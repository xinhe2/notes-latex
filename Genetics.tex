\documentclass{report}
\usepackage{amsmath,amssymb,fullpage}
\usepackage[version=3]{mhchem}
\usepackage{url}
\title{Notes on Biology}
\author{Xin He}
\begin{document}
%\maketitle
\tableofcontents
%\newpage

\def\Var{{\rm Var}\,}
\def\E{{\rm E}\,}
\def\Cov{{\rm Cov}\,}
\def\tr{{\rm tr}\,}

\newcommand\independent{\protect\mathpalette{\protect\independenT}{\perp}}
    \def\independenT#1#2{\mathrel{\setbox0\hbox{$#1#2$}%
    \copy0\kern-\wd0\mkern4mu\box0}}
    
\providecommand{\abs}[1]{\lvert#1\rvert}
\providecommand{\norm}[1]{\lVert#1\rVert}
%%%%%%%%%%%%%%%%%%%%%%%%%%%%%%%%%%%%%%%%%%%%%%%%%%%%%%%%%%%%
%%%%%%%%%%%%%%%%%%%%%%%%%%%%%%%%%%%%%%%%%%%%%%%%%%%%%%%%%%%%
\chapter{Genetics Background}
\section{Basic Principles of Genetics} 

Reference: [Human Molecular Genetics, 4th Ed, Chapter 3], [Thomas, Statistical Methods in Genetic Epidmeiology, 2004, Chapter 3]

Chromosome structure: 
\begin{itemize}
\item Cytogenetic map [Wiki]: Diagrams identifying the chromosomes based on the banding patterns are known as cytogenetic maps. Notations: p for short arm, and q for long arm, numbers indicat band, sub-band and sub-sub-band. 

\item Sex chromosomes: 
\begin{itemize}
\item Y-chromosome carries the gene sex determining region Y (SRY), the presence of which makes an embryo develop into a male. 
\item X-chromosome recombines only in females. There are two regions on the human X chromosome that are homologous with and recombine with the Y chromosome. These are called pseudoautosomal regions (PAR) because they act like autosomes. The larger (PAR1) is at the tip of the short arm; it is 2.6 Mb long and contains 14 genes. The smaller (PAR2) region on the tip of the long arm is only 320?kb long and contains four genes. 
\item Size of X-chr: 165 Mb, about 5\% of human genome.  Estimates of the number of genes on the X chromosome range from 600 to 1500. It has been claimed that the X is particularly rich in genes involved in intelligence and in sex and reproduction.
\end{itemize}
\end{itemize}

Chromosome abnormalities: 
\begin{itemize}
	\item If chromosome translocation or inversion happens that disrupts the disease gene, then it will lead to disease. Thus for patients with chrosome abnormalities in cytogenetic data, search for the breakpoints, which may reside in or close to the candidate genes. 
	\item Large chromosome deletions: often delete multiple genes, thus the patient may experience multiple genetic disorders, or one disorder and mental retardation (many deletion could lead to mental retardation because of the large number of genes involved in fetal brain development). 
	\item Array CGH: genome-wide meaurement of gene copy number amplifications or deletions. 
	\item \textbf{Remark}: because chromosome abnormalities are rare, thus it is likely that when chromosome abnormality occurs in a diseaes patient, there is a causal relation between the two. 
\end{itemize}

Principles of Mendelian Inheritance: 
\begin{itemize}
\item Segregation of alleles: each person carries two copies of each gene, one from each parent. Alleles at any given gene are transmitted randomly and with equal prob. 
\item Independent assortment: the alleles of different genes (not linked) are transmitted independently. 
\item The expression of two copies of a gene is indpendent of which parent they come from. The exception is imprinting. 
\end{itemize}

Hardy-Weinberg Equilibrium (HWE): 
\begin{itemize}
\item HEW in sex-linked loci: if a gene is X-linked, let the allele frequencies be $f(A) = p$ and $f(a) = q$. Then in males, the frequencies of A and a would be $p$ and $q$. In females, following HWE, the frequencies of AA, Aa and aa would be: $p^2, 2pq$ and $q^2$. 
\item HWE in general does not hold for loci associated with diseases. Thus deviation from HWE in cases of diseases is often taken as preliminary evidence of disease association.
\end{itemize}

Monogenic vs. multifactorial diseases:
\begin{itemize}
\item A spectrum of diseases, from monogenic to multifactorial/polygenic. In the intermediate, e.g. multiple loci with one making the major contribution. Note that the genetics of molecular traits (e.g. enzyme activity, gene expression) may be close to Mendelian traits. 
\item Dichotomous traits: suspectibility genes/loci. Continuous traits: eQTL. A trait that could be Mendelian or polygenic is called a complex trait. 
\end{itemize}

Patterns of Mendelian inheritance in pedigrees: note that males are hemizygous for loci in X or Y chromosomes. 
\begin{itemize}
\item Symbols for pedegrees: (1) Main symbols: males, females, mating, affection status. (2) Roman numbers for generations and Arabic numbers for individuals. (3) Other features: consanguineous mating, carrier status, probands (through which the family was ascertained.
	\item Autosomal dominant: an affected person usually has at least one affected parent. 
	\item Autosomal recessive: an affected persons are usually born to unaffected parents. 
	\item X-linked recessive: affects mainly males. 
	\item X-linked dominant: an affected father will only transmit to daughters but not sons. 
	\item Y-linked: affect only males. 
	\item Genes in the pseudoautosomal region (2.6Mb homologous regions between X and Y): inheritance pattern is similar to autosomal genes. 
	\item Mitochondrial inheritance: always transmitted from mothers (however, could be de novo mutations from unaffected mother). Also often highly variable clinical manifestation (a cell contains multiple mitochondrial and it is possible that some are normal). 
\end{itemize}

X-inactivation:
\begin{itemize}
\item In females, one of the two X-chromosomes is inactivated in each cell, and remains so in the entire life. Occurs probably in 10-20 cell stage. 
\item Phenotypic consequence of X-inactivation (mosaicism): (1) if the trait is highly localized (e.g. sweat gland/hair), female carriers may show patches of normal and abnormal tissues. (2) For some traits, the cells expressing the mutant copy may die, as a result, all surviving cells in females may be normal (e.g. mature B cells). (3) For X-linked recessive diseases, females occasionally get severely affected because of X-inactivation in critical cells. (4) X-linked dominant diseases: females often show variable phenotypes because some cells express normal X. 
\item An explanation of IQs in males and females based on X-inactivation:

\texttt{http://www.economist.com/news/science-and-technology/21568704-geniuses-are-getting-brighter-and-genius-levels-iq-girls-are-not-far}
It is easier for males to win the IQ lottery if they happen to get a single very strong version of the relevant genes on their X-chromosome. All female mammals are mosaics with respect to the X-chromosome. For a woman to have that same gene in all of her tissues, she would have to inherit the rare, very strong allele/s from both of her parents, which is statistically far less likely to occur. Similarly, males are also far more likely to lose this lottery, and wind up mentally retarded. 

\end{itemize}

Complications to the basic Mendelian inheritance patterns: 
\begin{itemize}
\item Locus heterogeneity: common in syndromes that result form a failure of a complex pathway. Ex. profound hearing loss, when two disease carriers marry, the children is often normal (complementation). 
\item Incomplete penetrance: define \textbf{penetrance} of a genotype, $G$, as $P(D|G)$, i.e. the probability of disease of an individual with genotype $G$. Not all diseases have 100\% penetrance, especially for dominant diseases. 
\item Age-related penetrance in late onset diseases: if a patient dies before the onset of the disease, then he/she may show as normal in the pedigree, complicating the analysis. 
\item Variable expression: may depend on the genotypic background and environment. 
\item Imprinting: for some autosomal diseases, the disease gene is inherited only from the father or the mother. Epigenetic changes in father or mother. 
\item Male lethality may complicate X-linked pedigree: e.g. Rett Syndrome. Affected males are often not born (miscarriage). 
\item De novo mutations: for diseases that prevent affected people from reproducting but new cases keep occuring, it is likely that a significant fraction is de novo mutations. Severe X-linked recessive diseases may also show a large fraction of de novo mutations. 
\item Mosaicism: if a mutation occurs early in life, then a significant fraction of cells may express the abnormal gene. When a large number of germ-line cells are mutated, this could produce multiple affected offsprings - the pedigree would mimic the recesive inheritance. 
\end{itemize}

Characterizing familial clustering of diseases: not all familial clustering is due to genetic factors, thus it is important to distinguish the two concepts
\begin{itemize}
\item Sporadic vs. familial cases: sporadic means isolated cases without a family history, and familial cases refer to cases with a positive family history. 
\item Hereditary cases: among the cases with a strong family history, those with a clear evidence of genetic factors. 
\item Example: breast cancer cases with genotyped BRCA1 and BRCA2. The majority of cases are sporadic. Among most familial cases, most are not hereditary. 
\end{itemize}

Polygenic threshold theory: [Human Molecular Genetics, 4rd Ed, Chapter 3]
\begin{itemize}
\item History: debate betweeen Mendelians and biometricians (Francis Galton). A paper by R.A. Fisher in 1918 settled the debate: characters governed by a large number ofindependent Mendelian factors (polygenic characters) would display precisely the continuous nature, quantitative variation, and family correlations described by the biometricians. DS Falconer extended this model to cover dichotomous characters.
\item Polyngenic theory: if a trait is determined by many loci (the sum of effects of all loci), then the trait distribution is approximately normal. The goal of this theory is to understand the influence of genetic and environmental factors, and the pattern of inheritance. However direct regression analysis is difficult, because neither genotype nor environmental variables are directly measured. 
\item Regression to the mean: one main observation in quantitative trait studies is: the offsprings of the individuals with extreme phenotypes, tend to have traits in the intermediate between the parent and the population mean. This is called ``regression to the mean''. This can be explained by that the parent may have many alleles that have extreme values (comparing with population mean), but because of random mating, the other parent has a smaller number of such alleles. 

\item Dicontinuous characters: assume there is an underlying continuous variable: susceptibility, and that there exists a threshold - when the suspectibility of an individual is higher than the threahold, the disease may develop. Susceptibility is a RV for individuals, similar to a quantitative character, while the liability threshold is fixed for a group. To model the risks: let $t$ be the threshold relative to the population mean in the unit of standard deviation (i.e. the $Z$ score), then the risk is: 
\begin{equation}
P(X \geq t) = 1-\Phi(t)
\end{equation}
where $\Phi(\cdot)$ is the CDF of normal distribution. Suppose we have a mutation in a gene, the individuals carrying this mutation have a different distribution of susceptibility - suppose its mean is $\mu$ and its standard deviation is the same (the mean is 0 for people without the mutation). Let $Z_{\mu}$ be the $Z$-score of the mean (i.e. measure of $\mu$ with the unit of standard deviation), then the risk of the mutant group is: 
\begin{equation}
P(X \geq t) = P(X' \geq t - Z_{\mu}) =1-\Phi(t-Z_{\mu})	
\end{equation}
The relative risk is thus: 
\begin{equation}
\text{RR} = \frac{1-\Phi(t-Z_{\mu})}{1-\Phi(t)} 	
\end{equation}
For example, at $t = 2$, the risk is about 2.3\%, and suppose $Z_{\mu} = 0.5$, the risk is 6.7\%, and the RR about 2.9. 

\item Model different risks in different groups: e.g. some diseases have higher risk in one gender than the other. This can be modeled by allowing different thresholds in different genders. 

\item Genetic model: suppose we have genotypes, AA, Aa and aa. Typically, we model the effect of a in the liability scale as $\alpha$, then the mean of the three genotypes are: $-\mu$, $\alpha-\mu$ and $2 \alpha - \mu$, where $\mu$ is chosen so that the population mean is 0. It can be shown that $\mu = 2 \alpha p$, where $p$ is the allele frequency of a. 

\item Q: what is the relationship between the polygenic threshold model and probit model for binary response? 
\end{itemize}

Penetrance function and genetic model: 
\begin{itemize}
\item Penetrance function: defined as $P(Y|G)$, where $Y$ is phenotype and $G$ genotype. Note that the penetrance function takes different forms depending on the phenotype. If the phenotype is binary, the penetrance is multinomial; if the phenotype is quantitative, the penetrance represents multiple continuous (e.g. normal) distributions. 
\item To characterize the penetrance function, we define different genetic models. Let $A$ be the wild type and $a$ be the mutant allele, then dominant means $P(Y|Aa) = P(Y|aa)$, i.e. a single copy of $a$ is sufficient to increase the risk; recessive means $P(Y|Aa) = P(Y|AA)$, i.e. need two copies of $a$ to increase the risk; additive (or multiplicative, depending on the scale) means $P(Y|Aa)$ is intermediate between $P(Y|AA)$ and $P(Y|aa)$. Codminant: means $P(Y|AA) \neq P(Y|Aa) \neq P(Y|aa)$. 
\end{itemize}

Genetic model of binary (dichotomous) traits: [UWash, b516, 2009] consider a locus with two alleles $A_1$ and $A_2$ (suppose $A_2$ increases the risk of diseaes), let $q_1$, $q_2$ be the frequency of the two alleles respectively. 
\begin{itemize}
	\item Penentrance: $f_{ij}$ is the probability of disease of the genotype $A_i A_j$. 
	\item Population prevalance of disease: also the weighted average of diseaes penetrance: 
\begin{equation}
K_p = q_1^2 f_{11} + 2 q_1 q_2 f_{12}	+ q_2^2 f_{22}
\end{equation}
	\item Genotype relative risk: the risk of disease of a genotype relative to the population average: $R_{ij} = f_{ij} / K_p$. 
	\item Dominant model: $f_{11} = k$, $f_{12} = f_{22} = k + c$. Recessive model: $f_{11} = f_{12} = k$, $f_{22} = k + c$.
	\item Additive model: $f_{11} = k$, $f_{12} = k + c$, $f_{22} = k + 2c$. 
	\item Multiplicative model: $f_{11} = k$, $f_{12} = rk$, $f_{22} = r^2 k$.
\end{itemize}

Genetic model of quantitative traits: consider a locus with 2 alleles $Q_1$ and $Q_2$. Typically, assume the trait of $Q_1 Q_1$ (reference) is 0, and then the trait of $Q_1 Q_2$ and $Q_2 Q_2$ are $a + d$ and $2 a$ respectively (or $a (1+k)$ and $2a$). We have: $k = d / a$, which measures the amount of dominance. Depending on the value of $k$, we have some special cases: 
\begin{itemize}
\item $k = 0$: no dominance;
\item $k = 1$: complete dominance;
\item $k = -1$: recessive;
\item $k > 1$: overdominance.
\end{itemize}   

Population genetics of Mendelian diseases: 
\begin{itemize}
	\item Model: apply the theory of selection-mutation balance, the frequency of the disease allele is related to the mutation rate and the selection coefficient: $\hat{q} = \sqrt{\mu/s}$ for recessive traits; and $\hat{q} = \mu/(hs)$ for dominant traits ($h$ is the degree of dominance). This allows one to estimate the mutation rate of diseases (e.g. for lethal diseases, $s = 1$). 
	\item The explanations of polymorphism in disease-causing alleles for Mendelian diseases: high mutation rate, late onset of symptoms (e.g. Huntington disease), heterozygote advantage (e.g. for cyarix fibrosis, mutation in membrance chloride channel, heterozygotes may be more resistent to certain infections, similar to sickel-cell disease). 
\end{itemize}

%%%%%%%%%%%%%%%%%%%%%%%%%%%%%%%%%%%%%%%%%%%%%%%%%%%%%%%%%%%%
\section{Genetic Epidemiology Overview}

Reference: [Thomas, Statistical Methods in Genetic Epidmeiology, 2004, Chapter 1, 3, 4]

Problems of complex trait genetics: 
\begin{itemize}
	\item Technonology and analytic methods for detecting the loci of complex traits. 
	\item Genetic architecture of complex traits: genetic vs non-genetic contribution, number of loci and effect size distribution, common vs rare variants, the comparison of genetic loci of related traits, etc.  
	\item Mechanism of complex trait genetics: how the polymorphism at the loci affects the trait: coding vs non-coding (or cis- vs trans-), types of genes, the mechanism of hot-spots (pleotropy), etc. 
	\item Application of the results of complex trait mapping: prediction of phenotypes, explanation of other complex traits, etc. 
	\item Evolutionary forces on complex traits: e.g. whether alleles are under natural selection. 
\end{itemize}

Genetic epidemiology background [Thomas04, Chapter 1]: an overview
\begin{itemize}
\item Descriptive epidemiology: forming the hypothesis of whether a trait/disease has a genetic component, by correlating trait with the properties that might represent genetics (e.g. race).
\item Aggregation analysis: using family data to test if a trait is genetic, or ``run in families''. 
\item Segregation analysis: fit the genetic model using family model, how many genes are involved (major gene vs. polygene), dominant or recessive, etc. 
\item Linkaging analysis: within families, co-segregation of traits and markers (which are close to causal loci).
\item Association mapping and functional studies. 
\end{itemize}

Descriptive epidemiolgy: 
\begin{itemize}
\item Principle 1. correlation of phenotypic traits and genetics. Typically, genes are not observed, we use ``markers'' that represent genetic similarity of groups, such markers often represent common ancestry, such as country of origin, race, etc.
\item Principle 2. the key challenge is the confounders such as environmental factors, culture and history, etc. Find the type of data where one (gene or environment) is controlled, e.g. migration studies (genes controlled). 
\item Background: incidence rate is the number of people caught a disease per year, often measured using person-years. It is different from prevalence. 

\item Variation of incidence rates across countries/regions: one of the most common studies of descriptive epidemiology. Ex. breast cancer rates vary nearly 20 fold across countries. 
\begin{itemize}
	\item Migration studies: the obvious problem with the country comparison is that many things are different. To control for that, study the incidence of migrants. 
\end{itemize}

\item Race and ethnicity: [Risch, 2002] argues the use of self-identified ethnicity in genetic epid. rather a purely genetic definition that would not encompass cultural factors that may be relevant. A special case of studies involving race is admixture studies, e.g. in African-Americans or Latinos. In particular, with such group, test the correlation between traits and the degree/gradience of certain genetic heritage. The advance of admixture studies is that the environmental factors are better controlled. 

\item Age effects: if a trait is genetic, then its age of onset tends to be earlier. Ex. comparison of age of onset in familial form vs. sporadic form of diseases. 

\item Time trends: because genetics cannot change in a short time, the changes of incidence in time are generally due to environmental factors. 
\end{itemize}

Family aggregation: the question is how to test if a disease ``runs in families''? 
\begin{itemize}
\item Case-control studies: if a disease ``runs in families'', then if one member gets the disease, it means that the chance that other members may also have the disease is higher. Two specific statistical strategies: 
\begin{itemize}
\item Randomly sample cases and controls (matched), obtain their families and compare the incidence of case-families vs. control-families. 
\item Treat the family history of case/control as ``exposure risk'', and test if the exposure increases the disease risk using the standard case-control comparison. 
\end{itemize}

\item Variance components method: for quantitative trait, statistical analysis of variance due to shared environmental or genetics. 

\item Twin studies: comparison of MZ and DZ twins. DZ twins are different genetically, but environment is controlled. 

\item Adoption studies: could be done in either ways: 
\begin{itemize}
\item Same gene, different environment: MZ twins but raised in separate families. 
\item Same environment, different gene: biological child vs. adopted child in the same families.
\end{itemize}

\item Inbreeding: correlation of incidence with the degree of inbreeding. 

\item \textbf{Lesson}: a major problem is to separate genetic and environmental effects, and this can be addressed to a large extent by the study designs. The easiest one is the case-control comparison where the two groups are matched. More generally, twin studies, adoption studies, admixture/inbreeding stuides all try to control one and study the effect of the other. 
\end{itemize}

Other steps in genetic epidemiology studies: 
\begin{itemize}
\item Segregation analysis: the goal is to fit the genetic model - single gene vs polygene model, dominant or recessive, etc. From the model, one can estimate the segregation parameters such as allele frequency and penetrance.

\item Linkage analysis: the difference with the segregation analysis that both disease loci and markers need to be modeled. 

\item Fine mapping: the regions identified from linkage studies are usually broad, use LD mapping to locate the segments shared by multiple haplotypes in cases. 

\item Association stuides: in addition to the standard case-control, association studies on families can reduce the problem of population substructure. Two common designs: case-sibling design (unaffected siblings as controls) and trios (TDT). 
\end{itemize}

Study designs in epidemiology: 
\begin{itemize}
\item Cohort design: choose a cohort, record the exposures, and then do follow-up. Advantage: less bias (see below). Disadvantage: may require large sample size, since only a small fraction in general will become cases. 

\item Case-control design: from cohort, find cases and controls, then compare risk factors (in the past). To choose cases and controls: 
\begin{itemize}
	\item Nested case-control design: for each case, find a best matched control: random sampling from the set of subjects who are at risk at the time the case occurred. 
	\item Case-cohort design: the controls are a random sample of the entire cohort, irrespective of whether or not they later become cases. The advantage: the same controls can be used for multiple case groups. 
\end{itemize}

\end{itemize}
 
Common issues/biases in epidemiology studies: 
\begin{itemize}
\item Selection bias: the selected samples are not representative of the entire population of interest. Ex. the controls may come from a particular geographic region that is different from cases in some unmeasured ways. 
\item Information bias: the information in cases and controls is somehow not comparable. Ex. recall-bias: cases and controls recall information in the past (risk factors) differently. 
\item Confounding: if a variable (1) is associated with the exposure of interest, and (2) conditional on the exposure, an independent risk factor of the disease, then this variable is a confounder. 
\end{itemize}

Modeling disease risks: 
\begin{itemize}
\item Motivation: the disease risk typically changes over time/age, thus we cannot use the standard Poisson process. 
\item Defintions: hazard and survival functions. Let $Y(t)$ be the status at time $t$, the absolute risk can be written as: $F(t) = P(Y(t)=1|Y(0)=0)$. The hazard function is similar to Poisson rate: 
\begin{equation}
\lambda(t) = \lim_{dt \to 0} 	P(Y(t+dt)=1|Y(t)=0) / dt
\end{equation}
The survival function is $S(t) = 1 - F(t) = P(Y(t)=0|Y(0)=0)$. 
\item Relationship between hazard and survival functions: the total rate from 0 to $t$ is the integration of $\lambda(u)$, and the survival function follows exponential distribution, so we have: 
\begin{equation}
S(t) = \exp\left( -\int_0^t \lambda(u) du\right)	
\end{equation}
On the other hand, the probability of having an event in $(t, t+dt)$ can be expressed in two ways. (1) No event up to $t$, followed by an event in $(t, t+dt)$, so this is $\lambda(t) S(t) dt$. (2) This is also: $P(Y(t+dt)=1|Y(0)=0) - P(Y(t)=1|Y(0)=0) = d F(t) = - dS(t)$. So we have: 
\begin{equation}
\lambda(t) dt = - \frac{dS(t)}{S(t)}	
\end{equation}
The relationship could also be derived using Fundamental Theorem of Calculus. 
\item Statistical problem of estimating hazard and survival function: our data would be time of cases for many subjects (or at multiple time points, the status of all subjects). The simplest strategy is to have multiple time windows, and the hazard at each window is constant, and can be computed simply by the frequency of cases. More complex statistical analysis is possible. 
\end{itemize}

Modeling relative risks: 
\begin{itemize}
\item Case-control design: the data is a 2 by 2 table: based on exposure status and affection status. Let the four entries b $A,B,C,D$ respectively, then the simplest test is the $\chi^2$ test or Fisher's exact test (for small samples). The odds ratio is, following definition: 
\begin{equation}
OR = \frac{D/C}{B/A} = \frac{AD}{BC}
\end{equation}
Alternatively, it can be interrpeted as the ratio of frequency of exposure in cases vs. controls. To obtain the confidence interval, we have: 
\begin{equation}
\Var(\log OR)	= \frac{1}{A} + \frac{1}{B} + \frac{1}{C} + \frac{1}{D}
\end{equation}
This follows from binomial distributions. The intuition is that the variance is dominated by rare events. 

\item Matched case-control design: equal number of cases and controls. We define a different 2 by 2 time: the exposure status in cases and in controls, and we test the null hypothesis if the diagonal elements (number of exposure in cases and controls) are equal. This is done by McNemar test. 

\item Cohort design: use the standardized incidence ratio (SIR). Let $z$ be the exposure variable, $s$ be other risk factors (e.g. age). We could define the hazard ratio as: 
\begin{equation}
\gamma_s = \frac{\lambda_s (z=1)}{\lambda_s (z=0)}	
\end{equation}
The expected number of cases under null model is: $E_z = \sum_s \lambda_s T_{zs}$, where $\lambda_s$ is the baseline rate and $T_{zs}$ is person-year. And the expected number of cases under alternative model is: $\E(Y_s) = \sum_s \gamma_s \lambda_s T_{zs}$. This motivates the definition of SIR: when $\gamma_s$ is a constant, we have: $\hat{\gamma} = Y_z/E_z$. 
\end{itemize}

Modeling modifers/interactions: 
\begin{itemize}
\item Single explanatory variable: suppose we have a binary phenotype $y$ and an explanatory variable $G$ (e.g. genotype). We use the logistic regression: 
\begin{equation}
\log \frac{P(y=1|G)}{P(y=0|G)} = \beta_0 + \beta_1 G	
\end{equation}
Plug in $G=0$ and $G=1$, we have: 
\begin{equation}
\log-\text{odds}(G=0) = \beta_0 \qquad \log-\text{odds}(G=1) = \beta_0 + \beta_1
\end{equation}
Subtracting the two: $\beta_1 = \log-\text{OR}(G=1)$, thus $\beta_1$ has a simple interpretation: the log odds ratio of $G=1$.  

\item Modifer variable: suppose we have an additional variable $E$, we have the model: 
\begin{equation}
\log \frac{P(y=1|G,E)}{P(y=0|G,E)} = \beta_0 + \beta_1 G + \beta_2 E + \beta_3 G\cdot E	
\end{equation}
To see what the parameters mean, first, we plug in different combinations of $G$ and $E$, and use $\log-\text{odds}(G=0, E=0)$ as the baseline odds:
\begin{equation}
\log-\text{OR}(G=1,E=0) = \beta_1, \log-\text{OR}(G=0, E=1) = \beta_2, \log-\text{OR}(G=1,E=1) = \beta_1 + \beta_2 + \beta_3
\end{equation}
From this, we see that $\beta_3$ represents the non-additive effect on the odds ratio: 
\begin{equation}
\beta_3 = \log-\text{OR}(G=1,E=1) - [\log-\text{OR}(G=1,E=0) + \log-\text{OR}(G=0, E=1)]. 
\end{equation}
Alternatively, $\beta_3$ has another interpretation representing the modification of the effect of $G$ due to $E$. To see this, we use the conditional OR (the baseline is defined under the condition): 
\begin{equation}
\log-\text{OR}(G=1|E=0)	= \beta_1 \quad \log-\text{OR}(G=1|E=1) = \beta_1 + \beta_3
\end{equation}
Note that the later OR is $\text{odds}(G=1,E=1) / \text{odds}(G=0,E=1)$. Thus we have: 
\begin{equation}
\beta_3 = \log-\text{OR}(G=1|E=1) - \log-\text{OR}(G=1|E=0).
\end{equation}
Thus $\beta_3$ measures the strength of the modifier effect. 
\end{itemize}

Matched design: conditional logistic regression [Thomas, chapter 4, page 85]: 
\begin{itemize}
\item Problem: suppose we are testing the effect of $Z$ on a response variable $Y$. However, there are potentially other confounding variables that might modify the risk $Y$. To deal with the problem, we use a 1:1 matched case-control, where each case and the matched control are identical in all aspects except that $Z$ is different. We can prove that under this matched design, even though the baseline risk of each pair is different (reflecting the other variables), the likelihood doesn't depend on these baseline risks, so they can be ignored. 

\item Model: consider the $i$-th pair, let $Y_{i1}$ and $Y_{i0}$ be the response variables for case and control, and let $Z_{i1}$ and $Z_{i0}$ for the explanatory variables. Let $\alpha_i$ be the baseline risk of the $i$-th pair, and $\beta$ the main effect to be estimated. We can show that: 
\begin{equation}
L(\beta) = \prod_i P(Y_{i1}=1|Z_{i1}, Z_{i0}, Y_{i0} + Y_{i1} = 1) = \frac{\exp(Z_{i1} \beta)}{\exp(Z_{i0} \beta) + \exp(Z_{i1} \beta)}	
\end{equation}
\end{itemize}

Generalized Estimating Equation (GEE) approach to correlated data [Robert Weiss' lecture, UCLA, Biostatistics 411]: 
\begin{itemize}
\item Motivation: we may have correlated data, examples: (1) longitude data: we have $N$ subjects, and for each subject, we may take $n$ measurements of a variable, clearly, the errors of these measurements tend to be correlated; (2) family data: we have $N$ families, and within each family, the random error terms (e.g. of a quantitative trait) may be correlated - due to shared environment, etc. However, instead of modeling the full likelihood, we want an approach that is robust to the error distributions. 

\item Background: Estimating Equation (EE) for parameter estimation. For the GLS problem (see Statistics Notes, ``Generalized Least Squares''), the EE can be written as: 
\begin{equation}
X^T \Sigma^{-1} (y - X \beta) = 0	
\end{equation}
It has three components: (1) $X^T$: this comes from $d\mu /d\beta$ (the chain rule), where $\mu = X\beta$ for linear model is the mean of $Y$; (2) $\Sigma^{-1}$; (3) the residual $Y - X\beta$. This motivates the GEE approach: even when the likelihood is not satisfied, we solve this form of equation to estimate $\beta$. 

\item Marginal model: we have data $Y_{ij}$, where $1 \leq i \leq N$ represents families/subjects, and $j$ represents individual observation within families/subjects. The mean, $\mu_{ij} = \E(Y_{ij})$ depends on explanatory variables $X_{ij}$ through a link function: $g(\mu_{ij}) = X_{ij} \beta$. The marginal variance: 
\begin{equation}
\Var(Y_{ij}) = v(\mu_{ij}) \phi	
\end{equation}
where $v(\cdot)$ is the variance function, and assumed known, and $\phi$ is a constant (1 for Poisson, Bernoulli; may need to be estimated). Correlation between $Y_{ij}$ and $Y_{ik}$ may depend on extra parameter $\alpha$ and means $\mu_{ij}$ and $\mu_{ik}$. The idea is to use the three components from the marginal model in the EE equation above (for each family): 
\begin{equation}
\sum_{i=1}^N \frac{d \mu_i}{d \beta} V_i^{-1} (Y_i - \mu_i(\beta)) = 0	
\end{equation}
where $V_i$ is the ``working covariance matrix''. Solving this equation gives the GEE estimator, $\hat{\beta}_{\text{GEE}}$. The working covar. matrix can be constructed as: 
\begin{equation}
V_i = A_i + \text{Corr}(Y_i)	
\end{equation}
where $A_i$ is the diagonal matrix of the marginal variance, and $\text{Corr}(Y_i)$ is the correlation matrix, dependent on the parameter $\alpha$. The variance of $\hat{\beta}_{\text{GEE}}$ can also be analytically determined (sandwich estimator). 

\item Extensions in genetics: we may be interested in estimating the parameters in the covariance, thus we need to extend GEE to higher-order moments. For instance, let $C_i$ be the observed covariance within families, and we are interested in estimating the parameters of the expected $C_i$: $\Sigma_i(\alpha) = \E(C_i|\alpha)$. Use the GEE: 
\begin{equation}
\sum_{i=1}^N \frac{d \Sigma_i}{d \alpha} W_i^{-1} (C_i - \Sigma_i(\alpha)) = 0	
\end{equation}
In genetics, we may need to solve both GEE (in terms of mean and covariance) simultaneously. 
\end{itemize}

%%%%%%%%%%%%%%%%%%%%%%%%%%%%%%%%%%%%%%%%%%%%%%%%%%%%%%%%%%%%
\section{Aggregation and Segregation Analysis} 

Reference: [Human Molecular Genetics, Chapter 15]

Aggregation analysis: deciding whether a non-Medelian character is genetic (``run-in-families''):
\begin{itemize}
	\item Family clustering: genetic disease tends to occur in families. Measured by $\lambda_R$, the risk to relative $R$ of an affected person (the entire population should be 1). 
	\item Shared family environment: a confouding variable, family clustering alone does not necessarily suggest genetic contribution. To estabilish genetic cause (especially for behavior disorders), may need twin studies or adoption studies. 
\end{itemize}

Segregation analysis: the mode of inheritance - Mendelian, oligogenic or polygenic, etc. 
\begin{itemize}
	\item Test: propose a genetic model (the number of loci, penetrance, etc.), compute the segregation ratio (the fraction of each phenotype) and compare the predicted vs and the observed ratio. Testing of statistical significance can be done with binomial/multinomial distributions.
	\item Example: three phenotypes with the genetic model (full penetrance): $C*G*$ - grey, $C*gg$ - black, $CC**$ - albino, and F1 is $CcGg$, then the segregation ratio is, $\text{grey} : \text{black} : \text{albino} = 9:3:4$. 
	\item Ascertainment bias: this is a problem when only families with diseases are genotyped. Thus the true distribution is truncated binomial, and need correction (e.g. will not be $3:1$ for simple Mendelian traits). 
\end{itemize}

\subsection{Aggregation Analysis}

Reference: [Thomas, Chapter 5]

Genetic relationship between relatives: 
\begin{itemize}
\item IBD: for a pair of individuals, the probability of sharing 0,1 or 2 alleles IBD: $\pi_0, \pi_1, \pi_2 (\nabla)$. The expected number of alleles sharing IBD is $\bar{\pi} = \pi_1 + 2 \pi_2$ is called the coefficient of relationship. 
\item Kinship coefficient ($\phi$): for a pair of individuals, the probability that a randomly selected pair of alleles, one from each individual, is IBD. Ex. for a sib pair, it is 1/4 (50\% the pair is from the same parent, and 50\% they are identical). 
\item The path method to determine the kinship coefficient: find all paths $p$ from the pair to a common ancestor, and let $M_p$ denote he number of meioses along that path. Then $\phi = \sum_p (1/2)^{M_p + 1}$. 
\end{itemize}

Testing family clustering of continous traits: 
\begin{itemize}
\item ANOVO approach: the factor is family, and we are testing if the family means are equal (if so, then family has an effect on the trait). The phenotype of the $j$-th member of the $i$-th family is:
\begin{equation}
Y_{ij} = \mu + X_i + E_{ij}	
\end{equation}
where $X_i$ is the family effect and $E_{ij}$ is the error (environment) term. 

\item Use family member correlation: for the $i$-th family, the covariance between members $j$ and $k$ can be expressed as (using components of covariance - see the Section on Quantitative Genetics): 
\begin{equation}
\Cov(Y_{ij}, Y_{ik}) = 2 \phi_{jk} \sigma_A^2 + \nabla_{jk} \sigma_D^2 + 	\phi_{jk} \sigma_I^2 + \gamma_{jk} \sigma_C^2 + \delta_{jk} \sigma_E^2
\end{equation}
where $\sigma_A^2, \sigma_D^2, \sigma_d^2$ are additive, dominance variance and interaction (between different loci) variance, $\sigma_C^2$ for shared environment and $\sigma_E^2$ for indepenent environment, and $\gamma_{jk}$ for the propotion of shared env. influence and $\delta_{jk} = 1$ if $j=k$ and 0 otherwise. Given this, we can construct the covariance matrix (e.g. for a nuclear family of four member) of a family, and use the data to fit the covariance matrix to estimate the variance terms.  
\end{itemize}

\subsection{Segregation Analysis}

Reference: [Thomas, Chapter 6]

Ascertainment: 
\begin{itemize}
\item Why could ascertainment be an issue? For example, suppose we are estimating the effect $\beta_1$: 
\begin{equation}
y = \beta_0 + \beta_1 x + \epsilon	
\end{equation}
where $x$ is binary. Then $\beta_1 = \E(y|x=1) - \E(y|x=0)$. If we only ascertain the extreme values, then $\beta_1$ may be overestimated. 

\item Different ascertainment strategies for binary traits: define $\pi$ as the probability that a random case in the population is chosen for study. Then we have complete ascertainment if $\pi=1$, or single ascertainment if $\pi \to 0$, i.e. any case has a certain probability to be chosen. Under single ascertainment, multi-case families have higher probabilites of being ascertained. 
\end{itemize}

Segregation analysis for sibship data: 
\begin{itemize}
\item Example: suppose we are testing if the gene is dominant or recessive. If dominant, then the probability of being affected in a sibling is $p = 1/2$; and if recessive $p = 1/4$. So we are testing whether $p = 1/2$ or 1/4. Given a sibship of size $s$, suppose $r$ siblings are affected, then we have: 
\begin{equation}
P(r|s, p) = \binom{s}{r} p^r (1-p)^{s-r}
\end{equation}
Suppose we have $N_{rs}$ siblings for the cases of $r$ affected in $s$ siblings, then the likelihood is: 
\begin{equation}
L(p) = \prod_r \prod_s P(r|s,p)^{N_{rs}}	
\end{equation}
This allows us to test $p$. 

\item Ascertainment correction: suppose we have single ascertainment, then a family is ascertained if $r \geq 1$. We have: 
\begin{equation}
P(r|s, A) = \frac{P(r|s) P(A|r,s)}{P(A|s)}	
\end{equation}
The term $P(r|s)$ is the binomial probability, $P(A|r,s) = 1 - (1-\pi)^r \approx \pi r$, which is roughly proportional to the number of affected siblings, and $P(A|s)$ can be computed by summing over $r$. 
\end{itemize}

Segregation analysis in a general pedigree with major gene: 
\begin{itemize}
\item Model: suppose we consider a pedigree, let $Y_i$ be the phenotype of the $i$-th member, and $G_i$ be its genotype (unobserved). Let $G_{m_i}$ and $G_{f_i}$ be the parent genotypes. The parameters of the model are $\Theta = (f, q)$ where $f$ is the penetrance function and $q$ allele frequency (in founders). The model can be written as: 
\begin{equation}
P(Y|\Theta) = \sum_g \prod_i P(Y_i|g_i) P(g_i|g_{m_i}, g_{f_i})	
\end{equation}
Note that for founders, the conditional probabilities are replaced by allele frequencies. 

\item Ascertainment: we need to modify the likelihood according to the ascertainment model: 
\begin{equation}
L_A(\Theta) = P(Y|\Theta, A, \pi) = \frac{L(\Theta) P(A|Y,\pi)}{P(A|\Theta, \pi)}	
\end{equation}

\item Elston-Stewart peeling algorithm: assume conditional independence of members given the parents, the we could formuate a Dynamic Programming algorithm where the subproblems are: $P(Y|G)$, where $G$ is the genotype of ``linkers'' (the individuals that link different families). 
\end{itemize}

Segregation analysis in a general pedigree with polygene: 
\begin{itemize}
\item Ideas: the traits are determined by the breeding values $Z$: the contribution of genes to the trait. And the dependency within families members can be captured by the correlation between $Z$'s. 

\item The breeding values $Z$: correlated among families members. Ex. assuming additive model, we have $\E(Z) = \frac{1}{2} [\E(Z_m) + \E(Z_f)]$, the mean of $Z$ is the average of parent means, and for the variance, $\Var{Z} = (\frac{1}{2} P_1 + P_2) V = \frac{1}{2}V$ where $V$ is the variance of parents. 

\item Model: we have the likelihood
\begin{equation}
L(\Theta) = \int \cdots \int \prod_i P(Y_i|Z_i=z_i) P(z_i|z_{m_i}, z_{f_i}) dz	
\end{equation}
We can use MVN for the distribution of $Z$: its mean is vector $\mu$, and covariance matrix $C$ captures the dependency of relatives. For example, for a nuclear family with two offsprings, $C$ is a $4 \times 4$ matrix, with the covariance between parent and child $1/2$, and the covariance between siblings also $1/2$. The trait can be written as a linear model of $Z$, or simply $\E(Y|Z) = \alpha + Z + \epsilon$ (then $Z$ need to be scaled). It's can be shown then that $Y$ follows MVN distribution. 
\end{itemize}

%%%%%%%%%%%%%%%%%%%%%%%%%%%%%%%%%%%%%%%%%%%%%%%%%%%%%%%%%%%%
\section{Recombination and Linkage} [Hartl, Principles of Population Genetics, Chapter 2,3, Section 9.2]

Linkage disequilibrium: 
\begin{itemize}
\item Model: consider two-loci, each with two alleles $A, a$ and $B,b$. Let $p_A, p_a$ be the frequency of $A$ and $a$, $p_B, p_b$ be the frequency of $B$ and $b$, $P_{AB}$ be the frequency of the haplotype $A B$ and so so. We know that if two loci are independently assorted, then we always have $P_{AB} = p_A p_B$. If the two are linked, this may not be true. We define $C$ as the fraction of recombination (defined as the fraction of recombinants, or the probability that any haplotype undergoes recombination), then we have the equation: 
\begin{equation}
P_{AB}' = (1-c) P_{AB} + c p_A p_B
\end{equation}
where $P_{AB}'$ be the frequency in next generation. If we define $D = P_{AB} - p_A p_B$, called linkage disequilibrium (LD), then we have: 
\begin{equation}
D_{t+1} = (1 - c) D_t
\end{equation}
Thus $D_t$ converges to 0 at the rate $1 - c$, i.e. strong recombination leads to quicker linkage equilibrium ($D = 0$). For any two loci with $D > 0$, we call the two at LD. 

\item Frequency of the gamete types: given the allele frequency, only a single parameter $D$ is sufficient to determine the gamete types:  
\begin{equation}
\begin{array}{ll}
P_{AB} & = p_A p_B + D\\
P_{Ab} & = p_A p_b - D\\
P_{aB} & = p_a p_B - D\\
P_{ab} & = p_a p_b + D
\end{array}	
\end{equation}

\item Remark: fraction of recombination: $0 \leq c \leq \frac{1}{2}$, the extreme case is the two loci are completely independent, e.g. when they are very distant. Note in this case, cannot conclude that always recombination, thus the fraction of recombinant is 1; in fact, there could be multiple recombinations between two loci, and the overall effect is one locus has no information of the other. 
\end{itemize}

Measure of linkage disequilibrium (LD): 
\begin{itemize}
\item $D'$: $D$ depends on allele frequencies, thus not comparable when allele frequency is different. Define $D'$ as the fraction of $D$ over the theoretical maximum of $D$ (at that allele frequency), $D_{\text{max}}$ if $D > 0$; and the fraction over the theoretical minimum if $D < 0$. The maximum and minimum of $D$ is given by: 
\begin{equation}
\begin{array}{ll}
D_{\max} & = \min\{ p_A p_b, p_a p_B\}\\
D_{\min} & = \max\{ -p_A p_B, -p_a p_b\}
\end{array}	
\end{equation}
Note: $D'$ has the property that $D' = 1$ if any of the four gametes has frequency 0, i.e. $D'$ is sensitive to rare haplotypes.  

\item $r^2$: defined as $D^2 / (p_A q_a p_B q_b)$. It has a simple interpretation: its square root is the correlation coefficient between the two alleles $A$ and $B$ (treating them as two RVs). 

\item Comparison of $D'$ and $r^2$: $D'$ measures the departure from linkage equlibrium, and $r^2$ measures the correlation/dependence of two loci. Thus at $D' = 0$ (equlibrium), we have $r^2 = 0$. However, as $D'$ increases, $r^2$ may take a range of values - there are many ways of moving away from equlibrium. Treat this as the independence of two binary RVs: $D' = 0$ means two RVs are independent; $r^2$ is large only when two RVs are very correlated, or one variable encodes information of the other (which may not be the case, e.g. $A = 1$ suggests $B = 1$, but not the other way around).

\item Example: when one haplotype is absent, $D' = 1$. $r^2$ on the other hand, depends on the genealogy/history: e.g. if $a$ and $b$ mutations happen to occur in the same chromosome in the ancestor, then they will show strong correlation; if mutations are new, then weak correlation.  
\end{itemize}
 
Haplotypes: 
\begin{itemize}
	\item Definition: a combination of alleles at multiple loci that are transmitted together on the same chromosome.
	\item Haplotype blocks: defined according to $r^2$. In human the haplotype blocks are of size a few tens of kb; in Drosophila, a few kb; in Arabidopsis, the order of 100 kb (intense inbreeding). 
	\item Haplotype resolution: in genotyping experiment, at each SNP, if it is heterozygous, it may not be clear which parent an allele is from, and as a consequence, it may be difficult to determine haplotype of a multi-SNP region. 
\end{itemize}

Causes of LD: 
\begin{itemize}
	\item Population admixture: if allele frequencies are different, then one locus may show LD with another. Suppose $p_A$ is large in one population, but not in the other, then an $A$ allele will suggest that it is more likely to come from the first population, and then the allele in the $B$ locus would tend to be that specific to the first population. An extreme case: suppose one population has fixed haplotype $AB$, and the other has fixed haplotype $ab$, then the mixture of two mutations will result in LD (no heterozygotes at the beginning).  
	\item Reduced recombination: from inbreeding (e.g. in plants), chromosome inversion (making recombination more difficult), etc. 
	\item Selection: a particular combination of alleles may have higher fitness. 
	\item Recent mutations: if mutation is recent, and recombination is low, then the two loci may be at LD. This is especially important in human genetics, where human populations grow exponentially in a short amount time of time, and the recombination rate is low, about 1 crossover per 100 Mb per generation (thus many mutations in a neighborhood without enough time to shuffle them).
	\item \textbf{Remark}: the association of $X$ and $Y$ may come from the common association of $X$ and $Y$ with some confounding variable (e.g. the group where the data instance comes from). In population genetics: population structure is a common confounding variable. 
\end{itemize}

Meiosis, recombination and origin of species [Peter Donnelly, May, 2016]
\begin{itemize}
	\item Recombination: hot-spots about 1-2kb, 30K hotspots in human. Contains motif of Zinc finger gene PRDM9. PRDM9 binds the motif, place H3K4me3 mark, recruit recombination machinary, and create ds break. The break is then resolved in two ways: 10\% cross-over, 90\% non cross-over. 
	
	\item PDRM9 in recombination: if there is a cis-change in one chromsome, asymmetry leads to inability of fixing ds break, and the offsprings have low fertility. Changing PRDM9 sequence would equilize both chromosomes and make the offsprings fertile again. Experimental proof from breeding two mouse strains.  
	
	\item Role in speciation: each change at PRDM9 motif creates asymmetry and pressure of PDRM9 change. However, if PDRM9 does not change quick enough, potential for creating hybrid incompability. 
\end{itemize}

%%%%%%%%%%%%%%%%%%%%%%%%%%%%%%%%%%%%%%%%%%%%%%%%%%%%%%%%%%%%
\section{Quantitative Genetics} 

Reference: [Felsenstein, Chapter 9], [Falconer \& Mackay, Introduction to Quant. Genetics, 4ed]

Overview: 
\begin{itemize}
\item Goal: the relative contributions of genetic and environmental influences; prediction of the traits of relatives; prediction of the response to selection (e.g. choosing parents of higher values of traits, what would be the distribution of offsprings); etc. 

\item Intuition: in general, we do not know the exact genetic basis of a quantitative trait. But we still be able to make predictions because: the traits of relatives are correlated (to different degrees depending on how close they are), and the extent of correlation reveals genetic vs environmental effects, and allows extrapolations (response to selection, other relatives, etc.). 

\item ANOVA perspective: Consider variation of some phenotype, $P$, it has two sources: genotype variation and environmental variation. We call the effect of genotype $G$, and the effect of environment $E$, then: 
\begin{equation}
P = G + E	
\end{equation}
From ANOVA perspective, $G$ represents the group effect and $E$ represents the intra-group variation. The total variance can be partitioned into the variance due to each source:
\begin{equation}
V_P = V_G + V_E	
\end{equation}
We now have a way of defining the contribution of one source to the total variation, e.g.: the ratio of $V_G$ over $V_P$ is the importance of genetic effects on the phenotypic variation (heritability).

\item Correlation between relatives: in general, the genotypes (groups) are not observed, so the standard ANOVA approach cannot be applied. The statistical idea is: the $G$ component is shared between relatives (to different extents), and this creates correlation between relatives. This would allow one to infer $V_G$, $V_E$, $h^2$, etc.  
\end{itemize}

\subsection{Variance of Phenotypes}

Quantitative genetic model: 
\begin{itemize}
\item Additive model: the phenotype value $P$ is the sum of effects contributed by each of the $n$ loci, plus an environment effect: 
\begin{equation}
P = \sum_{i=1}^n g_i + e
\label{eq:additive_model}	
\end{equation}
It is assumed that $E(e) = 0$. The parameter $g_i$ can be understood as the deviation from the mean introduced by the genotype at the $i$-th locus, i.e. the mean of the group defined by the genotype at the $i$-th locus, relative to population mean. 

\item Assumption 1: the effects of loci are additive, and there is no interaction among loci. 

\item Assumption 2: the genotypes at the $n$ loci are independent of each other. Effectively assume linkage equilibrium of the $n$ loci. 

\item Assumption 3: the environmental contribution to the phenotpye is indepdent of the genotype, and indepdent of environment contributions in other individuals. The later part of this assumption is the part that is most violated in practices, as environments of relatives are often correlated. 

\item Scale transformation: often need to transform the quantitative trait s.t. the additivity is held (and also for normality). 
\end{itemize}

Mean phenotypes in relatives: the goal is to determine how mean phenotypes of relatives are related. 
\begin{itemize}
\item Mean phenotype: use the assumption $E(e) = 0$, we have: 
\begin{equation}
E(P) = \sum_i E(g_i)
\end{equation}
Thus we only need to consider each locus independently. 

\item Inbreeding effects: let the genotype frequencies of $AA$, $Aa$ and $aa$ be $P$, $Q$ and $R$ respectively (which can be determined from inbreeding coefficient); and the phenotypes are $a_{11}, a_{12}$ and $a_{22}$. Then:
\begin{equation}
E(g) = P a_{11}	+ Q a_{12} + R a_{22} = p^2 a_{11} + 2 p (1-p) a_{12} + (1-p)^2 a_{22} + f p (1-p) [a_{11} + a_{22} - 2 a_{12}]
\end{equation}
Thus the mean phenotype is linearly related to the inbreeding coefficient $f$. The coefficient is proportional to the difference between the mean of the two homozygotes and the heterozygote.  

\item Inbreeding depression/hybrid vigor: for many traits, the heterozygote is often better than either parent (assuming homozygotes). This could be due to overdominance, or dominance at most of the loci (thus the heterozygote is always maximum between the two possible homozygote genotypes). 

\item Means of crosses and backcrosses: suppose we cross two inbred lines. We have $P_1 = a_{11}, P_2 = a_{22}$, and $F_1 = a_{12}$, but there may not be a simple relationship of $P_1$ and $P_2$ as we don't know dominance relation. The mean phenotypes of $F_2$ can be expressed in terms of the mean phenotypes of the parents and $F_1$: 
\begin{equation}
F_2 = \frac{1}{4}a_{11} + \frac{1}{2}a_{12} + \frac{1}{4}a_{22} = \frac{1}{2}	\left(\frac{1}{2} P_1 + \frac{1}{2} P_2 \right) + \frac{1}{2} F_1
\end{equation}
\end{itemize}

Additive and dominance variances: 
\begin{itemize}
\item Motivation: even through relatives share genetic contribution, not all genetic components can be inherited, so we need to partition the genetic effect into inheritable ones and the non-inheritable ones. 

\item ANOVA perspective: Suppose we have two factors, gene (represented by group $G$) and environment (represented by group $E$): 
\begin{equation}
Y = G + E	
\end{equation}
where $G$ is the effect of gene, and $E$ is effect of environment (since we are mainly interesed in gene, this is also the ``error variance'' within a group defined by genotype). To test if $G$ has any effect, ANOVA decomposes the variance of $Y$ into that due to $G$ (inter-group) and that due to $E$ (within-group). We can write this as:
\begin{equation}
\Var Y = \Var G + \Var E
\end{equation}
$G$ is related to the underlying alleles: $G = \mu_i - \mu$ with probability $p_i$, where $\mu_i$ be the mean of the $i$-th group and $\mu$ is the mean of the entire population. Note that $\E(G) = 0$. The variance of $G$ is given by: 
\begin{equation}
\Var G = \E (G^2) = \sum_i p_i (\mu_i - \mu)^2
\end{equation}
From this we see that, large effect $\mu_i - \mu$ leads to large $\Var G$ - thus we focus on the variance paritition to estimate how large the genetic effect is. 

\item Additive and dominant genetic values (Fisher's decompoisition): we assume an additive genetic model, given by Equation~\ref{eq:additive_model}. For simplicity, we consider a single gene with two alleles, $i$ and $j$. We want to write the genotype effect, $g$, as the sum of three parts: 
\begin{equation}
g = \mu + \alpha_i + \alpha_j + \delta_{ij}
\end{equation}
where $\alpha_i$, $\alpha_j$ are the departure from the mean due to the alleles $i$ and $j$, respectively. And $\delta_{ij}$ is the ``dominance deviation'' from group mean that is unexplained by $\alpha_i$ and $\alpha_j$. Define the additive genetic value $A = \alpha_i + \alpha_j$, and the dominance genetic value $D = \delta_{ij}$. \\
Note: our notation is $\alpha_i$ and $\alpha_j$ are RV's since $i$ and $j$ (allele) are random; however, $\alpha_1$, $\alpha_2$, $\delta_{11}$, etc. represent constants, e.g. $\alpha_1$ is the mean of the group (allele type 1), $\mu_1$, minus the overall mean $\mu$. 

\item Additive and dominance variances and heritability: in general we can write the phenotype as: 
\begin{equation}
P = \mu + A + D + E	
\end{equation}
where $A$ is called the ``breeding value'', and all terms are independent. Thus: 
\begin{equation}
V_P = V_A + V_D + V_E	
\end{equation}
We define the broad-sense heritability as: 
\begin{equation}
H^2 = \frac{V_G}{V_P}	
\end{equation}
and the narrow-sense heritability as: 
\begin{equation}
h^2 = \frac{V_A}{V_P}	
\end{equation}

\item Experimental determination of $V_G$ and $V_E$: 
\begin{itemize}
	\item In theory, one can determine $V_E$ by measuring the phenotypic variance of genetically identical strains. Then suppose we have $V_P$ from the normal variation in the population, then $V_G = V_P - V_E$. 
	\item Difficulities: $V_E$ may depend on the genotype of the identical strains. Furthermore, inbreed strains somtimes have higher (or lower) variance than the normal cross-bred population. 
\end{itemize}
\end{itemize}

Relating $V_A$ and $V_D$ to the genetic model: [Felsenstein, Chapter 9] 
\begin{itemize}
\item Genetic model: suppose we have a locus with 2 alleles $A_1$ and $A_2$, with the frequency of $A_1$ being $p$ and the other $q = 1 - p$. The genetic effect of the genotype $(ij)$ is denoted as $a_{ij}$. Our goal is to relate $V_A$ and $V_D$ to $p$ and $a_{ij}$'s. We first note that the genotype of a sample, $i$ and $j$, are random, and so the genetic values (thus we can define the variance of genetic values). Also note that $\E(\alpha_i) = 0$ and $\E(\delta_{ij}) = 0$. We have the relation between $V_A$, $V_D$ and genetic values: 
\begin{equation}
V_A = E[\alpha_i^2]	+ E[\alpha_j^2] = 2 E[\alpha^2] = 2 \sum_{i=1}^m p_i \alpha_i^2
\end{equation}
\begin{equation}
V_D = E[\delta_{ij}^2] = \sum_{i,j=1}^m p_i p_j \delta_{ij}^2	
\end{equation}
where $m$ is the number of alleles (the equation can be applied to multi-allele locus as well). Our problem next is to determine $\alpha_i$ and $\delta_{ij}$. 

\item Genetic values by group means: $\alpha_1$ is simply the mean phenotype of the group where one alleles is $A$ minus the population mean; and similary for $\alpha_2$. The population mean is: 
\begin{equation}
\mu = \sum_{ij} p_i p_j a_{ij}	
\end{equation}
assuming HWE. For two-allele case, we have: 
\begin{equation}
\alpha_1 = p a_{11} + (1-p) a_{12} - \mu	
\end{equation}
\begin{equation}
\alpha_2 = p a_{12} + (1-p) a_{22} - \mu	
\end{equation}
And for the dominance genetic value, we have: 
\begin{equation}
\delta_{ij} = a_{ij} - (\mu + \alpha_i + \alpha_j)	
\end{equation}
Plug-in $\alpha_1$ and $\alpha_2$, we have: 
\begin{equation}
\delta_{11} = (1 - p)^2 (a_{11} - 2 a_{12} + a_{22})	
\end{equation}
The expression for $\delta_{12}$ and $\delta_{22}$ are the same, but with $(1-p)^2$ replaced by $-p(1-p)$ and $p^2$ respectively. 

\item Genetic values by regression: we could consider the equation of the genetic value as a regression where the predictor $N$ is the number of $a$ alleles in the parents:
\begin{equation}
g = \mu + 2 \alpha_1 + (\alpha_2 - \alpha_1) N + \delta	
\end{equation}
where $\delta$ is the residual term. Thus the regression coefficients can be expressed through regression on three groups: the group $AA$ with average trait $a_{11}$; the group $Aa$ with average trait $a_{12}$; and the group $aa$ with average trait at $a_{22}$. Solving this regression leads to the same equations of $\alpha_i$ and $\delta_{ij}$ above. 

\item Common parametrization of genetic model [Falconer, Chapter 7]: this is given by: 
\begin{equation}
a_{11} = a \qquad a_{12} = d \qquad a_{22} = -a	
\end{equation}
If there is no dominance, $d = 0$; if $A_1$ is dominant over $A_2$, $d > 0$; if $A_2$ is dominant over $A_1$, $d < 0$. The degree of dominance is measured by $d/a$. Under this parameterization, we have the average effect of gene substitution: 
\begin{equation}
\alpha = p(a - d) + q(d + a) = a + d(q - p)	
\end{equation}
And the additive genetic values of two alleles:
\begin{equation}
\alpha_1 = q \alpha	\qquad \alpha_2 = -p \alpha	
\end{equation}
\end{itemize}

Extending the basic model of genetic and environment variance [Falconer, Chapter 8]:
\begin{itemize}
\item Correlation between genotype and environment: if the two variables are not independent, e.g. one genotype is more likely to be associated with one type of environment (e.g. in experimental populations, the treatment depends on the phenotype, thus the genotype of organisms), we have: 
\begin{equation}
V_P = V_G + V_E + 2 \Cov_{GE}	
\end{equation}
Intuition: when $G$ and $E$ are correlated, $V_P$ will be higher because the phenotype can now become more extreme (both $G$ and $E$ change in the same direction). 

\item Gene-environment interaction: when there is an interaction term, we have: 
\begin{equation}
P = G + E + I_{GE}	
\end{equation}
And then: 
\begin{equation}
V_P = V_G + V_E + 2 \Cov_{GE}	+ V_{GE}
\end{equation}
Again, $V_P$ is now higher because the interaction creates additional deviation from the additive model. 

\item Epistasis: suppose two loci control the genetic value, we could write genetic value as: 	
\begin{equation}
G = \mu_G + A + D + AA + AD + DD	
\end{equation}
where $A$ is the average effect of single alleles, $D$ is dominance deviation, $AA$, $AD$ and $DD$ the interactions. Variances: 
\begin{equation}
V_G = V_A + V_D + V_{AA} + V_{AD} + V_{DD}	
\end{equation}

\item Remark: in general, correlations and interactions among the components increase the variance.
\end{itemize}

\subsection{Heriability and Resemblance of Relatives}

Reference: [Felsenstein, Chapter 9], [Falconer \& Mackay, Introduction to Quant. Genetics, 4ed]

Covariance between relatives: 
\begin{itemize}
\item Indepdence among loci: suppose we consider the phenotypes of two relatives: 
\begin{equation}
\begin{array}{lll}
X & = & g_1 + g_2 + \cdots + g_n + e\\
Y & = & g_1' + g_2' + \cdots + g_n' + e'
\end{array}
\end{equation}
Assume the independence of loci, the independence of environment and genetic effects, we have: 
\begin{equation}
\text{Cov}(X,Y) = \text{Cov}(g_1, g_1') + \text{Cov}(g_2, g_2') + \cdots + \text{Cov}(g_n, g_n')
\end{equation}

\item Covariance in terms of $V_A$ and $V_D$: we consider only a single locus, there are only three situations in $g$ and $g'$: 
\begin{itemize}
\item $g$ and $g'$ has no allele IBD: no relationship between the two, thus $\text{Cov}(g,g') = 0$. 
\item $g$ and $g'$ has one allele IBD: we have $g = \alpha_i + \alpha_j + \delta_{ij}$ and $g' = \alpha_i + \alpha_k + \delta_{ik}$, the only nonzero terms are $\text{Cov}(\alpha_i, \alpha_i)$ and $\text{Cov}(\delta_{ij}, \delta_{ik})$. It can be proven that the latter term is 0 (intuitively, this is clear, as the dominance term is entirely determined by $A_j$ and $A_k$, but they are independent). Thus $\text{Cov}(g,g') = V_A / 2$
\item $g$ and $g'$ has two alleles IBD: this is the variance of $g$, $\text{Cov}(g,g') = V_A + V_D$
\end{itemize}
Suppose the probability of the cases above are $P_0, P_1, P_2$, respectively, we have: 
\begin{equation}
\text{Cov}(g,g') = P_1 (V_A / 2) + P_2 (V_A + V_D)	
\end{equation}
In terms of the covariance between the relatives: 
\begin{equation}
\text{Cov}(X,Y) = \left(\frac{1}{2} P_1 + P_2\right) V_A + P_2 V_D
\end{equation}
Remark: this is the key equation, and it can be understood without invoking the underlying genetics, i.e. expression of $\alpha$ and $\delta$ terms in terms of the underlying genetic values ($a_{11}, a_{12}, a_{22}$) and allele frequencies. The underlying genetics provide a rigorous proof of the relevant relations, notably $\text{Cov}(\delta_{ij}, \delta_{ik}) = 0$. 

\item Correlations and regression: the correlation and regression coefficient between two relatives $X$ and $Y$: 
\begin{equation}
\rho_{XY} = \frac{\text{Cov}(X,Y)}{\sigma_X \sigma_Y}	
\end{equation}
\begin{equation}
\beta_{XY} = \frac{\text{Cov}(X,Y)}{\text{Var}(X)}	
\end{equation}
when $\sigma_X = \sigma_Y$ (normally true except the case of offspring-midparent regression), we have $\rho_{XY} = \beta_{XY}$.  

\item Some cases: 
\begin{itemize}
\item Parent and offspring: we have $P_1 = 1, P_2 = 0$, thus $\rho_{OP} = \beta_{OP} = \frac{1}{2} h^2$; 

\item Between offspring and mid-parent: the covariance:
\begin{equation}
\Cov\left(\frac{P_1 + P_2}{2}, O\right) = \Cov(O,P) = \frac{1}{2} V_A	
\end{equation}
The variance of mid-parent:
\begin{equation}
\Var\left(\frac{P_1 + P_2}{2}\right) = 	\frac{1}{2} V_P
\end{equation}
Thus we have $\beta_{O \bar{P}} = h^2$. 

\item Half siblings: the covariance: 
\begin{equation}
\Cov_{HS}= 	\frac{1}{4} V_A
\end{equation}
Thus $\rho_{HS} = \frac{1}{4} h^2$; 

\item Full siblings: $P_1 = \frac{1}{2}, P_2 = \frac{1}{4}$:
\begin{equation}
\rho_{FS} = \frac{1}{2} h^2 + \frac{1}{4} \frac{V_D}{V_P}	
\end{equation}
Note that the correlation between ful sibs is greater than between parents and offsprings (the possibility of exactly the same diploid genotype in the two). 
\end{itemize}

\end{itemize}

General model: resemblance between relatives under gene-gene interactions and shared environment: 
\begin{itemize}
\item Gene-gene interactions: in the absence of gene-gene interactions, suppose the covariance between two relatives is given by: 
\begin{equation}
\Cov = r V_A + u V_D	
\end{equation}
With epistatic interactions, the covariance would be higher. Suppose we consider two relatives: 
\begin{equation}
G = A + D + AA	\qquad G' = A' + D' + AA'
\end{equation}
The covariance is thus: 
\begin{equation}
\Cov(G,G') = r V_A + u V_D + r^2 V_{AA}	
\end{equation}
where $r^2$ comes from the fact that if the probability that the two share a single gene is $r$, then the probability that the two share two genes (s.t. epistasis is shared) is $r^2$. More generally, we have more interaction terms:
\begin{equation}
\Cov = 	r V_A + u V_D + r^2 V_{AA} + ru V_{AD} + u^2 V_{DD} + r^3 V_{AAA} + r^2 u V_{AAD} + r u^2 V_{ADD} + u^3 V_{DDD} + \cdots
\end{equation}

\item Environmental covariance: for relatives, it may be important to model the shared environment, thus the environmental variance can be writted as: 
\begin{equation}
V_E = V_{Ec} + V_{Ew}	
\end{equation}
where $V_{Ec}$ is the between-group variance (contributing to the similarity between relatives in the same group).  
\end{itemize}

Estimating variance components and heritability: [Falconer, Chapter 10]
\begin{itemize}
\item Common procedures: parent-offspring regression; half-sib covariance: groups of half-sibs, each group with the same father; etc.  

\item ACE model for twin-studies: [Twin Study, Wiki] consider only the additive effect (no dominance, no epistasis): 
\begin{equation}
P = A + C + E	
\end{equation}
where $C$ is the contribution from the common environment and $E$ from the unique environment. Then for mono-zygotic (MZ) twins, the covariance is:
\begin{equation}
\Cov_{MZ} = \Cov(A + C + E, A + C + E') = V_A + V_C	
\end{equation}
For di-zygotic (DZ) twins, the covariance: 
\begin{equation}
\Cov_{DZ} = \Cov(A + C + E, A' + C + E') = \frac{1}{2} V_A + V_C	
\end{equation}
Solving $V_A$ and $V_C$: 
\begin{equation}
V_A = 2 (\Cov_{MZ} - \Cov_{DZ})	\qquad V_C = 2 \Cov_{DZ} - \Cov_{MZ}
\end{equation}
In terms of heritability: 
\begin{equation}
h^2 = 2 (r_{MZ} - r_{DZ})	
\end{equation}

\item Limitations: 
\begin{itemize}
	\item Shared environments among relatives: to control it, randomize the environment of samples in experimental populations. 
	\item Genotype-environment interactions. 
	\item Genetic interactions: among different loci. 
	\item Age and sex effects: e.g. parent-offspring regression, the traits may not be comparable, thus in general, half-sibs regression may be better. 
\end{itemize}
\end{itemize}

%%%%%%%%%%%%%%%%%%%%%%%%%%%%%%%%%%%%%%%%%%%%%%%%%%%%%%%%%%%%
\section{Genetic Mapping of Complex Traits}

Understanding genetics of complex trais: the fundamental problem is the genetic basis of individual variation. 
\begin{itemize}
	\item Genetic mapping: heritability of trait, finding the loci, environmental influences, the interactions among loci, etc. 
	\item Mechansims: molecular explanation of the genetics: causal loci, gene-gene interactions, gene-environment interactions, etc.  
\end{itemize}

Problems/challenges of quantitative trait mapping: 
\begin{itemize}
\item Reference: [Mackay \& Ayroles, NRG, 2009]
\item Identifying causal genes or CREs and QTNs (quant. trait nucleotide): not straightforward, as the QTL confidence interval or haplotype block map may be large. One idea: the allele frequency of SNP may indicate the importance, e.g. low frequency suggest selection, thus prioritize the SNPs among all candidates. 
\item Increasing the power to detect QTLs and QTNs: especially a problem if a huge number of hypothesis is tested (in the case of eQTL and dense markers). One idea is to group correlated traits. 
\item How QTNs affect phenotypes through molecular networks: joining eQTLs and QTTs (quant. trait transcripts). Idea: (1) causal inference procedure; (2) use co-expression network of QTTs. 
\end{itemize}

Strategies for human disease gene identification: mostly applicable to Mendelian traits [Human Molecular Genetics]
\begin{itemize}
	\item Positional evidence: through genetic linkage/association, also information such as chromosome abnormality (below), and expanded repeats (often found in disease genes in patients) can provide clues on the positional information. 
	\item Functional evidence: gene expression pattern, function in other species (animal model) or of paralogs, biochemistry and physiology, etc. Functional information is used for both priortization of genes (even with positional information), and for providing candidates. 
	\item Positional cloning: the process of starting with positional information and reaching the disease genes. The steps: candidate genes in the region, prioritization (functional information), mutation screening, and confirming the genes. 
	\item Confirming a candidate gene: require stronger evidence, most often mutation screening. Also restorization of function in vitro or the function in animal models. 
\end{itemize}

Challenges of personal genomics [personal thoughts]: 
\begin{itemize}
\item Finding causal variants: with whole-genome sequencing (WGS), very large number of variants may be found. How would causal variants of a complex trait be determined? This is particularly difficult given the heterogeneity of complex traits. Broad strategies: 
\begin{itemize}
\item Population genetics: it's possible that even though rare variants are rare in general population, they may be fairly common in patients. 
\item Recognize relations among genes to better deal with heterogeneity through (1) functional interactions among genes; (2) multiple variants/genes may affect the same intermediate/molecular traits. 
\item Stratification of phenotypes: once divide a phenotype into sub-phenotypes based on variables such as certain molecular signatures, the genetics of the sub-phenotype may be less heterogenous. 
\item Functional evidence of candidate loci: this is important for validating the findings. If additional functional data is available, e.g. the loci is associated with another trait known to be important for the disease (e.g. LDL for metabolic diseases). 
\end{itemize}

\item Molecular mechanism of causal variants: understanding how a genetic change (or multiple ones) leads to diseases (increase of disease risk). 
\begin{itemize}
\item Mutations in non-coding sequences: what are the targe genes? Mechanism of how non-coding mutation leads to change of target genes: e.g. TFBS change, nucleosomes, etc. 
\item Downstream effects: suppose $X$ is a causal gene (a regulatory protein), then $X$ may affect some functional genes (e.g. enzyme), which affect certain metabolite/stress level, and cause cell/tissue damage. Need to find such causal chain of events. Note that things such as stress level, cell damage may be reflected as levels of certain markers (e.g. DNA damage may be reflected by the level of DNA repair enzymes). 
\end{itemize}

\item Gene-environment interactions: 
\begin{itemize}
\item Inferrning causality from correlations between genotypes and different environmental variables. 
\item Recognizing interactions: e.g. the effect of genetic changes is only manifested in the presence of some environmental influence, and vice versa.  
\item Environmental effects may be manifested as changes of metabolites, gene expression, epigenetic states, etc. 
\end{itemize}

\item Prediction and personalization: risk prediction and prediction of response to treatments
\begin{itemize}
\item A better disease model would help prediction and personalization: e.g. in pharmacogenetics, both drug metabolism and drug target level could affect the efficacy of a drug, such knowledge can be encoded into some (non-linear) model. 
\item Better utilization of patient information: phenotype stratification, patient grouping, etc. 
\end{itemize}
\end{itemize}

Strategies of improving genetic mapping of complex traits [personal thoughts]: according to the general framework of regression analysis
\begin{itemize}
\item Incorporating more features: using imputation to add more SNPs; rare variants; CNVs. 

\item Feature expansion: epistasis between SNPs. 

\item Structure in features: haplotype and regional test; pathway analysis. 

\item Prior knowledge of features: SNP annotation (conservation, MAF, etc.); gene annotation and relations. 

\item Group structure: exploiting family structure; multi-population mapping and meta-analysis; phenotype stratification (by family, population, age of onset, etc.). 

\item Sparsity: SNP selection and causal variant analysis. 
\end{itemize}

Global analysis of genetic data: 
\begin{itemize}
\item Motivation: when we do not have confidence for individual genes, how do we draw conclusion about the overall genetics of a complex trait? This may happen, e.g. when we have case/control data where there are many weak loci; or we have de novo mutation data where most causal genes have only zero or one mutation. 

\item Enrichment pattern: some signals may be collectively detectable. Ex. (1) suppose we compute the statistic of each gene, and compare the distribution of the statistic of all genes with null distribution; (2) de novo mutation data: the total number of mutations across all genes may be higher than expected by chance. 

\item Genetic architecture: formal analysis of the data may allow one to infer genetic architecture of the trait, including the number of causal genes, the average effect size (or effect size distribution), and the variance explained by the data. 

\item Gene network analysis: patterns in the gene network, e.g. pathway test, whether candidate genes are highly connected, whethere they are linked to known genes, etc. 

\item Reference: [De novo mutations revealed by whole-exome sequencing are strongly associated with autism, Sanders, Nature, 2012], [Patterns and rates of exonic de novo mutations in autism spectrum disorders, Neale, Nature, 2012]

\end{itemize}

%%%%%%%%%%%%%%%%%%%%%%%%%%%%%%%%%%%%%%%%%%%%%%%%%%%%%%%%%%%%
%%%%%%%%%%%%%%%%%%%%%%%%%%%%%%%%%%%%%%%%%%%%%%%%%%%%%%%%%%%%
\chapter{Population Genetics}
\section{Genetic Polymorphism: an Overview} 

Ref: [Hartl \& Clark, Principles of Population Genetics, Chapter 1]

Challenges of population genetics: 
\begin{itemize}
\item Basic problems: how the pattern of genetic variations (SFS, heterozygosity, etc.) are related to the population history, mutational process and natural selection? How do we reconstruct the history from extant data (including the inference of time of events)? 
\item Strategy: we start with simple scenario, uniform population with one gene (two alleles), then solve more complex problems, non-random mating, population subdivision/migration, selection, recombination between multiple sites, and so on. 
\end{itemize}

Maintanence of polymorphism: 
\begin{itemize}
	\item Mutation-selection: most mutations are deleterious or nearly neutral, and are eliminated by natural selection (or random drift for neutral). 
	\item Balance: overdominance (heterozygosity is favored), frequency-dependent selection (rare alleles are favored), etc. 
\end{itemize}

Utility of polymorphism data: 
\begin{itemize}
	\item Polymorphism as markers: shared polymorphism may suggest shared ancestry. Thus polymorphism data can be used to infer the information about ancestry and history, e.g. migration/race, DNA fingerprinting and genealogy, phylogenetic tree reconstruction. 
	\item Inference of evolutionary process: population history (e.g. growth), selection on DNA sequences, disease association. 
\end{itemize}

Lessons/strategies of population genetics [peronsal notes]:
\begin{itemize}
\item Pattern of genetic relatedness or heterozygosity: basic process of random drift/inbreeding leads to increase of genetic relatedness or reduction of heterozygosity. So from pattern of relatedness, we can infer the underlying changes related to population size, population subdivision/migration, selection, and so on. 

\item Coalescence framework: the amount of genetic variations in a certain number of samples is due to mutations occuring during coalescence. So estimating coalesence time can lead to estimation of the amount of genetic variations. 
\end{itemize}

Questions of population genetics [personal notes]:
\begin{itemize}
\item Proof of expected SFS under infinite-sites model from coalescence:
\begin{equation}
\E(S_i) = \frac{\theta}{i}
\end{equation}
We start with $\E(S_1) = \theta$. One possible strategy is: after each coalesence event, ask how many nodes are internal and external, and count the contribution to the external lineage from both type of nodes. 

\item Defining $F_{ST}$ for genomewide data: from the relation of $F_{ST}$ and heterozygosity, we should define $F_{ST}$ using all loci, included the ones that are fixed in the population. However, comparing two populations, most of the sites should remain unchanged between the two and have $F_{ST} = 0$. Shall we include them? 

\end{itemize}

Departure of HWE: 
\begin{itemize}
\item HWE specificies the frequency of the heterozygotes, $P(Aa) = 2p(1-p)$, as a result of random mating. The variance of $X$ (the number of $a$'s in an individual) is given by: $\Var{X} = 2 p (1-p)$. 

\item Loss of heterozygosity (LOH): some factors may reduce the frequency of heterozygotes, creating departure from HWE. This may include: (1) recessive diseases: thus in patients, most are homozygotes $aa$; (2) inbreeding; (3) population admixture: e.g. mixture of two populations, one fixed with $A$ and the other fixed with $a$. The LOH due to population substructure is called the Wahlund effect. 
\begin{itemize}
	\item Consequence of LOH: increase of the variance of $X$. This can be understood as: due to LOH, the population has more $AA$ and $aa$'s, thus more deviation from the mean ($Aa$). 
\end{itemize}
\end{itemize}

Questions: 
\begin{itemize}
\item Genetic variation under neutrality: does inifite-site model incorporate recombination? What would be the level of variation considering mutation and recombination? 
\item Testing selection using polymorphism data: selective sweep vs. test based on neutral theory (e.g. Fu-Li test)?
\item Inferring age from polymorphism data: e.g. the HapMap common variants? 
\end{itemize}
%%%%%%%%%%%%%%%%%%%%%%%%%%%%%%%%%%%%%%%%%%%%%%%%%%%%%%%%%%%%
\section{Evolution of Infinite Populations} 

Ref: [Hartl \& Clark, Principles of Population Genetics; Felsenstein, Theoretical Population Genetics]

Goal: understand how evolutionary forces shape the long-term behavior of the population, as well as the dynamics of the changes. 

Mutation: 
\begin{itemize}
	\item Model: consider a single locus with two alleles $A$ and $a$. The mutation rates are $\mu$ ($A$ to $a$) and $\nu$ ($a$ to $A$) respectively. Let $p_t$ be the frequency of $A$ at generation $t$ (and $q_t$ be that of $a$), then $p_t$ can be solved with discrete difference equation:
\begin{equation}
p_t = \frac{\nu}{\mu + \nu} + (p_0 - \frac{\nu}{\mu + \nu}) (1 - \mu - \nu)^t
\end{equation}
	\item Result: because of recurrent mutations between the two, the population reaches equlibrium frequencies, whose ratios depend on the mutation rates. This should holds in the general case with multiple alleles. The rate of $p_t$ convergence to equilibrium is geometric, with rate $1 - \mu - \nu$.
\end{itemize}

Selection in haploids: 
\begin{itemize}
\item Discrete generations: let $p_A, p_a$ be the frequenices of $A$ and $a$ alleles, $w_A, w_a$ be the fitness of $A$ and $a$, and $p_A', p_a'$ be the frequencies in the next generation, we have: 
\begin{equation}
\frac{p_A'}{p_a'} = \frac{w_A}{w_a} \frac{p_A}{p_a} 
\end{equation}
Thus selection will drive the population towards the advantageous allele, with geometric rate equal to $w_A/w_a$. We could also write the dynamic equation in different ways, let $\bar{w} = w_A p_A + w_a p_a$, we have: 
\begin{equation}
p_A' = \frac{p_A w_A}{\bar{w}} \qquad p_a' = \frac{p_a w_a}{\bar{w}} 
\end{equation}
Or if we define $p = p_A$, we have the change of frequency in one generation: 
\begin{equation}
\Delta p = p (1-p) \frac{w_A - w_a}{\bar{w}}	
\end{equation}
The rate of change can be defined by the number of generations needed for the change of certain frequency: 
\begin{equation}
t = \frac{\ln \left(p_A^{(t)}/p_a^{(t)}\right) - \ln \left(p_A^{(0)}/p_a^{(0)}\right)}{\ln (1+s)}	
\end{equation}
which is roughly inversely proportional to $s$ when $s$ is small. 

\item Continuous generations: let $r_A, r_a$ be the growth rates of $A$ and $a$, we have: 
\begin{equation}
\frac{dp}{dt}	= (r_A - r_a) p (1-p)
\end{equation}
\end{itemize}

Selection in diploids: 
\begin{itemize}
\item Dynamic equations: similarly, the change of frequency can be written in three forms. Only need to replace $w_A$ and $w_a$ with the average fitness of the genotypes having at least one $A$ and $a$ respectively: $\bar{w}_A = p_A w_{AA} + p_a w_{Aa}$ and $\bar{w}_a = p_A w_{Aa} + p_a w_{aa}$. First, relative frequency ratio: 
\begin{equation}
\frac{p_A'}{p_a'} = \frac{\bar{w}_A}{\bar{w}_a} \frac{p_A}{p_a} 
\end{equation}
Second, the frequency in terms of average frequency: 
\begin{equation}
p_A' = \frac{p_A \bar{w}_A}{\bar{w}} \qquad p_a' = \frac{p_a \bar{w}_a}{\bar{w}} 
\end{equation}
Third, the change of frequency in one generation: 
\begin{equation}
\Delta p = p (1-p) \frac{\bar{w}_A -\bar{w}_a}{\bar{w}}	
\end{equation}

\item Multiplicative cases: the fitness of three genotypes are: $AA- (1+s)^2, Aa- (1+s), aa- 1$. The result is exactly the same as the haploid case with selection coefficient $s$. 

\item Recessive and dominant case: (1) Recessive: $AA- 1+s, Aa- 1, aa- 1$; (2) Dominant: $AA- 1+s, Aa- 1+s, aa- 1$. 

\item Overdominance [Nielsen \& Slotkin, Chapter 7]: the fitness of $Aa$ is higher than both $AA$ and $aa$, in this case, a stable polymorphism will be established. Overdominance is a special form of balancing selection. With overdominance, the frequency reaches equilibrium (stable polymorphism): 
\begin{equation}
\hat{f}_A = \frac{s_{aa}}{s_{AA} + s_{aa}}
\end{equation}

\item Underdominance: the fitness of $Aa$ is lower than $AA$ and $aa$, the population will be fixed on $A$ or $a$, depending on the initial frequency. 
\end{itemize}

Dynamics of fitness: how the average fitness of the population changes over time. Intuitively this should always increase. 
\begin{itemize}
\item Fisher's Fundamental Theorem of Natural selection: we consider haploid population with $k$ possible alleles, we have:  
\begin{equation}
p_i' = \frac{p_i w_i}{\bar{w}}	
\end{equation}
Use the equations of $\bar{w}$ and $\bar{w}'$: 
\begin{equation}
\bar{w}' - \bar{w} = \sum_i p_i' w_i - \sum_i p_i w_i = \frac{1}{\bar{w}} \left[ \sum_{i=1}^k p_i w_i^2 - \bar{w}^2\right]	
\end{equation}
Thus we have the basic equation of Fisher's Fundamental Theorem of Natural selection:
\begin{equation}
\Delta \bar{w} = \frac{\text{Var}(w)}{\bar{w}}	
\end{equation}
Thus, the increment of the mean population relative fitness is the ratio of the genetic variance in fitness to the mean fitness. This immediately implies that $\Delta \bar{w}$ will never decrease. 

\item Adaptive topography and fitness optimization: the function $\bar{w}(p)$ shows how the average fitness changes with the frequency of $A$ in the population, called ``adaptive topography''. Thus the evolution of the population can be visualized as the movement in the adaptive topography. This leads to an alternative way of showing $\Delta \bar{w}$ never decreases. For the diploid case, we can derive the following equation: 
\begin{equation}
\Delta p = \frac{p(1-p)}{2\bar{w}} \frac{d\bar{w}}{dp}	
\end{equation}
Then we have: 
\begin{equation}
\Delta \bar{w} = 	\frac{d\bar{w}}{dp} \cdot \Delta p = \left( \frac{d\bar{w}}{dp} \right)^2 \frac{p(1-p)}{2\bar{w}} > 0
\end{equation}
Therefore, the behavior of the population can be described as fitness optimization: $p$ will always move towards the direction where $\bar{w}(p)$ increases. However, this does not guarantee that the global optimum of $\bar{w}$ can be reached, and the final equilibrium depends on the initial frequency. 

\item Generality of the results: even though the Fundamental Theorem of Natural Selection and monotonicity of average fitness are proven only for simple cases, they are valid for much more general cases (the case of multiple alleles). 
\end{itemize}

Selection in multiple alleles: 
\begin{itemize}
\item Dynamics: let $p_i$ be the frequency of the $i$-th allele, the dynamics: 
\begin{equation}
p_i' = \frac{p_i \bar{w}_i}{\bar{w}}	
\end{equation}
which leads to: 
\begin{equation}
\Delta p_i = p_i \frac{\bar{w}_i - \bar{w}}{\bar{w}}	
\end{equation}

\item Equilibrium and fitness: at equilibrium, either $p_i = 0$ or $\bar{w}_i = \bar{w}$. However, not all equilibrium points are stable. Because $\bar{w}$ is nondecreasing, the local maximums must be stable equilibrium. 

\end{itemize}

Mutation-selection balance: [Hartl, Section 5.4; Felsenstein, Chapter 3]
\begin{itemize}
\item Haploid model: suppose the fitness of $A$ and $a$ are $1$ and $1 - s$ respectively. Suppose the mutation of $A$ to $a$ is $\mu$, and the back-mutation can be ignored. Suppose at $t$ generation, the frequency of $A$ is $p$, then after selection, its frequency becomes: 
\begin{equation}
p^* = \frac{p}{1 - (1-p)s}	
\end{equation}
This follows from a simple selection model of two genotypes. And after mutation, its frequency becomes: 
\begin{equation}
p' = p^* (1 - \mu)	
\end{equation}
Putting the two together, we have, in $t+1$ generation, the frequency of $A$ is: 
\begin{equation}
p' = 	\frac{p (1 - \mu)}{1 - (1-p)s}
\end{equation}
At equilibrium, we have $p' = p$, solving this equation, and we have the frequency of $a$ allele ($1 - p$):  
\begin{equation}
q_e = \mu / s	
\end{equation}

\item Diploid model: consider the case where $A$ is advantageous. Let the fitness of $AA$, $Aa$ and $aa$ be 1, $1 - hs$ and $1 -s$, respectively. Suppose the rate of $A$ to $a$ mutation is $\mu$, and the back-mutation can be ignored because of the low frequency of $a$ in the population. Because of recurrent mutations, $a$ cannot be completely eliminated. Let $\hat{q}$ be the equilibrium frequency of $a$, we have two cases: if $a$ is completely recessive ($h = 0$): 
\begin{equation}
\hat{q} = \sqrt{\frac{\mu}{s}}	
\end{equation}
If $h > 0$, then an approximation: 
\begin{equation}
\hat{q} = \frac{\mu}{hs}	
\end{equation}

\item Interpretation: when $\mu$ is very small (e.g. $2N\mu << 1$), the number of mutations per generation is much less than 1, and most will not survive next generations, so the population is essentially fixed at the advantageous allele $A$. In general, if $2 N \mu$ is not substantially less than 1, then there will be more mutations created in each generation, and natural selection will reduce, but not completely eliminate all of them. So at equalibrium, there will be deleterious allels, and its frequency will depend on the ratio of (1) the rate of introducing new mutations, $\mu$; and (2) the rate of eliminating mutations by selection ($\ln (1-s)^{-1} = s$). 

\end{itemize}
%%%%%%%%%%%%%%%%%%%%%%%%%%%%%%%%%%%%%%%%%%%%%%%%%%%%%%%%%%%%
\section{Evolution of Finite Populations} 

Ref: [Hartl \& Clark, Principles of Population Genetics; Felsenstein, Theoretical Population Genetics]. An introduction to population genetic theory [Crow \& Kimura]

Random drift: Wright-Fisher model with no selection and recurrent mutation: 
\begin{itemize}
\item Intuition of the process: at each generation, there is always a probability that some individuals leave no offspring. Thus over time, there is a growing tendency that all individuals come from a small number of ancestors; or the individuals are more likely to be genetically related over time. 

\item Inbreeding perspective: random drift will drive fixation of one allele in the population. This can be understood as the inbreeding effect: inbreeding is more likely within a finite population. Consider a population with two alleles, let $f_t$ be inbreeding coefficient (the probability that two random gametes are IBD), then: 
\begin{equation}
f_t = \frac{1}{2N} + (1 - \frac{1}{2N}) f_{t-1}
\end{equation}
Solving this equation: 
\begin{equation}
f_t = 1 - (1 - \frac{1}{2N})^t	
\end{equation}
Or equivalently, the heterozygosity (the probability that the two random gametes are different): 
\begin{equation}
h_t = (1 - \frac{1}{2N})^t	
\end{equation}
Thus the heterozygosity decreases at the rate of $(1 - 1/2N)$ and approaches 0 as $t \rightarrow \infty$. 

\item Probability of fixation: equal to $p$ when the initial frequency of the allele is $p$. Specifically, for new mutation, it is $1/2N$. The intuition is (coalescence): eventually, the population will be fixed with one ancestor, but since the chance is equal for all $2N$ ancestral alleles, the probability of fixing the allele of interest is $1/2N$. Another way to derive this: let $X_{t}$ be the frequency of $A$ alleles in time $t$, we have: 
\begin{equation}
E(X_t) = X_{t-1}
\end{equation}
from the Wright-Fisher process. Thus: $E(X_t) = \cdots = E(X_0) = p$, where $p$ is the initial frequency. 

\item Time to fixation: for a new mutation that is fixed, the time to fixation is approximately $4N$ generations. This can be understood from coalescence. For a new mutation that is lost, the time of loss is approximately $2 \ln(2N)$. 
\end{itemize}

Mutation-drift balance: [Hartl, Section 5.7; Felsenstein, Chapter 7] 
\begin{itemize}
\item Homozygosity under infinite-alleles model: with infinite allele model, the homozygosity is equal to the inbreeding coefficient, as every two alleles of the same type must be IBD. Let $F_t$ be the homozygosity at time $t$, then at time $t+1$, two random gametes have the same alleles if the same individual is sampled in $t$, or the two sampled individuals are IBD in $t$, and no mutations happened: 
\begin{equation}
F_{t+1} = (1 - u)^2 \left[ \frac{1}{2N} + \left(1-\frac{1}{2N}\right) F_t \right]	
\end{equation}
At equlibrium: 
\begin{equation}
F \approx \frac{1}{1 + 4 N u}	
\end{equation}

\item Homozygosity under finite alleles: this could be analyzed similarly. Suppose there are $K$ different alleles, the equilibrium homozygosity is given by: 
\begin{equation}
F \approx \frac{1 + 4 N u / (K - 1)}{1 + 4 N u K / (K - 1)}	
\end{equation}

\item Two-allele model: suppose there are recurrent mutations between $A$ and $a$. The idea is to analyze the fate of any new mutations. 
\begin{itemize}
\item $4 N \mu \leq 1$: the PDF of the allele frequency is bi-modal at $x = 0$ or $x =1$. At each generation, there are $2 N \mu$ mutations, but only $1/(2N)$ of these if fixed, so on average, $\mu$ mutations will be fixed per generation, or in other words, one mutation per $1/\mu$ generations will be fixed. On the other hand, each fixed mutation takes $4N$ generations to fix, thus if $4 N \leq 1/\mu$ (fixation occurs before new to-be-fixed mutations), the population will remain fixed most of the time. In general, if there are multiple alleles with possible mutations among each other, the population will follow a Markov chain of these alleles. 

\item $4 N \mu > 1$: with higher mutation rates, the probability that a popluation will be polymorphic is higher (the PDF of the allele frequency is single-peaked at intermediate frequency) - mutational equlibrium. 
\end{itemize}
\end{itemize}

Diffusion approximation of Wright-Fisher model: [Hartl, Section 3.2, 3.3]
\begin{itemize}
\item Model: let $\phi(p,x;t)$ be the pdf of the allele frequency at $x$ in time $t$, when starting at $x(0) = p$. Let $M(x)$ be the mean of the change of allele frequency ($\Delta x$) in one generation, and $V(x)$ be the varirance of $\Delta x$ in one generation. $M(x)$ is driven by the systematic force such as selection and $V(x)$ by random drift. 

\item Kolmogorov forward equation: analyze $\phi(p,x;t)$ in terms of the frequency at $t - \Delta t$, which could be $x - \Delta x$, $x + \Delta x$ and $x$. Multiply the probability of the three cases (time 0 to time $t - \Delta $) and the probability of transition from the three case to $x$ (in $\Delta t$), we could obtain the forward equation. 

\item Kolmogorov backward equation: starting from time $t = 0$, at time $\Delta t$, the frequency could be $p - \Delta p$, $p$ and $p + \Delta p$. Similarly, multiply the probability of transition to three cases (in time $\Delta t$) and the probability of generating the final frequency from these three cases (time $\Delta t$ to $t$), we could obtain the backward equation. 

\item Probability of fixation: at fixation, the time derivative is 0, and solve $u(p)$ in Kolmogorov backward equation (ODE of a single variable). 

\end{itemize}

Mutation-selection-drift: [Section 5.7]
\begin{itemize}
\item Model: consider the life cycle of a population. Starting from zygotes, selection determines the frequency of genotypes before reproduction, then mutation changes the frequencies when producing gametes, and random drift (sampling) happens at the step of selecting zygotes for the next generation. 
\item Probability of fixation for finite populations: starting at frequency $p$:
\begin{equation}
u(p) = \frac{1 - e^{-4 Nsp}}{1 - e^{-4Ns}}	
\end{equation}
The probability of fixation of a newly arised mutation is given by: 
\begin{equation}
u(1/2N) = \frac{1 - e^{-2s}}{1 - e^{-4Ns}} \approx \frac{2s}{1 - e^{-4Ns}}	
\end{equation}
Unless for extremely weak selection (say $4Ns < 2$), most new beneficial mutations are fixed with probability $2s$. The probability of fixation of deleterious mutations is low, however, heterozygosity of deleterious mutations can be quite high (as selection against heterozygotes is low or no selection). 

\item Fixation time: (not provided in the text) favorable selection will be fixed faster, however, the effect is limited. Most of fixtation time is spent at the beginning (coalescence process), however, selection is the weakest at the beginning. Thus fixation time would not be too different from $4N$.  
\end{itemize}

Relative strength of the evolutionary forces: [Felsenstein, Theoretical Population Genetics, VII.10]
\begin{itemize}
\item Intuitions of evolutionary forces: 
\begin{itemize}
\item Time scale: defined according to how fast allele frequency is changed by these forces. Mutation - $1/\mu$, selection - $1/s$ (in fact, valid only when the allele reaches intermediate to large frequency), random drift - $4N$. Comparison of two time scales could suggest which one dominates, when two forces are compared.
\item Selection: most effective when the allele frequency is large. At the beginning (e.g. a new favorable mutation occurs), selection is not very effective. 
\end{itemize}
 
\item Scenarios: consider two alleles with mutation rates $\mu$ being equal between the two. 
\begin{itemize}
	\item $4N \mu > 1$: there will be more than 1 new mutation in each generation, and this is faster than what random drift or selection can eliminate (selection is weak at the beginning), thus there will be substantial polymorphism, even in the presence of selection. This leads to mutation-selection balance or simply mutation balance (when selection is weak).  
	\item $4N \mu \leq 1$: either dominated by random drift (at weak selection, $\lvert 4Ns \rvert < 10$) or selection (when $\lvert 4Ns \rvert > 10$). In either case, population tend to be fixed (unless with overdominant selection that favors heterozygotes). 
	\item An important case: weak selection of finite population (nearly neutral theory). This is applicable in small populations, e.g. human. The fate of any new mutations is: initially, dominated by random drift (a small number of $a$ alleles, thus chance plays the important role); as the $a$ alleles accumulate to higher frequency, selection will stop it from getting higher (at large number of $a$ alleles, selection more important). Therefore, the probability of fixation of slightly deleterious mutations is very low, but the heterozygosity may be substantial.
\end{itemize}

\end{itemize}

\subsection{Diffusion model of Wright-Fisher process} 

Reference: [Sethupathy \& Hannenhalli, PRF tutorial, 2007], [Bustamante, Directional Selection and the Site-Frequency Spectrum, Genetics, 2001]

Goal: given a gene in a population under selection, mutation and random drift, how often a mutation is fixed or eliminated? What is the average time of absorption? In general, characterize the change of genetic composition. 

Model: the fitness of the 2 alleles are 1, and $1+s$ respectively. Let $f(x;,p,t)$ be the probability density function of the frequency of the derive allele ($x$) at time $t$ when the initial frequency equals to $p$. Then it can be computed by summing over all possible daf in the previous generation conditioned on the same inital frequency $p$: 
\begin{equation}
f(x+\delta x; p, t + \delta t) = \int_0^1{f(x;p,t) f(x + \delta x; x, \delta t) dx}
\end{equation}
Perform Taylor series expansion on both sides in $\delta t$ and $\delta x$ to derive the Kolmogorov forward equation: 
\begin{equation}
\frac{\partial{f(x;p,t)}}{\partial{t}} = \frac{\partial^2[b(x)f(x;p,t)]}{2 \partial x^2} - \frac{\partial[a(x)f(x;p,t)]}{\partial x}
\end{equation}
where: $a(x) dt = E(dx)$ and $b(x) dt = Var(dx)$, i.e. they are expectation and variance of the change of daf in one generation. In the neutral case, $a(x) = 0$, and $b(x) = x(1-x)/(2N)$. Similarly, we could compute $f(x,p,t)$ by summing over all possible daf in the previous generation conditioned on the same future daf $x$: 
\begin{equation}
f(x; p, t + \delta t) = \int_0^1{f(x;p + \delta p,t) f(p + \delta p; p, \delta t) d(\delta p)}
\end{equation}
Again do Taylor expansion and derive the Kolmogorov backward equation: 
\begin{equation}
\frac{\partial{f(x;p,t)}}{\partial{t}} = b(p)\frac{\partial^2[f(x;p,t)]}{2 \partial p^2} + a(p)\frac{\partial[f(x;p,t)]}{\partial p}
\end{equation}

Probability of extinction: the basic idea is to expression the extinction probability in terms of the function $f(x;p,t)$. In fact: the probability that daf is 0 at time $t$ given initial daf is $p$ can be expressed as:  
\begin{equation}
P_0(p,t) = \int_0^{0^+}{f(y;p,t)dy} = F(0^+; p,t)
\end{equation}  
where $0^+$ indicates $0+\epsilon$ ($\epsilon$ is a very small number) and $F(x;p,t)$ is the cdf of the function $f(x;p,t)$. Integrating over $dx$ in both sides of the Kolmogorov backward equation and plug in $x = 0^+$: 
\begin{equation}
\frac{\partial P_0(p,t)}{\partial t} = b(p) \frac{\partial^2[P_0(p,t)]}{2 \partial p^2} + a(p) \frac{\partial [P_0(p,t)]}{\partial p}
\end{equation} 
Let $t \rightarrow \infty$, then LHS should be $0$: 
\begin{equation}
0 = b(p) \frac{\partial^2[P_0(p)]}{2 \partial p^2} + a(p) \frac{\partial [P_0(p)]}{\partial p} 
\end{equation}
which could be analytically solved. 

Probability of fixation: similar to the case of extinction, one could solve $P_1(p)$. Assuming selection, but no dominance, no recurrent mutation, $a(x) = \gamma x(1-x)$, $b(x) = x(1-x)$ where $\gamma = 2N_e s$ ($N_e$ is the effective population size), this probability is: 
\begin{equation}
\label{eq:fixation_prob}
P_1(p) = \frac{1 - e^{-4N_e s p}}{1 - e^{-4N_e s}}
\end{equation}

Mean time of fixation or extinction: the basic idea is to relate the mean time of absorption to the fixation and extinction probabilities. Let $\phi(p,t)$ be the pdf of the absorption time, then: 
\begin{equation}
P_0(p,t) + P_1(p,t) = \int_0^t{\phi(p,t)dt}
\end{equation}
Or equivalently: 
\begin{equation}
\phi(p,t) = \frac{\partial [P_0(p,t) + P_1(p,t)]}{\partial t}
\end{equation}
Plug in the equations of $P_0(p,t)$ and $P_1(p,t)$ and take derivative of $t$: 
\begin{equation}
\frac{\partial \phi(p,t)}{\partial t} = b(p) \frac{\partial^2[\phi0(p,t)]}{2 \partial p^2} + a(p) \frac{\partial [\phi(p,t)]}{\partial p}
\end{equation} 
The average time of absorption $\overline{t}(p) = \int_0^{\infty}{t \phi(p,t)dt}$, could be obtained from this equation via integration by parts (see Text): 
\begin{equation}
-1 = b(p) \frac{d^2 \overline{t}(p)}{2 dp^2} + a(p) \frac{d \overline{t}(p)}{dp}
\end{equation}
Define $t(p,x)dx$ be the mean time that the daf spends in the interval $(x, x+dx)$ before absorption occurs, then $\overline{t}(p) = \int_0^1{t(p,x)dx}$, could solve $t(p,x)$ (see Text). Suppose $p = \frac{1}{2N_e}$ [Bustamante, Genetic, 2001], we have
\begin{equation}
\label{eq:daf_mean_time}
f(x) = t(p,x) \approx \frac{1 - e^{-2 \gamma (1-x)}}{(1 - e^{-2 \gamma})} \frac{2}{x(1-x)}
\end{equation}
where $f(x)dx$ is the expected time for which the daf is in the range $(x,x+dx)$ before absorption. If $s = 0$: $f(x) = \frac{2}{x}$. This can be shown by applying L'Hopital rule: limit of $f(x)$ when $\gamma \to 0$. 

Remark: 
\begin{itemize}
	\item The derivation of $t(p,x)$ from the equation of $\overline{t}(p)$ is still unclear.
	
	\item The equation of $f(x)$ is from Bustamante. The one from the PRF tutorial has a constant $N_e$ instead of 2 - probably wrong. 
\end{itemize}
 

%%%%%%%%%%%%%%%%%%%%%%%%%%%%%%%%%%%%%%%%%%%%%%%%%%%%%%%%%%%%
\section{Coalescence Theory} 

Reference: [Nielsen \& Slatkin, An Introduction to Population Genetics, Chapter 3. Hartl \& Clark, Principles of Population Genetics, Chapter 3]

Motivation: understanding the patterns of genetic variations, how they are related to the underlying population genetic processes/parameters. Ex. 
\begin{itemize}
\item Given two samples: the difference between two samples. Heterozygosity in a set of samples. 
\item Given many samples: how many variants are singletons, doubletons, etc. In general, the site-frequency spectrum. 
\end{itemize}

Coalescence in a sample of two ($n = 2$): 
\begin{itemize}
\item Motivation: (1) it is difficult to study the forward process, e.g. we can ask how long it takes to an individual to have more than one offspring. The number of offsprings of any indiviual follows $\text{Binom}(2N, 1/2N)$, which can be difficult to track. (2) Backward thinking is easier: every individual must have a parent, and it has probability $1/2N$ to be any specific one. So we trace the ancestors of each gene copy.

\item Coalescence process: the coalescence events occur at a rate of $1/2N$ - at each generation, the probability that two offsprings share the same parent is $1/2N$. The waiting time for the coalesence is geometric distribution with mean $2N$. The continuous version is that the coal. process occurs at a rate $1/2N$, and the waiting time is expoential distribution with rate $1/2N$. 

\item Coalescence with mutation: infinite-site model. We make the assumption that coalecence and mutational processes are independent. This is not always true, e.g. when there is selection, the coalescence rate will not be a constant. We now have two processes in parallel then: coalescence and mutation. Under this assumption, the number of difference between two samples, $\pi$, is the number of mutations that have occurred since their most recent common ancestor (MRCA): 
\begin{equation}
\E(\pi) = 2N \times 2 \mu = 4 N \mu = \theta	
\end{equation}
where $2N$ is the expected time of coalesence, and $2 \mu$ is the mutation rate. $\pi$ is thus called Tajima's estimator of $\theta$. Furthermore we can obtain the distribution of the number of segregating sites $S$: the probability of $S = j$ is the probability of $j$ mutational events occur before the coalescence. The two processes (coal. and mutation) occur at rates $1/2N$ and $2\mu$, respectively, thus the probability mutation occurs befores coal. is $\theta/(1+\theta)$. Effectively, we have $j$ mutational. events before coal., and this probability is given by geometric distribution: 
\begin{equation}
P(S = j) = \frac{1}{1 + \theta} \left( \frac{\theta}{1+\theta}\right)^j
\end{equation}

\item Coalescence with mutation: infinite-allele model. Homozygosity between two copies is equivalent to coalescence before mutation. This probability depends on the relative rates of the two: 
\begin{equation}
p = \frac{1/2N}{1/2N + 2 \mu} = \frac{1}{1 + \theta}	
\end{equation}
The expected heterozygosity is:
\begin{equation}
H = \frac{\theta}{1 + \theta}
\end{equation}
This can be viewed as a measure of the amount of genetic variation in a sequence: more variation, higher $H$. The value of $H$ is monotonically dependent on $\theta$: higher $\theta$, larger $H$.  
 
\item Effective population size: defined based on coalesence rate. Ex. given a population of changing size, we find the effective population size s.t. the expected rate equals to the rate under the changing size. Suppose we have a population that fluctuates in sizes: it has size $N_i$ in the $p_i$ proportion of time. Then we have: 
\begin{equation}
\frac{1}{2N_e} = \frac{p_1}{2 N_1} + \cdots + \frac{p_k}{2 N_k}
\end{equation}
So $N_e$ is the harmonic mean of $N_i$'s. 

\end{itemize}

Coalescence in a sample of $n$:
\begin{itemize}
\item Bugs-in-a-box analogy of coalescence [Hartl \& Clark]: random movement of $n$ bugs in a box of size $2N$ (i.e. $2N$ grids), when two bugs meet at the same grid, one of them will eat another (thus reducing the number of bugs). The process of going back in time can be modelled as the dynamic process of the change of bug numbers over time. 

\item Coalescence tree: the basic process is that any pair coalescences with rate $1/2N$. Which pairs coalscenced at each step is random, so the shape of the tree is stochastic. We are interested in the time of coalescence (at each step and overall) to estimate the amount of genetic variations (mutations in the coalescence tree). Suppose we have $k$ samples, the coalescence from $k$ to $k-1$ samples happens at rate: $1/2N$ times $k(k-1)/2$ (the number of pairs). Thus the expected time of coalescence when there are $k$ samples is: 
\begin{equation}
\E(T_k) = \frac{1}{\frac{1}{2N} \cdot \frac{k(k-1)}{2}} = \frac{4N}{k(k-1)}	
\end{equation}
The variance of $T_k$ is: $\Var(T_k) = \frac{16 N^2}{[k(k-1)]^2}$. The time to the most recent common ancestor (MRCA) is the sum of $t_k$: 
\begin{equation}
\E(T_{\text{MRCA}}) = \sum_{k=2}^n \E(T_k) = 4N \sum_{k=2}^n \frac{1}{k(k-1)} = 4N \left( 1 - \frac{1}{n}\right)
\end{equation}
The total tree length is the sum of all lineages: from $k$ to $k-1$, each of the $k$ individuals has a waiting time of $T_k$
\begin{equation}
\E(T_{\text{Tree}}) = \sum_{k=2}^n k \E(T_k) = 4N \sum_{k=1}^{n-1} \frac{1}{k}
\end{equation}
The expected number of segregating sites is: 
\begin{equation}
\E(S) = \mu \cdot	\E(T_{\text{Tree}}) = \theta \sum_{k=1}^{n-1} \frac{1}{k}
\end{equation}
This leads to Watterson's estimator of $\theta$: the number of segregating sites in $n$ samples divided by $\sum_{k=1}^{n-1} \frac{1}{k}$. 

\item Site frequency spectrum (SFS): we are interested in $S_i$, the number of segregating sites with frequency $i$ (i.e. occuring in $i$ out of $n$ samples). First, consider singletons. The number of singletons is equal to the number of mutations in the external lineage (lineage leading to a leaf node). The expected total length of external lineage is $4N$, so we have: 
\begin{equation}
\E(S_1) = \mu \cdot 4N = \theta
\end{equation}
More generally, we have $\E(S_i) = \frac{\theta}{i}$. 

\item Tree shape as a function of population size: the tree shape reflects the pattern of genetic variations, and it depends on population size. When population size is small, the coalescence rate is high, thus the lineages are short. Thus one can infer the population size changes from the tree shapes (which need to be consistent with the pattern of genetic variations). 
\begin{itemize}
\item With exponential growth of population size, the branches near the root are shorter. In the extreme case, the tree resembles a star phylogeny: all lineges are persistent. 
\end{itemize}
\end{itemize}

Coalescence with recombination: 
\begin{itemize}
	\item Ancestral recombination graph: the history of mutations and recombinations. With low mutation rates, it is possible to resolve recombination; not possible with high mutation rates (indistinguishable). 
	\item Inference: determine the coalescence tree consistent with sample is extremely difficult - random trees will not match the sample properties, e.g. LD patterns. Sampling techniques such as MCMC have been used. 
	\item Estimation of recombination rates: if coalesence approach is not applicable, one could compute some summary statistic that contains information of recombination rate. Ex. $4Nr$ (where $r$ is the recombination rate) is reflected in the value of $\text{Var}(S)$, where $S$ is the number of mismatches in every pair of nucleotide sequence (infinite-site model): recombination will reduce the variance of $S$. 
\end{itemize}
%%%%%%%%%%%%%%%%%%%%%%%%%%%%%%%%%%%%%%%%%%%%%%%%%%%%%%%%%%%%
\section{Inbreeding, Population Subdivision and Inference of Population History} 

Reference: [Nielsen \& Slatkin, Chapters 4-5; Felsenstein, Chapter 5; Hartl \& Clark, Chapter 9,10; Laird \& Lange, Chapter 3]

Motivating problems: we are generally interested in inferring the population history from extant pattern of genetic variations. Some examples: 
\begin{itemize}
\item Given one population, infer the history of population size changes. 

\item Given two populations (e.g. European and African), infer the time of divergence; and/or estimating the level of genetic flow/migration between the two. 

\item Given many populations, inferring the history of divergence among these populations (the relationship), and the ancestral alleles of MRCA. 

\item For each of these problems, it is important to keep in mind that the inference (pattern of genetic variations) can be complicated by factors such as population size changes and selection. 
\end{itemize}

General strategies for inferring population history and demography [personal notes; Nielsen \& Slotkin, Chapter 5]
\begin{itemize}
\item Pattern of genetic variations reveal population history: intuitively, samples from the same (sub)population (more recent common ancestor) are more genetically similar than samples of different populations, thus one can study the genetic similarity between samples to infer history. 
\begin{itemize}
\item Formally, suppose we have two populations, then the average difference between samples of the same population is $\pi_s = \theta$ and between samples of the different populations is $\pi_D = 2 \mu T + \theta$ where $T$ is the divergence time. 
\end{itemize}
	
\item Explorative analysis: instead of modeling the population genetic process, study the relationship between samples to reveal their similarity and infer the history. Ex. cluster samples by their genetic similarity to find samples with common ancestry. One could have a probabilistic model, but the model may be phenomenoloical, rather than population genetic. 

\item Summary statistics: e.g. pairwise difference $\pi$, number of segregating sites $S$, $F_{ST}$, homozygosity. Their values often reflect the underlying structure/process. One often derives the expectation of these values under a population genetic model, and compare that with the observed values. To determine the distribution of parameters, one can use simlation (see below). 

\item Likelihood: directly model the genetic data to infer the parameters. The challenge is generally the unknown coalescence trees linking the samples, and may need expensive sampling approach. 
\end{itemize}

Inbreeding: 
\begin{itemize}
\item Inbred lines: the goal is to create experimental animals/plants that are genetically identical. Procedure: long-term inbreeding. If starting population is heterozygous (e.g. $Aa$), then at each generation, the fraction of heterozygotes will be reduced by half according to Mendel law of segregation. Thus over long time, most loci will become homozygous.
	
\item The inbreeding coefficient of an individual is the probability that the two gene copies present at a locus in that individual are identical by descent, relative to an appropriate base population. In the base population, all gene copies are assumed not to be identical by descent.

\item Inbreeding effect: given the allele frequency of the base population in HWE, $p$ for the allele $A$, and the inbreeding coefficient ($f$), we could calculate the genotype frequencies. Thus a genotype is $AA$ if two alleles are IBD and the ancestral allele is $A$, or if two alleles are not IBD and both ancestral alleles are $A$, and similarly for other genotypes: 
\begin{equation}
\begin{array}{ll}
AA: & p^2 (1-f) + pf\\
Aa: & 2p(1-p) (1-f)\\
aa: & (1-p)^2 + (1-p)f
\end{array}
\end{equation}
With inbreeding, the value of $f$ will increase over generations, and eventually	reaches $1$, i.e. most are homozygotes.  

\item Loss-of-heterozygosity (LOH) for inbreeding: heterozygotes occur when the two alleles are not IBD and are different: 
\begin{equation}
P(Aa) = 2 p (1-p) (1 - F)	
\end{equation}
It can be shown that the variance of $X$ (number of $a$'s individuals) is increased: 
\begin{equation}
\Var{X} = 2 p (1-p) (1+F)	
\end{equation}
\end{itemize}

Population subdivision: Loss-of-heterozygosity. 
\begin{itemize}
\item Intuition of inbreeding and population subdivision: if for any reason, genetic exchange is limited to a smaller population, then there is always the effect of random drift, or increase of genetic relatedness (or LOH) over time. We can thus study LoH or genetic relatedness to explore/define inbreeding and population subdivision. 

\item Concepts: genetic differntation (difference of allele frequencies) defines subpopulations. Many populations have hierarchical population structure. 

\item Reduction of heterozygosity due to population subdivision (Wahlund effect): assume mating only occurs within subpopulations, then this is similar to inbreeding and reduces heterozygotes. Thus population subdivision will cause departure of HWE. 

\item Wright's $F$ statistic: to measure the reduction of heterozygosity, define: 
\begin{equation}
F_{ST} = \frac{H_T - H_S}{H_T}	
\end{equation}
where $H_S$ is the average heterozygosity of all subpopulations, and $H_T$ is the heterozygosity of the total population. In general, if $F_{ST} < 0.05$: very little genetic differentation; 0.05 to 0.15: moderate genetic differentiation, etc. 

\item Causes of population subdivision: debate over natural selection (adaptation to local environment) or random drift (including fixation of different founder alleles). 

\item Analysis of LOH for population stratification: suppose there are $K$ strata in the population, with allele frequencies $p_k$ each. The heterozogosity is: 
\begin{equation}
P(Aa) = 2 p (1-p) - 2 \Var{p_k}	
\end{equation}
where $\Var{p_k}$ is due to the difference of $p_k$ across strata. The variance of $X$ is: $\Var{X} = 2 p (1-p) + 2 \Var{p_k}$. 
\end{itemize}

Population substructure: defined by the difference of allele frequencies across the population. Several types:
\begin{itemize}
\item Population stratification: within a population, the individuals can be subdivided into multiple geographical/ethnic groups. 
\item Population admixture: mixture of populations with different ancestry, often due to migration. 
\item Population inbreeding: due to isolation of subgroups. The inbreeding coefficient $F$, is the probability that the two alleles from an individual is IBD. An extreme case is self-fertilizing plant, $F = 1$. 
\end{itemize}

Migration model: 
\begin{itemize}
\item Wright-Fisher model of migration: two populations that exchange genes, and we study how AFs change at each population. Let $m_{1 \rightarrow 2}$ be the probability that any gene copy in population 2 in one generation is replaced by some one in population 1, and similarly we have $m_{2 \rightarrow 1}$. Let $f_{A_1}$ and $f_{A_2}$ be the frequencies of $A$ in two populations. Then we have the recurrence of the allele frequency at generation $t+1$: 
\begin{equation}
\E[f_{A_1}(t+1)] = (1 - m_{2 \rightarrow 1}) f_{A_1}(t) + m_{2 \rightarrow 1} f_{A_2}(t)	
\end{equation}
The idea is that at each generation in population 1, each copy is either a migrant from population 2, or not. Similarly, we have: 
\begin{equation}
\E[f_{A_2}(t+1)] = (1 - m_{1 \rightarrow 2}) f_{A_2}(t) + m_{1 \rightarrow 2} f_{A_1}(t)	
\end{equation}
At equilibrium, we have $f_{A_1} = f_{A_2}$: eventually two population exchanging a lot of migrants will have the same allele frequencies. 
\begin{itemize}
\item Remark: there are variations of the basic scenarios of migration. For example, in the case of immigration, a small population is merged within a much larger population. 
\end{itemize}

\item Coalesence with migration: for simplicity, we assume that the two populations have equal size $N_1 = N_2 = N$, and the migration rates are also equal: $m_{1 \rightarrow 2} = m_{2 \rightarrow 1} = m$. We define $M = 2Nm$, as the number of migrants per generation. From the coalesence perspective, $m$ is the probability that any gene copy, when moving back in time, is from a different popoulation. Consider the problem of $n = 2$, we want to learn the expected coalescence time. Two scenarios: 
\begin{itemize}
\item Two samples are from the different populations ($D$): the only way to coalescence is first they trace back into the same population, then coalescence. The rate of migration is $m$, thus the waiting time is $1/2m$ for migration, so we have: 
\begin{equation}
E_D(t) = \frac{1}{2m} + E_S(t)
\end{equation}
where $E_S(t)$ is the coalescence time when two samples are from the same population. 

\item Two samples are from the same population ($S$): we have two processes, migration with rate $m$ and coalescence with rate $1/2N$. The probability that coalescence occurs before migration is: $p=\frac{1}{2N}/(\frac{1}{2N}+2m)$, and when this happens, the coalescence time is $2N$. The probability that coalescence occurs after migration is $1-p$, and when this happens, the coalescence time is $E_D(t)$. So we have: 
\begin{equation}
E_S(t) = \frac{\frac{1}{2N}}{\frac{1}{2N}+2m}	\cdot 2N + \frac{2m}{\frac{1}{2N}+2m}	\cdot E_D(t)
\end{equation}
\end{itemize}
Solving the two equations together we obtain: 
\begin{equation}
E_S(t) = 4N \qquad E_D(t) = \frac{1}{2m} + 4N
\end{equation}
Thus when the migration rate $m$ is small, it can take a long time to coalescence for two samples in different populations. The result about $E_S(t)$ is somewhat surprising: it does not depend on $m$. The intuitions are: (1) it is bigger than $2N$ because there is a small probability of migrating out, which increases the coalescence time. (2) When $m$ is small, it is a small probability to migrate out, but it takes longer to return, so the overall effect is independent of $m$. 

\item Remark: a special case is $m = 0$, and we should coalescence time equal to $2N$. However, when $m >0$, even if it is small, it extends the coalescence time by two fold. 

\item $F_{ST}$ and migration rates: to estimate $F_{ST}$, we need to determine heterozygosity, $H_S$ and $H_T$. When two samples are from the same population, the average heterozygosity is simply the number of pairwise difference divided by the number of sites $k$ (see below). Use $E_S(t) = 4N$, we have $\pi_S = 4N \cdot 2\mu = 2\theta$.  So we have: 
\begin{equation}
H_S = 2\theta / k	
\end{equation}
For $H_T$, again we use the coalesence time above to estimate the pairwise difference: 
\begin{equation}
\pi_D = \left(\frac{1}{2m} + 4N\right) \cdot 2\mu = \left( \frac{1}{4Nm}+2\right) \theta
\end{equation}
This leads to calculation of $H_T$. The final result of $F_{ST}$ is given by (general results): 
\begin{equation}
F_{ST} = \frac{1}{1 + 4 N m_T}	
\end{equation}
where $m_T$ is the total number of migrants into a population. For example, if we have $d$ populations with $m$ the pairwise migration rate, we have $m_T = (d-1) m$. The results say that $F_{ST}$ decays as $Nm_T$ increases: when there are significant migration, e.g. $N m_T > 10$, $F_{ST}$ is close to 0 - the population effectively evolves as a whole. When the migration is limited (e.g. with large geographic separation), $F_{ST}$ could get large. 

\item Remark: the relation between heterozygosity and $\pi$. Previously, when studying coalescence model (without migration), we estimate heterozygosity using the infite-allele model, which compares two events, mutation and coalescence, to obtain the expected heterozygosity at $\theta / (1 + \theta)$. When there are many sites in a sequence ($k$ is large), $\theta$ per site is close to 0, thus heterozygosity is equal to $\theta$, which is also the average number of pairwise difference, divided by the sequence length.  
\end{itemize}

Divergence model: 
\begin{itemize}
\item Process: an ancestral population splits into two populations at some point. Assumption: the population size does not change, i.e. $N_A = N_1 = N_2 = N$, for simplicity. The coalescence process is: each lineage is traced back to the divergence time, after that, the standard coalescence process. Given two samples, the coalescence time: 
\begin{equation}
E_S(t) = 2N \qquad E_D(t) = T+2N
\end{equation}
where $T$ is the divergence time. From this, we estimate the pairwise difference: 
\begin{equation}
\pi_S(t) = \theta \qquad \pi_D(t) = (T + 2N) \cdot 2 \mu = 2 \mu T + \theta
\end{equation}
The term $2 \mu T$ is the amount of genetic difference due to divergence. 

\item $F_{ST}$: we first obtain $H_S = \pi_S = \theta$ (ignore the term $1/k$), then we have: 
\begin{equation}
H_T = \frac{1}{2} (\pi_S + \pi_D) = \theta + \mu T	
\end{equation}
From these, we obtain: 
\begin{equation}
F_{ST} = 1 - \frac{H_S}{H_T} = \frac{T}{T + 4N}	
\end{equation}
So when $T$ is small, $F_{ST}$ is close to 0 (a single population); when $T$ is very large, $F_{ST}$ is close to 1. 
\end{itemize}

Isolation by distance: 
\begin{itemize}
\item Observation: for many species, $F_{ST}$ increases with distance in a way s.t. $F_{ST} / (1-F_{ST})$ is linear wrt. distance. This is called ``isolation-by-distance''. 

\item Explanation by the migration model: if the migration rate $m_T$ is a linear function of the distance (decreases with longer distance), this would explain the pattern. Alternatively, one can imagine many local populations, while migration occurs only between adjacent populations - this is called the Stepping-stone Model. 

\item Explanation by the divergence model: imagine a series of divergence events, each of which results in one spopulation splitting apart from another. And the new population occupies an area close to the parent population. This would lead to a linear relation between divergence time and distance, and the observed pattern of $F_{ST}$. 

\item \textbf{Remark}: in population genetics, there may be multiple scenarios that produces similar pattern of genetic variations. The reason is, for example, there may be multiple scenarios leading to change of effective population size (reduction of actual number, or limiting of genetic flow among individuals, etc.), which manifest as similar genetic patterns (increase of genetic relatedness). 
\end{itemize}

Allele frequency spectrum under migration/divergence model [personal notes]:
\begin{itemize}
\item Problem: The model provides the expected relation between $F_{ST}$ and migration/divergence parameters, e.g. divergence time. However, the pattern of genetic variation can be more complex, e.g. the same average $F_{ST}$ in a region could be a result of: many loci with small $F_{ST}$, or a smaller number of loci with large $F_{ST}$ (say fixed). How do we relate the frequency spectrum (e.g. the number of fixed sites at different alleles) with the migration/divergence parameters? 

\item Analysis of FS from coalesence: the time/position of mutational events in the coalesence tree determines its AF. Suppose we have $n$ samples from one population, the time to MRCA is about $4N(1-1/n)$. If there are no mutations in this time, the locus remains fixed at ancestral allele. The time of divergence is $T$, if $4N(1-1/n) > T$, the mutation happens before divergence, thus the site will appear fixed as the derived allele. 

\item Analysis of FS from forward models: e.g. given two populations diverged at timet $T$, we want to estimate the number of sites that are fixed in two different alleles in two populations (one ancestral, one derived). The idea is that: we have $\mu T$ new mutations in a region since divergence, and each mutation has probably $1/2N$ to be fixed, so roughly the number of fixed mutations is $\mu T / 2N$. 
\end{itemize}

Parametric bootstrapping to determine the distribution of estimators 
[Nielsen \& Slotkin, Chapter 5]: suppose we estimate some parameter $\theta$ using summary statistics (e.g. Tajima's estimator for $\theta = 4 N \mu$), and let $\hat{\theta}$ be the estimator
\begin{itemize}
\item Sample coalesence tree linking the samples: at each step, randomly sample time $t$ from expoential distribution, then sample the two nodes to coalescence.  

\item Sample mutations along the tree: Poisson process with rate $\mu$ (or using $2N$ as time unit). 

\item Compute $\hat{\theta}$ from each sampled tree (with mutations). 
\end{itemize}

Species tree and coalescence tree: (species could also mean populations within a species)
\begin{itemize}
\item Reciprocal monophyly: suppose we have multiple samples per species, but the divergence event is very ancient. Then at the point of divergence, all the samples within a species have coalescenced (thus can be treated as one individual). In this case, the species tree exactly matches the coal. tree. 

\item Incongruence (Incomplete Lineage Sorting): when the MRCA of all samples occurs before the divergence event (taking a long time back to reach the MRCA), then the species tree is incongruent with the coal. tree of all samples. 

\item Example of Incomplete Lineage Sorting: human-chimp-gorilla, the short divergence between MRCA of all three species to the ancestor of (human, chimp) can cause the problem. 
\end{itemize}

Infering population history from gene trees: suppose we are inferring the history of multiple populations of a common ancestor (when the divergence occurs, the effective population size, migration, etc.)
\begin{itemize}
\item Analysis using divergence model: we know that $\E_S(t) = 2N$ and $\E_D(t) = T + 2N$, when $T$ is large (or $N$ small), this leads to reciprocal monophyly; so the species/population tree is easy to infer. We could also say that population bottleneck (small $N$) eliminates ancestral variation. 

\item Incomplete lineage sorting can be resulted from short divergence time (relative to $2N$), or the gene flows. 

\item Example: from human gene tree, we are testing the hypothesis of out-of-Africa or multi-regional hypothesis (small $N$ of non-African group with some gene flow between African and non-African group).

\item Effect of recombination: instead of one coalescence tree, we may have multiple trees for different regions.  
\end{itemize}

Likelhood method for inferring population history: 
\begin{itemize}
\item Felsenstein equation: let $X$ be data, $G$ be the genealogy, and $\theta$ be the parameters of the model. Then the data can be used to infer the underlying processes:  
\begin{equation}
P(X|\theta) = \int_G P(X|G) P(G|\theta)	
\end{equation}
where $P(G|\theta)$ is given by the coalesence tree, and $P(X|G)$ is a result of mutations. 

\item Inference: MCMC to sample $G$. Another strategy is Approximate Bayesian Computation (ABC). The intution is that to sample $\theta | X$, MCMC uses $P(\theta|X) \propto P(X|\theta)$, while ABC assess how good $\theta$ agrees with $X$ based on summary statistics (simulate data from $\theta$, compute smmary statistics, and see if they are close to the observed values to decide if $\theta$ should be accepted). The advantage of ABC is that it does not directly sample the trees $G$.  
\end{itemize}

Inference of population structure through analysis of genetic similarity: 
\begin{itemize}
\item Population assignment by AFs: different populations differ in AFs. So we can assign a sample to population through the AFs of its markers. See the STRUCTURE algorithm below.  

\item PCA: imagine each population has a ``canonical'' profile (PC direction), and then each individual haplotype can be expressed as a linear combination of these profiles (the coefficients are PCs of that individual: the population composition). 
\end{itemize}

STRUCTURE algorithm: [Pritchard \& Donnelly, Genetics, 2000]
\begin{itemize}
\item Assumptions: alleles are in HWE, and in LE with each other. 

\item Model without admixture: suppose there are $K$ populations. The probability of the genotype of an individual, $X_i$ is given by $P(X_i|Z_i,P)$, where $Z_i$ is the population index of the individual and $P$ is the allele frequencies of the populations. Let $x_l^{(i,1)}, x_l^{(i,2)}$ be the genotype of the $i$-th individual of the $l$-th locus, we have: 
\begin{equation}
P(x_l^{(i,a)} = j | Z, P) = p_{z_ilj}	
\end{equation}
where $p_{klj}$ is the frequency of allele $j$ at locus $l$ in the population $k$. 

\item Model with admixture: define $q_k^{(i)}$ be the proportion of the individual $i$'s genome that originated from population $k$, and change $Z$ to: $z_l^{(i,a)}$ be the population of origin of the allele $x_l^{(i,a)}$. Then: 
\begin{equation}
P(x_l^{(i,a)} = j | Z, P, Q) = p_{z_l^{(i,a)}lj}		
\end{equation}
\begin{equation}
P(z_l^{(i,a)} = k | P,Q) = q_k^{(i)}	
\end{equation}

\item Inference: Dirichlet prior of the allele frequencies, and sample $Z$ and $P$ (and $Q$ for the model with admixture) through MCMC. 
\end{itemize}

Spatial locations of ancestry [Anand Bhaskar, May, 2016]
\begin{itemize}
	\item Problem of PCA: genetic distance does not match the spatial distance. Show by simulation. PCA minimizes the total pairwise genetic similarity. 
	
	\item Related work: SCAT, SPA, SpaceMix. 
	
	\item Model: let $\mu_l$ be the mean AF of locus $l$ (uniform across space). Assume actual AF at two locations correlate with each other with: 
	\begin{equation}
	\Cov(Q_l(z), Q_l(z')) = \eta(z -z')
	\end{equation}
	where $\eta$ is some function. Ex. isotropic decay, then covariance decays at rate $\exp[-\alpha_1 \norm{z - z'}^{\alpha_2}]$. Another model is directional decay. Idea is to do Taylor expansion so that locally, genetic similarity is a simple function of spatial distance. 
	
	\item Inference: use ideas from manifold learning. Basically, find pairs with smallest genetic distance, then create the spatial distance; then stitch the pairs to obtain the spatial distance of distantly related individuals. 
	
	\item \textbf{Remark}: how the parameters of the Gaussian process model relates to the population genetic processes?  
\end{itemize}

Introgressed tract of Nean. ancestry [Matthias Steinrücken, NHS meeting, 2018]
\begin{itemize}
	\item About 3\% of Nean. genome introgressed with human. To detect introgression: train CRF using African, European and Nean. as training; $S*$ statistics that detect divergence using panels.
	
	\item HMM: hidden states are ancestry, and transition represents introgressed Nean.
	
	\item Observation: less introgression in chr. X. Likely due to selection: DMI?
\end{itemize}

Spatial distribution of deleterious rare alleles [Dan Rice, Nov, 2019]
\begin{itemize}
	\item Observed pattern of rare deleterious alleles: plot AF vs. location, for CVs, fluctuation across mean in large regions; for RVs, much more localized. 
	
	\item Model of spatial AF of deleterious alleles: a mutant follows birth-death process, where the death rate (due to negative selection) is higher than birth rate. Also spatial random move. This can be modeled by a PDE, let $p(x, r, t)$ be the PDF of AF at $x$, location $r$ at generation $t$. At a given location, we can use W-F process, and we just need to additional terms to capture spatial diffusion: in-flux from neighbors and out-flux to neighbors.  
	
	\item Solving the PDE: marginalize locations (focus on AFs), and define the MGF of $p(x)$. Solve the MGF. Results: the spatial scale of AF distribution depends on $\sqrt{D/s}$, where $D$ is diffusion rate and $s$ selection coefficient. 
	
	\item Applications: (1) Test selection acting on RVs: derive neutral distribution. (2) Rare variant association test: adjusting for population structure due to different spatial locations. 
	
	\item Q: do we need selection to explain local spatial distribution of RVs? 
\end{itemize}

%%%%%%%%%%%%%%%%%%%%%%%%%%%%%%%%%%%%%%%%%%%%%%%%%%%%%%%%%%%%
\section{Recombination and Linked Selection}

LD in finite populations: [Hartl, Principles of Population Genetics, Section 9.2; Yang, Introduction to Statistical Methods in Modern Genetics, Appendix A]
\begin{itemize}
	\item Intuition: even without mutation and selection, LD could emerge in finite populations. To see why, we view the haplotypes as single polymorphic sites, and notice that: (1) Why not LE: in finite populations, certain haplotype tends to be fixed, and this pushes the population away from LE; (2) why not fixation: the frequency of haplotypes change at each generation, in fact, high frequency haplotypes may be more likely to be replaced due to recombination. The situation is similar to mutation-drift balance. 
	
	\item Model: consider two loci $u$ and $v$, and define $I$ as the probability that any two randomly selected gametes are IBD at $u$ and no recombination has ever happened between $u$ and $v$. We first claim without proof (see Yang) that: $I = r^2$. We could derive the recurrence equation of $I$, similar to the derivation of homozygosity under mutation-drift balance: 
	\begin{equation}
	I_{t+1} = (1 - c)^2 \left[ \frac{1}{2N} + \left( 1 - \frac{1}{2N}\right) I_t\right]	
	\end{equation}
	where $c$ is the recombination rate between $u$ and $v$. At equilibrium, the expected LD measure $r^2$ is given by: 	
	\begin{equation}
	E(r^2) \approx \frac{1}{1 + 4 Nc}	
	\end{equation}
\end{itemize}

Consequences of linkage: because of linkage, the evolution of one site could influence the evolution of a neighboring site. 
\begin{itemize}
	\item Clonal interference (in asexual organsisms) and Hill-Robertson effect: even if a favorable mutation occurs in one site, it may not get fixed because of a perhaps stronger favorable mutation in a neighboring site. 
	\item Genetic hitchhiking (selective sweep): Suppose a recent adaptive mutation occurs in the locus $A$, and is fixed very fast in the population, then an allele at a tightly-linked locus will also be fixed in the population, even if this allele may be neutral or deleterious. The consequence: a small region around the favored allele will be overrepresented in the population. 
	\item Background selection: suppose we analyze a site under neutral mutation, but a neighboring site is under negative selection, then some recombinations will be elimiated by selection. Thus the overall effect on the site of interest is effectively a reduction of population size (which reduces the level of polymorphism). 
	\item Evolutionary benefits of recombination: suppose there are multiple favorable alleles in the population, with low recombination or asexual organisms, they must be fixed in sequential order, thus effectively there is interference among favorable alleles. With recombination, different favorable allels can be brought together in rapid succession. 
\end{itemize}

Testing selection by LD patterns and haplotypes: [Hartl, Section 9.2]
\begin{itemize}
	\item Idea: selection of one site can change the polymorphism of the neighboring sites through hitchhiking. However, LD generated by hitchhiking persites for a relatively short amount of time, on the order of $0.4N$ generations, so the test is only applicable when selection is strong and recent, and linkage sufficiently tight. 
	\item Example: (Figure 9.8) a sequence segment with one highly similar haplotype in multiple copies, while other haplotypes are much more variable, suggesting that this haplotype is under selective sweep. 
\end{itemize}

Selective sweeps [Graham Coop, Population Genetics notes, Chapter 8]
\begin{itemize}
	\item Fully linked variants during selective sweep: suppose we have a new beneficial allele with selection coefficient $s$, it takes $\tau = 4 \log(2N)/s$ generations to fixation. This time is usually very short, comparing with $2N$. The effect is: loss of diversity after sweep, then slowly recover. 
	
	\item Recombination during sweep: Figure 8.3, genealogy tree (Figure 8.4): the ones from the selected lineage coalescence (short tree), and other lineages coalescence further back in time, with possible variants/mutations in the branches. The effect is: loss of diversity close to the selected variant, and recover diversity moving away from the selected variant (Figure 8.5)
	
	\item Quantitative analysis of loss of diversity due to selective sweep: consider two haplotypes, and we need to compute $\E(\pi)$, this is given by $T_2$, the coal. time of the two. This depends on which one of them or both are selected lineage: two main possibilities (1) One selected and one neutral; (2) Both are selected. The coal. time is $2N$ and $\tau$ respectively. The probabilities of (1) and (2) are given by $p_{NR}$, probability of no recombination. Note that since $\tau << 2N$, most of the diversity is from (1), when recombination occurs. Results: the diversity at recombination distance $r$ is given by: 
	\begin{equation}
	\E(\pi_r) = 4 N \mu (1-e^{-r \tau}) = \pi_0 (1-e^{-r \tau})
	\end{equation}
	where $\pi_0$ is the expected diversity if there is no linked selection. 
	
	\item Loss of diversity vs. recombination distance: intuitive analysis. Diversity is 0 if both haplotypes are the selected one. Diversity is $4 N \mu$ if one haplotype is new. The probability of this is just probability of recombination. Note that, the expected number of recombinations between selected variant and the variant at distance $r$ is $r \cdot \tau$. So probability of recombination is given by $1 - e^{-r \tau}$. 
	
	\item The extent of selective sweep: $s = $ 0.1\% will reduce diversity over 10's of kbs and $s$ = 1\% over 100kb. 
	
	\item Other signals of selective sweep: (1) Fay-Wu test: at intermediate distance, excess of high frequency alleles (from the selected haplotype). (2) As neutral diversity recovers, excess of low frequency alleles. (3) Local peaks of $F_{ST}$ between differentiated populations. 
	
	\item Soft sweep: Figure 8.12, multiple mutations selected, none completely sweep; or single standing variant (already multiple haplotypes at the time of selection) also lead to incomplete (soft) sweep. 
	
	\item Genomewide patterns of diversity due to linked selection: positive correlation of neutral diversity and recombination rates. 
\end{itemize}

Background selection (BGS) [Graham Coop, Population Genetics notes, Chapter 8]
\begin{itemize}
	\item BGS on fully linked variants: suppose we have a variant under negative selection. At this locus, we assume mutation-selection balance, so at equilibrium, we have: $q = \mu / (hs)$, where $hs$ is selection on heterozygotes. Only $1-q$ haplotypes contribute to diversity since $q$ deleterious haplotypes will be lost. The effect is similar to reduced population size  by $1-q$. Putting this together, we have: 
	\begin{equation}
	\E(\pi) = \pi_0 (1-q) = \pi_0 \left( 1 - \frac{\mu}{hs} \right)
	\end{equation}
	where $\pi_0$ is the expected diversity without BGS. 
	
	\item BGS with recombination: suppose we have a neutral locus that is at recombination distance $r$ away from a BCS locus, the diversity is: 
	\begin{equation}
	\E(\pi) = \pi_0 \left(1 - \frac{\mu hs }{2 (r+hs)^2}\right)
	\end{equation}
	More generally, a netural locus under the influence of multiple BGS loci at distance $r_i$, has diversity: 
	\begin{equation}
	\E(\pi) = \pi_0 \prod_{i} \left(1 - \frac{\mu_i h_i s_i }{2 (r_i+h_i s_i)^2}\right) \approx \pi_0 \exp \left(- \sum_i \frac{\mu_i h_i s_i }{2 (r_i+h_i s_i)^2} \right)
	\end{equation}	
	
	\item Approximation of BGS effects on linked variants: consider a large genomic region, with a neutral locus in the center with total recombination $R = \sum_i r_i$, and $U = \sum_i \mu_i$. We assume equal mutation (deleterious mutations only), recombination rates and strength of selection across all positions in the region. Under these assumptions: 
	\begin{equation}
	\E(\pi) \approx \pi_0 \exp(-U/R) = \pi_0 \exp(-\mu_{BP}/r_{BP})
	\end{equation}
	where $\mu_{BP}$ and $r_{BP}$ are per base mutation and recombination rates. Average $r_{BP}$ is about 2cM/Mb = $2 \times 10^{-8}$. 
	
	\item Genomewide pattern of BGS and B-factors: [McVicker, 2009] fits BGS with the genome, and estimates the effect of BGS (B-factors) across genome. The estimated mutation rate is high $7.4 \times 10^{-8}$. 
	
	\item Distinguishing hitchhiking from BGS: hitchhiking has systematic effects on SFS, distorting it towards rare minor alleles (slow recovery after sweep). An example (Figure 8.19): dip of diversity around syn. sites driven largely by BGS, but some fraction of dip around non-syn. substitutions from positive selection/sweep.
	
	\item Analysis (personal notes) of how BGS affects SFS: suppose we have a neutral locus $A$, linked with BGS locus $B$. Our goal is to estimate SFS at $A$ from PRF, and we need to study the survival time of an allele at $A$. Suppose we have AF at $A$ equal to $x$, we need to derive the new diffusion equation of $x$, by modifying $\E(\Delta x)$ and $\Var(\Delta x)$ in one generation to incorporate deleterious variants at $B$. 
\end{itemize}

Recombination rate variation within and between species [Molly Przeworski, 2017]
\begin{itemize}
	\item Part I: Recombination hotspots: double-strand breaks (DSB), orders of magnitude more often than average.
	
	\item Hotspots are not conserved: e.g. human vs. chimp, about 10\% are conserved. Also variations across individuals.
	
	\item Key role of PRDM9: N terminal Zn finger (DNA binding). Can predict PRDM9 binding sites from motif. Another domain has histone methyltransferase activity, H3K4me3, which often precedes recombination (help recruit recombination machinery). However, the mark itself is not sufficient to attract the machinery, so the recombination hotspots are not enriched near promoters.
	
	\item PRDM9 motifs are rapidly lost: if there are two alleles, one prefer PRDM9 binding, it will lead to differential transmission (high binding allele less likely to transmit) - meiotic drive. PRDM9 Zinc finger also evolves rapidly.
	
	\item Recombination (DSB) is reduced near TSS. This pattern is lost in PRDM9 KO mice.
	
	\item Part II: What drives recombination in species without PRDM9, e.g. chicken?
	
	\item Experiment: Sequence finches (N = 1-20) - no PRDM9, then infer recombination rates. (1) Found hotspots near promoters. (2) Hotspots are much more shared, 70\\%, between species.
	
	\item Background: GC-biased gene conversion: in DSB repair, heteroduplex forms (AG), but is more likely to be fixed as G.
	
	\item Elevation of GC content in hostpots, enriched near CG islands. Also find similar pattern in yeast.
	
	\item Model: in species lacking PRDM9, recombination machinery is more likely to target promoters (e.g. open chromatin or motifs). And they tend to be conserved because of selection on promoter function.
	
	\item PRDM9 in 225 vertebrates: lost in amphibilians, and several cases, some domains (KRAB) are lost. Zinc finger evolves rapidly only if the gene is intact. In species with partial ortholog (fish), H3K4me3 correlates with recombination rate. PRDM9 binding motif (predicted from sequence) not associated with H3K4me3; and PRMD9 motif not associated with high recomb. rate. So the partial PRDM9 does not direct recomb. in fish.
	
	\item Part III: Evolutionary consequences of recombination rate hotspots?
	
	\item Background: Hybridization between species is very common.
	
	\item Human-Nean: there is some desert of Nean. ancestry. Model: D-M incompatibilities, Nean has low Ne thus more deleteious mutations.
	
	\item Recombination rate and minor parent ancestry: low recombination rate, the minor parent ancestry is lost because of DMI, and some retained in high reocmb rate regions. The pattern of correlation between recombination rate and ancestry is observed in Human-Nean.
	
	\item Lesson: PRDM9 function is related to histone modification. Meiotic drive can lead to rapid evolution of PRDM9 binding sites.
	
	\item Lesson: relationship between recombination rate and ancestry, in low recombination rate regions, DMI will remove DMI-incompatible regions (higher load of deleterious mutations due to small population).
	
	\item Q: Selection of recombination rates: optimal level? If too high, break the good combinations too often.
	
	\item Q: In species without PRDM3, why recombination machinary evolves to recognize H3K4me3 mark or promoters? Is this due to CpG islands?
	
	\item Q: In birds (no PRDM9), are hotspots not close to promoters also highly conserved? Answer: not clear, because of uncertainty of defining hotspots.
\end{itemize}

The Effect of Strong Purifying Selection on Genetic Diversity [Cvijovic and Desai, Genetics, 2018]
\begin{itemize}
	\item Model assumption: a neutral site, perfectly linked to a site with deleterious mutation. The mutation is strong $N s$ at the level of 1000.
	
	\item BGS can distort SFS (Figure 1): (1) Excess of rare variants: even strong selection cannot purge deleterious alleles instantly. So at very low AF, the SFS is similar to neutral expectation under the original population size. (2) Excess of high frequency variants: similar to positive selection, once the alleles reach high frequency (few ancestral alleles), drift will dominate. Neutral alleles get fixed in a sweep-like fashion because they carry on average fewer deleterious mutations than the wild type.
	
	\item Implication of the distortion: not obvious in small samples, but the effect on high and low ends of SFS can be large with large sample sizes.
	
	\item Discussion: the model assumes single selection coefficient in BGS sequence $s$. The target sequence is neutral, but if negative selection, can be easily accommodated. If the BGS sequence has large variation of $s$, need more work.
\end{itemize}

Natural selection interacts with recombination to shape the evolution of hybrid genomes [Science, 2018]
\begin{itemize}
	\item Hybrid genomes, assess the ancestry of two parents. Ancestry from the “minor” parental species is more common in regions of high recombination and where there is linkage to fewer putative targets of selection.
	
	\item Model: ancestry from the minor parental species is more likely to persist when rapidly uncoupled from alleles that are deleterious in hybrids.
\end{itemize}

%%%%%%%%%%%%%%%%%%%%%%%%%%%%%%%%%%%%%%%%%%%%%%%%%%%%%%%%%%%%
\section{Neutral Theory and Detecting Natural Selection} 

Reference: [Hartl \& Clark, Principles of Population Genetics, Section 4.4, 4.5; Nielsen \& Slatkin, Chapters 8-9]

Principles of Detecting Selection: 
\begin{itemize}
	\item Strategy: the goal is to infer the evolutionary forces, in particular selection, on protein or DNA sequences, from the pattern of polymorphism. The general strategey is to derive these patterns under the assumption of no selection, and the departure in real data from these expected patterns would indicate selection or other evolutionary forces. This departure can often be measured as the amount of genetic variation, for example, positive selection in a sequence can reduce its genetic variation. 
	
	\item Measures of genetic variation: many measures such as $H$, heterozygosity, $\pi$, pairwise difference, $S$, number of segregating sites, and so on. More generally, we can use patterns of polymorphism to measure genetic variation: (1) allele-level pattern: the frequency of different alleles; (2) site-level pattern: how many sites are segregating and their frequencies; the number of different sites among pairs of sequences; etc. Two common patterns: allele frequency spectrum and site frequency spectrum (derived allele frequency, DAF, of each site). 
	
	\item Influence of natural selection on the patterns of polymorphism and divergence: basis of statistical test of selection. 
	\begin{itemize}
		\item Within-species polymorphism: generally, selection affects SFS, which may change the metrics of genetic variation, such as $\pi$. In general, $S$ is less sensitive to selection, while $\pi$ is. Selective sweep: reduce intermediate frequency alleles, leading to a reduction of $\pi$. Balancing selection: increase $\pi$. Negative selection: excess of rare variants, thus reducing $\pi$ and reduce $S$ to a smaller extent.
		\item Within-species haplotype: another signal of selection, nearby regions of selected sites show reduction of homozygosity. 
		\item Genetic differentiation between populations/$F_{ST}$: when a locus is adapted to a local environment, it may change AF significantly between population. This is a signature of positive selection. Ex. $F_{ST}$ is elevated in the lactase locus, relative to adjacant region in European samples. Another example, AF difference at EPAS1 gene between Han and Tibetan. 
		\item Divergence: generally, positive selection makes it easier to reach fixation while negative selection makes it harder. 
		\item Combining patterns of polymorphism and divergence: they may result from the same underlying selective force, thus can be combined to increase the power of detecting selection. 
	\end{itemize}
	
	\item Application of a test depends on specific problems/contexts: e.g. when finding genes underlying population difference within a species, choose SFS type of tests. When testing genes being selected in a local population, choose AF-based tests such as $F_{ST}$. When searching for genes underlying species difference, use dN/dS test. 
	
	\item Defining expected pattern of genetic variation: this generally depends on mutation rates and demography. While the absolute values can be difficult to obtain, sometimes we can still utilize information in them. For example, the relative mutation rates between sites (e.g. syn. vs. nonsyn.) and genes are possible to find. 
	
	\item The importance of controls: often the expected pattern of genetic variation under neutrality is not known, e.g. $\theta = 4 N \mu$ depends on the local mutation rate, which is often unknown. So we will need to define a control/contrast to test for selection. This is a very general idea that has many applications in population genetics and comparative genomics: 
	\begin{itemize}
		\item $d_N/d_S$ test of selection using divergence data. 
		\item Variant density (e.g. number of singletons) in synonymous vs. nonsyn. sites: could be used to estimate selection. Intuitively, few nonsyn. sites would suggest negative selection. Under neutrality, the ratio of the two is determined by mutation rates. 
		\item Constraint in non-coding sequences: choose neutral sequences as those that are non-DHS in any known cell types. 
	\end{itemize}
	
	\item Confounding factors on detecting selection: demography (change of population size, in particular). For example, negative selection will create an excess of rare variants, but population expansion will also create excess of rare variants. 
\end{itemize}

Neutral theory and negative selection:
\begin{itemize}
\item Neutral theory: postulate that most of the observed genetic difference is neutral, instead of adaptive. The theory does not exclude the role of negative selection, which does not contribute much to the observed difference. 

\item Negative selection and amount of genetic variation: in theory, negative selection could limit the amount of genetic variation. Ex. we consider $\theta$ (or $H$): with negative selection, the effective mutation rate (the effective portion of sequences that are allowed to mutate) is lower than the actual rate. However, when we compare the test sequence with the control (e.g. inter-species divergence), as long as the proportion of negatively selected sequence is the same, $\theta$ will be identical. 

\item $d_N/d_S$ test and why it is not enough: the main limitation is that the selection within species may have changed since divergence. Another challenge for this test is that: it is possible that only a small proportion of sites are under positive selection while most of the sites are under negative selection. 

\item Main tests for negative selection: Intra- and inter-species variation (MK test). Intra-species: fraction of rare variants, variant density (more useful for strong selection). Tajima's D and its extensions are not designed for testing negative selection. Remark: need a better way to integrate information from density and FRV.  
\end{itemize}

Pattern of genetic variations following positive selection: genetic hitchhiking
\begin{itemize}
\item Selective sweep: suppose we have a new advantageous mutation, then as it is getting fixed, the alleles linked to the mutation will also be fixed. Selective sweep will reduce genetic variation ($H$). Another way to understand it: because of selection, the coalesence time is shorter, thus fewer mutations. Recombination will reduce the effect of selective sweep: the further away from the selected locus, the less selective sweep. It can be shown that the recombination $c$ and selection cofficient $s$ determines the extent of sweep: at $c > s$, hardly any selective sweep. 
\begin{itemize}
	\item Intuitive analysis: suppose we have complete LD. Initially, we have an advantageous allele $a$, linked to a neutral site $b$. As $f_a$ increases from directional selection, $f_b$ also increases, and eventually both reach fixation.
	
	\item Recombination: with recombination, the efficiency of $f_b$ increase will be lower because with any new $a$ alleles, only a fraction of them have the $b$ allele. 
\end{itemize}

\item Partial sweep: advantageous allele reaches intermediate frequency but has not been fixed. The pattern is: polymorphism at sites close to the advantageous alleles is reudced, and strong LD is created between alleles on the same chromsome as the advantageous allele. Ex. G6PD (Figure 6.1 in [Nielsen \& Slatkin]). 

\item Associative overdominance: suppose we have overdominance (heterozygous advantage) at $Aa$, we ask what is its effect on a neighboring neutral site $B/b$. Intuitively analysis: 
\begin{itemize}
	\item Complete LD: suppose we start with a single allele $a$, initally, $f_a$ will increase becuase of heterozygous advantage, but it will reach equilibrium $\hat{f}_a$. Because of complete LD, the $b$ allele also increases in frequency and reach stable polymorphism. The overall consequence is that the genetic variation at $B/b$ locus  increases. 
	
	\item Recombination: As $f_a$ increases, $f_b$ also increases, but as $a$ reaches substantial fraction, $b$ will swithc its partner from $a$ to $A$ due to recombination. Formally, we can think of two populations $A$ and $a$, and there is migration of $b$ between the two populations, the rate of which depends on recombination rate and $f_a$. This allows us to formally analyze heterozygosity at the $b$ locus using the coalesence model of migration.  
\end{itemize}
The overall consequence of associative overdominance is the increase of genetic diversity (heterozygosity) at the linked neutral loci, opposite to the pattern of selective sweep.

\end{itemize}

Tests of selection from population genetic and inter-species data:
\begin{itemize}
\item HKA test: let $S$ be the number of segregating sites within the population, intuitively, with positive selection, $S$ should be reduced. We use $F$, the number of fixed difference between two species, to assess the change of $S$. For neutral sites, $\E(S) / \E(F)$ should be constant:
\begin{equation}
\frac{\E(S)}{\E(F)} = \frac{4 N \mu \sum_i 1/i}{ 2 T \mu + 4 N_A \mu}
\end{equation} 
where $T$ is the divergence time and $N_A$ the ancestral population size. So choosing two loci, we compare $S_1, F_1, S_2, F_2$ and do a $\chi^2$ test. If $S/F$ for a locus is significantly below the other one, this suggests selective sweep. 
 
\item MK test: similar to the idea of HKA test, instead of comparing two loci, we compare the syn. and nonsyn. sites of a region/gene. Specifically:
\begin{itemize}
	\item Under positive selection (selective sweep): $F_N$ is increased (mutations get fixed across species) and $S_N$ is reduced (less genetic variation), so $S_N/S_S < F_N/F_S$. 
	
	\item Under negative selection: (1) Moderate/weak selection: $F_N$ is reduced (harder to fix mutations), and $S_N$ not change much as moderate/weak selection does not affect much the number of variants, so $S_N/S_S > F_N/F_S$. (2) Strong negative selection: both $F_N$ and $S_N$ will be reduced (thus strongly deleterious mutations do not contribute), so MK test cannot detect. 
\end{itemize}

\item Remark: The measure of genetic variation is $S$, this does not take all the information in the data. 
\end{itemize}

Testing selection [John Novembre, HGEN 469]
\begin{itemize}
	
	\item I. Diversity vs. recombination rate: positive correlation. Divergence vs. $r$: no correlation, ruling out mutation. Neutral theory cannot explain this pattern. Possible explanation: genetic hitichiking; or background selection. 
	
	\item Effect of positive selection on linked loci: use figure to explain selective sweep. Reducing nearby diversity. After fixation, then accumulation of mutations, increasing diversity. However, number of variants recover faster than $\pi$ - signature of positive selection. The difference can be tested via Tajima’s D. 
	
	\item Explaining $r$-$\mu$ correlation using positive selection: low recombination rate region, larger impact of hitchhiking, thus lower diversity. 
	
	\item Background selection: effective population size in a region reduced. Similar effects of reducing diversity. Thus low recombination regions, larger impact from negative selection and lower diversity in nearby regions.  
	
	\item To distinguish the two: the site frequency spectrum could be different. In positive selection: excess of singletons. However, background selection can also result in similar, negative Tajima’s D. 
	
	\item Demography and selection: Tajima’s D is negative with growing populations. How to differentiate the two? 
	
	\item Fit a demography model using neutral data: then estimate the distribution of Tajima’s D. Alternatively, fit the empirical Tajima’s D from neutral sites. 
	
	\item References for detecting SFS changes (extension of Tajima’s D), SweepFinder, Nelson, 2005. 
	
	\item MKPRF: combination of divergence and SFS. Estimation of distribution of fitness effects. 
	
	\item II. Haplotype patterns for selection. Voight 2006. Idea: in a selected site, group by the alleles, at the site, homozygosity = 1, nearby regions show reduction of homozygosity. But the reduction is slow with sweep, comparing with neutral. Idea: assess the AUC of haplotype homozygosity, and normalize by neutral. IHS  test. 
	
	\item III. AF difference across populations. $F_{ST}$ based scan. 
	
	\item IV. comparison of methods. IHS better at detection of partial sweep. Detecting different processes: e.g. recent or older. 
	
	\item CMS: Sebati. Combining multiple signatures. 
	
	\item Lesson: population geneticists use genetic diversity as a main tool: how various forces affect diversity. In selective sweep, analysis of how diversity changes because of selection, and then how it changes due to mutations. Different patterns of different measures of diversity: number of variants and diversity, the latter is much more sensitive to selection. 
	
	\item Remark: about teaching (1) ask questions, e.g. why positive correlation between recombination and mutations? Motivates the work. (2) Use figures, e.g. to show the dynamics/ change over populations. (3) Link to what’s discovered. Ex. background selection, link to $4N \mu$. 
\end{itemize}

Detecting negative selection using polymophism data [personal notes]:
\begin{itemize}
	\item MK test: more powerful than Tajima's D and related test for detecting negative selection [Zhai \& Slatkin, MBE, 2009]. However, using divergence data.  
	
	\item INSIGHT: under negative selection, no intermediate and high-frequency variants. Weakness: not based on mutation rates, thus lower power; and strong assumptions. 
	
	\item $dN/dS$ test and variations: compare the number of NS and S sites within a gene, and compare with expectation. The expectation can be obtained from mutation rates (approximately 3.9:1). 
	
	\item Mutation rate based: RVIS, ncRVIS and Samocha2014. Obtain the  number of NS sites in a gene, and compare with expectation based on mutation rates (Samocha2014, ncRVIS) or gene length/total num. of variants (RVIS). For RVIS and ncRVIS: consider only the number of common NS sites - they are more sensitive to negative selection.
	
	\item Number of common variants reflects the strength of negative selection: we use PRF model to analyze the difference of SFS under neutrality or negative selection. We recall that the expected number of polymorphic sites with frequency $x$ is $g(x) = 2 N_e \mu f(x)$ and 
	\begin{equation}
	f(x) \approx \frac{1 - e^{-2 \gamma (1-x)}}{[N_e (1 - e^{-2 \gamma})][x(1-x)]}
	\end{equation}
	where $\gamma = 2 N_e s$. The ratio of $f_{\gamma}(x)$ and $f_0(x)$ (neutrality) is: 
	\begin{equation}
	\frac{f_{\gamma}(x)}{f_0(x)} = \frac{1 - e^{-2 \gamma (1-x)}}{1 - e^{-2 \gamma}}
	\end{equation}
	Suppose we have $\gamma = -5$ and $x = 0.01$, the ratio is 0.9; but at $x = 0.2$, the ratio is 0.13! So negative selection strongly limits the number of common variants, but has little power of removing rare and very rare variants.  
\end{itemize}

Molecular signatures of natural selection [Nielsen, ARG, 2005]
\begin{itemize}
\item Predictions of neutral theory:
\begin{itemize}
\item On inter-species divergence: new mutations arise at the rate $2 N_e \mu$, and each of them, if neutral, has a probability $1/2N_e$ to be fixed. Thus the number of fixed mutations between two species is proportion to $\mu$, the mutation rate.
\item On intra-species polymorphism: both the number of segragating sites ($S$) and the nucleotide diversity ($\Pi$) is proportional to $\theta = 4 N_e \mu$.
\end{itemize}

\item Estimating selection and test of neutrality: general strategies:
\begin{itemize}
\item For a gene to be tested, comparison of nonsyn. mutations vs. syn. mutations (neutral).
\item Using polymorphism data: under the neutral theory, the number of segregating sites (or average diversity/heterozygosity) is proportional to the mutation rate (roughly the length of sequence). Thus the number of segregating sites in syn. vs. nonsyn. sequences should be equal to the ratio of mutations of the two types of sequences.
\item Using polymorphism data: site-frequency spectrum (SFS) (Nielsen05, Figure 2). The tests such as Tajima's D identify the excess of rare variants (relative to neutral expectation), which is due to natural selection. However, the departure of SFS from neutral expectation can be also due to population growth, especially for human (inflated rare variants from recent expansion).
\item Extension of Tajima's D test: (1) Fu \& Li extended this test to take information regarding the polarity of the information into account by the use of an evolutionary outgroup. (2) Fay \& Wu suggested a test that weights information from high-frequency derived mutations higher. 
\item Using divergence data: under the neutral theory, the subsitution rate between two species should be proportional to the mutation rate (the substitution rate is related to the number of fixed sites under a molecular evolution model, such Jukes-Cantor model). Thus if the gene is neutral, the $d_N/d_S$ should be equal to 1.
\end{itemize}

\item Test of neutrality using both divergence and polymorphism data (MK test). The general idea can be applied in different ways: e.g. the polymorphism may be measured using diversity (or average heterozygosity).
\end{itemize}

The genetics of human adaptation: hard sweeps, soft sweeps and polygenic adaptation [Pritchard, Curr Biol, 2010]
\begin{itemize}
	\item Examples of human recent adaptation: height, e.g. pygmy, may be driven by food limitation or humidity. High altitude adaptation: to low O2 during pregnancy. Pigmentation: half a dozen genes found with strong genetic signals of selection.
	
	\item Conflicting patterns of selection learned from genome-wide data: (1) Genome-wide scan of selection: different methods often do not agree. (2) AF difference between populations: large AF diff. tend to be in genic than non-genic regions, suggesting selection (Figure 1A). (3) However, differentiation of AS closely matches historical population, suggesting drift (Figure 1B); and high F-ST SNPs do not show haplotype patterns of selection (Figure 1C), similar XP-EHH to random SNPs, and are often not fixed.
	
	\item Possible explanations: soft sweep model, where sweep occurs in a standing variant or multiple mutations. Consider mutation target sizes (number of possible beneficial mutations): show that when it is large, say $>100$, usually there will be standard variations when a new selection force arrives. Soft sweep does not produce classical pattern of sweep. When mutation target size is small, there may be long waiting time.
	
	\item Model of polygenic adaptation (Figure 3): short-term adaptation, selection on standing variations at multiple loci, so the AFs of these variants will increase, but not get fixed. This may also explain the pattern of AF difference (genic $>$ non-genic) due to linked selection. Once the population reaches new optimum, weak frequency-dependent selection: downward drift of some alleles will have to be balanced by upward drift of other alleles.
	
	\item Analysis: why standing variants may be more important for adaptation? Suppose we have a reasonably large mutation target size, then when selection force arrives, some standing variants will respond. As these variants increase in frequency, selection will be reduced.
	
	\item How to find real selection signals? "place more weight on sweep signals thatinclude variation at likely functional sites. Improved external information will likely help greatly in the coming years”; “test whether, looking across many loci, there is a significant tendency for alleles that increase the phenotype value to increase (or decrease) in frequency together”.
\end{itemize}

Recent Progress in Polymorphism-Based Population Genetic Inference [J Hered, 2012]
\begin{itemize}
\item Methods for testing the departure of neutrality: using the site frequency spectrum. Tajima's D, Fu-Li test.
\item Methods for testing selective sweep:
\begin{itemize}
\item e.g. Fay-Wu derives the expected frequency spectrum after a selective sweep.
\item Important to account for population demographic history.
\item Using haplotype (LD) information.
\end{itemize}
\item Estimating the extent of purifying selection:
\begin{itemize}
\item  Estimating selection on nonsynonymous mutations [Loewe et al, 2006]: using data of two species. idea: variants subject to sufficiently strong purifying selection will not increase significantly as effective population size increases, whereas neutral diversity is expected to increase proportionally with population size (the comparison of syn. sites in two species).
\item Simultaneous inference of selection and population growth from patterns of variation in the human genome [Williamson, PNAS, 2005]: Simultaneous inference of selection and population growth from patterns of variation in the human genome. Account for population growth.
\item  Estimating the rate of adaptive molecular evolution in the presence of slightly deleterious mutations and population size change [Eyre-Walker A, Keightley PD. MBE, 2009]: inference of selection, population history and account for beneficial mutations.
\end{itemize}
\end{itemize}

\subsection{Infinite-Allele Model}

Infinite-alleles model: often used to analyze the allozyme data. 
\begin{itemize}
	\item Model: each new mutation creates a new allele. The new alleles are constantly created from mutation, and also removed from the population by random drift (and selection, if it is modeled).  
	\item Homozygosity: we are interested in the genetic variation of the population, which is commonly defined as the homozygosity, (probability that two random alleles are the same). Under infinite-alleles model, homozygosity is equal to probability of IBD. Let $F_t$ be the probability of IBD at generation $t$, could derive the recurrent of $F_t$, and at equlibrium: 
	\begin{equation}
	\hat{F} = \frac{1}{\theta + 1}	
	\end{equation}
	where $\theta = 4 N \mu$ ($\mu$ is the mutation rate of the protein). The intuition is that: when $\mu$ is low, most of the alleles are the ancestral form, thus the IBD is high. 
	
	\item Allele-frequency spectrum: the polymorphism data of $n$ sequences can be characterized by the allele-frequency spectrum: the frequency of each unique allele. Let $k$ be the number of different alleles in a sample of $n$, then: 
	\begin{equation}
	E(k) = 1 + \frac{\theta}{\theta+1} + \frac{\theta}{\theta+2} + \cdots + \frac{\theta}{\theta+n-1}
	\end{equation}
	The probability of a particular allelic configuration is given by the Ewens sampling formula: the probability of a sample of size $n$ containing $k$ distinct alleles with $n_i$ copies of type $i, 1 \leq i \leq k$ is: 
	\begin{equation}
	P(n_1, n_2, \cdots, n_k, k) = \frac{n! \theta^k}{k! n_1 n_2 \cdots n_k S_n(\theta)}	
	\end{equation}
	where $S_n(\theta) = \theta (\theta +1) \cdots (\theta+n-1)$. 
\end{itemize}

Testing neutrality with infinite-alleles model: Ewens-Watterson test
\begin{itemize}
	\item Allele frequence spectrum: e.g. excess of common or rare alleles. This could be compared with the expectation under Ewens-sampling formula.  
	\item Homozygosity: the test statistic is $F$, the homozygosity. The statistical significance can be assess by simulation (sampling from a population according to the neutral infinite-alleles model). If $F$ is higher than expected by chance (more homozygosity), this may suggest: purifying selection that removes deleterious mutations; or growing populations. 
	\item Limitations: the test assumes that any two alleles that cannot be distinguished (e.g. by electrophoresis) must be IBD. The change of population size may also cause departure of the infinite-alleles model. 
\end{itemize}

\subsection{Infinite-Site Model}

Infinite-sites model: 
\begin{itemize}
	\item Model: the sequence is infinitely long, and each new mutation occurs in a different site of the sequence. The pattern of interest is: (1) the number of segregating sites; (2) the nucleotide mismatches of any two sequences in a sample. Both (1) and (2) are defined at per site level. 
	\item Number of segregating sites ($S$): this is the same as the number of mutations in the entire genealogy tree of $n$ sequences. The expected number of mutations is: 
	\begin{equation}
	E(S) = \mu E[\sum_{i=2}^{n} i T_i] = \theta \sum_{i=1}^{n-1} \frac{1}{i} 
	\end{equation}
	Note that the mutation rate $\mu$ above is the mutation rate of the entire nucleotide sequence. The variance of $S$ can be computed using the law of total variance (the distribution of the number of mutations conditional on the coalescence time). Let $X_i$ be the number of mutations in time $T_i$, then (condition on $T_i$, the number of mutations in $i$ branches are independent): 
	\begin{equation}
	E(X_i|T_i) = i \mu T_i \qquad \text{Var}(X_i | T_i) = i \mu T_i
	\end{equation}
	Thus, plug in the mean and varirance of $T_i$, we have: 
	\begin{equation}
	\text{Var}[X_i] = E[\text{Var}(X_i|T_i)]	+ \text{Var}[E(X_i|T_i)] = \frac{\theta}{i-1} + \frac{\theta^2}{(i-1)^2}
	\end{equation}
	The variance of $S$ is given by: 
	\begin{equation}
	\text{Var}(S) = \theta \sum_{i=1}^{n-1} \frac{1}{i} + \theta^2 	\sum_{i=1}^{n-1} \frac{1}{i^2}
	\end{equation}
	
	\item Number of mismatches between any two sequences ($\Pi$): this is the number of mutations in the tree relating the two sequences: 
	\begin{equation}
	E(\Pi) = 2 \mu E(T_2) = \theta	
	\end{equation}
	The variance of $\Pi$ can be found in Equation (4.19) or [Tajima83]. 	
	
	\item Nucleotide polymorphism and diversity: both $S$ and $\Pi$ depend on the sequence length. We could define nucleotide polymorphism as the average $S$ over the sequence length $L$: $S^* = S / L$; similarly we define nucleotide diversity as: $\pi = \Pi / L$.  
\end{itemize}
	
Testing selection using Site Frequency Spectrum (SFS): Figure 9.2 of [Nielsen \& Slotkin]. Different scenarios and how the SFS are changed:
\begin{itemize}
\item Negative selection: the SFS will be skewed towards low-frqeuency alleles comparing with neutral sites. Note: the patterns are different for strong and moderate/weak selection. Strong selection: reduce number of variants; moderate/weak selection: AF change (higher fraction of rare variants) without a large effect on the number of variants. 

\item Positive selection: the opposite effect, SFS skewed towards high-frequency alleles. 

\item Selective sweep: alleles linked to the advantagous site will increase and reach higher ferquency, while the unlinked alleles will have lower frequency. The result: an excess of both high- and low-frequency alleles. 

\item Balancing selection: opposite to selective sweep, more alleles of intermediate frequency. 
\end{itemize}
	
Tajima's D statistic: designed for testing selective sweep
\begin{itemize}
	\item Intuition: both $S$ and $\Pi$ can be used to estimate $\theta$. Thus under the neutral model: $\Pi - S/a = 0$ (where $a$ is the harmonic series). With selection, the two would not be equal: $S$ is not very sensitive to frequency of polymorphic sites (and thus less sensitive to selection), but $\Pi$ is sensitive - very rare or very common variants contribute less to $\Pi$ than intermediate frequency variants. 
	 
	\item Test: $D = \frac{\Pi - S/a}{\sqrt{V(\Pi - S/a)}}$. Needs simulation to get the null distribution of $D$. 
	
	\item Selective sweep: creates more rare and very common alleles, and this leads to lower $\Pi$, thus $D < 0$. Other population process such as population growth, which leads to an excess of rare alleles. 
	
	\item Balancing selection: leads to more nucleotide diversity (pairwse difference) relative to $S$ than expected, thus $D > 0$. Another interpretation: population contraction.   
\end{itemize}
		
Fu-Li test: 
\begin{itemize}
	\item Intuition: the numbers of singleton and non-singleton sites ($\eta_i$). With purifying selection, expect most polymorphic sites to be singleton, as it would be hard for the polymorphic sites to acqure more than one copy in the population due to selection. 
	\item Test: the number of singleton sites is equal to the number of mutations in the external branch ($\eta_e$), and non-singleton sites equal to that in the internal branch ($\eta_i$), where the external branch is defined as the branch from an internal node to the tip. Could show that: 	
	\begin{equation}
	E(\eta_e) = \theta \qquad E(\eta_i) = (a-1) \theta	
	\end{equation}
	Note that the results are independent of the sample size. The test statistic is $\eta_e - \eta_i / (a-1)$ normalized by its varirance. The $P$ value is often determined by simulation. 
\end{itemize}
			
\subsection{Poisson Random Field Model} 

Reference: [Sethupathy \& Hannenhalli, PRF tutorial, 2007], Directional Selection and the Site-Frequency Spectrum [Bustamante \& Hartl, Genetics, 2001]

Goal: given a sequence of interest (a protein, some genomic region, etc.) and its polymorphism spectrum data (the number of fraction of polymorphic sites with different daf), how to detect natural selection? 
\begin{itemize}
\item PRF model: assumption of unlinked loci. The expected frequency spectrum can be calculated directly using mathematical models. The model can be used for estimating selection coefficients for particular classes of mutations and test various hypotheses regarding selection.
\item Williamson et al [PNAS, 2005] applies PRF for human data, taking population history into account.
\end{itemize}

Model: let $X_i$ be the number of polymorphic sites with daf $i / n$, i.e. $i$ copies of derived alleles within $n$ samples, the problem is to determine the distribution of $X_i$. 
\begin{itemize}
%\item Relating expected number of polymorphic sites with the mean time of absorption: analogous to the following problems:
%\begin{itemize}
%\item Problem: particles are injected into a container with rate $\theta$ (average number of particles added per unit time is $\theta$, and each particle can survive only for time $t$, show that the expected number of particles at any given time is  $\theta t$\\
%Proof: any particle in the container was generated in time less than $t$ ago, and any particle generated in this period will be alive at this moment, therefore, the number of particles is equal to the number of particles added in time $t$. 	
%\item Problem: same as above, except that each particle can survive for a random period of time whose expectation is $t$, show that the same result holds\\
%Proof: should not change the expectation. 
%\end{itemize}
\item Survival time (transient distribution): from the diffusion model, we know that for a new mutation, its time of staying in frequency $(x, x+dx)$ is $f(x) dx$, where $f(x)$ is defined by Equation~\ref{eq:daf_mean_time}, which we repeat here:
\begin{equation}
f(x) \approx \frac{1 - e^{-2 \gamma (1-x)}}{(1 - e^{-2 \gamma})} \frac{2}{x(1-x)}
\end{equation}
where $\gamma = 2 N_e s$. When $s = 0$, we have $f(x) \approx 2/x$. 

\item Relating number of polymorphic sites with survival time: we are interested in the number of polymorphic sites whose daf are in $(x,x+dx)$, denoted as $g(x) dx$. Intuitively, if any mutation has a long survival time in $(x, x+dx)$, we would expect more sites in this range; and if we have high mutation rates $\theta$, we'd also expect more sites. We can show that $g(x) = 2 N_e \mu f(x)$. Suppose mutations are non-overlapping (e.g. we look a small segment), then we can analyze one mutation a time. After each new mutation arises, it will stay at frequency $x$ for $f(x)dx$ generations. But we do not have one mutation every generation, so we should multiply $f(x)dx$ by the probability we have a mutation in one generation $2 N_e \mu$. The same results would also hold for longer segments. So we have: 
\begin{equation}
g(x) = 2 N_e \mu f(x)
\end{equation}

\item Finite sampling: We need to further consider the sampling process in analyzing the actual polymorphism data. Note that $g(x)dx$ is the expected number of polymorphic sites with AF in $(x, x+dx)$, among these sites, the percent that generates $i$ copies in a sample of $n$ sequences is $\text{Binom}(i;n,x)$. Thus the expected number of sites with $i$ copies of derived allele is: 
\begin{equation}
F(i) = \int_0^1{g(x) \text{Binom}(i;n,x)dx} = 2 N_e \mu \int_0^1{f(x) {n \choose i} x^i (1-x)^{n-i} dx}
\end{equation}
The distribution of $X_i$ is Poisson distribution with mean $F(i)$ because all sites are iid. Note that in the above equation, we have binomial sample of $n$, instead of $2n$ - one can show that this does not change the results. 

\item Likelihood: the probability of observing $X = (X_1, \cdots, X_{n-1})$ is given as: 
\begin{equation}
L(\theta, \gamma) = P(X|\theta, \gamma) = \prod_{i=1}^{n-1}{P(X_i = x_i|\theta, \gamma)}
\end{equation}
This leads to a way of parameter estimation and LRT for neutrality. 

\item Extension to site-specific selection [Bustamante \& Hartl, Genetics, 2001]
\begin{itemize}
	\item Mixture of selected and neutral sites: change $f(x)$ to $(1-\pi) f_0(x) + \pi f_{\gamma}(x)$, where $\pi$ is the fraction of selected site, $f_0(x)$ and $f_{\gamma}(x)$ are the transient distribution of neutral and selected sites respectively.
	\item Individual site: inferrence in a mixture model.
\end{itemize}
\end{itemize}

Understanding SFS from PRF model: 
\begin{itemize}
	\item Neutral SFS under PRF vs. coalescence theory: e.g. we consider the expected number of singletons $F(1)$. Under neutral theory, this is $\theta = 4 N_e \mu$. We show that this is indeed the case with PRF: 
	\begin{equation}
	F(1) = \int_0^1 2 N_e \mu \frac{2}{x} {n \choose 1} x^1 (1-x)^{n-1} dx = 4 N_e \mu n \int_0^1 (1-x)^{n-1} dx = 4 N_e \mu 
	\end{equation}
	
	\item Comparing SFS under neutral vs. selection. 
\end{itemize}

Application to McDonald-Kreitman (MK) test: 
\begin{itemize}
\item{}MK test \\
Intuition: if there is negative selection, then there will be less polymorphism than divergence because more mutations are eliminated by natural selection than in the case of neutral; and similarly, there will be more polymorphism if there is positive selection. The polymorphism and divergence measures are both scaled by those of the neutral sites (synonymous sites). \\
Procedure: construct 2-by-2 table with number of polymorphic sites and number of fixed substitutions for both synonymous and replacement sites, and test if the ratio of polymorphism between replacement and synonymous is significantly larger than that of divergence. 
\item{}application of PRF to MK test
\begin{itemize}
\item{}number of polymorphic sites: apply the expected density function of daf and consider the sampling process (because some true polymorphic sites will not be sampled as polymorphic if only one allele is sampled). The expected number of polymophic sites with sample size $m$ is: 
\begin{equation}
H(m) = \int_0^1{g(x) P_m(x) dx} = \int_0^1{g(x) [1 - x^m - (1-x)^m] dx}
\end{equation} 
Apply this equation to both the synonymous ($s = 0$) and replacement sites. 
\item{}number of fixed substitutions: it has two parts, one from fixed substitutions (mutation multiplied by fixation probability) and the other from sampling: 
\begin{equation}
2 N_e \mu t_{div} P_{fixation} + \int_0^1{g(x) x^m dx}
\end{equation}
where $P_{fixation}$ is given by Equation \ref{eq:fixation_prob}. 
\end{itemize}
\end{itemize}


%%%%%%%%%%%%%%%%%%%%%%%%%%%%%%%%%%%%%%%%%%%%%%%%%%%%%%%%%%%%
\subsection{Inferring Negative Selection}

Approximation to the Distribution of Fitness Effects across Functional Categories in Human Segregating Polymorphisms [Racimo and Schariber, PLG, 2014]
\begin{itemize}
	\item Challenges of estimating distribution of fitness effects (DFE): confounded by demography; most methods depend on binary classification of neutral or selected genes. 
	 
	\item Estimating selection coefficients from SFS: use non-equilibrium demography to derive the expected SFS, let $f(x,t)$ be the frequency spectrum at frequency $x$ and time $t$, and let $g(x,t) = x(1-x) f(x,t)$. Solve the PDE of $g(x,t)$ from diffusion equation. Then the probability that a given site in a sample of $n$ has $i$ copies of the derived allele is: 
	\begin{equation}
	p_{i}(t) = \frac{f_i(t)}{\sum_{j=1}^{n-1} f_j(t)}
	\end{equation}
	where $f_i(t)$ is the frequency of variants with $i$ copies of allelies at time $t$. Comparing with PRF, this does not explicitly model absorption (extinction or fixation). To fit selection coefficients to observed SFS: create bins by C-scores assuming one DFE value for the entire bin, then use MLE to estimate DFE for that bin. 
	
	\item DFE for C-score bins (Figure 1A): substantial selection at $C > 30-35$ ($C$-scores are PHRED-scale), $s < 10^{-4}$ ($N(0) = 10000$ reasonable population size for human). 
	
	\item Background selection: divide sequences into 10 B-score bins. The DFE-C score relationship is robust to BGS, except the two highest bins (strongest BGS). Solution: fit a neutral demography model only in the exome, and then do C-score to DFE mapping in these regions.  
	
	\item DFE in different classes of elements: 
	\begin{itemize}
		\item Nonsynonymous variants: bimodal distribution. Note: PhastCons scores do not perform well as the scores are regional, thus cannot distinguish NS and S variants. 
		\item Synonymous: not neutral, however, could be due to background selection. 
		\item Non-coding elements: more uniform. 
	\end{itemize}
\end{itemize}

\subsection{Inferring Positive Selection}

Population differentiation as a test for selective sweeps (XP-EHH) [Chen and Reich, GR, 2010]
\begin{itemize}
	\item Background: selective tests, within population, use AFS or haplotype (EHH test). Intuition of EHH: Figure 1A, selection leads to increase of DAF of sites linked to selected sites. Comparing this with neutral expectation leads to selection inference. Importantly, the change of DAF under neutrality depends on the region (LD) size: small region, old allele, so expect larger changes on average; and large regions, young alleles, so smaller changes. 
	
	\item Background: Cross-population test of selection: (1) AF differentiation, or F-ST. However, F-ST has large variations under neutrality. (2) Use haplotype comparison: increased hom. at linked sites. 
	
	\item Model idea: Figure 1B, similar to EHH test, expect that it considers difference of AF b/t populations. 
	
	\item Model: of a single site, let $p_1$ be the AF of the objective population, and $p_2$ be the AF of the reference population. Our goal is to model $p_1$ distribution as a function of $p_2$, recombination rate $r$, selection coefficient $s$, and divergence time (time since selection event) $\omega$. The change of AF follows Brownian motion model, the frequency is expected to be increased to $1 - c + cp_1^*$, where $p_1^*$ is the frequency in the objective population before selection, and $c \approx 1 - q_0 r/s$, where $q_0 $is the initial allele frequency of A in population 1. The likelihood $f(p_1 | r, s, p_2, \omega)$ is given by Equation (4), see Figure 2 for different distributions of $p_1$ under neutrality vs. selection (closer to 0 and 1).
	
	\item Composite Likelihood ratio (CLR) test: test if $s = 0$ across all sites. They are not independent, so weigh the likelihood by their LD. 
	
	\item Analysis: how does the model encode the idea of calibrating $\Delta$ AF with LD/allele age? The parameters are fit using all sites in a window. Intuitively, the method uses low LD sites to learn about neutral pattern, including $\omega$, and use high LD sites to learn selection (expected to be deviated from neutrality). 
\end{itemize}

Signature of multiple-merger coalescence in genomic diversity data [Daniel Rice, Sep, 2016]
\begin{itemize}
	\item Background: nucleotide diversity $\pi$ depends on $T_c \mu$ where $T_c$ is the time since common ancestor and $\mu$ the mutation rate. Also $\pi$ is spatially correlated because $T_c$ of adjacent pairs are similar. The spatial correlation depends on recombination rate $T_c r$: higher $r$, lower spatial correlation. 
	
	\item Background: genealogy tree, balanced tree topology. With population bottleneck or positive selection, one has ``mergers'', or star-like topology. This picture changes with recombination: one has merger in the tree, but then the subtree coalescence with other neighbors. 
	
	\item Recurrent selective sweep: leads to multiple mergers in a tree. Signature: selective sweeps lead to increase of DAF, thus the variants with high DAF will be higher than expected. 
	
	\item Motivation: detection of multiple mergers without using recombination map, ancestral alleles. Let $\phi_i$ be the number of sites with AF $i$. The idea is that, with merging, we create multiple high AF variants (consider multiple mutations before the merging: they all have high AFs). So we can use average correlation between $\phi_i$'s as our statistic: it will be generally negative under neutral model, but could be $>0$ under multiple merge model.   
\end{itemize}

%%%%%%%%%%%%%%%%%%%%%%%%%%%%%%%%%%%%%%%%%%%%%%%%%%%%%%%%%%%%
\section{Population Genetics of Complex Traits}

A Population Genetic Signal of Polygenic Adaptation [PLG, 2014]
\begin{itemize}
	\item Model idea: suppose we have multiple populations. Let $Y_i$ be the average trait value in population $i$. We want to infer whether the trait is under selection. Define the neutral model of the trait, and test deviation. The idea is that we can obtain the neutral model of the AF, and the trait is related to the AF, so we can obtain the neutral model of the trait.   
	
	\item Model: let $p$ be the AF of an extant population and $p_A$ be that of ancestral population. Then $p|p_A$ follows normal distribution, with variance dependent on the time. For the trait in population $i$, we have $Y_i = \sum_l \beta_{l} p_{il}$ where $\beta_l$ is the effect size of SNP $l$, and $p_{il}$ the frequency of SNP $l$ in population $i$. One can show that $Y_i$ follows MVN.  
\end{itemize}

The osteoarthritis and height GDF5 locus yields its secrets [NG, 2017]
\begin{itemize}
	\item Fine-mapping GDF5 locus: transgenic mice, define an enhancer of 2.5kb, driving expression in long bones.
	
	\item Signature of positive selection: high LD near the SNP, suggesting selective sweep. Use phylogenetic analysis to infer the history: compare African, European and Neanthandal. The allele arrives in Africa, then spread to both Neanth. and European during the old and later out-of-Africa migration.
	
	\item Remark: does this result mean that height is under positive selection? If so, we’d expect many other loci showing evidence of selection. If not, what’s the selective force driving GDF5 locus?
\end{itemize}

From Adaptation to Disease [Jeremy Berg, 2018]
\begin{itemize}
	\item Part I. Detecting polygenic adaptation. Ref: Berg and Coop, 2014: polygenic score under selection, much stronger signature than individual SNPs.
	
	\item Detecting polygenic selection across populations: Let $Z$ be the polygnic score, Z is weighted average of effect size (by AF). Define $V(Z)$ as variance of $Z$ across populations, then $E(V(Z)) = V F_ST$. To see this, we can write $Z = G \beta$, where $\beta$ is effect size, then:
	\begin{equation}
	\Var(Z) = \Var(G^T \beta) = \beta^T \Var(G^T) \beta = \tr(\Var(G^T) \beta \beta^T)
	\end{equation}
	where $\Var(G^T)$ is related to $F_{ST}$. Intuitively, large selection leads to large AF difference, thus large variance of Z. Its expected value under neutrality depends on gene flow, etc.
	
	\item New model: Racimo, Berg and Pickrell 2018. Selection happens at particular lineages. Analysis of phylogenetic tree of populations and how polygenic scores change. Ancient DNA data to help with understanding polygenic score changes.
	
	\item Q: Can we develop a model of $H_1$ and use it to identify specific populations where the trait under selection? The model is far more complex, e.g. need to consider population size, when selection happens, etc.
	
	\item Part II. Why disease alleles persist? Does genetic architecture depend on fitness cost?
	
	\item Model: $s = \alpha \cdot t(P) \cdot S$, alpha: allele effect, and t(P) density at threshold of liability (population distribution of liability), S trait fitness. s: selection coefficient. Intuition: when $S$ increases, the distribution of liability shifts (so that prevalance is reduced), as a result, for any particular allele, it’s less likely to reach the liability, so $t(P)$ is weaker and overall, $s$ is constant.
	
	\item Show with simulations: when $S$ changes, $s$ returns previous level after some generations.
	
	\item Remark: intuitively, with stronger fitness cost, deleterious variants will be eliminated more often. So any deleterious allele will be present in an individual with fewer other deleterious variants - thus this allele is further from liability threshold, weakening selection.
	
	\item Chalk Talk. Goal: given AF vs. effect size distribution, learn about selection on traits.
	
	\item Remark: PRF kind of model, the site frequency spectrum for variants under a given selection.
	
	\item For directional selection: Power is proportion to $\alpha^2 \cdot p (1-p)$. Strong selection $p ~ 1/(\alpha \beta)$, where $\beta$ is selection.
	
	\item For stabilizing selection: $p ~ 1/\alpha^2$, so power is the same for different AFs. Remark: power is proportion to variance per SNP.
	
	\item Q: But in practice, power is higher for common variants, why?
	
	\item Directional selection: $\E(\Delta p) = \alpha \beta p q$, change of AF per generation. Disease: liability threshold, fitness depends on risk, which depends on liability.
	
	\item Stabilizing selection: $\E(\Delta p) = \alpha^2 \omega^2 pq(.5-p)$, where $\omega$ is strength of stablizing selection.
	
	\item Modeling Pleiotropy: treat pleiotropic effect as some additional term influencing selection. Long term, joint modeling of multiple traits.
	
	\item Discussion [personal notes]: Interaction can be very common under the liability threshold model: a variant near the threshold may have a large effect than a variant far away from the threshold. Interaction between variant and PRS: collider bias? 
	
	\item Model: how natural selection shapes AF vs. effect size of a trait? If the trait is correlated with another continuous trait, then it is possible that selection for the second trait would drive the first trait to higher prevalence even if the first trait is deleterious. Ex. ASD and education.
	
	\item \textbf{Lesson}: two challenges of studying evolution of complex traits: (1) Epistasis: the selection on one locus, depends on genetic background (how many deleterious alleles are present). This could be thought of as Collider bias: both alleles act on a common trait. (2) Pleiotropy: selection on one allele depends on its effect on all phenotypes it affects. 
\end{itemize}

Evolution of polygenic risk score using ancient DNAs [Maryn, NHS meeting, 2020]
\begin{itemize}
	\item Problem: during evolution, the risk alleles may change from random drift, e.g. some risk alleles may get lost, and new alleles may emerge (from low frequency to higher frequency with measureable effects). This turnover will change the population level PRS distribution. 
	
	\item Model: suppose we have $L$ risk alleles, with effect $\beta_l$, and $X_l(\tau)$ be the genotype of a sample back in time $\tau$. The genotype-phenotype model is written by:
	\begin{equation}
	Y(\tau) = C + \sum_l \beta_l X_l(\tau)
	\end{equation}
	The observed PRS can be written as: 
	\begin{equation}
	\hat{Y}(\tau) = \hat{C} + \sum_s \beta_s \hat{X}_s(\tau)
	\end{equation}
	where we sum over all observed variants. We can thus consider the change of the mean of PRS and variance of PRS. The mean change is the bias of PRS due to the change of the intercept. To model how the PRS distribution changes, we denote $Z(\tau)$ the vector of AF of all variants at time $\tau$ and $Z(0)$ the AFs in the current time. 
		
	\item Our primary analysis involves the computation of AF change conditioned on $Z(0)$. In particular, consider the cases where $Z(0)$ is rare (below GWAS AF cutoff) and $Z(\tau)$ is common and the opposite situation. Such cases lead to bias of PRS. Let $p(z, z')$ be the transition probabilities of population AF, we have:
	\begin{equation}
	\E(X_l(\tau) | Z_l(0)) = \int z (2z-1) p(z, z'; \tau) \frac{\rho(z)}{\rho(z')} dz
	\end{equation}
	where $\rho(\cdot)$ is the equilibrium frequency of $z$. 
	
	\item Remark: the model considers one sample a time, and treat ancient AF as unknown and marginalized. When we have multiple samples in ancient, we have some information of ancient AF. 
	
	\item Remark: PRS model also depends on LD. How LD structure evolves over time? 
\end{itemize}

Mutation-selection balance of polygenic traits [Jeremy Berg, NHS, 2020]
\begin{itemize}
	\item Motivation: in quantitative genetics, it is often assumed that the fitness effects of mutations are ``symmetric'': increase or decrease of a trait has the same fitness effects. But for polygenic disease trait, this is not true, as risk-increasing mutations are subject to stronger selections. 
	
	\item Model idea: mutations mostly increase the liability and reduces the fitness. This forces is balanced by the smaller portion of protective mutations and negative selection. At equilibrium, the fitness distribution of the population does not change. This is similar to mutation-selection balance, but across many loci.  
	
	\item Model assumptions: consider a single site, it can exist in two forms, protective or risk alleles. We consider the two alleles by a two-state Markov chain, with mutation rate $\mu$. The effects of new mutations follow certain distribution, denoted ad $p(\alpha)$, for effect size $\alpha$. At equilibrium, we expect many more sites exist in the protective allele form, and this proportion generally depends on $\alpha$. We denote $\rho_{\alpha}$ the proportion of sites fixed for the risk (or protective?) allele. 
	
	\item Model: how liability distribution changes in each generation? We would expect at equilibrium, the increase of liability by mutation is equal to the reduce of liability by selection:
	\begin{equation}
	\Delta_u \bar{z} = - \Delta_s \bar{z}
	\end{equation}
	where $z$ denotes liability. Suppose there are $L$ sites in total, We have $\Delta_u \bar{z} = 2 L \mu b$, where $b$ is the average effect on liability, over all sites (by their mutation effects). We write selection effect: 
	\begin{equation}
	\Delta_s \bar{z} = \int p(\alpha) \alpha \langle \Delta x | \alpha \rangle d\alpha
	\end{equation}
	where $x$ is the frequency of allele. The change of $x$ for sites with effect $\alpha$ is given by:
	\begin{equation}
	\langle \Delta x | \alpha \rangle = \int \langle \Delta x | x, \alpha \rangle \Psi(x | \alpha) dx
	\end{equation}
	where $\Psi(x|\alpha)$ is the AF distribution. We have $\Delta x | x, \alpha \rangle = \pi_{\alpha} S x (1-x)$, where $\pi_{\alpha}$ is the risk-scale effect size, and $S_{\alpha} = \pi_{\alpha} S$ is the selection coefficient. And for $\Psi(x|\alpha)$, we have:
	\begin{equation}
	\Psi(x | \alpha) = (1- \rho_{\alpha}) f(x | -\gamma_{\alpha}, \theta) + \rho_{\alpha} f(1-x|\gamma_{\alpha}, \theta)
	\end{equation}
	where $\theta = 4 N_e \mu$ and $f(\cdot)$ is given by the diffusion equation. 
	
	\item What have we learned from the model? Assuming effects are small, the results are insensitive to fitness cost of disease $S$. 
	
	\item Question: how is the model related to empirically observed effect size and AF relationship? 
\end{itemize}
%%%%%%%%%%%%%%%%%%%%%%%%%%%%%%%%%%%%%%%%%%%%%%%%%%%%%%%%%%%%
\section{Pattern and Rates of Germline Mutations}

Influences of mutation rates and patterns [personal notes]
\begin{itemize}
	\item Local mutation rates are influenced by both mutational supply (e.g. exposure to carcinogens) and DNA repair. There are multiple mechanisms of mutations and repairs, each with possibly distinct signatures/mutational spectrum. Ex. some mutational mechanisms are dependent on replications, others not. 
	
	\item Replication timing: strong influence on cancer/somatic mutation rates, moderate influence on germline rates. Mechanism: nucleotide store depleted during late replication, reducing the effectiveness of repair. 
	
	\item Distance to telomere. 
		
	\item Recombination rates. 
	
	\item Low complexity regions/repeats. 
		
	\item Transcription coupled repair. Signature: strand asymmetry. 
	
	\item Chromatin structure: strong association of DNM with DHS in [Autism WGS, Cell, 2012]. Also found association with DHS in [Francioli, 2015], but not significant after correction of GC. 
	
	\item Regulatory activities: association with K27ac in [Autism WGS, Cell, 2012]. However, this could be due to confouding factor: GC content or DHS. 
	
	\item DNA methylation and CpG: cytosine deamination.
	
	\item Local sequence context: tri-nucleotide context. GC content: DNA repair needs to separate two DNA strands, it is easier for AT (2 H-bonds) than GC. So generally higher GC content associated with higher mutation rates.  
	
	\item Hotspots: found in multiple studies. Also [Francioli, 2015] suggests that their mutational spectrum is different from non-clustered mutations. 
	
	\item Individual-level variation of rates: paternal age. Also found the difference in mutational spectrum (and influence of other factors) in older fathers vs. younger ones [Francioli, 2015].
	
	\item Remark: to model/understand the pattern of mutations, consider all mechanisms, and the effect of each feature on each mechanism. Sometimes a feature may impact multiple mechanisms, creating complex relationship with observed mutation rates, e.g. chromatin structure can affect mutational supply and repair. Another example: transribed regions. 
\end{itemize}

Properties and rates of germline mutations in humans [Campbell \& Eichler, TiG, 2013], Determinants of Mutation Rate Variation in the Human Germline [Ségurel \& Przeworski, ARG, 2014]
\begin{itemize}
	\item Four strategies of estimating mutation rates: 
	\begin{itemize}
		\item Severe Mendelian disease: mutation/selection balance to estimate how frequent new mutations occur. 
		\item Comparative genomics: need estimation of generation time (large uncertainty). Also influenced by GC-biased gene conversation, which acts analogously to selection. Also highly sensitive to the assumptions of ancestral polymorphism. 
		\item Genetic variation within human population: may be affected by other processes such gene conversion. 
		\item Pedigree sequencing. 
	\end{itemize}
	
	\item SNV mutation rates
	\begin{itemize}
		\item DNMs from families: estimate of 1.16E-8 per base pair per generation. 
		\item Human-chimp comparison: earlier estimate of 2.5E-8 in pseudogenes, and 1.82E-8 using inferred ancestry of nearby microsatellites. 
		\item Discrepancy may be due to (1) differences in filtering applied for SNVs or in sequencing methodology. (2) Change of life history in primates: number of spermatogenetic cell divisions per cycle, change of generation time, delayed onset of puberty and so on. 
	\end{itemize}
	
	\item Sources of mutations: replication or spontaneous (endogenous or exogenous). 
	\begin{itemize}
		\item Replication errors: incorrect base incorporation or slippage of polymerase (indels $<20$ bp). The repair process is facilitated in regions where DNA strans can be easily separated (AT-rich). 
		 
		\item Non-replicative errors: deamination of C is an important source. 
	\end{itemize}
	
	\item Nonrandom distribution of new mutations: 
	\begin{itemize}
		\item Shorter scale: Transitions outnumber transversions by twofold for de novo SNV. The rate of mutation at CpG dinucleotides has been observed to be ten- to 18-fold the rate of non-CpG dinucleotides, probably due to CpG methylation. 
		\item Intermediate scale:  Low-complexity repetitive DNA may also be mutagenic, consistent with sequence-dependent replication slippage. 
		\item Broader scale: (1) The higher mutation rates in or near protein-coding regions: perhaps from the higher GC content of these regions in combination with the effects of transcription-associated mutations (TAM). Also strand asymmetry in transcribed DNA. (2) Recombination: may introduce point mutations. 
		\item Other factors such as nucleosome occupancy and DNase hypersensitivity are correlated with broad-scale mutational patterns,  but the association could be due, at least in part, to confounders such as base composition.
	\end{itemize}
	
	\item Nonindependence of mutations: 
	\begin{itemize}
		\item Over 10kb scale: mutations in a single transmission more clustered, from IBD studies. 
		\item In the range of 100bp or less: excess of mutations in a single replication. Mechanism: DNA lesion (e.g. oxidized base), requres special polymeraise such as Pol zeta, which is error prone. 
	\end{itemize}
	
	\item Inter-individual variation of mutation rates: 
	\begin{itemize}
		\item Replication driven mutations: likely dominate most of germline mutations. In males, after puberty, about 23 replications / year in sperms. The ratio of male to female mutations $\alpha$ is about 4. However, the ratio $c$ based on number of cell divisions should be 9-14 depending on the time of conception. 
		
		\item Spontaneous mutations: proportional to time. 
		
		\item Possible explanations of $\alpha \neq c$: (1) Many mutations are spontaneous in both sexes. (2) Higher error rate per cell division before puberty. 
		
		\item Individuals may vary in their propensity to acqute mutations: (1) notably due to mutations in DNA mismatch repair genes. (2) Germline methylation.  
	\end{itemize}
	
\end{itemize}

The effects of chromatin organization on variation in mutation rates in the genome [Makova \& Hardison, NRG, 2015]
\begin{itemize}
	\item Why it is important to estimate local mutation rates? Inferring selection and interpreting somatic mutations. 	
	
	\item Regional variation in mutation rates (RViMR): demonstrated in multiple types of mutations using cross-species comparison, including base substitutions, small indels, TE insertions. Co-variation among different types was also found. 
	
	\item Genomic features that contribute to RViMR: \textit{GC content} (methylated CpG). Hight AT content - high sbustitution rates (more likely in \textit{heterochromatin}). Increase in rates close to \textit{telomeres} due to altered repairs in these regions. These genomic features (add male \textit{recombination rates}, exon density, etc.) explain $>50\%$ variability in substitution rates and 30\% in indel rates. 
	
	\item Chromatin structure/accessibility: depends on whether it has a greater impact on mutagenic or repair process. In general, increased access leads to increased exposure to DNA damaging agents, but also greater access of DNA repair enzymes. Overall, open chromatin has a lower mutational frequency, either due to fewer mutations or increased repair.  
	
	\item Results of chromatin accessibilty on mutation rates: 
	\begin{itemize}
		\item Pairwise analysis: nucleosome occupancy may affect substitution rates. Some possible mechanisms: DNA in nucleosomes are less prone to cytosine deamination. But the substitution pattern may also be influenced by selection that maintain optimal GC in core and linker regions. In general, some studies support a link between open chromatin and repressed mutations (due to repair), while other studies found the opposite pattern. 
		
		\item Multivariate analysis: use base substitutions, insertions, deletions and microsatellite repeat number alteration. Classify the four types of mutations into different types of regions (6) with HMM model. Ex. the ``hot'' state where the rates of all mutations are elevated - characterized by open chromatin, and ``warm'' states where deletion and substitution rates were mildly elevated - characterized by closed chromatin.  
	\end{itemize}
	
	\item Transcription coupled repair (TCR): lead to mutation bias (lower in transcribed strand) and overall lower rates. 
	
	\item Discussion: future work
	\begin{itemize}
		\item Experimental manipulation of enzymes to study the mechanims of mutation. 
		\item The influence of biased gene conversion on the relationship between mutation and chromatin. 
	\end{itemize}
	
	\item Remark [personal notes]: the relationship between mutation rates and chromatin accessibility seems complex, nonlinear. This could be due to (1) accessibility has different effects on exposure to mutational agents and on repair (e.g. repair is saturated at some point, while exposure is monotonic); (2) additional factors not accounted for: heterogenietity of open chromatin regions. It might be interesting to apply \textbf{causal inference} techniques to distinguish direct and indirect effects of various genomic features on mutaiton rates. 
\end{itemize}

Whole-Genome Sequencing in Autism Identifies Hot Spots for De Novo Germline Mutation [Michaelson \& Sebat, Cell, 2012]
\begin{itemize}
	\item Background: for structural variation (SV), the mutation rates are predominantly driven by meiotic recombination - nonallelic homologous recombination (NAHR) between tandem segmental duplications lead to SV hotspots. Human and chimpanzee comparison have found evidence that regional SNV mutation rate is influenced by GC content, recombination rate and chromosome-banding patterns. Data: WGS of 10 MZ twin-family samples, 40x. MZ improves the power of detecting DNMs. A total of 581 germline DNMs were detected in ten MZ twin pairs. Our estimate is lower than theoretical estimates by a factor of two, but consisent with other estimates. 
	\item Mutational clusters: defined as two or more DNMs within 100 kb. Found 10 clusters in all samples (one per family). 
	\item Factors influencing the mutation rates: find features that discriminate DNMs and random genomic sequences. The features were obtained from ESCs (histone mark, DHS), Mammary Epithelial Cells (K27ac, nucleosome occupancy) and LCL (replication timing). The most significant features were DNase hypersensitivity, GC content, nucleosome occupancy, recombination rate, simple repeats, and the trinucleotide sequence surrounding the site. Train a sparse logistic regression model to predict mutation rate (MI). 
	\begin{itemize}
		\item Mutability was greatest for CpG dinucleotides. 
		\item Factors act on different scales: larger scales such as nucleosome occupancy (100 bp), recombination rate ($10^4$ bp), and replication timing ($10^6$ bp). 
	\end{itemize}
	\item Relation between mutability and evolutionary conservation: a distinctly U-shaped relationship between mutability and sequence conservation: hypermutability is correlated with highly conserved sequence and low genetic diversity.
\end{itemize}

Genetic Variation Meets Replication Origins [Cell, 2012] on [Koren, AJHG, 2012]
\begin{itemize}
	\item Identify replication timing from WGS data of proliferating cells: early replication (S phase) tends to have higher sequencing depth than late replication ones. 
	
	\item Replication time QTL (rtQTL): phenotypes can be: gain or loss of replication orgin, change of replication zone length, etc. 
	
	\item Possible mechanisms of rtQTL: chromatin accessibility (affecting binding of replication proteins), enrichment of eQTL. 
\end{itemize}

Nucleosomes suppress spontaneous mutations base-specifically in eukaryotes [Chen and Xionglei He, Science, 2012]
\begin{itemize}
	\item Nucleosomal DNA has fewer cytosine deamination, 50\% decrease of the C $>$ T mutation rate. The rates of G $>$ T and A $>$ T mutations were also about twofold suppressed by nucleosomes.
	
	\item Remark: what is the implication on mutation rates in open chromatin regions? No nucleosome, so expect higher mutation rates. But also more accessible by DNA repair enzymes.
\end{itemize}

A framework for the interpretation of de novo mutation in human disease [Samocha \& Daly, NG, 2014]
\begin{itemize}
	\item Model: trinucleotide context for each base. There are $64 \times 3 = 192$ rates for each mutation in every trinucleotide context. 
	\begin{itemize}
		\item Estimation of mutation rate per base: intergenic regions that are orthologous between humans and chimps. Tally the number of all 64 trinucleotides, and for each SNP, considered the chimp allele to be ancestral and determined the trinucleotide (XY1Z) to trinucleotide (XY2Z) change. 
		\item Calibration: use the model to determine the relative mutability, then set the absolute mutation rate by genome-wide mutation rate of 1.2E-8. 
		\item Per-gene probability of mutation: look up the table to obtain the rate for each base, then sum appropriate mutations.The probability of a frameshift mutation: multiplying the probability of a nonsense mutation by 1.25 (the relative rate from ASD WES data).
	\end{itemize}
	\item Adjustment of the rates: use the number of synonymous singletons from ESP data. 
	\begin{itemize}
		\item Base model: gene length alone show $r = .835$, the base model, $r = 0.854$. 
		\item Depth adjustment for prob. of mutations in sequencing data (lower than theoreical rates because some bases are not covered well): multiply the estimated rates by a fraction 0.9 to 1, which is determined from the proportion of trios where every member has 10x coverage of a base. After this adjustment, $r = 0.891$. 
		\item Divergene adjustment: the idea is to adjust for regional variation. Define local divergence scores as the number of divergent sites between human and macaque over screened sites in a gene and up/down-stream 1M region.  Then use linear models to determine the best equation to adjust the per-gene probabilities of mutation to incorporate the divergence score. Results: $r = .91$. 
		\item Replication time: no significant effect, so not used. 
	\end{itemize}
	\item DNM rate: in exome, 1.67E-8 per base per generation. 
	
	\item Individual genes based on the estimated mutation rates: in 1,061 trios, (1) CHD8: 3 LoF and 1 missense, p.LoF = 1.76E-6, p.all = 3.2E-5. (2) TTN: 4 missense, p = 0.18.
	
	\item Validation using ESP data: correlate the number of rare syn. variants with predicted rates. Gene length alone showed high correlation (r = 0.880), but our full model showed significantly greater correlation, $r = 0.94$. 
	
	\item De novo mutations and IQ: ASD cases with below average IQ have significant excess of de novo LoF mutations, while those with above average IQ have no such excess. 
	
	\item \textbf{Using DNM rates to define constrained genes} (page 10 of Supplement): analogy of $dN/dS$ test. For every gene, we let $\mu_S$ be the rate of syn. mutation and $\mu_{NS}$ be the rate of nonsyn. mutation, and similarly, $N_S$ and $N_{NS}$ be the observed number of variants. Our goal is to estimate the expected number of variants $E_S$ and $E_{NS}$. Specifically: 
	\begin{itemize}
		\item Fit a linear model of $N_S$ vs $\mu_S$ across all genes in the genome. 
		\item Apply the linear model to a specific gene to obtain $E_{NS}$ using $\mu_{NS}$ from that gene. 
		\item Test deviation of $N_{NS}$ from $E_{NS}$ using chi-squared test. Then obtain the corresponding signed $Z$-score: positive score reflects negative constrainer (fewer nonsyn variants than expected). 
	\end{itemize}
	Note: the test uses missense variants. ExAC also has scores based on LoF variants, which is less reliable. 
	
	\item Defining the excessively constrained genes: should have at least 5 syn. singletons (some genes have low coverage and not enough variants identified), the syn. mutation fits the mutation rates, and there is significant deficit of missense singletons ($p < 0.001$). 
	\begin{itemize}
		\item 5\% of genes (1,003) are under strong selection (missense $Z$-score cutoff 3.09, $P < 0.001$), about half of which are in OMIM. 
		\item Constrained list: 2.3 fold enrichment of de novo ASD LoF, highly significant. 
		\item Missense $Z$ scores: genomewide average around 0.94, genes with dn-LoF in ASD 1.68 and with dn-LoF in ID 2.46.  
	\end{itemize}
	
	\item Comparison with other metrics: simple $dN/dS$ test using human data, only 377 highly onstrained genes ($P < 0.001$). The missense $Z$ scores correlate with RVIS. 
	
	\item Question: definition of $\chi^2$ statistic, use only NS or both NS and S variants? 
	
	\item Remark: the $dN/dS$ test compares two counts (syn. and nonsyn.), while the current approach for testing selection estimates the mutation rates, instead of using counts. 
\end{itemize}

Genome-wide patterns and properties of de novo mutations in humans [Francioli \& Sunyaev, NG, 2015]
\begin{itemize}
	\item Data: 250 Dutch parent-offspring families, 13-fold coverage. A total of 11,020 DNMs, with estimated sensitivity of 69\% and specificity of 95\%. 
	
	\item Effect of paternal age: twich more DNMs in children of 40-year old father than 20-year old. Replication timing was significantly associated with paternal age, whereas chromatin states and recombination rates not. DNMs in younger fathers biased towards late replicating regions, also DNMs in older fathers are more likely to be in exonic regions. 
	
	\item Regional variation: correlation with DHS and exonic regions (more enriched). However, no correlation after controlling for CpGs. Also no evidence of transcription-coupled repair (power is limited though). 
	
	\item Clustering of mutations: beyond correlation with epigenomic variables: 78 clusters (up to 20kb) of 2-3 mutations. Both within and between individuals. Also a unique mutational spectrum in these clusters: enriched with C$>$G mutations. 
	
	\item Within population variation: correlation with local recombination rates, after controlling for CpG sites and GC content. 
	
	\item Correlation with human-chimp substitution rate (HCCG model): correlation with observed DNM rates $r = 0.18$. After adjusting for local recombination rates, $r = 0.37$. Also relative frequency of trinucleotide mutations are constant between DNMs and human-chimp divergence. 
	
	\item Buidling mutation rate map: for any 1M region $i$ and type $t$ (8 types, including one $CG$ dinucleotide mutation), let $r_{ti}$ be the rate. The procedure:
	\begin{itemize}
		\item Test correlation of human-chimp substitute rate with observed DNM rates for a type $t$, if significant, use the human-chimp rate as the baseline of $r_{ti}$; otherwise, use the global rate of that type of mutation.  
		\item Correct for local recombination rates $\rho_i$, linear regression of $r_{ti}$ and $\rho_i$. 
		\item Add scaling factor $f_t$ for each type of mutation, matching the observed frequency of type $t$. 
	\end{itemize}
	
	\item Remark: 
	\begin{itemize}
		\item The model is largely based on HCCG, however, even after justing for recombination rates, the correlation with DNM rate in human is only 0.37. 
	\end{itemize}
\end{itemize}

DNA methylation level of sperm is a major influence on the rate of de novo germline mutations [Li \& Wu, 2015]
\begin{itemize}
	\item Data: 4,470 exonic DNMs (4,143 DNSs and 327 DNIs) of 4,039 ASD trios and 2,061 DNMs (1,936 DNSs and 125 DNIs) of 2,299 control trios 
	\item Distributions of DNMs: 
	\begin{itemize}
		\item C$>$T/G$>$A transition accounts for the majority (55\%) of SNVs. Of these, 62\% are located on CpG sites (CpGs), which is significantly higher than average level (12\%). 
		\item Comparison of 6 subsitution types: only two types of bps (AT) or (CG), and each can mutate to 3 bases, so a total of 6. Next we define the context: CpG or non-CpG, so a total of 9 types of substitutions (only 3 for CpG context). DNMR of CpG C>T/G>A are 25-100 folds higher than other eight sub-types of DNSs (SNVs-8). 
	\end{itemize}
	
	\item Association of sperm DNA methylation levels with DNM rate: 
	\begin{itemize}
		\item For both C$>$T/G$>$A and SNVs8 (DNA methyl. levels are defined in the proximal CpGs), $R^2 > 0.9$ for both (previous studies found association at $R^2$ no more than 0.2). Method: define 5 levels of DNA methylation bins, then estimate DNMR for each bin. 
		\item DN indels: not significantly associated with DNA methylation 
	\end{itemize}
	
	\item Non-sperm results: the methylation level of non-sperm samples ($r < 0.75$) were less correlated with the reference sperm sample.
	\begin{itemize}
		\item Table S6: CpG rate, correlation is close to 0.9 in non-sperm cells. But SNV-8: correlation is much lower (0.3 vs. 0.9).  
	\end{itemize}
	
	\item Gene-specific DNMR: Higher correlation with the number of syn. SNPs from ExAC, $r^2 = .92$. Comparison with other methods:  
	\begin{itemize}
		\item Samocha: 0.90. Full model: trinucleotide sites, sequencing depth, local divergence rates, and replication timing, 0.925.  
		\item Francioli: 0.90. 
	\end{itemize}
	
	\item DNA methylation in sperm explains most of mutability in DNMs using WGS data of 250 trios.
	\begin{itemize}  
		\item Studying the relationship between DNA methylation and DNMR using WGS data: We separately divided CpGs and non-CpGs into 20 groups, and counted the expected DNMR based on our DNA methylation model and the observed DNMR in each group. 
		
		\item Figure 4AB. Sperm DNA methylation level could explain 94.7\% and 92.5\% of mutability at CpGs and non-CpGs in whole genome. 
		
		\item Define clusters: If a 20-kb window contains $> 2$ mutations which showed significantly smaller distance than expected, we defined these mutations as clustered DNMs. 
		
		\item Figure 4C: clustered DNMs and non-clustered DNMs. It was showed that the expected site-level DNMR of non-clustered DNMs was significantly higher than that of genomic background, but lower than clustered DNMs. 
	\end{itemize}
	
	\item Candidate genes from TADA:
	\begin{itemize}
		\item Use the mutation rates with TADA: 38 genes at FDR $< 0.05$ and 141 genes at FDR $< 0.3$, comparing with 2 and 17 in controls. 
		\item The ASD candidate genes were involved in synaptic function, chromatin remodeling, transcriptional regulation, and Wnt signaling. 
	\end{itemize}
\end{itemize}

Timing, rates and spectra of human germline mutation [Rahbari \& Hurles, NG, 2016]
\begin{itemize}
	\item Background: germline mosaicism. If mutations during development of germ cells, these mutations will appear in a fraction of germ cells (the earlier the mutations, the higher fraction). As a result, the DNMs in offsprings may share mutations. Also note that if the mutations occur before the separation of germ cells and soma, they will manifest as somatic mosaicism. 
	
	\item Data: WGS of 3 families, about 20 individuals. 60-70 DNMs per genome, with rate 1.28 $\times 10^{-8}$. 
	
	\item Patenal age effect: for about half of DNMs, parental origin can be determined. Variation of paternal age effect: one family, 1.5 mutations per year, but other two, 3.2-3.6 per year. 
	
	\item Mutation rate model during gametogenesis: deep sequencing 567x. About 1.3\% DNMs of siblings are shared. This allows one to estimate that 3.8\% of mutations in parental germ cells are shared. Infer the mutation rate change: Figure 6. Pre-PGC: about 0.2-0.6 mutation per replication; post-PGC: 0.5-0.7; after puberty: 0.1-0.2 (23 replications per year).  
	
	\item Mutational spectra: No significant difference in the spectra of paternal and maternal mutations; and no difference in mutations from young and old fathers. Mutation signatures from cancer: signature 1 and 5 correlate well with the germline mutations. 
	
	\item DNA methylation in testis cell line: 25\% CpG are methylated (above 50\% reads). 13 CpG sites overlap with DNMs, of which 12 were methylated.  
\end{itemize}

An expanded sequence context model broadly explains variability in polymorphism levels across the human genome [Aggarwala \& Voight, NG, 2016]
\begin{itemize}
	\item Background: mutational spectrum. When we do not care about the absolute mutation rates, the pattern of mutations can be summarized in two ways: 
	\begin{itemize}
		\item Mutational spectrum: among all mutations, the frequency of $X_1$ to $X_2$ mutations, written as $f(X_1 \rightarrow X_2)$. So let the total number of mutations be $n$, the number of mutations of a type follows multinomial distribution. 
		
		\item Conditional mutational spectrum: among positions with nucleotide $X_1$, how often it mutates to $X_2$, written as $f(X_1 \rightarrow X_2 | X_1)$. The number of mutations of all sites with nucleotide $X_1$ is given by the multinomial distribution. 
	\end{itemize}
	The two are related by: 
	\begin{equation}
	f(X_1 \rightarrow X_2) = p(X_1) \cdot f(X_1 \rightarrow X_2 | X_1)
	\end{equation}
	where $p(X_1)$ is the frequency of base $X_1$. 
	
	\item Model: use data from 1000 GP. For each position at the genome, say C, it has three polymorphism, $C \rightarrow X$. Let $n_C$ be the number of sites with $C$ in the reference genome, and $n_{CX}$ be the number of polymorphic sites, then $n_{CX}$ follows multinomial distribution. We model this distribution. Consider 6 mutational types, then 3-mer, 5-mer and 7-mer models. Show that the 7-mer model provides a significant better fit using LRT.  
	
	\item Feature selection: for any of 6 mutational type $X_1 \rightarrow X_2$, let $S$ be its sequence context (7-mer). The rate of $X_1 \rightarrow X_2$ mutations depend on $S$. We use a regression model: response variable is the mutational frequency of $X_1 \rightarrow X_2$ for base $X_1$ under a given $S$ (i.e. conditional mutational spectrum defined above); and the explanatory variables are features of $S$. The intuition is that this mutational frequency depends on $S$. For each $X_1, X_2$, we then have $4^6$ data points: 
	\begin{equation}
	P(X_1 \rightarrow X_2|S) = \alpha + \beta_1 p_1^{C} + \beta_2 p_1^{G} + \cdots + \beta_n p_7^T + \epsilon
	\end{equation} 
	For each $S$, we encode it as a set of features: first, single nt. feature, e.g. whether the first position is C or not. Next, incorporate interaction features. Use stepwise regression to selection features. 
	
	\item Selected features from stepwise regression: first, show that the selected models explain 81\% of variation of rates across contexts, comparing with 30\% using only 3-mer model. Note: this ignores variation of rates within a certain 7-mer context. Some features include, polyA or T motif. 
	
	\item Impact of DNA methylation: correlate the average mutation probability vs. average methylation level among all 7-mer contexts. Found a weak correlation $R^2 = 0.33$. 
	
	\item Detecting selection of genes: using the estimated NS. mutation (probabilities), estimate the number of NS polymorphic sites per gene, and compare with the expected number ($Z$-score).  
	
	\item Significant constraints in neuropsychiatric disease genes, OMIM genes, but less so in immune, OR genes. The scores work as well as Samocha in autism DNMs, and better than RVIS. 
	
	\item Remark: the feature selection model does not use count data, rather, it uses linear regression of rates. 
\end{itemize}

Discussion with Kelly Harris [2016]
\begin{itemize}
	\item Polymerase zeta (low fidelity) may create clustered mutations ($<100$bp): when high fidelity enzymes gets stuck, polymerase will fix the DNA damage, but often introduce wrong nucleotides (often A's). The most common signature: GC $\rightarrow$ AA. Overall, these clustered mutations explain about 2\% of SNPs. 
	
	\item Different mutation spectrum in different populations: eg. TCC 50\% more common in Europeans than non-Euro. Possible explanation: match UV mutational signature. 
\end{itemize}

Statistical methods for identifying sequence motifs affecting point mutations [Zhu \& Huttley, review for Genetics, 2016]
\begin{itemize}
	\item Motivation: suppose we want to study if neighborhood (context) affects mutation rates. Let's say $C \rightarrow T$ mutations. The possible confounder is the genomic regions. For example, suppose the mutations happen more often in exons (due to chromatin structure, etc.) than other regions, and exons are more rich in T's. Then we compare the context of C mutations vs. C non-mutations, we will find more T's near C mutations than non-mutations, even though the adjacant T's do not direct affect mutation rates. 
	
	\item Idea: match each mutation with a non-mutation (of the same base) in 300 bp. Then compare the context. 
	
	\item Model: let $i$ be a base and $j$ be status (mutaiton M or reference R), $f_{ij}$ be the counts. Then model $\log f_{ij}$: the effect of base (some bases are more mutable), of status, adjacent bases and interactions. If there is an interaction between adjacent base and status, this suggests that context modifies the mutation rate. 
\end{itemize}

Decoding germline de novo point mutations [NG, 2016; on Parent-of-origin-specific signatures of de novo mutations, NG, 2016]
\begin{itemize}
	\item Data: 800 trios, 36K DNMs. Assign a substantial proportion of DNMs to their parental chromosomes of origin: 7k DNMs. 
	
	\item 80\% of mutations originating during spermatogenesis. Paternal effect: about 1 DNM per father age. 
	
	\item Male and females: distinct mutational signatures. Evidence of non-replicative origin of DNMs in females. 
	
	\item Some regions of the genome show an enrichment of maternally derived mutations. Perhaps some DNMs provide a survival advantage to mutant eggs in aging women? 
\end{itemize}

Extremely rare variants reveal patterns of germline mutation rate heterogeneity in humans [Carlson \& Zollner, biorxiv, 2017]
\begin{itemize}
	
	\item Extremely rare variants (ERVs) from WGS of 3,000 subjects, about 36M ERVs. Advantages: earlier studies using variants of all AFs (ERVs only 25\%), which are subject to other processes such as gBGC and selection. The ERVs are very young, estimated 1,200 years old. 
	
	\item Defining relative mutation rate: consider a mutation of $X_1 \rightarrow X_2$, suppose it occurs in a context of $S$. The relative mutation rate is: $P(X_1 \rightarrow X_2 |S)$, the probability that at $S$, how often it is mutated. Specifically, it is the number of times we observe ERV with $X_1 \rightarrow X_2$ divided by the number of times $S$ (including $X_1$) occurs in the reference genome. Ex. 7,548 $C>T$ or $G>A$ autosomal singletons occurring in an ATACGCA or TGCGTAT 7-mer motif, and there are 53,314 such motifs in the autosomal reference genome, the relative mutation rate is 7,548/53,314 = 0.1416.
	
	\item Mutation rates in mutation subtypes (mutation type in a 1, 3, 5, 7-mer context): 3-mer context, 96 subyptes. 7-mer: 24,576 subyptes (256 times of 3-mer). 7-mer rates can vary 400 fold. Ex. NTT(A$>$T)AAA has 6-fold higher rate than average A $>$ T. 
	
	\item Genomic features: (1) Binary features: histone marks, DHS, CpG islands, exons, lamin-associated domains (2) Continuous features: recombination rate, replication timing and GC content, in nearby 10kb window centered at site.
	
	\item Model incorporating genomic features: train a genomewide logistic regression model. For a given subtype, each matched site in the genome is either a ERV $Z_j = 1$ or not $Z_j = 0$. Genomic features are explanatory variables and fit using \texttt{speedglm}. Estimating parameters for each subtype separately: for 84\% of 24,576 subtypes, 10 times more ERVs than number of parameters, so OK. 
	
	\item Influence of genomic features: (1) H3K9me3 (heterochromatin mark), recombination rate and late replication time, all associated with higher mutation rates across all subtypes. (2) H3K36me3, DHS, GC content and CpG islands: effects vary depending on mutation type and context. H3K36me3: regulate DNA MMR; CpG islands associated with lower DNA methylation. (3) For CpG $>$ TpG subtypes: lamin-associated domain: higher rate (DNA hypermethylation); histone marks H3K4me1, me3 and K27ac: lower rate (DNA hypomethylation). 
	
	\item Prediction of DNMs from trio-sequencing: classification of DNMs (46K) and non-DNMs (1M) with logistic regression, using predicted rate as covariate. Show that 7-mer model with genomic features (all included) performs the best (Figure 4). 
	
	\item Prediction of genomewide relative mutation rates: use the logistic regression model (trained genomewide). All 14 genomic features are used - no feature selection. 
	
	\item Possible mechanisms: (1) TFBS may increase mutation rate (limit the access of repair proteins): CEBP motifs within DHS show 2 fold lower mutation rates. (2) NTT(A$>$T)AAA: associated with L1 EN (retrotransposon)-induced damage. 
	
	\item Remark: genomic features are derived from multiple somatic cells; better to use germline cells. 
\end{itemize}

Reduced intrinsic DNA curvature leads to increased mutation rate [Duan and Qian, GB, 2018]
\begin{itemize}
	\item Experiment: yeast mutant strains, only LoF mutations in a gene can survive. Sequence 1000 strains and map nonsense mutation freq.
	
	\item Calculation of DNA structure features: Figure S3. A table of 17 DNA structure features for each dinucleotide. Sliding window (10bp) to estimate the DNA structure features.
	
	\item Correlating DNA structure features with mutation rates: vary sliding window sizes. At 100 bp windows, DNA curvature shows strongest correlation.
	
	\item Yeast and TCGA data: similar patterns.
	
	\item Possible mechanism: mutation of mismatch repair genes does not alter correlation. Possibly due to different mutagen sensitivity.
\end{itemize}

Characterizing mutagenic effects of recombination through a sequence-level genetic map [Science, 2019]
\begin{itemize}
	\item Background: most DSB ends up with lateral transfer of homologous segments (sometimes observable as gene conversions if there are heterzogyous markers). A small percent ends up as cross-overs.
	
	\item Data: 3000 WGS trios. 200K DNMs and 4M cross-overs. 
	
	\item Human recombination rate map: 700bp resolution. Cross-over rates: vary with epigenomic factors. 
	
	\item DNM rates: DNMs increases by 1.39 and 0.38 with every paternal and maternal age. Within 1kb of cross-overs: 50 times higher DNM rates. Females: higher DNM rates even within 40kb of cross-overs. Also change of DNM spectrum. 
	
	\item GWAS of recombination rates: and other phenotypes, e.g. the average GC content within 500 bp of cross-over locations. Found 35 loci. 
	
	\item Remark: mutation rates strongly correlate with cross-over events. However, cross-over events are random, though their distribution are highly non-random. 
\end{itemize}
%%%%%%%%%%%%%%%%%%%%%%%%%%%%%%%%%%%%%%%%%%%%%%%%%%%%%%%%%%%%
\section{Evolution of Human Population} 

Reference: [Hartl, Principles of Population Genetics, Chapter 9, 10]

Overview: the patterns of genetic variations, including polymorphism, LD \& recombination, divergence, in genome and across populations, reveal the underlying forces such as selection, mutational processes, human demography:   
\begin{itemize}
	\item Mutation rate: depends on sequence content, e.g. GC content (mutational bias). 
	\item Recombination rate: depends on repeats (high repeat regions may have high recombination rates due to unequal cross-over and gene conversion), and chromosome location (e.g. telemere, centreme, etc.). 
	\item Selection: coding sequences, promoters, 3' UTRs, introns, functional RNAs, etc. 
	\item GC content: often garge variation across the genome: e.g. in human, varies from 35\% to 60\%. Call the regions of high local similarity of GC content isochores. GC content variation can be caused by utational bias, selection (different on different regions) and biased gene conversion.
\end{itemize}

Patterns of polymorphism and recombination in model organisms: 
\begin{itemize}
	\item Yeast: excess polymorphism (copy number variation) in subtelomeric regions. Also note that these regions are enriched for genes with functions in transport, fermentation, etc. 
	\item Fly: low recombination rates and polymorphism in centromeres, Y chromosome, telemere. Note Y chromosome effectively has a smaller population size, thus low polymorphism. 
\end{itemize}

Polymorphism and LD pattern in human genome (average): 
\begin{itemize}
\item Data: 
\begin{itemize}
	\item Earlier projects: sequencing of chosen genes/regions of multiple individuals. 
	\item HapMap project: about 6.1 million SNPs in samples from 90 Yorubans (Nigeria), 90 CEPH (European ancestry) and 45 Han and Japanese. Not phased. Indels underrepresented (about one per 10 SNPs). 
	\item dbSNP: all known SNPs and indel polymorphism. 
\end{itemize}

\item Variant density: (1) SNPs: 3M in each pair of individuals (1 in 1000 bp), 30M in our species. (2) CNVs: 3-7 large CNVs per individual, 5-10\% individual have CNV $>100$ Kb, 1-2\% have CNV $> 1$Mb. 

\item Ascertainment bias of common variants: in HapMap, a SNP is genotyped in large populations only if it is found at least twice in the sample. Thus rare alleles are missed in the chosen SNPs. The bias can be seen easily from site frequency spectrum: uniform distribution (as opposed to large number of rare alleles as expected by mutation-drift). Need correction for analysis of polymorphism: the site frequency is weighted by the probability of discovery of a SNP.

\item Site frequency spectrum: with full sequence data (not HapMap), the excess of rare alleles, or Tajima's $D < 0$, across the genome. Thus not caused selection, but by human population growth (recent mutations). 

\item Haplotype blocks: block structure (high LD, measured by pairwise $r^2$, within blocks, but low LD between blocks). However, note that, even with very close regions, $r^2$ could be small. 

\item Local recombination rates: recombination hotspots, 80\% of recombination falls in 10-20\% chromosome. 
\end{itemize}

Polymorphism and LD patterns across different populations: 
\begin{itemize}
	\item Polymorphism: about six genetic clusters of human, corresponding to major geographical regions. $F_{ST}$ is about 0.07. About 93-95\% of total genetic variation occurs bewteen individuals within each group, with only 3-5\% between groups. 
	\item LD: African populations have larger effective population size, increased level of diversity, and reduced LD. Even though LD patterns are different, a common set of SNPs are highly predictive of linked SNPs in all human populations. 
\end{itemize}

Culture in human evolution: 
\begin{itemize}
\item Examples of culture influencing human evolution: in general, they relax selective pressure, and may be responsible for accumulation of deleterious mutations in human genome. Examples: lactose tolerance, Vitamin C synthesis in chimp (lost because of fruits in diet), short-sightedness and spectacles. 

\item Inferring the role of culture in human evolution: a critial question is when this happened, before or after major migrations out of Africa. One scenario: gene-culture co-evolution leading to genetic difference in response to culture difference among different populations (e.g. ashkenazi jews might have high IQ - very controversial hypothesis).  
\end{itemize}

Understanding Site Frequency Spectrum in human population [personal notes]
\begin{itemize}
	\item Goal: we estimate the SFS under neutrality, this would help interpret the empirical patterns from genetic variation data. 
	
	\item Number of variants: this depends on the gene length, e.g. $L = 5000$ bp and $N = 10000$, $\mu = 10^{-8}$, we have: 
	\begin{equation}
	\theta = 2 N \mu L = 1
	\end{equation}
	So at $n = 1000$, we have 7.5 variants in the gene and 10 at $n = 10000$. Nevertheless, it was found that the actual number of variants is much higher (10 times), see [Nelson \& Mooser, Science, 2012].
		
	\item Percent of rare variants among all variants: use $\E(S_i) = \theta / i$, we can show that: 
	\begin{itemize}
		\item At $n = 1000$ and AF $< 0.05$ as cutoff: percent of RVs is 60\%. 
		\item At $n = 10000$ and AF $< 0.05$: percent of RVs is 69\%. 
	\end{itemize}
	With natural selection, this ratio is higher. 
	
	\item Percent of singletons: similarly
	\begin{itemize}
		\item At $n = 1000$: 13\% of variants are singletons. 
		\item At $n = 10000$: 10\% are singletons. 
	\end{itemize} 
	
	\item Rates of rare variants per individual: about 100 nonsyn. variants. 
\end{itemize}

Most rare missense alleles are deleterious in humans: implications for complex disease and association studies [Kryukov \& Sunyaeva, AJHG, 2007]	
\begin{itemize}
	
	\item Background: why deleterious mutations could accumulate in the human population? 
	\begin{itemize}
		\item Evolution is not effective at eliminating these mutations: late-onset diseases, balancing selection, change of environment and lifestyle (e.g. thrifty genes). 
		\item Mutation-selection balance: relatively weak selection combined with a high mutation rate. Implication: Common Disease-Rare Variants. 
		\item The distinction between the two hypothesis largely lies in the parameters: the rate of deleterious mutations (the mutation rate and the chance of being deleterious) and the strength of selection acting on them. 
	\end{itemize}
	
	\item Data: sequences of 37 genes in about 700 people (obese and healthy, but the genes are not obesity-related). 
	
	\item Estimating the fraction of strong deleterious mutations among new missense mutations: "strong deleterious" is defined as LoF, similar to non-sense or splice site. 
	\begin{itemize}
		\item Idea: if all missense mutations are strong deleterious, then in the set of such mutations, it will be very frequent relative to the nonsense or splice site mutations (the ratio is determined by the mutation rates). Otherwise, the ratio would be much smaller. 
		\item Result: among all known disease-causing mutations (HGMD, Mendelian disease), the ratio of missense over nonsense is $3.9:1$, but by mutation rate, the ratio should be about $19.7:1$. So the ratio of strong deleterious mutations (causing gene LoF) in missense is about 3.9/19.7 = 20\%. 
	\end{itemize}
	
	\item Estimating the fraction of effectively neutral muations among new missense mutations: 
	\begin{itemize}
		\item Idea: these mutations will be fixed at the same rate as the syn. mutations. Thus by comparing the rate of fixation (using human-chimp comparison) of missense vs. syn. mutations, we can estimate this fraction. 
		\item Results: about 27\% are effectively neutral. 
	\end{itemize}
	
	\item Estimating the fraction of mildly deleterious alleles among standing variants: we known that about 53\% of new missense mutations are mildly deleterious (1 minus 20\% and 27\%). But what about the fraction in existing variants? 
	\begin{itemize}
		\item Using sequences of 37 genes in human: we use polymorphism data, specifically, the number of segregating sites (which should be proportional to mutation rate under neutrality). Focus on singletons in this analysis. The $N_a / N_s$ ratio (the number of nonsyn. vs. syn. sites) is compared with the theoretical ratio (based on mutation rate, about 2.2). The observed ratio is 1.49, thus 32\% (1-1.49/2.20) of missense mutations are deleterious. 
		\item Using existing SNP data in human: for common SNPs, $N_a / N_s$ is close to neutral expectation. For rare variants (MAF less than 1\%), the majority (52\%-71\%) of amino acid substitutions are mildly deleterious. 
	\end{itemize}
	
	\item \textbf{Summary: among missense mutations, about 20\% are strongly deleterious, 27\% neutral and the rest (53\%) mildly deleterious}. 
	
	\item Lessons/Remark: 
	\begin{itemize}
		\item \textbf{General strategy for estimating selection/fraction of deleterious mutations: extension of $dN/dS$ test}. Under neutral model, the expected ratio of different types of variants (singletons) is determined by mutation rates (in some cases, these are known, e.g. ratio of syn. vs. nonsyn. mutations). The depature of this pattern is due to selection and can be used to estimate the strength of selection. 
		\item Limitation: assumption that under alternative model (selection), all singletons (variants) will be removed by selection. 
		\item Sources of data to estimate these fractions: human-chimp divergence (pattern of fixation vs. mutation), known disease-causing mutations, human polymophism data (pattern of singletons). 
	\end{itemize}
\end{itemize}

A map of human genome variation from population-scale sequencing, [1000 Genome Project, Nature, 2010]
\begin{itemize}
	
	\item Data: 1000 Genome Pilot Project, low-coverage (2-4) WGS of about 100 individuals and a small number of trios and WES of about 600 individuals. 
	
	\item Power analysis for variant discovery: 
	\begin{itemize}
		\item We estimated that 250 samples sequenced at low coverage would be needed to find 99\% of the synonymous variants in an individual, and with 320 sequenced samples 98.5\% of nonsynonymous and 96.3\% of LOF variants would be found.
	\end{itemize}
	
	\item Pattern of genetic variation: 
	\begin{itemize}
		\item We estimated that an individual typically differs from the reference at 10,000-11,000 nonsynonymous sites. 
		\item LoF variants per individual: in frame indels (190-210), premature stop codons (80-100), splice site disrupting variants (40-50), and deletions that shift reading frame (220-250). 
		\item Each genome is heterozygous for 50-100 variants classified by the Human Gene Mutation Database (HGMD)
		\item Putative functional variants had an allele frequency spectrum depleted at higher allele frequencies
	\end{itemize}
	
	\item Selection on the genes: 
	\begin{itemize}
		\item Comparison of diversity (measured by average heterozygosity) across different regions: lowest in exons, etc. reflecting selection. 
		\item Diversity/divergence: roughly constant for different elements (exons, introns, etc.), suggesting that variation in diversity can be explained that by divergence. Explanation: a certain fraction is under selection, the rest (polymorphic sites) are mostly neutral, thus both diversity and divergence are scaled by the same ratio (1 - the fraction of selected sites), so the ratio is constant. The results suggest that the common part of the allele frequency spectrum is dominated by effectively neutral variants. 
	\end{itemize}
	
	\item Applications of 1000 Genome data: 
	\begin{itemize}
		\item Impute untyped variants in GWAS: find better association. Ex. in one eQTL study, by imputation, the SNP significance is increased from $P$ value $10^{-8}$ to $< 10^{-15}$. 
		\item Filter of variants in the study of Mendelian diseases. 
	\end{itemize}
	
	\item Remark: the results do not show that most rare variants are neutral. The diversity (average heterozygosity) is dominated by common variants, so the results only suggest that common variants tend to be neutral. 
\end{itemize}

Genetic variation and population size: [Barton, PG, 2010] the neutral theory predicts that the nucleotide diversity (the number of mismatches between two sequences), $E(\Pi) = \theta = 4N \mu$, but the observed diversity varies only about an order of magnitude, while the population size varies far more. The possible explanations (for variations at very large populations):  
\begin{itemize}
\item Effective population size: in the short-term, the factors such as sex ratio may reduce population size by one order of magnitude; in the long term, population bottlenecks limit the effective population size. 
\item Selective sweeps: genetic hitchhiking reduces genetic variation as linked alleles are fixed together. Evidence: regions of low recombination rate have lower level of genetic variation. 
\item Deleterious mutations: eliminated by selection, and reduce genetic variations. 
\end{itemize}

Adaptation in large populations: [Barton, PG, 2010]
\begin{itemize}
\item Mutations at large populations: e.g. for Drosophila, the actual population size is much bigger than that suggested by the genetic diversity (about $10^6$). Even in a single orchard, the population size could reach $10^6$, and the size in a local area may reach $10^8$. Thus assuming mutation rate $\mu \approx 10^{-8}$, effectively, the mutation rate per nucleotide is: $2 N \mu \approx 2$, i.e. every possible mutation will occur in every generation in a local area. 

\item Adaptation of large populations: because of high mutation rates, benefical mutations (either from standing variation or from new mutations) are likely to appear independently mutliple times in a local area, following environmental changes (say, introduction of an insecticide). These mutliple mutations all contribute to adaptation and each affecting the surrounding DNA sequences (genetic hitchhiking). Thus after adaptation: the population is dominated by a few independent alleles, each associated with a distinct haplotype. 
\end{itemize}

An integrated map of genetic variation from 1,092 human genomes, [1000 Genome Project, Nature, 2012]
\begin{itemize}
	\item At the most highly conserved coding sites, 85\% of nonsynonymous (NonSyn) variants and over 90\% of STOP gain and splice-disrupting variants are below 0.5\% in frequency , compared to 65\% of synonymous (Syn) variants.
	
	\item The least conserved splice-disrupting variants show rare-variant load similar to synonymous and non-coding regions suggesting that these alternative transcripts are under very weak selective constraint.  
	
	\item The NonSyn to Syn ratio among rare ($<0.5\%$) variants is typically in the range 1-2 and among common variants in the range 0.5-1.5, suggesting that 25-50\% of rare NonSyn variants are deleterious. 
	\begin{itemize}
		\item Certain groups (e.g., ECM-receptor interaction, DNA replication and pentose phosphate pathway) show a substantial excess of rare coding mutations, which is only weakly correlated with the average degree of evolutionary conservation.
	\end{itemize}
\end{itemize}

Evolution and Functional Impact of Rare Coding Variation from Deep Sequencing of Human Exomes [Tennessen, Science, 2012]
\begin{itemize}
	
	\item Data: ESP, 111x coverage in 2440 individuals of European (n = 1351) and African (n = 1088) ancestry. 
	
	\item Purifying selection on rare variants: 
	\begin{itemize}
		\item Of the total SNVs, 57\% (285,857) were singletons, and SNVs with three or fewer minor alleles accounted for 72\% of all variants. 
		\item The overall site frequency spectra (SFS) and the SFS for both AAs (African) and EAs (European) are highly skewed, exhibiting a large excess of rare variants relative to the standard neutral model. Tajima's D was highly negative for both EAs (-2.12) and AAs (-2.10) and dropped precipitously as sample size increased. 
	\end{itemize}
	
	\item Functional impact of RVs: using Polyphen, SIFT, MutationTaster, GERP, PhyloP, etc.  
	\begin{itemize}
		\item About 47\% of all SNVs (74\% of nonsynonymous and 6\% of synonymous variants) are predicted to be deleterious by one or more method. 
		\item Overlap among methods is modest. For example, only 1\% of nonsynonymous variants are predicted to be functional by all seven methods, and variants predicted by any single approach are likely to have a high false-positive rate. 
	\end{itemize}
	
	\item Summary: most CV's are not under selection, but rare variants are. The evidence of RV selection comes from damaging effect analysis (complements the earlier ASHG'07 study that uses singlenton density and divergence data). 
\end{itemize}

An Abundance of Rare Functional Variants in 202 Drug Target Genes Sequenced in 14,002 People [Nelson \& Mooser, Science, 2012]
\begin{itemize}
\item Motivation: use sequencing data of 202 genes to study the pattern of genetic variation in human, including, excess of rare variants, inference of population demography, and natural selection. 

\item Data: 202 genes that are potential drug targets, in 14,002 people (from multiple disease cohorts), mostly European ancestry. 

\item Excess of rare variants (0.5\%): 
\begin{itemize}
	\item Among all varaiants, 95\% are rare, and more than 74\% observed in 1 or 2 subjects. 
	\item Watterson's estimator ($\theta_W$) is nearly 10 times larger than pairwise estimator (Tajima, $\theta_{\pi}$). Explanation: exponetial population growth, the coalescence have long terminal branches (a large number) and short roots - comparing with constant population size tree, there are more variants, leading to larger estimator. 
\end{itemize}

\item Inference of population demograhy and mutation rates: using only syn. sites. 
\begin{itemize}
	\item Current: $N_e = 10,000$ based on pairwise estimator, however, $\theta_W >> \theta_{\pi}$, signature of rapid growth. Using likelihood model with genealogy (MCMC for inference), find 1.7\% population growth rate, and effective European population size of 4.0 million. 
	\item Mutation rates: the same model can also estimate mutation rates. Median estimate: $1.38 \times 10^{-8}$. 
\end{itemize} 

\item Natural selection on genes: using the approach of AJHG'07, comparison of NS vs. S in variants with different AFs. Estimate that roughly 70\% of all NS singletons are sufficiently deleterious that they will never reach 5\%. Only 13\% of new NS mutations are so deleterious that they cannot be observed even as singletons in this sample size. 

\item Discussion: the estimation of demography ignores background selection on S variants. 
\end{itemize}

%%%%%%%%%%%%%%%%%%%%%%%%%%%%%%%%%%%%%%%%%%%%%%%%%%%%%%%%%%%%
\section{Evolution of Multi-State Systems}

Multi-state systems: the fitness of states and mutation rates among states could be very unequal, e.g. some states have high fitness, but mutationally less accessible. We are interested in which states/regions will be most likely: the high-fitness ones vs. the ones with high entropy. Two approximations:
\begin{itemize}
\item Mutation-selection balance: applicable if $4N \mu >> 1$. In the two-state case, the frequency of state $a$ would be proportional to $\mu(b,a)/F(a)$, where $\mu(b,a)$ is the mutation rate from $b$ to $a$, and $F(a)$ is its fitness. In the general case, mean-field approximation. 

\item Successive fixation: among all possible states, applicable if $4N \mu << 1$. In each time point, population is fixed at one state, and the system can be viewed as a Markov chain. The rate of switching state $a$ to $b$ is given by: 
\begin{equation}
u(a,b) = 2 N \mu(a,b) \frac{1 - e^{-2[F(b) - F(a)]}}{1 - e^{-4N[F(b)-F(a)]}}	
\end{equation}
The rate of switching to a favorable mutation with selection coefficient $s$ is approximately: 
\begin{equation}
u(a,b) = \mu(a,b) \frac{4Ns}{1 - e^{-4Ns}}
\end{equation}
The rate of switching to a deleterious mutation with selection coefficient $s$ is approximately: 
\begin{equation}
u(a,b) = \mu(a,b) \frac{4Ns}{e^{4Ns} - 1}
\end{equation}
Putting the two together, we have $u(4N \Delta s)$ as a monotonic function of $4N \Delta s$ ($\Delta s$ is signed): $u \approx \mu$ when $\lvert 4N \Delta s \rvert$ is close to 0; $u$ decreases to 0 about exponentially as $4 N \Delta s$ reduces to negative values; and increases exponentially as $ 4 N \Delta s$ increases to positive values.  

\item Low-fitness states: Under the mutation-selection balance model, a low-fitness state/region is likely if the mutational influx to this state/region is high. Under the population random walk model, fixation of low-fitness state/region is very unlikely, as probability of fixation of deleterious mutations is apprxoimately $4Ns / e^{4Ns}$  (very small if $4Ns > 10$). It should be understood however, under this scenario, even though low-fitness states are unlikely to be fixed, they could reach substantial frequency.

\item Weakly deleterious selections: (the fate of new mutations) for large populations, they will be eliminated by selection; for small populations, we see that, the probability of fixation is (use $e^{4Ns} \approx 1 + 4Ns$): $u(a,b) \approx \mu(a,b)$, thus weakly deleterious selection almost as likely to be fixed as neutral mutations. 
\end{itemize}

Successive fixation model: [Sella \& Hirsh, PNAS, 2005]
\begin{itemize}
\item Assumption: $N \mu << 1$, the mutation is rare, thus the population will undergo successive fixation (the linked polymorphism can be ignored). The probability of fixation of any site does not depend on other segregating alleles. Also assume that the mutation rates between any two states are equal. 

\item Equlibrium distribution of states: let $W_{ij}$ be the rate of substitution from state $i$ to $j$, and $f_i$ be the fitness of state $i$. Using the equation of substitution rate: 
\begin{equation}
\frac{W_{ij}}{W_{ji}} = \frac{f_j^{\nu}}{f_i^{\nu}}	
\end{equation}
where $\nu = 2N - 1$. This implies the detailed balance: 
\begin{equation}
W_{ij} f_i^{\nu} = W_{ji} f_j^{\nu}	
\end{equation}
Let $x_i = \ln f_i$ (called the additive fitness), the equilibrium distribution is given by: 
\begin{equation}
P_i^* \propto f_i^{\nu} = e^{-\nu(-x_i)}	
\end{equation}
This is analogous to the Boltzmann distribution: $-x_i$ can be viewed as the (negative) energy of a state, and $\nu$ is analogous to $\beta = 1/k_B T$. When population size is infinite (temperature at 0), the population will be fixed on the highest fitness state. 
 
\item Free fitness function: the dynamics of system (how probability distribution of states change) can be characterized by the free fitness function: 
\begin{equation}
G = \langle \ln f \rangle	 + \frac{1}{\nu} S
\end{equation}
where the first term is the average fitness over all states, and $S$ is the entropy of states, $S = -\langle \ln P \rangle$. So we could write $G$ as: 
\begin{equation}
G = \sum_i P_i \left[ \ln f_i - \frac{1}{\nu} \ln P_i \right]	
\end{equation}
The free fitness function should monotically increases, and converges to the equilibrium distribution defined above. The equlibrium state reflects the balance between the tendencies to increaese fitness and entropy (analogous to physical system at thermal equilibrium). 

\end{itemize}

Evolution of sequences: let $L$ be sequence length, the total mutation rate is thus $4NL \mu $: 
\begin{itemize}
	\item Short sequences (e.g. TFBSs): for large population ($4 N L \mu > 1$), this can be approximated by mutation-selection balance; for small population, may be apprxoimated by a Markov chain. 
	\item Long sequences (regulatory sequences or genes): generally $4NL\mu > 1$, thus substantial polymorphism. When $4N L \mu < 1$, and most mutations are neutral, then the population may be fixed at a single sequence most of the time (see below). 
	\item Limitation of monomorphism assumption: even at $4 N L \mu < 1$, the Markov chain model may be inadequate. If beneficial mutations are relatively common, then before fixation of the current mutation (at the level of $4N$ generations), a new ``fixable'' mutation may occur (more than once per $4N$ generations because of the increase of fixation probability of advantageous mutations). 
\end{itemize}

%%%%%%%%%%%%%%%%%%%%%%%%%%%%%%%%%%%%%%%%%%%%%%%%%%%%%%%%%%%%
%%%%%%%%%%%%%%%%%%%%%%%%%%%%%%%%%%%%%%%%%%%%%%%%%%%%%%%%%%%%
\chapter{Linkage Analysis}
\section{Introduction to Linkage Analysis}

Reference: [Thomas, Statistical Methods in Genetic Epidemiology, Chapter 7]

Recombination: [Human Molecular Genetics]
\begin{itemize}
	\item Recombination fraction: defined as the fraction of recombinants in an individual with heterozygotes in both loci, denoted as $\theta$. Always no more than 0.5 (when two loci are completely independent). Given a heterozygote $(A_1 B_1, A_2 B_2)$, the frequency of four genotypes in the gametes are: 
\begin{equation}
f(A_1 B_1) = f(A_2 B_2) = \frac{1-\theta}{2} \qquad f(A_1 B_2) = f(A_2 B_1) = \frac{\theta}{2}
\end{equation}
		\item Genetic distance is defined in the units of cenimorgan (cM), which corresponds to $1\%$ chance of recombination between two loci. In mouse/human, about $0.5 - 1$ cM/Mb.
	\item Mapping functions: map genetic distance to physical distance. This is not linear because of the possibility of multiple cross-overs (always no more than 0.5, but physcial distance can be arbitrarily large). Ex. Haldane's function: $w = -1/2 \ln(1 - 2\theta)$, where $\theta$ is the recombination fraction.
	\item Distribution of recombination: highly non-uniform. Ex. very high in the telomere regions of male chromosomes. In human, chromosomes consist of conserved blocks, typically, 20-50 kb, separated by 1-2 kb recombination hotspots (about 95\% of all recombinations). 
\end{itemize}

Principles of linkage analysis: 
\begin{itemize}
\item Linkage mapping in families: consider a polymorphic marker and the disease locus, the ratio of recombinants and non-recombinants in the offsprings (which can be detected through the combination of marker genotype and the disease state) reveals the genetic distance between the two. In particular, the presence of non-recombinants indicate the linkage between the marker and disease (more non-recombinants, tighter linkage). 

\item Intuition of linkage mapping: suppose we have a fully penetrant locus and a linked marker, then from the data, we can infer the complete genotype of all individuals. Suppose phasing is also not a problem, then we have the haplotypes. This allows us to estimate the fraction of recombinant and non-recombinant haplotypes, thus comparing it with random expectation ($\theta = 1/2$); furthermore, the fraction reveals information of the distance.  
	
\item Limits of linkage mapping: when no recombination is present, either from the small sample size (e.g. in family studies) or from the low recombination rates in most part of the genotypes, there will be only non-recombinants, therefore the linkage mapping cannot narrow down to smaller regions. Ex. with genetic distance $\theta = 0.01$, need at least 100 samples to have one recombination event, thus the resolution would be at the level of 1 Mb. 
	
\item Association with disease/LD mapping: the basic idea of enrichment of non-recombinants can be applied at the population level. When the marker and the disease allele is tightly linked, there is LD between the marker and the disease gene (the two co-segregate in a haplotype block). Since the disease genotype is not directly observed, this LD is manifested as the association of the marker(s) with the disease status.
	
\item Linkage vs. assocition (LD) mapping: let $M$ be a marker and $D$ be a disease locus, linkage mapping relies on recombination between $M$ and $D$ within families, and association mapping relies on the LD between $M$ and $D$ at population level. The two methods are different: 
\begin{itemize}
\item It is possible to have linkage but not association: $M$ and $D$ may be close, but not in the same LD block, then association would not identify $D$, but linkage mapping can. 
\item It is possible to have association but not linkage: population stratification may create false associations. 
\end{itemize}
	
\end{itemize}

Direct counting method: 
\begin{itemize}
\item Method: suppose we can infer the full haplotypes of all subjects, and find that the number of recombinants and non-recombinants are $r$ and $s$ respectively. Then we can use McNemar's test: $(r-s)^2/(r+s)$ follows $\chi^2$ distribution under the null model. 

\item Phasing problem: In practice, even for fully penentrant diseases, the direct counting method generally cannot be applied because the phasing is unknown. Under special cases, however, the phasing may be inferred, e.g. when one parent has identical alleles, then which one is transmitted to the child does not matter. 
\end{itemize}

Identity by state (IBS) and identity by descent (IBD) in sibling pairs: [Thomas, Chapter 7]
\begin{itemize}
\item IBD and type of relationships: Table 7.2 of [Thomas]. Let $\pi_0, \pi_1$ and $\pi_2$ be the probabilities of sharing 0, 1 and 2 IBD alleles. Then IBD for some common relationships are shown in Table~\ref{tab:IBD}

\item Computing IBD from marker genotypes of sib-pairs: let $Z_A$ be the number of shared IBD alleles at the marker $A$ between the sib-pair, and $M_o^A$ be the observed marker genotypes at $A$. We define: 
\begin{equation}
Q_a^A = P(Z_A = a|M_o^A)	
\end{equation}
To compute this, we first convert it to $P(M_o^A|Z_A = a)$ using Bayes Theorem. This term is then computed by summing over the parental genotype: 
\begin{equation}
P(M_o^A|Z_A = a) = \sum_m P(M_p^A=m|p^A) P(M_o^A|M_p^A=m, Z_A = a) 	
\end{equation}
where $p^A$ is the allele frequency in the population at the marker $A$ and $P(Z_A) = (1/4, 1/2, 1/4)$. Note that not genotypes are possible: we should only consider those that are consistent with $Z_A = a$. 

\item Computing IBS between sib-pairs: suppose we want to compute the probability that IBD between the sib-pair is $n$. This is done by summing over parental genotypes and offspring genotypes: 
\begin{equation}
P(\text{IBS}=n) = \sum_{M_p} P(M_p|p) \sum_{M_o} P(M_o|M_p) I(\text{IBS} = n|M_o)	
\end{equation}
where $p$ is the allele frequency and $I(\cdot)$ is the indicator function. 

\item Computing conditional IBD at different loci between sib-pairs: if we know IBD at one marker, then we should be able to get some information of the IBD at an adjacent marker. In the extreme case, when the two are completely linked, the IBD should be identical. More generallly, the conditional distribution would depend on the recombination fraction $\theta$. If we denote $s$ and $t$ the segregation indicators of the four alleles in the markers $A$ and $B$, the conditional IBD distribution:
\begin{equation}
T_{ab} = P(Z_B = b|Z_A = a) = \frac{\sum_{s,t} I(Z_A = a |s) I(Z_B = b |t) \theta^{r_{st}} (1-\theta)^{4-r_{st}}}{P(Z_A = a)}	
\end{equation}
where $r_{st}$ is the number of recombinations, which is derived from $s$ and $t$. 
\end{itemize}

\begin{table}
\centering
\begin{tabular}{l|lll}
\hline
Relationship & $\pi_0$ & $\pi_1$ & $\pi_2$ \\
\hline
MZ twins & 0 & 0 & 1\\
Full sibs & 1/4 & 1/2 & 1/4\\
Parent-offspring & 0 & 1 & 0\\
\hline
\end{tabular}
\caption{IBD for different relationships}
\label{tab:IBD}
\end{table}

Questions of linkage analysis: 
\begin{itemize}
\item How sequencing would benefit linkage analysis: because of the limit of resolution, it is hard to narrow down to individual genes even if we have the full sequences. The true benefits are probably the causal loci from the identification of very rare variants, and other information such as natural selection. 

\item In ASP analysis: how are the IBD status are determined? 
\end{itemize}

%%%%%%%%%%%%%%%%%%%%%%%%%%%%%%%%%%%%%%%%%%%%%%%%%%%%%%%%%%%%
\section{Parametric linkage analysis} 

Reference: [Human Molecular Genetics, 3rd Ed, Chapter 13, 14; Yang, Introduction to Statistical Methods in Modern Genetics, Chapter 2; Thomas, Chapter 7]

Two-point mapping: 
\begin{itemize}
\item Intuition: the fraction of non-recombinants (between the disease locus and the marker) indicates the extent of linkage between the two loci. To detect recombinants and non-recombinants, one of the parent must be a double heterozygote; if not the recombinantion will not change the probabilities of the haplotypes in the offsprings. 

\item Model: given the marker genotype and disease status of a pedigree, we test the hypothesis in terms of the genetic distance $\theta$. $H_0$: the marker is not linked to the disease, i.e. $\theta = 0.5$ vs. $H_A$: the marker is linked to the disease, i.e. $\theta < 0.5$. This is done through the likelihood ratio test. We define $g$ as genotype and $x$ as phenotype. $P(x|g)$ is the penetrance function. 

\item Likelihood [Thomas]: let $\mathbf{Y}$ be the phenotypes and $\mathbf{M}$ be the observed marker genotypes. Furthermore, we have parameters $\theta$ for recombination, and $\omega = (f, q)$ be the segregation parameters (penetrance and allele frequency). The likelihood is: 
\begin{equation}
L(\theta, \omega) = P(\mathbf{Y}|\mathbf{M}) = \sum_g \prod_i P_f(Y_i|G=g_i) P_{\theta}(G=g_i|M_i, g_{f_i}, g_{m_i}, M_{f_i}, M_{m_i})
\end{equation}
where $f_i$ and $m_i$ are father and mother of $i$, respectively. The last term will be replaced by $P_q(G=g_i|M_i)$, if $i$ is a founder. Usually most linkage analysis assume linkage equilibrium. The summation over $g$ (four possible genotypes per subject) can be done by the Elsteon-Steward algorithm. The transition probabilities in one generation depend on recombination (Table 7.1 of [Thomas]). In general, phasing is unknown and some marker genotypes may be unobserved. When a marker has two alleles, there are 10 possible hayplotypes (Table 7.1). 

\item Example: Suppose we are analyzing the data where $A$ and $D$ represents the disease status (affected or unaffected). We use $D,d$ for disease gene (recessive) and $1, 2,3,4$ for marker alleles. Consider the following observation: 
\begin{equation}
U14 \times U14 \rightarrow A11	
\end{equation}
The likelihood computation consists of the following steps: 
\begin{itemize}
\item Parent genotype: since the disease is recessive and one child is affected, the parent genotypes must be $Dd$ and $Dd$. Thus we have the mating type: $(Dd)/(14)$ and $(Dd)/(14)$ (where the parenthesis means the phase is unknown). 
\item Phasing: each parent could have two phases $D1/d4$ or $D4/d1$, thus there are a total of 4 mating types. 
\item Offsprings: conditioned on one mating type with phasing: $D1/d4 \times D1/d4$. The genotype of the offspring is $(DD)/(11)$ or $(Dd)/(11)$. The first can be only obtained through no-recombination and the second by recombination in one parent only, thus the probability is: 
\begin{equation}
P(A11 | D1/d4, D1/d4) = (1-\theta^2) / 4 + \theta (1 - \theta) / 2	
\end{equation}
\end{itemize}

\item Parameter estimation: generally linkage analysis assumes that the segregation parameters are are known from separate segregation analysis, thus $f$ and $q$ are fixed in estimation. This simplifies parameter estimation. In addition, the segregation parameters may be biased if using only the genotyped families (ascertainment bias).  

\item Importance of extended families: more information regarding phase is provided. Ex. when the phase of an individual cannot be resolved: $D A_1$ or $D A_2$, where $D$ is the disease allele; the genotype of $A$ locus at the parent carrying $D$ may help to resolve. 
\end{itemize}

LOD score: 
\begin{itemize}
\item Definition: the LRT score is defined as: 
\begin{equation}
Q = -2 \ln \frac{L(\theta=0.5)}{\max_{\theta} L(\theta)}	
\end{equation}
It is common in genetics to use LOD score, defined as: 
\begin{equation}
LOD = \log_{10} \frac{\max_{\theta} L(\theta)}{L(\theta=0.5)}
\end{equation}
For a particular $\theta$, often use: 
\begin{equation}
Z(\theta) = \log_{10} \frac{L(\theta)}{L(\theta=0.5)}	
\end{equation}

\item Distribution: $Q$ normally follows $\chi^2$ distribution with df 1; in our case, since $H_A$ is one-sided, the $p$ value will be half of that under the two-sided alternative hypothesis. Thus $Q$ follows a mixture distribution: 
\begin{equation}
Q \sim \frac{1}{2} \chi^2 + \frac{1}{2} \{0\}	
\end{equation}
Then $4.6 \times LOD$ follows the mixture distribution. The threshold for $LOD$ score is usually $+3$ for acceptable (and $-2$ for exclusion). This is used even when the number of markers is high [Lander \& Botstein, 1989]. 
\end{itemize}

Performance/power analysis: we study this with double backcross $(Aa)/(21) \times (aa)/(11)$, where $A,a$ are trait alleles and $1,2$ are marker alleles (phasing may or may not be known). 
\begin{itemize}
\item Strategy: to assess the power of a test, we use the distribution of the test statistic under $H_A$. This can be done in two manners: expected LOD score and the sample size requirement to achieve a certain power. The key to derive the distribution is to express the LOD score (or $Z(\theta)$) under a given $\theta$, in terms of probabilities that can computed from basic model parameters including $\theta$. 

\item Complete penetrance with phasing known: the mating type is $A2/a1 \times a1/a1$. Suppose there are $n$ offsprings from this type of mating (often in experimental animals/plants, or many families with the same mating types), we could test the recombination parameter $\theta$ using the fraction of non-recombinants. The sample size requirement, at the true $\theta = \theta^*$ vs. $H_0: \theta = 1/2$ is given by (from power calculation of binomial distribution): 
\begin{equation}
n = \left[ \frac{z_{\alpha} \sqrt{\theta_0(1-\theta_0)} + z_{\beta} \sqrt{\theta^*(1-\theta^*)}}{\theta_0 - \theta^*} \right]^2	
\end{equation}
Note that the threshold calculation can also be performed on the LOD score (which follows $\chi^2$ distribution, but the square root follows normal distribution). 

\item Incomplete penetrance: reduces the power (or increases sample size requirement). In this case where the penetrance is $\lambda$, we cannot use the simple binomial distribution. We have double backcross: $A2/a1 \times a1/a1$, the probability of the offsprings (let $A$ be affected and $U$ unaffected): 
\begin{equation}
\begin{array}{ll}
p_1 = P(A11) & = (1 - \theta) \lambda / 2\\
p_2 = P(A12) & = \lambda \theta / 2\\
p_3 = P(U11) & = (1 - \lambda + \lambda \theta ) / 2\\
p_4 = P(U12) & = (1 - \lambda \theta) /2
\end{array}	
\end{equation}
The likelihood function is: 
\begin{equation}
L(\theta) = \prod_{i=1}^4 p_i(\theta)^{n_i}
\label{eq:double_backcross}
\end{equation}
where $n_i$ are the number of individuals of the type $i$. This would allow us to compute the expected LOD as (using $E(n_i) = n p_i$): 
\begin{equation}
ELOD = n \sum_{i=1}^4 p_i \log_{10} \frac{p_i(\theta)}{p_i(0.5)}	
\end{equation}

\item Unknown phasing: reduces the power. This is similar to the case above, but need to replace the probabilities taking unknown phasing into account. The distribution of $Z$ is approximately normal because it is a sum of $Z$-scores for many families. The expectation and variance of $Z$ can be computed from Equation~\ref{eq:double_backcross}. 

\end{itemize}

Multi-point mapping: how can multiple markers help? 
\begin{itemize}
\item More informative families: if there is only one marker, then in some families, the marker may happen to be monomorphic in the affected parents, thus these families are not informative. With more markers, it is likely that for any family, at least some marker may be polymorphic and thus informative. 
	
\item Rare events: with more markers, say two, it is likely to observe double recombination events. In general, rare events are more informative. Ex. in estimating the parameter of a Bernoulli distribution, $p$, the variance of the statistic: 
\begin{equation}
\text{Var}(\hat{p}) = \frac{p(1_p)}{n}	
\end{equation}
It is clear that the variance is small at small value of $p$. Suppose we have two markers $B,b$ and $E,e$, and disease alleles $D,d$. Then in the offsprings of $BDE/bde$, the haplotypes $BdE$ or $bDe$ require two recombination events. 
\end{itemize}

Multipoint linkage analysis: [Thomas, Chapter 7]
\begin{itemize}
\item Order of markers and direct generalization of two point analysis: when the number of markers is small, we can directly generalize the two-point analysis. For simplicity, we consider only markers, say $A,B,C$. Consider one particular ordering $ABC$, the likelihood (consider only genotypes) would depend on $\theta_{AB}$ and $\theta_{BC}$ - bascially we consider all recombinations between $A$ and $B$, and between $B$ and $C$. We can also simplify the model by using a single parameter $x$, the location of $B$. Then both $\theta_{AB}$ and $\theta_{BC}$ can be written as functions of $x$ using the Haldane map function. 

\item Challange of multipoint linkage: when there are multiple markers, the likelihood calculation would involve summartion of $G$, which goes exponential with the number of markers. Furthermore, there is clearly dependency between markers, and the likelihood cannot be factored into individual markers. 

\item Intuition: there is some conditional independence between markers: e.g. suppose two markers, $A$ and $B$, between two subjects, are IBD, then one can infer that there is no recombination between the two markers and thus all markers between $A$ and $B$ are IBD. More generally, given the IBD status of two markers $A$ and $C$, the IBD status of any marker $B$ between $A,C$ would be conditionally independent of all the other markers. 

\item Computing IBD probabilities by HMM (Lander-Green algorithm): following our notation in the section ``IBS and IBD in sibling pairs'', the IBD probabilities of $B$ (a vector, $\pi_0$, $\pi_1$, $\pi_2$) in a sib pair is given by: 
\begin{equation}
\pi_B = Q_A T_{AB} T_{BC} Q_C	
\end{equation}
In sibpair analysis, we need to determine the IBD of the location of the disease locus $x$. Let it be $Z_x$, and we need to estimate $P(Z_x = z)$ where $z = 0,1$ or $2$. Suppose it's between marker $i-1$ and $i$, we can have the IBD status of $x$ as: 
\begin{equation}
\pi_x = P(Z_x | M, x) = \mathbf{1} Q_{1,l-1} T_{l-1,x} T_{x,l} Q_{l,L} \mathbf{1}
\end{equation}
where $Q_{l,m} = Q_l T{l,l+1} Q_{l+1} \cdots Q_{m-1} T_{m-1,m} Q_m$ represents the results of matrix multiplication of all markers between $l$ and $m$ inclusive. Now we can extend this to pedigrees. We use indciator variable $Z$ for possible IBD configurations (for each subject in the pedigree, which ancestral allele its genotype comes from). The corresponding IBD probabilities are denoted as $\Pi(Z_x)$, and can be computed similarly. 

\item Parameteric analysis: the likelihood can be represented in terms of the location $x$ and the segregation parameter $\omega = (f,q)$:
\begin{equation}
L(x, \omega) = P(Y|M)=\sum_g P(Y|G_x=g) P(G_x=g|M) = \sum_z P(Y|Z_x = z) \Pi(Z_x =z)
\end{equation}
Note that instead of summing over genotypes (which is exponential of the number of markers), we sum over the IBD configurations. The term:
\begin{equation}
P(Y|Z_x = z) = \sum_a \sum_{s|z} P(Y|G(s,a);f) P(a;q)	
\end{equation}
where $a$ is the founder alleles and $s$ segregation indicators (which alleles are transmitted). The summation is over all segregation indicators that are consistent with the IBD configruation $z$. 

\item \textbf{Lesson}: the genotype information is essentially equivalent to the ``history'' of the alleles, i.e. which ancestral alleles is inherited. This is technically the IBD status (IBD: whehther two alleles, in pairwise comparison, come from the same ancestor). In contrast to genotypes, IBD follows simpler rules, determined by Mendelian law of segregation and recombination. E.g. in sib-pairs, IBD follows simple distributions, which IBS depends on parental genotypes. 
\end{itemize}

%%%%%%%%%%%%%%%%%%%%%%%%%%%%%%%%%%%%%%%%%%%%%%%%%%%%%%%%%%%%
\section{Non-parametric Linkage Analysis} 

Reference: [Human Molecular Genetics, 3rd Ed, Chapter 15; Yang, Introduction to Statistical Methods in Modern Genetics, Chapter 2, 4]

Motivation: why do we need (or not need) non-parametric methods? 
\begin{itemize}
\item Incomplete penetrance: when penetrance is low and complex (e.g. in complex traits, or age-dependent), the parametric likelihood method may be difficult to apply. The common solution is the Affected Sib Pair (ASP) design. 

\item Low resolution: with parametric linkage methods, when $\theta$ is small, the recombination events will be very rare in a few generations, thus it is not possible to estimate small value of $\theta$. This limits the resolution of linkage methods. The idea is to effectively look at extended families or subpopulations (more recombinations), and look for association between markers and the disease status. 

\item Near-Mendelian families: even for complex traits, it may be possible that within families, the traits are Mendelian (e.g. breast-cancer). Thus parametric linkage analysis within families may identify genes of complex diseases.  
\end{itemize}

Intuitions of non-parametric linkage analysis: 
\begin{itemize}
\item If there is linkage, the markers should be correlated with the disease trait. To reformulate it, if we divide all individuals into two groups, affected and unaffected, then one allele of the marker should be overrepresented in the affected group. A special case is that among affected pairs of relatives, they tend to share the same alleles. 
\end{itemize}

Affected sib pairs (ASPs): 
\begin{itemize}
\item Strategy: suppose the marker is very polymorphic. A mating type is typically $12 \times 34$. Let $S_m$ be the number of alleles shared by two affected siblings, then under $H_0$ of no linkage: 
\begin{equation}
P(S_m = 0) = P(S_m = 2) = 1/4 \qquad P(S_m = 1) = 1/2	
\end{equation}
Let there be $n$ sib pairs with $n_0, n_1, n_2$ pairs sharing 0, 1 and 2 alleles. The departure from the null distribution (the ratio of $1:2:1$) thus indicates linkage. The simple test would be the $\chi^2$ test of departure. But this test is not very specific to $H_A$, so a better test is: the increased number of shared alleles if there is linkage:
\begin{equation}
Z = \frac{n_2 - n_0}{\sqrt{n/2}}	
\end{equation}
It is simple to show that $Z$ follows normal distribution under $H_0$. 

\item More than two affected sibs: the idea is similar, if the marker is linked to the disease locus, the number of shared alleles will be high. Consider the most frequent haplotypes, one from each parent, let $N_i$ be the count of these two haplotypes in the $i$-th family, then the test statistic is $N = \sum_i N_i$. 

\item IBD: when the markers are not very polymorphic, it is better to use haplotypes. The method would then need to recognize IBD shared by affected sibs. 
\end{itemize}

Extensions: 
\begin{itemize}
\item Relative pair methods: for more distant pairs, IBD are harder to determine, so use IBS instead. 

\item ASP for quantitative traits: the idea is similar, pairs with similar traits should have more allele sharing. Let the difference between two siblings of a pair be: $D_i = Y_{i1} - Y_{i2}$, we could regress $D_i$ with $\pi_i$ the expected proportion of alleles shared IBD. More generally, we assume that the traits of a pair follow bivariate normal distribution with mean $\mu \mathbf{1}$, marginal variance $\sigma^2$ and correlation coefficient $\rho$, which depends on the IBD status. 
\end{itemize}

High resolution mapping: 
\begin{itemize}
\item High resolution mapping with extended pedigrees: within an extended family (e.g. with a lot of inbreeding), the disease alleles in all patients generally come from the same ancestral mutation, therefore, all these disease carriers will share the same haplotype blocks. To define these blocks that contain the disease locus: 
\begin{itemize}
	\item Autozygosity mapping: for receissive diseases, a patient carries two identical blocks, thus the marker should be homozygous (in this case, also autozygous as they are identical by descent). Thus search for blocks where the markers are homozygous. 
	\item Shared haplotype blocks: search for such shared blocks in the haplotypes of all patients with the disease. 
\end{itemize}

\item Population association studies/LD mapping: 
\begin{itemize}
	\item Association of markers and diseases: generally speaking, association of disease locus and disease, and LD between marker and disease locus, create non-random association between marker and disease. Specifically, if marker is in tight LD with disease allele, then after a certain number of generations, there will be still a relatively large fraction in the population where the marker and the allele are in the same block, thus creating association with the disease. 
	\item Remark: possilbe causes of association between markers and diseases - natural selection (if an allele helps survival of that disease), population stratification, LD with true disease loci, etc. 
\end{itemize}
\end{itemize}
%%%%%%%%%%%%%%%%%%%%%%%%%%%%%%%%%%%%%%%%%%%%%%%%%%%%%%%%%%%%
\section{QTL mapping} 

Reference: [Lynch \& Walsh, Genetics and Analysis of Quantitative Traits, 1998]

Principles of QTL mapping: 
\begin{itemize}
\item Idea: (1) if we can measure the genotype, then we could test if the association bewteen the genetic variation at any locus and the phenotypic variation; (2) even if we do not have the complete genotype, the genetic markers close to the QTL can be a proxy of the true QTL, if the marker remains linked to the QTL. In QTL mapping, usually starts with crossing two inbred lines, thus creating LD (similar to population admixture).  

\item Experimental procedure: ``create'' genetic variations among different strains by crossing in different ways. In particular, crossing two inbred strains, and then apply backcross or intercross on $F_1$, and collect data on the many progenies. For any locus, the progenies will have one of several genotypes.  
	
\item Why does the procedure work, and how it depends on recombination? If a marker, $M_1$ is close to $L$ (the true QTL), then $M_1$ will be associated with the trait as $M_1$ allele will be coupled with $L$; a distant $M_2$ will not be associated with the trait because of recombination (alleles of $M_2$ are randomized). Therefore, recombination allows one to narrow down the range of $L$: high recombination rates, large samples (thus more recombinations), and more markers will lead to more precise localization of $L$. 
\end{itemize}

Experimental procedures of QTL mapping: crossing of two inbred lines, and $F_1$ is heterozygotes for both marker and QTL, with genotype $MQ/mq$, where $M/m$ is for marker alleles and $Q/q$ for QTL alleles. The gametes of $F_1$ however, could have four genotypes $MQ$, $mq$, $Mq$, and $mQ$ with probabilities dependent on recombination rate $c$. 
\begin{itemize}
\item Intercross ($F_2$ design): crossing $F_1$ with $F_1$. 
\item Backcross: crossing $F_1$ with one of the parental lines. Note that not all genotypes will be examined, e.g. with the parent $MMQQ$, the genotypes of the offsprings will have $MM$ and $Mm$, but not $mm$, so this method is not good for dominance effects. 
\item $F_t$ design: to increase the resolution of QTL mapping (more recombinations), by crossing $F_1$, $F_2$, etc. 
\item Recombinant Inbred Lines (RILs): take $F_1$ line through multiple rounds of selfing or multiple generations of brother-siste mating. The resulting lines have no within-ine genetic variance (each RIL represents a different multi-locus genotype). 
\end{itemize}

Designing experimental procedures [personal notes]
\begin{itemize}
\item In model organism studies, one has the freedom to design experiments to best acheive the goal of mapping trait loci. Considerations of experimental design may include: power, resolution, non-additive genetic model (e.g. dominance), gene-gene and gene-environment interactions, etc. 

\item Power consideration: eg. in $F_1$ intercross, the AF is 50\% and this leads to maximum power. 

\item Minimizing variance: e.g. small variance of trait leads to better power, so we control the environment to minimize environment-introduced variance. Or we choose two $F_0$ strains that are genetically similar, but different in our interest trait; this would minimize variance due to genetic background, when we analyze any particular locus. 

\item Resolution consideration: e.g. $F_t$ design achieves higher resolution. Another example, QTL generally does not tell the causal gene; thus combine QTL with eQTL to find the causal gene (expression trait).  

\item Creating more genetic variations: because of selection, some important loci may not have natural genetic variations. So use artificial selection or other means to introduce more gentic variations, in more genes/loci. 
\end{itemize}

Basic concepts for statistical methods of QTL mapping: 
\begin{itemize}
\item Conditional probabilities of QTL genotypes: we are interested in knowing $P(Q_k|M_j)$, where $Q_k$ is the QTL genotype and $M_j$ is the genotype of the marker. This is written as: 
\begin{equation}
P(Q_k | M_j) = \frac{P(Q_k, M_j)}{P(M_j)}	
\end{equation}
We consider several cases: 
\begin{itemize}
\item Single marker in $F_2$ design: In $F_1$, we have: 
\begin{equation}
P(MQ) = p(mq) = (1-c)/2 \qquad P(Mq) = P(mQ) = c/2	
\end{equation}
In $F_2$, we have the probability of $MM$, $Mm$, $mm$ be $1/4, 1/2, 1/4$ respectively. And the joint probability can be easily computed from multiplying the probabilities of gametes. So we have for example: 
\begin{equation}
P(QQ|MM) = (1-c)^2 \qquad P(Qq|MM) = 2c (1-c) \qquad P(qq|MM) = c^2	
\end{equation}

\item Two markers in $F_2$ design: suppose the distance between $M_1$ and $Q$ is $c_1$, and between $M_2$ and $Q$ is $c_2$. Assume there is no recombination inteferance, then $c_2 = c_{12} - c_1$, where $c_{12}$ is the distance between markers 1 and 2 (known). So there is only one unknown parameter. In $F_1$, we have: 
\begin{equation}
P(M_1 Q M_2) = (1 - c_1) (1 - c_2) / 2 \qquad P(M_1 q M_2) = c_1 c_2 / 2	
\end{equation}
Similarly, we could derive the conditional probabilities: 
\begin{equation}
P(QQ|M_1 M_1 M_2 M_2) = \frac{(1 -c_1)^2 (1 - c_2)^2}{(1 - c_{12})^2}	
\end{equation}

\item Single marker in backcross design: with crossing with $MMQQ$. This can be done similarly: 
\begin{equation}
\begin{array}{lll}
P(QQ|MM) = 1 - c & P(Qq|MM) = c	\\
P(QQ|Mm) = c & P(Qq|Mm) = 1 - c
\end{array}
\end{equation}
\end{itemize}

\item Genetic model: we assume the following parameterization of the means of three genotypes of the QTL: 
\begin{equation}
\mu_{QQ} = \mu + 2 a \qquad \mu_{Qq} = \mu + a ( 1 + k) \qquad \mu_{qq} = \mu	
\end{equation}

\item Marker-class means: the mean of a marker genotype $M_j$ can be expressed as: 
\begin{equation}
\mu_{M_j} = \sum_k \mu_{Q_k} P(Q_k|M_j)	
\end{equation}
where $Q_k$ is the $k$-th genotype of the QTL. For a simple case of single marker (two alleles), and $F_2$ design, we could compute $\mu_{MM}$, $\mu_{Mm}$ and $\mu_{mm}$, and have the following results: 
\begin{equation}
(\mu_{MM} - \mu_{mm}) / 2 = a ( 1 - 2c )
\end{equation}
\begin{equation}
\frac{\mu_{MM} - (\mu_{MM} + \mu_{mm})/2}{(\mu_{MM} - \mu_{mm})/2} = k ( 1- 2c)	
\end{equation}
Thus one strategy is to test for significant differences between mean trait values associated with different marker genotypes. 
\end{itemize}

Methods for single marker analysis: 
\begin{itemize}
\item Linear models: the idea is to test association bewteen marker genotype and the trait with ANOVA or regression: the difference of traits between genotypes carries information of the effect size and the distance. Let $z_{ik}$ be the trait value of the $k$-th individual of marker genotype $i$, we have:
\begin{equation}
z_{ik} = \mu + b_i + e_{ik}	
\end{equation}

\item Maximum likelihood (ML) method: 
\begin{itemize}
\item Mixture model: the distribution of trait values under marker genotypes can be modeled as a mixture of Gaussian distribution:  
\begin{equation}
P(z|M_j) = \sum_k P(z|\mu_{Q_k}, \sigma^2) P (Q_k | M_j)	
\end{equation}
where summation is over all possible genotypes of the QTL. The model has five parameters: marker means for three QTL genotypes, $\sigma^2$ and the recombination ratio $c$. Parameter estimation can be done through EM. 

\item Confidence interval of $c$: this is needed for defining the boundary of the QTL. It can obtained from (1) asympotic distribution of MLE; (2) bootstrapping procedure: sample with replacement. 
\end{itemize}

\end{itemize}

Interval mapping: 
\begin{itemize}
\item ML interval mapping: similar to the single marker mapping, by replacing the term $P(Q_k | M_j)$ with the condition probabilities with two markers. 

\item Approximating ML interval mapping by regression: let $M_i$ be the marker genotype of the $i$-th individual, the expected trait value for $M_i$ is given by the marker-class means equation: 
\begin{equation}
\mu_{M_i} = (\mu + a) P(QQ|M_i) + (\mu + d) P(Qq|M_i) + (\mu - a) P(qq|M_i)	
\end{equation}
where we have the genetic model: $\mu_{QQ} = \mu + a, \mu_{Qq} = \mu + d, \mu_{qq} = \mu - a$. So regression coefficients of the trait values, $z_i$ (which is $\mu_{M_i}$ plus an error term), on the marker genotype $M_i$, will give the relevant parameters, $a$, $d$ and $c$. 
\end{itemize}

Lessons from QTL mapping [personal notes]: 
\begin{itemize}
	\item Generally 100-200 individuals with 20-100 markers are able to detect QTLs. 
	\item Genetic architecture: 222 traits, almost half (45\%) of all traits had a QTL accouting for at least 20\% of the total phenotypic variance ($R^2 > 0.2$). When all detected QTLs are considered, the ratio was raised from 45\% to 84\%. However, it should be noted that when the QTL power is low, the effect may be (severly) estimated. 
	\item Dominance is common, but epistatis is rare. Possible explanations: the methods screen with significant single-locus effects, thus may miss many epistatic effects; the sample size for multi-locus genotype is small, thus reducing the power. 
\end{itemize}

Comments on QTL mapping [personal notes]: 
\begin{itemize}
	\item One major limitation of QTL (and eQTL for mapping GRNs): the crucial genes of the corresponding process may be highly conserved in the population, thus not enough genetic variations.
	\item Comparison of linkage mapping and association studies: in linkage mapping, only a small number of genetic variations are explored, thus not powerful in detecting common variants. Also linkage mapping relies on recombination events within the pedigree, thus more coarse-grained than association mapping, which relies on LD between nearby variants.  
\end{itemize}

Multiple loci in QTL mapping: [Sen \& Churchill, Genetics, 2001]
\begin{itemize}
\item Introduction:
\begin{itemize}
\item Motivation: a signal in QTL mapping may represent multiple linked QTLs, thus need a method to test and separate linked QTLs. Also need to model the effect of epistasis. 

\item Strategy: similar to single marker analysis, the trait values are determined by the QTL alleles, and the QTL alleles are related to the marker genotypes in a manner dependent on the QTL position. 
\end{itemize}

\item Genetic model: often assume additivity among different QTL, i.e. the mean of the normal distribution, $\mu$, of a genotype $g$ is: 
\begin{equation}
\mu_{g_1 \cdots g_p} = \mu + \sum_{i=1}^{p} \Delta_j g_j
\end{equation}
where $g_j$ is the allele at the $j$-th locus.

\item Probabilistic model: let $y$ be the trait value, $m$ be the marker genotype, these are the observed data. The unobserved data are: the QTL genotype, denoted as $g$, and there could be multiple QTLs in the region being tested. The unknowns are the parameters of the genetic model $\mu$ and the QTL locations, denoted as $\gamma$ (specified through recombination rates). The joint distribution: 
\begin{equation}
P(y, m, g, \mu, \gamma) = P(\mu) P(y|g, \mu) P(\gamma) P(m, g|\gamma)	
\end{equation}
where $P(y|g, \mu)$ is specified by the genetic model, and $P(m,g|\gamma)$ is specified according to the recombination distances. 

\item Inference: 
\begin{itemize}
\item Sampling QTL genotypes: 
\begin{equation}
P(g|y,m) \propto P(g|m) P(y|g) 	
\end{equation}
Sampling could be performed with importance sampling: the proposal distribution is $P(g|m)$, and the weighting according to $P(y|g)$. 
\item Sampling QTL locations and effects: $P(\gamma|y,m)$ and $P(\mu|y,m)$. 
\item Number of QTLs: determined via Bayesian model selection. 
\end{itemize}

\end{itemize}

Mapping functional allele series in multiparental populations [Wes Crouse, postdoc interview, 2020]
\begin{itemize}
	\item Motivation: haplotype based test can be more powerful than single marker analysis because: (1) Multiple causal variants: then the haplotype with multiple causal alleles can have a large effect. This is similar to block-level analysis in GWAS. (2) Epistasis: this would further increase the difference between haplotypes. 
	
	\item Idea: if we know which haplotypes have similar effects, we should group them, and do a contrast analysis. However, they are not known. We define haplotypes with the same effects as a functional allele. The goal is to find functional allele series (grouping of haplotypes). 
	
	\item Model: define $M$ as the group structure of haplotypes, then our model is $y = X M \beta + \epsilon$ where $X$ is haplotype and $\beta$ the effects of functional alleles. They key is to define a prior on $M$: assume causal mutations occur during the phylogeny of haplotype, and haplotypes with the same mutations belong to the same group. Let $T$ be the local phylogenetic tree (may vary across loci because of recombination), $L$ be the branch length and $\pi$ be when causal mutation occurs. Use a coalescence tree prior for $T$, and causal mutations follow Poisson process, then the prior of $M$ marginalize $T$ and $L$ and $\pi$. Show that $M$ follows Chinese Restaurant Process. 
	
	\item Note: the prior depends quite strongly on the rate of causal mutations $\alpha$. This parameter is chosen to have mostly 0 or 1 mutations. 
	
	\item Part II. mediation analysis in QTL. Find mediators of trans-expression QTL. Motivation: current mediation analysis requires several tests, which is not statistically satisfactory. Also Sorbet test cannot distinguish partial and complete mediation. 
	
	\item Model selection approach: define several models, including complete mediation, partial mediation, independent effects. Use a uniform prior for all models. 
	
	\item Results: some example, eg. trans-pQTL of a gene, mediated by cis-gene. 
	
	\item Remark: not account for reverse causality. Confounders may be OK: unless they are heritable. 
\end{itemize} 
%%%%%%%%%%%%%%%%%%%%%%%%%%%%%%%%%%%%%%%%%%%%%%%%%%%%%%%%%%%%
%%%%%%%%%%%%%%%%%%%%%%%%%%%%%%%%%%%%%%%%%%%%%%%%%%%%%%%%%%%%
\chapter{Association Mapping}
\section{Introduction to Association Mapping}

Reference: [A tutorial on statistical methods for population association studies, Balding, NRG, 2006; Genome-wide association studies for complex traits: consensus, uncertainty and challenges, McCarthy \& Hirschhorn, NRG, 2008; Handbook of Statistical Genetics, Chapter 36,37]

Concepts of GWAS analysis: 
\begin{itemize}
\item Indirect association between markers and traits: association between causal variants and genetic markers through LD and association between causal variants and traits.  

\item Population genetic perspective: in the association studies, the individuals are not really unrelated. In fact, they share a small number of common ancestors, and disease-causing mutations may descend from a small number of ancestral mutations. 

\item CDCV hypotheiss: Only common variants are genotyped in association studies (e.g. above 5\%). The fundamental assumption of GWAS is that the disease traits are influenced by common variants (Common Disease-Common Variants, CDCV). There is no statistical power to detect rare variants in GWAS (e.g. if only 1\%, then in a study involving 1,000 people, only 10 people will have the rare allele, not enough to detect association). 

\item Linkage vs association studies: two main problems with linkage studies (limited to families) for complex diseases: (1) not concentrated within families; (2) the causal variants are not shared among affected family members. Thus for complex disease, association studies are more powerful. 
\end{itemize}

Justification of CDCV: if some ancestral mutation has small effect on fitness, then it may accumulate in the population and reach relatively high frequency. The possible causes of common varirants: 
\begin{itemize}
	\item Many mutations have small effects on fitness due to the robustness of the system, the uncertainty of environment (time-varying selection), etc.
	\item Human population size is small: thus selection is weak. 
	\item Selection against many late-onset diseases may be small. 
\end{itemize}

Design of association studies: 
\begin{itemize}
	\item Design: family-based association studies: transmission-disequilibrium test (TDT), and population-based case control: GWAS. Multi-stage replication design (at each stage, narrow down candidate SNPs and increase sample size). 
	\item Control selection: should minimize the possible bias in case vs control group. 
	\item Tag SNP selection: because of LD, only a subset of SNPs need to be genotyped (50-70\% reduction). How tag SNPs are chosen depends on the LD structure. E.g. use $r^2 = 0.8$ as the cutoff for redundent SNPs that are adjacent. 
	\item Sample size: if $n$ is the sample needed for causal variants, then $n/r^2$ for indirect association where $r^2$ is between the marker and causal variant. 
	\item Prospective and retrospective design: prospective design - inidividuals are followed forward in time and disease events are recorded. Most studies are retrospective design.  
\end{itemize}

Possible biases of case-control studies: 
\begin{itemize}
	\item Selection bias: bias of sampling of cases and controls. E.g. cases are obtained nationally while controls are obtained locally. 
	\item Information bias: the measurement errors could be different in cases and controls. 
	\item Stratification or admixture: potentially confounding variables including ethnic groups/population structure and environmental variables. 
\end{itemize}

Preliminary analysis: 
\begin{itemize}
	\item HWE: often used for data control (discase SNPs not in HWE), but may also indiate disease association. Test of deviation: Pearson's $\chi^2$ test.
	\item Missing genotype data and haplotype reconstruction: possible if LD is strong. 
	\item LD and recombination rate: LD between two loci is commonly measured by $D'$ or $r^2$. A disadvantage of $D'$ is: sensitive to rare alleles. $r^2$ gives the sample size needed to detect the disease association (relative to the sample size required if directly type causal SNP). Summarizing LD in a region requires estimation of recombination rates. 
	\item Tag SNPs: could reduced the SNP set in the analysis, thus reduce d.f. of a test. 
\end{itemize}

Diagnostic plots:
\begin{itemize}
	\item Quantile-quantile (Q-Q) plot: comparison of the null distribution of the test statistic (e.g. $\chi^2$) with the observed distribution. Population stratification will result in deviation from the null across the entire distribution, whereas large-effect disease loci generate deviation at the highly significant end of the range. 
	\item Genome-wide Manhattan plot: test statistic or $P$ value across the genome positions. 
\end{itemize}

Test of association: single SNP
\begin{itemize}
\item Two d.f. test: Pearson $\chi^2$ test or Fisher's exact test (preferred with small sample size/rare alleles).  
\item One d.f. test: assume the risk is additive. Armitage test: test if the ratios of case/control are the same across all three genotypes (test slope of the line). Armitage test sacrifices power if the genotypic risks are far from additive in order to obtain better power for near-additive risks. 
\item Linear/logistic regression: the linear predictor is equal to $\beta_0$, $\beta_1$ or $\beta_2$. LRT against the null hypothesis $\beta_0 = \beta_1 = \beta_2$ has 2 d.f. (equivalent to Pearson 2-df test with large sample size). If assume the coefficients are linear, i.e. $\beta_1 = (\beta_0+\beta_2)/2$, a 1-df test that is equivalent to Armitage test. 
\end{itemize}

Test of association: multiple SNPs. This can be used to test, e.g. whether a gene is associated with the disease. 
\begin{itemize}
\item SNP-based logistic regression: suppose there are $L$ SNPs, treat each of them as a predictor and do regression. LRT: the null hypotheis requires for every SNP, $\beta_0 = \beta_1 = \beta_2$ (test with additivity assumption is similar). Step-wise selection or shrinkage method or tagging SNPs (feature selection) can be applied to deal with redundant SNPs. 
	
\item Haplotype-based method: avoid the problem of having too many predictors. 
\begin{itemize}
	\item Basic methods: test of 2 by $k$ contingency table ($k$ is the number of haplotypes); comparison of the proportion of case and control in $k$ haplotypes; regression anallysis with haplotypes treated as categorical variables. 
	\item Haplotype clustering: impose a structure on haplotype space to exploit possible evolutionary relationship among haplotypes. Ex. cluster haplotypes that are assumed to share a common ancestry and therefore convey a common disease risk. 
\end{itemize}
\end{itemize}

Multiple testing: 
\begin{itemize}
	\item Permutation procedure: use simulation to estimate the FP rate. 
	\item FDR: the expected number of false positives among significant associations. Treat actual distribution of $P$ values as a mixture of distributions under null (uniform) and alternative (skewed towards zero) hypothesis. Let $N$ be the number of SNPs, $\alpha$ be the significance level (uncorrected), and $k$ be the number of SNPs with $p$ value less than $\alpha$, then: $FDR = N \alpha / k$. 
	\item Bayesian: traditional approach discourages additional tests, because all tests will suffer from the multiple-testing penalty. Bayesian approach may have advantages, as the prior probability of association should not be affected by what tests are chosen to carry out. 
\end{itemize}

Epistatis: 
\begin{itemize}
	\item Test SNP pairs:  in the logistic regression framewok, allow interaction terms of two SNPs (four terms corresponding to additive/dominance contributions to epistasis). 
	\item Two-stage approach: single SNP test, then choose SNPs above a certain threshold for pair testing. 
	\item Bayesian model averaging. 
\end{itemize}

Strategies for dealing with confounding:  
\begin{itemize}
	\item Stratification: of the confounding variaible and test for association within each strata. To test for association: (1) assume the association parameter is constant across strata and allow the disease distribution varies across strata, and do contingency table test. (2) Logistic regression: allow disease status to depend on the stratum (additional categorical variable). 
	\item Group or individually matched study: may help stratification problem.  
\end{itemize}

Population stratification: 
\begin{itemize}
	\item Genomic control: the idea is that the scores of the SNPs are inflated, so make a correction by dividing a factor $\lambda$. The value of $\lambda$ is estimated by: compute the test statistic (e.g. Armitage test) of null SNPs, and compare the distribution with the expected null distribution. Limitation: only applicabile to the simplest single-SNP test. 
	\item Structured association: (1) infer the population structure - $K$ specific ancestral strata inferred using the STRUCTURE algorithm (MCMC); (2) test for associations conditional on subpopulations. However, the correct number of subpopulations is generally unknown. 
	\item Regression: incorporate null SNPs (or known race/ethnicit) as covariates in regression (these null SNPs will capture the effect of stratification: e.g. one null SNP may be a marker of some subpopulation). Could also use population structure variable (e.g. PCs from PCA) as covariates. This has the benefit of incorporating the effect of different environmental or demographic factors: such as diet. Thus in this sense, the marker SNPs are simply surrogates of these (unmeasured) factors. 
	\item Hierarchical model: estimate kinship, with or without subpopulation effect. The population structure can be diagnosed by PCA. 
\end{itemize}

Replication: 
\begin{itemize}
	\item Evaluate the selected signals with additional independent samples, under the same allele, the same phenotype and genetic model. Replication should be separate from fine-mapping: avoid using additional SNPs (e.g. those adjacent to the selected ones). However, in practices, investigators often combine the two studies. 
	\item In general, replication experiments only genotype the SNPs that are significant (or close) in the first (scan) phase of the study. Ex. see [Barrett et al., Crohn's disease GWAS, NG, 2008]
	\item Heterogeneity: not all signals can be replicated because of heterogeneity. The possible causes include: the difference in LD, in the allele distribution, non-additive interactions with other genetic variants or environmental exposures, etc. Ex. FTO gene is found to be associated with type 2 diabetes in one study (through association with weight), but not in a replicate study (which controls the weight of samples). 
\end{itemize}

Follow up analysis: 
\begin{itemize}
	\item Candidate genes or regulatory elements: the intervals containing putative causal SNPs are defined in terms of the flanking recombination hot spots. Usually the intervals contain multiple genes, or distant from any known genes. 
	\item Biological knowledge: e.g. expression of a gene is associated with the disease. 
	\item Resequencing and fine mapping: detect the putatively causal variants. 
	\item Application of GWAS results: (1) novel biological insights: which may suggest new biomarkers or therapeutic targets; (2) personalized medicine: which may improve diagnostics/prognostics and treatment (however, the predictive power, even all SNPs combined, is still very low). 
\end{itemize}

Remark: 
\begin{itemize}
	\item Question: the environmental influences in GWAS (similar to population stratification)? E.g. if a disease is strongly associated with a certain diet, and the case and control groups are imbalanced with this diet, then there will be false associations with the SNPs that tend to occur in the people with one type of diet. 
	\item The possible answers: (1) if the environmental influence is known, then model with multi-variate regression, and explain the trait using the environmental variable. (2) Otherwise, depending on whether environmental variable is associated with genetic backgrounds, if there is no association, then there wouln't be many false associations; if there is association, most interestingly, people with one genetic background tend to have similar life style/be subject to the same environmental influence, and tend to have certain diseases or not, then there will be false associations. A population stratification problem. 
\end{itemize}

Genetic mapping in human disease [Altshuler \& Lander,  Science, 2008]:
\begin{itemize}
\item Principles of GWAS: 
\begin{itemize}
	\item Common disease/common variants assumption (CD/CV): common diseases are caused by genetic variations at common variants (normally defined as allele frequencies $> 1\%$). The alternative hypothesis is: common diseases/rare variants (CD/RV), where common diseases are often caused by variantions at rare variants. The CD/CV assumption is the foundation of GWAS, as generally GWAS only measure genotypes at common variant SNPs. 
	\item Human haplotype structure (Figure 1): the heterozygosity rate is about 1 in 1000 bases, and about 90\% are common variants. The haplotype of SNPs are often highly correlated within regions with low recombination, forming the haplotype blocks. This reflects the fact the evolutionary history of human populations, the recent genetic variations in closely linked loci have not been shuffled by recombination. Therefore, a high level of LD within human genome (this allows ous to genotype only one SNP within a haplotype block). Particulary, the LD between markers (SNPs or structural variants, such as copy number variations, or CNVs) and disease loci means that the information of markers reveal information of the disease loci. 
	\item Analysis of GWAS: since the physical location information is not quantitatively known (when the founder mutation occurs, etc.), methods such as interval mapping cannot be applied. Instead, testing association between each SNP and the trait. 
\end{itemize}

\item Statistical power of GWAS: 
\begin{itemize}
	\item Problem of population structure: if the case group and the control group have different population structure, e.g. the case group is enriched with African people, then any SNPs typical of the African population will show significant associations. The idea is that: population structure is revealed by the patterns across a large number of SNPs. 
	\item Because of haplotype structure (the strong LDs among nearby SNPs), only 500,000 SNPs (out of milliions of SNPs) provide excellent power to test $>90\%$ of common SNP variations. 
	\item Sample size requirement: achieving 90\% power to detect an allele with 20\% frequency and a factor of 1.2 (increase the disease risk by 20\%) at a statistical signifance of $10^{-8}$ requires 8,600 samples. The sample size of the current GWAS is usually smaller (thousands). 
	\item Pooling data could yield higher power. 
\end{itemize}

\item Biological lessons from GWAS: 
\begin{itemize}
	\item In the vast majority of cases, the estimated effects are small, a factor of 1.1 to 1.5 per associated allele. And the total variants explain only a small fraction of inherited risk of disease, e.g. 10\% for Crohn's disease. 
	\item Many associations implicate non-protein coding regions, e.g. the regions at 8q24 associated with cancer, 300kb from the nearest gene. 
	\item Often multiple disease are associated with the same regions. 
\end{itemize}

\item Future directions: 
\begin{itemize}
	\item Increase the statistic power by: increasing sample size; the structural variants. The goal is to identify rare variants (this may be particularly important for disease). 
	\item Identify causal mutations and disease mechanism: GWAS typically yield regions of 10 to 100 kb. E.g. through resequencing or fine-mapping. And further studies of the disease mechanism (creating disease model in human cells or non-human animals). 
	\item Gene-gene interactions ang gene-environment interactions. 
	\item Disease-related intermediate traits: e.g. gene expression. 
	\item Genome sequencing of a large sample. 
\end{itemize}
\end{itemize}

Statistical Challenges in GWAS [Cantor \& Sinsheimer, AJHG, 2010; Moore \& Williams, Bioinfo, 2010]:
\begin{itemize}
\item Meta-analysis: 
\begin{itemize}
	\item Types of meta-analysis: (1) combine studies in two diverse populations; (2) combine studies that differ in covariates or in measure of traits (e.g. binary and continuous). 
	\item Considerations of meta-analysis/heterogenity: ascertainment of the sample (including ethnic stratification), the definition/measure of the trait, the statistics that summarize the association results. 
	\item Methods: 
\begin{itemize}
	\item SNP imputation
	\item Traditional approach: combine $p$ values, or other test statistics. Ned weighting to reduce the effect of heterogenity: fixed-effect model (effect size is constant) and random-effect model (effect size also varies). 
	\item Bayesian hierarchical models: parameters as random variables sampled from prior distributions; also incorporate the LD data and other evidence (biological functional information) as prior information.  
\end{itemize}
\end{itemize}

\item Epistatis: 
\begin{itemize}
	\item Difficulty: the large number of possible interactions. May need to restrict the search: e.g. only those with significant marginal effects. Involve a trade-off between miniminizing computation and the number of tests, and maximizing power. 
	\item Parametric methods: test departure from the additive model (between two loci). Penalized regression is one powerful way of selecting a small number of significant interactions. 
	\item Non-parametric methods: these methods directly or indirectly model interactions, including decision tree, multifactor dimensionality reduction (MDR), combinatorial partitioning, entropy/conditional entropy measures, logic regression, Bayesian partitioning, etc. 
\begin{itemize}
	\item Random forest: the standard implementation is conditional on marginal effects. The interpretation is not straightforward as the interactions are hidden in the tree. 
	\item MDR: designed to detect interaction even in the absence of strong marginal effect. The idea is to construct features by pooling genotypes from mulitple SNPs, and these features will make it easier for methods such as logistic regression to detect attribute dependencies. 
	\item Relief, RelieF, TuRF: attribute selection method. Assess the weight/quality of an attribute based on whether the nearest neighbor of a randomly selected instance from the same class and the nearest neighbor from the other class have the same or different values. 
\end{itemize}
\end{itemize}

\item GWAS pathway analysis (GWASPA): 
\begin{itemize}
	\item Motivation: genetic heterogeneity of complex traits - mutations of the same gene, or different genes of the same pathway, can lead to the same disease (especially in different ethnic groups). Thus GWASPA can increase power by combining these evidences. 
	\item Statistical considerations for GWASPA: sources of biases may include: SNP density in genes; gene size; pathway size. If not correcting for these factors, there will be bias toward finding the pathways that are large, well-known, with more genes that are densely covered by SNPs. 
	\item Methods/issues of GWASPA: 
\begin{itemize}
	\item SNP assignment: the most frequent approach is to select a SNP with the strongest association signal for a gene. Not optimal because still bias with large genes, and multiple association signals within a gene may be lost. 
	\item Score pathways: most method aggregate $P$ values of genes in the pathway in some way such as GSEA, $k$ most significant genes, etc. 
	\item Permutation test: the score is often assessed by permutation test, which may also adjust for many sources of bias. 
\end{itemize}
\item Challenge: SNP assignment - consider multiple independent association signals of a gene; correction of testing multiple pathways; epistasis in genes of a pathway; $P$-value based test (because genotype data are not always available); capture both rare and common variants. 
\end{itemize}
\end{itemize}
%%%%%%%%%%%%%%%%%%%%%%%%%%%%%%%%%%%%%%%%%%%%%%%%%%%%%%%%%%%%
\section{Statistical Analysis of Association: Background} 

Reference: [Laird \& Lange, The fundamentals of Modern Statistical Genetics, Chapter 7; Handbook of Statistical Genetics, Chapter 36,37]

Linkage and association: [Thomas, Chapter 9]
\begin{itemize}
\item Linkage: between disease locus and marker (co-segregation in families). The consequence of linkage (and co-segregation) is that the affected members in families tend to share markers. 

\item Association: LD between disease locus and marker. 

\item Comparison: for linkage analysis, we are testing $H_0: \theta = \frac{1}{2}$, where $\theta$ is the fraction of recombination, and for association analysis, we are testing $H_0: D' = 0$, where $D'$ is the measure of LD. 
\end{itemize}

Disease risk and genotype frequenices: 
\begin{itemize}
	\item Disease penetrance: for a genotype, the probability that the individual carrying this genotype is disease, denoted as $f_i = P(D|i)$, where $i = 0,1,2$ is the number of $A$ alleles (suppose $A$ is the disease allele and $a$ is the normal allele), and $D$ and $U$ denote disease or not. 
	\item Genotype frequency: the probability of a genotype in the population (including both cases and controls), denoted as $g_i$. If HWE holds, then $g_i$ can be calculated from allele frequencies. 
	\item Relative risk: measure the strength of association. The increase of disease probability relative to some reference genotype, denoted as: $\gamma_i = f_i / f_0$. 
	\item Odds and odds ratio: The odds of a genotype is defined as: 
\begin{equation}
\text{odds}(i) = \frac{f_i}{1-f_i}	
\end{equation}
The odds ratio is defined as the odds relative to some reference genotype (similar to relative risk), e.g.: 
\begin{equation}
		OR(i) = \frac{f_i}{1-f_i} / \frac{f_0}{1-f_0}
\end{equation}
For most diseases that are rare in population, we have $1 - f_i \approx 1$, thus $OR(i) \approx \gamma_i$, i.e. odds-ratio is approximately equal to relative risk. 
	\item Genotype frequencies in case and control: let $p_i = P(G=i|D)$ be the genotype distribution of case, and $q_i = P(G=i|U)$ be that of the control. We have the following relations using Bayes's Theorem: 
\begin{equation}
p_i = P(i|D) = \frac{P(D|i) P(i)}{P(D)} = \frac{f_i g_i}{\sum_i f_i g_i}
\label{eq:genotype_freq}
\end{equation}
\begin{equation}
q_i = P(i|U) = \frac{P(U|i) P(i)}{P(U)} = \frac{(1-f_i) g_i}{\sum_i (1-f_i) g_i}
\end{equation}
One could define $K = \sum_i f_i g_i$, which is the disease prevalence in the whole population. A consequence of these relations is that the relative genotype frequency of case/control is proportional to the disease odds. 
	\item Hardy-Weinberg equilibrium (HWE): in general not hold for loci associated with diseases. Thus deviation from HWE in cases of diseases is often taken as preliminary evidence of association. 
\end{itemize}

Genetic models: 
\begin{itemize}
\item Multiplicative penetrance model: the relative risk is multiplicative of the number of copies of the allele $A$, i.e. $f_i = \gamma_i f_0$ and $\gamma_2 = \gamma_1^2$. Under this model, association with disease does not lead to deviation from HWE. 
\end{itemize}

Testing for association at a single locus: two basic approaches based on different intuitions: 
\begin{itemize}
\item Disease risk perpsective: genetic variant increases disease risk. This corresponds to discriminant models in machine learning ($X \rightarrow Y$). 
\begin{itemize}
	\item Hypothesis testing based on genotype partitioning: the case/control ratios are different across genotypes. Armitage trend test: test if the ratios are the same across three genotypes (or whether slope is zero vs slope not equal to 1 in regression). It has df equal to 1 (effectively assuming additive risk in $H_A$). 
	\item Likelihood method: model the probability distribution of disease conditional on genotypes. Logistic regression: under the null hypothesis, the coefficient of the genotype variant is zero. 
	\item Bayesian: priors on genetic model (additive, dominant, etc.) and effect size. 
\end{itemize}

\item Genotype distribution perspective: disease group has a different genotype distribution than the case group. This corresponds to generative modesl in machine learning ($Y \rightarrow X$). 
\begin{itemize}
	\item Hypothesis testing based on genotype distribution: the genotype distributions are different in case and control groups - $\chi^2$ or Fisher's exact in contingency table. 
	\item Likelihood method: model the probability distribution of genotypes conditional on case/control group. Multinomial distribution: the coefficients of the genotypes under case are different from those under control. 
	\item Bayesian: priors on the genotype distribution. 
\end{itemize}

\item Analogy: this is similar to text classification problem. The main statistical challenge of much more parameters than observations is essentially the same. And the different methods here also parallle those in text classification: discriminative models such as SVM where response is a function of predictors; and generative models such as Naive Bayes. 
\end{itemize}

Allele test for additive model: [LL, Section 7.2]
\begin{itemize}
\item Problem: Consider a 2 by 3 table and let $r_i$ be the number of cases, and $s_i$ be the number of controls, $i = 0,1,2$. And the total number of cases is $R$, the total number of controls is $S$, and the number of genotype $i$ is $n_i$. And the total number is $N = R + S$. The null hypothesis is that the number of $A$ alleles is equal in case and in control. 

\item Idea: in the case group, the number of $A$ alleles follows a binomial distribution with sample size $R$ and in the control group, it follows an independent binomial distribution with sample size $S$. The problem is thus the test whether the probability parameters of two binomial distributions are identical. 

\item Test: let $p_{\text{case}}$ and $p_{\text{control}}$ be the $A$ frequency in the case and control group respectively, we are testing $H_0: p_{\text{case}} = p_{\text{control}}$. Let $\bar{p}_{\text{case}}$ and $\bar{p}_{\text{control}}$ be the average $A$ frequency in case and control, and our test would be based on $T = \bar{p}_{\text{case}} - \bar{p}_{\text{control}}$. Under $H_0$, $E(T|H_0) = 0$, and under the HWE assumption (each copy of allele is independent), the variance of $\bar{p}_{\text{case}}$ and $\bar{p}_{\text{control}}$ is the variance at each copy of allele divided by sample size. So:
\begin{equation}
\text{Var}(T) = \bar{p} ( 1 - \bar{p} ) \left( \frac{1}{2R} + \frac{1}{2S}\right)
\end{equation}
where $\bar{p} = (2 n_2 + n_1 ) / N$ is the average frequency of $A$. Thus we have the test: $Z = T/\sqrt{\text{Var}(T)}$ following standard normal distribution, or $X^2 = T^2/\text{Var}(T)$ following $\chi^2$ with df equal to 1. 

\end{itemize}

Armitage trend test: [Armitage test derivation, statgen.org; LL, section 7.2]
\begin{itemize}
\item The null hypothesis, the trends of cell counts in the case and in the control are the same. Or more precisely, if we assume $\bar{X}_{\text{case}}$ and $\bar{X}_{\text{control}}$ be the average number of $A$ alleles per individual, then under $H_0$, the two should be equal. 

\item Idea: the numbers of three genotypes in the case group follow a multinomial distribution $\text{Mul}(R;p_0, p_1, p_2)$; and similarly, they follow an independent distribution $\text{Mul}(S;q_0, q_1, q_2)$. We are thus testing if $p_i = q_i, i = 0,1,2$. 

\item Test: we first standarize the table s.t. the number of cases and controls are equal ($RS$). Our test statistic measures the difference of the number of $A$ alleles in cases and in controls (weighted by the sample size of case and control): 
\begin{equation}
U = \sum_{i=0}^2 x_i (S r_i - R s_i)	
\end{equation}
Under $x_i = i, i = 0,1,2$, the test statistic becomes: 
\begin{equation}
U = N(r_1 + 2 r_2) - R (n_1 + 2 n_2)	
\end{equation}
Clearly, under $H_0$: $E(U) = 0$. The variance of $U$ can be computed by: (1) only $r_i$ and $s_i$ are variables and the expression contains only their variance and covariance; (2) the variance and covirance of $r_i$ and $s_i$ under $H_0$ can be computed using properties of multinomial distribution: $(r_0,r_1,r_2)|H_0 \sim \text{Mul}(R; n_0/N, n_1/N, n_2/N)$ and $(s_0,s_1,s_2)|H_0 \sim \text{Mul}(S; n_0/N, n_1/N, n_2/N)$. 
\begin{equation}
\text{Var}(U) = \frac{(N-R)R}{N} \left[ N(n_1 + 4 n_2) - (n_1 + 2 n_2)^2 \right]
\end{equation}
At large $N$, we form the $\chi^2$ test statistic (or normal): 
\begin{equation}
X^2 = \frac{N\left[ N(r_1+2r_2) - R(n_1+2n_2)\right]^2}{(N-R)R\left[ N(n_1 + 4n_2) - (n_1 + 2n_2)^2 \right]} \sim \chi^2_1	
\end{equation}

\item An alternative way of deriving the test [LL, section 7.2]: we have $\bar{X}_{\text{case}} = (2r_2 + r_1)/R$ and $\bar{X}_{\text{control}} = (2 s_2 + s_1)/S$. Under $H_0$, the difference of the two is 0, and the variance can be computed as: 
\begin{equation}
\text{Var}(\bar{X}_{\text{case}} - \bar{X}_{\text{control}}) = \text{Var}(X) \left( \frac{1}{R} + \frac{1}{S}\right)
\end{equation}
The variance of $X$ (the number of $A$ allele) under $H_0$ can be easily calculated using that $X \sim \text{Mul}(N; n_0/N, n_1/N, n_2/N)$. This leads to the same test. 
\end{itemize}

Estimation of effect size: [LL, section 7.5]
\begin{itemize}
\item Notation: Suppose we are given a genotype and a reference genotype, denoted as $E$ (exposed) and $U$ (unexposed) respectively. The numbers of cases in $E$ and $U$ groups are $a$ and $b$ respectively, the numbers of controls in $E$ and $U$ are $c$ and $d$ respectively. 

\item The estimated odds ratio is given by: 
\begin{equation}
\text{OR} = \frac{a/c}{b/d} = \frac{ad}{bc}	
\end{equation}
The variance of the $\log \text{OR}$ is (see the section ``Estimating odds-ratio in a 2 by 2 table'' in Statistics Notes): 
\begin{equation}
\text{Var}(\log \text{OR}) \approx  \frac{1}{a} + \frac{1}{b} + \frac{1}{c} + \frac{1}{d} 	
\end{equation}

\item Remark: it is possible to test association using effect size (check if 0 is in the confidence interval) - this test is not equivalent to the trend test (which is based on the difference of risk, not ratio); in practice, usually do association test first, and if significant, calculate the effect size. 
\end{itemize}

Logistic regression:
\begin{itemize}
\item Model: let $G$ be genotype, $y$ be phenotype, and $\pi_i$ be the probability of disease given $G_i$: 
\begin{equation}
\pi_i = \frac{\exp(\eta_i)}{1 + \exp(\eta_i)}
\end{equation}
where $\eta_i$ is the linear predictor dependent on $G_i$. The log-likelihood is then given by: 
\begin{equation}
\ln f(\mathbf{y}|\mathbf{G},\mathbf{\beta}) = \sum_i \left[ y_i \ln \pi_i + (1 - y_i) \ln (1 - \pi_i) \right]	
\end{equation}
Typically, the linear predictor encodes additive and dominance effect of two alleles: 
\begin{equation}
\eta_i = \beta_0 + \beta_A z_{(A)i} + \beta_D z_{(D)i}	
\end{equation}
where $\beta_A$ denotes the additive effect of the allele $m$ (minor allele), $\beta_D$ the dominance effect, $z_{(A)i}$ is an indicator variable representing the additive component of $i$-th genotype (the number of copies of the minor allele), and $z_{(D)i}$ the dominance component of $G_i$ (only present for heterozygous genotype). Note: the model can be easily extended to incorporate other (e.g. env.) variables simply by adding these in the linear predictor.  

\item Significance: $\beta$ in the logistic regression model is the log-OR, and the significance of $\beta$ can be tested using $Z$ test or LRT (see section of Logistic Regression). 

\item Genetic models: 
\begin{itemize}
	\item Multiplicative (additive) model: the disease odds are multiplicative for the $a$ allele, thus we have: $\text{odds}(Aa) = \alpha (1 + \theta)$ and $\text{odds}(aa) = \alpha (1 + \theta)^2$. 
	\item Dominant model: $\text{odds}(AA) = \alpha$, $\text{odds}(Aa) = \text{odds}(aa)= \alpha (1 + \theta)$. 
	\item Recessive model: $\text{odds}(AA) = \text{odds}(Aa) = \alpha$, $\text{odds}(aa)= \alpha (1 + \theta)$.
\end{itemize}
\end{itemize}

Background for Summary Statistics [personal notes]:
\begin{itemize}
	\item Relation of effect size, $Z$-score and PVE under simple regression: $y = x \beta + \epsilon$, let $N$ be sample size, $\sigma_x, \sigma_y$ be the variance of $x$ and $y$. The variance of $\hat{\beta}$ is given by: 
	\begin{equation}
	s^2 = \frac{\sigma_y^2}{N \sigma_x^2} \Rightarrow s = \frac{\sigma_y}{ \sqrt{N} \sigma_x}
	\end{equation}
	The PVE of $x$ is given by:
	\begin{equation}
	\text{PVE} = \frac{\hat{\beta}^2 \sigma_x^2}{\sigma_y^2} \Rightarrow \sqrt{\text{PVE}} = \frac{\hat{\beta} \sigma_x}{\sigma_y}
	\end{equation}
	When both $x$ and $Y$ are normalized, then PVE is simply the square of $\hat{\beta}$. The Z-score of $\hat{\beta}$ is given by:
	\begin{equation}
	Z = \frac{\hat{\beta}}{s} = \frac{\sqrt{N} \hat{\beta} \sigma_x}{\sigma_y} \Rightarrow Z = \sqrt{N} \cdot \sqrt{\text{PVE}}
	\end{equation}
	Thus $Z$ score is simply $\sqrt{N}$ times the square root of PVE - this is true regardless of when $x$ and $y$ are normalized.
	
	\item $Z$-score is normalized: it is easy to check that $\Var(Z) = 1$. 
\end{itemize}

\subsection{Power of Association Studies}

Power analysis: [Neale, Statistical Genetics, Chapter 21]
\begin{itemize}
\item Armitage trend test [Slager \& Schaid, Hum Hered, 2001]: suppose $p_i$ and $q_i$ are genotype frequencies of cases and controls respectively. We know that under $H_A$, $r_0, r_1, r_2 \sim \text{Mul}(R; p_0, p_1, p_2)$ and $s_0, s_1, s_2 \sim \text{Mul}(S; q_0, q_1, q_2)$. Our statistic is: 
\begin{equation}
U = \sum_i x_i \left( \frac{S}{N}r_i - \frac{R}{N}s_i \right)	
\end{equation}
$U$ follows normal distribution asymptotically. Taking expectation of $U$ (using the expectation of $r_i$ and $s_i$ from multinomial distribution): 
\begin{equation}
E(U) = \mu_1 = N \phi (1 - \phi) \sum_i x_i (p_i - q_i)
\end{equation}
where $\phi = R/N$. The variance of $U$ (again using the properties of multinomial distribution): 
\begin{equation}
\text{Var}(U) = \sigma_1^2 = N \phi (1-\phi)^2 \left[ \sum_i x_i^2p_i - (\sum_i x_i p_i)^2 \right] + N \phi^2 (1-\phi) \left[ \sum_i x_i^2q_i - (\sum_i x_i q_i)^2 \right]
\end{equation}
The terms $p_i$ and $q_i$ can be related to the genetic parameters, according to Equation~\ref{eq:genotype_freq}. 

\end{itemize}

Calculation of power in case-control studies: 
\begin{itemize}
\item Distribution of test statistic: to calculate the power, need to compute the distribution of test statistic, $\chi^2$ in case-control studies, under $H_A$. This is equivalent to finding the distribution of the counts of two alleles in cases and controls respectively. The distribution of the $\chi^2$ statistic can be analytically computed under the simple additive model with only a single causal SNP. When the numbers of cases and controls are equal, this is given by [Spencer \& Marchini, PG, 2009]: 
\begin{equation}
E(\chi^2) \propto N \gamma^2 p (1-p) r^2	
\end{equation}
where $N$ is the number of cases and controls, $\gamma$ the effect size, $p$ the allele frequency of the risk variant and $r^2$ the correlation between the marker and the causal SNP. 
\end{itemize}

Factors affecting the power of case-control studies: power is assessed by the sample size required for a given significance and power level. See [Neale, Table 21.1-21.3]
\begin{itemize}
\item Genetic model, risk allele frequencies and genotype relative risk: these factors affect the genotype frequencies $P(G|D)$. Rare alleles require considerably more samples. The recessive models typically require many more samples. 
\item TDT vs. case-control: about the same number of TDT trios and case-control pairs. So the sample size requirement of case-control is $2/3$ of TDT. The main advantage of TDT is the control of population stratification. 
\item $D'$ and marker allele frequency: in the calculation of power, replace the genotype frequency of disease alleles with those of marker allele frequencies (by summing over the disease alleles, whose probabilities are related to marker allele frequencies through LD). Large $D'$ increases the sample size requirement; marker allele frequencies are optimal when they match the causal allele frequencies. 
\item Quantitative traits: discretization leads to a substantial loss of power.  
\end{itemize}

Simulating genotypes for power analysis: HAPGEN [Spencer \& Marchini, PG, 2009]: 
\begin{itemize}
\item Generative model of LD: suppose we want to model $n$ haplotypes $h_1, \cdots, h_n$, where the haplotype is size $S$. The idea is that given $k$ haplotypes, the next one can be viewed as a mosaic of existing haplotypes. We have: 
\begin{equation}
P(h_1, \cdots, h_n|\rho) = P(h_1|\rho) P(h_2|h_1,\rho) \cdots P(h_n|h_1, \cdots, h_{n-1}, \rho)	
\end{equation}
where $\rho$ represents the recombination parameters of the region. When we have $k$ haplotypes, in generating $h_{k+1}$: we have a hidden variable $X_j$ for the $j$-th position, indicating where it comes from, $X_j \in \{1, 2, \cdots, k\}$. $X_j$ is modeled as a HMM, where the transition probability depends on $\rho$. Also, the mutation process is modeled in the term $P(h_{k+1}|h_1, \cdots, h_k, X)$. 

\item Simulation procedure: 
\begin{itemize}
\item Sample a locate in the region as the disease locus. 
\item For a given disease model, and effect size, sample the genotype at the locus according to the disease status. If case, the genotype frequencies can be calculated from the risk allele frequency in the population (control) and the effect size. 
\item Starting at the disease locus, sample the rest of the two haplotypes: first sample $X_d$ for the disease locus, the probability should be high for the haplotypes with the same genotype at $d$; next sample $X_j$ for other positions using the HMM described above; Then the haplotype is sampled according to $X_j$ for all $j$'s, with mutations allowed. 
\end{itemize}

\item Power analysis: suppose we want to evaluate a method, or a SNP set, we simulate the data: at each simulation, chooses a disease locus, and set the effect size (and the disease model), then simulate the genotypes according the procedure above, and apply the test method on the simulated data. The procedure is repeated many times, with different disease loci. Finally, to obtain power for a given threshold of the test statistic, compute the fraction of times when the test statistic (at some locus in the region) exceeds the threshold. 

\end{itemize}

\subsection{Meta-analysis}

Methods for meta-analysis: 
\begin{itemize}
\item Reference: [LL, The Fundamentals of Modern Statistical Genetics, Chapter 11], [Bakker \& Voigh, Human Mol Genetics, 2008; Zeggini \& Ioannidis, Pharmacogenomics, 2009], [Evangelos \& Ioannidis, NRG, 2013]

\item Statistics background: see ``Meta-analysis'' in Statistics Notes. 

\item Fisher's method: drawbacks include, direction of effects, weighting of studies, inability to produce a summary effect.  

\item $Z$-score based: The most commonly used is the Liptak method, let $Z_i$ be the $Z$-score of the $i$-th study (the sign of $Z_i$ should reflect direction of effect), 
\begin{equation}
Z_{\text{meta}} = \sum_i w_i Z_i 	
\end{equation}
where the weights should satisfy $\sum_i w_i^2 = 1$. It can be proved (using sum of normal random variables) that if $Z_i \sim N(0,1)$, then $Z_{\text{meta}} \sim N(0,1)$. $w_i$ is often chosen to be $\sqrt{\frac{N_i}{N_{\text{total}}}}$. This is also equivalent to: inverse of the standard error of the regression coefficient. From linear model, we know that $\Var(\hat{\beta})$ is proportional to inverse of total variance in predictor, which is $2 N_i p(1-p)$ for SNPs. So the standard error is proprotional to $1/\sqrt{N_i}$. 
\begin{itemize}
	\item Correction for sample imbalance: [Bakker \& Voigh] the effective sample size (assume case and control are balanced) is determined by power analysis: should be equivalent to the actual sample (imbalanced) in terms of power. 
	\item Effective sample size formula: for any individual study, it is given by [Willer et al, METAL, Bioinformatics, 2010]
\begin{equation}
N_{\text{eff}} = \frac{4}{\frac{1}{N_{\text{case}}} + \frac{1}{N_{\text{control}}}}
\end{equation}
\end{itemize}

\item Fixed Effect model: inverse variance method, where the effect size is: log-OR for binary trait, and $\beta$ (regression coefficient) for quantitative trait. Advantage over $p$-value or $Z$-value based methods: maximize power. Most existing meta-studies (70\%) use the fixed effects model. 

\item Random-effect models: model the between-study heterogeneity: the effect in each study is a RV from a common distribution.  Formal Bayesian methods have been developed for random-effect models. 
\begin{itemize}
	\item Often not used in discovery effects because it loses power. 
	\item More appropriate in studies with expected heterogeneity, e.g. across different populations. 
\end{itemize}

\item Representation of meta-analysis results: forest plot, drawing the effect of each study (with confidence interval), and the summary effect. 

\item Diagnosis of meta-analysis: the goal is to understand the heterogeneity of results: e.g. whether a large effect in one study is an outlier or a true effect. Techinques: sensitivity analysis - how sensitive the results are to a single study; meta-regression: explain the observed between-study heterogenity using additional covariates; etc. 

\item Test of between-study heterogenity: Crochran's $Q$ statistic, and $I^2$. However, both are underpowered at $<20$ studies. 

\item Ranking the results in GWAS meta-analysis by effect size: while conceptually it is the right thing to do, in practice, the effect size is often so modest that this is difficult to. 

\item Examples: 
\begin{itemize}
	\item Crohn's disease meta-analysis: [Barrett, NG, 2008; Franke, NG, 2010] compute $Z$-scores from individual scans, and combine ''the scores across all six datasets (inversely weighted by variance)''. 
	\item T2D meta-analysis: [Ziggini, NG, 2008] Liptak method, where effective sample size is calculated by power analysis: choose parameters s.t. the power if 50\%, and choose effective sample size that reaches the same power.  
	\item Remark: the details of meta-analysis are not clear in either cases: (1) CD meta-analysis: inverse variance weighting is usually applied to effect size (OR), not $Z$-scores; (2) T2D meta-analysis: power of a study depends on parameters such as allele frequencies, effect size, significance level (type I error), not clear how parameters are chosen for ``equivalent power''. 
\end{itemize}

\end{itemize}

Practical issues of meta-analysis: 
\begin{itemize}
	\item Data quality issue: e.g when different studies use different platforms. 
	\item Imputation: it is common to perform imputation on a common set of SNPs (e.g. HapMap SNPs) with multiple studies. However, the uncertainty of imputation results should be considered. 
	\item Dealing with inflation/population stratification in meta-analysis. In each individual study: should apply the correction (genomic control or PCA, etc.), before performing meta-analysis. After meta-analysis: may also need correction because of issues such as sample overlap, etc. 
	\item Generally, use the same criteria for SNP filtering: AF $< 1\%$, $P < 10^{-5}$ for HWE, imputatino quality index $< 0.3$ will be removed [Evangelou]. 
\end{itemize}

Criteria of replication test: [LL, Section 11.5]
\begin{itemize}
\item Sample size of replication: generally need to be sufficiently large, i.e. at least 80-90\% power. Otherwise, high false negative, and dismissal of true findings. 
\item Genetic model: need to be the same effect, and the same genetic model in the replication test.  
\item $P$-value criterion: usually should be nominally significant, i.e. $P < 0.05$. 
\end{itemize}

Between-study heterogeneity: 
\begin{itemize}
\item Possible causes:
\begin{itemize}
\item Phenotype definition [Evangelos]: may affect the estimated effect sizes. Harmonization is often desirable. 

\item Population ancestry: assessment shows that GWAS-discovered variants show modest correlation in MAF between ancestries and different effects in different ancestry. A trans-ethnic meta-analysis that considers the distance between different ethnic groups. 

\item Population structure. 

\item Gene-gene or gene-environment interactions. 

\item Difference of designs of studies: differnet platforms, thus the same causal SNPs may be linked to tagged SNPs to different degrees. The phenotype defintion may be different: e.g.  FTO association with T2D (significant association with body mass index, but not if body mass index removed). 

\item Sex difference. 

\item Winner's curse: earlier studies may suggest stronger effects, and the most significant SNPs are likely to exhibit some regression-to-the-mean upon replication. 

\item Other biases or errors: e.g. sample selection (e.g. how control group is chosen, whether disease subjects are rigorously excluded). 
\end{itemize}

\item Assessing heterogeneity: e.g Corchran's Q statistic: 
\begin{equation}
Q = \sum_i w_i (d_i - d^F)^2	
\end{equation}
where $d_i$ is the study-specific effect size, $d^F$ is the summary effect size, and $w_i$ is the weight of each study. Note the limitations of each statistic: sensitivity to outliers, to the number of studies, etc. 
\end{itemize}

Random-Effects Model Aimed at Discovering Associations in Meta-Analysis of Genome-wide Association Studies [Han \& Eskin, AJHG, 2011]
\begin{itemize}
\item Motivation: in GWAS meta-analysis, it is often true that random-effect model (RE) is not as powerful as fixed-effect model (FE). Confirmed in simulation that even when there is heterogeneity, RE still loses power. Why? 

\item Problem of traditional random-effect method: traditional RE first estimates the summary effect and its confidence interval, then obtain its $p$-value. The problem is that when doing this, the model implicitly assumes $H_0: \mu = 0$ with the same $\tau^2$ as $H_1$, as confidence interval is based on RE model. The true $H_0$ should be $\mu=0, \tau^2=0$. 

\item New RE model: first show that LRT is equivalent to FE model, where we use CI to derive $p$-value: $S_{FE} = Z_{FE}^2$, where $Z_{FE}$ is the Z-score for the summary effect. Next, for traditional RE, the LRT statistic is equivalent to testing $\mu=0$ assuming $H_0$ and $H_1$ share the same $\tau^2$: $S_{RE} = Z_{RE}^2$. In the new RE model, do LRT, where $H_0: \mu=0, \tau=0$:
\begin{equation}
S_{New} = S_{FE} + S_{Het}
\end{equation}
where $S_{Het}$ test heterogeneity, and is similar to Cochran's $Q$. The distribution of $S_{FE}$ is chi-square, and $S_{Het}$ is mixture of 0 and 1-df chi-square. 

\item Results: when the heterogeneity is low, new RE has similar power to FE; when heterogeneity is high, new RE has higher power. Tradition RE always has lowest power. 
\end{itemize}

\subsection{Imputation and Haplotype Methods}

Reference: [Neale, Statistical Genetics, 2007, Chapter 17, 18]

Genotype imputation [Li \& Abecasis, ARGHG, 2010]
\begin{itemize}
\item Imputation in related individuals: 
\begin{itemize}
	\item Intuition: family memebers would share long stretches of haplotypes (several Mb) - IBD regions. So we will need only a small set of genetic markers to cover IBD regions, and if we identify the IBD region, we can impute untyped markers in the regions. 
	
	\item Application in linkage analysis: for disease-related loci, regions of IBD would be more similar in phenotypes than other regions. 
\end{itemize}

\item Imputation in unrelated individuals: assume reference panel of haplotypes are available (100-200 kb in length, typically 10-20 genotyped markers), then we infer which haplotype generates given SNP regions. If we know the haplotypes, we can infer the untyped SNPs.  

\item Accuracy of imputation: 
\begin{itemize}
	\item Procedure of estimating accuracy: masking the typed markers and impute. Comparison of imputed and masked SNPs.  
	\item Overall error rate (discrepancy): about 1-2\%. 
	\item Measuring accuracy: $r^2$ between the observed and imputed allele counts. It is directly translated to power calculation: e.g. 1000 imputed samples equivalent to 930 typed samples if $r^2 = 0.93$. 
\end{itemize}

\item Application: increasing power and fine-mapping. Ex. in a LDL GWAS. 

\item Application: meta-analysis of multiple stuides genotyped using different platforms. 

\item Practical considerations: 
\begin{itemize}
	\item Non-European samples: for some populations, use specific panel from HapMap, e.g. YRI for west Africa and JPT + CHB for east asian. For other samples, evalaute which reference panel works best (by masking/cross-validatin). Or use multiple reference panels. 
	
	\item Determining accurate imputed genotypes: use $r^2$ measure, the posbility that an imputed genotype is correct - this is not comparable across different MAFs (e.g. if MAF is low, then most of time it will be correct if one just assigns the major allele). 
	
	\item Association testing: allele count based association testing. 
\end{itemize}

\item Application: low-coverage sequencing data. Use imputation to combine information across individuals who share a haplotype stretch. Ex. 400 sequenced at 2x, using imputation, sites with MAF $>2\%$ can be genotyped with $>99\%$ accuracy. 
\end{itemize}



Genotype imputation for genome-wide association studies [Marchini \& Howie, NRG, 2010]
\begin{itemize}
\item Intuitions of imputation: each sample is a mosaic of multiple reference haplotypes. The challenge is to infer the source haplotype of every study sample. These source haplotypes should have these properties: \begin{itemize}
	\item They are more likely to be from frequent haplotypes in the reference panel. 
	\item There should be higher level of mosaicism in high recombination regions. 
	\item The source haplotypes should be similar to the sample genotypes. 
\end{itemize} 
These intuitions can be encoded by HMM model. 

\item IMPUTE v1: let $G$ be the genotype data and $H$ be the reference haplotypes. Each haplotype region in the study sample can be viewed as mosaic of the haplotypes in the reference sample. Let $Z_i$ be the source haplotype of study sample $i$, $\theta$ be the mutation rate parameter and $\rho$ the recombination rate parameters, we have: 
\begin{equation}
P(G_i|H, \theta, \rho) = \sum_{Z_i} P(G_i|Z_i, \theta) P(Z_i|H, \rho)
\end{equation} 
The term $P(G_i|Z_i, \theta)$ encodes the intuition that the genotypes should be similar to reference haplotypes (few mutations), and the term $P(Z_i|H, \rho)$ encodes the intuition about haplotype frequencies and recombination. The latter is based on a HMM. In inference, $\theta$ is fixed and the program estimates $\rho$ (allow variation across regions). The parameter $N_e$ must be specified by the user. 

\item IMPUTE v2: in v1, the haplotype phasing in the study sample is intetraged out($Z$ term). In v2, the program tries to leverage study samples in addition to reference samples, so it estimates haplotype phases explicitly: it alternates between phasing (of both study and reference samples) and imputation given phases. 

\item FastPhase: the idea is that at large sample size, there are many possible haplotypes, and that makes the imputation programs slow (quadratic of number of haplotypes). So assume haplotypes can form a small number of clusters. And do similar HMM for the haplotype clusters. Essentially, the model has two parts: first how genotype relates to the haplotype cluster; second how observed haplotypes relate to the clusters. Let $\alpha$ denotes the weights of clusters at each site $l$ (similar to haplotype frequencies): 
\begin{equation}
P(G_i|\alpha, \theta, r) = \sum_Z P(G_i|Z, \theta) P(Z|\alpha,r)
\end{equation}
where $r$ is recombination rate. And
\begin{equation}
P(G,H|\alpha, \theta, r) = \prod P(G|\alpha, \theta, r) P(H|\alpha, \theta, r)
\end{equation}

\item Factors affecting imputation accuracy: error rates about 5\%. Error rates depend on MAF (lower frequency higher error). Reference panel important: how close they are to the study sample. 

\item Association testing using imputed data: 
\begin{itemize}
	\item Frequentist approach: use expected allele count in regression model. Score test: likelihood method that marginalize the missing data. 
	\item Bayesian approach: let $D$ be the data, then we marginalize the imputed genotypes. Let $p_{ijk} = P(G_{ij}=k)$ be the imputation probability of the genotype of sample $i$ at site $j$. Then we have model evidence: 
	\begin{equation}
	P(D|M) = \int \left[\prod_{i}  \sum_k P(D|G_{ij=k}, \theta) p_{ijk}\right] P(\theta|M) d\theta
	\end{equation}
\end{itemize} 

\item Joint imputation and testing: the problem of separate imputation and testing is that the effect size will be underestimated as the data are imputed under the null model. However, the improved performance from using study samples in imputation outweights the advantage of joint association and testing. 
\end{itemize}

Bayesian methods for imputation: the idea is to use the probability distribution $P(H)$, where $H$ are the hidden haplotypes, to encode the criterion of fewer mutational events among haplotypes [Stephens \& Donnelly, AJHG, 2001]. 
\begin{itemize}
\item Inference: Gibbs sampling, sample haplotype of one individual given haplotypes of all other individuals. The algorithm consists of repeated steps of sampling $H_i$ from $P(H_i | G, H_{-i})$, where $H_{-i}$ stands for the haplotypes of all other individuals in the sample: 
\begin{equation}
P(H_i | G, H_{-i}) \propto P(H_i | H_{-i}) \propto \pi(h_{i1}|H_{-i}) \pi(h_{i2}|H_{-i}, h_{i1})
\end{equation}

\item Conditional distribution $\pi(h|H)$: the sipmler form would be multinomial distribution. A better form is to use coalescent modeling: the haplotypes should be clustered, i.e. only a few haplotypes can generate all observed ones; or the new haplotype $h$ should be the same or similar to some of the existing haplotype in $H$. Formally, the probability is computed from sampling of an ancestral haplotype, $\alpha$, and applying the mutational events (perhaps both substituation and recombination events): 
\begin{equation}
\pi(h|H) = \sum_{\alpha} P(\alpha) P(\alpha \rightarrow h) = \sum_{\alpha} \sum_s P(\alpha) P(s) P(\alpha \rightarrow h | s \text{ events})	
\end{equation}
where $s$ is the number of mutations from $\alpha$ to $h$ (follow geometric distribution). The first term is $r_{\alpha} / r$, where $r$ is the sample size and $r_{\alpha}$ the number of haplotypes of type $\alpha$; the second term encodes the geometric distribution of number of mutations:
\begin{equation}
P(s) = \left( \frac{\theta}{r+\theta}\right)^s \frac{r}{r+\theta}
\end{equation}
where $\theta$ is the mutation rate. And the last term is the $\alpha h$ entry of the matrix $P^s$, where $P$ is the mutation matrix. 

\item Other ideas: 
\begin{itemize}
	\item Partition ligation (PC): for large region, first do haplotype estimation in short segments (e.g. 10 bp), then ligate the segments. 
	\item Parental information: can be used to improve the inference, in particular, for heterozygote children, if the parent is homozygotes at multiple markers, the phasing may be determined. 
\end{itemize}

\item Remark: 
\begin{itemize}
	\item Generally highly accurate: the best method is PHASE v2.1: less than 6\% of errors of unrelated individuals, and 0.2\% for trio (parents) data. Other methods are slightly worse. 
	\item Limitations: not directly model the coalescence. E.g. for the PHASE method, the coalesence is treated at the level of prior. 
	\item Idea: use HMM to learn the block structure in the data. Ex. one state per marker (one state for each allele), the transition probabilities among the states would suggest which markers form haplotypes. 
\end{itemize}
\end{itemize}

Modeling Linkage Disequilibrium and Identifying Recombination Hotspots Using Single-Nucleotide Polymorphism Data [Li \& Stephens, Genetics, 2003]
\begin{itemize}
\item Motivation: block structure is not often obvious from pairwise LD. Need a model that uses all the data to study recombination. 
	
\item Problem: suppose we have a set of haplotype data, $h_1, \cdots, h_n$, how can we estimate the recombination rates? 

\item Intuitions of the model (Figure 2): we need to specify the conditional distribution $P(h_k|h_1, \cdots, h_{k-1})$. Intuitively, $h_k$ should be mosaic of haplotypes $h_1$ to $h_{k-1}$. This distribution should have these properties: 
\begin{itemize}
	\item The next haplotype is more likely to match a haplotypes that has already been observed many times. 
	\item The probability of seeing a new haplotype decreases as $k$ increases. 
	\item The probablilit of seeing a new haplotype increases as $\theta$ (recombination) increases. 
	\item The haplotype $h_k$ tend to differ from existing haplotypes by only a few mutations. 
	\item The next haplotype should be somewhat similar to existing haplotypes. 
\end{itemize}

\item Model: the total probability is $P(h_1, \cdots, h_n) = P(h_1) P(h_2|h_1) \cdots P(h_n|h_1, \cdots h_{n-1})$. To obtain the conditional distribution $P(h_k|h_1, \cdots, h_{k-1})$, assume that $h_k$ is a mosaic of existing haplotypes, and let $X$ be the variables of which haplotypes for $h_k$. We use a HMM for $X$. Specifically, the transition probability at position $j+1$:  
\begin{equation}
P(X_{j+1}=x' | X_j = x) = (1 - \exp(-\rho_j d_j/k)) / k
\end{equation}
when $x \neq x'$. The rate of total recombination at position $j$ is: $\rho_j d_j$, where $\rho_j$ is the recomibnation rate ($4 Nc$ where $c$ is recombination rate) and $d_j$ the distance. But the more existing haplotypes we have, the less likely we will need recombination (i.e. the more likely $h_k$ is an exact copy of some existing haplotypes), so the actual rate is $\rho_j d_j / k$. Next if we do have a recombination (choose a different haplotype), each of the $k$ haplotypes has equal probability. Similarly, we can derive the transition probability when $x' = x$ (skipped). 

\item Remark:
\begin{itemize}
	\item The model depends on the order of haplotypes in inference. In practice, try multiple orders and averaging. 
	\item See Discussion in Appendix A about the choice/justification of $\rho / k$ in the transition rate. 
\end{itemize} 
\end{itemize}

A new multipoint method for genome-wide association studies by imputation of genotypes [Marchini \& Donnelly, NG, 2007]:
\begin{itemize}
	\item Idea: 
	\begin{itemize}
		\item The LD patterns would allow one to infer the recombination history in the interested SNP, meanwhile, the genotype can be inferred if the recombination events are known: if no recombination, the SNP is determined from the haplotype; if one recombination, from the other haplotype (if there are only two); etc. 
		\item Approximating population genetic process: to infer the complete history in the genealogy is difficult. For each individual, the SNP depends on its immediate neighbor and the probability that a recombination event occurs between the two SNPs. This can be modeled as a simple HMM. 
	\end{itemize}
	
	\item Model: $N$ haplotypes (from HapMap), and $K$ individuals with $G_i$ the genotype of $i$-th individual: $G_{il} \in \{0,1,2,\text{missing}\}$ for the $l$-th SNP. Need to infer missing genotypes $G_M$ from the observed genotypes $G_O$: $P(G_O|G_M,H)$. Assume that: (1) $G_i$ is sampled independently. (2) The probability $P(G_i|H)$ is given by a HMM. Specifically, given $G_{il}$ at $l$-th locus, sample $G_{i,l+1}$ according to whether recombination occurs: if no recombination, from the current haplotype; if recombination, sample with equal probability from any of the $N$ haplotypes. The hidden state at a locus in HMM is thus the haplotype where the locus is sampled from. Suppose $Z_{il}^{(1)}$ and $Z_{il}^{(2)}$ are the two haplotypes of the $i$-th individual at $l$-th locus, we have: 
	\begin{itemize}
		\item Transition probability: $P((Z_{il}^{(1)}, Z_{il}^{(2)}) \rightarrow (Z_{i,l+1}^{(1)}, Z_{i,l+1}^{(2)}))$ is given by the recombination probability. The probability of the same haplotype is equal to the probability of no recombination plus the probability of recombination with the same haplotype. The former is given by: $\exp(-\rho_l/N)$, where $\rho_l = 4 N_e r_l$ ($r_l$ is the genetic distance per generation). This could be understood intuitively as per generation rate times number of generations (about $2N_e/N$: generating $2N_e$ copies from $N$ haplotypes). The later is: $(1+\exp(-\rho_l/N))/N$. 
		\item Emission probability: $P(G_i|Z_{i}^{(1)}, Z_{i}^{(2)},H)$ mimics the effect of mutation. Estimate the probability of mutation using an approximation to population genetics model.
		\item Remark: since the HMM needs to track two states (diploid) per step, thus the complexity is $O(n^2)$, where $n$ is the number of haplotypes.  
	\end{itemize}
	
	\item Testing association within a region: suppose there are $W$ SNPs in a region, we test the hypothesis: $M_0$ - no assication; $M_1$ - some SNP is associated with the disease. Let $P(S_i)$ be the probability that $S_i$ is associated given $M_1$ is true, then we have: 
	\begin{equation}
		BF_{region} = \frac{P(D|M_1)}{P(D|M_0)}	= \frac{\sum_i P(S_i) P(D|M_1,S_i)}{P(D|M_0)} = \sum_i P(S_i) BF(S_i)
	\end{equation}
	Thus $BF_{region}$ is a weighted average of BF of all SNPs. An alternative method is simply the maximum of BF in the region. 
	
	\item Simulation: for each of the SNPs in the ENCODE region (HapMap), create simulated case-control data assuming it is a causal SNP. Only show a subset of SNPs to any program (Affymetrix SNPs). The programs are assessed by the power at different type I error rate. 
	
	\item Results: 
	\begin{itemize}
		\item Data: WTCCC, using 10-Mb window for imputation.
		\item In simulation: imputation significant increases power for rare non-tagged SNPs, as well as common non-tagged SNPs (with smaller effect). Imputation makes small difference in the tagged SNPs. 
	\end{itemize}
	
	\item Remark: 
	\begin{itemize}
		\item Haploptypes: the model assumes all $N$ haplotypes are equally frequent for computation of transition probabilities. This may be invalid. 
		\item Testing association in a region: under $M_1$, if one SNP is associated, then its nearby SNP will show a pattern different from $M_0$ because LD, so not independent. This may create bias: e.g. a weak causal SNP in a region with high SNP density and large LD may create signals in multiple nearby SNPs, thus the average BF is high in this region. 
	\end{itemize}
\end{itemize}

A Flexible and Accurate Genotype Imputation Method for the Next Generation of Genome-Wide Association Studies (IMPUTE v2) [Howie \& Marchini, PLG, 2009]
\begin{itemize}
	\item Motivation: the performance of imputation algorithms largely depend on the haplotypes. Existing methods often fix a set of haplotypes from reference sample. The idea here is to use the study sample as well to reconstruct better haplotypes. 
	
	\item Method: let $G_i$ be the genotype of $i$, and $H_i$ be the haplotype indicators (state path in HMM). The idea is to sample the HMM paths, instead of integrating them out (i.e. phasing the study samples). This way, we'll be able to learn more haplotypes from combined reference and study samples. We alternate two steps: 
	\begin{itemize}
		\item Step 1: sample $H_i$ from: $P(H_i|G_i, H_{(-i)}, H_R)$ where $H_{(-i)}$ are all haplotypes in study samples except $i$ and $H_R$ are reference haplotypes. This step needs sampling diploid states, so time complexity $O(n^2)$.  
		
		\item Step 2: given the haplotypes, we need to impute the missing genotypes. At this step, we only deal with haplotypes, and for the missing genotypes, we need to evaluate the probability of data given each different possible genotype, so we use forward-backward algorithm with running time $O(n)$. 
	\end{itemize}
	
	
	\item Multiple reference samples: the typed SNPs may be different, and this will be takend into account. Also the reference samples may be unphased (diploid). 
\end{itemize}

MaCH: Using Sequence and Genotype Data to Estimate Haplotypes and Unobserved Genotypes [Li \& Abecasis, GE, 2010]
\begin{itemize}
	\item Model: instead of fixing a set of haplotypes, the method infers the haplotypes from study samples. Specifically, let $G$ be gentoype data and $S$ be the mosaic states underlying the unphased genotype (two values, one for each chromosome), then we have: 
	\begin{equation}
	P(G,S) = P(S_1) \prod_j P(S_j|S_{j-1}) \prod_j P(G_j|S_j)
	\end{equation} 
	The model then update $S$ of each individual at each step, using the current set of haplotype estimates for all individuals as template using Li-Stephens model. 
\end{itemize}

Motivations for haplotype-based methods for association mapping:
\begin{itemize}
	\item At single locus level, the association of SNP with the disease may be very weak. The association can be improved by using haplotypes [Zaykin et al, HTR]. 
	\item If group subjects by haplotypes (i.b.d.), then these subjects tend to be genetically homogeneous, and thus it would be easier to locate genetic factors (comparison within a group: better control). 
	\item Regional test: it may be desirable to test on the level of a region encompasing multiple SNPs, for taking advantage of the information in multiple SNPs, or for testing at the gene level. In these tests, haplotypes may not be explicitly modeled, instead, the LD structure is taken into account while modeling/weighting multiple markers [Mingyao Li \& Chun Li, ATOM, Bioinfo, 2008]. 
\end{itemize}

Haplotype-based analysis: 
\begin{itemize}
\item Method: the linear predictor in logistic regression is a function of haplotype: 
\begin{equation}
\eta_i = \beta_0 + \beta_{H_{i1}} + \beta_{H_{i2}} 	
\end{equation}
where $H_{i1}$ and $H_{i2}$ are the two haptoypes of the $i$-th individual. Typically these methods assume that the risks from two haplotypes are multiplicative. Since the haplotypes not observed, they are either resolved, or integrated over by treating as hidden variables (distribution determined from the genotype $G_i$). The likelihood can be written as: 
\begin{equation}
f(y|G,\beta, H) = \prod_i \sum_{H_i} f(y_i | H_i, x_i, \beta) f(H_i | G_i, H)	
\end{equation}
where $G$ is phenotype data, $H$ are haplotypes (unobserved), $H_i$ is the haplotype of the $i$-th individual. Also, the additional covariates can be easily incorporated into regression. The standard LRT can then be applied (both the parameters and the hidden haplotypes are maximized). 

\item Tests: commonly test the null hypothesis that the regression coefficients of all haplotypes equal to zero (omnibus test), or the coefficient of a particular haplotype is 0 (haplotype-specific test).

\item Two-stage strategies: first estimate the haplotype frequencites using both case and control data, and then estimate other parameters where the haplotype frequencies are fixed. 

\item Ascertainment bias: the affected individuals are overrepresented in the case-control samples, thus the estimation of high-risk haplotypes will be inflated. One could use only data from the case group, but information on rare high-risk haplotypes is missed. [Stram \& Thomas, Hum Hered, 2003] proposed a correction of the likelihood function.   

\end{itemize}

Haplotype clustering: 
\begin{itemize}
\item Idea: group similar haplotypes according to some similarity measure, and assign the same risk to all haplotpes within the same cluster. Essentially an approximation to the population genetic process. 

\item Allowing dominance effects: the underlying causal variants may have non-multiplicative effects. [Morris AP, AJHG, 2006] models the presence of causal variant as a hidden variable in the haplotype. All haplotypes in a cluster have the same probability of carrying a causal variant. 
\end{itemize}

Issues of haplotype-based methods: the performance of these methods vary with the strength of LD, the haplotype diversity (e.g. the fraction of common haplotypes), etc. 
\begin{itemize}
\item Haplotype diversity: sample size is effectively reduced in the partition based on haplotypes. In particular, the rare haplotypes increase the d.f. of the test, thus reducing the power.

\item Comparison of single-locus and haplotype methods: single-locus can be as powerful or more than multi-locus test when permutation test is used to control for LD [Roeder \& Devlin, GE, 2005]

\item Remark: haplotype based methods are similar to regression on multiple markers, allowing for marker interactions. 
\end{itemize}

Haplotype pattern mining (HPM) [Toivonen \& Kere, AJHG, 2000]
\begin{itemize}
	\item Haplotype patterns: to avoid the problem of having too many haplotypes, define haplotype patterns allowing wildcard symbols, e.g. $(*, 2, 5, *, 3, *, *, *, *, *)$. Then a haplotype, $(4, 2, 5, 1, 3, 2, 6, 4, 5, 3)$ would match this pattern. For any pattern, its association with the trait can be defined using the similar metric, e.g. $\chi^2$. 
	\item The search of best patterns can be done using e.g. \emph{apriori} algorithm. 
\end{itemize}

Haplotype trend regression (HTR) [Zaykin \& Ehm, Human Hered, 2002]
\begin{itemize}
\item Motivation: haplotypes may be in higher LD with the causal varirant than the individual markers. Ex. haplotypes of three loci: $A_1 B_1 C_1$, $A_2 B_1 C_2$, $A_1 B_2 C_2$ and $A_2 B_2 C_1$, suppose $A$ is the causal locus, clearly, $A$ is in LE with $B$ and $C$, but is in LD with the $BC$ haplotype. 

\item Methods: 
\begin{itemize}
	\item Simple test: if the haplotypes are certain, then we have $2 \times L$ table, where $L$ is the number of haplotype, and can use the standard $\chi^2$ test on the table. 
	\item HTR: better to incorporate the uncertainty of the haplotype inference. Thus in logistic regression (genotype to trait), instead of integer count (of the number of minor alleles), use fractional count, which is the expected number under the posterior distribution. 
\end{itemize}
\end{itemize}

Haplotype clustering allowing dominance effects: [Morris, AJHG, 2006]
\begin{itemize}
\item Intuition: similar haplotypes are clustered; furthermore, a cluster may be associated with the causal variant (thus disease risk) and the genetic model of the (hidden) causal variant can be implemented for haplotypes. Let $C_k$ be the $k$-th cluster, $Z_i^{(a)}$ be the $a$ allele (first or second) of the causal polymorphism of the $i$-th individual, then the association of a cluster to the causal variant is defined by: $\phi_k = P(Z_i^{(a)} = A | C_k)$, where $A$ is the causal variant. 

\item Notations: our data consists of the phenotype $y$, the genotype $G$, and other covariates $x$, the hidden variables are: $C$ - clusters $1$ to $K$; $H$ - haplotypes, $H_i = (H_{i1}, H_{i2})$ is the haplotype of the $i$-th individual; and $Z$ - the causal variant, $Z_i = (Z_{i1}, Z_{i2})$ is the causal variant genotype of the $i$-th individual. The parameters are: $h$ - haplotype frequencies; $\theta$ - model parameters. 

\item Model: several components: 
\begin{itemize}
	\item $C_k \rightarrow H_i$: haplotype clustering. 
	\item $H_i \rightarrow G_i$: the consitency of $G_i$ and $H_i$.
	\item $C_k \rightarrow Z_i$: association of haplotype cluster and causal variants, determined by $\phi_k = P(Z_i^{(a)} = A | C_k)$. 
	\item $Z_i, x_i \rightarrow y_i$: disease risk is a function of the causal variant and covariates. Genetics models are implemented in this distribution (e.g. dominance effects). 
\end{itemize}
The likelihood function: 
\begin{equation}
\begin{array}{lll}
P(y|G, x, h, \theta) & \propto & \prod_i \sum_{H_i} P(y_i | H_i, x_i, \theta) P(H_i|G_i, h)	\\
 & \propto & \prod_i \sum_{H_i} \sum_{Z_i} P(Z_i | H_i, \theta) P(y_i | Z_i, x_i, \theta) P(H_i | G_i, h)
\end{array}
\end{equation}
Note that the term $P(Z_i | H_i, \theta)$ is computed by summing over all possible clusters for $H_i$. 

\item Inference: MCMC algorithm to sample $P(\theta|D,M) \propto P(y|G, x, h, \theta) P(\theta|M)$. The key variables of interest are $\Psi_j$: the probability that the $j$-th haplotype carries a causal variant (by summing over all possible clusters of haplotype $j$).
\end{itemize}
%%%%%%%%%%%%%%%%%%%%%%%%%%%%%%%%%%%%%%%%%%%%%%%%%%%%%%%%%%%%
\section{Family-Based Methods}

Reference: [Thomas, Chapter 9], [Laird \& Lange]

Using family members as controls: 
\begin{itemize}
\item Idea: the case-control studies suffer from the problem of population stratification (cryptic relatedness, etc.). If we can use match every case with a family member as control, then the other confounding variables (diet, culture, etc.) will be matched. By Conditional Logistic Regression, the different baseline risks for different families can be ignored with a matched design. 

\item Design: often use siblings, pseudo-sibs (e.g. TDT) and cousins for controls. Not using other members to control age and sex. Note: pseudo-sibs mean the expected genotypes that are allowed by the parent mating types. 
\end{itemize}

Transmission disequilibrium test (TDT): 
\begin{itemize}
\item Motivation: TDT is designed to provide the internal control - the background of the same parents. Basically TDT compares the allele frequencies of markers transmitted to affected children: if no association, a particular allele is equally likely to be transmited or not, to the affected children. 

\item Test: for a particular alleles of a marker, let $T$ be the number of times it is transmitted to an affected child, and $R$ to be the number of times it is not. Under the null hypothesis, we expected $T = R$. We thus define the test statistic: 
\begin{equation}
Q_1 = \frac{(T-R)^2}{T+R}	
\end{equation}
With multiple alleles, we use $T_i$ and $R_i$ for the $i$-th allele respectively, and define the test as: 
\begin{equation}
Q_m = \sum_{i=1}^m \frac{(T_i-R_i)^2}{T_i+R_i}	
\end{equation}

\item Likelihood model of TDT: let $\beta$ be the effect size, and $R_G(\beta)$ for the RR of the genotype $G$. We have $N$ families (trios), and $G_i$, $Y_i$ for the genotype and phenotype of the child $i$ and $P_i$ be the parent genotypes. The likelihood: 
\begin{equation}
L(\beta) = \prod_i P(G_i|Y_i=1, P_i) = \prod_i \frac{P(Y_i=1|G_i) P(G_i|P_i)}{\sum_{G_i^*} P(Y_i=1|G_i^*) P(G_i^*|P_i)} = \prod_i \frac{R_{G_i}(\beta)}{\sum_{G_i^*} R_{G_i^*}(\beta)}	
\end{equation}
where $G_i^*$ is the permitted genotype based on the parent genotype. Use the simple log-additive model of RR (the RR of a genotype is the product of the RR of each allele), the likelihood can be written as: 
\begin{equation}
L(\beta) = \prod_{ij} \frac{r_{g_{ijt}}}{r_{g_{ijt}}+r_{g_{ijn}}}	
\end{equation}
where $r_g$ is the RR of the allele $g$, $g_{ijt}$ is the allele transmitted from parent $j$ of subject $i$ and $g_{ijn}$ is the corresponding nontransmitted allele. From the likelihood, we see that it is a function of the number of transmitted and nontransmitted alleles. 

\item TDT as a score test [Thomas, Chapter 9, p274]: from the likelihood function, assume $R_{G_i}(\beta) = \exp(g_i \beta)$, the log-likelihood: 
\begin{equation}
l(\beta) = \sum_i \left[ g_i \beta - \log \sum_{g \in G} \exp(g\beta)\right]	
\end{equation}
It can be shown that the score:
\begin{equation}
U = \sum_i \left[ g_i - \E(g_i|p_i)\right]	
\end{equation}
where $\E(g_i|p_i)$ is the expectation of $g_i$ given the parental genotypes $p_i$. So the TDT is a test of departure of the actual transmitted genotypes in the affected children vs. what is expected from Mendel's Laws. 

\item TDT is a test of both linkage and association: it can be shown that under either situation, $\theta = 1/2$ or $D' = 0$, the expectation of TDT statistic is 0. 
\end{itemize}

Analysis of TDT: 
\begin{itemize}
\item Power analysis: we need to compute the distribution of $T$ under $H_A$: the marker is linked with the disease locus with $\theta < 1/2$. Suppose there are two alleles at the disease locus $s$, and $m$ alleles at the marker. We denote $TM$ the transmitted marker and $NM$ the non-transmitted marker and $A$ the disease status ($A = 1$: affected). We define: 
\begin{equation}
t_{ij} = P(TM = i, NM = j | A=1)	
\end{equation}
Then the the probabilities of transmitting and non-transmitting $i$ to an affected child (to compute the distribution of $T$ and $R$) are: 
\begin{equation}
P(TM = i | A = 1) = \sum_{j=1}^m t_{ij} \qquad 	P(NM = i | A = 1) = \sum_{j=1}^m t_{ji} 
\end{equation}
The probability $t_{ij}$ can be computed by summing over the transmitted disease allele ($TD$), denoted as $s$: 
\begin{equation}
P(TM = i, NM = j, A = 1) = \sum_{s=1}^2 P(TH = si, NM = j) P(A = 1|TD = s)	
\end{equation}
where $TH$ stands for the transmitted haplotype. The first term can be computed from the meiotic process: 
\begin{equation}
P(TH = si, NM = j) = p_{si} p_j (1 - \theta) + p_{sj} p_i \theta
\end{equation}
and the second term from summing over the possible transmitted disease allele from the other parent ($OTD$):
\begin{equation}
P(A = 1|TD = s)  = \sum_{u=1}^2 P(A=1|TD =s, OTD = u) P(OTD=u)	
\end{equation}
\end{itemize}

Family-based Association Test (FBAT) [LL, Appendix B]
\begin{itemize}
\item Model: suppose we have multiple families, and in each family, we have data of multiple offsprings. Let $x_{ij}$ be the genotype of the $j$-th sibling of the $i$-th family, and $y_{ij}$ be the phenotype. Let $P_i$ be the parent genotypes of the $i$-th family. It can be shown from the likelihood (similar to TDT) that the score test
\begin{equation}
U = \sum_{i,j} \left[ y_{ij} - \E(y_{ij}) \right] \left[ x_{ij} - \E(x_{ij}|p_i) \right]	
\end{equation}
So the test is effectively a covariance between phenotypes and genotypes (departure from the mean - in the case of genotype, mean is conditional on parental genotypes). The variance: 
\begin{equation}
\Var(U) = \sum_{ij} \left[ y_{ij} - \E(y_{ij}) \right]^2 \Var(x_{ij}|P_i)
\end{equation}
Both $\E(x_{ij}|P_i)$ and $\Var(x_{ij}|P_i)$ are computed using Mendel's Law. 

\item Special case: when $y_{ij}$ is binary and only cases are considered, this reduces to TDT. 
\end{itemize}

A Bayesian approach to genetic association studies with family-based designs [Naylor \& Lange, GE, 2010]
\begin{itemize}
\item Model ideas: in FBAT, only transmission information is used (inference is conditioned on affected states and parental genotypes). However, there is additional information in the parental genotypes and child phenotypes: if parents carrying minor alleles are also likely to have affected children, then this SNP is likely associated. More formally, let $x$ be the child genotype, $y$ be their phenotypes and $P$ be the parental genotypes:
\begin{equation}
P(x,y,P) = P(x|y,P) P(y,P)	
\end{equation}
The first part uses information in transmission, and the second information of association between parental genotypes and child phenotypes. 

\item Model: we are testing hypothesis using posterior odds: 
\begin{equation}
\frac{P(H_1|x,y,P)}{P(H_0|x,y,P)} = \frac{P(x|y,P, H_1)}{P(x|y,P, H_0)}	\times \frac{P(H_1|y,P)}{P(H_0|y,P)}
\end{equation}
We compute the BF of the first part. The model is $y_i \sim N(\mu+a x_i, \sigma^2)$, where $a$ is prior effect size. For the second part, we have $P(H_i|y,P) \propto P(y|H_i,P)$, for $i = 0,1$. The marginlal likelihood is computed by: 
\begin{equation}
\E(y_i|P_i) = \mu + a \E(x_i | P_i)	
\end{equation}
and summing over $x_i|P_i$. 
\end{itemize}

Rare Variant Analysis for Family-Based Design [De \& Laird, PLoS ONE, 2013]
\begin{itemize}
\item Multi-marker FBAT: for the $j$-th marker, let $x_{ij}$ be its genotype at the $i$-th subject. The FBAT statistic can be expressed as: 
\begin{equation}
U_j = \sum_{i} \left( y_{i} - \mu) \right) \left[ x_{ij} - \E(x_{ij}|P_{ij}) \right]		
\end{equation}
And its variance: 
\begin{equation}
\Var(U_j) = \sum_i (y_i - \mu)^2 \Var(x_{ij}|P_{ij})	
\end{equation}
To extend to the multi-marker case, we let $(U_1, \cdots, U_M)$ be the vector of FBAT statistics, where $M$ is the number of markers. Each of $U_i$ is normally distributed under $H_0$, but $U_i$'s are correlated because of LD in a region. So $U$ would follow MVN and we need to obtain the covariance matrix of $U$. We first compute $V_E$, with its element: 
\begin{equation}
e_{jk} = \sum_i (y_i -\mu)^2 \left[ x_{ij} - \E(x_{ij}|P_{ij})\right]	\left[ x_{ik} - \E(x_{ik}|P_{ik})\right]
\end{equation}
The covariance matrix is then obtained from $V_E$ with appropriate normalization (see the paper). The test statistic $T = U^T V_A^{-1} U$ follows $\chi^2$ distribution with dof equal to the rank of $V_A$.  

\item Rare variant FBAT: we cannot use the multi-marker test because it will lose power (too many degrees of freedom). So we effective collapse the rare variants in the region. The test statistic is:
\begin{equation}
W = \sum_i (y_i - \mu) \left[ \sum_{j=1}^M (x_{ij} - \E(x_{ij}|P_{ij}))\right]	
\end{equation}
The variance of $W$ now needs to take the correlation among RVs into account: 
\begin{equation}
\Var(W) = \sum_i (y_i - \mu)^2 \left[ \sum_{j=1}^M \Var(x_{ij}|P_{ij}) + \sum_{j \neq k} \Cov(x_{ij}, x_{ik}|P_{ij}, P_{ik}) \right]
\end{equation}
Also note that weighting can be allowed, i.e. we can have $W = \sum_j w_j U_j$, where $w_j$ is the weight of the $j$-th variant and $U_j$ the FBAT statistic of $j$. 
\end{itemize}

Utilising Family-Based Designs for Detecting Rare Variant Disease Associations [Preston and Dudbridge, Ann Hum Genet, 2014]
\begin{itemize}
	\item The advantage of family studies: no population structure, enrichment of risk variants.
	
	\item Study design: treating trios as case-controls (1) Pseudo-case-control (PCC); (2) Unrelated controls (UCC). Use trios, and enriched trios (one affected child from each multiplex family).
	
	\item Main results: (1) Simplex family (trios): family tests (score tests) and case-control tests have equal power. (2) Multiplex family (enriched trios): UCC gives the best power. Explanation: in multiplex families, pseudo-controls are not typical of general population, and devoid of genotypic variations - this reduces power.
	
	\item Comparison of various RVAT methods: SSU family (including C-alpha and SKAT) performs best at low causal fraction $<40\%$, while KBAC best at higher proportions.
\end{itemize}

The nature of nurture: Effects of parental genotypes [Kong, Science, 2018]
\begin{itemize}
	\item Model of genetic nurturing: considering a single locus/SNP. Let $T$ and $NT$ be the genotype of transmitted and non-transmitted alleles of the parents, respectively. Let $X_O$ be the phenotype of interest of the offspring. We assume that genotypes can affect $X_O$ via parental nurturing, which is mediated by parental phenotypes $Y_P$. Note that $Y_P$ are mostly unobserved. Let $\delta$ be the effect of the transmitted alleles on the offspring phenotype (direct effect), our model can be written as:
	\begin{equation}
	X_O = T \cdot \delta + Y_P \cdot \gamma + \epsilon_{X_O} \qquad Y_P = (T + NT) \cdot \alpha + \epsilon_{Y_P}
	\end{equation}
	We can then plug in $Y_P$ into the equation of $X_O$: 
	\begin{equation}
	X_O = T \cdot (\delta + \gamma \alpha) + NT \cdot (\gamma \alpha) + \epsilon
	\end{equation}
	Denote $\eta = \gamma \alpha$ be the genetic nurturing effect. If we do regression of $X_O$ on $T$ and $NT$, the expected effects of $T$ and $NT$ are given by:
	\begin{equation}
	\E(\hat{\theta}_T) = \delta + \eta \qquad \E(\hat{\theta}_{NT}) = \eta
	\end{equation}
	The difference of the two estimates gives genetic nurturing effects. If we use a prior on $\delta$, we can estimate the contribution of genetic nurturing on phenotypes. 
	
	\item Effect sizes from GWAS: in GWAS, we consider only transmitted alleles, so the GWAS effect sizes include both direct and genetic nurturing effects. This is true for any single SNP, and for PRS. So we can partition the PRS into those from direct effects and those from genetic nurturing. 
	
	\item Estimating genetic nurturing in Education Attainment (EA): first construct GWAS PRSs. Then given family data, we can transmitted PRS and non-transmitted PRS, $poly_T$ and $poly_{NT}$ (using transmitted and non-transmitted genotypes, but the fixed weights). Then we can compute the association of $poly_T$ and $poly_{NT}$ with EA status. This gives the estimated $\hat{\theta}_T$ and $\hat{\theta}_NT$, respectively. For EA, the effects are 0.22 and 0.067 (or PVE 0.05 and 0.025). 
	
	\item Remark: the PRS model takes estimated weights/effects. However, the weights already include genetic nurturing effects. Shall we use the data to estimate these effects for each SNP? 	
	
	\item Incorporating associative mating with direct effects only: two loci case. Because of associative mating, genotype of one parent is correlated with genotype of the other parent in the same locus (cis), or any other loci (trans) - Figure 2. Consider two loci A and B. Let $A_{TM}, A_{TP}$ be the genotype of maternal and paternal transmitted alleles. Similar for $B_{TM}, B_{TP}$. Let $\delta$ and $\delta_B$ be the direct effects of the two loci. We now have the model:
	\begin{equation}
	X = \delta (A_{TM} + A_{TP}) + \delta_B (B_{TM} + B_{TP}) + \epsilon
	\end{equation}
	The correlation of $X$ with $A_T = A_{TM} + A_{TP}$ and $A_{NT} = A_{NTM} + A_{NTP}$ gives the estimated effects $\hat{\theta}_T$ and $\hat{\theta}_{NT}$. To compute these covariance/correlation, we note the dependency of the genotypes in the above equation include:
	\begin{itemize}
		\item Cis-correlation: $A_{TM}$ vs. $A_{TP}$, $A_{TM}$ vs. $A_{NTP}$, $A_{TP}$ vs. $A_{NTM}$ and $A_{NTM}$ vs. $A_{NTP}$. Note that $A_{TM}$ vs $A_{NTM}$ and $A_{TP}$ vs $A_{NTP}$ are independent.
		
		\item Trans-correlation: $A_{TM}$ vs $B_{TP}$, and $A_{TP}$ vs. $B_{TM}$. Note that $A_{TM}$ vs. $B_{TM}$ are mostly independent (ignored in the derivation).  
	\end{itemize}
	 We also introduce some notation, for the variance of genotypes:
	 \begin{equation}
	 \Var{A_{TP}} = \Var{A_{TM}} = \Var{B_{TP}} = \Var{B_{TM}} = v
	 \end{equation}
	 and the same value for non-transmitted alleles. With these assumptions, we can now show that:
	 \begin{equation}
	 \Var{A_T} = \Var(A_{TP}) + \Var(A_{TM}) + 2 \Cov(A_{TP}, A_{TM}) = 2 v ( 1 + \text{cor}(A_{TP}, A_{TM}))
	 \end{equation}
	 where $\Cov(A_{TP}, A_{TM}) \neq 0$ due to associative mating. And
	 \begin{equation}
	 \Cov(A_T, A_{NT}) = \Cov(A_{TP} + A_{TM}, A_{NTP}+ A_{NTM}) = 2 v \cdot \text{cor}(A_{TP}, A_{TM})
	 \end{equation}
	 where we use $\Cov(A_{TP}, A_{NTP}) = \Cov(A_{TM}, A_{NTM}) = 0$. And the covariance of $A_T$ and $B_T$:
	 \begin{equation}
	 \Cov(A_T, B_T) = \Cov(A_{TM} + A_{TP}, B_{TM} + B_{TP}) = 2 v \cdot \text{cor}(A_{TM}, B_{TP})
	 \end{equation}
	 To get estimated effect of $T$ on trait, we compute:
	 \begin{equation}
	 \Cov(X, A_T) = \Cov(\delta A_T + \delta_B B_T, A_T) = \delta \Var(A_T) + \delta_B \Cov(A_T, B_T) = 2v \left[ \delta (1 + \text{cor}(A_{TM}, A_{TP})) + \delta_B \text{cor}(A_{TM}, B_{TP})\right]
	 \end{equation}
	 Using these results, we can obtain that the extra effect due to asociative mating, i.e. $\hat{\theta}_T = \delta + \phi_{\delta}$, with 
	 \begin{equation}
	 \phi_{\delta} = \frac{\delta_B \text{cor}(A_{TM}, B_{TP})}{1 + 2 \text{cor}(A_{TM}, A_{TP})}
	 \end{equation}
	 Note that the denominator is close to 1. So the main contribution comes from $\delta_B$, and the extent of associative mating, characterized by $\text{cor}(A_{TM}, B_{TP})$.
	 
	 \item Associative mating using PRS: the analysis above is based on two loci. Now we suppose $A$ captures PRSs and $B$ the remaining loci. Then the effect sizes $\delta$ and $\delta_B$ are not equal, we denote: $\pi = \delta_B^2 / \delta^2$ as the ratio of PVE. We can also view $\sqrt{\pi}$ as the ratio of number of independent alleles in B vs. A. Then we have:
	 \begin{equation}
	 \text{cor}(A_{TM}, B_{TP}) = \sqrt{\pi} \text{cor}(A_{TM}, A_{TP})
	 \end{equation}
	 With this, we can obtain:
	 \begin{equation}
	 \frac{\phi_{\delta}}{\delta} = \frac{\pi \text{cor}(A_{TM}, A_{TP})}{1 + 2 \text{cor}(A_{TM}, A_{TP})}
	 \end{equation}
	 Now we can estimate $\phi_{\delta} / \delta$ for EA by: (1) Estimation of $\pi$: we know that direct effects explain 17.0\% PVE (using heritability analysis). And PRS explains 2.45\%. So $\pi = (17.0-2.45)/2.45 = 5.94$. (2) Using transmitted genotype data, we can estimate $\text{cor}(A_{TM}, A_{TP}) = 0.012$. 
		
	\item Incorporating associative mating in the full model: we can expand the analysis above to the full model with both direct and genetic nurturing effects. The results:
	\begin{equation}
	\frac{\phi_{\eta}}{\eta}  = 2 \times \frac{\phi_{\delta}}{\delta} 
	\end{equation}
	The factor of 2 is from non-transmitted alleles have the same nurturing effect as the transmitted allele. In summary, we have the measured effects as:
	\begin{equation}
	\E(\hat{\theta}_T) = \delta + \phi_{\delta} + \eta + \phi_{\eta} \qquad \E(\hat{\theta}_{NT}) = \phi_{\delta} + \eta + \phi_{\eta}
	\end{equation}
		
	\item Estimating genetic nurturing and associative mating in EA: $\hat{\eta}$ accounts for 75\% of $\E(\hat{\theta}_{NT})$, and is 32\% of $\hat{\delta}$.
	
	\item Parent-of-origin: we can do the analysis on separate parents to estimate genetic nurturing effects via P or M. For 7 traits, only height show difference in $\eta$ between parents, and it accounts for 45\% of $\E(\hat{\theta}_T)$. 
	
	\item Impact of genetic nurturing on $h^2_g$ analysis: by definition $h^2_g$ should include only direct effects. But using GREML will include the nurturing effects, so the results will be biased. 
	
	\item Nature of genetic nurturing effects: in the EA cases, parental EA may mediate some of the nurturing effects, but is only a small fraction. 
	
	\item \textbf{Remark}: is it possible that a fraction of GWAS associations are entirely driven by genetic nurturing effects? 
\end{itemize}

Estimating genetic nurture with summary statistics of multi-generational genome-wide association studies [Wu and Qiongshi Lu, review for PNAS, 2020]
\begin{itemize}
	\item Problem: can we estimate direct and indirect (genetic nurturing) effects, including paternal and maternal ones, from GWAS summary statistics of offspring phenotype vs. offspring genotype, maternal genotypes and paternal genotypes, respectively? 
	
	\item Model: Fig. 1. Our goal is to express the estimated effects from GWAS summary stats as functions of the true effect sizes, including indirect effects. Let $Y_O$ be the offspring trait, $G_O$ be the offspring genotype. Let $G_M$ and $G_P$ be the maternal and paternal genotypes. We have:
	\begin{equation}
	Y_O = \beta_{\text{dir}} G_O + \beta_{\text{ind}\_M} G_M + \beta_{\text{ind}\_P} G_P + \epsilon 
	\end{equation}
	Now we can plug in $G_M = T_M + NT_M$ and $G_P = T_P + NT_P$, we have:
	\begin{equation}
	Y_O = (\beta_{\text{dir}} + \beta_{\text{ind}}) G_O + (\beta_{\text{ind}\_M} NT_M + \beta_{\text{ind}\_P} NP_M) + \epsilon
	\end{equation}
	We now use $\hat{\beta}_O = (G_O^T G_O)^{-1} G_O^T Y_O$. Plug in the equation of $Y_O$ and use the fact: 
	\begin{equation}
	\Cov(G_O, G_{M,P}) = p (1-p) (1 + \alpha)
	\end{equation}
	where $p$ is AF and $\alpha$ is the correlation of maternal and paternal alleles from associative mating. From this, we have:
	\begin{equation}
	\E(\hat{\beta}_O) = \beta_{\text{dir}} + \left(1 + \frac{\alpha}{2 + \alpha}\right) \beta_{\text{ind}}
	\end{equation}
	Similarly, we can derive $\hat{\beta}_M$ - summary stats of maternal GWAS, and $\hat{\beta}_P$. From these equations, we can derive the MOM estimators of $\beta_{\text{dir}},\beta_{\text{ind}\_M}$ and  $\beta_{\text{ind}\_P}$. The paper describes several cases, depending on whether we have all three GWAS, or just two (e.g. by merging maternal and paternal GWAS). See Table 1.
	
	\item Standard errors of the estimators and accounting for sample overlap: see ``Variances and covariances among effect size estimators`` in Supplements. First show $\Var(\hat{\beta}_{\text{dir}})$ as function of variance of $\hat{\beta}_{O,M,P}$ (given) and pairwise covariance among the three. Then to obtain covariance, model sample overlap, using LDSC intercept term.
	
	\item Correlation among the estimators: there are correlations b/c they are all functions of the same summary statistics, $\hat{\beta}_{O,M,P}$. (1) Correlation between direct and indirect effect estimators. (2) Correlation between maternal and paternal indirect effects. 
	
	\item Analysis: is the model equivalent to the use of individual level data? The benefit of individual data is to do joint regression. However, note that $T_M$ and $NT_M$ (also paternal) are independent, so it makes no difference. The dependency of $T_M$ and $T_P$ due to associative mating is accounted for in the summary stats model. 
	
	\item Simulations: use direct effect of $\beta = 0.02$ or PVE 0.04\% (roughly in line with 10K causal SNPs, toatl h2g = 0.4). Indirect effect is somewhat smaller. Show the estimator is unbiased, and type 1 error calibrated. 
	
	\item Application to birth weight: Fig. 3, show that using GWAS with complete sample overlaps (between O and M), the results are similar to the results using orthogonal phenotypes. The standard errors would be larger if not account for sample overlap. 
	
	\item Application to EA and study of genetic correlations of EA and 45 other traits: using GWAS-O and GWAS-MP to estimate EA direct and indirect effects. No genome-wide significant associations. Do correlation with 45 traits: found some pairs with large correlation with indirect effects, and large correlation of direct effect with ASD. 
	
	\item Correlation of indirect effects of EA vs. other traits: (1) EA-smoking relationship: indirect effect correlates with less smoking. (2) EA indirect effect correlates with lower BMI, larger height, and lower risk of RA. 
	
	\item EA-ASD relationship: only direct effect has positive correlation with ASD risk. Confirmed by transmission bias of EA PRS in ASD probands. 
	
\end{itemize}
%%%%%%%%%%%%%%%%%%%%%%%%%%%%%%%%%%%%%%%%%%%%%%%%%%%%%%%%%%%%
\section{Polygenic Modeling}

Heritability [personal notes]
\begin{itemize}
\item Why heritability matters? If some factor is important for a trait, then it should explain the variation of that trait. Thus heritabilty is a fundamental way of quantifying the importance of one factor over some trait. The idea can be applied in many contexts: e.g. measuring the importance of one type of variances over another (say coding vs non-coding). 

\item Perspective of genetic and phenotypic similarity: if a trait is very heritable, then genetically similar individuals should have similar phenotypes, so the ratio of phenotypic covariance and genetic covariance reflects heritability. 

\item Perspective of effect sizes: if there are many variants of large effect sizes, then a large fraction of phenotypic variance can be explained by genetic variants. 

\item General considerations of heritability estimation: genetically similar individuals tend to be raised in similar environment, then the genetic and environmental effects are coupled/confounded. If we do not remove the confounding, we will overestimate heritability. 
\end{itemize}

Strategies of mapping heritability: 
\begin{itemize}
\item Partition of variance: we write the phenotype as: 
\begin{equation}
y = \mu + g + \epsilon
\end{equation} 
where $g$ is the random genetic effect of an individual (see below: could view genotype of any individual as random; or a combination of genotype and effect size). From this, we obtain variance of $y$: 
\begin{equation}
V_P = \Var(y) = \Var(g) + \Var(\epsilon) = V_A + V_E
\end{equation}
where $V_A$ is interpreted as genetic variance, or variance explained by genetic variation. One can imagine two individuals with identical genotype (monozygotic twin), then their phenotypic covariance is $V_A$. 

\item Estimating $V_A$ through genetic relation/similarity: the idea is that covariance between phenotypes of individuals (related ones) carry information of $V_A$, so we derive the covariance matrix of $y$ (vector) in terms of $V_A$. 

\item Estimating $V_A$ through effect sizes: if we write $g$ as a product of genotype ($Z$) and effect sizes ($u$): 
\begin{equation}
y = \mu + Z u + \epsilon = \mu + \sum_j Z_{j} u_j  + \epsilon
\end{equation}
Then obviously covariance of $y$ depends on $u_j$'s. So we can estimate $u_j$'s through covariance of $y$. To relate effect sizes with $V_A$, we use basic relation from regression model: 
\begin{equation}
\text{SSR} = \sum_j \hat{u_j}^2 \Var(Z_j)
\end{equation}

\item Remark: the problem is basically making inference of a random effect model, marginalizing the random effects $g_i$. Becuase the joint distribution of $y$ (of many individuals) follow MVN, the key is to estimate the covariance matrix of $y$. 
\end{itemize}

Problems of estimating heritability [personal notes]:
\begin{itemize}
\item GCTA/LMM: assumption of random effect (non-sparse). How would this affect the estimation if the assumption does not hold (most likely)? Intuitively, if there are a small number of large effect loci, then LMM would estimate small $\sigma$ (most loci have no effects), and this would underestimate $h^2$? 

\item Bayesian method: could use a mixture prior. Matthew's comment: the sparse Bayesian prior is sensitive to the non-sparse scenario (perform badly), while GCTA is relatively robust to the assumptions. Question: what if we use a mixture prior, with flexible $\pi$? 

\item Impact of SNP heritability (effect size) prior: GCTA and LDSC vs. LDAK prior. Justification of LDAK prior was based on tagging in the paper, however, this is not satisfactory. Possible explanation of the observation that low LD regions (high recombination rates $r$) tend to explain more heritability than expected based on population genetics: causal variants are likely deleterious 
\begin{itemize}
	\item Linked selection view: given a deleterious variants, in low $r$ regions, it reduces the effective population size, hence strength of selection and this will lead to more deleterious variants (higher effect size) in the region. 
	
	\item Epistasis view: Given a deleterious variant, in low $r$ regions, a nearby SNP is less likely to be deleterious because the haplotype containing two deleterious SNPs will be under strong purifying selection. 
\end{itemize}
What explains the discrepancy? 
\end{itemize}

Quantitative genetic background: 
\begin{itemize}
	\item Random-effect model of phenotypes: we typically write the trait as $P = g + e$ where $g$ is genetic effect (breeding value) and $e$ the environmental effect. The genetic effect can be decomposed as: 
	\begin{equation}
	g = \mu + \alpha_i + \alpha_j + \delta_{ij}	
	\end{equation}
	where $\alpha_i$ and $\alpha_j$ are the effects of the two alleles, and $\delta_{ij}$ the dominance effect. We consider the model of $P$ as a random effect model: it has a group effect (genetic) and individual error (environmental). The groups can be thought of possible genotypes, and the group effect is random because which group an individual belongs to is random. An analogy is: treating individuals with different dosages (groups) - this is usually a fixed effect model; but if the dosage is randomly assigned, then a random effect model. 
	
	\item How breeding values are related to IBD? Let $g$ and $g'$ be the breeding values of two individuals, then the covariance between the two is from the decomposition of $g$ above: 
	\begin{equation}
	\Cov(g,g') = \left( \frac{1}{2} P_1 + P_2 \right) V_A + P_2 V_D
	\end{equation}
	where $P_1$ and $P_2$ are the probabilities of sharing one or two alleles IBD between the two individuals. 
	
	\item Kinship coefficient: probability that two randomly chosen alleles of two individuals are IBD. It's easy to see that: 
	\begin{equation}
	\phi = \frac{1}{4} P_1 + \frac{1}{2}P_2	
	\end{equation}
	When the two individuals share one IBD allele, there is only probability of 1/4 that the two randomly chosen alleles are these two ones; when two alleles share two IBD alleles, the chance is 1/2. Some special case: MZ twins, $P_2=1$, thus $\phi = 1/2$. 
	
\end{itemize}

Using expected genetic relationship (pedigree) to estimate heritability [personal notes]  
\begin{itemize}
	\item Model: for the $i$-th individual, let $y_i$ be its phenotype, we have: 
	\begin{equation}
	y_i = \mu + u_i + \epsilon_i
	\label{eq:random_effect_model}	
	\end{equation}
	where $u_i$ is the genetic effect (breeding value) and $\epsilon_i$ the environmental effect. Notes that the equation ignores the dominance term. To compute the covariance of two subjects: $\Cov(y_i, y_j) = \Cov(u_i, u_j)$, we first note that this covariance is twice of the covariance from alleles. Next, for a pair of alleles, its covariance is 0 if independent, and $\sigma_a^2$ if IBD. Let $\phi_{ij}$ be the probability of IBD, so we have: 
	\begin{equation}
	\Cov(y_i, y_j) = \Cov(u_i, u_j) = 2 \phi_{ij} \sigma_a^2
	\end{equation}
	The kinship is given by: $\phi_{ij} = P_1 / 4 + P_2 / 2$. When $i = j$, we have: 
	\begin{equation}
	\Var(y_i) = \sigma_a^2 + \sigma_e^2
	\end{equation}
	In the matrix form, we have the covariance matrix: 
	\begin{equation}
	\Var(y) = 2 \sigma_a^2 \Phi + \sigma_e^2 I 	
	\end{equation}
	
	\item Estimating variances and heritability: when we have multiple individuals, then we can fit the MVN to the data, where the covariance matrix is given above. When the relationship among individuals are known, the kinship matrix is given, and one only need to estimate $\sigma_a^2$ and $\sigma_e^2$ terms. When the kinship matrix is unknown, one needs to estimate it first. 
	
	\item Remark: the derivation assumes one source of breeding values. But we can show that when we have multiple sources, the same results hold. In this case, the expected genetic relationship $\phi_{ij}$ between any two individuals is the same across all loci, and $\sigma_a^2$ is the sum of contribution of all loci. 
	
	\item Remark: this result is based on expected genetic sharing, and on known pedigree; in the general case of population design, we can still use similar idea, but the kinship matrix would have different interpreatations. 
\end{itemize}

Heritability and polygenic scores/effect sizes [personal notes]
\begin{itemize}
	\item Polygenic scores: defined as the sum of estimated sizes of all variants present in a sample. Polygenic scores can be used for a number of applications, e.g. estimating heritability, disease risk prediction, and so on. It is closely related to the mixed model approach to heritability. 
	
	\item Intuition: we are interested in variance of trait explained by genetics (Proportion of variance explained, or PVE). This is similar to $R^2$ in regression, and it is related to the effect size estimates. Intuitively, with higher heritability, we should see many variables or large effect sizes; conversely, if effect size estimates are close to 0, then heritability is low. 
	
	\item Heritability in terms of effect size estimates: assuming we have $n$ independent markers, then using the results from [Relationship between $R^2$ and regression coefficients, Statistics notes], we have:
	\begin{equation}
	\hat{\sigma_a}^2 = \text{SSR} = \sum_j \hat{\beta_j}^2 \Var(x_j) \qquad \hat{\sigma_e}^2 = \text{SSE} = \hat{\sigma}^2 = \Var(\hat{\beta}) (X^T X)
	\end{equation}
	This leads to an estimate of $h_a^2$.  
	
	\item Comparison of random effect model based estimate vs. summary statistics based estimate: both are based on the idea that if heritability is high, similar genotypes should lead to similar phenotypes. The difference is: (1) LMM uses implicity genotype similarity, measured by kinship coefficient; (2) effect size approach uses the explicity genotypes. 
	
	\item Remark: this estimate represents only chip-heritability (explained by tagged SNPs), not narrow-sense heritability. 
\end{itemize}

Heritability in the genomics era - concepts and misconceptions [Visscher \& Wray, NRG, 2008]
\begin{itemize}
	\item Why most heritability is based on additive? Most relatives share only 1 or 0 IBD, thus dominance that is based on sharing two copies is not relevant. 
	
	\item Heritability is not constant. Genetic variance depends on segregation in a population of the alleles that influence the trait, the allele frequencies, the effect sizes of the variants and the mode of gene actions. All these variables can differ across populations. 
	
	\item Comparison of heritability of traits? Ex. morphological vs. fitness traits (higher heritability). Heritabilty also higher in more favorable environments. 
	
	\item Why so much genetic variance is additive and why h2 is so large? Theory predicts that additive genetic variance should be depleted because of natural selection, and biology tells us that genes work in interactive pathways, which implies non-additive interaction variance. Possible answers: strong interactions not a problem if gene frequency is near 0 or 1. 
\end{itemize}

A Tool for Genome-wide Complex Trait Analysis (GCTA) [personal notes; \textsf{Yang }and Visscher, AJHG, 2011]
\begin{itemize}
\item Motivation: In the general case, we may not know the pedigree relationship; furthermore, even with pedigree, the actual genetic similarity (realized genetic sharing) may not be equal to the expected similarity inferred from IBD; so we will need to derive the heritability in terms of the realized genotypes. 
	
\item Theoretical framework: linear mixed model (LMM). For any indvidual, its trait: 
\begin{equation}
y_i = X_i \beta + Z_i u + \epsilon_i
\end{equation}
where $X_i$ is the vector of fixed effects, including covariates such as age and ex and known genotypes of causal variants, and $Z_i$ the genotypes of causal variants(standardized with variance equal to 1). The coefficients $\beta$ and $u$ represent the effect sizes. We makes the assumption that $u_j$ of the $j$-th variant is random: $u_j \sim N(0, \sigma_u^2)$. To do LMM analysis, we first subtract the fixed effects, ans assuem for now that we only deal with the remaining terms. The covariance between two individuals is: 
\begin{equation}
\Cov(y_i, y_k) = \Cov(Z_i u, Z_k u)
\end{equation} 
We expand the genetic loci: 
\begin{equation}
\Cov(Z_i u, Z_k u) = \sum_j \Cov(Z_{ij} u_j, Z_{kj} u_j) = \sum_j Z_{ij}  Z_{kj} \Cov(u_j, u_j) = Z_i Z_k^T \sigma_u^2
\end{equation}
where we uses the random effect of $u_j$: $\Cov(u_j, u_j) = \sigma_u^2$. When $i = k$, we have:
\begin{equation}
\Var(Z_i u) = Z_i Z_i^T \sigma_u^2
\end{equation} 
This allows us to write the covariance matrix of $y$:
\begin{equation}
\Cov(y) = Z Z^T \sigma_u^2 + I \sigma_e^2
\end{equation} 
This is the basic relation between covariance of traits and the effect size distribution. The model is also called ``variance component'' model, because the covariance of phenotypes is partitioned into two parts, one from genetic variation, and the other from environmental variation (or more generally, across-group variation and within-group variation, where group means genotype here). 

\item Alternative derivation: write in the vector form, $Y = X \beta + Z u + \epsilon$, our prior is $u \sim N(0, I \sigma_u^2)$. Ignoring $X\beta$, and treat $u$ as random and $Z$ fixed, we use the result from MVN: 
\begin{equation}
\Var(Y) = \Var(Zu) + I \sigma_e^2 = Z (I \sigma_u^2) Z^T + I \sigma_e^2 = Z Z^T \sigma_u^2 + I \sigma_e^2
\end{equation}

\item Remark: the interpretation is, suppose we have given genotypes, but our effect sizes $u$ are random, then on average, what do we expect about the relationship of phenotypes of two individuals? 

\item Estimating heritability: to relate $\sigma_u^2$ to heritability, we note that $Z_i u$ is a linear combination of a random vector, so we can obtain its variance (i.e. the variance explained by SNPs): 
\begin{equation}
\sigma_a^2 = \Var(Z_iu) = Z_i Z_i^T \sigma_u^2 = \sigma_u^2 \sum_j Z_{ij}^2 \approx m \sigma_u^2
\end{equation}
where we use the fact that $Z_{ij}$ is normalized s.t. its variance is equal to 1. To understand this, we could derive this by using the basic relationship between RSS (explained variance) and effect size from regression model: 
\begin{equation}
V_A = SSR = \sum_j u_j^2 \Var(Z_j) = \sum_j u_j^2 \approx m \sigma_u^2
\end{equation}
where we uses the fact that $\E(u_j^2) = \Var(u_j) = \sigma_u^2$ and $m$ is the number of causal SNPs. Note that this derivation makes the assumption that SNPs are independent, which is not actually needed. We define the genetic relationship matrix (GRM) $A$ as: 
\begin{equation}
A_{ik} = \frac{1}{m} \sum_{j=1}^m \frac{(x_{ij}-2p_j)(x_{kj}-2p_j)}{2p_j(1-p_j)}
\end{equation}
where $x_{ij}$ and $x_{kj}$ are raw genotypes. Then $A = Z Z^T / m$, so we can write covariance matrix: 
\begin{equation}
\Cov(y) = A \sigma_a^2 + I \sigma_e^2
\end{equation}
where $\sigma_a^2 = V_A$ is the additive genetic variance. So in practice, we first estiamte $A$ from data, then fit the data (covariance) to estimate $\sigma_a^2$ and $\sigma_e^2$. 

\item Chip heritability and related individuals: in the model, we use only genotyped SNPs, and the heritability is those explained by these SNPs, hence called chip-heritability. To see this, we note that the model is correct if the covariance matrix from causal SNPs ($ZZ^T$) is equal to that estimated from genotyped SNPs ($A$). This is OK if the samples are independent. However, when there are related subjects, even if we do not tag all the causal variants, we may still estimate the matrix $Z Z^T$ (only need a fraction of variants to estimate relatedness). Then we cannot say that the heritability is explained by our genotyped SNPs. So we should remove close relatives in the analysis (e.g. 0.025 as cutoff).    
 
\item Partition of genetic variance: suppose we partition our SNPs by groups, e.g. chromosomes, then each group explains a small amount of genetic variance ($\sigma_g^2 = m \sigma_u^2$, so roughly it is proportional to the number of SNPs). To estimate each of them, we have:
\begin{equation}
\Cov(y) = \sum_i A_i \sigma_i^2 + I \sigma_e^2
\end{equation}
where $\sigma_i^2$ is the contribution of the $i$-th group and $A_i$ is the GRM estimated from the SNPs in this group. The model is generally identifiable since $A_i$'s would be different in different pairs: e.g. when estimating $\sigma_1^2$, we use the pairs with the highest $A_1$. 

\item The impact of LD (personal notes): GCTA does not require independence of SNPs, as it accounts for all SNPs simultaneously. In the two key steps in the derivation: $\Var(Y)$ in terms of $\sigma_u^2$, and $\sigma_a^2 = m \sigma_u^2$, neither requires independence. LD only becomes a problem when causal variants and non-causal variants have different LD and MAF.  
\end{itemize}

Estimation and partition of heritability in human populations using whole-genome analysis methods [Vinkhuyzen \& Visscher, ARG, 2013]
\begin{itemize}
\item History: Galton, resemblance of relatives. Fisher’s theory: reconcile two schools. BLUE and BLUP for estimation and prediction in animal and plant breeding. 
	
\item Three designs: contrasting individuals of different genetic relatedness. All these methods share a common conceptual framework: for genetically similar individuals, how similar they are phenotypically. The difference lies on how genetic similarity is defined: expected or realized; family or population. 
\begin{itemize}
\item Twin design: use comparison of MZ twins and DZ twins. MZ twins should be more similar than DZ twins, if heritability is high.
\item With-family design: use full-siblings.  
\item Population design: genetically similar individuals from population (due to population history, they may share part of the genome, even if they are unrelated). 
\end{itemize}

\item Genetic relationship matrix ($G$): the accuracy of estimation methods depend on the estimateion of $G$. In general, the sampling variance of $G$ is small for close relatives, large for distant ones (Figure 1). However, bias is more of a problem for close relatives (confounding with environment).

\item Population design: we first need to estimate Genetic relationship matrix (GRM) - the matrix $G$ in the LMM framework. The term $G_{ik}$ is the average genetic relationship (covariance) of causal variants. If we assume the causal variants are indistinguishable from genetic markers (in terms of genetic relateness), then a simple estimation of $G$ as the average correlation of genotypes: 
\begin{equation}
G_{ik} = \frac{1}{m} \sum_{j=1}^m \frac{(x_{ij}-2p_j)(x_{kj}-2p_j)}{2p_j(1-p_j)}
\end{equation}
where $p_j$ is the AF of the marker $j$. The estimation can be improved by using IBD inferred from haplotypes. Once we have $G$, we use ML of phenotypes or Equation~\ref{eq:cov_decomp}. The heritability obtained in this way is determined by the tagged SNPs in a chip, thus called ``chip heritability''. 
\begin{itemize}
	\item Closely related relatives: should be removed. Shared environment.
\end{itemize}	

\item Comparison of designs:
\begin{itemize}
\item Pedigree or family design: estimated heritability has small sampling variance, but bias (due to environmental confounding) is a seriour concern. 

\item Population design: large samples so could obtain small sampling variance; and less bias (indpendent samples, so no shared environment). In fact, population design uses all pairwise comparison: so even if each pair carries relatively little information, the number of pairs is large (quadratic).  
\end{itemize}

\item The impact of LD: 
\begin{itemize}
	\item If causal variants not in LD with tag SNPs, matrix $G$ is not the same as the true $G$. Methods were proposed to weigh SNPs based on LD. 
	
	\item Population design: estimation of $h^2$ is driven by LD between casual and tag SNPs; or by distant IBD between unrelated individuals. The two descriptions are equivalent - causal variants not tagged will not contribute. For family design: estimation is driven by shared IBD in relatives – estimate total $h^2$ because markers track all causal variants.  
\end{itemize}

\item Example: height.
\begin{itemize}
\item Pedigree design: about 80\%
\item Significant GWAS loci: 10\%
\item Using estimated genetic similarity from all SNPs: 50\%. The implication is that SNPs with small effects remain undetected. Incomplete LD with GWAS SNPs explain the gap (using simulation). 
\item We should call ``hidden heritability'' instead of ``missing heritability''. 
\end{itemize}

\item Future directions:
\begin{itemize}
\item Analysis of multiple phenotypes: Genetic covariance can be estimated from subjects where only one phenotype is measured. 

\item The properties of causal variants as a class: effect sizes, allele frequency spectrum. This can help understand the evolutionary process and guide experiment design. 
\end{itemize} 

\item Discussion: (Dan Nicolae) environmental effects in population design. When we use PCA to remove the effects of population substructure, do we also remove the true genetic effects, thus underestimate heritability? 
\begin{itemize}
\item Thought: PCs are surrogates of culture, ethnicity, and so, so we are removing the environmental effects, not the genetic ones. However, different ethnicities are genetically distince, so we do remove some common genetics. The idea is that we remove the main difference between ethnicities, but use the remaining the genetic variations to estimate heritability (i.e. controlling/stratfying ethnicity). 
\end{itemize}

\item \textbf{Question}: using CVs can estimate G, but does it mean we ``explain'' the variation of traits? Even if we miss many causal variants, we may still be able to estimate G well. Ex. imagine we are applying the method to family data, then we don't need all SNPs to estimate IBD. 
\end{itemize}

Advantages and pitfalls in the application of mixed model association methods [Yang \& Price, NG, 2014]
\begin{itemize}
\item Applications of LMM: include
\begin{itemize}
	\item Estimation of $h^2$ due to common variants. 
	\item Control for genetic background (even if the samples are unrelated) to increase power. 
	\item Correct for population substructure and cryptic relatedness.
	\item Prediction of phenotypes.  
\end{itemize}
The paper focuses on the first three applications, and in particular, how different choices of using LMM affect the three issues (estimation of $h^2$, power and control for population substructure). 

\item Correct for population substructure: the idea is that the control of genetic background removes the effect of relatedness. In practice, this also controls for geographical population structure (controlling for the ancestry markers is similar to control for PCs). 

\item Computational costs of LMM: three steps, building GRM, estimating variance components and compute association statistics for each SNP.  Different computational strategies
\begin{itemize}
	\item Exact strategy: compute the variance components for each candidate marker being tested. Exists more efficient methods of doing this. 
	\item Computing variance components only once and use them for all markers: OK if all markers have small effects. The difference between exact and approximate strategies is large with pervasive relatedness and large effect sizes. 
\end{itemize}

\item Issue 1: including the candidate marker (MLMe) or not (MLMi). Mathematically, one should use MLMe, as MLMi control for the candidate while testing its effect. Use of MLMi would lead to a lower power: this can be shown by studying the mean association statistics $\lambda$ under different models. Using only linear regression: 
\begin{equation}
\lambda_{mean}(LR) = 1 + N h^2 / M
\end{equation}  
where $N$ is sample size and $M$ the number of markers. This is bigger than 1 due to polygenecity, however, when $N$ is not too big, it is close to 1. For MLMi, 
\begin{equation}
\lambda_{mean}(MLMi) = 1
\end{equation}
For MLMe:
\begin{equation}
\lambda_{mean}(MLMe) = 1 + \frac{N h^2}{(1 - r^2 h^2)M}
\end{equation}
where $r^2 \approx Nh^2/M$. The difference between MLMe and MLMi can be understood as testing different $H_0$: a SNP has no effect (MLMe); or a SNP's effect is explained by the random effect $N(0, \sigma_u^2)$. 

\item Issue 2: use a subset of markers. Conceptually, one can use a subset of markers to estimate $\sigma_u^2$, thus speed up the computation. Two ways of selecting markers: top $M_T$ markers, or random $M_R$ set of markers. Whether use one of the two (or all markers) depends on the two (somewhat competing) goals: (1) increase the power by controlling for genetic background; (2) correct for population substructure. Specifically: 
\begin{itemize}
	\item If the goal is to increase the power, choosing top $M_T$ markers is OK, and the value of $M_T$ can be chosen by maximizing the (out-of-sampe) prediction accuracy. 
	
	\item Choose a small $M_T$ or $M_R$ however may be insufficient to control for population substructure. 
\end{itemize}

\item Issue 3: ascertained case-control data. The control of genetic background may lead to loss of power in ascentained case-control samples, when the disease prevalence ($f$) is low. 
\begin{itemize}
\item Intuition [The Covariate Dilemma, PLoS Genetics, 2012]: suppose we have a non-confounding covariate (independent of test marker, but correlate to the trait), then include it may reduce power. Explanation: when $f$ is small, the test marker and the variable may become correlated in cases, so controlling for the variable also removes some of the effect of the test marker, reducing the power. 
\end{itemize} 

\item Recommendations: 
\begin{itemize}
	\item Excluding candidate markers (MLMe) in preference to including them (MLMi). 
	\item For randomly ascertained quantitative traits: general include all markers. When population stratification is not a significant concern, could use top markers. 
	\item Genome-wide significant markers should be conditioned out as fixed effects. 
\end{itemize}

\item Directions: 
\begin{itemize}
	\item Distinguishing between polygenic effects and incomplete correction for stratification (both lead to $\lambda_{mean} > 1$). 
	\item MLM methods for ascertained case-control data. 
	\item Use better prior model of effect sizes: mixture distribution. 
\end{itemize}

\item Questions: if large effect from population structure, then the effects of population ancestry markers are large. Including them as fixed effect should be more powerful? 
\end{itemize}

Concepts, estimation and interpretation of SNP-based heritability [Yang and Visscher, NG, 2017]
\begin{itemize}
	
	\item GREML (GCTA): not confounded by environment and epistasis. 
	
	\item LD in GREML: accounted for because all SNPs are fit together. LD pruning is thus unnecessary; with LD pruning, possible that the MAF spectrum of SNPs changes. 
	
	\item Bias due to LD or MAF: difference in LD and MAF of causal vs. non-causal variants can affect. Recommended GREML-LDMS: stratify by LD and MAFs and estimate h2 separately, at the cost of more parameters. LDAK: (1) Earlier version: implicitly RVs explain 10 times more variance than CVs. (2) Later version: more similar to GREML-LDMS. 
	
	\item Under neutral model: variance explained by MAF bins is proportional to the MAF bin size. Can use this to test for negative selection. 
	
	\item LDSC: sensitive to genetic architecture, cannot estimate contribution from rare variants, biased estimate due to incorrect LD matrices. 
	
	\item Simulation study of LDSC: one major gene explains half of PVE, the rest SNPs other half (true PVE = 0.5). Sample size 13K and 500K SNPs. Both GREML and HE regression give estimated PVE close to 0.5, but LDSC gives 0.37. Possible explanation: for all other SNPs, their relationship of effect sizes and LD scores can be fit by a line with slope 0.25. Having a single large-effect locus may not be able to overcome the bias in the slope. 
\end{itemize}

Improved Heritability Estimation from Genome-wide SNPs [Speed and Balding, AJHG, 2012]
\begin{itemize}
	
	\item GCTA model makes several assumptions: polygenicity, normal prior, the relationship of effect size and MAF, and independence of LD (each SNP makes equal contribution regardless of LD pattern). 
	
	\item Simulation to investigate how these assumptions may affect the estimated $h^2$ for GCTA: real genotypes from 2,500 cases and 2,500 controls, choose causal variants (different scenarios), and simulate phenotypes. The true $h^2$ is 0.5 or 0.8. 
	
	\item Modeling effect of LD on $\hat{h}^2$: let $w_j$ be the weight of variant $j$, choose $w_j$ s.t. $w_j + \sum_{j’} w_{j’} r_{jj’}^2 e^{-\lambda d_{jj’}}$ is constant over $j$ where $d_{jj’}$ is the distance between variants. The motivation is that the sum represents the total amount of tagging in a SNP. To implement this weighting, standardize genotype by changing $X_j$ (SNP j) to $\sqrt{w_j} X_j$. Then update the kinship matrix in GCTA.  
	
	\item Polygenicity: different numbers of causal variants from 1 to ALL. In almost all cases, the estimated $h^2$ is unbiased. However, when number of causal SNPs is small, GCTA estimates of standard error of $h^2$ is too low. The problem is fixed with weighted kinship matrix. 
	
	\item Relationship of effect size and MAF: $\Var{\beta_j} \propto [p_j (1-p_j)]^{\alpha}$, set $\alpha$ from -2 to 1. The results are relatively robust to the value of $\alpha$ used in the model (scaling genotype). 
	
	\item Normal prior: simulation under normal exponential gamma (NEG) prior. The results are robust. 
	
	\item Impact of LD on $\hat{h}^2$: (Figure 3) when causal variants are in weak LD regions, we underestimate contribution to $h^2$; in high LD regions, we overestimate contribution to $h^2$. This can be fixed with weighted kinship matrix. 
	
	\item Why pattern of LD matter? It is strongly linked to MAF: the signals from low-MAF variants are less replicated (tagged), comparing with high-MAF variants. So in diseases where rare variants dominant (e.g. bipolar disorder), we tend to underestimate $h^2$ and in diseases where common variants and/or high-LD regions (e.g. MHC for AIDs), we tend to overestimate $h^2$. 
\end{itemize}

Contrasting genetic architectures of schizophrenia and other complex diseases using fast variance-components analysis (BOLT-REML) [Loh and Price, NG, 2015]
\begin{itemize}
	\item Background: methods to fit variance component model, include:
	\begin{itemize}
		\item First-derivative methods: gradient descent, e.g. EM. 
		\item Second-derivative methods: calculation or approximation of Hessian matrix. While faster than first derivative methods, they are less robust when far from the optimum. Newton-Raphson (NR) method: Zhou and Stephens use faster method based on EVD. Average information (AI): average of Hessian and Fisher information matrix.
	\end{itemize}
	Also Monte Carlo methods: applied to both EM and second derivative. 
	
	\item Model: write the model as: 
	\begin{equation}
	y = \sum_k \sigma_k Z_k u_k + \sigma_0 u_0
	\end{equation}
	where $Z_k$ are normalized genotype, $u_k$ are normalized random effects, and $\sigma_k$ variance component parameters. The variance of $y$ is given by:
	\begin{equation}
	V = \sigma_0^2 I_n + \sum_k \sigma_k^2 Z_k Z_k^T
	\end{equation}
	This leads to the log-likelihood function:
	\begin{equation}
	l(\sigma_0^2, \sigma_1^2, \cdots, \sigma_K^2) = -\frac{1}{2} (\log \det V + y^T V^{-1} y)
	\end{equation}
	
	\item Computing gradient and Hessian: (1) Monte Carlo approximation of graident: MC REML. The gradient $\partial l / \partial \sigma_k^2$ is a function of $Z_k^T V^{-1} y$. This is BLUP estimates of $u_k$, effect sizes. In computation, replace expectation with Monte Carlo samples. (2) Approximation of Hessian: Average Information (AI). 
	
	\item Optimization: (1) Trust region methods: find regions where local quadratic model of log-likelihood breaks down. They can be detected by comparing true vs. approximate log-likelihood. (2) Convergence: MC AI REML truly converges rather than jump around parameters. 
	
	\item Standard errors: note that heritability and genetic correlations need to be scaled. 
\end{itemize}

LD Score regression distinguishes confounding from polygenicity in genome-wide association studies (LDSC) [Bulik-Sullivan and Neale, NG, 2015]
\begin{itemize}
	\item Model: let $\chi_j$ be the chi-square of SNP $j$, $l_j = \sum_k r_{jk}^2$ be the LD score of SNP $j$. We have this equation: $E(\chi_j^2) = N l_j h^2 / M + Na + 1$, where $h^2/M$ is the per SNP heritability and $a$ the effect due to population stratification or other confounding variables.
	
	\item Derivation from random effect sizes (personal notes): one possibility is to use RSS and with normal prior of effect sizes. Another way (equivalent) is: consider the normalized effect size, $u$ and estimated effect $\hat{u}_j$ for SNP $j$. We are interested in the chi-square statistic, which is just variance of $\hat{u}_j$. We use Law of Total Variance, treating $u$ as random: 
	\begin{equation}
	\Var{\hat{u}_j} = \E_u [\Var(\hat{u}_j | u)] + \Var_u [\E(\hat{u}_j | u)]
	\end{equation}
	The first term is simple: with standardized effect, its se = 1, so $\Var(\hat{u}_j | u) = 1$, and its expectation is 1. The second term, we have $\E(\hat{u}_j | u) = (Ru)_j$, where $R$ is the LD matrix. We consider the variance: 
	\begin{equation}
	\Var(Ru) = R \cdot \Var(u) \cdot R^T = \sigma_u^2 R I R^T = \sigma_u^2 R^2
	\end{equation}
	We now have: $\Var[(Ru)_j] = \Var(Ru)_{jj} = \sigma_u^2 R^2_{jj}$.Let $l_j = (R^2)_{jj}$ be the LD score of SNP $j$. So we have: 
	\begin{equation}
	\Var{\hat{u}_j} = 1 + \sigma_u^2 l_j
	\end{equation}
	Now plug in $h^2 = \sigma_u^2 M / N$, where $M$ is number of SNPs and $N$ sample size. This gives the LD score regression results. 
	
	\item Derivation in the paper: also use Law of Total Variance. However, instead of treating effect size as given, it treats $X$ (genotype) as given: 
	\begin{equation}
	\Var(\hat{\beta}_j) = \Var_X [\E(\hat{\beta}_j | X)] + \E_X [\Var(\hat{\beta}_j | X)]
	\end{equation}
	It then uses the random effect size in the equation.  
		
	\item Remark: the LD score of a SNP can be thought of the effective number of SNPs tagged by that SNP. Under polygenicity assumption, the more SNPs tagged, the higher observed effect the SNP should have. 
	
	\item Regression estimation: we regress the chi-square of each SNP with its LD score. The slope would suggest per SNP heritability and intercept the confounding effect. In practice, define LD scores in 1cM blocks. 
	
	\item Two statistical issues: (1) SNPs are not independent; (2) Variance of the chi-square different: SNPs with high LD scores have higher variance. To address this, use weighting of SNPs by $1/l_j$.
	
	\item Obtaining standard error: knife procedure, each time removing a block of 2,000 SNPs and re-estimate. 
	
	\item Sensitivity of LDSC on various factors: 
	\begin{itemize}
		\item Low frequency variants: effects cannot be captured by LD scores. Thus they will drive up the slope. 
		\item Long-range LD: leads to underestimation of LD scores, and as a result, drive up the slope and/or intercept. 
		\item Very large effect variants: drive up the slope. ``SNPs with very large effect sizes can result in large LD Score regression standard errors with an unconstrained intercept'' because ``linear regression deals poorly with outliers in the response variable'' (from the cross-trait LDSC paper). 
	\end{itemize}
	All these SNPs should be filtered. 
	
	\item Sensitivity of LDSC on LD matrices: (1) Difference of ref. and target LD is just noise (mean 0): increase intercept, and reduce slope. (2) Systematic difference: if LD in ref is smaller than LD in target on average, the intercept will be upward biased. 
	
	\item Interpreting LD score results: Generally, if intercept is close to 0, no confounding. (Table 1) for the null SNPs, their mean $\chi^2$ is the intercept term. It is very close to $\lambda_{GC}$ when all SNPs are null (in null simulations). If the mean $\chi^2$ of all SNPs is much larger than the intercept, it means that the results are mostly driven by polygenecity rather than confounding. 
	
	\item Simulation on polygenic architecture: unrelated cohort of 1,000 Swedes. Intercept close to 1, and the estimates of h2 is unbiased at different levels of causal proportions. However, when the proportion is very low, the s.e. becomes very large.  
	
	\item Simulation on confounding only: a useful result, under null model with population structure, the mean $\chi^2$ is given by:
	\begin{equation}
	\bar{\chi}^2 = 1 + b N F_{ST}
	\end{equation} 
	where $b$ is the correlation of phenotype and ancestry, $N$ is the sample size. To simulate confounding: 
	\begin{itemize}
		\item Continent level: choose one cohort as cases and another controls. 
		\item Country level: obtain PCs, and use first three PCs as phenotypes.  
	\end{itemize}
	In simulations, $\lambda_{GC}$ is large, and is close to intercept, but the slope is close to 0. 
	
	\item Simulation on polygenic and confounding: partition the chromosomes, with the first half containing causal SNPs, and the second half only null. Then simulation phenotypes: causal SNPs and environment correlated with PCs. 
	
	\item Intuition of LDSC is PCs already capture LD, so SNPs in high LD vs. low LD regions are “equal” in PC-space. So their inflated effects are independent of LD.
	
	\item Remark: LDSC model holds even if prior effect sizes are not normally distributed. 
\end{itemize}

Relationship between LD Score and Haseman-Elston Regression [Brendan Bulik-Sullivan, BiorXiv, 2015]
\begin{itemize}
	\item HE regression: the covariance between two individuals depends on heritability and their genetic relationship: 
	\begin{equation}
	\E(y_h y_i | X) = h_g^2 A_{hi}
	\end{equation}
	where $A = X X^T / M$ is the GRM. This leads to a closed form estimator of $h_g^2$, which is the sample $\Cov(y_h y_i, A_{hi})$ divided by sample $\Var(A_{hi})$.   
	
	\item Equivalence of HE regression and LDSC: the idea is that the numerator of the HE estimator can be related to the marginal SNP effect, and the denominator expressed as LD. So the HE regression is equivalent to LDSC with intercept constrained to 1 and regression weights $1/l$. 
	
	\item Possible to incorporate fixed effects. 
	
	\item Simulation results: $h^2 = 0.5$ on chr 2, 2000 samples. REML: no bias, SD = 0.05. LDSC with no intercept: bias = 0.02, SD = 0.06 (or MSE = 0.063). LDSC with intercept: bias = 0.02, SD = 0.09. 	 	
\end{itemize}

Contrasting the Genetic Architecture of 30 Complex Traits from Summary Association Data (HESS) [Shi and Pasaniuc, AJHG, 2016]
\begin{itemize}
	\item Treating effect sizes as fixed: e.g. suppose we are interested in $\sum_j \beta_j^2 = \beta^T \beta$, how can do estimate that when our power of estimating individual $\beta_j$ is low? In general, we avoid estimation of individual $\beta_j$, but construct an estimator of $\beta^T \beta$. 
	
	\item Strategy for estimating local heritability: treating genotype as random and effect sizes fixed. Given $y = X \beta$, we have: 
	\begin{equation}
	\Var(y) = \beta^T \Cov(X) \beta + \sigma_e^2 = \beta^T V \beta + \sigma_e^2
	\end{equation}
	where $V$ is the LD matrix. Assuming $\Var(y) = 1$, then $\beta^T V \beta$ is the heritability. Our problem is then to estimate this function defined on $\beta$ with observed effect sizes and LD matrix. 
	
	\item Estimator of local heritability: first obtain that $\hat{\beta} \sim N(V \beta, V (1 - h^2)/n)$, where $n$ is sample size. This leads to MOM estimator:
	\begin{equation}
	\E(\hat{\beta}) = V \beta \Rightarrow \hat{\beta}_{\text{MOM}} = V^{-1} \hat{\beta} \Rightarrow (\beta^T V \beta)_{\text{MOM}} = (V^{-1} \hat{\beta})^T V (V^{-1} \hat{\beta}) = \hat{\beta} V^{-1} \hat{\beta}
	\end{equation}
	However, its expectation is not exactly $\beta^T V \beta$. With some correction, we obtain the unbiased estimator and its variance, Equations (5) and (6) in the paper. 
	
	\item Dealing with LD matrix rank deficiency: Psuedo-inverse, $V^{pi}$, which effectively consider the $q$ eigenvectors of the EVD of the LD matrix, if the rank is $q < p$, where $p$ is the number of SNPs.   
	
	\item Regularization of LD matrix when it's obtained from external reference samples: we perform EVD of the LD matrix, and the estimator can be rewritten in terms of the eigendecomposition as: 
	\begin{equation}
	\hat{\beta}^T V^{pi} \hat{\beta} = \sum_{i=1}^q (1/w_i) (\hat{\beta}u_i)^2
	\end{equation}
	where $w_i$ and $u_i$ are eigenvalues and eigenvetors of $V$ respectively. The interpretation is: it is the weighted sum of the square of projected effect sizes along eigenvectors. The term is dominated by the first $k$ eigenvectors, so to regularize, consider only $k$ eigenvectors for LD derived from reference samples. 
	
	\item Simulation: 50K samples, chr. 22. When using external reference, biased estimator (as in LDSC), but $k = 30-50$ is optimal. When $k$ is small, generally underestimate $h^2$: intuitively, we fail to capture all the true effects in a LD block. 
	
	\item Discussion: reference-based LD, often estimated from smaller number of individuals, often miss subtle information in in-sample LD, so have lower ranks.  
	
	\item \textbf{Lesson}: possible to deal with over-parameterization in frequentist statistics, by considering the problem of estimating some function defined over the parameters.
	
	\item \textbf{Lesson}: LD projected effect sizes. Ex. suppose we have two SNPs in strong LD, $\hat{\beta}_1$ and $\hat{\beta}_2$ would be both large and highly correlated. Now we project on eigenvectors (rotating the coordinates), then the second direction has small projected effect. Its contribution to heritability would be low.  
\end{itemize}

A united framework for variance component estimation with summary statistics in genome-wide association studies (MQS) [Xiang Zhou, AAS, 2017]
\begin{itemize}
	\item Motivation: REML is slow, require summary statistics and lead to down-ward bias in case-control data. 
	
	\item Model and notation: let $X_i$ be the genotype of all SNPs in category $i, 1 \leq i \leq k$ and $y$ be phenotype. We assume they are all centered to mean 0. Our model of phenotype is:
	\begin{equation}
	y = \sum_{i=1}^k X_i \beta_i + \epsilon \qquad \epsilon \sim N(0, \sigma_{k+1}^2 M)
	\end{equation}
	where $M = I - 1_n 1_n^T/n$. The effect size $\beta_i \sim N(0, \sigma_i^2 I / p_i)$, where $p_i$ is the number of SNPs in category $i$. With this model, we can marginalize $\beta_i$'s. Let $g_i = X_i \beta_i$ be the total contribution of SNPs in category $i$: 
	\begin{equation}
	g_i \sim N(0, \sigma_i^2 X_i X_i^T / p_i) = N(0, \sigma_i^2 K_i)
	\end{equation}
	where $K_i = X_i X_i^T / p_i$ is the $n \times n$ GRM from SNPs in the category $i$. 
	
	\item MINQUE estimator: we have $k+1$ parameters $\sigma^2$ to estimate. Using MOM, we have, for any matrix $A_j$: 
	\begin{equation}
	\E(y^T A_j y) = \sum_{i=1}^k \tr(A_j K) \sigma_i^2 + \tr(A_j) \sigma_{k+1}^2
	\end{equation}
	It is easy to prove this equation using the result about quadratic form of random vectors. Note that RHS is amount of variation or covariance of $y$'s (scalar). Intuitively, we can choose $A_j$ to ``extract'' correlation of any two pairs of samples, then we have MOM estimation that matches sample covariance with expected covariance. 
	
	\item Choosing $A_j$ matrices: we choose $A_j$ to minimize squared error (MINQUE criterion). The optimal $A_j$ is given by Equation (5), however, it depends on the values of $\sigma^2$, which are unknown. Possible strategy: iterative update I-MINQUE, which is equivalent to REML.  In order to use summary statistics, use $A_j$ of the form: 
	\begin{equation}
	\tilde{A}_j = X_j W_j X_j^T / p_j \qquad \tilde{A}_{k+1} = M
	\end{equation}
	where $W_j$ is pre-specified $p_j \times p_j$ diagonal matrix. To simplify algebra, scale $W_j$ s.t. the average weight is 1. Depending on the choice of $W_j$, this leads to either HE regression or LDSC. 
	
	\item Equivalence to Hazeman-Elston regression: we choose $W_j$ be identity matrix (equal weights to all SNPs). This leads to HE regression. Now we have $\tilde{A}_j = X_j X_j^T / p_j = K_j / p_j$. We have:
	\begin{equation}
	y^T K_j y = \sum_j = \sum_j \sigma_j^2 \tr(K_j^2) + \sigma_{k+1}^2 \tr(K_j)
	\end{equation}
	Suppose $k = 1$, then LHS is $y^T K Y = \sum_{i,j} K_{ij} y_i y_j$, and RHS is $\sigma_1^2 \tr(K^2) + \sigma_2^2 \tr(K)$. When phenotypic covariance matches genetic covariance, we have large value of $y^T K y$, and this leads to large value of $\sigma_1^2$. Another way to understand this is: (ignoring constant $p_j$), we have
	\begin{equation}
	y^T \tilde{A}_j y = y^T X_j W_j X_j^T y = (X_j^T y)^T W_j (X_j^T y) = Z_j^T W_j Z_j
	\end{equation}
	where $Z_j$ measures covariance of $X_j$ and $y$ (Z-scores). When $W_j = I$, this term is now $Z_j^T Z_j$, which is the sum of Z-scores of all SNPs in category $j$. So if this value is large, it suggests that $\sigma_j^2$ is large. 
	
	\item Estimator and confidence interval: we can solve the linear system analytically and the solution is given by $\hat{\sigma}^2 = S^{-1} q$, where $S$ is roughly LD matrix, and $q$ is $k$-dim. vector. See Equation (13) and (14). It is easy to obtain the variance of $\hat{\sigma}^2$ as a function of $V(q)$ and $S$. Use approximation to compute $V(q)$, Equation (18). 
	
	\item Subsampling of $S$: exact computation of $S$ takes $O(pn^2)$ which dominates computation. Approximate $S$ using a subset of $m$ samples (or external panel). Possible to account for sampling error of $S$, however, it is generally much smaller than $V(q)$.  
	
	\item Using summary statistics: both $q$ and $S$ can be expressed as summary statistics: $Z$ scores of SNPs, and pairwise LD. Two strategies: (1) use only summary statistics: require some additional summary statistics; (2) block jackknife strategy in LDSC. However this can work poorly if blockwise independence is not satisfied.  
	
	\item Simulation results: comparison with BOLT-REML, LDSC, MQS-HEW (HE weighting) and MQS-LDW (LD weighting). LDSC not efficient. MQS-HEW and LDW are more efficient for small $h^2$ than large $h^2$, and works better when samples are independent. Type 1 error of testing if $\sigma^2 = 0$ is slightly inflated with normal asymptotic instead of mixture of chi-square. 
\end{itemize}

Reevaluation of SNP heritability in complex human traits (LDAK) [Speed and Balding, NG, 2017]
\begin{itemize}
	\item Background: under current models, GCTA and LDSC, heritability per SNP is constant. 
	
	\item Background: LDSC tend to have standard error 25-100\% higher because it has an extra parameters and it is moment-based. 
	
	\item Motivation of dependency of SNP heritability on LD: Figure 1. High LD regions, tag fewer causal variants, comparing with low LD regions. 
	 
	\item LDAK prior: let $h_j^2$ be the heritability of SNP $j$, we assume: 
	\begin{equation}
	\E(h_j^2) \propto [f_j (1-f_j)]^{1+\alpha} \times w_j \times r_j
	\end{equation}
	where $f_j$ is MAF, $w_j$ weight of SNP which penalizes SNPs in high LD regions, and $r_j$ genotype uncertainty. $\alpha$ is a parameter controlling the relationship of SNP heritability with MAF: constant effect means $\alpha = -1$.  
	
	\item Strategy for comparing priors: (1) Log-likelihood of data: compare with different $\alpha$ and different LD models. (2) SNP partitioning: into bins (100 Kb segments) of MAF and LD, then estimate the heritability of each bin with GCTA. 
	
	\item Effect of MAP on prior: Figure 2, LL of GWAS traits under different $\alpha$, for most traits, it is maximized around $\alpha = -0.25$. However, the impact of using $\alpha=-1$ on PVE estimation is small. 
	
	\item Effect of LD on prior: Figure 4, partition SNPs into low LD and high LD halves. Show that low LD half explains more than 50\% heritability and is more consistent with LDAK model than GCTA for most of the traits. Figure S12: generally, quantitative traits show less effect of LD on SNP heritability. 
	
	\item Results: in 19 traits, $h^2$ on average 43\% higher than GCTA. Also, DHS only explains 24\% heritability (instead of 79\%).   
\end{itemize}

Linkage disequilibrium–dependent architecture of human complex traits shows action of negative selection [Gazel and Price, NG, 2017]
\begin{itemize}
	\item Background: h2g enriched in DHS, etc, which have low LD. On the other hand, regions of low recombination rate, and high LD, are enriched with exonic deleterious variants (opposite).
	
	\item Levels of LD (LLD): low LLD regions have higher heritability (Figure 1). Use S-LDSC with continuous annotations.
	
	\item Explanation: driven by more recent common variants having lower LLD (positive correlation of age and LLD): the youngest 20\% of common SNPs explain 3.9 times more heritability than the oldest 20\%. For fixed set of variants, older alleles have low LLD (negative correlation), however, for new variants, we have positive correlation. Imagine a background of many variants, now new variants appear: the youngers one are generally in lower LLD with existing ones (positive correlation).
	
	\item Recombination and h2g: low recombination rates, selection less effective (BGS reduces population sizes), so variants more likely to be deleterious.
	
	\item Gazel LD annotations: MAF adjusted allele age, MAF adjusted LLD in African, and other LD related annotations, B-scores, nucleotide diversity, etc.
	
	\item Summary: both low LLD and low recombination rates can be associated with higher per SNP h2g, for different mechanisms.
\end{itemize}

Better estimation of SNP heritability from summary statistics provides a new understanding of the genetic architecture of complex traits (SamHer) [Speed and Balding, Biorxiv, 2018]
\begin{itemize}
	\item MOM estimation of heritability with LDAK prior: $\E(h_j^2) \propto q_j$, where $q_j$ is given by LDAK. Let $Z_j^2$ be the chi-square statistic of SNP $j$, we modify the LDSC equation by: 
	\begin{equation}
	\E(Z_j^2) = 1 + n_j (h_j^2 + \sum_{l \in N_j} r_{jl}^2 h_l)
	\end{equation}
	We plug in the LDAK prior, and normalize per SNP heritability: 
	\begin{equation}
	\E(Z_j^2) = 1 + u_j h^2_{\text{SNP}}  \text{ where } u_j = \frac{q_j + \sum_{l \in N_j} q_l r_{jl}^2} {Q}
	\end{equation}
	where $Q = \sum_j q_j$ is a normalizing constant. Estimation is done by MOM, with weighting of SNPs and estimation of standard error similar to LDSC. 
	
	\item Estimating confounding bias: use $\E(Z_j^2) = C(1 + 1 + u_j h^2_{\text{SNP}})$, and estimate $C$. 
	
	\item Comparing heritability models: use log likelihood of the SNPs assuming a diagonal matrix. 
	
	\item Enrichment: generalization of stratified LDSC.   
	
	\item Results of enrichment analysis: conserved regions are highly enriched (13 fold) for GWAS effects in LDSC, but only 1.7 fold by SamHer. 
	
	\item Prediction of risks: very modest (1 or 2\%) improvement of prediction accuracy (correlation) using LDAK vs. GCTA.  
\end{itemize}

Reconciling S-LDSC and LDAK functional enrichment estimates [Gazel and Price, NG, 2019]
\begin{itemize}
	\item LDAK model: use different ways of modeling LD. Baseline-LD: baseline annotations and LD related annotations. Gold standard: baselineLD and LDAK.
	
	\item Comparison of likelihood: LDAK lower than Gazal-LD and baseline-LD. Remark: not very comparable if the number of parameters are different.
	
	\item Simulation to assess enrichment estimates: S-LDSC + LDAK gives robust estimates (unbiased).
	
	\item UK biobank traits: LDAK consistently underestimate enrichment of functional annotations, because it assigns 0 h2g to 85\% of SNPs.
	
	\item Baseline annotations: from [Finucane, NG, 2015]. DHS, H3K27ac, etc.: union of ENCODE/Roadmap annotations from all cell types. 
\end{itemize}

Accurate estimation of SNP-heritability from biobank-scale data irrespective of genetic architecture [Hou and Pasaniuc, NG, 2019]
\begin{itemize}
	\item Background: (1) multi-component REML by MAF and LD stratification is very resource intensive, and ``it is unclear whether multi-component methods based on summary statistics produce accurate estimates of total SNP-heritability''. (2) Models that explicitly model MAF-LD dependency: sensitive to model mis-specification.
	
	\item GRE estimator: $\beta_i \sim N(0, \sigma_i^2)$. The goal is to estimate $h_g^2$, defined as PVE by all variants, or $\Var(x^T \beta) / \Var(y)$. Assuming $y$ is normalized with variance 1, then $h_g^2$ is just $\Var(x^T \beta)$. Use Law of Total Variance to partition this:
	\begin{equation}
	\Var(x^T \beta) = \E(\Var(x^T \beta | \beta)) + \Var(\E(x^T \beta | \beta)) = \E(\beta^T \Var(x^T) \beta) + \Var(\E(x^T) \beta)
	\end{equation}
	It is easy to see that the second term is 0, and let $V = \Var(x^T)$ be the LD matrix. So we have: 
	\begin{equation}
	h_g^2 | \beta = \beta^T V \beta
	\end{equation}
	where we treat $\beta$ as given. Of course $\beta$ is not given, so marginalizing $\beta$ gives $h_g^2 = \sum_{i=1}^M \sigma_i^2$. 
	
	\item Deriving GRE estimator: first we consider $\beta$ as fixed. We know from RSS the distribution of $\hat{\beta}|\beta$, so we use $\hat{\beta}$ to obtain the unbiased estimator of $\beta^T V \beta$. Roughly, we should use $V^{-1} \hat{\beta}$ to approximate $\beta$, so we have the estimator $\hat{\beta}^T V^{-1} \hat{\beta}$. This estimator is not unbiased, but with modest change, we have the unbiased MOM estimator (GRE estimator) as:
	\begin{equation}
	\hat{h}_{GRE}^2 = \frac{N \hat{\beta}^T V^{-1} \hat{\beta} - M}{N - M}
	\label{eq:GRE_estimator}
	\end{equation}
	where $N$ is sample size and $M$ number of variants. \\
	Proof: see Equation (1)-(3) of the paper. Briefly, treating $\hat{\beta}$ as RV, we have $\E(\hat{\beta}) = V \beta$ and $\Cov(\hat{\beta}) = \sigma_e^2 V / N$. Then we use the result of the expectation of Quadratic form of a random vector to obtain $\E(\hat{\beta}^T V^{-1} \hat{\beta})$. It is given by: 
	\begin{equation}
	\E(\hat{\beta}^T V^{-1} \hat{\beta}) = \frac{M}{N} \sigma_e^2 + \beta^T V \beta
	\end{equation}
	We note that $\sigma_e^2 + \beta^T V \beta = 1$, plug in and we can solve the GRE estimator. \\
	While we assume $\beta$ as fixed effect, it is easy to show that the estimator, when treating $\beta$ as random, is still unbiased. 
	
	\item Analysis: do we require $N > M$? The paper said, when $M > N$, we get negative estimates. However, when this happens, both denominator and numerator of the GRE estimation Equation~\ref{eq:GRE_estimator} are negative. And it's easy to plug in the expectation of $\hat{\beta}^T V^{-1} \hat{\beta}$ to see that $N-M$ term cancels out. The real reason that GRE works well in large samples is probably related to estimation error of GRE. In fact, if we ignore the LD, then the estimator is roughly $\sum_j \hat{\beta}_j^2$. So it does not do any shrinkage, and with large sample size, would be unbiased. At smaller $N$, however, the variance of $\hat{\beta}_j$ is large.  
	
	\item Genome-wide approximation: $N > M$ may not always satisfy, so we divide the genome into blocks, and estimate $h^2_g$ for each block then add. Let $p_k$ be the number of SNPs in block $k$, then we should use $p_k$ instead of $M$ in the equation above. 
	
	\item Dealing with rank deficient LD matrix: We should use pseudoinverse instead, and also use $q_k$, the rank, instead of number of SNPs in block $k$. So our final formula is:
	\begin{equation}
	\hat{h}_{GRE}^2 = \sum_k \frac{N \hat{\beta_k}^T \hat{V}_k^{-1} \hat{\beta_k} - q_k}{N - q_k}
	\end{equation}
	
	\item Simulation setting: test a number of different genetic architectures: different proportion of causal variants and MAF and how per SNP h2g depends on MAF and LD. The genetic architecture is defined as (assuming genotypes are standardized):
	\begin{equation}
	\sigma_i^2 \propto c_i w_i^{\gamma} [2 f_i (1-f_i)]^{1+\alpha}
	\end{equation}
	where $c_i$ is causal status of SNP $i$, $w_i$ the LD weight and $f_i$ the AF. With $\gamma = 0$: no dependency on LD, and $\gamma =1$: higher LD scores imply smaller effects. For $\alpha$: if $\alpha = -1$, it is the standard normal prior. Recommended value: $\alpha = -0.25$ by LDAK. 
	
	\item Simulations at large sample size: $N = $400K, 0.5M variants. GRE is unbiased under various scenarios (Fig. 2). LDSC, S-LDSC and SumHer all have bias under various architecture. 
	
	\item Simulations at small sample size: $N = $8K, 14K variants. Still unbiased, but large variations comparing with GREML (Fig. 3). LDAK may have large bias under some settings. The best method, GREML with different components, stratified by MAF and LD (GREML-LDMS-I).   
\end{itemize}


\subsection{Heritability from Family Studies}

Relatedness disequilibrium regression estimates heritability without environmental bias [Young and Kong, NG, 2018]
\begin{itemize}
	\item Background: (1) Kinship method: correlate genetic correlation (GRM by expected kinship) (2) Sib-regression: use random deviation from 50\% sharing, free from environment. 
	
	\item Intuition: relatedness of two individuals $i$ and $j$ (IBD sharing), $[R]_{ij}$, is largely determined by the relatedness of their parents, $[R_{\text{par}}]_{ij}$. However, due to random segregation, $[R]_{ij}$ may deviate from its expectation, $[R_{\text{par}}]_{ij}/2$. This deviation can causal additional phenotypic covariance between $i$ and $j$, and allow one to estimate heritability (higher heritability, higher additional covariance). 
	
	\item Variance decomposition: let $Y_i$ be the phenotype of sample $i$, and $g_i$ be direct genetic effect, $p_i$ be the environmental effect correlated with parental genotypes, including indirect genetic effect, and $e_i$ be the residual environmental effect (not correlated with parental genotype). We have: 
	\begin{equation}
	Y_i = g_i + p_i + e_i
	\end{equation}
	Note that $g_i$ and $p_i$ are correlated. So the variance is: 
	\begin{equation}
	\Var(Y) = \Var(g_i) + \Var(p_i) + \Cov(g_i, p_i) + \Var(e_i)
	\end{equation}
	The terms corresponding to Eq (1) in the paper. 
	
	\item Covariance decomposition: We consider two subjects with $Y_i = g_i + p_i + e_i$ and $Y_j = g_j + p_j + e_j$. Covariance has these components: $\Cov(g_i, g_j)$ which relates to $R_{ij}$; $\Cov(p_i, p_j)$ which relates to $[R_{\text{par}}]_{ij}$; $\Cov(g_i, p_j)$ and $\Cov(g_j, p_i)$, which relates to $[R_{\text{o,par}}]_{ij}$ (individual $i$ vs. parent of individual $j$). 
	
	\item RDR regression: to fit the model, we regress cov. between $Y_i$ and $Y_j$ vs. $R_{ij}$, $[R_{\text{par}}]_{ij}$ and $[R_{\text{o,par}}]_{ij}$. The reason that it is robust to environment effects is: the coefficient of $R_{ij}$ must be due to random Mendelian segregation since we control $[R_{\text{par}}]_{ij}$. 
	
	\item Simulation results: compare with kinship and sib-regression methods. Kinship is sensitive to genetic nurturing, maternal environment. Sib-regression has large s.e., and sensitive to epistasis and dominance. 
	
	\item Analysis: in GWAS, SNP effect should be the sum of direct and indirect genetic effects. Can we disentangle the two so that we can learn which variants may act indirectly? Let $P_i$ be parental genotype, $G_i$ be genotype of sample $i$ and $Y_i$ be phenotype, then we have: 
	\begin{equation}
	Y_i = \beta G_i + \gamma P_i + \epsilon_i
	\end{equation}
	where $\beta, \gamma$ are direct and indirect effects, respectively. We can thus learn the separate effects when $P_i$ is available (measured or imputed from relatives). 
	
\end{itemize}

\subsection{Methods of Linear Mixed Model}

EMMAX: Variance component model to account for sample structure in genome-wide association studies [Kang \& Eskin, NG, 2010]
\begin{itemize}
	\item Goal: testing individual SNPs while correcting for population substructure and cryptic relatedness. 
	
	\item Mixed effect model: suppose we are testing one SNP at a time, let it be $k$. For th $i$-th individual, let $x_{i,k}$ be the SNP's genotype, and $\beta_k$ be its effect size. The  phenotypic trait of the individual is: 
	\begin{equation}
	y_i = \beta_0 + \beta_k x_{i,k} + \eta_{i,k}	
	\end{equation}
	where $\eta_{i,k} = \sum_{s \neq k} \beta_s x_{is} + \epsilon_i$ is the total effect of genetic background and environment. In general, the effect of a single SNP is small, so $\eta_{i,k} = u_i + \epsilon_i$, where $u_i$ is a random effect, as in Equation~\ref{eq:random_effect_model}. 
	
	\item Procedure: three steps: 
	\begin{itemize}
		\item Estimation of kinship matrix $\hat{S}_N$. 
		\item Estimation of $\sigma_a^2$ and $\sigma_e^2$ using the random effect model, Equation~\ref{eq:random_effect_model}. Test if $\sigma_a^2 = 0$. 
		\item If $\sigma_a^2 \neq 0$: the model is reduced to linear model with dependent error term: 
		\begin{equation}
		y_i = \beta_0 + \beta_k x_{i,k} + \eta_i	
		\end{equation}
		where $\Var(\eta) \propto \hat{\sigma}_a^2 \hat{S}_N + \hat{\sigma}_e^2 I$. 
	\end{itemize}
\end{itemize}

FaST linear mixed models for genome-wide association studies [Lippert \& Heckerman, NM, 2011]
\begin{itemize}
	\item Model: following the standard LMM, we have 
	\begin{equation}
	y \sim N(X\beta, \sigma_g^2 K + \sigma_e^2 I)
	\end{equation}
	where $\sigma_g^2$ is the genetic variance and $\sigma_e^2$ environmental variance, $K$ is the kinship matrix. The challenge is that the likelihood function involves determinant and inverse of the covariance matrix, which is computationally expensive.   
	
	\item Computational speedup: let $\delta = \sigma_g^2 / \sigma_e^2$, then we write $\sigma_g^2 K + \sigma_e^2 I = \sigma_g^2(K + \delta I)$. Our idea is to factorize the matrix s.t. we do not have to recompute determinant and inverse of this matrix each time we update $\delta$ (the main parameter). We use the Spectral Decomposition of $K$, $K = USU^T$, where $S$ is diagonal and $U$ is orthogonal. Then:
	\begin{equation}
	K + \delta I = USU^T + \delta I = U (S + \delta I) U^T
	\end{equation}
	The determinant of this matrix: 
	\begin{equation}
	\det(U (S + \delta I) U^T) = \det U \cdot \det(S+\delta I) \cdot \det U^T = \det(S+\delta I)
	\end{equation}
	where we use the fact that $\det U \det U^T = 1$ since $U$ is  orthogonal. The inverse of the covariance matrix: 
	\begin{equation}
	[U (S + \delta I) U^T]^{-1} = U (S+\delta I)^{-1} U^T 
	\end{equation}
	Note that $S$ is diagonal, so both determinant and inverse of $S+\delta I$ is a simple function of $\delta$. Let $S = \text{diag}(S_{ii})$,  
	\begin{equation}
	\det(S+\delta I) = \prod_i (S_{ii} + \delta) \qquad (S+\delta I)^{-1} = \text{diag}\left(\frac{1}{S_{ii} + \delta}\right) 
	\end{equation} 
	
	\item Equivalent linear regression model: with the spectral decomposition, we can show that the original regression model (where errors are correlated) has the log-likelihood model that is exactly the same as the following model where errors are independent:
	\begin{equation}
	U^T y = U^T X \beta + E, \qquad E \sim N(0, \sigma_g^2(S + \delta I))
	\end{equation}
	Note that the covariance matrix of $E$ is diagonal. So if we rotate the data $y, X$ with $U$, then the new regression model is just a weighted linear regression. The log-likelihood is then simply weighted sum of squared error. See Equation 2 of the paper.
	
	\item Optimization: first solve $\beta$ and $\sigma_g^2$ in terms of $\delta$ (closed form); then do 1D optimization of $\delta$.  
	
	\item Running time analysis: suppose we test $s$ SNPs with sample size $n$. Then computation involves Spectral Decomposition $O(n^3)$, data rotation for each of $s$ tested SNPs, $O(n^2 s)$, and optimization for each SNP $O(Cns)$. If we fix $\delta$ (do not restimate for each SNP), then the optimization step takes $O(Cn)$. 
	
	\item When $K$ is low-ranked ($k$): we can exploit that to further speed up the computation. Roughly we have $O(n^2k)$, $O(nks)$ and $O(C(n+k))$ for each of the three steps. 
\end{itemize}

SNP Set Association Analysis for Familial Data [Schifano \& Lin, GE, 2012]
\begin{itemize}
	\item Motivation: SKAT (SNP-set test) does not account for family structure. Use a random effect model to dependency of relative. 
	
	\item Model: let $s_{ij}$ be the SNP set ($r$ SNPs) of the $j$-th indiviudal in the $i$-th family, $x_{ij}$ be the covariates and $y_{ij}$ be the phenotype of the individual $i,j$. The phenotype can be written as a mixed effect model: 
	\begin{equation}
	y_{ij} = x_{ij} \alpha + h(s_{ij}) + b_{ij} + \epsilon_{ij}	
	\end{equation}
	where $h(s_{ij})$ is the random effects of SNPs (each SNP has a random effect), and $b_{ij}$ the random effect due to family background. The genetic random effects are correlated among family members, and we assume $b_i = (b_{i1}, \cdots, b_{in_i})^T \sim N(0, 2 \Phi_i \sigma_b^2)$, where $\Phi_i$ is the kinship matrix. One can then form the score test. 
\end{itemize}

Efficient Bayesian mixed-model analysis increases association power in large cohorts (BOLT-LMM) [Loh and Price, NG, 2015]
\begin{itemize}
	\item Why LMM? Even if our goal is to perform association test (per SNP), there are two advantages: (1) Correcting for population structure, including cryptic relatedness. (2) By adjusting for all other SNPs, increase the power. 
	
	\item LMM for association test: suppose we test one SNP a time, for a SNP with genotype $x$,  
	\begin{equation}
	y = x \beta + g + e
	\end{equation}
	where $g$ is the genetic effect and $e$ the error. We use a random effect model for $g$. Note that $g$ should not include SNPs in LD with the test SNP. Using the variance component approach, we have: 
	\begin{equation}
	V = \Cov(y) = \sigma_g^2 K + \sigma_e^2 I 
	\end{equation}
	where $K$ is the GRM. Now this becomes a problem of Generalized Least Square (GLS), where the error term are correlated, or write in MVN: $y \sim N(x \beta, V)$. Using GLS result, we have: 
	\begin{equation}
	\hat{\beta} = (x^T V^{-1} x)^{-1} (x^T V^{-1} y)
	\end{equation}
	We can think of $V^{-1} y$ as the phenotype, corrected for dependency between samples. We can test if $\beta = 0$ using the score test, and the test statistic is: 
	\begin{equation}
	T = \hat{\beta}^2 / \Var(\hat{\beta}) = (x^T V^{-1} y)^2 / (x^T V^{-1} x)
	\label{eq:LMM_test}
	\end{equation}
	It follows chi-square distribution with df 1. Reference: Rapid variance components–based method for whole-genome association analysis, NG, 2012
	
	\item BOLT-LMM: replace normal prior of effect size with mixture of normal prior, inference of the posterior mean of effect size for each SNP, and regress out the predicted phenotype when testing each SNP. When test each SNP, remove all SNPs in the same chromosome in modeling the random effects. 
	
	\item Step 1: estimating the hyperprior parameters and the genetic effects. Major components: 
	\begin{itemize}
		\item First fit a standard variance component model to estimate PVE.  
		\item With a two mixture of normal prior (three parameters), constraint it using the total PVE, so there are two free hyperprior parameters. For each parameter combination, inference is based on cyclic coordinate descent: for each SNP, update its effect conditioned on the estimated effects of all other SNPs.
		\item Use cross-validation (based on prediction performance) to select among 18 parameter combination.
	\end{itemize}
	The result is residual phenotype (for each test SNP), regressing out the remaining chromosome.   
	
	\item Step 2: association testing. This involves slight modification of Equation~\ref{eq:LMM_test} above. We replace $y$ in the equation with the residual phenotype obtained from Step 1. 
	
	\item Q: if we explicitly regress out residual phenotype, why do we still need the GLS, which treats the residual as random? 
\end{itemize}

\subsection{Heriability Studies}

Common polygenic variation contributes to risk of schizophrenia that overlaps with bipolar disorder [Purcell, Nature, 2009]
\begin{itemize}
\item Problem: assess the contribution of common variants of small effects to a complex disease. 

\item Defining scores (page 19-20 of Suppl): divide the data into a discovery set, and a target set (the paper used males as discovery and females as target). From the discovery set, obtain the loci using a liberal threshold, e.g. $p < 0.1$ - these are called ``score alleles'' (instead of risk alleles). Then assess the collective contribution of these score alleles in the target set. The score is defined as: ``total score for each individual as the number of score alleles weighted by the log of the odds ratio from the discovery sample''. Formally, for each individual $i$ in the target samples, its score is defined as: 
\begin{equation}
S_i = \sum_j x_{ij} \cdot \log \lambda_j 
\end{equation}
where the summation is over all score alleles, $\lambda_j$ is the OR of the allele $j$ in the discovery set, and $x_{ij}$ the genotype of $j$ in subject $i$. 

\item Testing for association between score and disease in the target sample (page 21): regression in the target sample, estimate $R^2$ by comparing a model including the score and covariates versus a model including only the covariates (needed for controlled population stratification). 

\item The relation between score and disease risk [Wray, Prediction of individual genetic risk to disease from genome-wide association studies, GR, 2007]: the risk of getting affected in individual $i$ with genotype $G_i$ is given by
\begin{equation}
P(D_i|G_i) = f_0 \prod_j \lambda_j^{x_{ij}}
\end{equation}
where $f_0$ is the baseline penetrance. 

\end{itemize}

Using extended genealogy to estimate components of heritability for 23 quantitative and dichotomous traits [Zaitlen \& Price, PLoS Genet. 2013]
\begin{itemize}
\item Problem: in a dataset with some related and some unrelated invididuals, how do we estimate heritability? 

\item Model idea: define a threshold based on GRM, when $IBS > t$,  consider them related, we use IBD estimated from haplotype to obtain $h^2$; when $IBS < t$, we use GRM to obtain $h^2$. The statistical problem is: we cannot simply partition all pairs of individuals into related and unrelated, so our model should account for that. 
\end{itemize}

Most genetic risk for autism resides with common variations [Gaugler \& Buxbaum, NG, 2014]
\begin{itemize}
\item Background: the relation between explained variance, effect size/relative risk and allele frequency. Suppose we have a mutation/variant with known RR and AF, how much variance of the liability is explained by this variant in the population? We know that the proportion of variance explained (PVE) is given by (from linear model): 
\begin{equation}
R^2 = \frac{\sum_j {\beta}^2 \Var(x_j)}{\Var{(y)}} = \frac{\beta^2 p (1-p)}{\Var{(y)}}
\end{equation}
In liability model, we often assume $y$ is normalized, i.e. variance equal to 1. Next, the effect size $\beta$ is related to RR by this equation: 
\begin{equation}
\text{RR} = \frac{1-\Phi(t-\beta)}{1-\Phi(t)} 	
\end{equation}
where $t$ is the threshold (determined from the baseline penetrance). For autism, assume baseline penetrance of 1/68, we have $t = 2.18$. 

\item Overall heritability: using a large sample from Sweden, 1.6M families, including 14K ASD cases. Estimate heritability from recurrence in MZ twiwns to first cousins: 54\%. 

\item Heritability due to common variants: 3K subjects include 466 cases and 2.5K controls, use GCTA to estimate, remove all related samples, result 49.4\%. 

\item Heritability due to both common and rare variants: include more closely related individuals, result 52.4\%. 

\item Heritability due to non-additive effects: 3\%. 

\item Variance of liability due to de novo mutations: 2.6\%. For example, for LoF mutations, we use RR = 2.42 and exposure ($q$) 0.053, we obtain the variance explained by LoF as 1.1\% (my own calculation 0.7\%). For missense mutations, we use RR = 1.1 and exposure 0.397, we have the explained variance 0.04\%. 

\item Question: how to estimate the heritability due to common and rare variants? 

\item Remark: different studies could obtain very different estimates of heritability. Ex. for common variants, the estimate from PGC is only 17\%. The general issues of consideration for estimating heritability: ascertainment, prevalence (could be different in different samples/populations), control of environment. 
\end{itemize}

Partitioning heritability of regulatory and cell-type-specific variants across 11 common diseases [Gusev \& Price, AJHG, 2014]
\begin{itemize}
\item Partitioning heritabilty in the presence of LD: suppose we have multiple categories of variants (similar to chromosomes), and we want to partition $h^2$ into categories, but variants in different categories could be in LD. We use similar partition when the variants are not in LD: 
\begin{equation}
y = \sum_i \sum_{s \in S_i} W_s \beta_s^i + e
\end{equation}
where $i$ is the $i$-th category, $S_i$ the set of all variants in $i$, $W_s$ is the genotype of SNP $s$, and $\beta_s^i$ the effect size. The random effect model: $\beta_s^i \sim N(0, \sigma_i^2)$ and $\sigma_i^2$ is category-specific effect size parameter. We could do the joint estimation similar to partition into chromosomes. Under this model, LD is taken into account: whenever we estimate the contribution of $i$-th category, it is conditioned on all other categories, automatically accounting for LD. 

\item Estimating enrichment of disease loci from $h^2$ estimation: the ratio of $h^2$ from a category $i$ vs. the ratio of category $i$ in terms of number of SNPs defines the enrichment of $i$. It is equivalent to the the relative risk that a SNP in category $i$ is causal in comparison to an average SNP. 

\item Estimating enrichment of disease loci from summary statistics: count the enrichment of number of SNPs above a threshold vs. random expectation. 
\begin{itemize}
	\item Remark: this estimation does not account for the power of studies, i.e. even if a SNP is causal, it may not reach the p-value threshold. So the enrichment estimated in this way will always be lower than the true enrichment. 
\end{itemize}

\item Annotation data: 6 primary categories, coding, UTR, promoter, DHS in 217 cell types (only distal ones), intronic and intergenic. 

\item Simulation studies: assume 10\% of SNPs are causal, randomly sample them from one category only, and see if the enrichment methods can recover the category. Results: 
\begin{itemize}
	\item $h^2$ based enrichment always recovers the true category, however, $p$-value based enrichment fails for DHS (large categories). Explanation: $p$-value enrichment is less sensitive, and the low LD in DHS region also makes it hard to detect enrichment. 
	
	\item Using imputed SNPs always better than only genotyped SNPs.
	
	\item When all causal variants are in coding regions, UTR and promoters are still enriched in $p$-value approach but not $h^2$ approach.  
\end{itemize}

\item Enrichment analysis in 11 common diseases (WTCCC) using heritability and $p$-values: variants with MAF $>1\%$, using $h^2$ enrichment, DHS is 5.1x enriched (79\% of $h^2$) and coding sequence is 13.8x, or 8\% of $h^2$ (Figure 3.). Using $p$-value enrichment, depending on the $p$-value cutoff, the enrichment of coding is highest (around 2 at stringent $p$-value), and the enrichment of DHS is very modest, around 1.1-1.2 at very stringent p-values. 

\item Contribution of enhancers: show that for DHS, a large fraction of heritability is due to enhancers. 

\item Lessons: 
\begin{itemize}
	\item Enrichment of causal variants can be significantly underestimated when the study power is not taken into account: with the $p$-value approach, at a given threshold, many true loci will not pass the threshold, leading to an underestimation. 

	\item How to control for LD: could be a problem, e.g. promoters and coding sequences may be in close LD, and thus can be difficult to separate out the effects using $p$-value approach (or simple regression in general). The strategy is to include them in a single regression model, thus adjusting for other variants in LD.  	
		
	\item Enrichment of causal variants: can be done using heritability or PEV (proportion of explained variance) under a linear model (instead of logistic regression). The results can be used as a prior for Bayesian anaysis of functional variants. 
\end{itemize}

\end{itemize}

Heritability estimation: comparison of GCTA vs. Sparse prior [Xiang Zhou, UMich, 2019]
\begin{itemize}
	\item H2g estimation: (1) Low polygenicity: normal prior has large s.e., and sparse prior works very well (small s.e.). (2) Mid to high polygenicity: normal prior is unbiased, but sparse prior has downward bias, because sparse prior will miss many effects. 
	
	\item Prediction: biobank scale. Large effect SNPs, treated as fixed effects, then the rest with random effects (normal prior). 

\end{itemize}
%%%%%%%%%%%%%%%%%%%%%%%%%%%%%%%%%%%%%%%%%%%%%%%%%%%%%%%%%%%%
\subsection{Polygenic Prediction} 

Disease classification from genotype [Jakobsdottir \& Weeks, PG, 2009] 
\begin{itemize}
	\item Two basic ways of measuring a set of SNPs for disease association/diagnosis: 
	\begin{itemize}
		\item Risk analysis: typically in a setting where multiple SNPs are identified, and the SNPs are assessed by the increased risk/odds of the (combined) SNPs. 
		
		\item Classification performance: suppose we use a genotype $G$ to classify: i.e. declare case if the genotype is $G$; declase not if the genotype is not $G$. Let the disease status be $A$ and $U$ (affected, or unaffected). The performance is measured by two quantities: true positive and false positive fractions: 
		\begin{equation}
		\begin{array}{l}
		TPF = P(G|A)\\
		FPF = P(G|U)
		\end{array}
		\end{equation}
		Thus the performance of the classifer is determined by the genotype frequencies in the cases and controls. 
	\end{itemize}
	
	\item Relation between odds ratio (OR) and classification performance: 
	\begin{itemize}
		\item Use the relation between odds and genotype frequencies ($G_0$ is the reference genotype): 
		\begin{equation}
		OR = \frac{\text{odds}(G)}{\text{odds}(G_0)} = \frac{P(G|A)/P(G|U)}{P(G_0|A)/P(G_0|U)}	
		\end{equation}
		Note that $P(G_0|A) = 1 - TPF$ and $P(G_0|U) = 1 - FPF$. Solving this equation for a given OR leads to the ROC of performance (only one of $TFP$ or $FPF$ can be fixed). 
		
		\item Analysis: as an approximation, we assume the reference genotype does not increase disease risk and equally distributed in cases and controls, then $\text{odds}(G_0) = 1$, and $P(G_0|A)/P(G_0|U) = 1$, then we have: 
		\begin{equation}
		\frac{P(G|A)}{P(G|U)} \approx OR	
		\end{equation}
		Even for large $OR$, its predictive power depends on the genotype frequencies (in control): (1) common alleles: large $P(G|U)$, thus high FPR, and high TPR; (2) rare alleles: small $P(G|U)$, thus low FPR, but also low TPR. Ex. AUC = 0.76 for a huge OR of 50 (e.g. at $FPF = 2.4\%$, the TPF is only 55\%).
		
		\item Why low risk alleles are not good predictors of diseases? In most GWAS studies, the alleles have OR below 2. We consider an allele with relatively low frequency in the control (if too common, FPR would be too high), say, $P(G|U) = 0.1$, and $OR = 2$, then according to our approximation, $P(G|A) = 0.1 \times 2 = 0.2$, i.e. the sensitivity is only 20\%. 
		
		\item Why multiple low risk alleles may not be sufficient for disease classification? Suppose we have two risk alleles with $P(G|U) = 0.2$ and $OR = 2$ for both, and we follow AND rule for classification (OR rule would give too many false positives). With single risk allele: $FPR = 0.2, TPR = 0.4$; with both alleles: $FPR = P(G_1 \wedge G_2|U) = 0.2 \times 0.2 = 0.04$, and $TPR = P(G_1 \wedge G_2|A) = 0.04 \times 2^2 = 0.16$. In general, AND rule would lead to low TPRs. 
	\end{itemize}
	
	\item Classification performance for several common diseases: e.g. in AMD data set, using an additive model of three variants (highly significant association) leads to AUC at 0.79. In T2D, a model of 12 SNPs has AUC of only 0.64. 
\end{itemize}

Multi-ethnic polygenic risk scores improve risk prediction in diverse populations [Márquez-Luna \& Price, review for PLG, 2017]
\begin{itemize}
	\item Motivation: how do we predict phenotype in a population with small sample size? There is larger data from a different population, but the LD patterns are different, which could reduce the accuracy. 
	
	\item Background: Polygenic risk score (PRS), usually do LD pruning and thresolding.  
	
	\item Analysis: assume no change of true causal variants and their effects. Consider a region, and let $i$ be its causal variant. Suppose in the training data, $j$ is the top SNP (could be equal to $i$) and its observed effect $\hat{\beta}_j$. In the target data, the true phenotype of an individual is $y = x_i \beta_i$, and the predicted: $\hat{y} = x_j \hat{\beta}_j$. We take expectation, and use the RSS model (assume the same standard errors in $i$ and $j$): $\E{\beta}_j = R_{ij} \beta_i$. We have: 
	\begin{equation}
	\E(\hat{y}) = x_j R_{ij} \beta_i \qquad y = x_i \beta_i
	\end{equation} 
	In the target data, LD between $x_i$ and $x_j$ is $R_{ij}'$. We note that the different LD pattern could cause problem: suppose $R_{ij}$ is large, but $R_{ij}'$ is small, then even if $j$ is a good proxy of $i$ in the training data, it is not in the testing data. 
	
	\item Analysis: the impact of ancestry. Ancestry could be correlated to phenotype, and adding it could improve accuracy. Intuitively, in the target population, because of random drift, $j$ might no longer be a good proxy of $i$. But the overall ancestry may still be correlated with many causal variants. 
	
	\item Model: train effect size model using both popluations. Then use the weighted average of PRS scores from the two effect size models. The weights are learned by (1) use the validation data, and compute adjusted $R^2$ to account for additional dof (when comparing methods in the validation data). (2)cross-validation. 
	
	\item Model: incorporating ancestry. Similar idea, use linear combination of PRS and the top PC. The weight parameters are learned from validation data. 
	
	\item Simulation study: use EUR and LAT data from WTCCC2. Show that EUR and LAT have somewhat similar performance even though EUR sample size is much larger. EUR $+$ LAT improves upon both, and EUR $+$ LAT $+$ PC further improves by 10\% or so. 
	
	\item T2D risk prediction: similar to simulations. The PC part has little improvement over EUR $+$ LAT, probably due to the fact that EUR effect sizes already captured most of the genetic ancestry. 
\end{itemize}

Improved polygenic prediction by Bayesian multiple regression on summary statistics [Lloyd-Jones and Visscher, NC, 2019]
\begin{itemize}
	\item Model: similar to RSS, let $b$ be the estimated effect sizes, and $\beta$ be the true effects, the model is given by:
	\begin{equation}
	b = D^{-\frac{1}{2}} B D^{\frac{1}{2}} \beta + D^{-1} X^T \epsilon
	\end{equation}
	where $D$ is diagonal matrix of $x_j^T x_j$, and $B = D^{-\frac{1}{2}} X^T X D^{-\frac{1}{2}}$. Using ASH prior for $\beta_j$. The model makes inference with Gibbs sampling. The conditional distribution of $\beta_j$ given others is in page 38 of Supplement. Let $\delta_j$ be the class indicator of $\beta_j$, and define $l_{jc} = x_j^T x_j + \sigma_e^2 / (\gamma \sigma_{\beta}^2)$, then
	\begin{equation}
	f(\beta_j | \delta_j = c, \theta_{-\beta_j}, y) \propto \exp \left[ -\frac{1}{2} \frac{(\beta_j - \hat{\beta}_j^2)}{\sigma_e^2 / l_{jc}} \right]
	\end{equation}
	where $\hat{\beta}_j$ is the posterior mean of $\beta_j$. It is given by $\hat{\beta}_j = x_j^T w / l_{jc}$, where $w$ is the residual regressing out all other parameters except $\beta_j$. In the actual model, sample $\delta_j$ first, marginalizing all parameters (PIPs), then sample $\beta_j$. 
	
	\item Right-hand site (RHS) update: note that to in each update, need to only compute $\hat{\beta}_j$. The computation of residual $w$ is expensive, however, we only need $r_j = x_j^T w$, so we only need to store $r^* = X^T y - X^T X \beta$, which is $p \times 1$ vector. Once we have $r^*$, we can compute $r_j$ as:
	\begin{equation}
	r_j = r_j^* + x_j^T x_j \beta_j 
	\end{equation}
	Note that computation of $r^*$ is faster and requires much less memory: since $X^T y = D b$ is proportional to $b$ (summary statistics), and $X^T X$ is just the LD matrix, which is local/sparse. At each iteration, once we update $\beta_j$ using the current $r_j$, we can update $r^*$ as: 
	\begin{equation}
	\Delta r^* = X^T x_j \Delta \beta_j
	\end{equation}
\end{itemize}

%%%%%%%%%%%%%%%%%%%%%%%%%%%%%%%%%%%%%%%%%%%%%%%%%%%%%%%%%%%%
\section{Gene-Gene and Gene-Environment Interactions}

Reference: epistasis [Cordell, NRG, 2009; Carlborg \& Harley, NRG, 2004]; multi-locus methods [Hoh \& Ott, NRG, 2003; Onkamo \& Toivonen, Human Genomics, 2006]

Discussion with Margit Burmeister [Umich, 2019]
\begin{itemize}
	\item Examples of gene-environment interactions: a variant about additive behavior: in smokers, decrease BMI, because it makes one more additive to smoking, which reduces appetite. But it  increase BMI in non-smokers.
	
	\item Confounding of genetics and environment: e.g in UK, some areas, low SE status, and different genetic groups. So genetics/regions/environment become confounded.
	
	\item Taste preference: highly genetic, e.g. broccoli bitterness. This may influence the risk of other traits, e.g. BMI (because one avoids bitter vegetables).
	
	\item Testing gene env. interactions via PRS and MR: e.g. PRSs of taste or smoking, and use as a surrogate for env.
	
	\item Remark: big question is how gene-environment interaction “works”? Is mediation by env. a common mechanism?
\end{itemize}

Multi-locus methods: motivations and benefits 
\begin{itemize}
	\item Regional test: use all information in a region (a gene) to test association, this is a generalization of haplotype-based tests.
	\item The SNPs may have non-linear interactions s.t. single SNP-based test may not be sensitive to detect them. 
	\item Fine mapping: in a candidate region, multiple SNPs may be associated with the trait because of LD. Using multi-locus methods may help to identify the causal variants (removing the effect of correlated, but non-causal SNPs). 
\end{itemize}

Background on epistatis: 
\begin{itemize}
\item Definition of epistasis: this is broadly defined as the effect of one allele on the phenotype depends on the allele at another locus. Example: the coat color of pig: the effect of the allele at the MC1R locus is dominated by the KIT locus, i.e. MC1R allele expresses phenotype only with a particular allele in the KIT locus. 
	
\item Biological basis of epistatis: many possible mechanisms of epistasis, including (but not limited to): 
\begin{itemize}
	\item Redundancy: e.g. two proteins play similar function, thus only when mutating both proteins, an effect can be seen. 
	\item Dominance/masking: e.g. in a biochemical pathway, one enzyme is rate limiting, thus mutation of another enzyme is invisible unless the rate limiting enzyme is also mutated. 
	\item Synergism: e.g. two proteins may be synergistic (in signal transduction or gene regulation), thus mutating two proteins has a effect similar to mutating only one of them. 
	\item Network behavior: the positive and negative feedbacks in the network create non-linearity, e.g. negative feedback may make the system robust to single mutations, but not to double mutations. 
\end{itemize}

\item Ubiquitous nature of epistasis: 
\begin{itemize}
	\item From the study of QTL: the proportion of the genetic variance in F2 that results from epistasis ranged from 0 to 81\% with a mean of 38\% for the 18 traits studied. 
	\item Difficulty of detecting epistasis: from the perspective of genotype partitioning, epistatis is detected from the frequencies of genotype combinations, which are exponentially lower than the frequencies of single genotype. Therefore, considerably larger sample size is generally needed. 
\end{itemize}
 
\end{itemize}

Testing for epistasis/interactions: two basic ideas of statistical interaction
\begin{itemize}
\item Disease risk perspective: the disease risk of the genotype combination is not the same as the sum of the risks of individual genotypes. 
\begin{itemize}
	\item Genotype paritioning: e.g. multifactor dimensionality reduction (MDR) method, compare the case/control ratio under different genotype combinations. 
	\item Likelihood method: e.g. epastasis option in PLINK: comparison of different logistic regression models, where the interaction term(s) are set to zero. 
\end{itemize}

\item Genotype distribution perspective: the frequency of genotype combination is not the product of that of individual genotypes (the genotypes at two loci are not independent). 
\begin{itemize}
	\item Dependency/correlation of different predictors (loci) in case-only studies: this test has higher power than the regression models. However, two loci may be correlated for other reasons: LD, or genotype combinations may be related to viability. 
	\item Likelihood method: the data are sampled from multinomial distributions. Under null model (no interaction): the probabilities of a genotype combination can be factorized, or equivalently, the probabilities of multiple loci can be multiplied. Ex. BEAM. 
\end{itemize}
\end{itemize}

Issues of testing epistasis: 
\begin{itemize}
	\item Alternative epistatis models: there are a number of cases where epistatis may occur. A particular important situation to consider is: whether the two loci have marginal effects. It is possible that two loci has strong epistais but each loci itself has no main effect, but it is not clear how common this situation is. 
	\item Test for interaction vs. test for association allowing for interaction: in some cases, the later may be more interesting. A natural approach is to average over all possible interacting loci of the locus of interest. 
	\item Single locus test as filter: this is related to the issue whether a single locus has marginal effect. If not, filtering based on marginal effect may have low power. Ex. stepwise logistic regression model performs poorly under the experiment of S [Zhang \& Liu, NG, 2007]. 
	\item Multiple testing: permutation test can be theoretically used, but not feasible if the number of tests is large. 
\end{itemize}

Comparison of methods: (some methods are evaluted by using WTCCC data of Crohn's diease, pre-filter with $p$ value $< 0.1$ to limit the number of SNPs to about 10,000) 
\begin{itemize}
	\item Random forest: No clear sharp signals (thus even worse than single-locus analysis). And no clear greater insights than single-locus analysis. 
	\item MDR: use TuRE for attribute selection, then apply MDR. Very sensitive to random seed, thus the perofmrance does not seem very stable. CPM is similar, but for quantitative traits. 
	\item BEAM: similar to single locus analysis, find a single significant SNP interaction, however, the two SNPs are tightly linked, thus probably from LD. 
	\item SVM: [Waddell, SIGKDD, 2005] 3,000 SNPs in 40 cases/controls. The resulting SVM is not very interpretable (150 SNPs). 
\end{itemize}

Overview of gene-environment interactions: 
\begin{itemize}
\item Graphical model perspective of gene-environment interaction analysis: suppose we have genotype $G$, environment $E$ and disease phenotype $D$, then we have the simple model: 
\begin{equation}
G \rightarrow D \leftarrow E, \qquad G \leftrightarrow E
\end{equation}
where the second relation describes the possible dependence between $G$ and $E$. This perspective could help some questions in G-E analysis. For example, even if $G$ and $E$ are independent in the population, and there is no interaction between $G$ and $E$, conditioned on $D$, they are dependent (easy to prove using three normal RVs and check the conditional correlation between $G$ and $E$). 
 
\end{itemize}

Non-hierarchical logistic models and case-only designs for assessing susceptibility in population-based case-control studies [Piegorsch \& Taylor, Stat Med 1994]: 
\begin{itemize}
\item Motivation: the standard test of gene-environment interaction is the logistic regression model testing the coefficient of the interaction terms. However, this test has three d.f. (the main effects of gene and environment, interaction effect), can we have a test with a single d.f.? 

\item Intuition: suppose $G$ and $E$ are independent in the population, and $G$ and $E$ work independently to raisk the disease risk, then in the disease group (cases), when we see $G$ as a risk allele, we don't have more information of $E$ (beyond what we konw from the disease state $D$, thus $E$ should be higher), i.e. $G$ and $E$ are independent in cases.  

\item Relation between interaction term and ORs: consider the discrete model where genotype $G =i$ and environment exposure $E = j$ (with $i=1,j=1$ the reference state), and the disease state $D$ or $\bar{D}$. Let $\Psi_{ij}$ be the OR of the genotype $i$ and environment $j$ combination, i.e.:
\begin{equation}
\Psi_{ij} = \frac{P(D|G=i,E=j)}{P(\bar{D}|G=i,E=j)} \cdot \frac{P(\bar{D}|G=1,E=1)}{P(D|G=1,E=1)}
\end{equation}
Similarly, we could define $\Psi_{i1}$ and $\Psi_{1j}$. Under the logistic model, the interaction term is related to the OR: 
\begin{equation}
\exp(\gamma_{ij}) = \frac{\Psi_{ij}}{\Psi_{i1} \Psi_{1j}}	
\end{equation}
When $G$ and $E$ act independent, we have $\Psi_{ij} = \Psi_{i1} \Psi_{1j}$. Our goal is to show that this is equivalent to: $G$ and $E$ are indepndent in cases. 

\item Case-only test: we apply the Bayes rule to the odds of $(i,j)$ combination:
\begin{equation}
\frac{P(D|G=i,E=j)}{P(\bar{D}|G=i,E=j)} = \frac{P(G=i,E=j|D)P(D)/P(G=i,E=j)}{P(G=i,E=j|\bar{D})P(\bar{D})/P(G=i,E=j)} = c \cdot \frac{P(G=i,E=j|D)}{P(G=i,E=j|\bar{D})}
\end{equation}
where $c = P(D)/P(\bar{D})$ is a constant. Plug in this into the equation of the interaction term: 
\begin{equation}
\frac{\Psi_{ij}}{\Psi_{i1} \Psi_{1j}} = \frac{P(G=i,E=j|D) P(G=1,E=1|D) P(G=i,E=1|\bar{D}) P(G=1,E=j|\bar{D})}{P(G=i,E=1|D) P(G=1,E=j|D) P(G=i,E=j|\bar{D}) P(G=1,E=1|\bar{D})}	
\end{equation}
If the disease is rare, and $G$ and $E$ are independent in the population, then the four terms conditioned on $\bar{D}$ will be canceled out. We have (rearranging the terms and use the conditional probability terms $P(E|G,D)$): 
\begin{equation}
\frac{\Psi_{ij}}{\Psi_{i1} \Psi_{1j}} = \frac{P(E=j|D, G=i) P(E=1|D, G=i)}{P(E=1|D, G=i) P(E=j|D, G=1)}		
\end{equation}
which is exactly the OR of genotype $i$ treating $E$ as the phenotype. 

\item Higher power using the case-only test: it can be shown that the case-only test is more powerful or has a lower standard error when estimating the interaction than the standard logistic regression test. The reason is probably due to (1) lower df. in the case-only test (no the main effect terms); (2) using the independence of $G$ and $E$ in the population (similar to the fact that modeling the joint distribution may be more powerful than the regression approach modeling conditional distribution only. 

\item Summary of the case-only test of interaction: if $G$ and $E$ are independent and the disease is rare, then to test if $G$ and $E$ have interactions to influence the disease risk is equivalent to test if $G$ and $E$ are independent in the case-only samples, and the case-only test is more powerful. 
\end{itemize}

Multifactor dimensionality reduction (MDR) [Ritchie \& Moore, AJHG, 2001]
\begin{itemize}
	\item MDR procedure: (1) Prediction rules: for any multi-factor genotype class (e.g. genotype combination of two loci), determine for each combination, whether it increases or decreases the risk. And the rule for predicting disease is simply the union of all combinations that are associated with high risk. Thus this is a simple Boolean function. (2) Assessing a combination of multiple loci: predictive accurancy under 10-fold cross validation. 
	\item Data: 200 breast cancer patients, about 10 polymorphic loci in the coding regions of the candidate gene(s). 
	\item Results: a four-locus interaction associated with breast cancer. 
\end{itemize}

Tree-based association of alcoholism: [Ye \& Zhang, BMC Genet, 2005]
\begin{itemize}
\item Data: 1,614 family members, 4,720 SNPs, response: alcohol dependence. 
\item Classification method: decision tree with the covariates: sex, parental phenotypes, and the SNP markers. 
\item Results: The pruned tree at the significance level of 0.00001: the top nodes are sex and mother phenotype, the rest are SNPs. 
\end{itemize}

Comparison of tests of association between a region and a trait: [Roeder \& Devlin, GE, 2005]
\begin{itemize}
\item Problem: given a set of SNPs in substantial LD, test if the set is associated with a trait. The idea is that by using all information in the LD region, it may perform better than the single locus test. 

\item Tests: 
\begin{itemize}
	\item $T_P$: suppose $T_l$ is the test statistic of the $l$-th locus, then define $T_P = \max \{ T_1, \cdots, T_L \}$. The threshold is determined by permutation of cases and controls. 
	\item $T_S$: a smoothed version of $T_P$, fit a smooth curve $g(\cdot)$ to $T_1, \cdots, T_L$, and let $S_l = \hat{g}(b_l)$ be the value of the $l$-th SNP at the curve. Define $T_S = \max \{ S_1, \cdots, S_L \}$. 
	\item $T_R$: let $Y$, the phenotype, be a linear function of genotypes in all loci: $g(\mu) = \beta_0 + X \beta$, where $g$ is the logistic function, $\mu = E(Y)$ and $\beta$ are the regression coefficients of all loci. $T_R$ is defined as the test statistic of the hypothesis: $\beta = 0$ simultaneously for all loci. This is also the Hotelling $T^2$ statistic: the genotype frequency of all loci are equal under case and control.  
\end{itemize}

\item Results/Discussion: 
\begin{itemize}
\item $T_R$ may be more powerful than haplotype-based association test [Chapman \& Clayton, Hum Hered, 2003]. Need to check what version of haplotype-based test is used: e.g. parsimony of the haplotypes. $T_R$ suffers from a very flexible alternative hypothesis, thus a penalty is power. 
\item $T_P$: perform at least as well and usually better than other test statistics. 
\item $T_S$: the effect of sommothing is that it is more robust than other statistics, most powerful when a dense set of SNPs is genotyped and a causal variant is located with this region. But it has little power when the signal is very local.
\item $T_P$ and $T_S$: increase power with more SNPs; but $T_R$ will reduce power with more SNPs (more flexible alternative hypothesis). 
\end{itemize}

\item Question: in $T_R$, the genotype frequencies do not follow Guassian distribution, how would Hotelling $T^2$ test apply? 
\end{itemize}

Two-stage approach for testing epistatis [Marchini \& Cardon, Genome-wide strategies for detecting multiple loci that influence complex diseases, NG, 2005]
\begin{itemize}
\item Simulation models: consider three models (Figure 1). Ex. two-locus interaction with multiplicative effect, the odds of the genotype with both loci minor allele is: $\alpha (1 + \theta)^4)$, where $\alpha$ is the odds of baseline genotype, and $\theta$ the increase of odds by any single minor allele (in either locus). 

\item Simulation: assume that only one pair of loci has interaction. $L = 300,000$ markers, $n = 1,000$ to 4,000 cases and controls; MAF varies from 0.05 to 0.5; $r^2$ from 0.5 to 1.0; and $\lambda$ (the marginal effect) from 0.2 to 1.0. In general, the results will depend on these parameters, so need to examine their influences. Note that from genotype frequency (MAF and HWE) in the control, we can estimate the frequencies in the case group (by using the disease odds). For each parameter setting, repeat the simulation 1,000 times. The results are evaluated by the power of tests, i.e. the fraction of successfully detecting association in 1,000 simulations. 

\item Tests: the two-stage model first selects SNPs using single-locus test, with some weak threshold. Then test any pair of loci using LRT where the alternative hypothesis would use full interaction model, i.e. logistic regression with 9 parameters (intercept, additive and dominant terms per loci, and four interaction terms). The threshold is determined by Bonferroni correction. Thus for interaction test, the penalty is very high. 

\item Results: 
\begin{itemize}
	\item With no-interaction model, the single-locus test has significantly higher power than two-stage or full interaction tests, especially at low MAF. 
	\item With the interaction models: the two-stage and full interaction methods perform similarly, and both better than single-locus test. 
\end{itemize}
\end{itemize}

Bayesian inference of epistatic interactions in case-control studies (BEAM) [Zhang \& Liu, NG, 2007]
\begin{itemize}
\item Model: 
\begin{itemize}
	\item Assume the genotype data are generated from a probabilistic model. Three groups of loci: group 1, no effect; group 2, main effect only; group 3, interactions. The index variables (which loci belong to which group) are unknown and to be inferred. Assume independence of loci, however, modeling LD is also possible by using a Markov chain, where the emission of the genotypes would depend on the adjacant marker. 
	\item Likelihood/evidence of model: the data is sampled from a multinomial distribution with three parameters for control. For case: (1) In group 1: case distribution is the same; (2) In group 2: the control distribution is a different multinomial distribution; (3) In group 3: multiplying probabilities over genotype combinations. In all computations, the parameters in the multinomial distribution are assumed to have Direchlet prior, and can be integrated out. 
	\item B statistic: log. of Bayes factor as the test statistic for any marker set. 
\end{itemize}

\item Results: BEAM outperforms the two-stage logistic regression methods on the models with no main effects, using simulation data. No epistatis is found in AMD data (small size with less than 100 cases/controls. 

\item Remark: the method is sensitive to LD between two markers: the probability of the genotype at two loci will not be independent, thus the test will lead to FP results. 
\end{itemize}


Epistatic interactions in GWAS using PPI network [Emily \& Schierup, EJHG, 2009]
\begin{itemize}
\item Analysis: test epistaitic interactions on SNP pairs, where associated genes have PPIs. The test is based on LRT of two models: H0 - independent effect of two SNPs; H1 - H0 and interaction between two SNPs. The logistic regression is used to map genotype to phenotype.  

\item Data: WTCCC data. PPI network from STRING database containing 71k interactions. After filtering (non-HWE, etc.), the number of SNP pairs (with PPI in associated proteins) for each disease was about 3 million. In contrast, the number of all possible SNP pairs is 125 billions. 

\item Results: 4 epistatic interactions were identified. in Crohn's disease: 8 significant SNP pairs, all from a single pair of genes: they are in LD with genes APC and IQGAP1. Both have function in cell adhesion and regulating cell migration, and both regions have been related to CD before. 
\end{itemize}

Gene-Environment Interaction in Genome-Wide Association Studies [Murcray \& Gauderman, Am J Epiderm, 2009]: 
\begin{itemize}
\item Motivation: the case-only test of gene-environment interaction is more powerful, however, it assumes the independence of $G$ and $E$ in the population. The standard regression test does not make the assumption, but is less powerful. The idea is to use the case-only test test to filter, then apply the regression test only on the filtered list (no independence assumption, and less penalty of multiple testing). 

\item Two-step procedure: 
\begin{itemize}
\item Step 1. combine the cases and controls, and test the indepdence of $G$ and $E$, i.e. $\text{logit} P(E=1|G) = \gamma_0 + \gamma_g G$, test if $\gamma_g = 0$. Suppose there are $m$ SNPs with $P < \alpha_1$. 
\item Step 2. the standard test: 
\begin{equation}
\text{logit} P(D=1|G,E) = \beta_0 + \beta_g G	+ \beta_e E + \beta_{ge} G E
\end{equation}
Test if $\beta_{ge} = 0$. The significance level is $\alpha / m$, where $\alpha$ is the desired type I error. 
\end{itemize}

\item Independence of the Step 1 and Step 2 tests: if the two tests are dependent, then we will have inflated type I error. It can be shown that, the two test statistics are asymptotically independent (using the normal approaximation of the test statistic, and compute the correlation coefficient). Note that: this is not true if we use cases only in Step 1. 

\item Remark: intuitively, the two tests are independent realizations of the interaction term. The first test is about the independence of $G$ and $E$, and the second test is about how $G$ and $E$ together influence the risk of $D$. 
\end{itemize}

Screen and clean: a tool for identifying interactions in genome-wide association studies [Wu \& Roeder, GE, 2010]:
\begin{itemize}
\item Computational procedure(Figure 2): 
\begin{itemize}
	\item Step 1: a set of SNPs to analyze
	\item Step 2: Lasso screening retains a subset of SNPs
	\item Step 3: the new dictionary involves all the subset of SNPs plus all pairwise interactions
	\item Step 4: Lasso screening again
	\item Step 5: cleaning, traditional hypothesis testing on the remaining terms from Step 4. 
\end{itemize}

\item Simulation: generate SNPs following specified LD. For simpler simultations: 1000 SNPs split into 200 blocks (5 SNPs per block); for large simulations: 100K SNPs with LD by Markov chain. The genetic models tested include: main effect only, interaction effect with varying number and strength of interactions. 

\item T1D analysis with WTCCC data: three pairs of SNPs, two in MHC region. The SNPs are close to each other in all three cases, though LD is very low. 
\end{itemize}

A Flexible Bayesian Model for Studying Gene-Environment Interaction [Yu \& Liang, PLoS Genetics, 2012]: 
\begin{itemize}
\item Model: gene-environment interaction by assuming there are multiple clusters (sets of genotypes), and the environmental effect may be different in different clusters. 
\begin{itemize}
	\item Inference: the number of clusters (K) unknown.
\end{itemize}

\item Remark: The better performance in simulation (comparison with single SNP or PC test): relies on multiple SNPs interacting with environment. Not clear how this is true in general. 
\end{itemize}

SBERIA: Set Based gene EnviRonment InterAction test for rare and common variants in complex diseases [Jiao \& Peters, submitted to GE, 2013]:
\begin{itemize}
\item Background: difficulty of aggregating signals: tell signals from noise and how to determine the direction of the signals. To deal with this issue in the SNP set test:
\begin{itemize}
\item Han and Pan 2010: used the signs of the marginal effect to determine the direction of the main effect. 
\item Lin and Tang (2011) used the corresponding regression coefficient plus a constant as the weight for each marker. 
\item Cai, Lin and Carroll 2012: proposed to weight each marker based on the z-score of its effect. 
\end{itemize}
One common characteristic of these methods is that the statistics used to weight the markers are not independent of the main effect test. Hence, permutation is needed to estimate the null distribution and maintain the correct type I error, which is computationally intensive.

\item Background: correlation screening to identify GxE interactions, and the test is independent of the usual regression-based test [Murcray \& Gauderman, Am J Epiderm, 2009]. 

\item Basic model and benchmark methods: suppose we are testing the interaction of a group of SNPs with environment ($E$). Let $G$ be the genotype vector, we have the model: 
\begin{equation}
\text{logit}(D) = \alpha_0 + \alpha_1 E + \alpha_2 G + \sum_j \beta_j (E G_j)
\end{equation}
where $\beta_j$ is the coefficient of the $j$-th interaction term. The null hypothesis is $\forall j: \beta_j = 0$. The simplest tests are: the LRT with df. $m$ (the number of SNPs) - $\chi^2$ distribution, and the individual SNP test (min-$P$ value) - use permutation to assess significance. 

\item Model: similar to the aggregate test of the main effect (burdent test, EREC, etc.), we assume $\beta_j = w_j \rho$, with $w_j$ the weight of the $j$-th SNP. The model becomes: 
\begin{equation}
\text{logit}(D) = \alpha_0 + \alpha_1 E + \alpha_2 G + \rho \sum_j w_j (E G_j)
\end{equation}
Our test is $H_0: \rho = 0$. Similar to EREC, it is better to learn $w_j$ from data, instead of using fixed weights. However, this creates inflation of type I error. To address this problem, use the correlation screening to test each SNP, and set $w_j$ to 1 or -1 (depending on the direction) or 0 (if not passing a certain threshold). The threshold is chosen s.t. under $H_0$, about 10\% of SNPs pass the threshold. 

\item Simulation: consider four different scenarios, with the first threee of GWAS (common variants) setting and the last rare variant setting.  
\begin{itemize}
\item Scenario 1: 21 SNPs in LD, no main effect, two SNPs have interaction effect. Consider three settings $\beta_1 = \beta_2, \beta_1 = -\beta_2, \beta_2 = 0$. 
\item Scenario 2: 20 independent SNPs, each has main effect (e.g. pathway). Two SNPs have interaction effect, and similar setting for $\beta_1$ and $\beta_2$. 
\item Scenario 3: 21 SNPs in LD, no main effect, two SNPs have interaction effect with similar $\beta$. The difference, $E$ is correlated with one of the two interaction SNPs. 
\item Scenario 4: 10 independent SNPs (rare), each has main effect. 2-8 SNPs have interaction effects (could be both positive and negative, the ratio is varied). 
\end{itemize}
In S1-3, compare with min-P and LRT. In S2, compare with the genetic risk score (GRS) method: test the interaction of GRS and $E$. In S4, compare with the simple version of burden test: the burden is simply defined as the sum of rare alleles. 

\item Application: in a cancer data of 10K cases and controls, 25 known SNPs associated with the phenotype. Treat the 25 SNPs as a group and test the interaction with smoking. Found a SNP with low $p$-value (less than 0.01), while the GRS method gives a higher $p$-value (about 0.05). 

\item Analysis: why the method performs better than the simple LRT? Similar to EREC, it uses $w_j$ from the data, thus reducing the df. of the test: in simulations, the naive LRT has df. of 21. The question is that: in these situations, unlike the burden test, there is no obvious aggregation of signals across SNPs (e.g. in S1, only two causal SNPs with opposite signs). Where does the power come from? 
\begin{itemize}
\item Similarlity to James-Stein estimator: better estimator when one tries to simulateously estimate multiple parameters, even if the parameters are completely independent. However, there is no obvious shrinkage? 
\item Bayesian interpretation: $w_j$ is (almost) entirely from the data, thus the model is similar to a noninformative Bayes prior. 
\end{itemize}
\end{itemize}

Test for interactions between a genetic marker set and environment in generalized linear models (GESAT) [Lin \& Lin, Biostatistics, 2013]
\begin{itemize}
\item Model: let $\mu_i$ be the expectation of phenotype $Y_i$, $X_i$ be covariates, $E_i$ be scalar environment variable, $G_i$ be the genotype vector and $S_i = (G_i E_i)^T$ be the interaction term. The model: 
\begin{equation}
g(\mu_i) = X_i \alpha_1 + E_i \alpha_2 + G_i \alpha_3 + S_i \beta
\end{equation}
Use the variance component model, assuming $\beta_j \sim N(0, \tau^2)$, and we are testing $H_0: \tau = 0$. This can be tested using the score test. 

\end{itemize}

Polygenic model of epistasis [Andy Dahl, NHG, 2020]
\begin{itemize}
	\item Defining coordinated epistasis: our phenotype model is
	\begin{equation}
	y \sim \sum_s G_s \beta_s + \sum_{s, s'} (G_s \cdot G_{s'}) \omega_{s,s'} + \epsilon
	\end{equation}
	where $\beta$ are single SNP effect, and $\Omega$ are interaction effects. A simple model is to use a normal prior for $\Omega$ on all SNPs or a subset of SNPs. Our goal is to model the dependency of $\Omega$ on $\beta$: intuitively, for instance, SNPs with large/non-zero $\beta$ should also tend to have large/non-zero interaction terms. Define:
	\begin{equation}
	\gamma(\Omega, \beta) = \text{Cor}_{s \neq s'}(\Omega_{s s'}, \beta_s \beta_{s'})
	\end{equation}
		
	\item Pathway interpretation of coordinated epistasis: imagine phenotype depends on several latent variables $z_k$ and there are interactions among $z_k$'s. Assuming SNPs act on $z_k$, we can see that the interact term of SNPs depends on the interaction between pathways. Ex. suppose there is only interaction term between pathways $k$ and $k'$, then we have:
	\begin{equation}
	\Omega_{s s'} = \beta_s^{(k)} \beta_s^{(k')} \alpha_{k k'}
	\end{equation}
	where the first two terms are SNP to pathway effect and the last term is pathway interaction. 
	
	\item Implications of coordinated epistasis, non-zero $\gamma$ on evolution: deviation of trait distribution from normal in the population. If each SNP segregate independently, we will see an excess of large values of trait.  
	
	\item Inference of $\gamma$: MOM estimator, even/odd chromosome tests. Let $P_E$ and $P_O$ be the PRS of even and odd chromosomes, respectively. Fit the regression model:
	\begin{equation}
	y \sim \beta_E P_E + \beta_O P_O + \gamma P_E P_O
	\end{equation}
	One can prove that $\gamma$ estimates coordinated epistasis. 
	
	\item Results: UKBB 26 traits, several of them show strong signals of coordinated epistasis. 
	
	\item Remark: if the relationship between $z_k$ to the trait is non-linear, e.g. threshold function, it may also cause interactions among SNPs in the same pathway. 
\end{itemize}
%%%%%%%%%%%%%%%%%%%%%%%%%%%%%%%%%%%%%%%%%%%%%%%%%%%%%%%%%%%%
\section{Extensions of Association Analysis}

Bayesian genetic mapping [personal notes]
\begin{itemize}
	\item Statistical background: we have a regression model with the prior: 
	\begin{equation}
	y | \beta \sim N(X \beta, \sigma^2 I) \qquad \beta \sim N(\beta_0, V_0)
	\end{equation}
	Marginalize $\beta$: 
	\begin{equation}
	y \sim N(X \beta_0, \sigma^2 I + X V_0 X^T)
	\end{equation}
	
	\item Individual-level model: we follow the model of BIMBAM. We assume genotype is centered, but not normalized. Suppose the configuration is $\gamma$. We have: 
	\begin{equation}
	y | \beta_{\gamma} \sim N\left(X\beta, \frac{1}{\tau}I\right) \qquad \beta \sim N(0, \sigma_a^2/\tau I_{\gamma})
	\end{equation}
	where $I_{\gamma}$ is the diagonal matrix with diagonal terms given by the indicators $\gamma$. This leads to the marginal likelihood: 
	\begin{equation}
	y | \gamma \sim N\left(0, \frac{1}{\tau} I + \frac{\sigma_a^2}{\tau} X I_{\gamma} X^T\right)
	\end{equation}
	Simlarly, we have $y|\gamma = 0 \sim N(0, I/\tau)$. One can then compute BF for a configuration. 
	
	\item Summary level model: we have $\hat{\beta_j}$ and $s_j$ from unnormalized genotypes, with $\beta_j \sim N(\hat{\beta_j}, s_j^2)$. 
\end{itemize}

Relationship between PVE and prior effect sizes [personal notes]
\begin{itemize}
	\item Individual level data: usually, one uses a regression model: 
	\begin{equation}
	Y = X \beta + \epsilon \qquad \epsilon \sim N(0, 1/\tau)
	\end{equation}
	following BIMBAM. The prior effect size is usually defined on the scale of $1/\tau$, assuming BVSR, we have, for causal variants, $\beta \sim N(0, 1/\tau \cdot \sigma_a^2)$, and $\beta = 0$ for non-causal variants. Suppose genotypes are normalized, the variance explained by a single causal variant is: $\pi \sigma_a^2 \cdot \frac{1}{\tau}$. This leads to the PVE across all $m$ SNPs as: 
	\begin{equation}
	\text{PVE} = \frac{m \pi \sigma_a^2 \cdot \frac{1}{\tau}}{m \pi \sigma_a^2 \cdot \frac{1}{\tau} + \frac{1}{\tau}} = \frac{m \pi \sigma_a^2}{1 + m \pi \sigma_a^2}
	\end{equation} 
	If we define prior effect size on the scale of phenotypic variance, i.e. $\beta \sim N(0, \sigma_a^2 \sigma_y^2)$, where $\sigma_y^2$ is the variance of phenotype, then we have: PVE $= m \pi \sigma_a^2$. 
		
	\item Summary level effect sizes and standard errors: we should assume that effect sizes are defined on the scale of residual variance or $\sigma_y$. Even though they are unknown, they will disappear in PVE and in calculation of statistical evidence (BF of a SNP or a configuration in fine-mapping). We have the same relationship of PVE and $\sigma_a^2$ above. 
	
	\item Summary level $Z$-scores: we have $\hat{\beta}_i \sim N(\beta_i, s_i^2)$, where $\hat{\beta}_i$ and $s_i$ are estimated effect size and standard error, respectively. Let $N$ be sample size and $v_i = 2 f_i (1-f_i)$ be the variance of genotype, we know $s_i^2 = \sigma_y^2 / (N v_i)$. From this, we have the PVE: 
	\begin{equation}
	\text{PVE} = \sum_i \beta_i^2 v_i / \sigma_y^2 = \frac{1}{N} \sum_i \frac{\beta_i^2}{s_i^2}
	\end{equation}
	Now we defined standardized effect size as: $u_i = \beta_i / s_i$, and assume $u_i \sim N(0, \sigma_u^2)$ for causal variants and 0 for non-causal variants. So $\E(u_i^2) = \pi \sigma_u^2$, and we have: 
	\begin{equation}
	\text{PVE} = \frac{1}{N} m \pi \sigma_u^2
	\end{equation}
	So when summary statistics are $Z$-scores, we should consider effect size at the $Z$-score scale, which depends on sample size. 
	
\end{itemize}


\subsection{Basic Fine Mapping}

Background [personal notes]:  
\begin{itemize}
	\item Prior variance and PVE explained: let $\beta_j \sim N(0, \sigma_{\beta}^2 \sigma_y^2)$ be the effect of SNP $j$. If genotype data is normalized (FINEMAP paper), the PVE explained by $j$ is given by:
	\begin{equation}
	\text{PVE}_j = \frac{\Var(\beta_j)\Var(X_j)}{\Var(Y)} = \frac{\sigma_{\beta}^2 \sigma_y^2}{\sigma_y^2} = \sigma_{\beta}^2
	\end{equation} 
	Typically, even a causal SNP explains a small amount of PVE. Ex. default parameter of FINEMAP is $\sigma_{\beta} = 0.05$, or explained PVE 0.25\%. 
	
\end{itemize}

From genome-wide associations to candidate causal variants by statistical fine-mapping [Schaid, NRG, 2018]
\begin{itemize}
	\item Initial GWAS results: LocusZoom plot, lead or index SNPs.
	
	\item Why need fine-mapping? Simulations with 1000 cases and 1000 controls, at effect size 1.5 and AF 50\%,  the causal variant has 79\% chance of being the lead SNP; but at effect size 1.1 and AF 5\%, only 2.4\% chance.
	
	\item LD and SNP statistics: ideally, the causal variant has the lowest p-values and the p-values of other SNPs decay with LD. However, in practice, often not the case. Ex. APOE locus for LOAD, the association statistics change in complex patterns.
	
	\item Remark: this is possible. Consider a relatively large haplotype, there may be recombination events within the block, leading to non-monotonic relation of LD.
	
	\item Forward regression to determine if there are multiple signals in a region: a challenge is to determine the threshold in later steps, some use the same 5E-8, some uses more liberal thresholds of 1E-5 or 1E-4. Limitations of this approach: (1) Multiple testing: if there are m SNPs, then after k steps, $mk$ tests are performed. (2) Low power of detecting secondary signal: dramatic loss of power if the correlation is 0.2 or higher (Figure 3).
	
	\item Heuristic LD approach for fine-mapping: top SNP and all within LD threshold; or all SNPs in the haplotype block of the lead SNP. Limitation: arbitrary threshold; block boundaries not easily defined.
	
	\item Penalized regression: better than forward regression. Limitation: high correlation combined with sparse models reduces the chance of selecting the causal variant.
	
	\item Bayesian approach (Figure 2C): Model marginal likelihood. (1) PIP: caution when there are multiple highly correlated SNPs, then PIP of each SNP may be small. Use the number of causal SNPs, or sum of PIPs (2) Credible set: the minimum set of SNPs that contains all causal SNPs with probability $\alpha$. Significant advantages of Bayesian fine-mapping: PIPs, higher power of mapping SNPs with smaller effects.
	
	\item Trans-ethnic fine-mapping: SNP effects are often consistent across populations. Multiple European ancestries: little gain of power. Incorporating African ancestry helps because of much narrower LD. Trans-ethnic analysis often done with random effect model on summary statistics with METASOFT [Han and Eskins, AJHG, 2011]; followed by Bayesian fine-mapping (Figure 2D).
	
	\item Remark: this approach assumes a single causal configuration. A better approach may be to model possibly different causal variants and effect sizes across populations.
	
	\item Incorporating annotations in Bayesian fine-mapping: most use a small set of annotations. CAVIAR-BF. To assess the performance: compare the size of Bayesian credible set. In PAINTOR paper, reduce from 17 to 13 SNPs per region.
\end{itemize}

Identifying Causal Variants at Loci with Multiple Signals of Association: Cavier [Hormozdiari \& Eskin, Genetics, 2014]
\begin{itemize}
	\item Review of current fine-mapping methods: (1) Single causal SNP assumption. (2) Conditional approach: iterative selection of SNPs, at each step, conditioned on the ones previously chosen. The problem: selection of SNPs in close LD is arbitrary. 
	
	\item Cavier: the idea is to choose causal SNP set that covers most of the posterio prob. Let $S$ be the statistics of SNPs, and $c$ be indicator vector. Assume $P(c)$ follows binomial distribution, and $P(\hat{S}|c)$ is given. The evidence of a $c$ is given by its posterior $P(c|\hat{S})$. Given a set of SNPs $K$, the total posterior prob. of $K$ is given by the sum of posterior of all $c$ consistent with $K$, called confidence level of $K$. Our goal is to choose the minimum set $K$ with confidence level above a threshold. 
	
	\item Algorithm: to reduce computational burden, two ideas (1) Limit to at most 6 causal SNPs per blocks. (2) at each iteration, choose a SNP that increase the posterior prob. the most. 
\end{itemize}

FINEMAP: efficient variable selection using summary data from genome-wide association studies [Benner and Pirinen, Bioinfo, 2016]
\begin{itemize}
	\item Summary statistics on binary traits: use $Z$ scores, the phenotypic variance for quantitative traits is 1; for binary traits, $\sigma^2 = 1/(\phi (1-\phi))$ where $\phi$ is the proportion of cases among all samples.
	
	\item Prior model: let $\gamma$ be the indicator configuration and $\lambda$ be the effect sizes. For a causal variant, $\lambda \sim N(0, s_{\lambda}^2 \sigma^2)$ where $\sigma^2$ is phenotypic variance and $s_{\lambda}^2$ is user-defined parameter, default $0.05^2$ for quantitative trait. The prior for $\gamma$ is: let $p_k$ be the probability of having $k$ causal variants, and each configuration is equally likely given $k$. The distribution of $p_k$ is basically binomial with probability of success $1/m$, where $m$ is the number of variants in the region. 
	
	\item Computational problem of evaluating marginal likelihood: in a region with $m$ SNPs, the likelihood involves $m \times m$ matrix. By factorize the probability into causal and non-causal SNPs, we can reduce the computation to $k \times k$ matrix, where $k$ is the number of causal SNPs.
	
	\item Shotgun Stochastic Search (SSS): once reach a configuration, compute the unnormalized prob. of all neighboring configurations (defined by simple operations); then move to one configuration with probabilities determined by these unnormalized probabilities. All these prob. computation results will be stored for future use (hash table). Comparison with MCMC: once in a dense region, SSS explores the entire neighborhood.
\end{itemize}

Prospects of Fine-Mapping Trait-Associated Genomic Regions by Using Summary Statistics from Genome-wide Association Studies [Benner and Pirinen, AJHG, 2017]
\begin{itemize}
	\item APOE causal variants of LDL-C: true results from in-sample LD, two causal SNPs. With ref. LD from 99 individuals (Finnish from 1000GP): several FP SNPs, and one likely causal variant has PIP close to 0.
	
	\item Intuition why mismatched LD can cause problems: if the external panel underestimates LD, it will try to explain some association as independent causal SNPs (FPs). If it overestimates LD, it will not be able to distinguish two SNPs, so the true causal SNP will have to split evidence with another SNP (both reduced PIP for causal, and FP for another SNP).
	
	\item Simulation study (Figure 4): (1) GWAS of sample size 5000: ref. panel of 100 individuals significantly lower AUC, but 1000 is sufficient; (2) GWAS of sample size 50K (UK Biobank): ref. panel of 100 is not enough; 1000 will have significantly lower AUC; 5,000 will be close to optimal.
	
	\item LDstore software: to share LD.
\end{itemize}

\subsection{Bayesian Statistical Methods} 

Reference: Bayesian statistical methods for genetic association studies [Stephens \& Balding, NRG, 2009]

Motivation for Bayesian methods: 
\begin{itemize}
\item $p$ value does not take into account the power of test (i.e. how likely the alternative hypothesis is). Ex. testing two SNPs, one with high MAF, the other low MAF. For the former, the test is simpler because of larger sample size (if it is associated) thus the same $p$ value should be attached with higher confidence than the latter.

\item Also, to determine a threshold of significance, one should take into account the number of true associations, not the number of tests (our probability should not be affected by how many tests we perform).
\end{itemize}
 
Bayesian testing of a single SNP in GWAS: 
\begin{itemize}
\item Bayes factor (BF) and posterior probability of association (PPA): under Bayesian framework, the quantity of interest is $PPA = P(\text{there is association}|D)$. We first obtain posterior odds (PO): $PO = BF \cdot \pi/(1 - \pi)$, where $\pi$ is the prior probability of association, and the ratio is the prior odds. In general, $\pi$ is taken to be in the range of $10^{-4}$ to $10^{-6}$, therefore BF generally needs to be large, say, greater than $10^5$ to be significant. 

\item Computation of BF: need to compute $P(D|M_1)$ and $P(D|M_0)$ assuming the logistic regression model. 
\begin{itemize}
\item BF for the additive model [WTCCC, Nature, 2007]: suppose $M_0$ denotes the model of no association, $M_1$ denotes the model with an additive effect on the log-odds scale. Then we have: 
\begin{equation}
BF_1 = \frac{P(D|M_1)}{P(D|M_0)} = \frac{\int P(D|\theta_1,M_1) P(\theta_1|M_1)d\theta_1}{\int P(D|\theta_0,M_0) P(\theta_0|M_0)d\theta_0}	
\end{equation}
where the parameters $\theta$ are log. of odds ratio. We use a logistic regression model for the likelihood. Let $N$ be the sample size, $Z_i$, $Y_i$ be the genotype and phenotype of the $i$-th individual respectively, then: 
\begin{equation}
P(D|\theta) = \prod_{i=1}^N p_i^{Y_i} (1 - p_i)^{1-Y_i}	
\end{equation}
where for $M_1$, we have: $\theta_1 = (\mu, \gamma)$, and $\log p_i/(1-p_i) = \mu + \gamma Z_i$; for $M_0$, we have: $\theta_0 = (\mu)$, and $\log p_i/(1-p_i) = \mu$. The prior distributions are specified by $\mu \sim N(\alpha_1, \beta_1)$, in practice, choose $\mu \sim N(0,1)$; and $\gamma \sim N(\alpha_2, \beta_2)$. To evaluate the marginal likelihood $P(D|M)$, use the Laplace approximation, and the result is a function of $\hat{\theta}$ (MAP estimator of $\theta$). 	

\item Averaging genetic models: could consider multiple models: additive, dominant, recessive, and general model. Weighting the models by their prior probabilities: even small weights on non-additive models can allow the identification of large non-additive effects without substantial reduction in the ability to detect near-additive effects. 

\item Averaging effect size: the prior distribution of the effect size (log-OR) is usually assumed to be $N(0,\sigma^2)$ ($\sigma^2 = 0.2$ is a common choice). However, this distribution decays rather fast; to allow for large effect size, replace with a mixture of normal distribution. 
\end{itemize}

\item Dependence of BF: one factor is the MAF, the same $p$ value corresponds to large BF if MAF is large (or the power is high). Another factor is the prior standard deviation ($\sigma$) of the effect size: could have a big impact if MAF is low. 

\item Why Bayesian methods do not require multiple testing correction? Suppose we are testing $n$ SNPs in a region, assume that the global prior odds is fixed (e.g. there is one disease SNP among $n$ SNPs), then the prior odds of each SNP is $1/n$. This amounts to correction of testing multiple hypothesis (as $n$ increases, the BF needs to increase to reach the same level of posterior odds). 

\end{itemize}

Imputation-based analysis of association studies: candidate regions and quantitative traits. (BIMBAM) [Servin \& Stephens, PG, 2007]
\begin{itemize}
\item Model: phenotype is associated with genotype ($q$ SNPs) by a standard regression. 
\begin{equation}
y_i = \mu + \sum_j \beta_{j} x_{ij} + \epsilon_i	
\end{equation}
where $\epsilon \sim N(0, 1/\tau)$. Assume the genetic effects of the $j$-th SNP are: $(0, a_j(1+k), 2a_j)$ for three genotypes, respectively. 

\item Prior distributions: 
\begin{itemize}
	\item Prior for phenotype mean and variance $(\mu, \tau)$: two priors are considered. The $D_1$ prior is defined as $p(\mu, \tau) \propto 1/\tau$. 
	\item Prior for QTNs: $p_0$ is the probability that no QTN is present; $p(l)$ is the distribution that there are $l$ QTNs, in this paper, choose $p(l > 2) = 0$. 
	\item Prior for effect size: assume $a_j \sim N(0, \sigma_a^2/\tau)$. $\sigma_a$ should be relatively small. The paper choose $\sigma_a = 0.5$, i.e. one allele may increase the trait by $0.5 \sigma$. It was suggested in Discussion that a smaller $\sigma_a$ should be used. The default BIMBAM averages BF for four values: $\sigma_a = 0.05, 0.1, 0.2, 0.4$. 
\end{itemize}
		
\item Inference: Let $M_{\gamma}$ be a specific model of which SNPs are associated with the disease, where $\gamma$ encodes $q$-dim vector (each SNP associated with the disease is called QTN). Consider the special case where only 1 QTN is associated, then there are $q$ possible models: $H_j$ - SNP $j$ is the QTN. Then: 
\begin{equation}
BF = \frac{1}{q} \frac{\sum_j P(\bf{y}|\bf{G},H_j)}{P(\bf{y}|\bf{G},H_0)}
\end{equation}
Thus the overall BF is the mean of the single-SNP BFs.  
	
\item ``Bayes/non-Bayes compromise'': use the $p$ value of BF through permutation. 
	
\item Bayesian analysis with imputation: imputation of untyped SNPs using standard methods. The uncertainty of imputation can be incorporated into the inference, e.g. summing over the untyped SNPs in computation of $P(\bf{y}|\bf{G},H_j)$ (where $\bf{G}$ contains only typed SNPs). 
\end{itemize}

Application of Bayesian approach in other tasks: 
\begin{itemize}
	\item Imputation: the uncertainty of imputed SNPs can be taken into account using Bayesian approach. 
	\item Fine mapping: the goal is to determine the truely associated (or causal) SNP(s) in a region of LD. This is a problem of Bayesian variable selection where the predictors may be correlated. The advantage of Bayesian method is: the probabilities of each variable can be quantified. This is often necessary because given the correlations, the true feature can be difficult to determine (the methods such as Lasso will choose a few, but this provides a supferfically simple solution). 
	\item Meta-analysis: the random-effects model, where the effect across different studies/population is treated as random variable. 
\end{itemize}

Polygenic modeling with Bayesian sparse linear mixed models [Zhou \& Stephens, 2013]
\begin{itemize}
	\item BSLMM model: the model can be written as: 
	\begin{equation}
	y | X, \beta \sim N(X \beta, \tau^{-1}) \qquad \beta_j \sim \left[\pi N\left(0, \frac{\sigma_a^2}{\tau}\right) + (1-\pi) \delta_0\right] + N\left(0, \frac{\sigma_b^2}{\tau}\right)
	\end{equation}
	We can write $\beta_j = \tilde{\beta}_j + \gamma_j$ where $\tilde{\beta}_j$ is the sparse genetic component, and $\gamma_j$ the polygenic component. We can rewrite the model as: 
	\begin{equation}
	y = X \tilde{\beta} + u + \epsilon \qquad u \sim N(0, \sigma_b^2 \tau^{-1}K) \qquad \epsilon \sim N(0, \tau^{-1} I_n)
	\end{equation}
	where $K = X X^T / p$ is the GRM (columns of $X$ are centered but not standarized). 
	
	\item Prior specification: the parameters are $\pi, \sigma_a^2, \sigma_b^2$. The prior of $\pi$ is similar to BVSR. For the effect size parameters, we specify the prior in terms of PVE and PGE (the fraction of sparse effects among all genetic variance). Let $h$ be the PVE and $\rho$ the PGE, both have $U(0,1)$ prior. The expected genetic and environmental components are: 
	\begin{equation}
	V_B = \frac{\sigma_a^2}{\tau} p \pi s_a \qquad V_P = \frac{\sigma_b^2}{\tau} s_b \qquad V_E = \frac{1}{\tau}
	\end{equation}
	where $s_a$ is the average variance of SNP genotype, and $s_b$ the average of the diagonal term of $K$. This leads to Equations 16 and 17 in the paper. 
	
	\item Estimation of PVE: comparison of LMM, BSLMM and BVSR. When the number of causal SNPs is relatively large, BVSR performs poorly, significantly underestimating the PVE than LMM and BSLMM. 
	
	\item Prediction of phenotypes: in real data, for some diseases, HT, CAD, T2D and BD, BVSR does not perform well and BSLMM and LMM have similar performance. In other diseases, CD, RA and T1D, BVSR and BSLMM both perform well, better than LMM. 
\end{itemize}

%%%%%%%%%%%%%%%%%%%%%%%%%%%%%%%%%%%%%%%%%%%%%%%%%%%%%%%%%%%%
\subsection{Gene, Pathway and Network Association Test}

Goal: from GWAS data, find pathways/gene modules that are associated with diseases. 

Problems of gene and pathway association test: 
\begin{itemize}
\item Gene/region test: test the significance of a gene/region, combining the evidence of all SNPs in the gene/region. Need to address the bias problem: the size, SNP density, etc. of genes are different.

\item Testing given gene groups: in general, assign $P$-values of each gene in the group, and test the significance of group (e.g. GSEA).
  
\item Finding significant groups in a list of significant SNP/loci: assign genes to the SNPs first and find significant gene groups. The first step may not assign the unique gene to a SNP. 

\item Network test: in the gene network (e.g. PPI) with connectivity information, find significant subnetworks. 
\end{itemize}

Challenges of pathway association test:
\begin{itemize}
\item Gene bias: the size, SNP density and LD structure are different in different genes. Larger genes are more likely to be close to some significant SNPs, so the method of assigning most significant SNPs in the neighborhood to a gene may be biased. Need normalization: convert the statistic of any gene (e.g. its most significant SNP) into some value that normalizes the gene size and LD, e.g. permutation of case and controls and compute the $p$ value of the statistic of that gene [Holmans09].

\item Assign multiple genes adjacent to a single SNP: if this is not allowed, this is probably not realistic; if this is allowed, however, this may skew the results. Ex. one SNP is adjacent to 10 genes that are functionally related, then these 10 genes may show enrichment in GWASPA. 
 
\item Multiple independent association signals/causal variants: not a seriour problem according to [Wang07], as most often if a gene has multiple significant SNPs, they are generally located in the same LD block. However, it is not clear if this is also true for weakly associated SNPs (as multiple rare causal variants should be common). 

\item Heterogeneity of effects: need to consider different cases: (1) one gene has the dominant effect, or (2) a number of genes each has modest effect. The two scenarios are indistinguishable in the test of [Wang09]: e.g. whenever a gene set contains a MHC variant, the gene set is significant, even if the rest of genes are completely random. 

\item Verification of pathways: the same idea of replication may be applied. 

\item Other issues: 
\begin{itemize}
	\item Epistatis of genes in a pathway: not explored. 
	\item Rare variants: how would the analysis extend to rare variants?
\end{itemize}
\end{itemize}

Approaches of gene and pathway tests and analysis: we focus on the analysis using summary statistics, since these are most useful in practice [personal notes]  
\begin{itemize}
\item Difference between gene and pathway tests: under the gene test, there could be strong LD between SNPs, and usually the true number of causal SNPs is small (though not necessarily 1). While at the pathway leve, the statistics of genes are usually independent, and the proportion of causal genes can be relatively large. 

\item Existing approaches to gene test: two general approaches: (1) The top SNP or more generally, top $H$ SNPs. (2) Some combination of the test statistics of all SNPs, including Fisher's method, summation of $\chi^2$ statistic, and so on. Also in general, need to correct for type-1 error.

\item Analysis: depending on the true model, different method may perform better. Ex. if there is a single causal SNP, then the best SNP methods may perform the best. In this case, the sum of $\chi^2$ statistic would likely be dominated by noises, especially when the number of SNPs is large. On the other hand, if there are multiple causal SNPs (each with modest signal), the best SNP method would suffer. 

\item Existing approaches to pathway test: more similar to anaysis of gene sets in gene expression data. Could use top genes, or use overall departure of $p$-value distribution from null (e.g. KS test). 

\item Ideas: explicit model the alternative model, and use Bayesian approach for the gene-level test. 
\end{itemize}

Analysing biological pathways in genome-wide association studies [Wang \& Hakonarson, Nat Rev Genet, 2010]
\begin{itemize}
	\item General procedure: 
	\begin{itemize}
		\item Mapping SNPs to genes: different thresholds
		\item Pruning SNPs: identify independent SNPs from LD regions. 
		\item Gene-based test statistic: often min-P, some methods use multiple markers
		\item Pathway enrichment: distribution of test statistics or hypergeometric test
		\item Adjust for gene size and pathway size: often permutation.
	\end{itemize}
	
	\item Self-contained test: use pathway and null background. Competitive: comparison between pathways. The concern of self-contained test: inflation.
	
	\item Challenges of pathway analysis
	\begin{itemize}
		\item Effect heterogeneity: distinguish the case where one major gene drives the effect. Idea: remove known associated genes and test pathway, e.g. remove TCF7L2 from Wnt signaling, for T2D. 
		\item Bias introduced by permutation: esp. for permutation of p-values. 
		\item Multiple pathways that are non-independent. FWER may be overly conservative. 
	\end{itemize} 
\end{itemize}

Functional and genomic context in pathway analysis of GWAS data [Mooney \& Wilmot, TIG, 2014]
\begin{itemize}
	\item List of existing software for GWAS pathway analysis (Table 1). 
\end{itemize}

Pathway-based approaches for analysis of genomewide association studies.
[Wang \& Bucan, AJHG, 2007], Diverse genome-wide association studies associate the IL12/IL23 pathway with Crohn Disease. [Wang \& Hakonarson, AJHG, 2009]: 
\begin{itemize}
\item Assign $P$-values of genes: Each SNP is assigned to its closest gene if it is within 500 kb upstream/downstream, otherwise, the SNP will not be assigned to any gene. For each gene, choose the most significant SNP, according to the test statistic ($\chi^2$ value), and assign this test statistic to this gene.  

\item Enrichment score (ES): similar to GSEA, test for enrichment of the query gene group in the top-ranked list. Rank all $N$ genes by their test statistic: $r_{(1)}, \cdots, r_{(N)}$, and for any gene set $S$, compute the enrichment score, $ES(S)$, as the weighted Kolmogorov-Sminov like running sum statistic. 

\item To address for possible biases (e.g. gene size bias): do permutation test. In different gene sets, the gene sizes are different, thus some sets may have higher $ES$ scores simply because they have larger genes. Random shuffling of case and control labels and repeat the analysis, calculate the ES of the gene group for any permutation, $ES(S,\pi)$. The score of each group is then normalized by the mean and standard deviation under random shuffling: 
\begin{equation}
Z(S) = \frac{ES(S) - \text{mean}[ES(S,\pi)]}{\text{SD}[ES(S,\pi)]}	
\end{equation}
	\item Multiple hypothesis testing correction: the FDR for a threshold $Z^*$ is computed as: 
\begin{equation}
FDR = \frac{\% \text{ of } (S,\pi) \text{ with } Z(S,\pi) \geq Z^*}{\% \text{ of observed } S \text{ with } Z(S) \geq Z^*}	
\end{equation}

\item Results: Parkinson disease: new glycan-related pathways. The relationship between glycobiology and nuero-degenerative disease has been reported recently. 
\end{itemize}

Pathway analysis of seven common diseases assessed by genome-wide association [Torkamani \& Schork, Genomics, 2008]
\begin{itemize}
	\item Assign $P$-values of genes: Each SNP was assigned to a gene within 5 kb, if mapped to multiple genes, assign to a single one according to coding $>$ intron $>$ upstream, etc. And for each gene, choose the most significant SNP.
	\item Enrichment test: take top 2.5\% genes and use hypergeometric test for each gene group. 
	\item Results: WTCCC data. (1) Bipolar disorder: sulfotransferase, HDL metabolism, glutamate receptors; (2) Hypertension: dopamine signaling, PKA and cAMP signaling pathways, calineurin signaling (regulation of lipid metabolism), cell-cell interactions and cytoskeletal remodeling. 
	\item Biological lessons from pathway analysis: 
	\begin{itemize}
		\item Most of seven diseases have weak risk factors in general signaling pathways: G-protein, cAMP, calcium signaling, etc. Also TFs, transcriptional regulatory factors, membrane recetpros are common risk factors. 
		\item Each disease often has multiple aspects: metabolic, neurological or inflmamatory. Ex. Bipolar disorder: sulfotransferase is related to clearing of dopamine in extracellular space; Alzheimer's disease: ApoE. 
	\end{itemize}
\end{itemize}

Pathway and network-based analysis of genome-wide association studies in multiple sclerosis [Baranzini \& Barnes, HMG, 2009]
\begin{itemize}
	\item Assign $P$-values of genes: lowest $P$ value of all SNPs mapping to a given gene. Different ways of combining multiple $P$ values of SNPs to a gene-wise $P$ value are tested, including Fisher's method of combining $P$ values, and a method that corrects for the number of SNPs and for LD, but none performed better. 
	\item Enrichment test: load the $P$ values of genes to PPI network, and search for subnetworks where significant genes are overrepresented [Ideker02]. The modules are then tested for GO terms and KEGG pathways. 
\end{itemize}

[Elbers \& Onland-Moret, GenetEpid, 2009]
\begin{itemize}
	\item Assign $P$-values of genes: SNPs are mapped to haplotype blocks (defined by $r^2 > 0.25$). Within a block, most have multiple genes. To find the best genes within each block: use a reference gene network and find genes related to genes in other selected regions with an interaction $P$ value below 0.05. 
	\item Enrichment test: once genes are assigned to pathways, use the standard tools to test if genes in a certain pathway are overrepresented in the list of candidate genes. 
	\item Results: 
	\begin{itemize}
	\item DGI and WTCCC data of T2D, with a weak threshold on individual SNPs ($P < 0.003$). The remaining about 2,000 SNPs were then used for analysis: gene assignment and pathway enrichment test. The results from the two datasets overlapped in pathways, but frequently different genes in the same pathway were picked. Thus may be difficult to replicate, but shared pathways. 
	\item Randomly selected genes: usually result in significantly overrepresented pathways. 
	\item Possible biases: large genes, genes in large LD blocks, pathways with more genes. That large pathways are preferred may also come from the fact that with larger set, statistical evidence is stronger. Permutation and bootstrapping should improve the results. 
\end{itemize}
\end{itemize}

[Holmans \& Craddock, AJHG, 2009]
\begin{itemize}
	\item Assign $P$-values of genes: a SNP is assigned to a gene if it lies within 20kb of 5' or 3' of that gene. Allow a SNP to be assigned to multiple genes.
	\item Enrichment test: [Holmans09-Fig1]
\begin{itemize}
	\item Test statistic: first define the list of significant genes (use all significant SNPs, and extract the associated genes), of size $N$. For each gene list, the test statistic $T$ is the number of significant genes in that list.  
	\item Statistical significance: randomly sample $N$ genes (by random sampling of SNPs) $M$ times, and count $T$, the number of significant genes in the test list, in each of the $M$ replicates. 
\end{itemize} 
\end{itemize}

[Yu \& Chatterjee, GE, 2009]
\begin{itemize}
\item Combination of $P$ values: (test the null hypothesis that none is significant) the rank-truncated product (RTP) method: 
\begin{equation}
W(K) = \prod_{i=1}^K p_i	
\end{equation}
where $K$ is a predetermined integer (truncation point). The adaptive RTP (ARTP) method chooses the $K$ that gives the minimum $P$-value. The significance of the ARTP statistic needs to be assessed by permutation test to account for the multiple testing of different values of $K$. 

\item Strategy of pathway association: (1) obtain the gene-level $P$ values using ARTP; (2) combine the gene-level $P$ values of the pathway using ARTP. If step (1) uses the $P$ value based on known null distribution, then no permutation on (1) is needed; otherwise, would need multi-level permutation to assess the significance of the final statistic. 
\end{itemize}

Logistic kernel machine model [Wu \& Lin, AJHG, 2010]: 
\begin{itemize}
	\item Logistic regression of the trait on all SNPs in a gene, with a kernal function. The kernel can measure the similarity of genotypes of individuals. The df. is determined s.t. the high LD region has a lower df. 
\end{itemize}

A Versatile Gene-Based Test for Genome-wide Association Studies (VEGAS) [Liu \& Macgregor, AJHG, 2010]
\begin{itemize}
\item Test statistics for a gene: let $\chi_i^2$ be the 1-df chi-square statistic of the $i$-th SNP of the gene, the gene-level test statistic is simply $\sum_i \chi_i^2$. 

\item Determining null distribution: use simulation to obtain the null sample, then calculate the test statistic. To get the null sample, generate MVN sample $Z \sim N(0, \Sigma)$: this is accompolished by first sample independent standard normal random vectors, then multiply that by $C$, where $C$ is the Cholesky decomposition of $\Sigma$: $CC^T = \Sigma$. 
\end{itemize}

MISA: Bayesian model search and multilevel inference for SNP association studies [Wilson \& Schildkraut, Ann Applied Stat, 2010]
\begin{itemize}
\item Model: phenotype is associated with genotype ($q$ SNPs) by a standard regression. Let $M_{\gamma}$ be a specific model of which SNPs are associated with the disease, where $\gamma$ encodes $q$-dim. vector. The BF is given by: 
\begin{equation}
BF(H_A:H_0) = \sum_{M_{\gamma}}	BF(M_{\gamma}:H_0) P(M_{\gamma}|H_A)
\end{equation}
The posterior probability of a single SNP, or a gene, can be marginalized from $P(M_{\gamma})$. 
	
	\item Prior: BetaBinomial distribution of the size of $M_{\gamma}$. Choose: BetaBinomial$(1, \lambda S)$ in this work where $S$ is the total number of SNPs and $\lambda$ is a parameter, the global prior odds are $1/\lambda$. At $\lambda = 1$: the global prior odds of there being at least one association of 1, and the marginal prior odds of any single SNP of $1/S$. 
	
	\item Inference: for any model, define its fitness as the log of the (unnormalized) posterior probability. Sample models based on their fitness using the Evolutionary Monte Carlo algorithm. 
	
	\item Data: genotype only in candidate genes, 2129 women at 1536 SNPs in 170 genes on 8 pathways. 
\end{itemize}

Gene-Based Tests of Association [Huang \& Arking, PLG, 2011]: 
\begin{itemize}
\item Motivation: Gene-based and related multi-marker association tests have generally under-performed single-locus tests when assessed with real data. A general drawback of methods that attempt to exploit the structure of LD to reduce the number of tests, for example through PCA, is the loss of power to detect low-frequency alleles. Methods that consider multiple independent effects often require that the number of effects be pre-specified, which loses power when the tested and true model are different.

\item Methods:
\begin{itemize}

\item A model $M$ is defined as the subset of $K$ SNPs in a gene with $P$ total SNPs that are permitted to have non-zero regression coefficients. The prior $P(M)$ assumes that each of the SNP is equally likely to be causal ($f$). The effective number of SNPs ($T$) is smaller than the number of SNPs because of LD. Equation (5). 

\item The model likelihood: $P(Y|M,X)$. Integration over $\beta$: dependent on the SSE at MLE of $\beta$ (similar to Laplacian approximation?) Integration over $\tau$ ($1/\sigma^2$): Gamma function or steepest descent approximation (?). The final log-likelihood is given by Equation (10): it is a function of $N$, $K$, $\Sigma$ (genotype matrix), MLE of $\sigma$, $A$ and $B$ (the limit of $\tau$ and $\beta$ respectively). (Note: SSE is absorbed to MLE of $\sigma^2$.)

\item GWiS strategy: find $M$ that maximize $P(M|X,Y)$, Equation (12). The terms involving $K$ provide a Bayesian penalty for model performance, but also make this an NP-hard optimization problem

\item Optimization: First is a greedy forward search, essentially Bayesian regularized forward regression, in which the SNP giving the maximal increase to the model likelihood is added to the model sequentially until all remaining SNPs decrease the likelihood. The second is a similar heuristic, except that the initial model searches through all subsets of 2 SNPs or 3 SNPs. 

\item GWiS is designed to select a single model for each gene. An alternative related approach would be to test for the posterior probability of the null model against all other models (BIMBAM). Unfortunately, the number of terms increases exponentially fast with model size, and the brute-force approach does not scale to genome-wide applications. 

\item Models tested: minSNP - p-value for the best single SNP; BIMBAM - very similar to minSNP; VEGAS; LASSO - genes with at least one SNP selected. 

\end{itemize}

\item Results:
\begin{itemize}

\item The performance of GWiS depends on the genetic architecture of a disease or trait: higher power if genes house multiple independent causal variants, and lower power if each gene has only a single causal variant. In practice, the loss of power was so slight as to be virtually undetectable.

\item Of the other methods, minSNP-P and BIMBAM had similar performance that degraded as the true model included more SNPs. The VEGAS test did not perform well, and the LASSO method performed worst.

\end{itemize}

\item Discussion: 
\begin{itemize}

\item By gathering multiple independent effects into a single test, GWiS has greater power than conventional tests to identify genes with multiple causal variants. GWiS also retains power for low-frequency minor alleles. 

\item Bayesian methods can be computationally expensive. GWiS minimizes computation by evaluating only the locally optimal models of increasing size in a greedy forward search. This appears to be an approximation compared to previous Bayesian methods that sum over all models

\item Why VEGAS does not perform well? Presumably because the sum over all SNPs creates a bias to find causal variants in LD blocks represented by many SNPs and to miss variants in LD blocks with few SNPs. Also VEGAS is sensitive to low frequency alleles: its power drops two fold when MAFs drop from 50\% to 5\%. Reason: low-frequency SNPs lack correlation with other SNPs, reducing the contribution to the VEGAS sum statistic. 

\item Why LASSO does not perform well? Also sensitive to low-MAF SNPs. Lasso shrinks the regression coefficients, thus a SNP with large regression coefficients but low MAF may be missed. For the same variance explained, this SNP is penalized more than SNPs with smaller regression coefficients but higher MAF (not a problem for GWiS whose penalty is based on the number of selected SNPs). 
\end{itemize}

\item Remark: 
\begin{itemize}
\item Lesson: approximation of model likelihood in Bayesian: Laplacian approximation? 

\item Lesson: searching for models is an important bottleneck of Bayesian approach, heuristic search. 

\item Need to check how well the method performs with rare variants. 

\end{itemize}
\end{itemize}

MAGENTA: Common Inherited Variation in Mitochondrial Genes Is Not Enriched for Associations with Type 2 Diabetes or Related Glycemic Traits [Segre \& Altshuler, PLG, 2010]
\begin{itemize}
	\item Gene-level statistics: (1) MinP, and obtain gene score (Z-score); (2) Control for confounders: regression of gene scores vs. possible confounders e.g. gene-size. Step-wise multiple linear regression. 
	
	\item Gene set enrichment: variation of GSEA. (1) Number of genes with p $<$ a threshold, use 95 percentile. (2) Gene set p-value: fraction of genes with p $<$ threshold for a gene set, and compare with random gene sets of the same size. 

\end{itemize}

GATES: A Rapid and Powerful Gene-Based Association Test Using Extended Simes Procedure [Li \& Sham, AJHG, 2011]
\begin{itemize}
\item Motivation: problems of existing approaches for gene-based test, (1) best single-SNP statistic: not normalize by gene size. (2) Fisher's method (or similar) to combine test statistics of SNPs: independence assumption not hold, thus need simulations to get correct type I error. 

\item Simest test: let $p_{(1)}, \cdots, p_{(m)}$ be the $p$-values of the $m$ SNPs of a gene, we define the gene-leve $p$-value as: 
\begin{equation}
p_G = \min_j \frac{m_e p_j}{m_{e(j)}}
\end{equation}
where $m_e$ is the effective number of SNPs, and $m_{e(j)}$ is the effective number of SNPs among the top $j$ SNPs. 

\item Incorporating prior weights of $p$-values: similar to the Simes test except that:
\begin{equation}
p_G = \min_j \frac{m_e p_j}{\sum_k^j w_{(k)} }
\end{equation}
where $w_{(k)}$ are non-negative weights summed to $m_e$. 
\end{itemize}

Incorporating Biological Pathways via a Markov Random Field Model in Genome-Wide Association Studies [Chen \& Zhao, PLG, 2011]: 
\begin{itemize}
\item Idea: if a gene is associated with a disease, other genes in the same pathway tend to be associated as well, since both would disrupt the same pathway. Furthermore, genes that are directly linked in a network should have the same tendancy of association. 

\item MRF model: given $n$ genes in a network $G = (V,E)$ (edges given), suppose $S_i$ is the association status (binary) of the $i$-th gene, we first define the prior distribution of $S$ under $G$: 
\begin{equation}
P(S|\theta_0) = \frac{1}{z(\theta_0)} \exp\left[ h \sum_i I_1(S_i) + \tau_0 \sum_{(i,j) \in E} (w_i + w_j) I_{-1}(S_i) I_{-1}(S_j) +  \tau_1 \sum_{(i,j) \in E} (w_i + w_j) I_{1}(S_i) I_{1}(S_j) \right]	
\end{equation}
where $\theta_0$ are hyperparameters, and $I_1(S_i)$, $I_{-1}(S_i)$ are indicator variables, $w_i = \sqrt{d_i}$ is the square root of the degree of the $i$-th gene. A typical parameter setting $(h, \tau_0, \tau_1) = (-1, 0.25, 0.01)$ (used in simulation) would penalize a lot of associations (prefer simpler model), and reward the edges connecting two associated genes. And the degree parameters $w_i$ would put more influence on highly connected genes. 

\item Posterior distribution: suppose we have the gene-level statistics $y_i$ (in the experiment, derived from PCA regression at gene level). Assume given $S_i = 0$, $f_0(y_i) \sim N(0,1)$. Further assume that $f_1(y_i)$ follows a normal distribution $N(\mu_i, \theta_i^2)$, with conjugate priors of $\mu_i$ and $\sigma_i^2$, we could obtain the distribution of $y_i$ under hyperparameters $\theta_1$. The posterior distribution of $S$ given $y$ is: 
\begin{equation}
P(S|y, \theta_0, \theta_1) \propto f(y|S, \theta_1) P(S|\theta_0)	
\end{equation}

\item Parameters: values chosen based on simulations. However, they could be estimated with an empirical Bayes approach. The marginal likelihood is: 
\begin{equation}
L(\theta_0, \theta_1|y) = \sum_S f(y|S,\theta_1) P(S|\theta_0)
\end{equation}
Thus choose parameters to maximize $L$. In practice, summing over $S$ is slow, so do maximization over $S$. 

\item Inference: Maximize the posterior distribution $S$. This could be done via maximizing the conditional probability (MCP), the label of $S_i$ given observed data and labels of all the other nodes. FDR control: direct posterior probability approach. 

\item Experiment: 
\begin{itemize}
	\item Simulation: given a network, and select genes for labels. Sample genotypes and phenotypes (require parameters of MAF and effect size). The performance is evaluated by the AUC of classifying genes. 
	\item Crohn's disease: compare the method (posterior means) with the $p$-value based approach for ranking genes. Improved AUC (Figure 6). 
\end{itemize}

\end{itemize}

Integrating genetic and gene expression evidence into genome-wide association analysis of gene sets [Xiong \& Furey, GR, 2012]:
\begin{itemize}
\item Idea: Pathway analysis. Give more weights to genes that are differentially expressed in cases vs. controls. 

\item Methods: for any gene, define (1) SNP set association score: maximum statistics and (2) differential expression score. The gene association score is some combination of these two scores (e.g. $Z$-score sum). The pathway association score is then defined as the weighted K-S statistic. 
\end{itemize}

Integrated Enrichment Analysis of Variants and Pathways in Genome-Wide Association Studies Indicates Central Role for IL-2 Signaling Genes in Type 1 Diabetes, and Cytokine Signaling Genes in Crohn's Disease [Carbonetto \& Stephens, PLG, 2013]
\begin{itemize}
\item Testing enrichment of a single pathway: we do the analysis one pathway a time. For each SNP, let $a_j$ be its annotation (of the test pathway, binary), the inclusion prob. (prior) of SNP $j$ in association studies is: $\text{logit}(\pi_j) = \theta_0 + a_j \theta$. To test if $\theta > 0$, we use BF by comparing two models: $\theta > 0$ vs. $\theta = 0$. 

\item Testing enrichment of a combination of pathways: after the first, choose multiple pathways, then test combination of pathways. Only OR logic is considered, i.e. set $a_j$ to 1 when SNP $j$ is assigned to at least one of the enriched pathways. 

\item Use pathway enrichment to redo association analysis: after we have enrichment pathways (or combinations), we compute Posterior inclusion probability (PIP) of each SNP, taking into account the data and prior (annotations). For each SNP, we have multiple PIPs, one from each enrichment pathway. The PIPs are combined by averaging, weighted by BFs. 

\item Remark: limitations
\begin{itemize}
	\item Only consider OR logic when combining pathways. In practice, it is likely that AND of pathways is more important. 
	\item Combine PIPs from multiple pathways: the results can be easily dominated by single enriched pathway (weighted by BF). The average PIP is never more than each of the PIPs, thus a gene with multiple enriched pathways does not benefit from multiple annotations. 
\end{itemize}

\end{itemize}

Detecting Association With Network (DAWN) [Li Liu thesis defense]: 
\begin{itemize}
\item Network construction from co-expression: partial neighborhood selection (PNS) algorithm. First choose a set of likely candidates: nodes with small $p$-values from TADA, further filter out isolated notes. Next, for each node, choose its neighbors using Lasso (predict the expression of this node). Choose regularization parameter by fitting the power law degree distribution (the best value that leads to scale-free network). 

\item Hidden MRF (HMRF) model: the Ising prior, however, only reward when both nodes are risk genes and no penalty for non-risk neighbors. 

\item Remark: 
\begin{itemize}
\item PNS algorithm: miss the correlated genes. 
\item Scale-free to select network parameters: unproven? 
\item HMRF: no penalty term, thus not normalize the degree of nodes. 
\end{itemize}
\end{itemize}

A Method for Gene-Based Pathway Analysis Using Genomewide Association Study Summary Statistics Reveals Nine New Type 1 Diabetes Associations [Evangelou \& Wallace, GE, 2014]
\begin{itemize}
\item Goal: gene test and pathway test using only summary statistics. 

\item Gene test: the statistic has two alternatives. (1) Minimum $p$-value; (2) Fisher's method of combining $p$-values: 
\begin{equation}
FM = -2 \sum_j \log p_j
\end{equation}
In both cases, using MVN to sample the $p$-values of SNPs under $H_0$. 

\item Pathway test: let $r_i$ be the ranks of the genes (divided by the total number of genes), Fisher's method: $-2 \sum_i \log r_i$. An alternative is Adaptive rank truncated product method (ARTP): similar to Fisher's method except that only the top $H$ gene will be used. 
\end{itemize}

Fast and rigorous computation of gene and pathway scores from SNP-based summary statistics [Lamparter \& Bergmann, review for AJHG, 2015]
\begin{itemize}
\item Motivation: both gene-based test and pathway-based test. For gene-based test, VEGAS is a popular method but it relies on simulation to obtain $p$-values. For pathway-based test, existing methods rely on threshold and hypergeomic test, which depends on the threshold parameter. 

\item Gene-based test: the basic idea is to obtain the analytic distribution of VEGAS test. Specifically, let $Z_i$ be the $Z$-score of the $i$-th SNP, then under null, 
\begin{equation}
Z \sim N(0, \Sigma)
\end{equation}
where $\Sigma$ is given by the LD. We form the test statistic as the sum of $\chi^2$ over all SNPs: 
\begin{equation}
T_{sum} = \sum_i z_i^2
\end{equation}
To obtain its distribution, the idea is to conver the vector $z$ to a vector of independent random variables. Let the eigenvalue decomposition of $\Sigma$ be $\Sigma = \Gamma \Lambda \Gamma^T$, then we define
\begin{equation}
y = \Lambda^{-1/2} \Gamma^T z
\end{equation}
It is easy to show that $y \sim N(0, I_n)$ ($\Lambda^{-1/2}$ is for scaling). We have: 
\begin{equation}
\sum_i z_i^2 = z^T z = z^T \Gamma \Gamma^T z = y^T \Lambda y \sim \Sigma_i \lambda_i \chi^2_1
\end{equation}
where $\lambda_i$ is the $i$-th eigenvalue. The result is a weighted sum of independent $\chi^2$ distribution. Another test statistic is: 
\begin{equation}
T_{max} = \max_i z_i^2
\end{equation} 
Its distribution can be determined by: $T_{max} \geq t$ iff $z_i \geq \sqrt{t}$ for each $i$, so this amounts to a rectangular integration over multivariate normal. 

\item Pathway test: first make the gene-level statistic independent. To do that, do ``gene-fusion'', i.e. treat close genes (or LD) as a single ``fusion gene''. Next, to determine the pathway score, convert the $p$-value of each gene to $\chi^2_1$ statistic, then use the sum as the test statistic of the pathway. Two variations: chi-squared-method or empirical sampling method. The difference is likely: (1) chi-squared-method assumes independence of genes (so that the null distribution is valid); (2) gene-score p-values are determined empirically, instead of from theoretical null (more conservative). 

\item Results: fusion method avoids the inflation problem (using simulated random phenotype data to show that) of existing pathway tests. The pathway test also is more powerful than existing pathway tests, hypergeometric or rank-sum: using replication type of analysis (the truth from a larger dataset), and using the number of significant pathways as the metric. 

\item Lessons: (1) The analytic distribution of sum of chi-square: convert to independent random variables; (2) Binary test loses information and power. 
\end{itemize}

Biological interpretation of genome-wide association studies using predicted gene functions (DEPICT) [Pers and Franke, NC, 2015]
\begin{itemize}
	\item Reconstitute gene sets: about 14,000 gene sets, each gene has a membership probability.
	
	\item Gene prioritization: get a set of genes in the trait-associated loci $S$. For each gene, a vector representing its membership probability of all gene sets. Then scoring one gene: correlation of the membership prob. vector with all genes in $S$.
	
	\item Gene set enrichment test: for any gene set, sum over the membership over all genes in $S$, then test the significance.
	
	\item Remark: a poor man’s way of doing Bayesian hierarchical model: learn likely gene sets from all trait-associated loci; then score a gene by its similarity with likely genes.
\end{itemize}

MAGMA: Generalized Gene-Set Analysis of GWAS Data [de Leeuw, PLG, 2015]
\begin{itemize}
	\item Procedure: (1) Gene level analysis: obtain gene p-values. (2) Pathway analysis: use gene p-values and gene correlation matrices (accounting for LD between genes).
	
	\item Gene-level analysis using individual level data: for SNPs in a gene, do PC first, and use the top PCs to represent the gene. Regression of PCs with phenotype, and use F-test for the gene, H0: no PC is associated with phenotype.
	
	\item Gene-level analysis using summary statistics: use mean or max. chi-square. For mean $\chi^2$: an approximate distribution is available. For max $\chi^2$: use permutation - permute phenotype labels and do association.
	
	\item Gene set analysis: let $Z$ be Z-score of a gene, regression of $Z$ vs. features of genes. Self-contained: use only one gene set a time, test if deviation from $N(0,1)$. Competitive test: use gene set as a feature, do regression analysis.
	
	\item Generalized gene set analysis: default in MAGMA, testing gene set, conditioned on gene size, gene density (number of SNPs, or number of genotype PCs).
	
	\item To account for LD between genes, the error model of $Z$ are correlated, using Generalized Least Square. $\epsilon \sim MVN(0, \sigma^2 R)$. The correlation matrix $R$: approximated by using the correlations between the model sum of squares (SSM) of each pair of genes from the gene analysis multiple regression model, under their joint null hypothesis of no association.
	
	\item Lesson/Remark: dimensionality reduction in testing variable set association. What are alternative approaches? E.g. how to do testing with Lasso or Bayesian? Perhaps G-prior for Bayesian variable selection?
\end{itemize}

Pathway analysis using RSS [Xiang Zhu, 2015]
\begin{itemize}
	\item RSS model: the likelihood is given by RSS model: 
	\begin{equation}
	\hat{\beta} | \beta, S, R \sim N(SRS^{-1}\beta, SRS)
	\end{equation}
	The model is applied to LD blocks. Use large LD blocks with shrinkage estimation s.t. the LD matrix is block-diagonal. The LD blocks tend to be large: some many include $>100$ genes, a total of about 1,000 blocks. 
	
	\item Testing association of pathway: let $\beta_j$ be the true effect of variant $j$. Use sparse parior 
	\begin{equation}
	\beta_j \sim (1- \pi_j) \delta_0 + \pi_j N(0, \sigma^2), \quad \text{logit}(\pi_j) = \theta_0 + a_j \theta 
	\end{equation}
	where $a_j$ is the pathway annotation of $j$: 1 if $j$ belongs to the pathway being test, and 0 otherwise. The results are $P(\theta|\hat{\beta}, S, R, a)$, the BF of pathway (if $\theta = 0$) and evidence of each SNP/gene $P(\beta_j|\hat{\beta}, S, R, a)$.  

	\item Computational problem: need to integrate out $\beta_j$, but this cannot be done analytically, so rely on Variational Bayes.
	
	\item Analysis of height GWAS: $\theta_0 = 2.0$ (prior 0.01) and $\theta = 0.75$ (OR $=5.6$) for the top pathways. Note: compare this with Carbonette \& Stephens: the posterior was inflated, $\theta_0 = 0.001$ and $\theta=10^3$ for T1D. 
	
	\item Multiple pathways: if a gene/SNP belongs to multiple pathways, only use the strongest pathway, instead of additivity assumption. 
\end{itemize}

\subsection{Other Tests}

A Novel Test for Recessive Contributions to Complex Diseases Implicates Bardet-Biedl Syndrome Gene BBS10 in Idiopathic Type 2 Diabetes and Obesity [Lim \& Daly, AJHG, 2014]
\begin{itemize}
\item Model: our data are the counts of aa in cases and in controls (also the counts of non-aa). The expected frequency of aa in cases depends on the population frequency of aa and the relative risk. We derive this expected frequency for cases (controls are simpler), then perform the LRT using binomial distribution. In the LRT, the parameter $\gamma$ (RR) needs to be determined. 

\item Estimating $P(aa)$: simplest case, $P(aa) = P(a)^2$. Two exceptions:
\begin{itemize}
	\item Population substructure that causes systematic departure from HWE: $P(aa) = F P(a) + (1-F) P(a)^2$. 
	\item Local departure from HWE due to hemizygous deletions or systematic genotyping errors. 
\end{itemize}
If the observed rate of $aa$ in control subjects exceeds the expected corrected $P(aa)$, simply use the observed rate (to be conservative). 

\item Remark: the power comes from estimated $P(aa)$. Normally this would be $P(a)^2$, which is very small for rare variants. 
	
\item Questions: the parameters under the model: (1) population frequency (would be a problem for rare variants), e.g. aa might have 0 count in controls. (2) Relative risk: maximization? (3) Why need EM? 
	
\end{itemize}

Benchmarker: An Unbiased, Association-Data-DrivenStrategy to Evaluate Gene Prioritization Algorithms [Fine and Hirschhorn, AJHG, 2019]
\begin{itemize}
	\item Background: limitations of current approaches for evaluation of gene prioritization algorithms. (1) Gold standard gene sets: often hard to obtain, and may be biased towards well-characterized genes. (2) Independent GWAS datasets: not always available.
	
	\item Benchmarker strategy (Figure 1): compare several methods, use LDSC to estimate the enrichment of heritability in the prioritized genes. (1) Obtain list of prioritized genes: take one chromosome (e.g. chr1), train on all other chromosomes, then prioritization on chr1 (correlation of membership vector vs. genes in trait-associated loci in other chr's). (2) For all prioritized genes: assess enrichment of h2  using S-LDSC. Use SNPs near genes.
	
	\item Results: three variations of DEPICT, performance of gene sets $>$ GEO co-expression and GTEx co-expression. Relatively small overlap. Intersection of genes by multiple methods performs better.
	
	\item Comparison of DEPICT vs. MAGMA: similar performance, but gene overlap is not high. Intersection much better than each alone.
	
	\item Comparison with NetWAS: NetWAS uses PPI network to prioritize genes. Both DEPICT and MAGMA significantly better than NetWAS.
\end{itemize}

%%%%%%%%%%%%%%%%%%%%%%%%%%%%%%%%%%%%%%%%%%%%%%%%%%%%%%%%%%%%
\section{Incorporating Variant Annotations}

Strategies of testing and incorporating annotations [personal notes]
\begin{itemize}
	\item Basic strategy: if a feature is predictive of causal variants, then the variants with this feature should generally have higher signifiance in GWAS than those without this feature. 
	
	\item Complication: (1) LD across variants; (2) bette to compare the distribution of summary statistics: enrichment analysis often requires discretization. 
	
	\item Analysis of how LD may affect the enrichment analysis (in the general sense): two scenarios where LD can lead to inflation of enrichment: 
	\begin{itemize}
		\item A causal SNP in an enhancer, and the enhancer contains multiple SNPs in LD. Then a causal SNP may contribute multiple times in computing the fold enrichment. 
		
		\item A SNP in an enhancer, and in LD with a causal SNP (not in an enhancer): proper analysis should remove the enhancer-SNP as its effect is due to causal SNP. 
	\end{itemize}
	
	\item Statistical strategy: general idea is that the annotations increase the prior of a variant (indicator or effect size). And the likelihood of data (summary statistics) depends on the indicators or effect sizes.  
	
	\item Polygenic model: the first scenario, the model would assume that the enhancer contains multiple causal SNPs, so each one would receive lower effects. In the second scenario, the model would infer that the true causal SNP (not in enhancer) has larger effect, and that may be due to LD with enhancer-SNP. The model does allow one to explain ``away'' SNPs in LD with causal SNPs that are in enhancers.
	
\end{itemize}

Review of methods [personal notes]
\begin{itemize}
	\item fgwas: one causal variant per LD. 
	\item PAINTOR: use only associated loci. Model distribution of test statistics (Z-scores) as MVN.  
	\item LD-score regression: annotations change effect size variance. Cannot model penalty. 
	\item CAVIAR: only fine-mapping, finding the interval that covers all causal variants with high probability. 
	\item CAVIAR-BF: similar to CAVIAR, combine the idea of BIMBAR. 
	\item Schaid paper: similar to PAINTOR, with different prior for causal indicators (L1, L2 and elastic net). 
	\item Kellis group [Yue Li]: logistic prior, the innovation is cross-trait. However, only capture cross-trait correlation via annotation weights. Should model how effect sizes are correlated.
	\item PICS.   
\end{itemize}

Enriching the analysis of genomewide association studies with hierarchical modeling: [Chen \& Witte, AJHG, 2007]
\begin{itemize}
	\item Idea: for each SNP, also consider its extra information: conservation, functional category, etc, which can weight SNPs. To assess the effect of the extra information, use a hierarchical model where the effect of a SNP depends on these general evidences. 
	\item Model: for $M$ SNPs, let $\beta_m$ be the effect of the $m$-th SNP (regression coefficient). Then $\beta_m$ depends on the $K$ factors, including conservation (PhastCons score), the functional category (mRNA, UTR, or intron, etc.), and all these types of information in adjacent SNPs. Model $\beta_m$ through a regression: 
	\begin{equation}
	\beta_m = \sum_{j=1}^K \pi_j Z_{mj} + u_m
	\end{equation}
	where $\pi_j$ is the effect of the $j$-th factor, $Z_{mj}$ is the $j$-th factor of the $m$-th SNP, and $u_m$ follows normal distribution with mean 0. The variance of $u_m$ thus indicates heterogeneity of $\beta_m$. The model can be extended to allow dependence among SNPs: $u_m$ are no longer independent, instead $U$ follows a multivariate normal distribution. 
	\item Result: test association of SNPs and gene expression, choose a few genes where the causal SNPs are known (from functional annotation, e.g. in promoters). Compare the ranking of SNPs from single-marker analysis and the current method: those close to the true causal SNP should rank higher in the new method. 
	\item Remark: the hierarchical model here is unusual, $\beta_m$ is modeled separately for each SNP. A probably better way is the model in [Veyrieras \& Pritchard, PG, 2008]
\end{itemize}

Disease gene identification by PPI [Lage \& Brunak, NBT, 2007]: 
\begin{itemize}
	\item Goal: rank candidate genes in the linkage interval of diseases. 
	
	\item Methods: 
	\begin{itemize}
		\item Idea: if a protein X is involved in a disease, then a protein interacting with X may be involved in a related disease.
		\item Data: training data - 963 genes involved in 1,404 distinct phenotypes. For each phenotype, construct a linkage interval (randomly chosen the size s.t. the average number of genes in one interval matches the data in the application data). Application data: 870 intervals with no candidate genes assigned. 
		\item Disease similarity: from literture mining - similarity of the associated profiles of semantic terms (e.g. UMLS). 
		\item Gene scoring in a linkage interval: for each gene, extract all interacting partners, and score a gene according to the number of partners that are known to be associated with similar diseases (thus considering both PPI strength and disease similarity). The score is a Bayesian probability that the gene is involved in the disease given the data (PPI and partner-disease association). 
	\end{itemize}
	
	Results: 
	\begin{itemize}
		\item Evaluation: precision and recall in 1,404 phenotypes. 
		\item Application: on 870 unassigned intervals. Assign high confidence genes to 91 intervals. 
	\end{itemize}
\end{itemize}	

All SNPs are not created equal: genome-wide association studies reveal a consistent pattern of enrichment among functionally annotated SNPs [Shork, Dale, PLG, 2013]
\begin{itemize}
	\item Enrichment of loci in different categories: consider exon, 5' UTR, 3' UTR, upstream sequence (1k or 10k) and downstream sequence. Stratified Q-Q plot to show the pattern of enrichment of signal in different categories. 
	
	\item Quantification of enrichment: true discovery rate (TDR) is estimated as $1 - p/q$, where $p$ is the fraction of expected SNPs under a given thrshold and $q$ the observed number. Another measure is: sample mean $z^2 - 1$. 
	
	\item Stratified FDR (s-FDR) approach: effectively use different $\pi_0$ (or threshold) to control FDR for each category. Show that at $\alpha = .05$, the increased proportion of SNPs due to sFDR ranges from 20\% (height) to 300\% (SCZ). 
\end{itemize}

Partitioning heritability by functional category using GWAS summary statistics: Stratified LD score regression (S-LDSC) [Finucane \& Price, NG, 2015]
\begin{itemize}
	\item Assumption: the heritability of a SNP is the sum of the heritabilities of the categories that this SNP belongs to.  
	
	\item Model: suppose both genotype and phenotype are standarized, define heritabilty of a category $C$ as $h^2(G) = \sum_{j \in C} \beta_j$. The summary statistic is $\hat{\beta}_j = X_j^T y$, plug in $Y = X \beta + \epsilon$, we have:
	\begin{equation}
	\hat{\beta}_j = \sum_k \hat{r}_{jk} \beta_k + \epsilon_j'
	\end{equation}
	where $\hat{r}_{jk}$ is the LD between SNP $j$ and $k$, and $\epsilon_j'$ has mean 0 and variance $\sigma_e^2 / N$. The assumption of the effect size: it has mean 0 and variance
	\begin{equation}
	\Var{\beta_j} = \sum_{c: j \in c}\tau_c
	\end{equation}
	where $c$ is an index of category. This allows us to derive the expectation of $\chi_j^2 = N \hat{\beta}_j^2$. The results: 
	\begin{equation}
	\E(\chi_j^2) = N \sum_{c: j \in c} \tau_c l(j,c) + 1
	\end{equation}
	where $l(j,c)$ is the weight of category $c$, $l(j,c) = \sum_{k \in C} r_{jk}^2$. This is a system of equation of $\tau_c$ and we solve it via multiple regression. Additional details of the regression: e.g. weighting SNPs, block jackknife for obtaining standard errors.  
	
	\item How would the model work? Suppose we have only two categories (one feature): then for SNPs not in LD with this category, its $\tau_C = 0$, thus we expect generally lower $\chi_j^2$. For SNPs in LD with this category, we would expect higher $\chi_j^2$. 
	
	\item Results: enrichment of main annotations across 9 phenotypes (Figure 4). The enrichment is defined as proportion of heritability of a category divided by the proportion of SNPs. The strongest categories are conserved (12), enhancer (4), fetal DHS (3), coding (7), etc. 
	
	\item Figure 7. comparison of methods for testing enrichment, causal proportion is 0.005, fgwas performs poorly (very low power). 
	
	\item Remark/questions:
	\begin{itemize}
		\item Additivity of effect size variance: cannot model the negative effect (some annotations reduce the effect size). Additivity assumption cannot capture AND logic (interaction terms) - similar to logistic regression, but it is relatively easier to incorporate interaction terms. 
		\item Distribution of $\chi^2$: the model fitting part uses linear regression, which is based on normal error model (not true here). Because of this, the confidence interval is obtained from jackknife.   
	\end{itemize}
\end{itemize}

Weighting sequence variants based on their annotation increases power of whole-genome association studies [Sveinbjornsson \& Stefansson, review for NG, 2015]
\begin{itemize}
	\item Motivation: incorporate weights of variants in multiple testing, where weights depend on annotations (damaing effects). 
	
	\item Intuition: if we can collect all causal variants, and their categories of annotations, we can simply calculate the enrichment of causal variant in each category. Since we do not know, we can collect all associated loci, and then model the uncertainty of causal variants.   
	
	\item Estimating category enrichment: assume all variants are partitioned into disjoint categories. For category $c$, we define $q_c$ be the probability of a non-causal variant being in category $c$, and $p_c$ the probability of causal variant being in $c$. The enrichment is defined as $e_c = p_c / q_c$. To estimate this enrichment, we model the likelihood of data (summary statistics). Suppose we are given association loci, and assume each locus contains exactly one causal variant (let $k_i$ be the index/position of the causal variant of locus $i$). Let $c_m$ be the category of a variant at $m$. The likelihood at locus $i$: 
	\begin{equation}
	P(y|g) = \sum_m P(y|g_m, k_i = m) p(k_i = m)
	\end{equation}
	where $p(k_i = m)$ is given by: 
	\begin{equation}
	p(k_i = m) \propto p_{c_m} \prod_{m' \neq m} q_{c_{m'}}
	\end{equation}
	as $k_i = m$ iff $m$ is causal (with probability $p_{c_m}$) and all other variants are non-causal (with probability $q_{c_{m'}}$). Effectively, we treat categories a variant belongs to as data, and model the probabilities of these ``category variables''. We multiply this likelihood over all loci and do MLE. 
	
	\item Weighted Bonferroni correction: for a variant $j$, use weight $w_j = e_{c_j}$ where $c_j$ is the category of $j$. It is easy to prove that sum of $w_j$ is 1 (actually sum of $p_c$, over all categories). Then follow KMR's weighted multiple testing procedure. 
	
	\item Data: WGS of 2,636 Icelanders, 96 quantiative and 123 case-control phenotypes. Use MAF $>0.1\%$, find 14.2 M variants. Bonferroni correction threshold would be $3.5 \times 10^{-9}$. 
	
	\item Results of category enrichment: first find all variants with $p < 10^{-8}$, then for each of them, extract LD variants with $r^2 > 0.2$. A total of 700 or so association signals (loci). Fitting the model lead to estimation of enrichment. LoF: 186, missense: 51, synonymous: 6.2, upstream/downstream 5k: 5.0, 5' or 3' UTR: 2.8, intronic: 0.6, intergenic: 0.4. Regulatory annotations: DHS - 2.4 and 3.6; enhancers (DHS and ChromHMM) - 7.4 and 7.0. 
	\begin{itemize}
		\item The confidence interval for synonymous and UTR are reall large and so the enrichment of these two categories may not be significant. It's likely that this is due to LD: they are very close to nonsyn or LoF variants, and so their effects are harder to disentangle from nonsyn. variants. 
		\item Conservation: CADD scores $>5$ show no enrichment, GERP scores show modest enrichment: 1.4 fold with scores $\geq 2$ in non-coding sequences. 
		\item Comparison with LD score regression: the main difference is LD score regression considers only coding variants, while this method divides them into LOF, missense and synonymous, which have very different enrichments. 
	\end{itemize}
	
	\item Results of association analysis: first find all settled associations ($p < 10^{-10}$), then among the rest, use standard Bonferroni finds 146, and the weighted finds 172. Thresholds reduced, e.g. LoF, $6.3 \times 10^{-7}$. Almost all new coding variants are real.  
	
	\item Remark: the enrichment defined in this way is similar/equivalent to the enrichment in [Gusev, AJHG, 2014], which define enrichment as: the probability that a variant is causal given that it is in category $c$ vs. the probability that a variant is causal. Let $Z_j$ be the indicator variable of variant $j$, then $p_c = P(j \in c|Z_j = 1)$ and $q_c = P(j \in C|Z_j = 0)$, we have: 
	\begin{equation}
	P(Z_j=1|j \in c) = \frac{P(Z_j=1) P(j \in c|Z_j=1)}{P(j \in c)} \Rightarrow \frac{P(Z_j=1|j \in c)}{P(Z_j=1)} \approx \frac{p_c}{q_c}
	\end{equation}
	
	\item Remark: the method of estimating enrichment requires a large number of association loci, so it cannot be used to estimate the enrichment for individual GWAS data. 
	
	\item Lesson: frequentist framework of incorporating priors, weighted multiple testing. To determine weights, use something similar to empirical Bayes. In the GWAS context, the enrichment of causal variants in a particular category can be directly used as weights. 
\end{itemize}

Disentangling the Effects of Colocalizing Genomic Annotations to Functionally Prioritize Non-coding Variants within Complex-Trait Loci (GoShifter) [AJHG, 2015]
\begin{itemize}
	\item Motivation: enrichment analysis that control for gene density, LD (trait-associated SNPs often mapped to high LD regions). Also colocalization of annotations.
	
	\item Local annotation shifting: (1) Assessing overlap of GWAS SNPs with annotation X: for each GWAS index, extend to SNPs in LD $r^2 > 0.8$. Estimate the proportion of these SNPs overlapping X. (2) Obtain null distribution: the idea is to randomize the positions of $X$. This is achieved by shifting $X$ within each locus (defined by GWAS SNPs and LD proxies).
	
	\item Stratified annotation shifting: testing enrichment of $X$ conditioned on another annotation $Y$.
	
	\item Application: use GWAS significant SNPs associated with height, RA and breast cancer.
	
	\item Remark: the method does not account for LD uncertainty, and it only uses genome-wide significant SNPs.
\end{itemize}

GREGOR: evaluating global enrichment of trait-associated variants in epigenomic features using a systematic, data-driven approach [Schmidt and Willer, Bioinfo, 2015]
\begin{itemize}
	\item Overview: LD pruning to obtain a set of independent GWAS index SNPs. Next we evaluate the overlap of these SNPs with a feature and compare that with a set of matched SNPs. To account for LD: expand to LD proxies ($r^2 $>$ 0.7$), only for testing if a SNP overlap with a feature.
	
	\item Method (Table 1): let $s$ be the number of index SNPs overlapping a feature (where overlapping allows LD proxy). The problem is to obtain the significance of $s$. For each index SNP, we obtain a set of $m$ matched SNPs with the same LD, MAF and gene distance. Then for SNP set $i$, by chance index SNP $i$ will overlap the feature with probability $p_i$, which is the proportion of feature overlap among (m+1) SNPs in SNP set $i$. Then the null distribution of $s$ is the sum of Bernoulli RVs $\sum S_i$, where $S_i \sim \text{Bern}(p_i)$. Show that significance from permutation results are similar to analytic results.
\end{itemize}

GWAS Analysis of Regulatory or Functional Information Enrichment with LD correction (GARFIELD) [review for NG, 2016]
\begin{itemize}
	\item Goal: testing enriched features/annotations among GWAS loci. 
	
	\item Input: GWAS summary statistics, annotations, LD statistics and MAF, TSS distance. 
	
	\item Data preparation: annotations of independent set of SNPs
	\begin{itemize}
		\item LD pruning: the goal is to find an approximately independent set of SNPs. Two SNPs are independent if $r^2 < 0.1$ We start with most significant SNPs, remove variants with $r^2 > 0.1$ and within 1Mb (all SNPs in LD). And repeat this process with the next significant SNPs. About 6\% of variants were left after this step. 
		
		\item Annotation: a variant would have a feature if any of its tagged SNPs $r^2 > 0.8$ and within 500kb (including itself) has that feature. 
	\end{itemize}
	
	\item Testing feature enrichment: choose SNPs at given $p$-value thresholds from 0.1 to $10^{-8}$. Assessing statistical significance of fold enrichment: based on permutations, match MAF, distance to TSS and number of tagged LDs ($r^2 > 0.8$). Specifically, for $N$ variants, we have $N$ p-values, we just permute these p-values. But to control for confounders, match SNPs in MAF, etc, (125 bins) and only permute SNPs within a bin.
	\begin{itemize}
		\item Comparison with Maurono: (1) LD pruning; (2) null set: match features. In Maurono, only LD tagging association. 
	\end{itemize} 
	
	\item Multiple testing correction: take into account the correlation among features. Estimate the number of independent annotations, then Bonfonerrino correction. 
	
	\item Joint modeling of multiple annotations (in Revision): choose a threshold $T$ for SNP $p$-values. Do a logistic regression of SNP association (passing threshold or not) with annotations, adjusting for MAF, TSS, and number of proxies (LD). Model selection: forward variable selection.  
	
	\item Data: GWAS of 27 phenotypes including 3 diseases and 24 quantiative traits. 
	
	\item Remark: Analysis of procedure: If LD pruning is too aggressive, many causal SNPs may be lost. Scenarios: 
	\begin{itemize}
		\item Most sign. SNP is causal, but there is another causal SNP (in enhancer) in LD $r2 = 0.2$ with the causal SNP. 
		
		\item Start with most significant SNPs: the assumption is that they are causal SNPs. Possible that SNP in an enhancer (causal), but a nearby SNP ($r2 = 0.7$) has higher significance. Then after pruning, causal SNP is lost. In tagging step, the highest SNP would not receive the annotation.  
	\end{itemize}
	Other confounding variables in fold enrichment analysis: 
	\begin{itemize}
		\item LD: in high LD regions, better tagging, thus likely to have higher effects and higher density of significant SNPs. High LD might also be correlated with enhancers.
		
		\item MAF: higher MAF means higher power, and thus likely to have higher density of sig. SNPs. It may also be correlated with enhancers: e.g. mutation rates may be higher, and thus affect the mutation age of alleles (hence MAF).
		
		\item Distance to TSS: near TSS, likely more sig. SNPs, and also enriched with enhancers. 
		
		\item Additional ones? Ex. GC content of regions, mutation rates.
	\end{itemize}
	GARFIELD analyzes effectively one annotation a time (another reviewer), and this may cause false discoveries when annotations are highly correlated. 
	
	\item Remark: the main difference with other approaches is: obtain a set of approximately independent variants with the annotation (LD tagging), then in null simulation, also match the independent variants. 
\end{itemize}

Heritability enrichment of specifically expressed genes identifies disease-relevant tissues and cell types [Finucane and Price, NG, 2018]
\begin{itemize}
	
	\item Stratified LD score: $\beta_j = N(0, \tau_0 + \sum_k \tau_k I(i \in C_k)$, $\tau_0$ is the average effect. Continuous extension of LD score: $l(i, k) = \sum_j a_k(j) r^2(i,j)$. 
	
	\item Gene expression matrix: for each gene in a tissue, t-test, regression of expression over tissue (category, e.g. brain). For every tissue, choose top 10\% genes as tissue-specific genes. Then expand to 100kb nearby. 
	
	\item Test tissue-specific annotation, controlling promoters, coding and other general annotation. 
	
	\item Validation with chromatin marks: use all sequences active in a tissue. 
	
	\item Results: BMI, only brain is enriched. T2D, only pancreas, but not adipose. The test likely has low power. 
\end{itemize}

\subsection{Fine Mapping with Variant Annotations}

Joint analysis of functional genomic data and genome-wide association studies of 18 human traits (fgwas) [Pickrell, The American Journal of Human Genetics. 2014]
\begin{itemize}
	\item Goal: test enrichment of a certain annotation (e.g. enhancers in a particular tissue type) in GWAS, and inference of indidivual SNPs (``re-weighting'' based on functional annotation). 
	
	\item Idea: an indicator variable of each SNP (or block, considering LD), and the prior of the indicator depends on functional annotation. Similar to Sherlock, where the prior depends on the status of eQTL. 
	
	\item Model: let $y$ be the data, the summary statistics of all SNPs. We divide the genome into blocks s.t. the blocks can be considered independent (2.5Mb on average). For the $k$-th block, let $\Pi_k$ be the prior that it contains one causal SNP, the likelihood: 
	\begin{equation}
	P(y) = \prod_k \left[(1-\Pi_k) P_k^0 + \Pi_k P_k^1\right]	
	\end{equation}
	where $P_k^0$ and $P_k^1$ are the probabilities of the block under null and alternative hypothesis. Within a block, we assume there is only one causal SNP, then the probability of a block is a sum of probability over all SNPs, weighted by the prior of SNPs: 
	\begin{equation}
	P_k^1 = \sum_k \pi_{ik} P_{ik}^1	
	\end{equation}
	where $\pi_{ik}$ is the prior of SNP $i$ in block $k$ and $P_{ik}^1$ is the prob. of this SNP under $H_1$. The prior depends on the functional annotations of SNPs or blocks. For the $k$-th block: 
	\begin{equation}
	\text{logit}(\Pi_k) = \kappa + \sum_l \gamma_l I_{kl}	
	\end{equation}
	where $I_{kl}$ indicates if block $k$ has the $l$-th block-level annotation, and $\gamma_l$ is the effect of $l$ annotation. Similarly we can define the prior of $\pi_{ik}$. 
	
	\item Computing BFs: the model would require computation of BFs per SNP. This is done by Wakefield's approximation, which depends on the estimated $\beta$, its standard error $V_i$, and a prior of the effect size: $\beta \sim N(0,W)$. In the paper, the prior parameter $W$ is fixed at 0.1 (small effect sizes). For case-control studies, approximate $V_i$ from Wakfield paper. 
	
	\item Inference: penalized log-likelihood (penalty because many annotations are used). The penalization parameter is learned using cross-validation: split all chromosome region 10-fold. 
	
	\item Reweighting: for any SNP, estimate its posterior prob. of association (PPA), that uses annotations as prior. The PPA is effectively ``reweighting'' of GWAS summary statistics. The PPA threshold (0.9) is chosen so that the true discovery rate is comparable to Bonferroni correction. 
	
	\item Data: use ImpG to impute the summary statistics of 18 traits. Annotations: 450 including 402 DHS maps. 
	
	\item Enrichment or depletion of GWAS loci: use HDL GWAS as an example, the enriched categories are enhancers, exons, TSS. Other examples: for platelet volume/count, open chromatin annotation is enriched with $\log_2$ enrichment betwen 2 and 3. A general pattern: repressed chromatins are depleted of GWAS loci: $\log_2$ from -1 to -2. 
	
	\item Results of reweighting: increaes of 5\% of more SNPs from using annotations. 
	
	\item Question: how to do cross-validation for general estimation problems when there are no labels involved? 
	
	\item Remark: 
	\begin{itemize}
		\item The model assumes one causal variant per region, which is much bigger than a single LD region (5,000 SNPs). Furthermore, the only regional level annotation seems to be gene-density, which is not very informative. 
		\item LD issue: the method is not aimed for fine-mapping, thus it does not model the LD structure of a locus.
		\item Robustness to the effect size prior $W$: the paper shows the estimation of annotation parameters is robust, but what about PPA for individual SNPs?  
	\end{itemize} 
\end{itemize}

Integrating Functional Data to Prioritize Causal Variants in Statistical Fine-Mapping Studies (PAINTOR) [Kichaev \& Pasaniu, PLG, 2014]
\begin{itemize}
	\item Motivation: integrating functional annotation and GWAS summary statistics to do fine-mapping. The difference with existing approaches: using summary statistics; aiming at fine mapping (fgwas does not model LD). 
	
	\item Model: consider a locus, let $Z$ be the summary statistics of all SNPs, and $A$ be annotations. Define $C$ as the latent indicator variable for which SNPs are causal (at most three), the distribution of $C$ depends on $A$ through a logistic model with parameters $\gamma$. The distribution of $Z$ given $C$ follows MVN: for the alternative model, it depends on the effect size parameter $\lambda$ (or non-centrality parameter). To simplify, assume $\lambda$ is equalt to $Z$-scores. The likelihood per locus:
	\begin{equation}
	P(Z|\gamma, \lambda, A) = \sum_C P(Z|C, \lambda) P(C|\gamma, A)
	\end{equation}
	
	\item Comparison of PAINTOR and fgwas for estimating annotation parameters: similar results, but fgwas is less efficient (higher standard error) and does not always converge.  
	
	\item Results of fine-mapping: by using annotations, reduce the number of variants per locus from an average of 17.5 to 13.5 (90\% confidence set). 
\end{itemize}

GPA: A Statistical Approach to Prioritizing GWAS Results by Integrating Pleiotropy and Annotation [Chung \& Zhao, PLG, 2014]
\begin{itemize}
	\item Motivation: use both multiple GWAS datasets and annotations to improve the finding of causal variants. 
	
	\item Model: let $P$ be the $p$-values of SNPs, and $A$ be annotations. The idea is that causal variants have different distributions of $P$ and $A$ comparing with null variants. Let $Z$ be the true underlying association variables. When we consider multiple phenotypes, $Z$ of each variant is a configuration (true for one trait, false for another, etc.). Let $j$ be an index of SNPs, the likelihood: 
	\begin{equation}
	P(P,A) = \prod_j \sum_l P(P_j|Z_j = l) P(A_j|Z_j=l) = \prod_j \sum_l P(P_j|Z_j = l) \prod_d P(A_{jd}|Z_j=l)
	\end{equation}
	where $l$ is one configuration of association states. The distribution $P(P_j|Z_j)$ is: uniform under null model and Beta distribution under alternative model. The distribution $P(A_{jd}|Z_j)$ follows Bernoulli distributions. 
	
	\item Application: five GWAS data of psychiatric diseases, and annotations include CNS gene (expression in CNS), eQTL and TFBS. 
	
	\item Remark: the model assumes conditional independence of multiple annotation dataset, and this does not work well when there are mutliple correlated annotation data. 
\end{itemize}

Cross-Population Joint Analysis of eQTLs: Fine Mapping and Functional Annotation [Wen, PLoS Genet, 2015]
\begin{itemize}
	\item Model: same as Torus model [Wen, AJHG, 2016]. 
	
	\item EM-MCMC algorithm: let $G, Y$ be genotype and phenotype data, and $D$ be annotation data. Let $\Gamma$ be the configurations, and $\alpha$ be the enrichment parameters of the annotations. The algorithm computes the MLE of $\alpha$, treating $\Gamma$ as missing data. The complete data log-likelihood is given by:
	\begin{equation}
	\log P(Y, \Gamma | G, D, \alpha) = \log P(\Gamma|D, \alpha) + \log P(Y|G, \Gamma)
	\end{equation}
	Now we take expectation over $\Gamma | Y, G, D, \alpha^t$ to compute $Q(\alpha | \alpha^t)$. Note that the last term does not have $\alpha$ in it, so it's a constant term when maximizing $\alpha$, so we can ignore it. The resulting algorithm is: use MCMC to sample $\Gamma|Y, G, D, \alpha^t$, and then take expectation (PIP) and do logistic regression of PIP vs. annotations. See Section S.2 in Text S1. 
\end{itemize}

Efficient Integrative Multi-SNP Association Analysis via Deterministic Approximation of Posteriors (Torus and DAP) [Wen, AJHG, 2016]
\begin{itemize}
	\item Model: let $\gamma$ be the configuration, the prior of $\gamma = 1$ is related to the annotation by logistic regression, with $\alpha$ the enrichment parameters.  
	
	\item Inference: (1) Estimation of $\alpha$: using EM algorithm, treating $\gamma$ as missing data; (2) Locus level discovery; (3) Fine-mapping on these loci: compute $P(\gamma| D, \alpha)$, where $D$ is full data and $\alpha$ enrichment/prior parameters. 
	
	\item EM algorithm for parameter estimation: in E-step, the method computes PIP for each SNP; in M-step, regression of PIPs (response variables) vs. variant annotations. Note that in the E-step, to compute PIP, we essentially need to fine-mapping for each block. This is difficult, so this computation is done either by MCMC (earlier work, MCMC-embedded EM), or by DAP-1. In DAP-1 approximation, the PIP of SNP $i$ is given by Equation C2. Let $\pi_k$ be the prior of SNP $k$, and $B_k$ be its BF. Then the PIP is:
	\begin{equation}
	P(\gamma_i=1 | y, G, \alpha) = \frac{\sum_k \pi_k B_k}{1 + \sum_k \pi_k B_k} \cdot \frac{\pi_k B_k}{\sum_k \pi_k B_k}
	\end{equation}
	The first term is the probability that there is at least one causal variant, and the second term is the probability of SNP $i$ is causal given that there is at least one. Note: in the denominator of the first term, we have 1 instead of $1 - \sum_k \pi_k$, this is because null model and single-effect model differs by a constant $\pi_k$. 
	
	\item Approximation of posterior: our goal is to infer $P(\gamma|D, \alpha)$. This is given by: 
	\begin{equation}
	P(\gamma | D, \alpha) = \frac{P(\gamma|\alpha) BF(\gamma)}{\sum_{\gamma'} P(\gamma'|\alpha) BF(\gamma')}
	\end{equation}
	where $BF(\gamma)$ is the BF of a configuration. The difficulty is to evaluate the normalizing constant $C$ (denominator), which involves summing over all $\gamma$'s. This is done by considering only models (set of causal SNPs) that cover most of the probability mass. 
	
	\item Adaptive DAP algorithm: let $s$ be the size of a model. We need to evaluate the normalization constant: 
	\begin{equation}
	C_s = \sum_{\norm{\gamma}=s} P(\gamma|\alpha) BF(\gamma)
	\end{equation}
	Suppose we have $\Omega_s$, the set of models whose size $\leq s$. For next step, we add a SNP if its posterior (conditioned on all chosen SNPs) $\geq \lambda$, with default $\lambda = 0.01$. Stopping condition: if for a value of $s$, adding more SNPs does not change $C$ much.
	
	\item Behavior of DAP (notes): suppose we have three causal variants, $A$, $B$ and $C$. In addition, $A'$ is in close LD with $A$. At $s=2$, we should choose: $(A,B), (A,C), (B,C), (A',B), (A',C)$. If we choose only the best model at each step, which is $A$ at $s=1$, we will miss $(A',B), (A',C)$. In general, a model reported by DAP should be conditionally independently. 
	
	\item Simulation results: 1,500 genes, with 50 cis-SNPs each, $n = 343$. Use $\alpha_0 = -4$. Vary $\alpha_1$ (enrichment parameter). Compare: best-case: use true labels to regress with annotations in the EM step; DAP-1 and adaptive DAP. Results: for $\alpha_1$ estimate, DAP-1 has somewhat larger SE than adaptive DAP, but not much larger (Fig. 1), both are roughly unbiased. $\alpha_0$ is slightly under-estimated, -4.6 vs. -4.0 - expected because of power limitation. 
	
	\item \textbf{Remark}: in real data, causal SNPs are not uniformly distributed, they are likely clustered around causal genes. DAP-1 approximation may significantly underestimate $\alpha_0$, but it is not clear its estimates of enrichment parameters are biased. 
\end{itemize}

Incorporating Functional Annotations for Fine-Mapping Causal Variants in a Bayesian Framework Using Summary Statistics (Caviar-BF) [Chen and Schaid, Genetics, 2016]
\begin{itemize}
	\item Model: similar to DAP, $y = X \beta + \epsilon$, where $\epsilon \sim N(0, 1/\tau)$. Let configuration be $c$, use logistic prior for $c_i$: 
	\begin{equation}
	\log \frac{P(c_i=1|A, \gamma)}{1 - P(c_i=1|A, \gamma)} = \gamma_0 + A_i \gamma_1
	\end{equation}
	where $A_i$ are annotations of SNP $i$ (vector). Use normal prior for $\gamma_1$: $N(0, \lambda^{-1})$, where $\lambda$ is penalization parameter. The prior effect size variance is defined as $N(0, \sigma_a^2 1/\tau)$. 
	
	\item Summary statistics version: BF calculation, the BF of a configuration $c$ depends on $S$ (variance of genotypes), $\Sigma_X$ (LD), $z$-scores, $N$, sample size and $\nu$, the prior effect size, or $\beta_i \sim N(0, \nu 1/\tau)$ for causal variants. 
	
	\item MAP estimation of $\gamma$ (annotation parameters): EM algorithm, the M step is effectively penalized logistic regression. The term in the E-step is the PIP of SNPs. 
	
	\item Selection of annotations: use L2, L1 and elastic net. Parameters chosen by BIC, AIC or Cross-validation. Shown that BIC has too much penalty, AIC better, and CV best.
	
	\item Multiple loci: for individual level model, could be multiple loci of the same trait or same locus of multiple traits - the former has different regression model (adjusting all other loci when fine-map one). For summary level data, two scenarios are the same.  
	
	\item Running time: exhaustive search of SNPs, 13 hours to fine-map a locus with 3 causal variants. 
	
	\item Results: comparison of causal variants identified, show CAVIAR-BF with elastic network, 5-fold CV, has the best performance. Q: how is this defined? Likely by PIP. 
	
	\item Calibration of PIPs: compare the true proportion of causal variants vs. average PIPs, show that PAINTOR has inflated PIP. 
\end{itemize}

A Scalable Bayesian Method for Integrating Functional Information in Genome-wide Association Studies (bfgwas) [Yang and Abecasis, AJHG, 2017]
\begin{itemize}
	\item Background: PAINTOR is very slow, can be used to fine-map small regions only. 
	
	\item Model: $Y = X \beta + \epsilon$, where $\beta_i \sim \pi_i N(0, \tau^{-1} \sigma_i^2) + (1-\pi_i) \delta_0$. Consider $K$ non-overlapping annotations, and all variants in category $q$ share the same $(\pi_q, \sigma_q^2)$. Let $A$ be annotations, our goal is to infer $\pi, \sigma, \gamma, \beta | Y, X, A$, where $\gamma$ is the indicators of causal variants (configuration). The posterior is given by: 
	\begin{equation}
	P(\pi, \sigma^2, \beta, \gamma | Y, X, A) \propto P(\pi) P(\sigma^2) P(\gamma| \pi, A) P(\beta|A, \gamma, \sigma^2) P(Y|X, \gamma, \beta)
	\end{equation}
	
	\item EM-MCMC idea: given $\pi, \sigma^2$, we can sample $\gamma, \beta$ for each genomic block independently (OK if each blocks explain small phenotypic variation). And once $\gamma, \beta$ are obtained (some form of summary, e.g. posterior mean), we can estimate $\pi, \sigma^2$ efficiently (similar to MLE). 
	
	\item E-step: MCMC of $\gamma, \beta$ given $\pi, \sigma^2$. The conditional distribution of $\gamma, \beta$: 
	\begin{equation}
	P(\beta, \gamma | Y, X, \pi, \sigma^2) \propto P(\gamma| \pi) P(\beta|\gamma, \sigma^2) P(Y|X, \gamma, \beta)
	\end{equation}	 
	This the Bayesian linear regression and we can integrate out $\beta$ analytically. The problem then becomes sampling from the conditional distribution of $\gamma$. At each step, the proposal distribution does one of three things (with prob 1/3 each): 
	\begin{itemize}
		\item Randomly add a new variant, with higher probability of sampling top SNPs based on marginal association statistics. 
		\item Randomly delete a SNP in the current $\gamma$. 
		\item Randomly switch a SNP in the current $\gamma$: replace it with a neighboring SNP. The selection of SNP is based on conditional association statistics of the SNPs (conditioned on all current SNPs except the switch SNP). 
	\end{itemize}
	From the MCMC, we can be obtain the posterior summary (mean) of $\gamma$ and $\beta$. 
	
	\item M-step: inference of $\pi, \sigma^2$ given the results from the E-step. We obtain the MAP first, and then use Fisher information to approximate the posterior. We have the conditional posterior of $\pi$ as: 
	\begin{equation}
	P(\pi|\gamma) \propto P(\pi) P(\gamma|\pi)
	\end{equation}
	The expected log-posterior-likelihood of $\pi$ can be obtained by integrating out $\gamma$ above. So the objective function of $\pi$ also depends on the posterior mean of $\gamma$ for each SNP. Similarly, for $\sigma^2$, we express its expected log-posterior-likelihood in terms of posterior mean of $\gamma$ and $\beta$ and maximize. 
	
	\item Implementation: typically 5 iterations of EM, and 50K MCMC per block. Running time: for 30K individuals, 9M SNPs, require 5K CPU hours. 
	
	\item Lessons: computational efficiency is gained by MCMC, using EM (marginalizing indicators for SNPs); taking advantage of block structure. 
\end{itemize}

Genetic fine mapping incorporating functional annotation: a Random Effects approach [Fisher and Liu, review for AJHG, 2018]
\begin{itemize}
	\item Motivation: use LDSC to estimate enrichment of heritability across annotations, then use that prior to obtain the posterior of effect sizes. Then test if effect size is 0 using Wald statistic. 
	
	\item Model: use LDSC prior, $\Var(\beta_j) = k \sum_c a_{jc} \tau_c$ where $\tau_c$ is the effect of annotation $c$ and $a_{jc}$ the $c$-annotation of SNP $j$. $k$ is scale parameter: the enrichment of heritability in a region, comparing with the genome-wide average. The method will first estimate $\tau_c$ using LDSC, then compute the posterior mean of $\beta_j$. Treating it as a statistic, and obtain the variance of the estimator, and do the Wald test. 
	
	\item Remark: ignoring the $k$ term, and make some simplifying assumption (each subject has the same residual variance), the results reduce to RSS-p. 
	
	\item Simulation: only one locus, Hapgen2 to simulate large number of genotypes. The annotation parameters we chosen from GIANT study of BMI. Choose one causal SNP in the region: at different LD (high or low), or different prior effect size $\sum_c a_{jc} \tau_c$. Then simulate the phenotype. 
	
	\item Simulation results: better than GWAS, Lasso, PAINTOR, GenoWAP (Hongyu's method). GenoWAP very poorly (often much worse than just plain GWAS). PAINTOR seems to overemphasize annotations (if the causal SNP has high annotation scores, PAINTOR works well). 
	
	\item Remark: a problem with polygenic model is that the prior is non-zero for every SNP, and the posterior is always non-zero. The solution here is to treat posterior estimate as test statistic, and do hypothesis testing of whether it equals to 0. 
\end{itemize}

Fine-mapping type 2 diabetes loci to single-variant resolution using high-density imputation and islet-specific epigenome maps [Mahajan and McCarthy, NG, 2018]
\begin{itemize}
	\item Data: large T2D with 74K cases. 243 loci, and 403 signals. 
	
	\item Use conditional regression from GCTA to infer multiple independent signals in a locus: use only summary statistics and LD. The LD is estimated from 6,000 unrelated individuals in UK Biobank of white British origin. Ex. TCF7L2 has 8 signals. 
	
	\item Unweighted fine-mapping method: on 380 distinct signals (excluding 23 signals), considering 500kb nearby region on either side of each signal. For each SNP, obtain its effect size and standard error. In regions with multiple signals, the effect size and standard errors were obtained from conditional analysis. Then use Wakefield's formula to obtain BF for each SNP, $\Lambda_j$, with prior effect variance 0.04. The PPA of variant $j$ is then given by: 
	\begin{equation}
	\pi_j = \frac{\Lambda_j}{\sum_k \Lambda_k}
	\end{equation}
	The 99\% credible set was then constructed by: ordering all variants by decreasing PPA, and include variants until the cumulative PPA reaches 99\%.  
	
	\item Fine-mapping: number of signals per locus (Figure 3a). 99\% credible set of all loci (Figure 3b). Median of 42 variants per credible set. PPA distribution of variants in credible set (Figure 3c). At 51 signal, one variant has $> 80\%$ PIP. 
	
	\item Impact of reference panel on fine-mapping: compare HRC vs. 1000GP, the results are similar. 
	
	\item Enrichment of enhancers: fgwas, log-OR 1-2 for active enhancers in islet and adipose; Or 2-8 fold in islet promoters, enhancers and coding sequences. 
	
	\item Fine-mapping with fgwas: 15 chr. annotations, and use backward variable selection. Fine-mapping (modified fgwas): (1) regions with single signal: use 1Mb region around lead SNPs. (2) Regions with multiple signals: analysis of each of the distinct signals. 
	
	\item Results of fine-mapping with functional annotations (Figure 6): median credible set size reduced from 42 to 32. Number of variants with PIP $>80\%$ increases from 51 to 73. 
	
	\item Examples: for fine-mapped SNP, use cis-eQTL to find the target gene and validate its effect. 
	
	\item Lesson: to assess fine-mapping results, use all variants included in credible sets, and see how their PIPs change by priors, reference panels, etc. 
\end{itemize}

Integration of human pancreatic islet genomic data refines regulatory mechanisms at Type 2 Diabetes susceptibility loci [Thurner and McCarthy, eLife, 2018]
\begin{itemize}
	
	\item Data: ATAC-seq, 17 samples and DNA methylation 10 samples in islet cells. 
	
	\item Enrichment of open chromatin regions in GWAS of T2D and fasting glucose (FG): Figure 3B-D. Single-feature enrichment and joint enrichment test. Include the islet epigenomic annotations and CDS, TSS, conserved. 
	
	\item Fine-mapping of T2D loci: found significant regions, and 99\% credible sets. Using epigenomic data helps with fine-mapping: reduction of credible sets and increase of PIP of the top SNP (Figure 4AB). Assess the contribution of annotations (e.g. islet enhancers) to PPA of each locus: e.g. for some locus, signal is primarily driven by islet enhancers, suggesting importance of insulin secretion, rather than insulin resistance (Figure 4C). 
	
	\item Several examples: Allele imbalance in top SNPs (PIP $> 0.1$): 3/20 show signals. In three examples (Figure 5), likely causal variants are found: high PIP, eQTL evidence, motif changes, Hi-C interaction (in addition to allele imbalance). 
	
	\item Lesson: (1) To demonstrate the value of epigenomic annotations: change of PIPs and credible sets. Assess the contribution to PIPs. (2) Case analysis: independent evidence of causal SNPs, such as eQTL, motif disruption, Hi-C, allele imbalance. 
\end{itemize}

Functionally-informed fine-mapping and polygenic localization of complex trait heritability (PolyFUN) [Weissbrod and Price, BioRxiv, 2019]
\begin{itemize}
	\item Model: assume the variance of prior effect size is the same across SNPs, then the probability of being causal is proportional to the variance of effect size. Let $a_i$ be annotations and $\beta_i$ be effect, we have: $\Var(\beta_i | a_i) \propto P(\beta_i \neq 0 | a_i)$. The problem is then to estimate the polygenic effect size of $\beta_i | a_i$. To do this: partition SNPs into non-overlapping bines, and estimate the SNP heriability by each bin $b$, then then specify $\Var(\beta_i | a_i)$ for all SNPs in that bin. 
	
	\item Details: (1) using regularized (L2) S-LDSC to estimate heritability for each bin. Training with even chromosomes. (2) Estimation of heritability of SNPs in odd chromosomes. Avoid Winner's curse. (3) Reestimation of h2 for target chromosomes: ensure robustness to model mis-specificiation.
	
	\item Simulations: real genotype from UKBB. 3Mb blocks, 10 cuasal loci explaining 0.05\% h2. (1) Calibration of PIPs: false discoveries of SNPs above PIP threshold (e.g. 0.5). CAVIARBF significantly inflated. (2) Power: number of true causal SNPs above a PIP $>0.5$. PolyFun + FINEMAP slightly more than PolyFun + SuSiE, and best (37\% more than others). PolyFun + SuSiE much faster. 
	
	\item Discussion: examples of coding and non-coding SNPs near the same genes (Table S22). 
	
	\item Lesson: to evaluate fine-mapping methods, simulate blocks with causal variants, then assess calibration/FDR and power at given PIP cutoff. Note: not need to simulate null blocks. 
\end{itemize}
%%%%%%%%%%%%%%%%%%%%%%%%%%%%%%%%%%%%%%%%%%%%%%%%%%%%%%%%%%%%
%%%%%%%%%%%%%%%%%%%%%%%%%%%%%%%%%%%%%%%%%%%%%%%%%%%%%%%%%%%%
\section{Population Structure and Association Studies}

Localizing components of shared transethnic genetic architecture of complex traits from GWAS summary data (PESCA) [Shi and Pasaniuc, 2020]
\begin{itemize}
	\item Background: summary statistics model in terms of Z scores. Let $Z$ be Z-score, $R$ be LD matrix and $\gamma$ be configuration, then we have:
	\begin{equation}
	Z | \gamma \sim N(0, n \sigma^2 R \cdot \text{diag}(\gamma) \cdot R + R)
	\end{equation}
	where $n$ is sample size, $\sigma^2 = h^2 / \abs{\gamma}$, $\abs{\gamma}$ is the number of causal SNPs. 
	
	\item Background: EM algorithm in Bayesian variable selection. If our goal is to estimate hyperparameters, then we can use EM, however, the $Q$ function would need to sum over all configurations. This is done by MCMC (bf-gwas) or DAP-1 (DAP).  
	
	\item Background: multivariate Bernoulli distribution (MVB). Related to logistic regression, but in multivariate case. 
	
	\item Model: let $f$ be parameters of MVB distribution, and $C_i|f$ be the configuration of SNP $i$ (2-dim vector). Given $j$ population, we have $Z_j | C_j$ are conditionally independent. Let $V_j$ be the LD matrix of population $j$. We have:
	\begin{equation}
	Z_j | C_j \sim N(0, V_j + \sigma_j^2 V_j \text{diag}(c_j) V_j)
	\end{equation}
	This gives the marginal likelihood $P(Z_1, Z_2 | f)$, summing over all $C_j$'s. 
	
	\item Inference: by EM algorithm: E-step, in calculating $Q$, averaging over all configurations (posterior of configurations) - simply do $L$ times. $M$-step has a closed form solution. How? 
	
	\item Simulation: 9000 SNPs in chr. 22. Genotype normalized. (1) Sample configurations $c$. (2) Sample effect size of causal variants: $\beta_i | c_i \sim N(0, c_i h^2 / \abs{C} )$, where $\abs{C}$ is the number of causal SNPs. (3) Sample error terms: total variance of trait is 1, so error is sampled from $N(0, 1 - h^2)$. (4) Compute $Z$-scores of all SNPs. 
	
	\item Results: European and Asian, 9 traits. (1) for BMI, 10\% SNPs and 2-3\% for lipids - seems too high. (2) Most of variants $>80\%$ have shared effects in both populations, and effect directions are highly correlated. 
	
	\item Remark: not model correlation of effect sizes. In estimation of $f$: not clear if $h^2$ is estimated. 
	
	\item Remark: standard error of $f$ (Table 1): Seems to be too small. 
	
	\item Remark: how to reconcile with lack of portability of PRS? Possible explanation: large effect SNPs are shared, as shown in the paper, but small effect ones are less shared, which contribute to the PRS difference. 
	
\end{itemize}

\subsection{Population Stratification}

Reference: [New approaches to population stratification in genome-wide association studies, Price \& Patterson, NRG, 2010]; [Laird \& Lange, Chapter 8]

Overview of population stratification: 
\begin{itemize}
	\item Problem: the case and control group may have different proportions of different subpopulations, thus a SNP that has different allele frequencies in two subpopulations may exist at different frequencies in cases and controls, causing false association signal. 
	\begin{itemize}
		\item Ex. suppose African population are more likely to eat junk food and have less exercise because of lower income, thus more likely to develop obesity, then African population may be overrepresented in the case group. 
		\item In general, the individuals are more likely to be related in the case group (since they share the genetic disorder) than in the control group. 
	\end{itemize}
	
	\item Statistical characterization of the problem: suppose genotype $x$ is independent variable and $Y$ (disease) response variable. We have a confounding variable $Z$, which represents the race/ethnicity. If $Z$ is correlated with $x$ (SNPs are correlated with race), and correlates with an independent risk factor (diet, culture), then $Z$ is a confounding variable that needs to be controlled. If not, there will be false association. Note that for a confouding variable to create false associations, we need both conditions. 
	
	\item Detection: in general, if there is a population bias in the cases vs. controls, many SNPs will have different allele frequencies thus associated with the phenotype (thus low $p$ values). So the in QQ plot, the observed distribution of the association statistic will depart from the theoretical null distribution. This could be measured using genomic control $\lambda_{GC}$, defined as the median $\chi^2$ across SNPs divided by the median under the theoretical null distribution. 
	
	\item Basic strategies of dealing with population bias: 
	\begin{itemize}
		\item Filtering putative related individuals. 
		
		\item Genomic control: the idea is to discount the statistic of a SNP. Intuitively, if a SNP has different AFs in the different subpopulations, and the subpoputions are different in cases/controls, then the statistic of the SNP should be discounts. 
		
		\item Ancestry matching: if we know the ancestral subpopulation of subjects, then we could test the association between SNP and phenotype, conditioned on the same subpopulation. This could be implemented via regression with subpopulation as a covariate, or other forms of ancestry matching. PCA is one form of ancestry matching. 
		
	\end{itemize}
	
	\item The challenge of crypic relatedness: population stratification can be caused by two sources: (1) Additional covariates (that correlate with SNPs) such as diet, culture. (2) Family genetic background: e.g. the cases contain a family, then all unique SNPs in this family may be enriched. PCA is a way of controling (1), however, crypic relatedness in (2) cannot be solved by PCA.  
	
\end{itemize}

Correcting for population structure by LMM [personal notes]:
\begin{itemize}
	\item The standard approach for correcting population structure and relatedness. Use a random effect to capture genetic background: the effect is correlated between samples, matching their genetic relatedness (similar to group structure in a typical LMM setting). 
	
	\item Does LMM capture env. confounders? Not directly. But it is reasonable to believe that the genetic random effects have similar group/relatedness structure as env. confounders. 
\end{itemize}

Sources of population stratification: 
\begin{itemize}
	\item Genetic drift: since the population divergence. This produces systematic shift of the observed distribution, and can be addressed by genomic control. 
	\item Natural selection: produces markers with unusal allele frequency differences that lie outside the expected distribution. Genomic control is inadequate. 
	\item Family structure and cryptic relatedness: this may be a more important explanation of spurious association [Devlin \& Roeder, Biometrics, 1999]. 
\end{itemize}

Genomic control: because of the relatedness of individuals, particularly in the cases, the test statistic may be inflated [Devlin \& Roeder, Biometrics, 1999] 
\begin{itemize}
	\item Idea: correct the null distribution of the test statistic by modeling its distribution under population structure and cryptic relatedness. Let $T$ be the numerator of the Armitage tend test statistic (i.e. the difference of the numbers of $A$ in cases and controls), we could derive the distribution (variance) of $T$ under two scenarios: population structure and cryptic relatedness.
	
	\item Population strcture: suppose there are $m$ populations and $a_1, \cdots, a_m$ in the cases and $b_1, \cdots, b_m$ in the controls. And let $p_1$, $p_2$ be the population frequency of $A_1$ and $A_2$ alleles and $R$ the total number of cases. $F$ is the inbreeding coefficient. We could derive the distribution of $T$ by considering the distriution of the number of $A_1$ alleles in each case or control: 
	\begin{equation}
	\text{Var}(T) = 4 R p_1 p_2 (1 + F) + 4 F p_1 p_2 \sum_k \left[ a_k (a_k - 1) + b_k (b_k - 1) - 2 a_k b_k \right]	
	\end{equation}
	
	\item Cryptic relatedness:  let $F_1$ and $F_2$ be the inbreeding coefficients in the cases and controls respectively, then we could derive: 
	\begin{equation}
	\text{Var}(T) = 2 R p_1 p_2	\left[ 2 + (F_1 + F_2) (2R - 1)\right]
	\end{equation}
	
	\item Genomic control: defined as the ratio of $\text{Var}(T)$ over the theoretirc distribution under $H_0$. It can be estimated with a large number of SNPs. Let $Y^2$ be the Armitage trend test statistic, and $Y_i^2$ be the statistic of the $i$-th locus. Then $Y_i^2 / \lambda \sim \chi^2_1$. So the estimate $\hat{\lambda}$ can be obtained by taking the median of $Y_i$. 
	
	\item Correction: divide the $\chi^2$ statistic by $\lambda_{GC}$ (and then compare with the theoretical null distribution). $\lambda_{GC} \approx 1$, no stratification; $\lambda_{GC} > 1$, stratification. Generally, $\lambda_{GC} < 1.05$ is considered benign. Note that inflation in $\lambda_{GC}$ is proportional to the sample size. 
	
	\item Remark: 
	\begin{itemize}
		\item A common inflation factor is applied to all SNPs, however since the SNPs differ in their allele frequencies across ancestral populations, doing this will lose power. Ideally, we want the SNPs whose AF differ a lot are heavily discounts, while other SNPs are not penalized much. 
		\item Will address genetic drift, but not unusual markers. Also not maximize power when family structure or cryptic relatedness is present. 
	\end{itemize}
\end{itemize}

Genomic control: [LL, Section 8.2]
\begin{itemize}
	\item Motivating example: suppose the sample consists of a mixture of two populations, however, the mixing fractions are different in cases and in controls ($\lambda$ and $\lambda'$ respectively). Givne a locus unrelated to disease, its allele frequencies in the two populations are $p_A$ and $q_A$, then the total frequency in the cases is: $\lambda p_A + (1 - \lambda) q_A$, and in the controls: $\lambda' p_A + (1 - \lambda') q_A$. If $\lambda \neq \lambda'$, the two frequencies are different, then the expectation of $U$ (in trend test) would not be equal to 0. 
	
	\item GC correction: suppose $X^2$ is the trend test statistic of markers, and $\lambda$ is the inflation factor (across all control markers) indicating the extent of population stratification. $\lambda$ can be estimated from the test statistic of $L$ control markers (not associated with the phenotype): 
	\begin{equation}
	\hat{\lambda} = \text{median}(X^2_1, \cdots, X^2_L)/0.456	
	\end{equation}
	We could adjust test statistic by: $X^2/\hat{\lambda} \sim \chi^2_1$. 
\end{itemize}

Association Mapping in Structured Populations: [Pritchard \& Donnelly, AJHG, 2000] 
\begin{itemize}
	\item Test: infer the subpopulations in the sample and the association statistics are computed by stratifying the subpopulations. The idea: the effect of a SNP is assessed only within a stratum/subpopulation, and the effects of all strata are combined. Specifically, let $P_0$ be the allele frequencies in the control, and $P_1$ be those in the case: 
	\begin{equation}
	P_0 = \langle p_{kj}^{(0)} \rangle, 1 \leq k \leq K \qquad \text{: the frequency of $j$-th allele in the $k$-th subpopulation at control}	
	\end{equation}
	And similary for $P_1$. The hypothesis of $P_0 = P_1$ can be performed by LRT: 
	\begin{equation}
	\Lambda = \frac{P(C|\hat{P_1}, \hat{P_0}, \hat{Q})}{P(C|\hat{P_0}, \hat{Q})}	
	\end{equation}
	where $C$ is the genotype, $Q$ is the population origin of individuals: $q_k^{(i)}$ is the proportion of genome of the $i$-th individual from the $k$-th subpopulation. The distribution of $\Lambda$ is obtained through simulation (under the MLE of the parameters). 
	
	\item Remark: the STRAT approach (allowing fractional membership) is computationally intensive, and may not be applicable to genome-wide studies.
\end{itemize}

Principal components analysis corrects for stratification in genome-wide association studies (EIGENSTRAT): [Price \& Reich, NG, 2006] 
\begin{itemize}
	\item Population structure analysis: clearly many SNPs are highly correlated (LD or shared ancestry), so a small number of latent variables for ancestral population (subpop.) may be needed. Suppose we have $u$ and $v$ as latent variables, the $j$-th SNP is: 
	\begin{equation}
	x_j = \beta_j u + \gamma_j v + \epsilon_j
	\end{equation}
	So $\beta_j$ is roughly the ``average'' genotype of the subpop. corresponding to $u$ (allele frequency), and $\gamma_j$ the average genotype of $v$. In this case, the eigenvectors represent the genotype/allele frequency (with appropriate standardization) of subpopulations. 
	\begin{itemize}
		\item Note: if there are $k$ subpopulations, only $k-1$ latent variables will be needed (as the total fraction sums to 1). 
	\end{itemize}
	
	\item Testing association: the PCs of subjects can be used as covariates in the testing of genotype-phenotype correlation. Ex. suppose we are regression $x_i$ (the SNP) on $y_i$ (phenotype), we add one covariate, $u_i$, the PC: 
	\begin{equation}
	x_i = \beta y_i + \gamma u_i + \epsilon_i	
	\end{equation}
	Then we test if $\beta = 0$ or not. Equivalently, we could regression $x_i$ on $u_i$ and obtain the residual, and $y_i$ on $u_i$ and obtain the residual, and test the association between two residuals. 
	
	\item Remarks: 
	\begin{itemize}
		\item Comparison with GC: the SNPs are discounted differently. If a SNP does not have different AF in a subpopulation (relative to the population average), then $x_i$ is independent of $u_i$ in the equation above, then no need of population stratification/discount. 
		
		\item Comparison with STRAT: continuous axis of genetic variation, i.e. $u$ and $v$ vary continuously. 
		
		\item These approaches are good for population-level confounding, but inadequate for family structure and cryptic relatedness. Even though these approaches may detect the bias in case vs. control, the power may be lost as any putative disease association will resemble a strong instance of the bias.  
		
	\end{itemize}
\end{itemize}

Testing for genetic associations in arbitrarily structured populations [Song \& Storey, NG, 2015]
\begin{itemize}
	\item Idea of Genotype Condtional Association Test (GCAT): the existing method (LMM) to correct for population structure does not correct for environmental variables that confound with structure. Correct for population structure by estimating the expected AF under the ancestry of any individual: then test if the AF changes with disease states.  
	
	\item Logistic factor analysis (LFA): estimating the AFs under given ancestry. Consider $m$ variants and $n$ individuals, let $x_{ij}$ be the genotype of SNP $i$ and individual $j$, and $\pi_{ij}$ be the corresponding AF (indvidual-specific AF). The idea to estimate $\pi_{ij}$ is that it depends on the ancestry of the individual $j$, and how SNP $i$ depends on the ancestry groups. Let $L_{ij} = \text{logit}(\pi_{ij})$, then we use a factor model: 
	\begin{equation}
	L = AH
	\end{equation}
	where $A$ is $m \times d$ matrix representing how SNPs are determined by the population structure, and $H$ is $d \times n$ matrix is the projection of individuals on population structure. To estimate $A$ and $H$, use the model: $x_{ij} \sim \text{Bin}(2, \pi_{ij})$. 
	
	\item Association testing: test if $x_{ij}$ correlates with $y_j$ with logistic regression (inverse regression), adjusting population structure with $\text{logit}(\pi_ij)$: 
	\begin{equation}
	\text{logit}\left[ \frac{\E(x_{ij}|y_j, z_j)}{2} \right] = a_i + b_i y_j + \text{logit}(\pi_ij)
	\end{equation}
	Test $H_0: b_i = 0$ using LRT. 
	
	\item Questions: 
	\begin{itemize}
		\item Validation of LFA model for estimating $\pi_{ij}$: most markers may not be ancestry informative, does it make sense to assume that its logit is a linear function of the latent factors representing the population structure? 
		\item Advantage of LFA over PCA? 
		\item How LFA controls for envrionmental variables that confound with genetic population structure? 
	\end{itemize}
	
	\item Remark:
	\begin{itemize}
		\item Similar to UNICORN in modeling the dependency of AF on population structure. How to incorporate population control in the GCAT model? 
		\item Specical form of the inverse regression model, and relation to TADA2 model: the idea of individual-specific AF? 
	\end{itemize}
\end{itemize}

Correcting subtle stratification in summary association statistics, [Bhatia and Price, bioRxiv, 2016]
\begin{itemize}
	\item Motivation: significant inflation in summary statistics (from LDSC) due to residual population stratification. 
	
	\item PC loading regression: the phenotype $y$ depends on both PC and SNP of interest: 
	\begin{equation}
	y = \beta_{PC} PC + \beta x + \epsilon
	\end{equation}
	Meanwhile, PC can be written as: $PC = \gamma x + \Psi$  where $\gamma$ is the loading of SNP. From here, we obtain the inflated (uncorrected) effect size is related to the true effect and extent of stratification: 
	\begin{equation}
	\beta_{\text{STRAT}} = \beta_{PC} \gamma + \beta
	\end{equation}	
	So if we regress the uncorrected effect size with PC loading $\gamma$, the slope would be $\beta_{PC}$, the extent of stratification and the corrected/true effect size $\beta$. 
	
	\item Issue: using estimated PC loading from external reference $\gamma$ leads to biased estimate and reduce power. In practice, this limits the number of PCs one can use in the regression. 
\end{itemize}

%%%%%%%%%%%%%%%%%%%%%%%%%%%%%%%%%%%%%%%%%%%%%%%%%%%%%%%%%%%%
%%%%%%%%%%%%%%%%%%%%%%%%%%%%%%%%%%%%%%%%%%%%%%%%%%%%%%%%%%%%
\subsection{Admixture Mapping}

Principles of admixture mapping [personal notes, Guimin's talk] 
\begin{itemize}
	\item Principle: if a causal locus is associated with another locus in some way (not only genotype, could be any other properties of the locus), then the linked locus will be associated with disease. 
	
	\item Admixed genome and local ancestry: genomes of admixed population are ``mosaic genomes'' of two populations, e.g. for African American, it would admixture of West African and European.
	\begin{itemize}
		\item Global ancestry: for any individual we can define its global ancestry as, on average, the proprtion of genome that comes from population 1. 
		\item Local ancestry: for any individual, at any point in the genome, we can define the similar proportion. For any individual, clearly, local ancestry could vary greatly across locations. However, averaging over many people, local ancestry is generally close to the genomewide average (global ancestry).  
	\end{itemize} 
	
	\item ALD and BLD: local ancestry of adjacent loci should be correlated (from the same ancestry block that has not been broken by recombination), and we can this ALD block. The genotypes of loci in ALD blocks are not necessarily correlated. BLD block: represent background correlation in genotypes.  
	
	\item Concept of admixture mapping: it relies on two associations: 
	\begin{itemize}
		\item At causal locus, if AFs in the two ancestry populations are different, say $q_1 > q_2$, then in the cases, since the locus will be enriched, it means it is more likely from population 1, leading to elevated local ancestry at the causal locus. 
		\item ALD block: other loci in the ALD block will have correlated local ancestry, thus their local ancestry will also be associated with diseases. 
	\end{itemize}
	
	\item Example: suppose we have a locus that is present only in population 1 but not 0, $q_1 = 1, q_2 = 0$. In admixture population, the fraction of population 1 is 0.2, then the average AF would be 0.2. In the cases, it will be higher, let's say 0.3. Now because the allele is only present in population 1, it means that the local ancestry at this site is 0.3, higher than average (0.2). 
	
	\item Formal analysis: let $q_1, q_2$ be the AF in two ancestral populations, let $\pi$ be the fraction of population 1. Then in admixed population, the AF is: 
	\begin{equation}
	q = \pi q_1 + (1-\pi) q_2
	\end{equation}
	The same equation means that if given $q, q_1, q_2$, we can estimate local ancestry as: 
	\begin{equation}
	\pi = \frac{q - q_2}{q_1 - q_2}
	\end{equation}	
	Now in the cases, the AF will be elevated, so its AF becomes $\gamma q$. The local ancestry becomes: 
	\begin{equation}
	f = \frac{\gamma q - q_2}{q_1 - q_2}
	\end{equation}
	It is easy to check that if $q_1 > q_2$, $f > \pi$, and if $q_1 \approx q_2$, $f \approx \pi$. So admixture mapping depends on the difference of AF in ancestral populations. 
	
	\item Local ancestry graph: typically, we estimate average local ancestry in the cases at each point of the genome. For most places, this average is equal to global average (when sample size is large enough). However, the disease ALD will show elevated or reduced average local ancestry. Effectively: local ancestry is similar to genotype, and average local ancestry is similar to AF. 
\end{itemize}

Admixture Mapping Comes of Age [Winkler, ARGHG, 2010]
\begin{itemize}
	\item Comparison of ALD with LD: usually much shorter, because the admixture is much more recent. 
	
	\item Inferring local ancestry (Figure 6): suppose the AF (of allele 1) is higher in Ancestral population A and lower in B. Intuition: if we have a blocks of 1’s, more likely to be from A. Block of 0’s more likely from B. Implementation with HMM: emission, AF. Transition: recombination and population proportion. 
	
	\item Admixture mapping: correlation of local ancestry with the phenotype. Similar to normal GWAS, except that we are correlation local ancestry instead of genotype. Rationale: cases, disease allele is enriched, which means that local ancestry is more likely population 1. So in the graph of local ancestry, the fraction of population 1 is elevated above global ancestry (Figure 3). 
\end{itemize}

Combining admixture mapping and association test [Guimin Gao, Dec, 2015]
\begin{itemize}
	\item Association test controlling ancestry: in association testing, one should control for global ancestry (population stratefication). Also, at local regions, control for local ancestry: otherwise any loci in ALD may be falsely associated with disease. Let $D$ be the disease locus and $A$ be a locus in ALD: then genotype of $A$ $\leftrightarrow$ local ancestry of $A$ $\leftrightarrow$ local ancestry of $D$ $\leftrightarrow$ genotype of $D$ $\leftrightarrow$ disease, where $\leftrightarrow$ means correlation. So we need to control for local ancestry in association testing (covariate).  
	
	\item Admixture test: correlation of local ancestry with phenotype. Beause ALD is often much larger than LD, the admixture test can only identify large regions.  
	
	\item Combining two tests: the idea is that using admixture test to find local regions, and then use association test to better define the SNP. Use Generalized Sequential Bonfonerri (GSB) procedure: let $p_i$ be the p-value of $i$-th SNP in admixture test, then set $w_i = 1/p_i$, and do weighted multiple testing correction on the association results. 
	\begin{itemize}
		\item Variation: smoothed weights, $w' = (1-\lambda)w + \lambda \alpha$ where $\alpha$ is a global parameter. 
	\end{itemize}
	
	\item Remark: $w_i$ acts as a soft filter (very strong effect on non-significant SNP). Thus if admixture test has low power, it will lose information. Generally need to derive the optimal weights to maximize the power of study. 
\end{itemize}
%%%%%%%%%%%%%%%%%%%%%%%%%%%%%%%%%%%%%%%%%%%%%%%%%%%%%%%%%%%%
%%%%%%%%%%%%%%%%%%%%%%%%%%%%%%%%%%%%%%%%%%%%%%%%%%%%%%%%%%%%
\section{Sequencing Studies and Methods for Rare Variants}

Sequencing studies in human genetics: design and interpretation [Goldstein, NRG, 2013]
\begin{itemize}
\item Applications of NGS: (1) Mendelian diseases (refractory to linkage); (2) undiagnosed childhood diseases; (3) common diseases. 

\item Functional prioritization of variants: ``narrative potential''. Ex. a control genome, all variants $<1\%$ AF, 237 rare missense variants, 86\% are considered damaging by one of four algorithms (PolyPhen, SIFT, GERP, Blosum62), and 32\% can be connected to some phenotypes in OMIM or HGMD.  

\item Aggregate test of association: ``the impact of incorporating differnet types of prior information into different types of tests has not been systematically evaluated''.  

\item Functional evaluation: animal models and iPSC. Ex. iPSC-derived cardimocyte platform for assessing the effect of mutations. 
\end{itemize}

Reference: [Bansal \& Schork, Statistical analysis strategies for association studies involving rare variants, NRG, 2010], [Cooper \& Shendure, Needles in stacks of needles: finding disease-causal variants in a wealth of genomic data, NRG, 2011], [Kiezun \& Sunyaev, Exome sequencing and the genetic basis of complex traits, NG, 2012]

Importance of rare variants: the evidence can be summarized:
\begin{itemize}
	\item Population expansion: likely to have resulted in a large number of segregating, functionally relevant, rare variants that mediate phenotypic variation. 
	
	\item Purifying selection [Kiezun12]: the number of observed variants is much higher than is predicted by the neutral model with constant population size. This is partly explained by population growth, but also by purifying selection. Thus rare variation is enriched for evolutionarily deleterious, and thus functional, variants. Among rare variants, missense variants predicted to be damaging are more prevalent than variants predicted to be benign. 
	
	\item Cancer genetics: the discovery of rare independent somatic mutations within and across genes that contribute to tumorigenesis may parallel the functional effects of inherited variants that contribute to congenital disease. 
	
	\item Mendelian diseases: the identification of multiple rare variants within the same gene that contribute to largely monogenic disorders such as cystic fibrosis and BRCA1- and BRCA2-associated breast cancer. 
	
	\item Sequencing studies: that focus on specific genes have shown that collections of rare variants can indeed associate with particular phenotypes. 
\end{itemize}

Scenarios where common and rare variants influence phenotype: [Figure 2, Bansal, NRG, 2010]: 
\begin{itemize}
	\item Variants at a single locus with common alleles are more frequent in cases then controls. 
	\item Multiple rare variations contribute to the phenotype such that the collective frequency of these variations is greater in cases. This would create a greater diversity of haplotypes or DNA sequences among the cases. This is the extreme allelic heterogeneity (EAH) setting. 
	\item Multiple rare variations contribute to the phenotype but act in a synergistic fashion, such that cases are likely to have more similar DNA sequences compared to controls. 
	\item Multiple rare variations contribute to a phenotype but the variations contributing to the phenotype reside in specific genomic regions. This situation would create greater sequence diversity among the cases, but only in the relevant genomic regions.
\end{itemize}

Basic strategy of testing:
\begin{itemize}
	\item Collapsing strategy: in its simplest form, counting the frequency of rare variants at any position in the genomic region of interest, in cases and controls, and compare the differences. Better strategies would weight the variants in some way (e.g. by allele frequency).  
	
	\item Functional predictions: use protein structure information, sequence conservation and motif conservation to build models that generate a probability that a particular variant is functionally important. For example, nonsense variants should be prioritized above non-conserved missense variants. Similarly, missense variants should be prioritized above synonymous variants. A number of tests allow the inclusion of prediction scores in test statistics, including the VT test, KBAC, SKAT, the rare variant weighted aggregate statistic (RWAS) and the likelihood ratio test (LRT) [Kiezun12]. 
	
	\item Different methods differ in the way rare variants are weighted, or the assumptions about the effect sizes. 
	\begin{itemize}
		\item WSS test [Madsen09] assumes effect size proportional to $1/x(1 - x)$, where $x$ is the allele frequency. 
		\item The sequence kernel association test (SKAT) [Wu11] simulation framework uses effect size proportional to $-\log(x)$. 
		\item Variable threshold (VT) test [Price10] simulations use a demographic history model with a range of possible values of strength of selection leading to different relationships between effect size and $x$. 
	\end{itemize}
	
	\item Regression-based collapsed variant and conditional tests: If a set of rare variants each individually explain only a small fraction of the variation of the trait, they could be combined into a single predictor variable, e,g. a dummy variable. Could also include other factors in regression model, such as covariate effects, the effects of previously identified common variants or other collapsed sets of rare variants.  
	
\end{itemize}

Methods based on summary statistics:
\begin{itemize}
	\item CAST method [Morgenthaler07]: a version of the collapsing approach in which the frequency of individuals carrying any one of several rare variants is contrasted between case and control groups. Then use the standard contingency table-based chi-square or Fisher's exact tests for obtaining p-values. An extension of the CAST method is combined multivariate and collapsing (CMC) method [Li and Leal].   
	
	\item Weighting by frequencies: Madsen and Browning proposed a statistic for testing a pre-specified collapsed set of variants that leverages weighting of each variant by its frequency, thus allowing one to include variants of any frequency into the collapsed set. 
	
	\item Optimal or variable weighting [Price, AJHG, 2010]: in a procedure resembling that of Madsen and Browning. Price et al. showed that their method is more powerful than approaches that consider fixed weights. In addition, they argued that the use of the predicted functional impact of each individual non-synonymous coding variant could be leveraged in their model.
	
	\item Incorporating the direction of effects [Han and Pan]: for example, protective or deleterious, this can be implemented in a regression model framework. 
	
	\item Haplotype analyses: comparing haplotype frequencies between, for example, case and control groups. Haplotype analyses require phase information, which is not trivial to obtain for genotyped rare variants or variants derived from sequence data. 
\end{itemize}

Approaches based on similarities among individual sequences: 
\begin{itemize}
	\item Motivation: the general nucleotide background or context within which a rare variant can influence a phenotype may be important.
	
	\item Assessing the strategy: such strategies can be as powerful, if not more so, than some traditional tests of association in many settings involving common variations. However, the performance of these methods when many rare variants and no common variants are considered is unknown. In addition, a limitation of these methods is that a specific DNA similarity or distance measure or metric must be chosen and this can be problematic. 
	
	\item Searches for optimal sets of variations: one could potentially search for a subset of variants that maximally discriminates between cases and controls.  Such methods are problematic in that the determination of an optimal subset of variants based on group differences can be computationally intensive.
	
	\item Choosing a DNA sequence similarity measure: difficult because ultimately, functional nucleotide content determines gene activity, rather than the phylogenetic origins of those nucleotides. Thus, in theory, similarity measures that build off the functional features and functional capacities of the affected genes associated with DNA sequence are likely to be more appropriate for association studies. 
	
	\item Family-based linkage analyses: consider the consistency of within-family sharing of specific transmitted chromosomal segments among affected family members rather than the consistency or similarity of the nucleotide content of these segments across different families. However, not all approaches to linkage analysis are very powerful, and this is especially true for non-parametric approaches involving small families. 
\end{itemize}

Multiple regression and data-mining methods: 
\begin{itemize}
	\item Basic method [Morris and Zeggini]: the use of a simple tally of the number of rare variants possessed by an individual across a large region as a predictor of a phenotype against the use of a simple indicator of the possession of any rare variant.
	
	\item Problems with the simple regression approach: LD, multicolinearity, many potential predictor variables to choose from if many individual common and rare variants, as well as collapsed sets of variants, are considered. 
	
	\item Newer regression methods: (1) regularization and shrinkage methods to control for colinearity and overfitting; (2) One possible solution to this problem is to devise methods that combine elements of many different regression procedures, such as the 'bridge' (GPS) regression procedure of Friedman; (3) ``ensemble'' methods that combine the results of different regression and prediction methods. 
	
	\item Logic regression: may be a particularly attractive regression-based approach, at least in theory, for the analysis of rare variants. Use additional variables, constructed from logical operators such as 'AND' and 'OR' that connect and combine sets of variants into potential predictors of the phenotype. The issues include computational burden; difficulty in obtaining $p$-values for each potential independent variable (or individual rare variant compared to a collapsed group of rare variants); and the identification of the optimal, and hence the biologically most plausible, set of genetic predictors. 
\end{itemize}

Related issues: 
\begin{itemize}
	\item Multiple hypothesis testing [Kiezun12]: could use permutation to obtain the threshold. For larger sample sizes, the permutation threshold would be closer to the Bonferroni threshold, asymptotically approaching it as the sample sizes increase. 
	
	\item Power of methods [Kiezun12]: most existing studies (up to early 2012) are underpowered. May require up to 10,000 samples to obtain satisfactory power. 
	
	\item Evaluation and simulation [Bansal11]: need to be assessed in a wide variety of contexts, not just the EAH setting. The best approach will be to take real sequence data obtained from many individuals (e.g. 1000 Genomes Project) and simulate phenotypes based on variants in those sequences, making assumptions only about phenotypic effect sizes and interactions between variants.
\end{itemize}

Directions: 
\begin{itemize}
	\item Methods that can accommodate covariates, previously identified genetic factors, allelic heterogeneity and different sets of collapsed variants simultaneously are clearly advantageous.
	
	\item Methods that can account for subtle synergistic effects of many loci within a defined region and/or different forms of variation that might contribute to gene function, such as those rooted in sequence or functional similarity, are also likely to be appropriate.
	
	
	\item Identifying causality of rare variants in a set: may be more pronounced than it is in assigning causality to a single common variant.
\end{itemize}

Evaluating the functional impact of rare variants [Cooper11]: 
\begin{itemize}
	\item Goal: two related but separable questions: whether a given variant has a functional effect at the molecular level and, if so, whether that functional alteration is deleterious to the organism. 
	
	\item Evolution as the best measure of deleteriousness: 
	\begin{itemize}
		\item Two considerations are essential. First, sequence conservation is not a predictor of deleteriousness per se, but rather it is conservation in excess of neutral expectations. Second, the 'phylogenetic scope' of the compared sequences has substantial effects. 
		\item The assumption of purifying selection: functional divergence will lessen the correlation between past constraint and present-day deleteriousness.
	\end{itemize}
	
	\item Predicting the effects of protein-coding sequence changes: 
	\begin{itemize}
		\item Nonsense and frameshift mutations are the most obvious candidates
		\item Considering non-synonymous variants, the simplest and earliest approaches to estimate deleteriousness use discrete biochemical categorizations such as 'radical' versus 'conservative' amino acid changes
	\end{itemize}
	
	\item The case for non-coding variation analysis.
	\begin{itemize}
		\item However, non-coding variants constitute the overwhelming majority of human genetic variation, and most weak-effect causal variants are non-coding.
		\item Additionally, evolutionary analyses demonstrate that approximately fivefold more non-coding positions exist than coding positions in human genomes that have been subject to purifying selection
		\item As for protein-altering variants, comparative genomics is a central component in deleteriousness prediction for non-coding variants.
	\end{itemize}
	
	\item Experimental approaches: 
	\begin{itemize}
		\item Projects such as the Encyclopedia of DNA Elements (ENCODE) are applying diverse assays in many cell types and conditions to generate functional annotations at a genome-wide scale, including protein-coding genes, non-coding RNAs and cis-regulatory elements
		\item New strategies whereby variants in regulatory sequences, RNAs and proteins can be studied in a highly multiplexed fashion. 
		\item Detailed but generically assayed molecular phenotypes may be useful to capture and measure protein function. For example, cells may be perturbed by overexpression or knockdown of specific genes and subsequently subjected to high-throughput assessments, such as RNA-seq. 
	\end{itemize}
	
	\item Important directions: 
	\begin{itemize}
		\item A unified, quantitative and predictive framework to estimate the prior probabilities for any given mutation to be both functionally relevant and disease relevant.
		\item Variant interactions: protein-protein interaction, gene co-expression networks, coupled with both literature and automated annotation of pathways and gene functions, are crucial to tackle this challenge.
	\end{itemize}
	
\end{itemize}

Summary of rare variant association tests [personal notes]
\begin{itemize}
	\item Multivariate test: the test statistic has $M$ degree of free, where $M$ is the number of variants. When $M$ is large, the test loses power. 
	
	\item Combining summary statistics of multiple variants: best when signal is sparse. When there exists joint effect of multiple variants, the test loses power. 
	
	\item Random effect model (variance component): this increases power (over the multivariate test) by borrowing information across variants, effectively reducing DF (especially when variants are highly correlated?). However the test loses power when the signal is sparse, as it has to pay for the many non-informative variants (because the prior distribution assumes that many variants have effects). 
\end{itemize}

Exome sequencing and the genetic basis of complex traits [Kiezun \& Sunyaev, NRG, 2012]
\begin{itemize}
	\item The relative power to detect association depends on factors such as the number and proportion of causal variants, their population frequencies and their effect sizes, as well as the directionality of effects, the number of genes contributing to the trait, etc.
	\item Effect size assumptions:
	\begin{itemize}
		\item WSS test assumes effect size proportional to $1/q(1-q)$
		\item VT: a demographic history model with a range of possible values of strength of selection leading to different relationships between effect size and $q$.
		\item Sequence kernel association test (SKAT): effect size proportional to Beta function of $q$.
	\end{itemize}
	\item A number of tests allow the inclusion of prediction scores in test statistics, including the VT test, KBAC, SKAT, the rare variant weighted aggregate statistic (RWAS), and the likelihood ratio test (LRT).
	\item To date, no published candidate gene study reported P values that would be significant in the context of the complete exome. This is particularly notable, because some candidate gene studies used much larger sample sizes (thousands of individuals) than ongoing exome sequencing studies (hundreds of individuals).
	\item Extrapolation of effect sizes and frequencies from published studies shows that thousands of individuals are required to reach acceptable statistical power.
\end{itemize}

In search of low-frequency and rare variants affecting complex traits [Panoutsopoulou \& Zeggini, HMG, 2013]
\begin{itemize}
	\item Population isolates: rare variants may have drifted up in frequency (random drift) and linkage disequilibrium (LD) tends to be extended. Example: Iceland-based deCODE study. Association of a rare functional variant (R19X) in the APOC3 gene with HDL-C and triglycerides levels was first detected in the Amish founder population.
	\item Rare variant reference panels: 1000GP, ESP, UK10K (high-depth WES of 6000 and low-depth WGS of 4000 well-phenotyped individuals).
	\item Variant weighting: imputation quality. MAF. Function prediction.
	\item Meta-analysis: based on study-specific summary statistics rather than individual-level data.
	\begin{itemize}
		\item SKAT meta-analysis: They combine single-variant score statistics first across studies and then within a region. They also require between-variant covariance-type relationship statistics (such as LD structure) for each region, as well as MAF of variants.
		\item Meta-analysis at gene level.
	\end{itemize}
	\item Rare variants show increased population specificity. Existing methods to correct for population stratification at common variants such as principal component analysis and genomic control have not been shown to effectively control stratification at rare variants.
\end{itemize}

Rare-Variant Association Analysis: Study Designs and Statistical Tests [Lee \& Lin, AJHG, 2014]
\begin{itemize}
	\item Notation: GLM with link function $h(\mu)$ 
	\begin{equation}
	\E(h(\mu_i)) = \alpha_0 + \alpha^T X_i + \beta^T G_i
	\end{equation}
	where $X_i$ is covariate and $G_i$ genotype. The score statistic of the marginal model for variant $j$ is: 
	\begin{equation}
	S_j = \sum_{i=1}^n G_{ij} (y_i - \hat{\mu}_i)
	\end{equation}
	where $\hat{\mu}_i$ is the estimated mean of $y_i$ under $H_0$. 
	
	\item Burden test: it can be shown that the score statistic under $H_0: \beta= 0$ is:
	\begin{equation}
	Q_{\text{burden}} = \left(\sum_j w_j S_j\right)^2
	\end{equation} 
	Adaptive burden test: use MLE of $\beta$ as the weight of a variant. 
	
	\item Variance component test: 
	\begin{equation}
	Q_{\text{SKAT}} = \sum_j w_j^2 S_j^2 
	\end{equation}
	
	\item Omnibus test: for $0 \leq \rho \leq 1$, 
	\begin{equation}
	Q_{\rho} = (1-\rho) Q_{\text{SKAT}} + \rho Q_{\text{burden}}
	\end{equation}
	where $\rho$ can be interpreted as pair-wise correlation among the genetic effect coeffients $\beta_j$. Adaptive test that uses an optimal value of $\rho$ that gives that minimum $p$-value. 
	
	\item Comparison of test: gene based test can lose power when a very few of the variants in a gene are associated, when many variants have no effect, and when causal variants have low frequency. 
\end{itemize}

Rare variant association studies: considerations, challenges and opportunities [Auer \& Lettre, Genome Medicine, 2015]
\begin{itemize}
	\item Advantages of population isolates: 
	\begin{itemize}
		\item Population bottleneck: rare variants may reach higher frequencies because of founder effect (larger genetic drift, loss of genetic diversity). 
		
		\item Environmental and culture homogeneity. 
	\end{itemize} 
	
	\item Family studies: 
	\begin{itemize}
		\item General idea: co-segregation analysis of variants and phenotypes. The challenge is that one will have a large number of pathogenic variants. Ad hoc filtering; also the MAF in the population. 
		
		\item Comparison of TDT vs. case-control: similar power at low frequency. 
	\end{itemize}
	
	\item Problems with stratification: RVs may have different patterns, and often stronger stratification. 
	
	\item Imputation: with large HRC, 30,000 projects, one can impute variants with MAF as low as 0.01\%. 
\end{itemize}

The increasing importance of gene-based analyses [Circulli, PLG, 2016]
\begin{itemize}
	\item Why do we need the gene based test? We cannot focus only on case-only variants, and then say that they are extremely rare in controls. 
	
	\item Family studies: co-segregation analysis, requires LOD 3.3 or higher. Multiple families with different co-segregation variants in the same linked gene can provide strong support (Ref 3).
	
	\item Coverage imbalance between cases and controls can create false signals.  
\end{itemize}

Genetic architecture: the shape of the genetic contribution to human traits and disease [Timpson and Richards, NRG, 2018]
\begin{itemize}
	\item Genetic architecture differs between phenotypes (Figure 1): T1D vs. T2D, while T2D has mostly low effect SNPs, T1D has some SNPs with large effects. Also comparison of Vitamin D vs. LDL: Vitamin D is mostly oligogenic while LDL is highly polygenic.
	
	\item Limitations of region-based rare variant testing: difficulty with replication; methods should be tailored by genetic architecture, which is unknown and varies; most WGS variants have no effects, and multiple testing burden; the direction of effects unknown.
	
	\item Review of rare variant studies: (1) UK10K: WGS, across 60 traits, burden and variance-component tests on regions. Not a single new region not already found by single SNV analysis. Note: low coverage. (2) T2D: WES in 12K samples. No regions found. (3) Height: 83 rare variants from single SNV analysis vs. 3 regions. (4) MI: no new regions beyond single SNV test.
	
	\item Summary: ``We anticipate that, of the methods currently available, this method (single SNV test) will enhancer our knowledge of genetic architecture the most''.
	
	\item Utility of small effect SNVs: could still lead to drug targets, e.g. PCSK9. Just need large perturbation of proteins.
	
	\item Gene-environment interactions: BMI study, environmental variables explain 14\% variation. But only smoking shows interaction with genotype, and other covariate genetic interaction effects account for less than 1\% of total phenotypic variance.
	
	\item Remark: all the limitations of region-based testing can be addressed by a Bayesian framework, which learns trait-level parameters and incorporates priors on the variant effects.
\end{itemize}

\subsection{Rare Variant Association Tests}
Methods for Detecting Associations with Rare Variants for Common Diseases: Application to Analysis of Sequence Data [Li \& Leal, AJHG, 2008]
\begin{itemize}
	\item Motivation: In the presence of allelic heterogeneity, although the power of linkage analysis is not influenced, association studies based on LD mapping will inevitably be low-powered. Low frequencies of functional variants result in low $r^2$ values.
	
	\item Genetic model: used for simulation: 
	\begin{itemize}
		\item Independence of rare variants: Usually, rare mutations occur on different haplotypes within a locus, therefore, correlation between variants is low. 
		
		\item Penetrance of the locus of the wild type: denoted by $f_0$, is the probability of an individual being affected if the genotypes across all variant sites are wild-type aa. If the assumption is made that wild-type genotypes at different sites have the same penetrance, the relationship can be simplified to $f_0= M f_{0i}$, where $M$ is the number of rare variants. 
		
		\item Risk model: Rare variants also high-risk, and they independently affect phenotype. At each variant, multiplicative, dominant or recessive model. 
	\end{itemize}
	
	\item Methods: 
	\begin{itemize}
		\item Single marker test: standard chi-square test. Remark: correction for many hypothesis. 
		
		\item Multiple-Marker Test: e.g., the Fisher product method, Hotelling's $T^2$ test, or multiple logistic regression. Hotelling's $T^2$ test is used here, with the null hypothesis that none of the variants is associated with disease susceptibility. Remark: a large degree of freedom. 
		
		\item Collapsing method: Due to the rarity of variants, the probability of carrying more than one variant for an individual is low, and the method collapses genotypes across all variants, such that an individual is coded as 1 if a rare allele is present at any of the variant sites and as 0 otherwise. Then the data consist of a set of 0's and 1's in the cases and in the controls, and the test is whether 1's are enriched in cases than in control. A standard $\chi^2$ test can be applied. 
		
		\item CMC method: combines collapsing and multivariate tests. Markers are divided into subgroups on the basis of predefined criteria (e.g., allele frequencies), and within each group, marker data are collapsed (individual coded as 1 or 0 depending on whether he has a rare variant). A multivariate test (e.g., Hotelling's $T^2$ test) is then applied. 
	\end{itemize}
	
	\item Results: comparison of methods 
	\begin{itemize}
		\item Single marker test: lowest power when there is no misclassification of variants. Not only does this test pay a penalty for multiple testing, but also affected by the low allele frequency at each variant, where the power for each individual test is low. 
		\item The power for Hotelling's $T^2$ test is superior to that for the single-marker test but is less powerful than that for the collapsing method. The improvement of power for the collapsing method is due to an enrichment of signals across variants and the single univariate test performed.
		\item However, collapsing methods are not always robust to misclassification of nonfunctional variants, and power loss can be substantial. multivariate tests are more robust in the presence of misclassification of nonfunctional variants
		\item In the presence of common variants, it can be advantageous to analyze both common and rare variants simultaneously with the CMC method; including rare variants in the analysis can greatly increase power if the rare variants have high genotypic RRs and are either numerous or not extremely rare.
	\end{itemize}
	
	\item Other considerations: choosing variants, collapsing, etc. 
	\begin{itemize}
		\item Before statistical analysis of sequence data can be carried out, the first step is quantifying which variants are potentially functional or neutral, e.g. from bioinformatic analysis. 
		\item Choosing functional variants: a number of studies have demonstrated that alleles with a wide range of frequencies are involved in disease etiology. If high-frequency functional variants are removed from or high-frequency nonfunctional variants are included from the analysis, the effect on power can be very detrimental. 
		\item Criteria for collapsing: the CMC method can be used to analyze data on the basis of allele frequencies or certainty of functionality. Even when classification is made on later, it is still inadvisable to collapse rare and high-frequency variants because, as previously discussed, if functionality classification is incorrect, then a large penalty in power can be incurred.
		\item Direction of effect: When all of the functional variants confer high risk or are protective, collapsing will enrich the signal. However, the signal will be weakened if some variants are protective whereas others increase disease risk. Although this situation is probably uncommon, when prior information is available on high-risk and protective variants it should be taken into account when deciding how to collapse variants. 
		\item For rare variants, it is reasonable to assume that within a locus they reside on different haplotypes. The application and the validity of the single-marker test, Hotelling's $T^2$ test, the collapsing method, and the CMC method are not altered by the presence of LD. 
	\end{itemize}
	
\end{itemize}

Simultaneous analysis of all SNPs in genome-wide and re-sequencing association studies [Hoggart \& Balding, PLG, 2008]: 
\begin{itemize}
	\item Motivation: 
	\begin{itemize}
		\item Multi-marker analysis: can improve over single-SNP tests, since a weak effect may be more apparent when other causal effects are already accounted for, but also because a false signal may be weakened by inclusion in the model of a stronger signal from a true causal association.
		\item Computation: Bayesian stochastic search methods have been used to tackle variable selection problems, typically using the ``slab and spike'' prior formulation. However, normal MCMC appraoch is too slow for GWAS-scale problems. 
	\end{itemize}
	
	\item Method:
	\begin{itemize}
		\item Shrinkage through Bayesian prior: continuous prior distributions with a sharp mode at zero, often referred to as ``shrinkage'' priors, to the regression coefficients. Consider two prior distributions, the Laplace, or double exponential distribution (DE) and a generalisation of it, the normal exponential gamma distribution (NEG), which has a sharper peak at zero and heavier tails. 
		\item Inference: seek only the posterior mode(s) rather than the full posterior distribution of the regression coefficients.
	\end{itemize}
	
	\item Results: 
	\begin{itemize}
		\item Application to T2D GWAS: captured the same significant loci as the single-SNP analysis but at the cost of many fewer false positives.
	\end{itemize}
	
	\item Discussion: 
	\begin{itemize}
		\item The main benefits: our analysis returns only the best SNP characterising the effect of a single detectable causal variant, thus suitable for fine-mapping. The NEG analysis improves on the single-SNP ATT (Armitage trend test) analysis, most notably in terms of false positives, and also in terms of power.
	\end{itemize}
\end{itemize}

Lasso logistic regression in GWAS [Wu \& Lange, Bioinfo, 2009]
\begin{itemize}
	\item Application of Lasso in GWAS with large number of SNPs: (1) Set the penalty parameter $\lambda$ s.t. effectively only a given number of predictors will appear in the regression; (2) after learning (the small number of) predictors using Lasso, reestimate the parameters; (3) assessing significance of each predictor: LRT with this predictor has zero effect or not, and compute $P$ value based on $\chi^2$ distribution (statistically incorrect).  
	
	\item Interaction effect: only the predictors predicted from the previous steps (ie. those with large marginal effects) will be used in the test. 
	
	\item FDR: B-H procedure to correct for multiple testing. 
	
	\item Results: 
	\begin{itemize}
		\item Largely similar with the univariate test: similar $P$ values for most SNPs. The main difference: multiple SNPs associated with the trait in LD region. Under Lasso procedure, none of these SNPs have high significance (as removing any of these can be compensating by the remaining SNPs). 
		\item Interaction effect: most of them are insignificant. 
	\end{itemize}
	
	\item Remark: Lasso is plausible in GWAS data, but few benefits are demonstrated. 
\end{itemize}

A groupwise association test for rare mutations using a weighted sum statistic [Madsen \& Browning, PLoS genetics, 2009]: 
\begin{itemize}
	\item Idea: extend the naive collapsing method. Accentuate mutations that are rare in the unaffected individuals, so that the test is not completely dominated by common mutations. 
	
	\item Model: let the weight be the $j$-variant be $w_j$, it is given by: 
	\begin{equation}
		w_j = \sqrt{q_j (1 - q_j)}	
	\end{equation}
	where $q_j$ is the MAF of the $j$-th variant (Bayesian posterior mean) in the controls. Then the genetic score (load) of the $i$-th individual is: 
	\begin{equation}
		\gamma_i = \sum_j \frac{x_{ij}}{w_j}	
	\end{equation}
	where $x_{ij}$ is the genotype of the $j$-th variant of the $i$-th individual. One could then compare $\gamma_i$ in cases vs. controls, specificially, using the total rank of cases among all subjects as the test statistic (same as Wilcoxin test). Obtain $p$-value by permutation. 
	
	\item Model explanation: 
	\begin{itemize}
		\item Weighting: the weight of the $j$-variant is in fact the standard deviation of $x_{ij}$, which follows $\text{Ber}(q_j)$. Thus the genetic score is the normalized count of all rare variants. 
		\item Permutation test: $q_j$ is calculated from the controls only (if using both cases and controls, the estimate of $q_j$ will be deflated/higher than the real values for risk variants). However, this may result in inflation of $p$-values, so need to use permutation to obtain the null distribution. 
	\end{itemize}
	
\end{itemize}

An evaluation of statistical approaches to rare variant analysis in genetic association studies [Morris \& Zeggini, GE, 2010]: 
\begin{itemize}
	\item Methods: 
	\begin{itemize}
		\item Rare variant test 1 (RVT1): the predictor is the proportion of rare variants at which an individual carries a minor allele.
		\item Rare variant test 2 (RVT2): the predictor is the presence/absence of a minor allele at any rare variant within an individual.
	\end{itemize}
	
	\item Results: Our simulations clearly indicate that tests based on the accumulation of minor alleles at rare variants are always more powerful than conventional tests applied to SNPs present on GWA chips, particularly in the presence of substantial allelic heterogeneity.
\end{itemize}

Pooled association tests for rare variants in exon-resequencing studies [Price \& Sunyaev, AJHG, 2010]: 
\begin{itemize}
	\item Motivation: 
	\begin{itemize}
		\item Variable allele-frequency threshold: [Li \& Leal, AJHG08] pick a fixed allele-frequency threshold and perform an association test on the set of variants below that threshold, giving them each equal weight. The potential value of a statistical approach that uses a variable allele-frequency threshold.
		\item Computational predictions of the functional effect of amino acid changes. The test gives higher weight to allelic variants predicted to be functionally significant. 
	\end{itemize}
	
	\item Log likelihood ratio and the weighted genetic score: we first note that the test statistic of the weighted collapsing test can be defined as the total burden in cases:  
	\begin{equation}
		T = \sum_j C_j w_j	
	\end{equation}
	where $C_j$ is the count of the $j$-variant in cases, and $w_j$ its weight. To see its relation to LRT, let $R_j$ be the OR of the i-th SNP. $p_j$ is the allele frequence in controls, and $q_j$ in cases; The two are related by $R_j$:
	\begin{equation}
		R_j = \frac{\frac{q_j}{1-q_j}}{\frac{p_j}{1-p_j}}	
	\end{equation}
	Assume independence of SNPs, and a generative model of genotypes, we could compute the LLR of causal (with specified ORs) vs null model.
	\begin{equation}
		L = \prod_{j=1}^D \frac{q_j^{C_j} (1-q_j)^{(N_+ - C_j)}}{p_j^{C_j} (1-p_j)^{(N_+ - C_j)}}	= \left( \frac{1 - q_j}{1- p_j}\right)^{N_+} \prod_{j=1}^D R_j^{C_j} 
	\end{equation}
	where $N_+$ is the sample size of cases. Take the log., we have: 
	\begin{equation}
		l = \sum_j C_j \log R_j + N_+ \sum_j \log	\frac{1-q_j}{1-p_j}
	\end{equation}
	Thus the weight of the $j$-th variant is $\log R_j$ (log-OR). 
	
	\item Methods: 
	\begin{itemize}
		\item Fixed threshold approach: $\sum_i C_i \xi_i$, where $\xi_i$ is an indicator variable that is equal to 1 if the frequency of SNP $i$ is below a specified threshold (1\% or 5\%) and is equal to 0 otherwise. 
		
		\item Weighted Approach: $\xi_i$ is the inverse square root of expected variance based on allele frequencies $p_i$ computed from controls only. 
		
		\item Variable-Threshold Approach: The intuition is that there exists some (unknown) threshold $T$ for which variants with (MAF) below $T$ are substantially more likely to be functional. Thus, we compute a $z$-score $z(T)$ for each threshold $T$, define $z_{\text{max}}$ as the maximum $z$-score across values of $T$, and assess statistical significance of $z_{\text{max}}$ by permutations on phenotypes, 
		
		\item Incorporating functional effects: use the PolyPhen-2 probabilistic score $S$, and convert it to posterior probabilities $p(S)$ of being functional for each SNP. The $p(S)$ is estimated from two distributions: the neutral set (substitutions that were fixed in the human lineage after divergence from chimpanzee) and the damaging set (known disease-causing missense mutations). These recalibrated posterior probabilities $p(S)$ were applied as weights in the regression.  
	\end{itemize}
\end{itemize}

Association screening of common and rare genetic variants by penalized regression [Zhou \& Lange, Bioinformatics, 2010]:
\begin{itemize}
	\item Motivation:
	\begin{itemize}
		\item The problem with Lasso: the lasso is too stringent for rare variants. Shifting some of the lasso action to a group Euclidean penalty makes it easier for weak or low-frequency predictors to enter a model.
		\item When we pass to penalized estimation, model selection is emphasized over hypothesis testing. The multiple hypothesis testing problem reappears in replication, but in a more benign form because the number of genes and SNPs of interest drop dramatically.
		\item Here, we discuss how to incorporate group penalties that make it easier for related predictors to enter a model once one of the predictors does. 
	\end{itemize}
	
	\item Lasso regression background:
	\begin{itemize}
		\item To put the regression coefficients on an equal penalization footing, all predictors should be centered around 0 and scaled to have approximate variance 1. 
		\item Logistic regression is handled in a similar manner. Instead of equating the loss function to a sum of squares, we equate it to the negative loglikelihood.
		\item Group Lasso: if a parameter enters a model, then it does not strongly inhibit or encourage other associated parameters entering the model. In other words, the local penalty around 0 for each member of a group relaxes as soon as the regression coefficient for one member moves off 0.
		\item Problem with group Lasso: Euclidean group penalties run the risk of selecting response-neutral predictors. As soon as one predictor from a group enters a model, it opens the door for other predictors from the group to enter the model.
	\end{itemize}
	
	\item Methods: 
	\begin{itemize}
		\item If SNP $j$ belongs to group $G$, it should experience penalty $\lambda_E ||\beta_G||_2 + \lambda_L |\beta_L|$. If it belongs to no group, it should experience penalty $\lambda |\beta_j|$, where $\lambda = \lambda_E+\lambda_L$. The objective function is given by: 
		\begin{equation}
			f(\theta) = L(\theta) - \lambda_L ||\beta||_1 + \lambda_E ||\beta_G||_2	
		\end{equation}
		\item As we demonstrate, both kinds of penalties (Lasso and Eucledian) are compatible with coordinate descent, which is by far the fastest optimization method in sparse regression.
	\end{itemize}
	
	\item Results: 
	\begin{itemize}
		\item Breast cancer data: candidate gene study, with genes in DNA repair (DSBR) pathway. 399 Caucasian: There were 196 affected and 203 unaffected individuals. 148 SNPs from the DSBR pathway were grouped by gene (17 genes).  Although most of the SNPs in this dataset are common, 4 have MAFs $<1\%$, 5 have MAF between 1\% and 5\% and 13 have MAF between 5\% and 10\%.
		\item In the case of the pure lasso, SNPs enter the model singly, and in the case of the pure group penalty, genes enter the model with their full sets of SNPs. In the mixed cases, we see that either single SNPs or sets of SNPs grouped by gene enter the model. When a group enters in the mixed cases, it need not contain all of the SNPs in that gene.
	\end{itemize}
	
	\item Remark: 
	\begin{itemize}
		\item The main benefit is: choose causal genes (accompolished by group level penalty) and the functional variants within group (by individual Lasso penalty). 
		\item Why group Lasso selects genes? Intuition: a group is not selected if none of the features is associated, this is similar to Hotelling's $T^2$ test: $\beta_1 = ... = \beta_k = 0$ within the group. 
	\end{itemize}
\end{itemize}

A data-adaptive sum test for disease association with multiple common or rare variants. [Han \& Pan, Hum. Hered, 2010]:
\begin{itemize}
	\item Background: 
	\begin{itemize}
		\item Global test: jointly testing on the multiple $\beta_j$ parameters with the null hypothesis $H_0: \beta_1 = ... = \beta_k = 0$ by one of the three asymptotically equivalent tests: the likelihood ratio test, the Wald test and the score test. Under $H_0$, any of the three test statistics has an asymptotic chi-square distribution with DF $= k$. The generalized Hotelling's $T^2$ test is closely related to the score test. A potential problem with the above tests is the power loss due to large DF. 
		\item In contrast to a global test, another extreme is to conduct a single-locus test for each SNP. The method is equivalent to choosing the univariate test for $\beta_{M,j}$ with the minimum $p$ value, and is hence also called UminP method.
		\item The Sum Test: while using all the SNPs, it adopts a key and generally incorrect working assumption that all SNPs are associated with the disease with a common OR. This is equivalent to regressing $Y$ on a new covariate that is the sum of the genotypes of the multiple SNPs. 
	\end{itemize}
	
	\item Methods:
	\begin{itemize}
		\item If the signs of $\beta_{M,j}$ (MLE of $\beta_j$) are quite different, it may result in power loss. Hence, before applying the Sum test, based on some ad hoc heuristic, one needs to choose the codings of the SNPs to maximize the number of their positive pairwise correlations. 
		\item A Data-Adaptive Sum Test: Hence, a natural approach is to choose the coding of each SNP j based on the sign of $\beta_j$ (MLE), which is data-adaptive and may lead to inflated type I error rates if no adjustment is made with the null distribution. 
	\end{itemize}
	
\end{itemize}

Testing for an Unusual Distribution of Rare Variants [Neale \& Daly, PLG, 2010]: 
\begin{itemize}
	\item Idea: under $H_0$, each variant is equally likely in cases or controls; under $H_1$, a fraction of variants may be more likely in cases than in controls (riks variants), or the opposite (protective variants). The difference lies in the overdisperson: more extreme (unbalanced) variants under $H_1$ than under $H_0$
	
	\item Test: contrasts the variance of each observed count with the expected variance, assuming the binomial distribution. The test statistic: 
	\begin{equation}
		T = \sum_i T_i = \sum_i [(y_i - n_i p_0)^2 - n_i p_0 (1 - p_0)]	
	\end{equation}
	where $y_i$ is the count of the $i$-th variant in cases, and $n_i$ is the total count of the $i$-th variant in both cases and controls, and $p_0$ is the null model (e.g. 1/2 if there are equal numbers of cases and controls).
	
	\item Remark: fail to take the effect size into account, e.g. a LoF variant with total count = 3, the imbalance is very small, thus contribute little to the test statistic (even if all are in cases).
\end{itemize}

Rare variant association testing for sequencing data using the sequence kernel association test (SKAT) [Wu \& Lin, AJHG, 2011]: 
\begin{itemize}
	\item Model: GLM of the phenotype $y_i$ as a function of genotype $G_i$ and covariates $X_i$. Consider the quantitative trait: 
	\begin{equation}
		y_i = X_i \alpha + G_i \beta + \epsilon_i	
	\end{equation}
	The naive test of $H_0: \beta_1 = \cdots = \beta_D$ suffers from high dof., thus low power. Assume the effect size follows the distribution, $\beta_j \sim N(0, w_j \tau)$, where $w_j$ is the weight of the $j$-th variant (the model can be generalized to any distribution with mean 0, variance $w_j \tau$). Test the hypothesis $H_0: \tau = 0$. This is the test of random effect under the variance component model. 
	
	\item Test: we define the $n \times D$ matrix of the genotype, $G = (G_{ij})$, where $G_{ij}$ is the genotype of the $j$-th variant of the $i$-th subject. Let $W = \text{diag}(w_1, \cdots, w_D)$ be the weights of the variants. The score test statistic is given by: 
	\begin{equation}
		Q = (y - \mu)^T K (y - \mu)	
	\end{equation}
	where $\mu$ is the predicted mean under $H_0$, and $K = G W G^T$. We can write $K$ as: 
	\begin{equation}
		K(G_i, G_{i'}) = \sum_j w_j G_{ij} G_{i'j}	
	\end{equation}
	thus $K(\cdot,\cdot)$ is the kernel function, measuring similarity between the subjects $i$ and $i'$. Under the null hypothesis, $Q$ follows a mixture of chi-square distributions. 
	
	\item Weighting: choose the form $\sqrt{w_j} = \text{Beta}(q_j;a_1,a_2)$ where $q_j$ is the MAF of the $j$-th variant (cases and controls combined). When $a_1 = a_2 = 1$, this leads to $w_j = 1$ for any $j$. When $a_1 = a_2 = 0.5$, this corresponds to: $w_j = 1/(q_j (1-q_j))$. The default/recommended choice is $a_1 = 1, a_2 = 25$, which would put some weights on the relatively common (MAF between 1\% to 5\%) variants. 
	
	\item Model analysis: relation to the tests of individual variants. Let $g_j$ be the vector of the genotype of the $j$-th variant. $Q$ is a weighted sum of the individual score statistics for testing for individual variant effects: 
	\begin{equation}
		Q = \sum_j Q_j = \sum_j w_j S_j^2	
	\end{equation}
	where $S_j = g_j^T (y - \mu)$ is the individual score statistic for testing the marginal effect of the $j$-th marker under the individual linear or logistic regression model. 
	
	\item Kernel function: could incorporate additional prior information or epistatic effects: the weighted linear kernel (the basic model), the weighted quadratic kernel (epistatic), and the weighted identity by state (IBS) kernel.
\end{itemize}

A probabilistic disease-gene finder for personal genomes: VAAST [Yandell \& Reese, GR, 2011]: 
\begin{itemize}
	\item Motivation: The utility of Amino Acid Substitution (AAS) approaches for variant prioritization is well established (Ng and Henikoff 2006); combining AAS approaches with aggregative scoring methods thus seems a logical next step.
	
	\item Methods: 
	\begin{itemize}
		\item Basic model: suppose there are $k$ sites, each of which has one RV. The $k$ sites are unlinked. Generative model of genotypes: the genotype at $i$-th site is simply a Bernouli trial. Under $H_0$: the probability is a constant (the MAF in the total samples); under $H_A$: two probabilities in cases and in controls respectively. The test is LLR. 
		\item Collapsing: to increase power, $m$ collapsing categories: the intuition is when sampling at the $i$-th category, may sample an RV at any of the site within the category. Equation (1). 
		\item Incorporating AA substitution: under the generative model, add the probability of sampling a certain type of changes. Under $H_0$: at the $i$-th site, the probability that this change will not affect phenotype; under $H_A$: the probability that this change will affect the phenotype. Equation (2). 
		\item P-value: if unlinked, LLR follows chi-square distribution. If linked, do permutation test to get P-value. 
		\item Synonymous and noncoding variants: Normalized Mutational Proportion (NMP) in vertebrate (primate) genome alignments. 
	\end{itemize}
	
	\item Results:
	\begin{itemize} 
		\item AAS frequencies among known disease-causing alleles in OMIM and AAS frequencies in healthy personal genomes differ from the BLOSUM model of amino acid substitution frequencies.
		\item VAAST scores a wider spectrum of variants than existing AAS methods. SIFT (Kumar et al. 2009), for example, examines nonsynonymous changes in the context of multiple alignments of homologous proteins. Because not every human gene is conserved and because conserved genes often contain unconserved coding regions, an appreciable fraction of nonsynonymous variants cannot be scored. 
		\item Test on Miller's syndrom (a few genomes): the true gene is ranked in the top. 
		\item Test on Crohn's disease: NOD2, with both rare ($ < 5\%$) and common variants. Vary the number of individuals and assess the power by using 500 bootstrapped samples. Its estimated power is 89\% for NOD2 using as few as 150 individuals ($\alpha = 0.05/21,000= 2.4 \cdot 10^{-6}$, where 21,000 is the number of genes). By comparison, the power of GWAS is $<4\%$ at the same sample size. 
		\item Test on LPL, a gene implicated in hypertriglyceridemia (HTG): a data set of 438 re-sequenced subjects, rare variants only. Similar trend, the VAAST methods are better than GWAS and MB test [Madsen \& Browning, 2009]. 
	\end{itemize}
	
	\item Remark: 
	\begin{itemize}
		\item Very similar to CMC in overall apporach: collaping variants, and the tests are similar. In CMC: $H_0: \beta_1 = ... = \beta_p$ and $H_A$ use the MLE of $\beta$; in VAAST, $H_0: p_j^A = p_j^U$, and $H_A$ uses MLE of $p_j$. Since $\beta_j$ and frequcy ratio of $p_j$ are closely related, the two methods are very similar. 
		\item Contributions: use of AA subtitution data; synonymous and non-coding variants and other features such as use of pedigree. 
	\end{itemize}
\end{itemize}

A New Testing Strategy to Identify Rare Variants with Either Risk or Protective Effect on Disease [Ionita-Laza \& Lange, PLG, 2011]
\begin{itemize}
	\item Idea: test the allele frequency difference in cases vs. controls for each variant, then combine the results using Fisher's method. 
	
	\item Model: let $(k,k')$ be the number of copies of minor alleles in controls and in cases for a variant. The data can be then summarized as a table of $n_{k k'}$, the number of variants falling into the $(k,k')$ cell. To test each variant, we use a (variation) of $p$-value: the probability that the variant occurs less than or equal to $k$ times in controls, and more than or equal to $k'$ times in cases, under $H_0$: 
	\begin{equation}
		p(k,k') = \text{ppois}(k, \hat{f}) \times	(1 - \text{ppois}(k', \hat{f}))
	\end{equation}
	where $\hat{f}$ is the expected frequency under $H_0$. The information of all variants are combined: 
	\begin{equation}
		S = \sum_{k,k'} -n_{k k'} \log [p(k,k')]
	\end{equation}
	
	\item Extensions: to incorporate protective variants, using the two-sided test to derive $p$-values. To incorporate externtal information, let $\phi(v)$ be the probability that $v$ is a risk variant, multiply $\phi(v)$ in the equation of the test statistic above. 
\end{itemize}

A general framework for detecting disease associations with rare variants in sequencing studies [Lin \& Tang, AJHG, 2011]: 
\begin{itemize}
	\item Contribution: relative to [Li08], [Madsen \& Browning, PLG, 2009], [Price10], a score test that enhancers power, normal approximation (thus no need of permutation) and accomodate covariates. 
	
	\item Basic model: 
	\begin{itemize}
		\item Notation: $n$ subjects with $m$ SNPs. $Y_i$: phenotype of $i$-th subject; $X_{ji}$: $j$-th SNP of $i$-th subject; $Z_{ji}$: $j$-th covariates of $i$-th subject. $Y_i$ are related to $X_i$ and $Z_i$ through logistic regression. We can write $\beta_j = \tau \xi_j$, where $\tau$ is a scalar constant, and $\xi_j$ is called the weight function (assume given in the testing). 
		\item Test: the score statistic for testing the null hypothesis $H_0$: $\tau = 0$. Under $H_0$, the test statistic $T=U/V^{1/2}$ is asymptotically standard normal. 
	\end{itemize}
	
	\item Forms of weight function: The true value of the weight function is unknown and must be determined biologically or empirically. If we set $\xi_j=1$, then $T$ is a burden test, which counts the total number of rare mutations each subject carries.  
	\begin{itemize}
		\item $C$ test:  the constant weight function. 
		\item $F_u$ test: $\xi_j = [p_j (1 - p_j)]^{-1/2}$, where $p_j$ is estimated from unaffected individuals. This weight function was proposed by [Madsen and Browning, 2009].
		\item $F_p$ test: similar to $F_u$ test except that $p_j$ is estimated from pooled samples of affected and unaffected individuals.  
		\item Fixed threshold tests: assume common variants are not associated with the phenotype, set $\xi_j=0$ if $p_j>c$, where $p_j$ is the (MAF) of the $j$th SNP, and $c$ is a given threshold. At $c = 0.01$, $T_1$ test; at $c = 0.05$, $T_5$ test. 
	\end{itemize}
	
	\item Variable threshold test: We consider $K$ choices of $\xi$, which could correspond to different thresholds or different types of weight function, or both. Under $H_0$, the random vector $T(U_1, \cdots, U_K)$ is approximately $K$-variate normal with mean 0 and covariance matrix. For the two-sided test, we consider the maximum of the absolute test statistics. 
	
	\item EREC (estimated regression coefficients) method for weight function: estimate weight function from data (important if the mutations have opposite effects on phenotypes). 
	\begin{itemize}
		\item If the choice of the weight function is not proportional to $\beta$ or $\xi$ is estimated from the data, then $U$ is no longer the score statistic. However, the test statistic $T$ is asymptotically standard normal under $H_0$ regardless of how $\xi$ is determined. 
		\item Naive method: The optimal choice of $\xi_j$ is $\beta_j$, which is unknown. We can estimate $\beta_j$ from the data, e.g. its MLE. The main problem is that $\beta_j$'s are highly variable (because the individual variants are very rare) and can be quite different from the true values.
		\item Compromise: set $\xi_j = \hat{\beta_j} + \delta$ , where $\delta$ is a given constant. The corresponding test statistic $T$ will be asymptotically standard normal as long as $\delta$ is nonzero.
	\end{itemize}
	
	\item Permutation test: important in small samples. In the absence of covariates, we simply permute the phenotype values $Y_i$'s and calculate the test statistic $T$ for each permutation.
	
	\item Discussion: 
	\begin{itemize}
		\item Comparison: Wald tests tend to be overly conservative (resulting in substantial loss of power) whereas likelihood ratio tests tend to be too liberal (resulting in excessive false-positive findings), especially for small $n$ and low MAFs.
		\item VT approach: improves upon that of [Price10] in three aspects: (1) it uses more powerful test statistics, (2) it can accommodate covariates, (3) it can be implemented by normal approximation instead of permutation.
		\item EREC test: recommended for general use. Similar power to the tests assuming the same direction of effects when that assumption holds and is much more powerful than the latter when that assumption fails. Outperforms the HP, C-alpha and SKAT tests. 
		\item Our theory allows incorporation of any prior knowledge into the weight function. Efficient use of functional or bioinformatics information requires further investigation. It would be worthwhile to explore Bayesian methods.
	\end{itemize}
	
	\item Remark: 
	\begin{itemize}
		\item Normal approximation: works well for large samples (in experiment, about 2,000 subjects). This is not the case in many NGS studies. 
		\item Not allow one to update the weight of SNPs: weighting is based on a sum of MLE and a constant. However, weights (the optimal values should be the actual effects of SNPs) should be part of the inference process.  
	\end{itemize}
\end{itemize}

Hierarchical Generalized Linear Models for Multiple Groups of Rare and Common Variants: Jointly Estimating Group and Individual-Variant Effects [Yi \& Liu, PLG, 2011]
\begin{itemize}
	\item Background: Bayesian Analysis of Rare Variants in Genetic Association Studies [Yi \& Zhi, GE, 2011], using Bayesian GLM, rather than predetermining the weights of variants as previous methods, our approach jointly models the overall effect and the weights of multiple rare variants and estimates them from the data.
	
	\item Idea: define multiple groups of variants (a group could be all rare missense variants), and assume each group acts as a whole and could have a different effect. Conceptually, group can be thought of as a latent variable. Next, the effect of a variant to a group is not fixed: variants may have different effect in a group. 
	
	\item Model: suppose we have $K$ groups, the effect of the $k$-th group to the trait is $g_k$. For a variant $j$ in the group $k$, its effect on this group (the burden of this group) is $\alpha_j$. We also have covariants $x_{ij}$ for the $j$ covariate of the subject $i$, and the effect $\beta_j$. The trait of the $i$-th subject is given by:
	\begin{equation}
		y_i = \sum_j x_{ij} \beta_j + \sum_k g_k \sum_{j \in G_k} \alpha_j Z_{ij} + \epsilon_i
	\end{equation}
	where $Z_{ij}$ is the genotype at marker $j$. For binary trait, we use GLM and replace $y_i$ as $\eta_i$ (the linear predictor). The prior of $g_k$ is normal with mean 0. The prior of $\alpha_j$ is normal with mean $\mu_j$, whose value is pre-specified (otherwise, $g_k$ and $\alpha_j$ are coupled, and unidentifiable). 
	
\end{itemize}

Testing Rare Variants for Association with Diseases: a Bayesian Marker Selection Approach [Zhang \& Deng, Ann Hum Genet, 2012]
\begin{itemize}
	\item Naive Bayes model: variant-specific prior of q, and relative risk:
	\begin{itemize}
		\item Risk variant: $1/\gamma \sim \text{Beta}(\gamma_{l1}, \gamma_{l2})$,
		\item Protective variant also allowed.
	\end{itemize}
	\item Choose Beta(0.5,0.5) as the prior.
	\item Remark:
	\begin{itemize}
		\item The default prior of RR makes little sense. At this prior, the relative risk is often as large as 100 for risk variants. For protective variants, most of the prob. mass of this prior is either close to 0 (highly protective) or close to 1.
		\item In general, the prior is critical, and the method does not demonstrate the benefit of a good prior, and does not show how to obtain a good prior.
	\end{itemize}
\end{itemize}

Incorporating prior biologic information for high-dimensional rare variant association studies. [Quintana \& Conti, Hum Hered, 2012]
\begin{itemize}
	\item Contributions:
	\begin{itemize}
		\item Bayesian risk indeix (BRI) method: uncertainty of variants, and multi-level inference of both gene/region and individual variants.
		\item iBRI method: integrate prior knowledge of variants as covariates, and allow multiple regions.
	\end{itemize}
	\item BRI model: $n$ subjects with $p$ variants and $q$ covariates. Note all variants are causal, so we let $\gamma$ be the subset of causal variants, and $M_{\gamma}$ denote the corresponding model. To incorporate direction of effects, let $\gamma_v = 1$ (risk) or -1 (protective), if the variant $v$ is in the subset $\gamma$. Given a model $M_{\gamma}$, the burden of the $i$-th subject is called risk index, defined as $X_{i,\gamma} = \sum_{v=1}^p \gamma_v G_{iv}$, where $G_{iv}$ is the genotype of the $i$-th subject at variant $v$. The model is then:
	\begin{equation} 
		\text{logit}(Y_i = 1) = \beta_0 + \beta Z_i + \beta_{\gamma} X_{i,\gamma} 
	\end{equation}
	\item The model can be easily extended to allow multiple regions.
	\item iBRI model: suppose we have covariates of variants, then we could have a second-level model relating the prior probability of whether $v$ is included to the covariants. The probit model is used.
	\item Inference: first, the prior of $M_{\gamma}$ is set s.t. the probability of at least one variant is included stays constant with the number of variants increased. Second, model selection tasks are: at least one variant in a gene is included (gene-level test using BF) and for any particular variant, it is associated (variant-level test). Finally, to sum over all models $M_{\gamma}$: to MH algorithm, and use Gibbs sampling to sample the second-level model parameters.
	\item Application to breast cancer data: 640 cases and 1272 controls, test BRCA1 gene (more than 100 rare variants). Gene-level association is extremely high with iBRI model, and 30 with non-informative BRI model. Table 1 shows the top variants: variant BF and supporting counts in cases/controls. In particular, the BF of BRI model of a variant with 2/0 counts is 13, but with iBRI, the BF increases to 600. This is because many LoF mutations exhibit associations, thus the prior probability is increased.
	\item Running time: 3h for 100K iteration on a single gene region (134 variants).
	\item Remark: comparison of our Bayesian approach:
	\begin{itemize}
		\item Extremely slow, cannot be applied to genomewide.
		\item Very simple prior information used: LoF and missense.
	\end{itemize}
\end{itemize}

Identifying genetic marker sets associated with phenotypes via an efficient adaptive score test [Cai \& Carroll, Biostatistics, 2012]
\begin{itemize}
	\item Model idea: the variance component test loses power when the signal is sparse, so we estimate the effect of each SNP using data, and then down-weigh the SNPs likely non-informative. 
	
	\item Remark: using a better prior to reflect the sparsity of informative SNPs, e.g. Slice-and-slab prior, can address the problem of variance component model? 
\end{itemize}

The Value of Statistical or Bioinformatics Annotation for Rare Variant Association With Quantitative Trait [Byrnes \& Li, GE, 2013]
\begin{itemize}
	\item Goal: evaluation of weighting schemes for RVAT.
	\item Phenotype independent weighting scheme:
	\begin{itemize}
		\item Collapsing approach (0 or 1) and burden approach
		\item Madsen-Browning frequency weighting
	\end{itemize}
	\item Phenotype dependent weighting scheme:
	\begin{itemize}
		\item EREC-like weighting: first do regressoin to find $\hat{\beta}_j$ (either single-variate or multi-variate), then do burden test, with the wegiths equal to $\hat{\beta}_j$'s.
		\item Penalized regression: Lasso, Elastic net, SCAN (penalize smaller coefficients more heavily than larger ones). After regression, the estimated coeff. will be used as weights.
	\end{itemize}
	\item Statistical test: score test, and the significance determined through permutation.
	\item Simulation: for each region/gene, $m$ causal variants, each variant is either risk or protective (with probability $r$). The effect size depends on $q$ (the link functions). Also simultate the setting where bioinformatic tools can predict functionality of variants - the true variants has 90\% prob. of being functional. In addition, there are a certain number (at least 1/3) of functional but non-causal variants.
	\item Results: in the absence of bioinformatic tools, vaiable selection methods (Lasso, EN) significantly outperform the rest.  With such tools, the power of burden-type of tests is significantly boosted.
	\item Remark:
	\begin{itemize}
		\item Phenotype dependent weighting using variable selection technique: better than independent weighting scheme, as expected (similar to random effects). However, all methods are slow because one needs permutation.
		\item A major challenge is to combine statistical variable selection with bioinformatic predictions.
		\item The effect-MAF functions in simulation are not realistic: small $q$ should imply a larger effect size. Also under the simulation setting, power often reaches close to 100\%, clearly unrealistic.
	\end{itemize}
\end{itemize}

A unified mixed-effects model for rare-variant association in sequencing studies. MiST [Sun \& Hsu, Genetic Epidemiology, 2013]
\begin{itemize}
\item Motivation: incorporation of variant characterisics and modeling variant-specific effect (heterogeneity). 

\item Model: let $X_i$ be the covariates, $G_i$ be genotypes, we have the usual GLM: $g(\E(Y_i)) = X_i \alpha + G_i \beta$. A prior model of $\beta_j$ as: 
\begin{equation}
\beta_j = Z_j \pi + \delta_j
\end{equation}
where $Z_j$ is a vector of variant annotations and $\delta_j \sim N(0, \tau^2)$. Thus the effect of a variant depends on its characteristics, but also allow individual variant effect. To allow for weighting, let $Z_j = w_j$, which depend on MAF of $j$. The null hypothesis is $H_0: \pi = 0, \tau = 0$. 

\item Inference: instead of LRT, which is difficult for alternative model (estimating $\tau^2$), use score test. Two test statistic, one for $\pi=0$, the other for $\tau = 0$. To combine the two (independent) test, use Fisher's method (more powerful). 

\item Simulation: fix number of variants (10), with different MAFs. Simulate different models: all deleterious, both positive and negative effects, and a small fraction of causal deleterious variants (3 out of 10). Show MiST has better power than SKAT-O. 

\item Dallas Heart Study results: use annotations: missense, nonsense, frameshift. Found a couple of significant genes,with lower $p$-values than not using annotations. Also, $p$-value for $\tau$ is large, suggesting that variant effect heterogeneity is not important. 

\item Remark: limitations of the model 
\begin{itemize}
	\item No borrowing information across genes: the ability to estimate $\pi$ for each individual gene is probably limited in practice. 
	\item Additive effect of annotations. 
	\item Prior of effect size: mean 0, but for rare variants, most causal variants should increase the risk. 
\end{itemize}
\end{itemize}
 
A Variational Bayes Discrete Mixture Test for Rare Variant Association [Logsdon, GE, 2013]
\begin{itemize}
\item Model: the idea is to model the heterogenity of effects, or a mixture of causal and neutral variants. Suppose there are a proportion of causal variants in a gene, let $Z_j$ be the indicator of causal variants. $Z_j$ follows Bernoulli distribution with prob. $p$. The model regresses $y$ with the sum of $Z_j$. Test the model $p = 0$ or $\theta = 0$ (the effect of causal variants). 

\item Remark: this is a special case of mixture prior (Slice-and-slab prior): $\beta_j$ is a mixture of 0 and non-zero distribution. Here it uses the simplest non-zero distribution (1). 
\end{itemize}

Meta-analysis of Gene-Level Associations for Rare Variants Based on Single-Variant Statistics [Hu \& Lin, AJHG, 2013]
\begin{itemize}
	\item Idea: obtain single variant statistics, and test the null hypothesis that none of the variant is associated with the trait. The single variant statistics can be easily combined across studies with meta-analysis. 
	
	\item Model: suppose we have a gene with $m$ variants, let $Y$ be the phenotype of $n$ individuals and $G_j$ be the genotype of $n$ individuals, then roughly speaking, the score statistic of variant $j$ is: 
	\begin{equation}
	U_j \propto \Cov(Y, G_j) = \sum_i Y_i G_{ij}
	\end{equation}
	Under $H_0$, $U$ follow MVN distribution, with mean 0 and covariance dependent on LD. Specifically, the covariance of $U_j$ and $U_k$ under the null model is a linear combination of terms $\Cov(G_j, G_k)$, which is basically the LD matrix. 
	
	\item Remark: the paper also derives the results under various burden tests. The burden test statistic is a function of $U$, and the MVN distribution of $U$ should allow us to derive the distribution of other test statistics. 
\end{itemize}
 
Design of DNA pooling to allow incorporation of covariates in rare variants analysis [Weihua Guan, PLoS ONE, 2014. Review]: 
\begin{itemize}

\item Idea of DNA pooling: pool all DNA of cases and controls, respectively, then sequence all cases, and all controls. From the reads, infer the AFs in cases and in controls, respectively. Test the difference assuming binomial distributions. 

\item Limitations: individual identification is lost, and cannot control for covariates.

\item Model idea/procedure: divide the samples based on covariates (i.e. group samples with similar covariates). Then sample genotypes from each group (pool) based on estimated AF in the group from the reads data. The last step is association test with regression: imputed genotypes and the covarates in each group. 

\item Estimating AF in each group: let $p$ be the AF in a group of $K$ subjects, and $m$ be the number of alleles of $a$. Suppose we have $n$ reads in the pool, $x$ of which have allele $a$. Suppose the error rate is $e$, then the probability of observing a read with allele $a$ is: 
\begin{equation}
d = \frac{m}{2K} (1-e) + (1-\frac{m}{2K}) \cdot e
\end{equation}

The conditional distribution $x|m$ follows $\text{Bin}(n, d)$. The distribution of $m$ is simply $\text{Bin}(2K, p)$. From these, we can estimate $P(m|x)$, the actual AF (counts) given the read data. 

\end{itemize}

The Power of Gene-Based Rare Variant Methods to Detect Disease-Associated Variation and Test Hypotheses About Complex Disease, [Moutsianas \& McCarthy, PLG, 2015]
\begin{itemize}
	
	\item Some known cases of rare variants: NOD2 for Crohn's Disease. PCSK9 for coronary heart disease. LPL for hypertriglyceridemia (154 missense variants with MAF $<0.1\%$, T1 association test). MTNR1B for T2D (13 variants, OR = 5.5, KBAC test). 
	
	\item Simulation of genotype: HAPGEN2, mimic the observed SFS in 202 genes. Average length of coding sequences. 
	
	\item Simulation of genetic architectures: sample up to 35 causal variants per loci, VE = 1\% (variance explained, 1\% is among the strongest loci found, TCF7L2 for T2D). Three main scenarios (AR1-AR3): strong, moderate or weak selection on causal variants, leading to inverse correlation between AF and effect size. AR4-AR5 are variations where only rare variants are causal. AR6: 50\% risk and protective variants.
	\begin{itemize}
		\item Frequency-effect size distribution: from simulations under selection on disease (T2D parameters: prevalence 8\% and heritability 45\%)
		\item Sampling causal variants: Variant effects were sampled until the cumulative variance explained (VE) on the liability scale by each locus reached the desired threshold 
	\end{itemize}
	
	\item Power of methods: 1.5K cases and 1.5K controls
	\begin{itemize}
		\item Power at 2.5E-6 is low ($<20\%$) for all genetic architectures. Also low at $\alpha = 0.05$. MiST and SKAT-O are best, followed by KBAC. 
		\item At $\alpha = 0.05$: KBAC is best. This high sensitivity can be used, e.g. for candidate gene studies. 
	\end{itemize}
	
	\item Comparison of single-variant and gene-based tests: the comparative advantage of gene-based tests was most evident when there is strong purifying selection against causal alleles. Ex. AR4, power of single variant test is $<5\%$ vs. gene-based test 20\%. Under AR2 and AR3, single-variant test performs better, though each method detects unique loci. 
	\begin{itemize}
		\item Conclusions: single variant and gene-based association methods should be jointly employed for maximal power across divergent locus architectures (union in Figure 3). 
	\end{itemize}
	
	\item Sensitivity to fraction of causal variants: burden tests are highly sensitive to the fraction, while MiST and SKAT-O are relatively immnue. 
	
	\item Effect of sample size: Even in 10K case-control samples, power remains modest (60\% at $\alpha = 2.5 \times 10^{-6}$). 
	
	\item Concordance between methods: high concordance between SKAT-O and MiST (0.92), between SKAT-O and KBAC (0.86 under AR2). The scenario where KBAC performs best: aggregate skew of case vs control counts. 
\end{itemize}
%%%%%%%%%%%%%%%%%%%%%%%%%%%%%%%%%%%%%%%%%%%%%%%%%%%%%%%%%%%%
\subsection{Rare Variant Studies}

Rare coding variants in the phospholipase D3 gene confer risk for Alzheimer's disease [Cruchaga \& Goate, Nature, 2014]; TERM2 variants in Alaheimer's disease [NEJM, 2013]
\begin{itemize}
	\item Strategy: use WES in families to identify candidate genes first, then use WES in case-control to find the risk genes. 
	
	\item Family study: families with history of LOAD in four or more members. Choose 14 (out of 800) families to do WES, then choose candidates by: segregation of variants with status; filtering by MAF and function; shared by multiple families. Identify a single variant in PLD3. 
	
	\item Case-control study to map PLD3: sequencing in 2,000 cases and 2,000 controls. Justification of PLD3: expression lower in LOAD, PLD1 and PLD2 previously implicated in APP trafficking and LOAD. Also functional study in N2A cells that express the gene: show that it affects APP and amyloid-$\beta$. 
	
	\item Pattern of rare variant in PLD3 and TERM2:
	\begin{itemize}
		\item Exists 1-2 relatively common (0.5-1.5\%) variants that drive the burden. But by themselves, not strong enough.  
		\item Relatively common variants could be neutral, e.g. TERM2, a variant 25/31. This reduces the power of burden: TERM2 $p = 0.02$.  
		\item Possible RVs with large effects: e.g. 4/0 and 3/0 in PLD3 and 6/0, 3/0 in TERM2. 
	\end{itemize} 
\end{itemize}

Whole-exome sequencing identifies rare and low-frequency coding variants associated with LDL cholesterol [Lange, AJHG, 2014]
\begin{itemize}
	\item Data: 2000 subjects, including 300 with extremely high and 247 extremely low LDL levels. 
	
	\item RVAT: (1) 5 burden (CMC) test: including burden test on non-silent variants at 4 AF threshold, and burden test on LoF; (2) SKAT-O: LoF, LoF and missense, LoF and probably damaging. 
	
	\item Single variant results: only one variant in APOE. 
	
	\item Burden test: 3 genes reach threshold in stage 1 and several sub-threshold ones replicated. Find the best test for each gene. The patterns of association: 
	\begin{itemize}
		\item Relatively high frequency variants (about 1\%): could drive association signal. But often neutral. 
		
		\item Very low frequency variants (0.1\%): there are cases with clear signal, e.g. 5/1, 6/1, 8/0 in PCSK9. 
		
		\item For some genes, the signal is dominated by a few relatively common variants, e.g. PCSK9, PLD3; for other genes, overall burden of very rare variants, e.g. LDLR (non-silent), and APOB (LoF). 
	\end{itemize}
\end{itemize}

Exome sequencing identifies rare LDLR and APOA5 alleles conferring risk for myocardial infarction [Nature, 2015]
\begin{itemize}
	\item Study design: single variant association using arrays (less than 5\%); rare variant burden test. 
	
	\item Data: about 4000-6000 cases (early-onset MI) and controls, respectively. 
	
	\item Negative results with single variant test. Gene-based test: limit to MAF $<1\%$, several categories, NS, deleterious (PPH2, broad, strict), and LoF. 
	
	\item Pattern in APOA5 and LDLR: (1) APOA5: burden in both relative common variants and singletons. (2) ORs: about 2 for NS, and 4-13 for LoF. (3) PPH2 and other annotations: sometimes increase the OR, but sometimes not (APOA5). 
\end{itemize}

A large genome-wide association study of age-related macular degeneration highlights contributions of rare and common variants [Fritsche, NG, 2016]
\begin{itemize}
	\item Study design: exome chip. Do sequencing to find out RVs of some candidate genes, then include these RVs in the exome chip. Sample size: 17k cases and 17k controls.  
	
	\item Single variant analysis: 34 independent loci. Most associated variants are common, and 7 are rare: frequency from 0.01\% to 1\%, and OR from 1.5 to 47.6. 
	
	\item RVAT: conditioned on single variants, four genes show significant burden. 
	\begin{itemize}
		\item CFH: 88/38, with seven RVs with signal 6/0, 10/0, 8/0, 5/0, etc. 
		\item CFI: 213/82, with many RVs with signal, 17/2, 9/0, 6/0, 36/10, etc. 
		\item TIMP3: 29/1, with clear RVs 14/0, 5/0, 4/0, etc. 
		\item SLC16A8: driven by a signle splice variant 370/278. 
	\end{itemize}
	 
	 \item Gene priortization score (GPS) table: gene annotations include expression in retina, eye phenotype in mouse, RVAT burden, drug target, etc. 
	 
	 \item \textbf{Remark}: a simple strategy of combining CV and RV is, limit the analysis to genes with some CV signal, thus reducing multiple testing burden. 
\end{itemize}

Exploring the genetic architecture of inflammatory bowel disease by whole-genome sequencing identifies association at ADCY7 [Luo \& Anderson, NG, 2017]
\begin{itemize}
	\item WGS low coverage on 4K IBD patients. Imputation from these sequences into existing GWAS cohorts of 16K cases and 18K controls. 
	
	\item Single SNP analysis: a 0.6\% missense SNP in ADCY7 that doubles the risk of UC. 
	
	\item Burden analysis: low coverage, so correct for sequencing depth. A significant burden of damaging variants in a known Crohn's disease gene NOD2. 
	
	\item Lesson: low-coverage sequencing, then imputation on very large cohorts, seems a good practical strategy. 
\end{itemize}
%%%%%%%%%%%%%%%%%%%%%%%%%%%%%%%%%%%%%%%%%%%%%%%%%%%%%%%%%%%%
\section{Copy Number Variations and Structural Variations}

Statistical tests of CNVs [personal notes]
\begin{itemize}
	\item CNV burden analysis: test if CNVs (deletions or duplications, or separately) are enriched in cases than in controls. Standard test: Fisher's enrichment test. Multiple metrics of burden: CNV number/rate, CNV size and gene content.
	
	\item Testing individual CNVs or CNV regions (CNVRs): Fisher's enrichment test: individuals with a CNV vs. without a CNV. To control for systematic difference bewteen cases and controls, use regression analysis [Pinto, Nature2010; Raychaudhuri, PLG2010; CCRET]. 
	
	\item Gene set analysis: similar to testing individual CNVRs, define a variable for gene set disruption due to CNV and test its correlation with phenotype status using Fisher's enrichment test or regression [Pinto, Nature2010; Raychaudhuri, PLG2010; CCRET]. 
	
	\item Remark: CNV-level analysis and gene set analysis are typically done separately in literature. The ideal approach is to combine them: e.g. first infer CNVs, then translate the knowledge of CNVs into genes. 
	
	\item Common problems: 
	\begin{itemize}
		\item Multi-gene CNVs: a CNV spanning multiple genes provides evidence of multiple genes, and the true evidence of a single gene is lower than CNV-level evidence. 
		
		\item Clustering of genes in a pathway: if we count the number of times a gene is disrupted, this could create spurious signal at the pathway level (as one CNV is counted multiple times). 
		
		\item Bias of CNV number and size in cases vs. controls: need to account for when analyzing a specific pathway.  
	\end{itemize}
\end{itemize}

Copy-number variation and association studies of human disease [McCarroll \& Altshuler, NG, 2007]: 
\begin{itemize}
	\item Challenges of genotyping CNVs:
	\begin{itemize}
		\item Most reported CNV locations actually correspond to the locations of CNV-containing regions (CNVRs), generally the genomic coordinates of a BAC probe, set of oligonucleotide probes, or fosmid from which a variant was discovered. A reported CNVR is consistent with a large number of potential variants 
		\item Until approaches for genome-wide CNP genotyping mature, a placeholder strategy may be to rely on raw hybridization measurements as an approximation to an unknown, underlying genotype.
	\end{itemize}
	
	\item Using SNPs as markers for CNPs: 
	\begin{itemize}
		\item One would genotype the CNP in the HapMap (or other reference) samples and assess whether nearby SNPs were able to capture the CNP through linkage disequilibrium. Using available SNP data and PCR-based genotyping of deletion polymorphisms, initial studies found that deletion polymorphisms are generally ancestral and are tagged by SNPs. 
		\item CNV calling algorithms: iPattern implements a non-parametric density-based clustering model that integrates intensity data across samples to assign individual samples to distinct copy number states. QuantiSNP uses an Objective Bayes (OB) Hidden-Markov Model (HMM) approach for CNV callings
	\end{itemize}
	
	\item Testing disease association of common CNPs: 
	\begin{itemize}
		\item For common CNPs, statistical tests will involve a straightforward comparison of allele frequencies between affected individuals and controls in a population cohort; between transmitted and untransmitted chromosomes in families with affected offspring; or between affected and unaffected siblings. 
		\item Some CNPs seem to involve more than three copy-number classes, and therefore more than two copy-number alleles. Need to extend methods: reduce to binary case (e.g. copy number $< 4$ vs. $\geq 4$); logistic regression; etc. 
	\end{itemize}
	
	\item Testing disease association of rare CNPs:
	\begin{itemize}
		\item Difficulty: given the existence of hundreds of rare CNVs with apparent frequencies of less than one percent, even in a well designed study it will frequently occur that a CNV is present (for example) in 3/200 cases and 0/200 controls. Such results are expected to occur by chance in a genome-wide search. 
		
		\item Collapsing: In the case of rare coding SNPs, a framework is typically used in which nonsynonymous SNPs are examined based on their a priori likelihood of functionality. In the case of CNVs, similar paradigms may be useful: for example, pooling just those CNVs confirmed as affecting a candidate gene's coding sequence and nearby highly conserved elements. 
		
		\item CNV burden analysis [Pinto et al, Nature, 2010]: use CNV size, CNV rate, etc. as measurs of burden, then compare the burden measures in cases and in controls. Ex. in cases, for a gene to be tested, there may be 10\% chance of having a rare CNV, while the chance is 2\% in the controls. 
	\end{itemize}
	
\end{itemize}

Human copy number variation and complex genetic disease [Girirajan \& Eichler, ARG, 2011]
\begin{itemize}
	\item Definition and detection of CNVs: CNVs typically between 1Mb (chromatomal aberrations) and 50 bp (indels). Detection relies on SNP microarrays, or CGH. The limit of CNVs using SNP microarrays is about 20-30 kb. 
	
	\item CNVs in population: 
	\begin{itemize}
		\item Between any two individuals, bp difference due to CNV is $>100$ larger of SNPs. 
		\item Common CNPs (copy number polymorphism), $>1\%$, often in multicopy number states (0 to 30). 
		\item Rare CNVs (many may be de novo), often larger ($>100$ kb), under strong selective pressure.
		\item Large CNVs are individually rare (more than 71\% CNVs larger than 100kb are rare), but collectively common. 65 - 80\% individuals carry at least a large CNV, 1\% carry a large CNV $>1$Mb in size. 
		\item De novo CNV rates: 8K - 25K bases of genomic DNA were added or lost during each transmission. 
	\end{itemize}
	
	\item Mechanisms of CNVs: commonly caused by nonallelic homologous recombination (NHAR) between high-identity segmental duplications, causing ``hotspots''. The frequency of these events partly determined by the size and sequence identity of the flanking segmental duplications. 
	\begin{itemize}
		\item Segmental duplication: large ($>1$kb) blocks that have nearly identical sequences ($>95\%$), as a result of duplication. SDs can be tandem or interspersed, and can be interchromosomal or intrachromosomal. 
		
		\item A large fraction of CNV disease burden is contributed by other nonhomologous mechanisms (non-recurrent or lower frequency), such as microhomology/microsatellite- mediated break-induced repair. Establishing the disease relevance is harder for these CNVs. 
	\end{itemize}  
	
	\item Evidence supporting association of rare CNV and diseases: 
	\begin{itemize}
		\item Frequency of pathogenic CNVs (known to be associated with diseases, called ``genomic disorder'') in children with developmental delays (Figure 2). Ex. 16p11.2 appears in 4\%. Many of them are recurrent, and highly penetrant. 
		
		\item In individuals with developmental delays or IDs, 10\% have de novo CNVs of size 500kb to 12Mb. 
		
		\item Many recurrent CNVs are associated with \textit{variable penetrance and expressivity}. Ex. 15q13.3 with developmental deley, autism, SCZ, epilepsy. 
		
		\item Example of pathogentic nonrecurrent CNVs: in a SCZ study, 362-kbp microduplication overlapping VIPR2 was detected in 2.5\% cases, but only 0.03\% of controls.  
	\end{itemize}
	
	\item Linking rare CNVs to rare diseases: 
	\begin{itemize}
		\item Clinical diagnosis: large CNVs are considered pathogenic if they arise de novo in proband, are rare ($<1\%$). Diagnositc yield between 5-20\%. 
	\end{itemize}
	
	\item Linking rare CNVs to common complex diseases:
	\begin{itemize}
		\item De novo CNVs in Autism: Sebat et al, 165 case families with 99 control families, 7.2\% de novo CNVs in cases vs. 1.02\% in controls. 
		
		\item De novo CNVs in SCZ: some studies fround 2-3 times burden in SCZ, while another study fails to find using 1,000 cases and 1,000 controls in SCZ. One study of 359 SCZ families, 8-fold enrichment in de novo CNVs, and 1.5 fold in rare inherited CNVs. In Simplex families, de novo CNVs more common, in multiplex families, burden of rare inherited CNVs. 
		
		\item Possible sources of difference in CNV burdens across studies: sample quality, cell line artifacts, probe resolution, GC content, lack of genotype in- formation, subphenotype characterization and clinical heterogeneity, age of onset of disease, and platform-specific biases. 
		
		\item A model of de novo and inherited CNVs in autism: (1) One study found evidence of strong dominant transmission. Model: de novo CNVs in low-risk families, while inherited CNVs (from mother) in high-risk families. (2) ANother study found 4-fold enrichment of de novo CNVs in cases in multiplex families. Model: sensitized genetic background, requires de novo CNVs. 
	\end{itemize}
	
	\item Methods for studying rare CNVs in complex diseases (Figure 4):
	\begin{itemize}
		\item De novo CNVs (FIgure 4a).  
		\item Comparion of CNV frequency in cases and in controls. Figure 4b.
		\item Gene-based or sliding window approach: for each gene, compare the number of CNVs covering this gene in cases vs. in controls. Figure 4c. 
		\item Comparison of CNV rates in cases vs. controls at a given CNV size (Figure 4d): could be used for estimating OR. 
		\item Pathway-analysis: Figure 4e. 
	\end{itemize} 
	
	\item Model of rare CNVs (genetic architecture): multiplex cases occur in high-risk familes. In low-risk families, autism may occur with de novo high-risk CNVs.  
	
	\item CNPs in human population: CNPs are 4-10 fold enriched in regions of segmental duplications. Bi-allelic or multicopy CNPs. 
	\begin{itemize}
		\item Mechanisms leading to CNPs: microhomology of sequences at the breakpoints, NAHR,andL1 retrotransposition. Almost all cases of CNPs are associated with SDs. 
		
		\item CNPs are enriched with immune genes and environment response pathways, suggesting possible positive selection. Some example that CNP frequency difference across population may be driven by positive selection: CCL3L1 CNV in HIV, more common in Afrian ancestry; OCLN, which is required for hepatitis C viral entry; a taste receptor cluster on chromosome 12. 
	\end{itemize} 
	
	\item Role of CNPs in diseases:
	\begin{itemize}
		\item Assessing CPNs. For SNP arrays: often lack of probes in duplication rich regions, almost half of deletions are not captured by high-density SNP arrays. Also all array approaches are based on reference genome, missing many variations (e.g. novel insertions).  
		
		\item Methods for studying CNP association with disease:  (1) For low copy numbers: first obtain integer copy numbers, then test association (Figure 5a). (2) For high copy number CNVs: may not be able to assign integer copy numbers, compare the distribution in cases vs. controls. 
		
		\item Using SNPs to test association of CNPs: mostly focus on biallelic CNPs, the majority of which are in high LD with SNPs. However, CNPs in segmental duplication-rich regions show less LD to SNPs. 
		
		\item CNPs are often associated with immune diseases (Table 1), but also examples of T2D, obesity, CHD.   
	\end{itemize}
\end{itemize}

Autism genome-wide copy number variation reveals ubiquitin and neuronal genes [Glessner \& Hakonarson, Nature 2009]
\begin{itemize}
	\item Data: Autism Case-Control (ACC) cohort of 859 ASD cases and 1,409 controls. For replication, use AGRE (1,336 cases) and 1,110 controls. On average, 15.5 CNV calls per individual (similar in cases and in controls).  
	
	\item 16p11.2: relatively common in controls (3), and not statistically significant. 
	
	\item Segment-based scoring approach: scan the genome for SNPs, and measure the SNP copy number changes in cases vs. controls using Fisher's exact test. Then for a CNV region (CNVR), find the local minimum and use permutation to assess $p$-values. 
	
	\item Report 8 CNVs (Table 2): most CNVs have 0-1 copies in controls, and 3-4 in cases. OR about 5 for those that occur in controls. Most have $P$-values from 0.001 to 0.04. (Unadjusted). Each CNV contains about 1-2 genes. Four genes are ubiqutin gene family. 
	
	\item Gene-based scoring approach: for each gene, consider the CNV calls affecting this gene region in cases and in controls. Identify 7 further genes at incresaed frequency of CNVs in ASD cases vs. controls.  
\end{itemize}

Functional impact of global rare copy number variation in autism spectrum disorders. [Pinto \& Scherer, Nature, 2010]
\begin{itemize}
	\item Data: rare CNVs (less than 1\%, greater than 30k in size) in 996 ASD cases (876 trios) and 1,287 controls. 
	
	\item CNV burden analysis in cases vs. controls: use three measures of burden per individual, the number of CNVs, the estimated CNV size, and the number of genes affected by CNVs. Use permutation to assess the burden. Only the last one (genic CNVs) show a burden of 1.19 (1.26 if deletions only).  
	
	\item Methods for gene set burden analysis: 
	\begin{itemize}
		\item Compare CNV frequency in cases vs. controls: define the proportion of indivdiuals containing CNVs overlapping the test gene set, and compare using Fisher's exact test.  
		
		\item Regression analysis for CNV effect on phenotype: regress of phenotype on the number of genes in the test set overlapping CNVs. 
	\end{itemize}
	
	\item CNV burden analysis with ASD gene lists: known ASD genes (36) - burden is 1.8, or 4.3\% in cases versus 2.3\% in controls. ID genes ($>100$): burden 2.1. 
	
	\item Gene set burden analysis: Novel sets include: GTPase/Ras signaling, microtubule cytoskeleton. Functional map of gene sets: gene set as nodes and overlap between sets as edges. 
	
	\item ASD candidate genes: based on de novo CNVs in cases but not controls, SHANK2, SYNGAP1 and DLGAP2. 
\end{itemize}	

Accurately Assessing the Risk of Schizophrenia Conferred by Rare Copy-Number Variation Affecting Genes with Brain Function [Raychaudhuri \& Daly, PLG, 2010]
\begin{itemize}
	\item Motivation: possible biases for CNV-based gene set analysis if the analysis is performed at gene level: 
	\begin{itemize}
		\item Brain genes tend to be bigger: case only analysis to compare the relative frequency of gene sets could be biased. 
		\item Different CNV rates and sizes in cases vs. controls: if CNVs are more common in cases, then any gene set may have higher chance of being more frequent in cases than in controls. 
		\item Clustering of genes in a single pathway in the chromosome: if we count the number of genes affected by a CNV in cases vs. controls, we may double count the same genes. 
		\item Loss of information of CNV rates affecting a gene: if we compare the number of genes in a pathway affected by CNVs in cases. vs. the number in controls, a gene overlapped CNV in both cases and controls would not contribute, regardless of the rates of CNVs in cases or controls. 
	\end{itemize}
	
	\item Model: let $y_i$ be the phenotype of individual $i$, and $g_i$ be the count of gene within a pre-specified gene set affected by a cnv in $i$. We regress $y_i$ on $g_i$, while controlling for confounders: $c_i$, the number of CNVs of individual $i$ and $s_i$, the average size of these events. 
	
	\item Simulations: simulate CNVs of cases and controls. Randomly place genes but brain genes are bigger. Introduce bias: e.g. larger CNVs in cases, or higher CNV rates in cases. Test the enrichment in brain genes: high type I error in the models that do not account for the bias. 
	
	\item Gene size bias: brain genes tend to be bigger. In CNV controls, genes affected by rare CNVs are involved disproportionately in brain function. Fisher's exact test: using no. of genes (fraction of disrupted brain genes vs. fraction of brain genes in the genome). 
\end{itemize}

Convergence of genes and cellular pathways dysregulated in autism spectrum disorders [Pinto \& Scherer, AJHG, 2014]
\begin{itemize}
	\item Data: 2,845 ASD familes and 2,640 control families. Rare CNVs: 1\% and 30k or larger in size. 6.8K CNVs in total. 
	
	\item Global burden in rare CNVs: 1.41 fold. CNV sizes: 188kb in cases and 159kb in controls.  
	
	\item De novo CNVs: 102 rare de novo CNVs, 4.7\% of cases have at least one de novo CNV vs. 1-2\% in controls. The average length of de novo CNVs: 1.17 Mb is larger in probands than siblings (0.55Mb), and affect more genes (3.8 fold) in cases than in controls. Gene content: about 18 in de novo CNVs. 
	
	\item CNV burden in genes implicated in ASD and ID: OR = 4.1 for CNVs overlapping dominant or X-linked genes. OR = 12.6 for Pathogenic CNVs and OR = 23.1 for pathogenic deletions. More than 50\% of pathogenic CNVs are de novo. 
	
	\item CNV burden for gene sets: FMRP targets (n = 842) and PSD genes (n = 1,453) carried a significant excess of both deletions (OR = 2-3) and duplications (OR = 1.5-2) in affected subjects. Gene sets show excess in deletions but not duplications: high brain expression (OR = 1.89), dominant neurological diseases and orthologous genes associated with abnormal phenotypes in heterozygous knockout mice (OR = 2.9), haploinsufficient genes (pHI $>0.35$, OR = 1.4) 
	
	\item Predictive model of disease risk: identify factors (burden) that correlate with disease status. Use logit model, where the explanatory variables are: number of genes in CNVs, number of deletions, number of brain-expressed genes in deletions, etc, and the outcome is the disease status. Results: number of deleted brain-expressed genes correlate with disease status (Figure 3C). 
	
	\item De novo SNVs (LoF) and de novo CNVs converge on functional gene networks: use NETBAG to identify genes in de novo CNVs, then assess the overlap of this gene list with de novo LoF genes. Found 11 common genes, such as NRXN1, SHANK2 and RIMS1 (synpatic rab3a-binding protein, new finding). 
	
	\item Lessons: 
	\begin{itemize}
		\item \textbf{Key statistics of CNVs}: (1) De novo CNVs: about 4\% in probands and 1-2\% in controls; size about 1Mb and 18 genes. (2) Inherited CNVs: rare CNVs about 1-2 per subject; size 150-200 kb and 1-3 genes. 
		\item \textbf{Logit modeling}: as alternative way of burden analysis. Good for controlling other variables.  
	\end{itemize} 
\end{itemize}

Refining analyses of copy number variation identifies specific genes associated with developmental delay [Coe \& Eichler, NG, 2014]	
\begin{itemize}
	\item Data: aCGH data of 29,085 primarily pediatric cases with intellectual disability, developmental delay and/or ASD in comparison to 19,584 adult population controls. 
	
	\item Burden of rare CNVs (1\%, $\geq 500$kb): burden for deletions, OR = 5.09, for duplications, OR = 1.76.  Analysis of 2,086 transmissions showed that likely deleterious CNVs were transmitted preferentially from mothers.
	
	\item CNV analysis: some CNVs are associated with SDs, and they are recurrent, so we use standard case-control comparison. Other CNVs may overlap, but have different breakpoints in different subjects (Multiple Breakpoints, or MB), and we need a different analysis. 
	
	\begin{itemize}
		\item CNV level analysis: counts in cases and controls, Fisher's exact test (Table S2). 2,000 CNVs in 55 regions. 19 regions were likely pathogenic and reach nomial significance. Most (40/55) are genomic hotspots flanked by segmental duplication.
		
		\item Gene-level analysis: 1,945 genes enriched for deletions and 2,633 genes enriched for duplications at $P < 0.01$, one tailed Fisher's exact test. 
		
		\item Region analysis: many genes were clustered, so perform region level analysis, using Fisher's exact test. Regions are defined using rare case CNVs of $>250$kb: find unique breakpoints to define the boundaries. Use simulation to obtain $p$-values. Found 14 significant regions, most not flanked by SDs. 
	\end{itemize}
	
	\item Methods for combined CNVs and SNVs: for each gene, count the number of times the SNV (or CNV) occurs in cases vs. in controls. Let $a$ and $b$ be the number of cases with and without CNVs, and $c$ and $d$ the numbers of controls. Define similar numbers $a_2, b_2, c_2, d_2$ for SNVs. Then test statistic is:
	\begin{equation}
	Z = \frac{a}{a+b} + \frac{a_2}{a_2+b_2}
	\end{equation}
	Assess $Z$ using controls: hypergeometric distributions. 
	
	\item CNV and SNV analysis: 
	\begin{itemize}
		\item In genes with single LoF (247 genes): 43 show excess of deletion in CNV case-control data, the burden is OR = 1.15, not significant. In 21 genes with 2 or more LoF mutations, OR = 2.72.	
	\end{itemize}
	
	\item Remark:
	\begin{itemize}
		\item The combined test: the treatment of CNVs is exactly the same as that of SNVs. Ex. if SNVs and CNVs data have the same sample size, then the test statistic is basically the total number of events affecting a gene (SNV and CNV). No effort has been done to address the fact that a multi-gene CNV only ``partially support'' a gene.  
		
		\item The de novo SNV information is not used, which is significantly more informative. 
	\end{itemize}
\end{itemize}

Novel Findings from CNVs Implicate Inhibitory and Excitatory Signaling Complexes in Schizophrenia [Pocklingtong \& Owen, Neuron, 2015]
\begin{itemize}
	\item Data: 11K cases and 16K controls from three datasets. CNVs called from genotyping arrays. Analyses are based on large, rare CNVs ($>$ 100 kb, frequency $< 1\%$). 
	\item Background: common variants from GWAS explain only 1/3 of heritability, and the 108 GWAS loci explain only a small fraction. 
	
	\item Intuition: to test if a gene is associated with risk, we check among all CNVs affecting this gene, how many are in cases vs. in controls. If this ratio is large, it suggests that this gene is likely a risk gene. But we need to compare this ratio with some background. Also we should control for the possible difference between the background CNVs and the CNVs covering this gene: e.g. CNV size. This is best done by a regression model. 
	
	\item Testing association of a feature (gene or gene set) with phenotype: suppose we have $n$ CNVs (CNV occurrences, more precisely). For each CNV, it has some features, gene size, whether it contains a gene, etc. We want to test if a feature increases the chance that the CNV occurs in cases. Use logistic regression model. 
	
	\item Gene set and single-gene analysis: use the same logistic regression model. (1) 134 predefined CNS gene sets from MPI Mammalian Phenotype (MP) database. (2) For single gene test, the problem: the $p$-values will be highly correlated for genes of the same recurrent CNV. 
	
	\item Enrichment of de novo nonsyn. mutations in gene sets: define a minimum CNS gene set (capturing all gene set enrichment singal). In this set, a burden of about 2 in deletion only, and duplication only. 
	
	\item Remark/Question:
	\begin{itemize}
		\item In Table 5, why enrichment of DNM is lower in combined set, about 1.2, vs. deletions or duplications along (about 2)? 
		\item Problem of applying logistic regression on single gene anlaysis: correlation of adjacent genes. 
	\end{itemize}
\end{itemize}

A New Method for Detecting Associations with Rare Copy-Number Variants (CCRET) [Tzeng, PLG, 2015]
\begin{itemize}
	\item Motivation: while cnv-enricnment-test (PLINK) addresses some of the biases in CNV analysis, it does not address the effect heterogeneity issue: between-locus (across CNVs) and within-locus (deletion vs. duplication of the same CNV). 
	
	\item Problems: (1) test CNVR dosage effect (DS): could be applied to one CNVR (CNV region) or a group or all CNVRs, to test if dosage disruption has any effect. (2) Test gene set (GI: gene interaction): test if disruption of genes in a set has any effect. 
	
	\item Model idea: similar to PLINK CNV test, control for difference of CNV size and numbers between cases and controls using fixed effects. But instead of collapsing all genes of the same pathway, consider the heterogeneity of possible effects. Similarly, for burden analysis of CNV dosage effect, consider heterogeniety of dosage effects across CNVs. 
	
	\item GI effect model: we are testing a particular pathway. Let $Y_i$ be phenotype of individual $i$, and $\mu_i = \E(Y_i)$. Let $\tilde{Z}_i^{len}$ the average length of $i$. Let $Z_{im}$ be the disruption indicator of gene $m$ in $i$. We have the model: 
	\begin{equation}
	g(\mu_i) = \beta_{Len} \tilde{Z}_i^{len} + \sum_m \beta_{m} Z_{im}
	\end{equation}
	where $g(.)$ is the link function, $\beta_m \sim N(0, \tau^2)$ is the random effect of gene $m$. The model is equivalent to a model where we use a random effect variable that is correlated across individuals (covariance matrix is given by the kernel). 
	
	\item DS effect model: similar to GI model, but use CNVRs instead of genes. 
	
	\item Simulation: CNV data of 2,000 cases and 2,000 controls, sampled from $>6$K individuals in TwinGene study. Identify 1,757 CNVR from 2,000 samples.  
	\begin{itemize}
		\item DS model study: sample 600 causal loci from 1757 CNVRs. For each causal CNV, sample its effect $\beta_m$ from a constant (no heterogeneity), or from a mixture of risk and protective effects (change the proportion). The effect size/OR is from 1-7. 
		
		\item GI model study: sample from 69 PSD genes that intersect with 1700 CNVRs (among a total of 668 PSD genes). Then do similar sampling for their effects: OR is chosen 1.5. 
	\end{itemize}
	
	\item Remark: some issues with the model
	\begin{itemize}
		\item The GI model does not address the issue that a CNV carries information of multiple genes. So a CNV covering three genes can contribute to the evidence of all three genes. 
		
		\item Multiple counting problem in gene set analysis: $Z_{im}$ in the DS model encodes whether gene $m$ is disrupted in subject $i$. However, when two genes in a pathway are physically close and disrupted by a CNV, the two genes will be counted twice.
		
		\item CNVR is defined as (sort of) the union of multiple overlapping CNVs. However, what is more interesting is the intersection of multiple CNVs with supporting evidence. By focusing on CNVR, we lose information and actually make it harder to identify causal genes.  
	\end{itemize}
\end{itemize}

CNV-association meta-analysis in 191,161 European adults reveals new loci associated with anthropometric traits [Mace, NC, 2017]
\begin{itemize}
	\item Data: UKBB (110K) and GIANT (56K), BMI, height, etc. Association test: infer dosage for each probe, and test each of about 1M probes. Multiple testing correction: 29K independent tests.
	
	\item Genomewide significant hits: 7 CNVs, with AF 0.01-0.07\%.
	
	\item Example: 16p11.2 region, about 9 genes, similar p-values for many probes in this region (Figure 2).
	
	\item MC4R region: also coding SNP show association. Independent signals.
	
	\item For all CNVs found, except 16p11.2, CNV signal and common SNPs are independent.
\end{itemize}

Contribution of copy number variants to schizophrenia from a genome-wide study of 41,321 subjects [Marshall and Sebat, NG, 2017]
\begin{itemize}
	\item Data: 20K cases and 20K controls. CNVs: at least 10 probes, $>20$kb and MAF $<1\%$.
	
	\item Global burden and gene set burden: use number of genes as burden metric, global burden = 1.2 (strongest). Use number of CNVs as burden metric, OR = 1.03. Significant gene burden (esp. deletions) in 15 out of 40 gene sets.
	
	\item Individual loci/genes: (Table 1) 16 loci at BH FDR $< 0.05$. About half are known. AF in controls: 0-7, and OR 3-20. Also a small set of protective CNVs.
	
	\item CNV breakpoint analysis: use breakpoint as analysis unit.
\end{itemize}

The impact of structural variation on human gene expression [Chiang and Hall, NG, 2017]
\begin{itemize}
	\item Data: 147 GTEx samples ini 13 tissues. Test 9000 common SVs. Found 5000 associations at 3K genes, and 1.6K SVs.
	
	\item 10\% of SV-eQTLs: change exons, most show effect consistent with SV classes.
	
	\item Estimating contribution of common SVs to gene expression: (1) Joint analysis with SNVs: estimate about 3.5\% of eQTLs are due to SVs. (2) Use LMM to estimate h2g from SVs: estimate about 7\%. (3) For individual SVs: 30-50 times more likely to be eQTLs vs. SNVs.
	
	\item GWAS: 52 loci where causal SV-eQTLs are in LD ($>$0.5) with GWAS variants. Some examples (Fig. 4): 1.4Kb deletion in an intron, LD (0.7) with GWAS SNP of RA.
	
	\item Rare SVs on gene expression: rare SVs are 16 times more likely to be close to gene expression outliers than SNVs.
	
	\item Lesson: about 7\% contribution of eQTLs are from common SVs. Some SNP associtions are likely driven by common SVs. Rare SVs probably also important, though under-powered to detect.
\end{itemize}

Properties of structural variants and short tandem repeats associated with gene expression and complex traits [Jakubosky and Frazer, NC, 2019]
\begin{itemize}
	\item Data: 350 iPSC samples, WGS, and association of SNVs, indels and SVs with expression. Classes of SVs: DEL, DUP, MEI (mobile element insertion), STR, and mCNVs (multi-allelic). Total of 42K common SVs.
	
	\item LD between common SNVs and SVs: for DEL and STR, 80\% common SVs are tagged by common SNVs, however, for DUP, only 30\%.
	
	\item Mapping eQTLs: use all variants or use only SVs. About 6000 eGenes shared: about 14\% of them lead eVariants are SVs. This suggests that at least a certain fraction of eQTLs are explained by common SVs.
	
	\item Association with SV sizes: for SVs $>$50kb, the chance of being lead eVariant has OR = 3.
	
	\item Spatial distribution of eQTLs: 2\% in exons, 9\% in promoters and 17\% introns.
	
	\item Enrichment of eQTLs (SVs) near chromatin loops: use Pc-HiC data from iPSC. Variants overlap with distal anchors (the other anchor is promoter) are 3-5 times more likely to be eQTLs, about 5-6\% vs. 1-2\% by chance (Fig. 5e).
	
	\item Multi-gene SVs: SVs that are in chromatin loops of multiple genes: more likely to be multi-gene eQTLs.
	
	\item Comparison with GWAS: 40\% of eVariants (SVs) associated with two or more eGenes were in strong LD with a GWAS variant vs. 20\% for eVariants associated with only one eGene.
\end{itemize}

Mapping and characterization of structural variation in 17,795 human genomes [Abel and Hall, Nature, 2020]
\begin{itemize}
	\item Data: 17K WGS, 40\% European, 9\% Finnish, 24\% African, 16\% Latino. NIH Common Disease (CCDG), PAGE (population) and Simons Diversity Panel.
	
	\item Computational workflow: Ext. Figure 1. per sample variant discovery, merging and break point refinement, per sample re-genotyping and copy number, cohort level VCF.
	
	\item SV distribution per sample: Figure 1b, 4000 SVs, 35\% deletions, 25\% mobile element insertions (MEIs), 11\% tandem duplications. AFS similar to SNVs, with most common. About 100 rare SVs per sample, and 10 ultra-rare SVs. Total of 200K rare SVs.
	
	\item Burden of rare deleterious SVs: coding sequences. Total of 42K rare SVs affect genes: 9K deletions alter gene dosage, 26K function changes (e.g. single exon deletion), 7K increase gene dosage. Most rare SVs are deletions (55\%) and insertions (42\%). 23\% of all SVs affect $\geq$ 2 genes. Mean 4.2 rare SV-altered genes per sample, vs. 33.6 by small indels and SNVs.
	
	\item Method for analyzing burden of rare deleterious SVs: for a given variant class (SNV, indels, del, dup), use singleton rate (percent of singletons among all variants) as a measure of selection. Also use CADD and LINSIGHT to define impact scores of variants. Then assess singleton rates among different variant classes and different impact scores. Use syn. SNVs as control: about 40\% singleton rates
	
	\item Burden of rare deleterious SVs: noncoding sequences. Figure 3c: high-confidence LOF in SNVs, singleton rate $>$ 0.6 and at the top impact score, singleton rate $>$ 0.7. For Del: singleton rates in coding are even higher, for noncoding lower, but similar to high-confidence LoF. For Dup: weaker, in noncoding, similar to missense SNVs. In all variant classes, impact scores modest increase of singleton rates.
	
	\item Estimation of rare deleterious variants in an individual: Figure 3d.  Use the impact-score threshold based on LOF SNVs to choose a cutoff for defining deleterious rare variants. 120 deleterious rare variants per individual, $>60$\% are SNVs, and 17\% SVs (majority are deletions). A given rare SV is 841-fold more likely to be strongly deleterious than a rare SNV. Median length of rare deleterious SVs is 4.5kb vs. 2.8 kb for all rare SVs.  Top 50\% of noncoding rare SV-deletions are as deleterious as LOF SNVs and indels: measured by singleton rates.
	
	\item Ultra-rare SNVs: about 0.01 per sample, mega-sized.
	
	\item Deletion and duplication sensitivity scores: (1) for each gene: based on the observed frequency of CNVs. (2) For 1kb windows: scores correlate with LINSIGHT scores and other annotations. Scores: roughly, for each window, define presence and absence of SVs. Also each window we have annotations, so we can define log-OR for each annotation.
\end{itemize}

Increased burden of ultra-rare structural variants localizing to boundaries of topologically associated domains in schizophrenia [Halvorsen and Sullivan, NC, 2020]
\begin{itemize}
	\item Data: WGS from 1162 Swedish schizophrenia cases and 936 ancestry-matched population controls.
	
	\item Burden analysis: URVs defined as once in the study, no in Gnomad. Do aggregated burden analysis.
	
	\item SNVs and indels: (1) coding: burden in LOF only, and larger in pLI genes, OR = 1.2. (2) noncoding: promoters, conserved sequences (CDTS and GERP) within putative CREs - FIRE, ATAC, etc. No burden.
	
	\item Ultra-rare SVs: total of 7K DEL, 2K DUP and 700 INVs. Overall burden in DEL, OR = 1.08. Specific classes: TAD boundary (brain), H3K27ac, H3K4me3, ATAC, FIRE, Hi-C loops (intersected with different gene sets, including GWAS, FMRP, etc.). Only TAD boundary shows significant enrichment (94 SVs) OR = 1.6, CTCF OR = 1.4,  K27ac OR = 1.2.
	
	\item Estimation of h2g by WGS: pedigree based about 0.6. SNP heritability: common, about 0.48. Use WGS, about 0.52, however, large SE.
\end{itemize}

%%%%%%%%%%%%%%%%%%%%%%%%%%%%%%%%%%%%%%%%%%%%%%%%%%%%%%%%%%%%
%%%%%%%%%%%%%%%%%%%%%%%%%%%%%%%%%%%%%%%%%%%%%%%%%%%%%%%%%%%%
\chapter{Cross-Phenotype Analysis and Mendelian Randomization}
\section{Multi-Trait Analysis}

Multi-trait analysis: an overview [personal notes, Debashree Ray's paper]
\begin{itemize}
	\item Problem: suppose we have $K$ traits, in $n$ subjects. We are interested in the association of a SNP with the $K$ trait, let $\beta_k$ be the effect of the SNP on the $k$-th trait. The $K$ traits can have correlation structure. The model can be written as:
	\begin{equation}
	Y_{ik} = \beta_k x_i + \epsilon_{ik}
	\label{eq:MANOVA}
	\end{equation} 
	where $i$ is the index of subject and $k$ that of the trait. The error term of the $i$-th subject is iid: $\epsilon_i \sim N(0, \Sigma)$, where $\Sigma$ captures the covariance among traits. 
	
	\item Intuition why testing multiple traits can be more powerful than univariate test: (1) suppose we have a pleiotropic SNP, and we consider the case $K = 2$, and for simplicity, assume $\beta_1 = \beta_2=\beta$. So comparison of $Y_1$ and $X$, and $Y_2$ and $X$, both provide information of the shared parameter $\beta$. In particular, when $Y_1$ and $Y_2$ are uncorrelated (after removing $X$), both $Y_1$ and $Y_2$ provide independent information of $\beta$. (2) Surprisingly [Stephens13], even if only $\beta_1 > 0$, if $Y_1$ and $Y_2$ are correlated, incorporating $Y_2$ can be still useful: when testing $Y_1$ on $X$, we conditioned on $Y_2$, this removes some of the variance of $Y_1$, thus increase the power. A more formal analysis: in this case, our model is: 
	\begin{equation}
	Y_1 = \beta_1 X + \epsilon_1 \quad Y_2 = \epsilon_2
	\end{equation}
	And $\epsilon_1$ and $\epsilon_2$ are correlated. Then the model can be rewrite as something like: $Y_1 = \beta_1 X + \gamma Y_1 + \epsilon_1'$. So the joint model is equivalent to conditional test, controlling for $Y_2$. 
	
	\item Example: suppose we are testing association of a SNP with height, and we have weight data, which is correlated with height. Height varies in the population, not only by genetics, but also by environment such as diet. While we do not have diet, weight is a good marker of diet. So incorporating weight as covarite would remove the variation of height due to environment, thus improving the power. When weight itself may be associated with the same SNP, controlling weight when testing association with height would remove some signal, so multivariate analysis would be preferred.  
	
	\item Univariate tests: univariate test of each trait, then combine the results together. Ex. let $p_k$ be the $p$-value from the $k$-th trait, then we can combine them using Fisher's method, or minimum $p$ value. The null distribution however needs to take into account the dependence between traits. 
	
	\item MANOVA: we test the model in Equation\ref{eq:MANOVA}. Consider the case of $K = 2$. We have: 
	\begin{equation}
	Y_1 = \beta_1 X + \epsilon_1 \quad Y_2 = \beta_2 X + \epsilon_2
	\label{eq:MANOVA2}
	\end{equation} 
	where $\epsilon_1, \epsilon_2$ follows MVN with covariance matrix specified by $\sigma^2$ and $\rho \sigma^2$. The hypothesis being tested: $H_0: \beta_1 = \beta_2 = 0$, and $H_1: \beta_1 \neq 0$ or $\beta_2 \neq 0$. The test can be done using LRT, which is equivalent to the MANOVA test statistic (Wilk’s Lambda), see [Ray \& Basu, 2015]. Intuitvely, test statistic is the sum of squared error, divided by the total variance of $y$ (sum of squared error plus the explained variance of $y$) - this is similar to the $F$-test for regression: explained variance divided by the mean squared error (unexplained). When test statistic is large, it means that $X$ can explain a lot of variance, thus $H_1$ should be accepted. 
	
	\item Comparison of uni- and multi-variate approach: we consider four scenarios (true model) and understand which approach is preferred:
	\begin{itemize}
		\item Scenario 1: $\beta_1 > 0, \beta_2 = 0, \rho \rightarrow 0$, $Y_2$ does not provide any information of $Y_1$, and we still need to pay for extra d.o.f. under the multivariate test, so the univarite test is more powerful. 
		\item Scenario 2: $\beta_1 > 0, \beta_2 = 0, \rho \rightarrow 1$: $Y_2$ provides some information of $Y_1$, thus conditioned on $Y_2$, we can reduce the variance of $Y_1$, increasing the power of detecting $\beta_1$. In 2D plot of $Y_1, Y_2$, the data points of different $x$ would be more separable for $X$ in $Y_1$, when we condition on $Y_2$. In [Stephens, PLoS ONE, 2013], this is captured by $Y_U$: unassociated traits, but important to adjust. 
		\item Scenario 3: $\beta_1 = \beta_2 > 0, \rho \rightarrow 0$: pleiotropic SNP with independent traits, both trait provide independent information of $\beta$. So the multivariate test is more powerful. 
		\item Scenario 4: $\beta_1 = \beta_2 > 0, \rho \rightarrow 1$: the two traits are redundant, so adding $Y_2$ does not help much with learning $\beta$. The test will lose power [Ray \& Basu, 2015]. Alternative scenario is the ``indirect association'' model of [Stephens, PLoS ONE, 2013]: when $Y_1 = \beta_1 X + \epsilon_1, Y_2 = Y_1 + \epsilon_2$, then $\beta_1 = \beta_2$, and if $\epsilon_1 >> \epsilon_2$, we have $\rho \rightarrow 1$. With indirect association, multivariate tests also loses power, since $Y_2$ is entirely redundant/noninformative for $\beta_1$. 
	\end{itemize}
	
	\item Test that incorporates trait correlation only through distribution: the test statistic is some kind of combination of the multiple tests, but consider only the correlation through the distribution of the test statistic. For example, Sum of Squared Score (SSU) test. Let $U_k$ be the score statistic of the $k$-th test, and we have: 
	\begin{equation}
	U_k = \frac{1}{\hat{\sigma}_0^2} Y_k^T X
	\end{equation}
	where $\hat{\sigma}_0^2$ is the MLE of $\sigma^2$, $Y_k$ is the $n$-dim vector of the $k$-th trait, and $X$ the $n$-dim vector of genotype. The SSU statistic is $T = \sum_k U_k^2$. Under $H_0$, $T$ has an approximate
	asymptotic scaled and shifted chi-squared distribution. 
	
	\item Comparison of MANOVA and SSU-type of approach (not univarite, but incorporate correlation only through the test distribution): the SSU statistic will have a large variance under $H_0$ because of the correlation $\rho$ between two traits (intuitively, when $U_1$ is large by chance, $U_2$ will also be large because of correlation, thus $T$ will be even larger). 
	
	\item Importance of phenotypic covariance: n many practical problems: e.g. eQTL, the covariance among multiple traits are NOT due to biological pleiotropy, and removing this covariance (or source of trait variance) would be important to achieve higher power. 
	
	\item Remark: 
	\begin{itemize}
		\item In the model here, $\Sigma$ reflects the covariance of traits after removing the effects of the SNP; it may be different from the ``marginal'' covariance among traits. 
		\item To visualize: plot $Y_1$ and $Y_2$, and use different colors for values of $X$ (discrete). The value of $\beta$ is reflected through the difference of mean of $Y_1$ or $Y_2$, between different groups defined by $X$.  
	\end{itemize}
\end{itemize}

Questions/ideas about multi-trait analysis: 
\begin{itemize}
	\item Can we determine which test to use, univariate or multi-variate, before we do the test? One complication is that in the model $\rho$ is defined as correlation after correcting for $x$, but we would not know this until we test for the SNP. 
	
	\item Idea: use shared heritability of two traits (instead of phenotypic correlation) to set the prior of effect sizes. 
	
	\item How do we find eQTL hotspot? The vast majority of expression traits are not correlated (even though considerable covariance exist) and for each SNP, it is associated with only a small number of traits. This seems to suggest Scenario 1 is most applicable, thus we prefer univariate test. Can we do clustering analysis first, then for the cluster, apply the multivariate test? 
	
	\item Group-level test: suppose we are interested in testing if a SNP affects a set of traits, can we use a random effect model of $\beta_k$, and test the mean of $\beta_k$? Also, can we use the random effect model to improve the power of testing indivudial trait (borrow information from the group)? 
\end{itemize}

Moving toward system genetics through multiple trait analysis in genome-wide association studies [Shriner, Frontiers in Genetics, 2012]
\begin{itemize}
	\item Dimension reduction: PCA, and association on eigen-traits. 
	
	\item Multivariate methods: (1) MANOVA, Seemingly Unrelated Regression (similar to MANOVA, except that the predictors can be different/unrelated). The common assumption is that the errors are independent across individuals, but correlated for different traits of the same individual. (2) GEE. 
	
	\item Categorical and mixed outcome: 
	\begin{itemize}
		\item Bivariate logistic regression: four parameters $\pi_{ij}$ where $i,j=0,1$. Only three are independent: 2 marginal prob and the OR that relates the two variables. 
		
		\item Bivariate probit regression. 
		
		\item Mixed continuous and binary traits: one possible model: we use a bivariate normal distribution, one for continuous trait, the other as underlying hidden variable for binary trait. Allow correlation between the two traits. 
	\end{itemize}
	
	\item Biological pleiotropy: a small number of loci affect many traits, but most affect a small number of traits. 
\end{itemize}

Pleiotropy in complex traits: Challenges and strategies [Solovieff \& Smoller, NRG, 2013]
\begin{itemize}
	\item Examples of cross-phenotype (CP) associations: cancer, autoimmune diseases and psychiatric diseases.
	\begin{itemize}
		\item AID: 44\% of SNPs associated with one AID associated with another AID. 
		\item CACNA1C: CP effect on bipolar and SCZ. 
	\end{itemize}
	
	\item Biological pleiotropy: true effects on multiple phenotypes. Multiple scenarios: single causal variant, or different causal variants of the same gene. Interpretation of CP: could be distinct effects of the same allele in different cell populations; or truly mulitple consequences. 
	
	\item Mediated pleiotropy: causal variant affects $P_1$ which affects $P_2$. Ex. SNPs to LDL to MI risk. 
	
	\item Spurious pleiotropy: design artefact; or causal variants in different genes (but in LD). Design issues may be: 
	\begin{itemize}
		\item Ascertainment bias: e.g. patients with two diseases may be more likely to be recruited in the study. 
		\item Phenotype misclassification: e.g. SCZ and bipolar. 
		\item Shared controls: e.g. population stratification or batch effect, that affect shared controls but not cases. 
	\end{itemize}
	
	\item Establishing genetic overlap between traits: often the first step in the CP analysis. Approaches: polygenic scores, or LMM. 
	
	\item Multivarate approaches: require all phenotypes to be measured on the same individuals and often require individual level data. 
	\begin{itemize}
		\item Multivaraite regression (MANOVA) for continuous traits. Extension: GEE for to allow non-normally distributed phenotypes. 
		\item Multiple categorical traits: log-linear model, Bayesian network. 
		\item Ordinal regressino: genotype as outcome and phenotypes as predictors. 
		\item Dimension reduction: PCA or canonical correlation analysis, find the linear combination of traits that explain the covariation with genetic variants. 
	\end{itemize} 
	
	\item Univaraite approaches: 
	\begin{itemize}
		\item Step-wise analysis: use significant SNPs from one trait, then test its association with another trait (use smaller number of SNPs). Underpowered. 
		\item Fixed effect meta-analysis. Problem: same directions of SNPs. 
		\item Random-effect meta-analysis: allow heterogeneity, but still has problem with effects of opposite directions.
		\item Cross-phenotype meta-analysis (CPMA): explicit testing of CP: null hypothesis of no additional association, vs. alternative hypothesis of two or more associations. Also deal with opposite effects. 
		\item Subset based meta-analysis: exhausitvely evalaute all possible combinations of non-null models for association. 
		\item Testing multiple associations using minimum (univariate) $p$-values across multiple traits. TATES (Extended Simes). 
		
		\item Regional test: use the number of trait associations with $p$ value less than a threshold in a region.   
	\end{itemize}
	
	\item Example of CP analysis in practice: PGC study (Box 2). 
	\begin{itemize}
		\item Polygenic analysis: find significant overlap between SCZ, bipolar and MDD. Also ASD and SCZ, to a lesser extent.
		\item Univariate GWAS of five traits.  
		\item Fixed-effect meta-analysis: identify some CP effects (bar-plot). 
		\item Identify which phenotypes drive the association: use multivariate approach to search for subset of traits (model selection) - log-linear model approach. 
	\end{itemize}
	
	\item Characterizing CP: 
	\begin{itemize}
		\item Fine-mapping to distinguish biological pleiotropy and spurious pleiotropy. 
		\item Identifying mediated pleiotropy and MR. 
	\end{itemize}
	
	\item Applications of CP analysis: 
	\begin{itemize}
		\item Disease classification. 
		\item Drug repurposing: target common biological pathways in related disorders. 
		\item Drug side effects: e.g. anti-TNF therapy for Crohn and UC, but may increase the risk of multiple sclerosis. 
	\end{itemize}
\end{itemize}

Pervasive Sharing of Genetic Effects in Autoimmune Disease [Cotsapas and Daly, PLG, 2011]
\begin{itemize}
	\item Cross-Phenotype Meta-analysis (CPMA): the idea is to do something like fixed-effect analysis, but use $p$-values. Under $H_0$, $p$ is uniformly distributed, thus $-\ln(p)$ is exponentially distributed with rate 1. Under $H_1$, we assume $-\ln(p)$ follows exponential distribution with $\lambda > 1$. We assume $\lambda$ is the same for all diseases, thus allow us to do LRT: 
	\begin{equation}
	CPMA = -2 \ln \frac{P(D|\lambda=1)}{P(D|\lambda = \hat{\lambda})}
	\end{equation} 
	CPMA follows $\chi^1$ with df 1 under $H_0$. Comparing with Fisher's method of combining $p$-values, it does not pay the penalty of high df.
	
	\item Analysis of 7 AID data: start with 107 SNPs associated with at least one disease, a total of 44 SNPs have CPMA $p < 0.01$, while the expected number is only 1. 
	
	\item Remark: the key assumption is $\lambda$ is the same across multiple studies, or $p$-value distributions under $H_1$ are the same. This is not true, for example, when the power of the studies are different. 
\end{itemize}

A Unified Framework for Association Analysis with Multiple Related Phenotypes [Stephens, PLoS ONE, 2013; Michael Turchin talk]
\begin{itemize}	
	\item Model: suppose we test association of a SNP with $d$ traits, possibly related. The key idea is to distinguish direct and indirect associations. Let $\gamma$ be a parition of traits into three groups: direct association $Y_D$, indirect association $Y_I$, and unassociated $Y_U$. The relationship:
	\begin{equation}
	g \rightarrow Y_D \rightarrow Y_I, \quad Y_U \rightarrow Y_D, \quad Y_U \rightarrow Y_I
	\end{equation}
	Given $\gamma$, we can obtain the marginal likelihood (and BF) of $\gamma$. This allows us to test any specific hypothesis by model averaging: e.g. whether $g$ is associated with any trait. 
	
	\item Testing association: to assess if a SNP is associated with some trait, we compute the BF for a particular $\gamma$, where $H_0$ is the SNP is not associated with any trait: 
	\begin{equation}
	BF_{\gamma} = \frac{P_1(Y_D|Y_U, g)}{P_0(Y_D|Y_U)}
	\end{equation}
	It can be shown that the BFs for different $\gamma$ can be compared to assess which $\gamma$ is more likely. Also note that: for different SNPs, $\gamma$ may be different. 
	
	\item Model with summary statistics: the model needs covariance between phenotypes, how do we obtain that from summary statistics? Idea: consider estimated effect sizes of a SNP wrt. the two phenotypes: $\beta_1 \propto x^T y_1$ and  $\beta_2 \propto x^T y_2$. If $y_1$ and $y_2$ are correlated, then $\beta_1$ and $\beta_2$ of null SNPs are also correlated. In fact, one can show the relationship of two correlations. So the procedure is: estimation of how effect sizes under null are correlated (removing large effect SNPs). 
	
	\item Importance of distinguishing $Y_I$ and $Y_U$: when we have multiple traits, but SNP is only associated with some of them, and the traits are not highly correlated, doing multivariate analysis actually loses power. So association analysis should be done only on $Y_D$, while explaining $Y_I$ or incorporating $Y_U$ in association. 
	
	\item Relation to MANOVA: suppose we ignore $Y_I$ for now. Consider two traits, Weight $W$ and Height $H$. There are two alternative models $\gamma$: $g \rightarrow W, g \rightarrow H$ (both direct association), and $g \rightarrow W \leftarrow H$ (one indirect association). The first is MANOVA, and the second is testing association with $W$ while controlling $H$. The two tests are similar, when $g$ explains only a small fraction of variance in $W$. 
	
	\item Regression model: the term $P(Y_D|g)$ is given by Bayesian multivariate regression. Consider a case with a signle SNPLet $V$ be the covariance matrix of traits, the prior of $\beta$ (one for each trait) follows $N(0, cV)$, where $c$ is a constant. 
	
	\item Question: when $Y_D$ and $Y_I$ each has multiple traits, the relationship could get complicated, e.g. a trait in $Y_I$ depends on some trait in $Y_D$ but not other ones. 
	
	\item \textbf{Remark}: 
	\begin{itemize}
		\item The model analyzes each SNP independently, without enforcing certain relationship between phenotypes.
		
		\item The assumption of prior in Bayesian regression: the prior of effect size follows the same covariance as the traits. This may not be the best prior model. 
	\end{itemize}
\end{itemize}

Genome-wide Association Analysis for Multiple Continuous Secondary Phenotypes (SMAT) [Schifano \& Lin, AJHG, 2013]
\begin{itemize}
	\item Motivation: if we have multiple correlated secondary phenotypes in case-control, we can jointly analyze the data to increase the power (shared effects of one SNP on multiple traits) and also detect pleiotropic effects. To model the correlation among the traits, use GEE (similar to longtidual data analysis). 
	
	\item Basic model: let $y_{ij}$ be the $j$-th phenotype of the $i$-th subject, $x_i$ be the covariates, $s_i$ be the genotype, we have: 
	\begin{equation}
	\E(y_{ij}|x_i, s_i) = x_i \beta_j + s_i \alpha_j	
	\end{equation}
	where $\beta_j$ is the effects of covariates on the $j$-th phenotype, and $\alpha_j$ the effect of SNP on the $j$-th phenotype (assuming heterogeneous effects on the phenotypes). The correlation among $y_{ij}$'s can be modeled using GEE. 
	
	\item Scaled marginal model: if the traits are positively correlated and measure the same underlying trait, we can assume a one-DF model. Let $\sigma_j^2 = \Var(y_{ij}|x_i, s_i)$ be the phenotype-specific variance, then we assume that the SNP has a shared common effect on the means of the scaled phenotypes: 
	\begin{equation}
	\E(y_{ij}|x_i, s_i) /\sigma_j = x_i \beta_j + s_i \alpha		
	\end{equation}
	And we can test $H_0: \alpha = 0$. 
	
	\item Need to account for ascertainment issue (case-control sampling). 
	
	\item Remark: can we use a random effect model of SNP effects on the $M$ phenotypes, assuming that the effect of the same SNPs on different phenotypes tend to be similar? 
\end{itemize}

Identification of risk loci with shared eff ects on five major psychiatric disorders: a genome-wide analysis [PGC, Lancet, 2013]
\begin{itemize}
	\item Data: 33K cases and 27K controls of European ancestry, five disorders, ASD and ADHD for children, and BPD, MDD and SCZ for adult-onset diseases. 
	
	\item Fixed-effect meta-analysis: combine all five disorders to increase the power of detection. Use weighted $Z$-score approach, where the weight is given by the inverse of standard error (or square root of the sample size). Results: 4 independent regions reaching genomewide significance. 
	
	\item Omnibus test: meta-analysis allowing different effects in different disorders. Testing $H_0: \beta_1 = \cdots = \beta_5 = 0$ vs. $H_1: \exists i, \beta_i \neq 0$, using 5 df. LRT. The test statistic is effectively the sum of $\chi^2$ for all five disorders. Results: no significant loci. 
	
	\item Model selection: for SNPs reaching genomewide significance, and for SNPs previously identified for BPD and SCZ, we want to test if they have pleiotropic effects on other disorders, and what is the best-fit model (i.e. which subset of disorders are associated). We compare 13 different models (Table S1), and choose the one using BIC. A model describes a configuration of the risk of a SNP in all disorders, e.g. 
	\begin{equation}
	\text{ASD = SCZ} \neq 0, \text{ ADHD = BP = MDD = 0}
	\end{equation}
	In other words, suppose for a SNP, its effect in five diseases are $\beta_1, \cdots, \beta_5$, each model specifies some constraint on $\beta_j$'s. And the likelihood is given by $P(y_j|x_j, \beta_j)$, where $y_j$ and $x_j$ are phenotype and genotype data of $j$-th disorder. \\
	Results: forest plot, four significant SNPs all have pleiotropic effects. For other less significant SNPs or earlier identified BP, SCZ SNPs, a moderate fraction of SNPs have pleiotropic effects in at least one more disorder.  
	
	\item Polygenic risk score analysis across disorders: define one disorder as discovery set (e.g. SCZ), find all score alleles at a p-value threshold, then assess their contributions in the target set (e.g. ASD). Results: 
	\begin{itemize}
		\item Three adult-onset disorders have larger overlap, especially SCZ and BP. The overlap of ASD and SCZ and ASD vs. BP, are still significant, but reduced. 
		
		\item In all cases, $R^2$ explained variance from SNPs in another disorders are small, generally $<2\%$. 
	\end{itemize} 
	
	\item Limitations of the study: (1) Model selection: in some cases, the best fit and second-fit models are similar; (2) Diagostic misclassification may happen, though it cannot explain all the observed disease overlap. 
\end{itemize}

An atlas of genetic correlations across human diseases and traits [Bulik-Sullivan and Neale, NG, 2015]
\begin{itemize}
	\item Definitions: genetic covariance defined as $\sum_j \beta_j \gamma_j$ where $\beta_j, \gamma_j$ are effect sizes of SNP $j$ on the two traits. Genetic correlation: normalized, -1 to 1, it is asymptotically proportional to MR estimate. 
	
	\item MOM estimation: $E(z_{j1} z_{j2} | l_j)$ is a linear function of $l_j$ (LD score), and the intercept represents the overlap among samples from two traits.
	
	\item Derivation [personal notes]: we have $y_1 = X \beta + \delta$ and $y_2 = Z \gamma + \epsilon$, where $\beta, \gamma$ are effect sizes. Note that we use different genotype variables, since the samples are mostly independent. The summary statistics:
	\begin{equation}
	\begin{array}{lll}
	\hat{\beta}_j & = & (X_j^T X_j)^{-1} [X_j^T X \beta + X_j^T \delta]\\
	\hat{\gamma}_j & = & (Z_j^T Z_j)^{-1} [Z_j^T Z \gamma + Z_j^T \epsilon]
	\end{array}
	\label{eq:LDSC_estimated_effects}
	\end{equation}
	Now we consider the covariance of the two, using Law of Total Covariance. We consider the conditional distributions with given $\beta, \gamma$, then marginalize them.
	\begin{equation}
	\Cov(\hat{\beta}_j, \hat{\gamma}_j)	= \E_{\beta, \gamma}[\Cov(\hat{\beta}_j, \hat{\gamma}_j) | \beta, \gamma] + \Cov_{\beta, \gamma}(\E(\hat{\beta}_j | \beta_j), \E(\hat{\gamma}_j | \gamma_j))
	\end{equation}
	The first term captures covariance of $\hat{\beta}_j$ and $\hat{\gamma}_j$ due to sample overlap, and the second term captures the covariance due to genetic correlation of $\beta$ and $\gamma$. For the first term, we have: 
	\begin{equation}
	\E_{\beta, \gamma}[\Cov(\hat{\beta}_j, \hat{\gamma}_j) | \beta, \gamma] = \Cov(X_j^T \delta, Z_j^T \epsilon) = \Cov\left(\sum_i X_{ij} \delta_i, \sum_{i'} Z_{i'j} \epsilon_{i'} \right)
	\end{equation}
	For all non-overlapping samples, the term is 0. And for the second term, we have:
	\begin{equation}
	\Cov_{\beta, \gamma}(\E(\hat{\beta}_j | \beta_j), \E(\hat{\gamma}_j | \gamma_j)) = \Cov_{\beta, \gamma}(X_j^T X \beta, Z_j^T Z \gamma)
	\end{equation}
	Note: we could also just take Equation~\ref{eq:LDSC_estimated_effects}, and compute the covariance of the two, marginalizing $\beta, \gamma$, i.e. treating both $\beta,\gamma$ and $\delta, \epsilon$ as random. 
	
	\item Results in PGC: same individual-level data. Comparison with REML: LDSC and LDSC with constrained intercept. LDSC has significantly larger std error. LDSC with constrained intercept has somewhat larger standard error and the mean estimate sometimes differ quite a bit. The difference is larger for traits with small sample sizes.
\end{itemize}

Detection and interpretation of shared genetic influences on 40 human traits [Pickrell, NG, 2016]
\begin{itemize}
	\item Data: summary statistics of 43 GWAS, including some unpublished ones from 23andMe. First map all variants associated with each trait separately: predefined blocks, and one variant per block (fgwas). 
	
	\item Variants associated with pairs of traits: method to detect such variants is similar to COLOC. Estimate $\pi_j$'s (for each model) from all blocks. Note: $\pi_i$ here are sensitive to sample size, and represent detectable shared genetic influence. Observation that many variants are related to multiple traits, with no obvious relationship of effect sizes.  
	
	\item Overlap between pairs of traits: Use the proportion of variants that affect one trait, would also affect the other trait (asymmetric). 
	
	\item Causal relationship between traits: all variants ascertained in one trait $X$, and compare the effect sizes in two traits: $\hat{\beta}_{XX}$ and  $\hat{\beta}_{XY}$. Obtain the rank correlation $\rho_X$. Similarly, obtain the rank correlation using variants ascertained on trait $Y$, $\rho_Y$. Intuition is that if $X \rightarrow Y$, then $\rho_X$ should be large, but $\rho_Y$ is close to 0 ($Y$ can be affected by many factors other than $X$). Testing is simple: likelihood model of 2 rank correlations, under causal model, one of them should be 0; under non-causal model both are 0 or the two are equal.  
	
	\item Remark: 
	\begin{itemize}
		\item Estimating genetic overlap: the gain of power comes from a large prior for the model where a SNP is shared between traits. Comparing this with an alternative prior based on correlated effect sizes. 
		
		\item Causality inference: same difficulties are Sherlock. Ex. other possible non-causal model not included is that the two traits share some genetic influence. 
	\end{itemize}
\end{itemize}

Playing Musical Chairs in Big Data to Reveal Variables’ Associations [Ashard, NG review, 2016]
\begin{itemize}
	\item Why current multivaraite methods such as MANOVA are not successful? (1) Testing composite null: $\beta=0$ for all traits, not informative. (2) When a SNP is associated with only a small fraction of traits, multivariate test could lose power (Scenario 1 in the univariate vs. multivariate analysis: other traits are not informative, but we have to pay for high d.o.f). 
	
	\item Analysis: when we test one trait $X$ on $Y$, we decide if we adjust for covariates (other traits). When a covariate $C$ is uncorrelated with $X$, but explains some variance of $Y$, then adjusting for $C$ is advantageous (Figure 1A, B). However, when $C$ itself is affected by a SNP $X$, then adjusting for $C$ can lead to errors: 
	\begin{itemize}
		\item Suppose $X$ has an effect on $Y$: When $X \rightarrow C$, and $C \rightarrow Y$ (or both $C$ and $Y$ affected by some hidden $U$), then adjusting for $C$ would remove some effect of $X$ on $Y$. This loses power. Figure 1C. 
		
		\item Suppose $X$ has no effect on $Y$: since $C$ and $Y$ are correlated, let's say it is created by a confounder $U$: $Y \leftarrow U \rightarrow C$. But we have $X \rightarrow C$, this leads to the collider case: $U \rightarrow C \leftarrow X$. So adjusting $C$ will create dependency of $U$ and $X$, but $U \rightarrow Y$, then $X$ and $Y$ becomes correlated conditioned on $C$. 
	\end{itemize} 
	The general idea is thus: for any covariate $C$ associated with $Y$, if it is independent of SNP $X$, we adjust for $C$, otherwise not. 
	
	\item Notation: let $\beta$ be the effect of $X$ on $Y$. Let $\gamma$ be the correlation between $Y$ and $C$. We let $\delta$ be the association of $X$ and $C$. Our model can be written as: 
	\begin{equation}
	\E(Y) = X \beta + C \gamma \qquad \E(C) = X \delta
	\end{equation}
	
	\item Problem with $p$-value based filtering: we could test if $X$ has a non-zero effect on $C$. However, $p$-value from this test is used to reject the null (zero-effect), while our goal is to reject the alternative (zero-effect). Say, we reject $C$ at 95\% level, but there could be many covariates that we fail to reject because of limited power. Call these type 1 covariates (associated with $X$). There is an additional problem, however, caused by type 2 covariates (not associated with $X$). If there covariates happen to have large $\hat{\delta}$, we will not include these covariates (fail to adjust). 
	
	\item Claim: failure of adjusting for type 2 covariates using p-value filter would lead to inflated type 1 error when testing association between $X$ and $Y$. We consider the null model: $\beta = 0$ and $\delta = 0$ (type 2 covariates). First consider the case with only one covariate. Suppose $\hat{\delta}$ is large. Under $H_0$, we have $Y = C \gamma + \epsilon_Y$, but since $\hat{\delta}$ is large, $C$ and $X$ would appear associated. Then $Y$ and $X$ would appear associated. %This itself would not create inflation. Next we consider the case with many correlated covariates.  
	
	\item MC algorithm: we should then include $C$ if $\hat{\delta}_l$ is not rejected. We consider two intervals, in the unconditional test, we obtain inclusion area for $\hat{\delta}_l$ (95\% confidence interval). But we also need to consider the conditional test of $\hat{\delta}_l$: when we know $\hat{\beta}$ (marginal effect of $X$ on $Y$) and $\hat{\gamma}$ (correlation of $C$ and $Y$), we can know $\delta$ better. Intuitively, $\hat{\beta}$ and $\hat{\delta}$ are correlated because $Y$ can $C$ are correlated: the larger $\gamma$ is, the higher $\hat{\beta}$ and $\hat{\delta}$ are correlated. 
	So we consider the conditional distribution $\hat{\delta}_l | \hat{\beta}, \hat{\gamma}$, under null model $\beta = 0, \delta = 0$: 
	\begin{equation}
	\E(\hat{\delta} | \hat{\beta}, \hat{\gamma}, \beta = 0, \delta =0) = \hat{\gamma} \hat{\beta} 
	\end{equation}
	The two tests have two inclusion areas, and we take the union of the two as the inclusion area (OK to include/adjust $C_l$ if its $\hat{\delta}_l$ is in the area). 
	
	\item Results: work on a large number of phenotypes. 
	
	\item Comparison of MC algorithm and [Stephens, PLoS ONE, 2013]: the covariates $C$ that are included are similar to $Y_U$ in [Stephens13]. The covariates that are not included (due to effect of SNP on them) are treated as multiple $Y_D$'s, and multivariate analysis is used in Stephens13. 
	
	\item Q: Figure 3 unclear: is the unconditional inclusion area in (b,c) the same as the one in (a)? If so, the inclusion area is larger than the simple case. This does not address the problem of limited power? Also, 
	
	\item Remark: a general question is how do we estimate an effect when we have both direct measurements and indirect ones. Ex. suppose we want to estimate $X \rightarrow Y$, and we can estimate their association directly, say $\beta$. But we also have $Z$, which correlates with $Y$, and we have $X \rightarrow Z$. There could be multiple such $Z$'s. How do combine all the information to estimate $\beta$? 
	
	\item \textbf{Lesson}: in general, we should adjust for covariates associated with $Y$ to increase the power and avoid false association, e.g. adjusting for ancestry PCs or PEER factors in eQTL. However, when the covariates are genetic, we should not adjust for them. 
\end{itemize}

Genome-wide associations for birth weight and correlations with adult disease [Horikoshi and Freathy, Nature, 2016]
\begin{itemize}
	\item GWAS of birth weight: 150K samples, 60 significant loci. Explain 2\% of variation, chip-heritability 15\%.
	
	\item Genetic correlation of BW and other traits: positive with anthorpometric and obesity related traits; but negative with CAD.
	
	\item Pattern of genetic correlations (effect sizes) across traits (Figure 2): choose BW loci, and plot their effect sizes on a number of related traits. The effect sizes of SNPs show cluster patterns (generally sparse).
	
	\item Understand genetic correlation between BW and blood pressure: mostly negative correlation across loci, however, some level of heterogeneity. Specific locus: e.g. a SNP with large effect on both BW and blood pressure, it is cis-eQTL of CYP17A1.
	
	\item Lesson: overall genetic correlation hides the heterogeneous pattern of effect sizes.
\end{itemize}

Multi-trait analysis of genome-wide association summary statistics using MTAG [Turley and Benjamin, NG, 2018]
\begin{itemize}
	\item Model: for SNP $j$, its estimated effect (vector) $\hat{\beta}_j | \beta_j \sim N(\beta_j, \Sigma_j)$, where $\Sigma_j$ captures correlation of estimation error. And the prior $\beta_j \sim N(0, \Omega)$. 
	
	\item Inference: from Bayesian perspective, we can infer $p(\beta_j | \hat{\beta}_j, \Omega)$. MTAG uses Generalized Method of Moment (GMM) estimator: let estimator of $\beta_j$ be linear combination of $\hat{\beta}_j$, then solve $\E(\hat{\beta}_j) = \hat{\beta}_j$. 
	
	\item Estimation of $\Sigma_j$: use LDSC to estimate the intercept term for the diagonal element of $\Sigma_j$. Then use bivariate LDSC to construct the non-diagonal terms. 
	
	\item Estimation of $\Omega$: the marginal distribution $\hat{\beta}_j \sim N(0, \Sigma_j + \Omega)$. So $\Cov(\hat{\beta}_j) = \hat{\beta}_j \hat{\beta}_j^T = \Sigma_j + \Omega$. Then for all SNPs, estimate the average $\Omega$. 
	
	\item Analysis: potential problem, when SNPs are truly null for one trait, but non-null for others, the effect for the null trait would be biased away from 0. This may lead to high FDR when the GWAS for the two traits have very different power. 
	
	\item Simulation study: the bias from ignoring sampling variation in estimators of $\Omega$ and $\Sigma_j$ are small. 
	
	\item Results on three behavior traits: (1) Increase of power on individual traits: number of significant loci (after LD clumping) for each of the three traits (Figure 4). (2) Improving polygenic prediction using independent data: MTAG has better $R^2$ than GWAS (Figure 6A). 
	
	\item Remark: a major limitation is the homogeneous assumption of $\Omega$. The method is only applied to traits with high genetic correlation 0.7.  
\end{itemize}

Genome-wide association study results for educational attainment aid in identifying genetic heterogeneity of schizophrenia [Bansal and Koellinger, NC, 2018]
\begin{itemize}
	\item Background: Education attainment (EA) is negative correlated with SZ, but has positive genetic correlation. Why?
	
	\item Proxy phenotype analysis: ascertain on EA GWAS, 500 SNPs, about 130 associated with SCZ at $p < 0.05$, and 21 passing Bonf. threshold (highly enriched).
	
	\item Sign inconsistency of 21 loci: half, half. Also about half show colocalization (using fine-mapping).
	
	\item Prediction of clinical phenotypes: if using SZ PRS, not correlated. If group them by sign consistency with EA, significant correlation with SCZ severity.
	
	\item Lesson: use PRS defined on subset of related SNPs may better capture biologically relevant factors
\end{itemize}

Type 2 diabetes genetic loci informed by multi-trait associations point to disease mechanisms and subtypes: A soft clustering analysis [Udler, PLoS Med, 2018]
\begin{itemize}
	\item Background: previously, unsupervised/hier. clustering, using half of SNPs here. Subtyping T2D: in contrast to our clusters of genetic loci, these clusters are defined using clinical data and biomarkers at the time of diabetes diagnosis.
	
	\item Data: 94 T2D variants (with LD pruning) and 47 T2D related traits, including  glycemic traits from MAGIC, anthroprometric traits from GIANT (e.g. height, BMI), adipose tissues, birth weight and lipid levels, leptin, and fatty acid traits (e.g. omega 3). Also 10 clinical outcomes such as stroke/CAD and kidney disease.
	
	\item Bayesian NMF clustering: to deal with the non-negativity constraint, duplicate the columns (47 traits) to have both positive and negative. Then NMF on 94 x 94 matrix. Learn 5 clusters.
	
	\item Testing association of clusters with clinical outcomes: for each cluster, select SNPs based on weights in NMF using cutoff of 0.75. To test association: "genetic risk score (GRS) for each cluster with each GWAS trait or outcome (“GWAS GRS”) was performed using inverse-variance weighted fixed effects meta-analysis using summary statistics from GWAS”.
	
	\item Clustering results: 5 dominant factors/clusters. No. selected SNPs per cluster: 5-30.
	
	\item Association of cluster GRSs vs. clinical outcomes: beta-cell cluster associated with stroke, lipid related cluster associated with blood pressure.
	
	\item Clusters are distinctly enriched for tissue enhancers or promoters (Figure 2): e.g. cluster 1 (beta-cell) are enriched in enhancers of islet and liver.
	
	\item Application of clusters to patients with T2D: T2D patients with many phenotypes. First, association of cluster GRS with traits, e.g. beta-cell cluster with BMI and CRP.  Ex.  those with extreme GRS in the Beta Cell cluster (N = 1,068) had decreased BMI, HC, and WC comparing with all patients.
	
	\item Analysis: why NMF is a poor tool for GWAS factor analysis? (1) Non-negativity constraint: not applicable to GWAS. (2) Sparsity: important: variant effects on factors should be sparse. (3) Using Z-scores: not scaled properly. The linear relationship should be defined in terms of effect sizes, just as in MR. Z-scores of variants depend on allele frequencies, which vary across variants.
\end{itemize}

Pleiotropic mapping and annotation selection in genome-wide association studies with penalized Gaussian mixture models (iMAP) [Zeng and Xiang Zhou, Bioinfo, 2018]
\begin{itemize}
	\item Model: let $\beta_j$ be the effect of SNP $j$ on two traits (vector). It follows mixture of bivariate normal distribution with prior weights $\pi_{00}, \pi_{01}, \pi_{10}, \pi_{11}$ with covariance $\Sigma$. Fit the parameters, including $\beta$, $\gamma$ and $\Sigma$ by EM. 
	
	\item Incorporating annotations: prior of mixing weights $\pi_{jk}$ for SNP $j$, pattern $k$ (4 patterns, 00, 01, 10 or 11), depends on $X_j$ (annotations) by a multinomial logistic regression with linear term $X_j b_k$ for annotation $k$. Use penalized regression  $L_1$ norm on $b_k$'s. The objective function is the sum of $L(b)$ the log-likelihood, defined in terms of $\E(\gamma_{jk})$ (response variable in mlogic regression), and penalty of $b_k$'s. Use general optimization technique.  
\end{itemize}


Discovery of shared genomic loci using the conditional false discovery rate approach (condFDR) [Hum Genet, 2019]
\begin{itemize}
	\item Motivation: given two phenotypes, if they show significant polygenic overlap, how do we leverage that to find more associations?
	
	\item Step 1: conditional QQ plot. QQ plot of primary phenotypes, conditioned on p-values of SNPs in the second phenotype.
	
	\item Step 2 (Box 1): Computing FDR for each strata of SNPs, based on p-values in the second phenotype. Compute the local FDR for each SNP at each bin, based on the distribution of p-values in that bin.
\end{itemize}

Genomic SEM Provides Insights into the Multivariate Genetic Architecture of Complex Traits [Grotzinger, Nature Hum Behavior, 2019]
\begin{itemize}
	\item Model: $y = \Lambda \eta + \epsilon$, where $y$ is $k \times 1$ effect vector on $k$ phenotypes, and $\Lambda$ is the loading matrix, and $\eta$ is $m \times 1$ vector (SNP to factor effect).
	
	\item Fitting SEM: obtain the covariance matrix $\Sigma(\theta)$ of SNP to phenotype effects,  where $\theta$ are parameters (loading matrix) and equate to the observed covariance matrix $S$. Estimation of $S$ is based on LDSC, allowing for sample overlap (residual correlation).
	
	\item Simulation: (1) Use only summary statistics - details not clear. (2) Use GCTA to simulate genotype and phenotype data: 10K causal variants.
	
	\item Learning SNP effects: once factor model is learned, plug in individual SNP (one at a time), and learn their effects on factors.
	
	\item Confirmatory factor analysis (CFA) on 5 psychiatric traits: a single general latent factor, called $p$-factor.
	
	\item Exploratory factor analysis (EFA): use 2 or 3 factors, and do model comparison.
	
	\item Multivariate GWAS: 128 loci of the $p$-factor in the case of 5 psychiatric traits, 27 new loci.
	
	\item Polygenic scores of factors: use the general latent factor in 5-phenotype case, show it predicts better the symptoms.
	
	\item Analysis: how this compares with FLASH? The model still assumes homogeneity of SNP effects - same distribution on factors across all SNPs. Imagine we have two factors, A has large effects on group 1 traits, and B large effect on group 2 traits. But a SNP can only affect factor A or B, but not both. For a SNP affecting A: it will have large effect on group 1, but no effect on group 2. However, with polygenic model, the SNP is expected to affect B, so expected to have some effects on group 2 as well, so in order to explain the observed lack of effect of the SNP on group 2, the model will have to fit with a smaller effect of factor B on group 2 traits. In simpler words: a SNP acts on only A should not be used when we estimate the effects of factor B on other traits; but by including them and assume non-zero effects, we may push down the estimates of factor B on other traits.
	
	\item Analysis: alternatively, in truth, we expect 2 distinct clusters of SNPs, some have large effects on group 1 only, and the other cluster large effect on group 2 only. But the model expects a homogenous group of SNPs.
\end{itemize}

Characterisation of the genetic architecture of immune mediated disease through informed dimension reduction [Burren and Wallace, biorxiv, 2020]
\begin{itemize}
	\item Motivation: (1 )the SNP effect estimates must be on the same scale. (2) Variable correlation between input dimensions (SNPs) due to LD; (3) All SNPs are expected to show small deviations between studies due to random noise, different genotyping platforms and data processing decisions. 
	
	\item Background: DeGA method, LD-thinning and p-value cutoff $p < 0.001$. However, this will make the results dominated by larger GWAS. 
	
	\item Model: use PCA. To deal with challenge with (1), use $\hat{\beta}$ divided by $\sigma_{MAF}$, the variance of estimation from MAF, but not sample size. To deal with the challenges (2) and (3), favor SNPs that are likely causal variants. Do fine-mapping separately on all traits, and compute the weighted average of PIPs, $w$. For a SNP, its input to PCA is: $\hat{\gamma} = w \hat{\beta} / \sigma_{MAF}$.  
	
	\item Selection of driver SNPs for PCs: most elements are close to 0, so use hard thresholding.
	
	\item Projection of new GWAS dataset to PC space: learn the contribution of PCs to the new trait.
	
	\item Importance of weighting SNPs (Figure 2): if do not use weighting, in PCA, traits are ordered by their data source (UKBB would be clustered), rather than trait types.
	
	\item Application to immune traits: 14 traits, PC1 is about autoimmune axis.
	
	\item Application to a large set of traits in UKBB: group the traits by their projections (Figure 3). Some clear patterns: IBDs form a cluster, some cancers form a cluster. Also use correlation of PCs with the phenotypes to interpret PCs.
	
	\item Remark: the SNP weighting by PIP is not a good strategy. In high LD regions, PIPs would be small. So low PIP would reflect LD, rather than true effects. 
\end{itemize}
%%%%%%%%%%%%%%%%%%%%%%%%%%%%%%%%%%%%%%%%%%%%%%%%%%%%%%%%%%%%
\section{Mendelian Randomization}

Questions about MR [personal notes]
\begin{itemize}
	\item Suppose we have a valid IV of $X$ to $Y$, and we know a confounder $Z$, should we adjust for $Z$ in the MR analysis? Conceptually, MR addresses exactly the problem of confounding. 
	
	\item Power of MR: how the power depends on the strength of IV (measured by PVE)? 
\end{itemize}

Problems of MR: summary data [personal notes]
\begin{itemize}
	\item Adjusting for confounders: if we know the confounder $U$, and we adjust $U$ when estimating $\hat{\beta}_{X|G}$ and $\hat{\beta}_{Y|G}$, then the ratio estimate is unbiased.
	
	\item Confounding assumption: we cannot directly test if $G \rightarrow U$. 
	
	\item 2-sample MR: is weak IV still a problem? 
	
	\item 2-sample MR: if we use data of exposure to select the strong IV(s), then do MR, we can suffer from selection bias.  
\end{itemize}

Analysis of MR and its issues: 
\begin{itemize}
	\item The effect of not including confounding variables in MR. Our true model is: 
	\begin{equation}
	X = \alpha G + \gamma_X U + \epsilon_X \qquad Y = \beta X + \gamma_Y U + \epsilon_Y
	\end{equation}
	where $\epsilon_X$ and $\epsilon_Y$ are independent. We plug in the first equation into the second: 
	\begin{equation}
	Y = \beta \alpha G + (\beta \gamma_X + \gamma_Y) U + (\beta \epsilon_X + \epsilon_Y)
	\end{equation}
	Note that $G$ and $U$ are assumed to be independent, however, in finite sample, they can be correlated. This leads to biased estimate when regressing $Y$ vs. $\hat{x} = \alpha G$ (see weak IV bias in the MR book). Note that we assume $\hat{\beta}_X = \alpha$, which does not account for finite-sample. 
	
	\item Reverse causation: MR does not address this and it is possible that the results are due to reverse causation. This violates MR assumption ($G$ affects $Y$ without going through $X$), but it may not be detectable. One can show that the 2SLS or ratio estimate leads to estimated effect of $1/ \beta$, where $\beta$ is the effect of $Y$ on $X$. Let the true model be $G \rightarrow Y \rightarrow X$,
	\begin{equation}
	Y = \alpha G + \epsilon_Y \qquad X = \beta Y + \epsilon_X
	\end{equation}
	We have $\hat{\beta}_{X|G} = \beta \alpha, \hat{\beta}_{Y|G} = \alpha$, so the ratio is $1/\beta$.  
	
	\item Pleiotropy without causal effects (independent of confounding): suppose our model is $X \leftarrow G \rightarrow Y$, without relationship between $X$ and $Y$. Then MR will give non-zero estimate. 
\end{itemize}

Weak IV bias in MR [personal notes]
\begin{itemize}
	\item Causal model: let $\alpha_j$ be the effect of SNP $j$ on $X$, and $\beta$ the causal effect of $X$ on $Y$. Let $\alpha_U, \beta_U$ be the effect of $U$ on $X$ and $Y$. 
	
	\item Weak IV bias under 2SLS model: assuming there is a single IV. Let $\Delta G$ be difference of genotypes, $\Delta X$ and $\Delta Y$ be the difference of exposures and outcomes. We first show that in the first regression model, the estimate of $\alpha_1$ ($G$ to $X$ effect) is biased:
	\begin{equation}
	\hat{\alpha}_1 = \frac{\Delta X}{\Delta G} = \frac{\alpha_1 \Delta G + \alpha_U \delta U}{\Delta G} = \alpha_1 + \frac{\alpha_U \Delta U}{ \Delta G}
	\end{equation}
	So when fitting the first regression model, we introduce a bias to $\alpha_1$. Let $\tilde{X} = \hat{\alpha}_1 G$ be the predicted $X$. Next, we consider the second regression, the estimated causal effect is:
	\begin{equation}
	\hat{\beta} = \frac{\Delta Y}{\Delta \tilde{X}} = \frac{\Delta Y}{\hat{\alpha}_1 \Delta G} = \frac{\beta \Delta X + \beta_U \Delta U}{\Delta X} = \beta + \frac{\beta_U \Delta U}{\alpha_1 \Delta G + \alpha_U \Delta U}
	\end{equation}
	So the situation is similar to the analysis using ratio estimator. When $\alpha_1$ is small, this leads to a bias close to $\beta_U / \alpha_U$ (non-zero). 
	
	\item Weak IV bias under 2SLS model with multiple IVs: likely the same problem: when estimating $\hat{\beta}$, both $\Delta Y$ and $\Delta \tilde{X}$ have the term $\Delta U$. 
	
	\item Can we correct weak IV bias using cross-validated prediction? Probably not. As shown above, the problem is introduced by having an estimated $\alpha_1$ that incorporates $\Delta U$. Leave-one-out prediction model would change little of $\hat{\alpha}_1$. 
\end{itemize}

Mendelian Randomization: New Applications in the Coming Age of Hypothesis-Free Causality. [Evans \& Smith, Annual Review Of Genomics And Human Genetics, 2015]
\begin{itemize}
	\item Concept of MR: the genotype (IV) randomizes subjects, i.e. all possible covariates are balanced out. Think of this as: subjects randomly assigned to genotype-defined groups, then assess outcome. 
	
	\item \textbf{Assumptions of MR}: (Figure 1) let $Z$ be genetic instrument, $X$ exposure and $Y$ outcome. The causal diagram: 
	\begin{equation}
	Z \rightarrow X \rightarrow Y \qquad X \leftarrow U \rightarrow Y
	\end{equation}
	where $U$ is a confounder. The three assumptions are: (1) $Z$ must be associated with $X$, (2) $Z$ not associated with $U$ (lack of edge between $Z$ and $U$), (3) $Z$ not associated with $Y$ except through $X$ (lack of edge from $Z$ to $Y$ directly). 
	
	\item Difficulty of MR: no pleiotropic effect of IV. Ex. robust association of genetic risk scores of CRP and cancer, however, pleiotropy cannot be ruled out. 
	
	\item Two-sample MR: the estimated effect is $\hat{\beta}_{GY}$ divided by $\hat{\beta}_{GY}$, where the two estimated effects may come from two studies. 
	
	\item Two-step MR (Figure 3) and mediation: motivation is to study if the effect of an exposure on outcome is mediated by methylation. Two step MR involves the use of IV for both exposure and intermediate variable.   
	
	\item Studying trait pairs with MR: start with finding traits with genetic correlations. The simple approach would use genetic risk scores as IVs, but it almost always violates the MR assumption. 
	
	\item Network MR: multiple SNPs and risk factors, learn the relation of variables, and use IVs to estimate causal effects. 
\end{itemize}

Fulfilling the promise of Mendelian randomization [Pickrell, biorxiv, 2015]
\begin{itemize}	
	\item Skepticism of MR: Arguably no new causal relationship has been identified with this approach and subsequently verified in a RCT. 
	
	\item Skepticism of MR: assumption that the genetic variants have a direct effect on one trait (the “causal” trait), but only an indirect effect on the other (the “caused” trait). This assumption is hard to validate a priori, and we have many examples of pleiotropy. 
	
	\item Simulations to show that using multiple loci may lead to false positives. Model: $G \rightarrow U$, where $U$ is a confounder, and $U \rightarrow X, U \rightarrow Y$. Even if only 5-20\% of all loci of $X$ influences $U$, MR will often find $X \rightarrow Y$. 
\end{itemize}

Problems of MR: individual level data [personal notes]
\begin{itemize}
	\item Testing MR assumption: (1) if $U$ is observed, we can test if an IV is valid, by testing if it's associated with $U$. However, we need to take power into account: even if the result is negative (no association), it does not prove that the IV is valid. (2) We may not know if $U \rightarrow X$ or $X \rightarrow U$, if it's the latter, then this does not violate the assumption of MR ($U$ lies the path from $X$ to $Y$).  
	
	\item Adjusting for confounders: when $U$ is observed, we can adjust for $U$ in regression models. When $U$ is unobserved, we can model $X_i$ and $Y_i$, and allow the error terms to be correlated. 
	
	\item Residual (pleiotropic) effect of $G \rightarrow Y$. Suppose we have a single IV. It is easy to show that a model with $G \rightarrow Y$ effect is not identifiable. 
	
	\item Weak IV bias. Q: can we remove weak IV bias, by modeling the correlated errors (hence confounders) between exposure and outcome?  
	
	\item Multiple correlated IVs (LD).  
\end{itemize}

Core concepts and limitations of Mendelian randomization [de Leeuw and Posthuma, review for NRG, 2020]
\begin{itemize}
	\item Basics of MR: for variant $j$, let $\gamma_{Xj}$ and $\gamma_{Yj}$ be the estimated effect of $j$ on $X$ and on $Y$. (1) NOME assumption: IV effect on $X$ is given and no measurement error. (2) Estimation: IVW, where weight of the estimated causal effect $\beta_j$ is proportional to $1/s_j^2$, where $s_j$ is the s.e. of $\gamma_{Yj}$. 2SLS: very similar to using PRS as IVs.  
	
	\item Violations of MR: 
	\begin{itemize}
		\item Horizontal pleiotropy (InSIDE) Figure 1E.
		\item A shared factor: Figure 1F, and Figure 2C. Note: Figure 2C is a special case of Figure 1F b/c $X$ would almost always have other variants. Also Figure 1G and Figure 2D. 
		\item Population structure (Figure 1H).
		\item LD (Figure 1I)
		\item Reverse causality: Figure 2A and 2B. Note that with reverse causality model, we expect that $X$ should have some other variants that do not act through $Y$. Overall, we would still have heterogeneity of causal estimates. 
	\end{itemize}
	
	\item Instances of exposures and outcomes are imperfect proxies for those directly involved in the causal effect (Box 3, Figure 2e-i). This could be due to measurement error, canalization (outcome pushed back to equilibrium), different tissue context of gene expression. 
	
	\item Collider bias: two important case. Figure 2I: $G$ and $D$ (confounder) acts on $R$, conditioning on $R$ induces correlation of $G$ and $Y$ even in the absence of $X$ to $Y$ effect. Figure 2J: $X$ affects $R$, then conditioning on $R$ removes causal effects. 
	
	\item Testing implied constraints to validate MR assumption:
	\begin{itemize}
		\item Reverse causality: if effect sizes are normalized, and either direction is true, then we would expect that the causal effect is within -1 to 1. To see this:
		\begin{equation}
		Y = X \beta + \epsilon
		\end{equation}
		Then $\beta^2 \cdot \Var{X} \leq \Var{Y}$, so $\beta^2 \leq 1$. This allows one to determine direction. Methods for doing this: Steiger test, BayesMR. 
			
		\item Testing potential confounders: difficult in practice. If $D$ (confounder) is observed, then we can adjust $D$ when estimating $\gamma_{Xj}$ and $\gamma_{Yj}$. A variant that acts on $X$ and $Y$ through $D$ will have $\gamma_{Xj} = \gamma_{Yj} = 0$, so they will not be used as IVs. 
		
		\item Negative control population: exposure is constrained to a single value, so exposure cannot mediate genetic effect of a variant on the outcome. So if $G_j$ is associated to $Y$, it must act independently of $X$ (so invalid IV). Ex. to test alcohol consumption vs. mortality, use a population who do not drink alcohol. There are methods designed to use negative control population: Pleiotropy-robust MR (PRMR) method. 
	\end{itemize}
	
	\item Methods: testing heterogeneity, let $\hat{\beta}_j$ be estimated causal effect of variant $j$, test if they are all equal - Q statistic. MR-PRESSO. GLIDE: very similar. Then pruning: individual deviation for each variant can be used to determine if it should be pruned. 
	
	\item Valid subset methods: assuming the majority of variants are instrument. Weighted median method. ZEMPA (zero modal pleiotropy assumption): the largest homogenenous subset of variants are valid instruments. Modal MR method: estimate the mode of a smoothed distribution of $\hat{\beta}_j$. 
	
	\item Methods: modeling deviations. MR-Eggar, BayesMR. CAUSE. 
	
	\item MR in practice: 76\% studies evaluate heterogeneity to some extents, but few did this in testing, and only one study used a robust estimation method. 
\end{itemize}

\subsection{Mendelian Randomization: Methods for using Genetic Variants in Causal Estimation [Burgess \& Thompson]}

Instrumental Variable (IV) approach: [Chapter 1,2]
\begin{itemize}	
	\item IV approach: suppose we want to study the causal effect of exposure $X$ on the outcome $Y$ from observational data. Consider a variable that satisfies these conditions: 
	\begin{itemize}
		\item It is associated with $X$;
		\item it is not associated with any confounder;
		\item it does not affect the outcome, except possibly via its effect on exposure. 
	\end{itemize}
	Then it is an IV. Suppose we divide the data into groups by IVs, then we can compare the group with high $X$ vs. the group with low $X$: if the outcomes are different, we show that $X$ causally influence $Y$. The intuition is that \textit{if the conditions are satisfied, then the groups defined by IV are random (wrt to $Y$), except the difference of $X$. So any difference in the outcome must be due to $X$}. 
	
	\item Examples of IV: geographical locations are often used as IV. Ex. two adjacent states that are very similar in all aspects, and one of them implements a policy and we want to know if the policy affects outcome. Another example of IV: policy changes. 
	
	\item Analogy to Randomized Clinical Trial (RCT): IV essentially generalizes RCT. In RCT, each subject is assigned randomly into one of two groups, treatment or control. The key of RCT is that the assignment of groups (exposure) is independent of the outcome. We can define $G$, the group where a subject belongs to, as an IV. Then RCT can be viewed as an IV approach: which group (genotype) an individual belongs to (receive) is random (wrt phenotype), determined by Mendelian segregation. Difference: in RCT, the effect of $X$ on $Y$ can be directly obtained, while in IV approach, it needs to be inferred, as the IV might have a small effect on the exposure (as in the case of genetic variants), which does not reflect the effect of exposure on outcome. 
	
	\item Confounding and endogeneity: when there is a variable $Z$ that is associated with $X$, and also may have an effect on $Y$, then $Z$ is a confounder. If confouner is not accounted for, this leads to endogeneity: the regressor $X$ will be correlated with the error term in regression (due to $Z$), and this leads to a biased estimate the influence of $X$ on $Y$. 
\end{itemize}

Mendelian randomization: basic concepts [Chapter 1,2]
\begin{itemize}
	\item Genetic variants as IVs: most genetic variants are found to distribute randomly in the population. In other words, if we divide the population by alleles, then the groups do not different substantially in a way that could impact the outcome of interest (Note: of course, for causal alleles, the groups would differ - but the assignment of alleles to individuals are random). 
	
	\item Violation of the assumptions: the conditions for genetic variants as IV would include random mating and lack of selection. This could be violated, e.g. if a variant is under selection, then the groups $G$ and $g$ will differ in some substantial way, e.g. $g$ has lower fitness, then this violates IV assumptions (the variant could affect a trait through fitness). Another example, variant $g$ is lethal in males (X-chromosome), but not in females, then the variant $G/g$ will differ in gender composition. 
	\begin{itemize}
		\item Remark: if IV of $X$, say $G$, have broad impact (not through affecting $X$), then it is not a valid IV of $X$. On the other hand, it is OK if $G$ is pleiotropic per se, as long as all these effects are mediated via $X$. 
	\end{itemize}
	
	\item Examples: 
	\begin{itemize}
		\item Study if CRP level has a causal influence on the risk of coronary heart disease (CHD): using cis-SNPs of CRP as IVs. 
		\item Study if alcohol influences the risk of cancer. Alcohol consumption is often correlated with smoking, so difficult to study. Use Mendelian randomization, the variant in the gene ALDH2 (alcohol metabolism). 
	\end{itemize}
	
\end{itemize}

Different views of IV approach [Chapter 3]
\begin{itemize}
	\item Causal inference from manipulation: ``no causation without manipulation''. To distinguish conditional proability $Y | X = x$ with the statement that $X$ has a causal effect on $Y$, use $Y | do(X=x)$, where $do()$ operators means manipulate the value of $X$. 
	
	\item Counterfactual view: $X$ has an effect on $Y$, means that suppose $X$ takes a different value (from the observed one), will $Y$ be different? The challenge is that this conterfacutal (alternative universe where $X$ is different) is not observable - the ``fundamental problem of causal inference''. The strategy is to use \textit{``exchangable'' data that mimics the alternative universe}. For RCT, the treatment and control group are exchangable, so we can say, even though the control group does not receive treatment, we can use the information from the treatment group to see the alternative universe where the control group does receive treatment. Similar logic applies for IV. 
	
	\item Probabilistic graphic model (DAG): causal model can be represented by DAG, where each edge has a causal interpretation, and no edge means conditional independence. The basic IV approach can be represnted as: 
	\begin{equation}
	G \rightarrow X \rightarrow Y, \qquad X \leftarrow C \rightarrow Y
	\end{equation}
	where $C$ is confounder. IV must satisfy the conditions: not associated with $C$, and not affect $Y$ (except through $X$). $G$ is not a valid IV if $G$ affects $Y$ through some intermediate variable, or there exists $D$ s.t. $D \rightarrow G$ and $D \rightarrow Y$. $G$ is still valid if there exists a collider $E$. In general, we summarize the condition of IV as: $d$-separation between $G$ and $Y$, or intuitively: \textbf{if $G$ is associated with some variable $D$, and $D \rightarrow Y$ (not through $X$), then $G$ is not a valid IV} (because the groups defined by $G$ may differ in $D$, which could influence $Y$). 
	
\end{itemize}

Possible violations of IV in Mendelian randomization [Chapter 3]: basically the existence of $D$ s.t. $D$ associates with $G$, and $D \rightarrow Y$ independently of $X$. 
\begin{itemize}
	\item Biological mechanisms: pleiotropy. Ex. to test BMI $\rightarrow$ T2D, we can use SNP associated with BMI (e.g. FTO) as IV. However, if the FTO variant has broad effects, e.g. blood pressure, then this is invalid. 
	
	\item Canalization: genetic variation may trigger compensatory changes. 
	
	\item Non-genetic inheritance: LD (e.g. variants associated with the causal variants may affect other variables), effect modification (the effect of $X$ on $Y$ depends on some covariate, which is not accounted). 
	
	\item Study design problems: the study design might introduce variables $D$ that violate IV. Examples: population stratitification (ancestry is $D$) and ascertainment. 
\end{itemize}

Validation of IVs in in Mendelian randomization [Chapter 3]
\begin{itemize}
	\item Statistical validations: check the three conditions. Ex. test if an IV is associated with known confounders. The problem: IV may associate with a covariate that lies in the causal path from $X$ to $Y$ (mediator), and in this case, association with the covariate is OK. 
	
	\item Biological considerations: Bradford Hill criteria (Table 3.1), some important ones: 
	\begin{itemize}
		\item Consistency: using multiple genetic variants (IVs) lead to the same estimate of causal effects. Or use non-genetic variables, e.g. a drug that manipulates the exposure. 
		
		\item Specificity: the IV should not change many different things. 
		
		\item Gradient: the genetic effect on the exposure and the genetic effect on the outcome are proportional. 
	\end{itemize}
	
	\item Example: suppose we are studying LDL $\rightarrow$ heart disease risk and we use a SNP associated with LDL as IV. To make sure it is a valid IV, we check if SNP is associated with known risk factors of heart disease, e.g. obesity. If it is the case, and the effect is not mediated via LDL, then the SNP is invalid IV. On the other hand, if the SNP is associated with a risk factor downstream of LDL, e.g. HDL, then it is OK. 
\end{itemize}

Estimating causal effect: overview [Chapters 3 and 4]
\begin{itemize}
	\item Testing causal effect: when $G$ is a valid IV of $X$, then to test $X$ on $Y$, we simply test if $Y$ is associated with $G$. 
	
	\item Assumptions of causal effect estimation: monotonicity, the effect of $X$ on $Y$ is monotonic.  
	
	\item An example of analyzing causal effect estimation problem: exposure is smoking, outcome cancer rate. The unobserved confounder is diet: poor diet may correlate with smoking, and increase the rate of cancer. 
\end{itemize}

Ratio of coefficient method for estimating causal effect [Chapter 4]
\begin{itemize}
	\item Idea: the strength of causal effect can be written as: 
	\begin{equation}
	\frac{\Delta Y}{\Delta X} = \frac{\Delta Y/\Delta G}{\Delta X/\Delta G}
	\end{equation}
	
	\item Binary $G$: the coefficients are then simply $\bar{Y}_1 - \bar{Y}_0$ and $\bar{X}_1 - \bar{X}_0$, and the ratio of the two (slope in the plot of $Y \sim X$) is the causal effect. 
	
	\item Intuition why this works: in the smoking example, we cannot simply correlate cancer rate and smoking, because of the confounder. Now suppose we create groups by genotype $G$, then compare $\bar{Y}_1$ and $\bar{Y}_2$, then with each group, the mean of $Y$ averages out the confounder (which on average is no different in the two groups). 
	
	\item Continuous $G$: it is the ratio $\hat{\beta}_{Y|G} / \hat{\beta}_{X|G}$. Note that this ratio is derived from the estimated effect size: thus it alreadys controls for covariates, can use summary statistics (without individaul level data), and can be estimated from two different datasets. 
	
	\item Confidence interval: could be derived from the ratio of normal RVs. Or use bootstrapping. 
	
	\item Problem of the ratio of coefficient method: the mean of the ratio is not finite. It reflects the fact that the denominator has a certain probability of being 0. 
	
	\item Issue with case-control studies: when the disease is rare, case enrichment will induce correlation between IV and confounders, thus estimation of the causal effect is biased.  
	
	\item Power of the method: suppose IV explains $2\%$ of the exposure, and we need 100 samples to study the correlation between $X$ and $Y$, then we need 100/2\% = 20,000 samples for the IV approach. 
\end{itemize}

Two-stage and likelihood methods [Chapter 4]
\begin{itemize}
	\item Intuition: generalization of the ratio method when $G$ is binary. When $G$ has multiple groups, intuitively, we should obtain $\bar{X}_k$ and $\bar{Y}_k$ for the $k$-th genotype, and regress the two. In the general case, we have one group per data point, so we will need to regression $y_i$ against $\hat{x}_i$, where $\hat{x}_i$ is the mean of exposure under the given $G_i$ (genotype). So this leads to the following method. 
	
	\item Two-stage least square (2SLS) method: it can be used for multiple IVs. First, we regress $x_i$ on $G_i$: $x_i = G_i \alpha + \epsilon_{X_i}$, then we have the predicted value $\hat{x}_i$, then we regress $y_i$ on $\hat{x}_i$:
	\begin{equation}
	y_i = \beta_0 + \beta_1 \hat{x}_i + \epsilon_{Y_i}
	\end{equation}
	where $\epsilon_{X_i}$ and $\epsilon_{Y_i}$ are indepenent error terms. Justification: suppose $y_i = \beta_0 + \beta_1 x_i + \epsilon_{Y_i}$, then we plug in $X$, we have: $y_i = \beta_0 + \beta_1 (G_i \alpha) + \beta_1 \epsilon_{X_i} + \epsilon_{Y_i}$, so the regression coefficient of $Y$ against $G \alpha$ is still $\beta_1$. 
	
	\item Problems of two-stage methods: (1) this model does not account for confounders. (2) The uncertainty of the first-stage regression has to be taken into acccount. While this does not affect the mean estimate, the standard error term needs to correct for this uncertainty. 
	
	\item Full information maximum likelihood model (FIML): we simply have two regression models, 
	\begin{equation}
	x_i = G_i \alpha + \epsilon_{X_i} \quad y_i = \beta_0 + \beta_1 x_i + \epsilon_{Y_i}
	\end{equation}
	The difference with the previous model is that $\epsilon_X$ and $\epsilon_Y$ are not independent, and the model will need to estimate the covariance matrix. The intuition that the full likelihood model works is that: the confounding variable is adjusted through the correlated error term. 
	
	\item Limited information maximum likelihood (LIML): find $\beta_1$ that minimize RSS of $(y_i - \beta_1 x_i)$ vs. $G_i$. The intuition is that under the MR causal model, $Y \independent G | X$, thus if we adjust $X$ in $Y$, the residual and $G_i$ should be uncorrelated. 
	
	\item Bayesian method: the model is similar to FIML, except that $y_i$ depends on the expected value of $x_i$ not $x_i$: 
	\begin{equation}
	x_i = G_i \alpha + \epsilon_{X_i} \quad y_i = \beta_0 + \beta_1 G_i \alpha + \epsilon_{Y_i}
	\end{equation}
	The joint distribution of $X_i, Y_i$ is bivariate normal, with the error term correlated. Inference can be done via MCMC (possible in WinBUGS). 
	
\end{itemize}

Statistical issues for IV analysis [Chapter 4]
\begin{itemize}
	\item Covariates: if a covariate is not correlated with the IV and not in the causal path between exposure and outcome, then we should include covariates. This is called \textit{exogenous regressor} in ecometrics. This will improve the precision of the estimate - more efficient. It is easy to check that not including covariates will not create a biased estimate of causal effect. 
	
	\item Weak IVs: literature has used F-test as a measure of strength, $F < 10$ is considered weak. However, this is not recommended for several reasons. Ex. F-statistic depends on sample size. 
	
	\item Overidentification test: when there are multiple IVs, we can test if some IV has effect not through the exposure. However, these tests generally have low power in detecting violation of IV assumptions. 
	
	\item Endogeneity test: test if the observational and IV estimates are the same. Not recommended because of power (even if the result is not significant, it does not mean that the two estimates are the same). 
\end{itemize}

Examples of MR analysis [Chapter 5]
\begin{itemize}
	\item Fibrinogen and CHD: IV is a SNP in the promoter of Fibrinogen. However, the IV is also associated with ApoB/A1 with $p = 0.01$. It does not pass multiple testing threshold (ambiguous). Possible that ApoB/A1 acts between fibrinogen and CHD. 
	
	\item BMI and blood pressure: use two SNPs of BMI (FTO and MC4R) as IVs. 
\end{itemize}

Weak instruments and finite-sample bias [Chapter 7]
\begin{itemize}
	\item Example of weak IV bias: partition the data into smaller subsets, and do MR, and use meta-analysis to combine the results from smaller studies [Table 7.1]. Very large bias in the estimated effects. 
	
	\item Intuition of weak IV bias: in MR, $G$ is supposed to capture difference of exposure, but may capture difference of confounders by chance (a finite-sample effect). Our estimator is $\beta = \Delta Y / \Delta X$, where $\Delta$ is the difference of $X$ or $Y$ in different genetic subgroups. If true value of $\Delta X$ is known, there should be no bias. 
	
	\item Analysis of weak IV bias: our true model is: 
	\begin{equation}
	X = \alpha_1 G + \alpha_2 U + \epsilon_X \qquad Y = \beta_1 X + \beta_2 U + \epsilon_Y
	\end{equation}
	where $\epsilon_X$ and $\epsilon_Y$ are independent. The ratio estimator is given by: 
	\begin{equation}
	\hat{\beta}_1 = \frac{\Delta Y}{\Delta X} = \beta_1 + \frac{\beta_2 \Delta U}{\alpha_1 + \alpha_2 \Delta U}
	\end{equation}
	where $\Delta$ means the difference in genetic subgroups. When sample size is large, because $G$ and $U$ are independent, then $\Delta U \to 0$, so unbiased. With finite sample, $\Delta U$ may not be 0. When $\alpha_1$ is large (strong IV), the bias is small. But when $\alpha_1$ is small relative to $\alpha_2 \Delta U$, the bias is close to $\beta_2 / \alpha_2$. In summary, in the one-sample case, finite sample (variation of $U$) leads to difference of $Y$ in genetic groups, and the direction is the same as the confounder effect. When true causal effect is 0, this leads to the increased FP rate. 
	
	\item Analysis: the situation is opposite in the two-sample setting. In this case, the chance variation of $\Delta Y$ and $\Delta U$ would be uncorrelated, so the bias is in the direction of null.  
\end{itemize}

Possible strategy to overcome weak IV bias [Chapter 7]
\begin{itemize}
	\item Choose large IVs using F-statistics (for IV). If use single IV, in general, if $F > 10$, generally OK. However, if one has multiple IVs (2SLS method), and choose the stronger IVs by F-test can lead to selection bias (winner's case). 
	
	\item Multiple IVs: not always better than single IV, because some IVs may be weak and lead to bias. 
	
	\item Adjusting for measured confounders improves estimate. 
\end{itemize}

Using multiple IVs [Chapter 8]
\begin{itemize}
	\item In general, using multiple IVs helps: reduce weak IV bias, and improves the power, which depends on how much variation of exposure is explained by IV. 
	
	\item Allele scores: if use the weights from regression in the same data, this is the same as 2SLS. However, this suffers from Winner's curse. Better to use external data to obtain weights. If external data not available, could use cross-validation: calculate allele scores for an individual using weights estimated from independent samples.   
\end{itemize}

Summary statistics in multiple studies [Chapter 9]
\begin{itemize}
	\item Problem setting: summary statistics of $\hat{\beta}_X$ and $\hat{\beta}_Y$ for possibly multiple variants in multiple studies. In some studies, only $\hat{\beta}_X$ or $\hat{\beta}_Y$ are available. 
	
	\item Multiple variant single study: for variant $k$, we have: 
	\begin{equation}
	\left( \begin{array}{l}
	\hat{\beta}_{Xk}\\
	\hat{\beta}_{Yk}
	\end{array}
	\right) = N \left(
	\left( \begin{array}{l}
	\xi_k\\
	\eta_k
	\end{array}
	\right),
	\left(
	\begin{array}{ll}
	\sigma_{Xk}^2 & \rho \sigma_{Xk} \sigma_{Yk} \\
	\rho \sigma_{Xk} \sigma_{Yk} & \sigma_{Yk}^2
	\end{array}
	\right)	
	\right) \qquad \eta_k = \beta_1 \xi_k
	\end{equation}
	where $\sigma_{Xm}$ and $\sigma_{Ym}$ are standard errors. The coefficient $\rho$ captures the correlated effects of $\hat{\beta}_{Xk}$ and $\hat{\beta}_{Yk}$ in the same samples. To see this, we consider the estimated effects: 
	\begin{equation}
	\hat{\beta}_X = (G^T G)^{-1} G^T X \qquad \hat{\beta}_Y = (G^T G)^{-1} G^T Y
	\end{equation}
	When there is confounder $U$, then $X$ and $Y$ are correlated (we consider $G$ as fixed/given), so $\hat{\beta}_{Xk}$ and $\hat{\beta}_{Yk}$ are correlated. The inference can be done via likelihood ratio and to test $\beta_1$ with LRT. The parameters $\xi_k$ and $\eta_k$ and $\rho$ are nuisance parameters. 
	
	\item Fixed effect model: To combine across studies, we assume the model above for each study, but $\xi_k, \eta_k$ may differ for each variant with the same $\beta_1$. A study with only exposure and outcome can still add information by having the model: 
	\begin{equation}
	\hat{\beta}_{Xm} = N(\xi_m, \sigma_{Xm}^2)
	\end{equation} 
	for study $m$. 
	
	\item Random effect models: instead of treating $\xi_{km}$ and $\eta_{km}$ as parameters, we assume a common normal distribution. Similarly we can use a normal prior for $\beta_{1m} \sim N(\beta_1, \tau^2)$. 
	
	\item Special case: 2-sample MR, with one study of exposure and the other outcome. The same model can be used. When there is no sample overlap, we have $\rho = 0$. 
\end{itemize}

\subsection{MR, Mediation and Related Methods}

Mediation analysis in Genetics [personal notes]
\begin{itemize}
	\item Ref: GMAC paper [Yang and Lin, GR, 2016]. 
	
	\item Model: suppose we are testing if a trans-eQTL is mediated by a cis-gene, we have SNP $G$, $X$ and $Y$ with the model: 
	\begin{equation}
	G \rightarrow X \rightarrow Y \qquad G \rightarrow Y
	\end{equation}
	where $G \rightarrow Y$ describes the unmediated effect (horizontal pleiotropy from MR perspective). 
	
	\item Adjusting for possible confounders: suppose we have $U$ (observed) associated with $X$ and $Y$, there are two scenarios: 
	\begin{equation}
	X \leftarrow U \rightarrow Y \qquad X \rightarrow U \leftarrow Y
	\end{equation}
	In the first case, $U$ is a confounder, and should be adjusted. But in the second case, $U$ is a collider, and should not be adjusted. To distinguish the two, note that when $U$ is a confounder, $L_i$ and $U$ should be uncorrelated. When $U$ is a collider, $L_i$ should affect $U$. 
	
	\item The impact of LD: suppose $L_i$ is not a causal variant, is adjusting $L_i$ itself sufficient? No. Let $L_j$ be the true causal variant, which may have unmediated effects. Our causal digram:
	\begin{equation}
	L_i \rightarrow X \rightarrow Y \qquad L_i \leftarrow V \rightarrow L_j \qquad L_j \rightarrow \{X, Y\}
	\end{equation}
	where $V$ is a hidden variable for LD. If there is no $L_j \rightarrow Y$ effect, then adjusting $L_i$ would block all backdoor path from $X$ to $Y$. However, when $L_j$ has unmediated effect, adjusting $L_i$ is not enough, since there is a backdoor path: $X \leftarrow L_j \rightarrow Y$
	
	\item Does significant mediation imply causality? No. See the notes in \texttt{Statistics.pdf}, ``Does Mediation imply causality?'' Significant mediation could happen if: 
	\begin{equation}
	G \rightarrow X, \qquad X \leftarrow U \rightarrow Y, \qquad G \rightarrow Y
	\end{equation}
	Biologically, this may happen when $G$ is eQTL of a gene $X$, and is also in LD with a causal variant of $Y$. 
	
	\item Comparison with MR: mediation analysis allows $G$ to influence $Y$ through other paths (with effect size $\delta$). Under the MR assumption, without this pleiotropic effect, association of $G$ with $Y$ implies $X \rightarrow Y$. 
\end{itemize}

An integrative genomics approach to infer causal associations between gene expression and disease (LCMS) [Schadt \& Lusis, NG, 2005]: 
\begin{itemize}
	\item Motivation: QTL studies often map to regions with multiple genes. Even if a single gene can be mapped, additional evidence is needed to show that the gene is causal to the trait. The idea is to use expression data to bridge the gap: 
	\begin{itemize}
		\item Intuitively, if the expression of a gene in QTL is also correlated with the trait, then this gene is likely to be involved in the trait.
		\item If the expression of a gene (somewhat related to this process) is linked to the QTL, then this QTL is functional (as opposed to other QTLs that may be be false positives). 
	\end{itemize}
	The issue is to decide the causality. 
	
	\item Model: let $L$ be QTL (genotype), $R$ be transcript, and $C$ be clinical trait, both eQTL and cQTL are mapped to the same loci, thus we know that change of $L$ leads to change of $R$ and $C$, the problem is to distinguish three models:
	\begin{equation}
	M_1: L \rightarrow R \rightarrow C \qquad M_2: L \rightarrow C \rightarrow R \qquad M_3: L \rightarrow C, L \rightarrow R, C \leftrightarrow R
	\end{equation}
	The three models are denoted as: causal, reactive and independent model, respectively. The likelihood based causality model selection (LCMS) chooses the model by computing the likelihood of the three models: 
	\begin{equation}
	\begin{array}{ll}
	P(D|M_1) = P(L) P(R|L) P(C|R) &	P(D|M_2) = P(L) P(R|L) P(C|R) \\
	P(D|M_3) = P(L) P(C|L) P(R|L,C)
	\end{array}
	\end{equation}
	Since the three models have different complexities (one more parameter in $M_3$), use AIC to choose a model. 
	
	\item An alternative test: conditional independence test. Under $M_1$, $L \independent C|R$, so the corresponding partial correlation coefficient (PCC) should be 0; similarly under $M_2$, $L \independent R|C$. Test the two PCCs to determine which test should be accepted: if one PCC is 0, accept $M_1$ or $M_2$; if neither, accept $M_3$. The problem is: if only one PCC is used, the test is incomplete, e.g. suppose we test PCC between $L$ and $C$ given $R$, but when $M_2$ is true, the test loses power; if both tests are used, it is not clear how they are combined. 
	
	\item Intuition of LCMS: suppose $M_1$ is true, and suppose $L$ is strongly correlated with $R$, but $R$ correlation with $C$ is weak. Then $M_2$ is a poor fit, because the correlation of both $L$ and $C$ and the correlation of $C$ and $R$ are both weak. For $M_3$, it is more complex, but it does not add much explanation beyond $M_1$. In general, suppose we have three variables with $X \rightarrow Y \rightarrow Z$, with the linear model between the two edges (coefficient $\beta$ and $\gamma$), then
	\begin{equation}
	\Var(Z|x) = \gamma^2 \sigma_Y^2 + \sigma_Z^2	
	\end{equation}
	which is larger than both $\sigma_Y^2$ and $\sigma_Z^2$. So a model with the edge $X \rightarrow Z$ would not be a good fit (large variance). In general, the likelihood method will try to put the variables with strong correlation into direct edges. 
	
	\item Data: 111 $F_2$ mouse from crossing two strains. Each mouse is: (1) genotyped for 139 microsatellite markers; (2) the liver expression data is profiled; (3) is phenotyped for the disease trait (obesity). 
	
	\item Multi-step procedure: (Figure S3)
	\begin{itemize}
		\item Step 1: define the QTL of obesity. Found four QTL. 
		\item Step 2: find a list of diff. expressed genes (correlated with phenotype). 440 genes were identified. 
		\item Step 3: test if the four QTL are also eQTL of these 440 genes. There were 113 genes which overlap with at least two QTL (a total of 267 eQTL-gene pairs). The FDR of having two overlap is 0.4\% (compared with 15\% when requiring only one shared QTL), by permutation of genotypes. 
		\item Step 4: LCMS test of 113 genes, the ratio of three models are 50\% (causal), 9\% (reactive) and 41\% respectively. 
	\end{itemize}
	
	\item Remark: 
	\begin{itemize}
		\item The main contribution to success comes from QTL overlap: with the requirement of having two or more QTL, the gene list is limited to 113 genes, and half of them were tested causal. 
		\item Correlation between genes is a major problem: LCMS test cannot distinguish them. That's why 4 QTL of obesity supports the finding of more than 50 genes. A related issue is that the model does not distinguish cis- and trans-eQTL. 
		\item Multi-step procedure: information in many weaker eQTL and cQTL are not used. 
	\end{itemize}
\end{itemize}

Bias in causal estimates from Mendelian randomization studies with weak instruments [Statistics in Medicine, 2011]
\begin{itemize}
	\item Weak IV problem from a causal model perspective: we know that correlation of $G$ and $U$ leads to a backdoor path: $G \leftrightarrow U \rightarrow Y$. Even when $G$ and $U$ has no confounder, stochastic/chance correlation can still lead to the backdoor path, and association between $G$ and $Y$. 
	
	\item Simulation to investigate weak IV bias: setting, no causal effect of $X$ to $Y$. Error of $U, X, Y$: variance 1. Effects of $U$ on $X$ and $Y$ = 1. Vary $\alpha_1$, the IV strength, from 0.05 to 0.55 (Figure 1, Table 1). At $\alpha_1 = 0.05$, or F-statistics 1.07 (slightly above the null expectation), often non-zero $\beta_1$ (causal effect). But at F-statistics at 8, relatively free of bias.  
	
	\item Multiple IVs: reduce the variance of estimator, but increase the bias. 
	
	\item Discussion: IVs should be ascertained before doing MR (estimate effect sizes to both $X$ and $Y$). 
\end{itemize}

Mendelian randomization analysis with multiple genetic variants using summarized data [Burgess \& Thompson, GE, 2013]
\begin{itemize}
	\item Problem: given summarize statistics of variants $X_k$ and $Y_k$ for variant $k$, how do we estimate the effect of $X$ on $Y$, assuming the variants are valid IVs?
	
	\item Likelihood model (Eq. 3): similar bivariate model as Sherlock2. Let $\xi_k$ be the true effect of $k$ on trait $X$, then the expected effect on trait $Y$ is $\beta \xi_k$. The difference is that: not allow residual effect of $k$ on trait $Y$; no prior on $\xi_k$ (treated as fixed). 
	
	\item Comparison of summary statistics vs. individual level data: similar performance when assumptions hold. The summary method overstates precision with LD. Weak instrument bias. 
	
	\item Remark: we could investigate weak instrument bias. Our model is $X = \alpha_1 G + \alpha_2 U$ and $Y = \beta_1 X + \beta_2 U$. We can derive $\hat{\beta}_{IV}$ and show that its expectation is not the same as $\beta_1$. 
\end{itemize}

Multivariable Mendelian randomization: the use of pleiotropic genetic variants to estimate causal effects. [Burgess and Thompson, AJE, 2015]
\begin{itemize}
	\item Motivation: often hard to find valid IVs because they have pleiotropic effects on multiple risk factors. Ex. LDL, HDL and TG.
	
	\item Different scenarios: Figure 2, vertical and functional pleiotropy. Also there could be causal effects among risk factors (Figure 3).
	
	\item Model: Two stage LS with individual level data. First do multiple regression to build predictive model of multiple risk factors using all putative IVs. Then regression of trait vs. predicted risk factors. Note: this can be done when the risk factor genetic data and trait genetic data are collected in different samples.
	
	\item Likelihood method when we only have summary statistics: Equation (1), we model the summary statistics of SNP to each risk factor, and SNP to trait. The joint distribution is MVN, where correlation among different risk factors and traits capture sample overlap. To incorporating LD, summary statistics of related SNPs are correlated.
	
	\item Results on HDL, LDL and TG on CHD risk: in joint analysis with likelihood model, only LDL and TG show non-zero effects.
	
	\item The impact of causal effects among risk factors: MVMR estimates direct effects, misleading results on total effects.
\end{itemize}

Bias due to participant overlap in two-sample Mendelian randomization [Burgess and Thompson, GEpi, 2016]
\begin{itemize}
	\item Weak IV bias under 2SLS: consider two stage model:
	\begin{equation}
	X = G \alpha_1 + \epsilon_X \qquad Y = \tilde{X} \beta_1 + \epsilon_Y 
	\end{equation}
	where $\tilde{X} = G \hat{\alpha}_1$ is the least square estimator of $X$. When $G$ and $U$ are correlated in finite samples, $\epsilon_X$ is now correlated with $U$ and $\epsilon_Y$ is also correlated with $U$, leading to correlation of $\epsilon_X$ and $\epsilon_Y$. Let their covariance be $\sigma_{XY}$. Also define $\mu$ be the concentration parameter, a measure of IV strength and $K$ the number of IVs used. The expected bias is then: 
	\begin{equation}
	\text{Bias of 2SLS} = \frac{\sigma_{XY} (K-2)}{\sigma_{X}^2 \mu} = \frac{\sigma_{XY}}{\sigma_{X}^2 \E(F)}
	\end{equation}
	where $\sigma_{X}^2$ is the variance of $X$ and $\E(F)$ is the expected F-statistic of the 2SLS estimator of $X$. 
	
	\item Analysis: weak IV bias is a form of overfitting. It is clear that the problem is the prediction of $X$ is correlated with the error of $X$ (which captures $U$). In this case, over-fitting is due to weak IV. In BLUP prediction [GBAT paper, Xuanyao Liu], over-fitting is due to the BLUP model. 
	
	\item Weak IV bias under 2SLS in two-sample setting: in the 2-sample setting, $\sigma_{XY}$ should generally be close to 0, so the bias is close to 0.  

	\item Question: does 2SLS reduce the bias? The claim that multiple IVs increase the bias [Burgess and Thompson, Stats in Med, 2011]. 
\end{itemize}

Orienting the causal relationship between imprecisely measured traits using GWAS summary data (MR-Steiger) [Hemani, PLG, 2017] 
\begin{itemize}
	\item Comparison of Mediation and MR: (1) Mediation: fit $Y \sim G + X$, where $X$ is observed. (2) MR: the causal diagram can be understood as: $G \rightarrow \tilde{X} \rightarrow X \rightarrow Y$, where $\tilde{X}$ is the genetic component of $X$ due to $G$. We fit $Y \sim \tilde{X}$. So the differences are:
	\begin{itemize}
		\item Mediation is affected by confounding and measurement error, since it models $X$. 
		\item Mediation based Causal Inference Test (CIT): test both directions. 
	\end{itemize}
	
	\item CIT: a set of four conditions to conclude the direction is $X \rightarrow Y$. The conditions reject $G \rightarrow Y \rightarrow X$ and $X \leftarrow G \rightarrow Y$. It establishes $G \rightarrow X \rightarrow Y$ by CI of $G$ and $Y$ given $X$. \\
	Problem of CIT: if there is a confounder $U$ affecting both $X$ and $Y$, then $G$ and $Y$ are dependent when conditioned on $X$, because $X$ is a collider. 
	
	\item MR causal test: two tests, first MR test to assess causal relationship. Next, Steiger test: compares correlation of $G$ and $X$, $\rho_{gx}$ vs. correlation of $G$ and $Y$ $\rho_{gy}$. The idea is that if $G \rightarrow X \rightarrow Y$, then $G$ and $X$ would be more correlated. 
	
	\item Mediation-based CIT is sensitive to measurement error: let $X_O$ be the observed $X$, our model is: 
	\begin{equation}
	G \rightarrow X \rightarrow Y \qquad X \rightarrow X_O
	\end{equation}
	It's clear now that $X_O$ no longer blocks the path from $G$ to $Y$, thus CI does not hold.  
	
	\item Remark: Steiger filtering refers to: filtering variants by correlations.
	
	\item \textbf{Lesson}: CIT that relies on CI is problematic in practice because of confounder and measurement error. 
\end{itemize}

A framework for the investigation of pleiotropy in two-sample summary data Mendelian randomization. [Bowden and Thompson, Stat Med. 2017]
\begin{itemize}
	\item Notation: standard MR, let $\beta$ be the causal effect, $\gamma_j$ be the IV effect size on exposure, and $\Gamma_j$ be the IV effect on outcome. Let $\alpha_j$ be the direct effect on outcome, and $\Psi_j$ be the effect on confounder $U$. Also let $\kappa_X$ and $\kappa_Y$ be the effect of $U$ on exposure and outcome, respectively. 
	
	\item Model: assume $\gamma_j$ is known (measured). The estimated effect on outcome:
	\begin{equation}
	\hat{\Gamma}_j = \alpha_j + \beta \gamma_j + \epsilon_j \qquad \epsilon_j \sim N(0, \sigma_{Yj}^2)
	\end{equation}
	For simplicity, assume $\alpha_j = 0$. Let $\hat{\beta}_j = \hat{\Gamma}_j / \gamma_j$ be the estimated causal effect from SNP $j$, the IVW estimator is weighted average of $\hat{\beta}_j$, where weight is given by:
	\begin{equation}
	w_j = \frac{1}{\Var (\hat{\beta}_j)} = \frac{\gamma_j^2}{\sigma_{Yj}^2}
	\end{equation}
	So the weight depends on both IV strength and the s.e. of the effect estimate on outcome. Putting this together, we have the IVW estimator:
	\begin{equation}
	\hat{\beta}_{\text{IVW}} = \frac{\sum_j \gamma_j \Gamma_j \sigma_{Y,j}^{-2}}{\sum_j \gamma_j^2 \sigma_{Y,j}^{-2}}
	\end{equation}
	Note: see Equation (1) of [Burgess and Thompson, Mendelian Randomization Analysis With Multiple Genetic Variants Using Summarized Data, GE, 2013]. 

	\item Random effect IVW: if we assume $\alpha_j \sim N(0, \sigma_{\alpha}^2)$, and use standard random effect meta-analysis, this is the additive random effect IVW. However, this is not good in MR setting: as this model generally put more weights on the outliers. Recommend to use multiplicative random effect model: where the the model can be written as:
	\begin{equation}
	\hat{\Gamma}_j = \alpha_j + \beta \gamma_j + \phi^{1/2} \epsilon_j \qquad \epsilon_j \sim N(0, \sigma_{Yj}^2)
	\end{equation}
	where $\phi$ is estimated from the heterogeneity. 
\end{itemize}

Distinguishing genetic correlation from causation across 52 diseases and complex traits (LCV) [O’Connor \& Price, NG, 2018]
\begin{itemize}
	\item Model: see Figure 1. Let $q_1$ and $q_2$ be the effects of $L$ (latent factor) on $Y_1$ and $Y_2$ respectively ($q_k$ are scalars). Let $\pi$ be the SNP to $L$ effect and $\gamma_1, \gamma_2$ be non-mediate effect of SNP to $Y_1, Y_2$ (these are distributions). Effect sizes are normalized: ``$\alpha$ and $\pi$ (but not $\gamma$) are normalized to have unit variance, and all random variables have zero mean''. 
	\begin{equation}
	\Var(\pi) = 1 \qquad \Var(\pi q_k + \gamma_k) = 1 \Rightarrow q_k^2 + \Var(\gamma_k) = 1, \text{ where } k = 1,2
	\end{equation}
	The interpretation of $q_k^2$ is then the proportion of heritability of trait 1 (or 2) that is mediated by the latent factor. 
	
	\item Assumptions of LCV: needs several independence assumptions, but weaker than independence of $(\pi, \gamma_1, \gamma_2)$. Most notably, it requires: SNPs with a large mediated effect do not tend to also have an additional non-mediated effect.
	
	\item Genetic causality proportion (GCP): when $q_1 = 1$, $\gamma_1 = 0$, this means trait 1 is causal to trait 2. Similarly, when $q_2 = 1$, we have trait 2 is causal to trait 1. In general, $q_1 > q_2$ means that trait 1 is ``partially causal'' to trait 2, through the latent factor. To define this, use gcp, defined as: 
	\begin{equation}
	\text{GCP} = \frac{\log \abs{q_2} - \log \abs{q_1}}{\log \abs{q_2} + \log \abs{q_1}}
	\end{equation} 
	Equivalent definition is: $q_2^2 /q_1^2 = (\rho_g^2)^x$. 
	
	\item Auxiliary test: estimate proportion of heritability explained by the correlated component. It performs poorly when two traits have unequal power and unequal polygeneicity, as expected (Table S2): highly inflated type I error.  
	
	\item Inference with MOM estimator: Use this equation (derivation in Methods): 
	\begin{equation}
	\E(\alpha_k^2 \alpha_1 \alpha_2) = \kappa_{\pi} q_k^2 q_1 q_2 + 3 \rho_g, k = 1, 2
	\end{equation}
	where $\kappa_{\pi} = \E(\pi^4) - 3$ and $\rho_g$ is genetic correlation. Intuitively, if $q_1^2 > q_2^2$, we should have $\E(\alpha_1^3 \alpha_2) > \E(\alpha_1 \alpha_2^3)$, this allows us to infer the difference of $q_1$ and $q_2$, hence GCP. Specifically, first estimate $\rho_g$, and estimated mixed fourth moments accounting for errors in $\hat{\alpha_1}$ and $\hat{\alpha_2}$. We define, for $k = 1, 2$: 
	\begin{equation}
	\hat{\kappa_k} = \E(\alpha_k^2 \alpha_1 \alpha_2) - 3 \rho_g
	\end{equation}
	If there is no population structure or sample overlap, this term can be easily estimated, just observed moments subtracting 3 $\rho_g$. In real data, we use this result (Equation 6 of paper): 
	\begin{equation}
	\E(\hat{\alpha}_1 \hat{\alpha}_2^3 | \alpha_1, \alpha_2) = \alpha_1 \alpha_2^3 + 3 \alpha_1 \alpha_2 \sigma_2^2 + \alpha_2^2 \sigma_{12} + 3 \sigma_{12} \sigma_2^2
	\end{equation}
	where $\sigma_2$ is the intercept of LDSC and $\sigma_{12}$ the intercept of cross-trait LDSC. This allows us to estimate $\E(\alpha_1 \alpha_2^3)$ by computing expectation of $\hat{\alpha}_1 \hat{\alpha}_2^3$ subtracting the last three tersm of the RHS of this equation. Intuitively, the difference of $\hat{\kappa_1}$ and $\hat{\kappa_2}$ should reflect the value of GCP. We could obtain a MOM estimator of GCP $x$ (not what the paper does). From Equation (2) of the paper, we have:  
	\begin{equation}
	\E(\hat{\kappa_1}) = \kappa_{\pi} q_1^3 q_2 \qquad \E(\hat{\kappa_2}) = \kappa_{\pi} q_1 q_2^3
	\end{equation}
	We notice that $q_2^2 /q_1^2 = (\rho_g^2)^x$, so we have: 
	\begin{equation}
	\frac{\hat{\kappa_1} - \hat{\kappa_2}}{\sqrt{\hat{\kappa_1}^2 + \hat{\kappa_2}^2}} = \frac{q_1^2 - q_2^2}{\sqrt{q_1^4 + q_2^4}} = \frac{1 - \rho_g^{2x}}{\sqrt{1 + \rho_g^{4x}}}
	\end{equation}
	This allows one to estimate $x$ from test statistics, $\hat{\kappa_1}$ and $\hat{\kappa_2}$, and $\rho_g$. Note that the LHS of this equation is very similar to the test statistic $S(x)$ in the paper (Equation 7). 
	
	\item Dealing with LD: in the estimation of moments above, e.g. $\hat{\alpha}_1 \hat{\alpha}_2^3$ over all SNPs, we weigh the SNPs by $\max\{1, 1/l_i\}$ where $l_i$ is the LD score of SNP $i$. This weighting scheme will reduce the weights of SNPs with high LD (otherwise, they would be over-counted). 
	
	\item Inference with likelihood: the paper uses a different estimator. It normalizes $\hat{\kappa_1} - \hat{\kappa_2}$ using $x$ s.t. the resulting distribution is approximately standard normal when $x$ is the true value of GCP. However, I am not able to show this. Suppose it is true, let $S(x) / \sigma_{S(x)}$ be the ``normalized'' test statistic, then $S(x) / \sigma_{S(x)} \sim N(0,1)$ when $x$ is the true value. Since $x$ is unknown, we compute $S(x) / \sigma_{S(x)}$ for each value of $x$, and we should have: the true value of $x$ gives standard normal (high degree t-distribution in the paper), but not other values of $x$. This leads to the likelihood-based estimator of the paper. 
	
	\item Remark: to understand the estimation procedure, consider the simple problem of estimating $\mu$ in a normal distribution $N(\mu, \sigma^2)$. We define $T(\bar{x}, \mu) = (\bar{x} - \mu) / \sigma$, then it should follow $N(0,1)$ under the true value of $\mu$, assuming $\sigma$ is known. To estimate $\mu$, we compute $T(\bar{x}, \mu)$ for different values of $\mu$, and the true value of $\mu$ is obtained by maximizing the likelihood: $P(T(\bar{x}, \mu) | N(0,1))$ under different values of $\mu$.  
	
	\item Accounting for LD: in the estimator, a SNP is weighted by the inverse of its LD score. 
	
	\item Methods compared: (1) Two-sample MR: ascertain SNPs significant for exposure, then estimate correlation coefficient with intercept 0. (2) MR-Egger: similar, but fitted intercept. (3) Bidirectional MR: ascertain SNPs for both exposure and outcome, then for each estimate $r_1$ or $r_2$, and test the difference. 
	
	\item Analysis of MR models: Two-sample MR captures any genetic correlation between two traits, even if gcp = 0. Bidirectional MR: simply consider the difference of $r_1$ and $r_2$, it does not account for different genetic architecture and power of the two traits. 
	
	\item Simulation procedure: Section 1.4 in Supplement. Assume $q_1$ and $q_2$ are given. Step 1. Simulate $\pi$ and $\gamma_k$ ($k = 1,2$). Sample $\pi$ from spike-and-slab distribution with mean 0 and variance 1. Specifically: 
	\begin{equation}
	\pi \sim p_{\pi} N(0, 1/p_{\pi}) + (1- p_{\pi}) \delta_0
	\end{equation} 
	where $p_{\pi}$ is a specified parameter. It's easy to check that variance of $\pi$ is 1 using the results about variance of mixture distribution [Wiki, mixture distribution]. Sampling of $\gamma_k$ is similar: use spike-and-slab distribution with mean 0 and variance $1 - q_k^2$, specifically: 
	\begin{equation}
	\gamma_k \sim p_{\gamma_k} N(0, (1-q_k^2)/p_{\gamma_k}) + (1- p_{\gamma_k}) \delta_0
	\end{equation}
	Step 2. Simulate effect sizes in the usual scale. Let $\beta_k$ be the effect size of trait 1 and 2. Per SNP heritability, or the variance of $\beta_k$ (assuming standardized genotypes), should be $h_k^2/M$, where $h_k^2$ is hertiability and $M$ number of SNPs. So should scale $\alpha_k$ from the previous step as: 
	\begin{equation}
	\beta_k = \frac{h_k}{\sqrt{M}} (q_k \pi + \gamma_k)
	\end{equation}
	Note: the equation of effect size scaling in this step is wrong, but the code is correct. Step 3. Simulate summary statistics without LD: $\hat{\beta_k} \sim N(\beta_k, s_k^2)$, where $s_k$ is the standard error of the estimator, which depends only on sample size for a given AF of SNP. Note: under LCV, variance of effect sizes of shared SNPs, $q_1^2 / p_{\pi}$, and SNPs acting only on trait 1, $(1-q_1^2)/p_{\gamma_1}$, can be different. 
	
	\item Connection with CAUSE: when $h_k^2$ and $M$ are given, LCV has five free parameters, $p_{\pi}, p_{\gamma_1}, p_{\gamma_2}, q_1, q_2$, while CAUSE has four: proportion of causal variants $p_M, p_Y$, and sharing parameters $q, \eta$. Note that the effect size variance of causal variants $\sigma_M^2$ and $\sigma_Y^2$ (for simplicity, we use spike-and-slab instead of ASH), are determined once $p_M$ and $p_Y$ are given: 
	\begin{equation}
	p_M \sigma_M^2 = h_M^2 / M \qquad p_Y \sigma_Y^2 = h_Y^2 / M
	\end{equation}	
	The difference of the number of parameters is because in LCV, the shared variants and variants acting only on trait 1 could have different effect size variance. Now suppose we have LCV parameters, $p_{\pi}$, $p_{\gamma_k}$, $q_1$, $q_2$ and we will determine CAUSE parameters. %Note $p_{\gamma_k}$ here are proportions of variants acting only on trait $k$. To make the model realistic, we can add the constraint that the effect sizes of shared variants and trait 1 (or 2) only variants have the same variance: 
	%	\begin{equation}
	%	\frac{q_1^2}{p_{\pi}} = \frac{1 - q_1^2}{p_{\gamma_1}} \qquad \frac{q_2^2}{p_{\pi}} = \frac{1 - q_2^2}{p_{\gamma_2}}
	%	\end{equation}
	%	So effectively, $p_{\gamma_1}$ and $p_{\gamma_2}$ are not free parameters - they are determined by $q_1, q_2$ and $p_{\pi}$. 
	First, the proportion of causal variants for trait $M$ and $Y$: 
	\begin{equation}
	p_M = p_{\pi} + p_{\gamma_1} \qquad p_Y = p_{\pi} + p_{\gamma_2}
	\end{equation}
	The sharing parameters of CAUSE: $q$ is the proportion of shared variants among all variants acting on $M$, and for $\eta$, it is the ratio of effect size of shared SNPs on trait $M$ and $Y$, respectively 
	\begin{equation}
	q = \frac{p_{\pi}}{p_{\pi} + p_{\gamma_1}} \qquad \eta = \frac{q_2 \sqrt{h_2^2/M}}{q_1 \sqrt{h_2^2/M}} = \frac{q_2 h_2}{q_1 h_1}
	\end{equation}
	Now suppose we know CAUSE parameters and we will determine LCV parameters. The proportion of causal variants: 
	\begin{equation}
	p_{\pi} = p_M \cdot q \qquad p_{\gamma_1} = p_M (1-q) \qquad p_{\gamma_2} = p_Y - p_M q
	\end{equation}
	Next we consider effect size variance of shared variants on trait 1/$M$: it should be $\sigma_M^2$ under CAUSE, and $1/p_{\pi}$ (variance on $L$) times $q_1^2$ (variance on trait 1) times $h_1^2/M$ (scaling parameter). So we have: 
	\begin{equation}
	q_1^2 \cdot \frac{1}{p_{\pi}} \cdot \frac{h_1^2}{M}	= \sigma_M^2
	\end{equation}
	We use the equation $p_M \sigma_M^2 = h_M^2 / M$, and $p_M q = p_{\pi}$ to obtain: 
	\begin{equation}
	q_1 = \sqrt{q} \qquad q_2 = \eta q_1 h_1 / h_2 = \eta \sqrt{q} h_M/ h_Y
	\end{equation}
	It's easy to check that with this transformation, the effect size variance of shared SNPs and SNPs acting only on trait 1 are the same: 
	\begin{equation}
	\frac{q_1^2}{p_{\pi}} = \frac{q}{p_M q} = \frac{1}{p_M} \qquad \frac{1 - q_1^2}{p_{\gamma_1}} = \frac{1-q}{p_M (1-q)} = \frac{1}{p_M}
	\end{equation} 
	
	\item Null simulation with uncorrelated effects: 50,000 independent SNPs. Some SNPs may be shared by two traits, but genetic effects are uncorrelated. LCV and all other methods control type I error well (Figure 2a). 
	
	\item Null simulation with correlated effects: certain percent of SNPs are shared between two traits, and $q_1 = q_2$ (but non-zero). Also vary power and polygenecity of the two traits. All MR methods fail in this case, but GCP performs well (Figure 2bc). 
	
	\item Causal simulation: Choose $N_1 = N_2 = 25,000$, 5\% SNPs are causal to trait 1, $q_1 = 1, q_2 = 0.2$. The power of LCV is much higher than other MR methods (Figure 3).   
	
	%\item Also consider violations of model assumptions: (1) Two latent variables. (2) SNP efects drawn from a mixture of multi-variate normal distributions. 
	
	\item Application to real data: 52 traits. 32\% of trait pairs show nominal significance $P < 0.05$, and 59 pairs at FDR $< 0.01$, including 30 with GCP $> 0.6$ (Table 1). Some of the 30 pairs include: lipid and TG on MI, LDL on bone marrow density. 
	\begin{itemize}
		\item Some pairs in Table 1 are likely not causal: Triglycerides and Platelet distribution width. High cholesterol and Red blood cell count. Triglycerides and Reticulocyte count, Eosinophil count. Balding and number of children. MI and breast cancer. 
		
		\item Autism and education attainment: gcp = 0.13. 
		
		\item MR: MR reported significant causal relationships (1\% FDR) for 271 of 429 trait pairs, including 155 reciprocal pairs of traits. 
	\end{itemize}
	
	\item Analysis: problems of GCP. GCP is only vaguely connected to causality. It tests asymmetry $q_1 > q_2$. It's possible that GCP is large even when $q_1$ is significantly smaller than 1. Ex. $q_1 = 0.5, q_2 = 0.05$, gcp = 0.62. Also, when $q_2$ is very small (regardless of $q_1$), conceptually $L$ has no effect on $Y_2$, but GCP will converge to 1, e.g. $q_1 = 0.5, q_2 = 0.00001$, GCP = 0.88.  
	
	\item Analysis: why the LCV estimator is robust to unequal polygeneicity (Table S2)? Simulation setting: 1\% of variants affect $L$, 2\% affect $Y_1$ only and 8\% affect $Y_2$ only, with equal effect size distributions. If we interpret $q_1$ and $q_2$ as percent heritability explained by $L$ (the auxilary test), we would find GCP $>0$. However, the LCV estimators works well as long as $q_1 = q_2$, which are effect sizes of $L$ on $Y_1$ and $Y_2$. To see this, let $\pi_L$, $\pi_1$, $\pi_2$ be the percent variants affecting $L$, $Y_1$ only and $Y_2$ only. Then: 
	\begin{equation}
	\E(\alpha_1^3 \alpha_2) = \pi_L \E(\alpha_1^3 \alpha_2 | L) + \pi_1 \E(\alpha_1^3 \alpha_2 | Y_1) + \pi_2 \E(\alpha_1^3 \alpha_2 | Y_2)  
	\end{equation}
	Note that the last two terms are 0 because $\alpha_1^3 \alpha_2 \neq 0$ if and only both $\alpha_1$ and $\alpha_2$ are non-zero. The same is true for $\E(\alpha_1 \alpha_2^3)$. This explains that only variants acting on $L$ contribute directly to the test statistic.
	
	\item Remark: other problems of LCV
	\begin{itemize}		
		\item LCV does not distinguish exposure and outcome, instead the two traits are ``interchangable labels''. 
		
		\item LCV does not infer specific SNPs contributing to the partial causality. 
	\end{itemize}
\end{itemize}

Causal associations between risk factors and common diseases inferred from GWAS summary data [Zhu and Yang, NC, 2018]
\begin{itemize}
	\item Generalized SMR (GSMR) model: multiple independent SNPs, we expect similar estimate of $b_{xy}$. So the method does Generalized Least Square (GLS), regression of $\hat{b}_{zy}$ vs. $\hat{b}_{zx}$, accounting for different standard errors and LD between SNPs. This is equivalent to a weighted mean of $\hat{b}_{xy(i)}$ over all SNPs. 
	
	\item HEIDI-outlier test: choose the SNP with the strongest effect one exposure among all SNPs based on $\hat{b}_{xy}$ - choose the ones in the third quantile. Then test if another SNP has a different $\hat{b}_{xy(i)}$, and remove outlier SNPs with $p < 0.01$.  
	
	\item Application: 7 risk factors and a large number of phenotypes in UKBB and another dataset. Confirm: (1) BMI on T2D, CVD and hypternsion (2) LDL to CAD and dyslipidemia. (3) Blood pressure and hypertensive diseases and CVD. Note that: LDL vs. LOAD is not significant after removing the outlier using HEIDI. 
	
	\item Adjusting for covariates (other risk factors) with GWAS summary statistics. Not change results much.  
	
	\item Remark: the step of removing pleiotropic SNPs with HEIDI is arbitrary. A non-causal risk factor may share a fraction of SNPs with outcome. This may inflate the causality claim. 
\end{itemize}

Bayesian variable selection with a pleiotropic loss function in Mendelian randomization (JAM-MR) [Gkatzionis and Newcombe, review for PLG, 2019]
\begin{itemize}
	\item Background: JAM for fine-mapping. G-prior for $\beta$ in linear regression. Inference of $\gamma$ by MCMC. 
	
	\item Background: extension of Bayesian inference without likelihood. Let $\theta$ be parameter of interest, we infer $\theta$ not via likelihood but with loss function: 
	\begin{equation}
	P_l(\theta|D) \propto \pi(\theta) \exp \left(-w l(D,\theta) \right)
	\end{equation}
	where $\pi(\theta)$ is the prior, $l(D, \theta)$ is the loss function and $w$ a weight parameter. 
	
	\item JAM-MR model: our idea is to do variant (IV) selection on data of $X$ (mediator). We penalize the variants with different effects (outliers). Introduce a loss function in JAM that favors variants with similar $\theta_j$, where $\theta_j$ is the ratio estimator from SNP $i$. The loss function: 
	\begin{equation}
	l_1(\hat{\theta}, \gamma) = \frac{1}{P_{\gamma} - 1} \sum_{j:\gamma_j = 1} (\hat{\theta}_j - \hat{\theta}_{\gamma})^2
	\end{equation}
	where $\hat{\theta}_{\gamma}$ is the mean of univariate causal effect estimate in model $\gamma$. The final posterior combines: prior of $\gamma$ (Beta), summary statistics based likelihood of $X$, and loss function. 
	
	\item Causal effect estimation: (1) For each model $\gamma$, use meta-analysis, inverse variance weighting, to combine $\hat{\theta}_j$ for all $j \in \gamma$. (2) Combined estimate of all $\gamma$'s: weighted average of all models, where the weights are determined by the posterior $p(\gamma|D)$. 
	
	\item Analysis: how variable selection for $X$ and consistency of causal effect estimates are combined? 
	\begin{itemize}
		\item When penalty $w$ in the loss function is low: the model chooses most SNPs, including some ``invalid'' SNPs that have pleiotropic effects. 
		\item When penalty $w$ increases: the model chooses good SNPs, but also remove SNPs with pleiotropic effects. The optimal range is $w$ about 0.5-5 of sample size $N_1$. 
		\item When penalty $w$ is large: even valid SNPs are not chosen by the model, leading to large standard errors of estimated effects. 
	\end{itemize}
	When the number of pleiotropic SNPs with consistent effects gets large, it's possible that the method chooses these SNPs, rather than valid SNPs. These models will have high posterior probabilities, along with true models. 
	
	\item Simulation setting: Equations (1) - (3), $\beta_{X,j}$ are IV effects sizes, always positive. $\delta_j$ is the direct effects (horizontal pleiotropy), and $\alpha_j$'s are effects on confounder. Use 50 IVs of trait 1, and 15 of them are pleiotropic. Simulate two values of $\theta$ (causal effect) 0 and 0.5. Four settings (Table 1) for pleiotropic variants: (1) Balanced pleiotropy: $\delta_j$s are symmetric. (2)-(4) Directional: $\delta_j$ are positive/constant or $\alpha_j$ non-zero. 
	
	\item Note: in directional settings of simulation, $\delta_j$ are not proportional to $\beta_{X,j}$'s. Similarly, $\alpha_j$ are random across variants. So the simulations are easier than correlated pleiotropic effects. 
	
	\item Results of scenario 1: all methods perform well except MR-Egger. MR-Egger has large standard errors, so it has high MSE, but low type I error. However, JAM-MR has considerable type I errors (12-22\%). 
	
	\item Results of scenario 2-4: IVW, MR-Egger have large standard error and MSE. Weighted median and mode estimator, MR-pressor generally perform well. JAM-MR seems to have lowest errors.  
	
	\item Application in real data: two blood pressure and CHD. Causal effects by most methods, except mode estimator. 
\end{itemize}

An integrative analysis of GWAS and intermediate molecular trait data reveals common molecular mechanisms supporting genetic similarity between seemingly unrelated complex traits (Sherlock-II) [Gu and Hao Li, 2019]
\begin{itemize}
	\item Scoring a gene: ascertain eQTL with $p < 10^{-5}$, then the score defined as: $s = - \sum_i \log_{10} (p_i)$ where $p_i$ is GWAS p-value, $1 \leq i \leq n$. LD pruning: $r^2 < 0.2$, and min. of 100kb between two SNPs. Note: truncate GWAS p-value at $10^{-9}$ makes the results less dominant by single SNPs. 
	
	\item Computation null distribution: (1) Single SNP case: basically, the percent of SNPs with p-values falling into an interval. (2) General case: sum of independent RVs, so we can analytically obtain the null distribution. 
	
	\item Correcting for pleiotropic SNPs: SNPs with pleiotropic effects (genes) are more likely to have small p-values in GWASs. To adjust for this, for any SNP (single-SNP case), its null distribution is obtained by getting the percent of SNPs with p-values in a range, only those SNPs matching number of genes. In the equation, $C_i$ is number of genes SNP $i$ is assigned to. 
	
	\item Dealing with multiple tissues in eQTL: choose the tissue with the strongest eQTL. 
	
	\item Results of gene-phenotype and metabolite-phenotype associations: 2000 gene-phenotype associations in 74 phenotypes and 400 metabolite-phenotype associations. (1) RA: 3 out of 5 genes are all in HLA regions, and the other two are plausible (however, not previously known). (2) several metabolite-trait association, often driven by multiple SNPs (Figure S1). Some justification of metabolites. 
	
	\item Phenotype similarity based on gene-phenotype relations: some expected, e.g. same phenotype, or closely related (e.g. LDL and HDL, BP and SCZ). However, some unexpected, e.g. breast cancer and insulin. 
	
	\item Genetic mechanisms linking seemingly unrelated phenotypes (Figure 3): (1) Age-at -menarche and BMI and childhood obesity: TGF-beta or MAP signaling in top genes. (2) Pathway enrichment test: find pathways contribute most to phenotype similarity, Pearson correlation of phenotypes using only genes in a pathway. RA and CD: antigen presentation, inflammatory cell death. (3) Insulin and breast cancer: cAMP/cGMP signaling. (4) T1D and T2D: apoptosis.  
	
	\item Biclustering analysis: a module of 10 genes or so and 7 diverse phenotypes (Figure 4), likely driven by insulin signaling. 
\end{itemize}

Mapping robust trans-associations via cross-condition mediation analysis and validating trans-associations of trans-genes for GWAS SNPs (CCmed) [Yang and Lin Chen, 2019]
\begin{itemize}
	\item Multi-tissue mediation analysis (CCmed): first need to establish association of locus $L_i$ with cis-gene $C_i$. To do this, association test of $\alpha_C$ using all SNPs in $L_i$ (using variance component test). Results expressed as $F_{ik}$ for locus-cis-gene $i$ at tissue $k$. Next do the mediation analysis: estimating $C_i$ to $T_j$ effect $\beta_1$ while adjusting for $L_i$. The results are expressed as $Z_{ijk}$ for cis-trans pair $ij$ in tissue $k$. Then estimating probability of mediation for pair $ij$ in at least $K_1$ tissues:
	\begin{equation}
	P_{\text{med},ij} \geq Pr(\alpha_C \neq 0 \text{ in all K tissues}) \times Pr(\beta_1 \neq 0 \text{ in at least $K_1$ tissues})
	\end{equation} 
	The two probabilities are computed from Primo. 
	
	\item Finding trans-genes targeted by GWAS SNPs using mediation: one can use GWAS SNPs for $L_i$ above and find the trans-genes targeted by GWAS SNPs. The problem is that GWAS SNP may be in LD with a cis-eQTL of $C_i$, but $C_i$ is not the true target. Let $G_i$ be GWAS SNP at locus $i$, and $L_i$ the eQTL. Comparing with CCmed, the first step test association of $G_i$ with $C_i$, but adjusting for $L_i$. However if $L_i$ and $G_i$ in close LD, $r^2 > 0.5$, then not adjust for $L_i$. The second step also adjust for $L_i$. Finally, estimation of the probability of mediation is similar, but find the best configuration (over $K$ tissues). 
	
	\item Validating trans-genes using MR: MR-Robin. Model: $L_i \rightarrow X \rightarrow Y$. Let $\beta_{yi}$ and $\beta_{xi}$ be the effect of SNP $i$ on $X$ (expression) and $Y$ (GWAS), then:
	\begin{equation}
	\beta_{yi} = (\gamma+ \gamma_i) \beta_{xi}
	\end{equation}
	where $\gamma$ is the causal effect and $\gamma_i$ reflects the deviation due to LD and horizontal pleiotropy. Use a random effect for $\gamma_i$. It is difficult to do this for multiple tissues, so use reverse regression of tissue-specific eQTL vs. GWAS effects. For tissue $k$,  
	\begin{equation}
	\beta_{xik} = (\theta + \theta_{i}) \beta_{yi} + \epsilon_{xik}
	\end{equation}
	
	\item Results of CCmed-GWAS: in SCZ, use all SNPs in 108 loci, and eQTL from multiple brain tissues. Found 1400 genes. 
	
	\item Validation of trans-genes using cis-signals: (1) Cis-eQTLs of 1400 genes are enriched with low GWAS p-values. (2) The genes are enriched with PrediXcan predictions.  
	
	\item Validation of trans-genes by MR-Robin: 40 genes. 
	
	\item Remark: issues with CCmed include, the estimation of mediation probability is based on a conservative lower bound: it requires $L_i$ to be cis-eQTL of $C_i$ in all tissues. This limits the applicability of the method. Could use enumeration strategy like CCmed-GWAS. 
	
	\item Remark: possible issues with CCmed-GWAS. Not enough to address LD: in general, it requires $G_i$ to have independent effect beyond $L_i$. This reduces the power of detecting $G_i$ effects. Having a LD condition (don't adjust for $L_i$ if in close LD) helps, but it has the risk of failing to adjust for LD. In practice, a large number of SCZ genes found, so it's OK to be somewhat conservative. 
	
	\item Remark: MR-Robin. Assumptions are: same effect of gene on trait across all tissues (same $\theta$). A strong assumption. 
\end{itemize}

MR accounting for weak effects and pleiotropy using profile likelihood [Jingshu Wang, NHS, 2019]
\begin{itemize}
	\item Motivation: visual inspection of effect size correlation, often see over-dispersion, better explained through pleiotropic effects of SNPs. 
	
	\item Profile likelihood approach: let $\Gamma_j$ be the GWAS effect of variant $j$ to outcome, and $\gamma_j$ the effect to exposure, and $\alpha_j$ the pleiotropic effect, then:
	\begin{equation}
	\Gamma_j = \beta \gamma_j + \alpha_j
	\end{equation}
	We assume $\alpha_j \sim N(0, \tau^2)$. This allows one to have $P(D|\beta, \gamma_1, \cdots, \gamma_p, \tau^2)$. Profile likelihood: replace $\gamma_j$ in likelihood with its MLE, and still obtain unbiased estimator of $\beta$. 
	
	\item Weak IVs and selection bias: include variants with $\gamma_j = 0$ will reduce power, but does not bias the estimate. Need to use independent data to estimate effect size $\gamma_j$. Show the effects of selection bias: use BMI male $>$ BMI female, expect effect size 1, however, MR methods may obtain $<1$ estimate because of winner's curse (not account for standard errors of effects). Note: current methods may show $<1$ estimate even with $p < 1E-8$ selection of IVs - due to winners' curse. 
	
	\item Experiments that show the benefit of weak IVs: reduce CI of estimates. Show that with weak IVs, obtain similar estimates to strong IVs (with larger CI). 
	
	\item Detecting multiple modes in profile-likelihood function: e.g. reverse causality, mode at 0 and $1/\beta$. 
\end{itemize}

Phenome-scale causal network discovery with bidirectionalmediated Mendelian randomization [Brown and Knowles, review for NG, 2020]
\begin{itemize}
	\item Causal model: $D$ traits, and data of $N$ samples. Let $X$ be the genotype matrix of $N \times M$ dim. We have a causal graph $G$ (sparse), with $G_{ij}$ the direct causal effect of $Y_i$ on $Y_j$. Let $\beta$, $M \times D$ matrix be the direct effects of SNPs on traits, and $\gamma$ be unexplained effects: we have
	\begin{equation}
	Y = Y G + X \beta + \gamma
	\end{equation}
	Let $\hat{\beta}$ be the estimated genetic effects, $\hat{\beta} \propto X^T Y$. One can show that $\hat{\beta}$ is related to $G$ and $\beta$ by: 
	\begin{equation}
	Y(I-G) = X \beta \Rightarrow X^T Y = X^T X \beta (I-G)^{-1} \Rightarrow (X^TX)^{-1} X^T Y = \beta (I-G)^{-1} \Rightarrow \E(\hat{\beta})  = \beta (I-G)^{-1}
	\end{equation}
	If we denote $\hat{R}$ be the estimated total effect of one trait on another - estimates from 2SLS MR, then we can also show how $\hat{R}$ is related to $G$:
	\begin{equation}
	\hat{R} = \frac{1}{N} (X\hat{\beta})^T Y = \frac{1}{N} (X\hat{\beta})^T YG + \frac{1}{N} (X\hat{\beta})^T X \beta + \frac{1}{N} (X\hat{\beta})^T \gamma  
	\end{equation}
	Taking expectation on both sides. The first in the RHS is $\E(\hat{R}) G$. Claims that the second term is diagonal, and the last term has mean 0. This leaves:
	\begin{equation}
	\E(\hat{R}) = \E(\hat{R}) G + D[\beta (I-G)^{-1}]
	\end{equation}
	where $D$ is diagonal matrix. This leads to the equation for estimating $G$: $G = I - R^{-1} D[1/R^{-1}]$.
	
	\item Analysis: is $(X\hat{\beta})^T X \beta$ diagonal? We note that $ X \hat{\beta}$ is the PRS of traits, so we denote as $\hat{Y}$. We are interested in if $(X \beta)_i$ for trait $i$ $\hat{Y}_j$ for trait $j$ are independent. Diagonal means that if SNP $k$ has a direct effect on $Y_i$, i.e. $\beta_{ki} \neq 0$, then SNP $k$ will not contribute to the PRS of another trait $Y_j$. This seems to be what the model assumes. However, this is not true because $\hat{\beta}$ captures the total effect: so if there is a causal path from $Y_i$ to $Y_j$, then SNP $k$ will have an effect on $Y_j$.
	
	\item Estimation of $R$: (1) variant weighting - similar to Steiger filtering. Then use Eggar regression to estimate $R$. For trait $i$ and $j$: first determine the putative IVs, all SNPs associated with $i$ at a given threshold. Then for these SNPs, compare their effect sizes for $i$ and $j$ using Welch test. The weight is determined by the test statistic: SNPs with larger effect difference will have higher weights. (2) Use two sets of statistics: one set to do SNP selection, and the other for estimating effects. In UKBB, use male and female associations. 
	
	\item Estimation of $G$ by sparse inverse regression (inspre): the goal is to estimate an approximate $R^{-1}$, which can be plugged in to the estimation equation for $G$. To estimate $R^{-1}$, we find $U$, and $V$ s.t. $UV = I$, $U$ is close to $\hat{R}$ and $V$ is sparse. This is done by solving the optimization problem in Equation (9).  
	
	\item Simulation using pairs of traits: three settings, no pleiotropy, independent pleiotropy and correlated pleiotropy, among 5000 causal SNPs, 1000 are pleiotropic, with $\rho_g = 0.2$. Show that the weighted Eggar regression is better than Eggar regression: both FP and power. The FP rate at setting 3 seems relatively low, 0.085 at targeted type I error rate 0.05. 
	
	\item Simulation of causal graph: random graph, graphs with hubs, and graphs with high in-degrees. F1 scores of bimmer much higher than elasic net on Eggar. 
	
	\item UKBB results summary: 400 traits. Found 8K pairs from the matrix $R$, at FDR 0.05. Then using the sparse graph (inspre): limit to 2K pairs. 
	
	\item Some examples of UKBB results: some causal effects can be explained by mediations. Ex. Time spent on TV (TSWT) has causal effect on other traits, walking pace, wheezing in the chest, father's death age. These can be explained by TSWT on BMI. Another example: age of first sex (AFSI) on knee pain, wheezing in the chest, etc. which can be explained by BMI. Possible mechanism: age of puberty. 
		
	\item Heart disease sub-network. Effect of WBC on heart disease via direct effect, cholesteroal and BP. Note: found BP $\rightarrow$ cholesterol.   
	
	\item Hemoglobin sub-network: Hb effect on bleed gum and cardic arrhythmia. For bleeding gum: likely by platlet. For the latter, likely anemia. 
 
	\item The highest trait by out-degree: WBC. Effect on neuroticism, suffer for nerves, anxiety. Remark: likely by stress/cortisone, a confounder.  
	 
	\item BMI is the second highest trait by out-degree: BMI found to have an effect on vegetable intake, and other dietary behavior. 
	
	\item Q: in weighted Eggar regression, the weights can be negative? Presumably truncated at 0. 
\end{itemize}

Simultaneous estimation of bi-directional causal effects and heritable confounding from GWAS summary statistics [Darrous and Kutalik, review for NC, 2020]
\begin{itemize}
	\item Model notations: effects of $G$ are denoted as $\gamma_x, \gamma_y, \gamma_u$. Causal effects: $\alpha_{x \rightarrow y}$ and $\alpha_{y \rightarrow x}$. Confounder effects: $q_x, q_y$. Observed summary statistics $\hat{\beta}_x, \hat{\beta}_y$, with variant index $k$.
	
	\item Prior distributions for genetic effects: proportion of non-zeroeffects denoted as $\lambda_x, \lambda_y, \lambda_u$. However, because of LD, the proportions are higher, denoted as $\pi_x, \pi_y, \pi_u$. 
	
	\item Model of true effect sizes: assume the genetic effects are given. Let $\rho_k$ be the LD of SNP $k$ with all other SNPs (vector), and $\gamma_u, \gamma_x, \gamma_y$ be genetic effects of all SNPs (vectors). Then the true effect $\beta_{k}^y$ would have three parts: from variants acting on $U$, on $X$ and on $Y$ directly:
	\begin{equation}
	\beta_{k}^y = (\alpha_{x \rightarrow y} q_x + q_y) (\rho_k \cdot \gamma_u) + \alpha_{x \rightarrow y} (\rho_k \cdot \gamma_x) + (\rho_k \cdot \gamma_y)
	\end{equation}
	Similarly, we can obtain the model of $\beta_{k}^x$, where it has contribution from reverse causal effects. 
	
	\item Likelihood model of summary statistics: marginalizing genetic effects. The variance of genetic effect now has LD score of a variant in it. And the joint distribution of the $\hat{\beta_x}$ and $\hat{\beta_y}$ now follows mixture of 8 normal distributions. The 8 components correspond to how a variants is associated with 3 traits. The terms of the covariance matrix captures the variance and correlation of genetic effects.
	
	\item Remark: treatment of LD. The model considers only marginal statistics, but the effect sizes would be correlated of SNPs in LD, so it is still not a model fully account for LD. 
	
	\item Inference: MLE of model parameters. Reparameterize some parameters, $t_x = h_u q_x, t_y = h_u q_y$ as scaled confoudner effects. Also $i_x, i_y$ similar to LD score intercept term. Limit the parameter ranges $t_x$ and h2g in [0,1], $t_y, \alpha$ are [-1,1].   
	
	\item Identifiability analysis. What matters is the ratio of effects size: $t_y/t_x$ under confounder effect and $\alpha_{x \rightarrow y}$ under causal effect. Under the model, expect two lines (Fig. S2), one from causal effect, and the other from confounder. When confounding is severe (large effect, or very heritable), or when two confounders with opposite signs: bimodal likelihood.
	
	\item Obtaining SE of the parameters: use MLE or jackknife. In simulations, jackknife is slightly more conservative. In real data: bimodal likelihood occurs more often, and jackknife leads to large SE, so use LRT instead.  
	
	\item Remark: comparison with CAUSE. The value of $q$ is roughly $\pi_u / (\pi_u + \pi_x)$, the proportion of all variants of $X$ that acts through $U$. 
	
	\item Remark: the model has identifiability problem under broad scenarios: (1) Simple case, causal effect, no confounder. This leads to full correlation of effect sizes. This can be explained by a causal effect, or a fully shared confounder, $q = 1$. (2) More generally: a causal effect and a confounder with opposite effect (proportion $q$). This leads to the effect of $Y$ be close to 0 for variants acting on the confounder. We could explain this pattern by: no causal effect, and a confounder effect with proportion $1 - q$.   
	
	\item Results: see review file. Main findings: lower SE of parameters comparing with standard MR. General agreement with standard MR in real data, difference largely due to the use of genomewide markers by LHC.
\end{itemize}

%%%%%%%%%%%%%%%%%%%%%%%%%%%%%%%%%%%%%%%%%%%%%%%%%%%%%%%%%%%%
%%%%%%%%%%%%%%%%%%%%%%%%%%%%%%%%%%%%%%%%%%%%%%%%%%%%%%%%%%%%
\chapter{Epigenetics}
\section{Overview of Epigenetics}

Epigenetics flipping the genetic switch [Neil Lamb]
\url{http://hudsonalpha.org/wp-content/uploads/2014/04/epigenetics.pdf}
\begin{itemize}
	\item Figure 1. mechanism. Agouti gene: normally ON briefly. But if consitintutely on, agouti mouse. This phenotype is caused by demethylation of the promoter of the Agouti gene. 
	
	\item Figure 2. transgenerational inheritance. Agouti phenotype passed to next generation. Dietary supplement with vitamin (methyl groups) during pregenency and nursing in Aguoti mice: offsprings show normal phenotype.  
\end{itemize}

A New Kind of Inheritance [Skinner, Scientific American, Aug, 2014]
\begin{itemize}
	\item Epigenetic actors: DNA methylation (28M sites), histone modification, ncRNA (interacting with epigenetic marks). Epigenetic marks get copied during replication. 
	
	\item Epigenetic inheritance from pesticide: endocrine disruptors (pesticide) on pregnant rats, the offsprings have smaller testeis and few sperms. Passes to next generation: more than 90\% show abnormalities. Evidence of epimutation that interfere with gonad development in male embryos. 
	
	\item Epigenetic inheritance from famine: using data of 300 people born in 1890, 1905, 1920, women whose paternal grandmothers experienced one of these feast-famine swings as young children had markedly higher rates of fatal cardiovascular disease. Similar observations have been made in descendants of a Dutch population that experienced famine during World War II.
	
	\item Mechanism of epigenetic inheritance: (1) First wave of reprogramming, removal of all methyl marks except imprinted gene, after conception. (2) Second wave during primordial germ cell development: essentially complete, including imprinted genes. Epimutation may happen in (2) so that they are protected, similar to imprinting. 

\end{itemize}

%%%%%%%%%%%%%%%%%%%%%%%%%%%%%%%%%%%%%%%%%%%%%%%%%%%%%%%%%%%%
\section{Imprinting and Maternal Effect}

Background [Fangyuan Zhang talk, Jan, 2015]
\begin{itemize}
	\item Imprinting: About 1\% of human genome is imprinted, only 90 genes have been detected.
	
	\item Maternal effect: the effect of a variant depends on the genotype of the mother. 
	
	\item Experimental techniques: for imprinting, can use mice to study (control mating); for maternal effects, use assisted reproduction that use surrogate mothers. 
\end{itemize}

Methods for Detection of Parent-of-Origin Effects in Genetic Studies of Case-Parents Triads [Weinberg, AJHG, 1999]
\begin{itemize}
	\item Motivation: suppose we have a disease susceptibility locus, where $a$ is the risk allele. The effect of $a$ may depend on its parent of origin: e.g. it may increase the risk only when it is inherited from father, this is called imprinting. Meanwhile, there may be maternal effect (prenatal effect). 
	
	\item Intuition of detecting imprinting: suppose we have children of genotype $A/a$, where $A$ is from mother and $a$ father, and of genotype $a/A$. If there is imprinting, the risk is different in the two genotypes (or the fraction of cases). The problem now is that we cannot directly test the different risks since we are conditioned on the affected children. 
	
	\item Extension of TDT - transmission assymetry: in TDT, the distorion of transmitted vs. non-transmitted measures the RR of the disease allele. So we simply stratify the data by parent-of-origin of the alleles. Then we have a 2 by 2 table: T vs. DT where T is from father or from mother, and the differential distortion can be tested by Fisher's exact test. This is called TDT$_{\text{MvsF}}$. The test however is not strictly valid when both parents are heterozygous. To deal with that, we remove these parents, and the test is called ``transmission asymmetry test (TAT)''. 
	
	\item Model of trios incorporting imprinting and maternal effect: we only consider the most informative cases where one of the parent is heterozygous and the other homozygous. There are four scenariors (similar to TAT): we write down the genotype combinations (mother then father), and the relative risk of the child.  
	\begin{itemize}
		\item AA $\times$ Aa $\rightarrow$ A/A (010): $1$
		\item AA $\times$ Aa $\rightarrow$ A/a (011): $R_p$
		\item Aa $\times$ AA $\rightarrow$ A/A (100): $S_1$
		\item Aa $\times$ AA $\rightarrow$ a/A (101): $I_m R_p S_1$
	\end{itemize}
	where $R_p$ is the RR of $a$ in father, and $I_m$ is the imprinting effect, $S_1$ is the maternal effect.  
	
	\item Parental Assymetry Test (PAT) and Parent-of-origin LRT (PO-LRT): the most informative statistic is given a certain mating type, the relative ratio of $A/a$ vs. $a/A$ in the child. In LRT, we consider the relative frequency of the two as $P(M>F|\text{mating type},C) / P(MCF|\text{mating type},C)$, which depend on the parameter $I_m$, and we test using LRT (PO-LRT). A simple approarch is: suppose there is no maternal effect, then given the mating type combination, under $H_0: I_m = 1$, the relative ratio of $A/a$ is 0.5. And we test if the counts of $A/a$ and $a/A$ are equal - this is PAT.  
	
	\item Comparison of tests: 
	\begin{itemize}
		\item Both TAT and PAT are valid only when there is no maternal effect. 
		\item PAT is more powerful than TAT. 
		\item LRT is more robust, but significantly loses power when there is no maternal effect. 
	\end{itemize}
\end{itemize}

Joint detection of association, imprinting and maternal effects using all children and their parents: LIME [Han \& Lin, EJHG, 2013]
\begin{itemize}
\item Background: 
\begin{itemize}
	\item Different designs such as case-parent trios, and may include multiple children (including unaffected), and general pedigrees.
	\item Different tests: (1) Nonparametric test: assumption of no maternal effect. (2) Parameteric test: stringent assumptions such as mating symmetry and parental allelic exchangeability (e.g. PAT is valid only when two assumptions hold).  
\end{itemize}

\item Motivation: testing both maternal effect and imprinting using likelihood; incorporate additional siblings. The study uses both case famlies and control families.  
	
\item Partial likelihood model: e.g. consider a family of two children, where one of them is affected. Let $M,F,C_1,C_2$ be the genotypes, and $D_1=1, D_2$ be the disease trait of the children. We are interested in the condtional prob.
\begin{equation}
P(M,F,C_1, C_2, D_2|D_1) = P(M,F,C_1,D_1) P(C_2|M,F) P(D_2|M,F,C_2)/P(D_1)
\end{equation} 
This likelihood can be factorized s.t. only part of them is dependent on the imprinting parameters, and the other nuisance parameters. So our inference can be based on the partial likelihood containing imprinting effect. 

\item Intuition: the main nuisance parameters are frequencies of genotypes (mating types). To avoid them in testing, we use the idea similar to two sample Poisson test: given a mating type, we compare the frequency of cases vs controls, the number of cases conditioned on the total number of events follows Binomial distribution whose parameter depends only on the relative risk parameters, but not genotype frequencies. 
\end{itemize}

Identifying Heterogeneous Transgenerational DNA Methylation Sites via Clustering in Beta Regression [Shengtong Han talk, 2014]
\begin{itemize}
\item Problem: given the methylation data of many CpG sites in $n$ trios, we want to identify different inheritance patterns: e.g. some sites are average of parents, some are new methylation sites, some follow from one of the parent (imprinting). The problem is to find such patterns in an unsupervised fashion. 

\item Idea: we care about the inheritance patterns, not the absolute levels, so we model the relation of offspring and parents (using a linear model), then cluster the sites by their relations (coefficients). One issue is that statistically, we shouldn't simply average methylation levels of multiple observations of one site (not normally distributed); so instead we model the distribution of methylation level. 

\item Model: let $O_j, M_j, F_j$ be the average methylation levels of offspring, mother and father at the $j$-th site, they are related by: 
\begin{equation}
O_j = \gamma_{0j} + \gamma_{1j} M_j + \gamma_{2j} F_j
\end{equation}
Then the cofficients form clusters. To simplify, instead of creating a model for $\gamma_j$'s, we simply assume there are $K$ clusters, and each cluster has a set of values of $\gamma$'s. To refine the model, instead of averaging observed methylation level, we directly model the methylation data. Let $y_{ij}$ be the observed methyl. level of offspring $i$ in the $j$-th site, and $Z_{1ij}, Z_{2ij}$ be that of parents $i$. We model them as Beta distribution: 
\begin{equation}
y_{ij} \sim \text{Beta}(\alpha_j^o, \beta_j^o), \quad Z_{1ij} \sim \text{Beta}(\alpha_j^M, \beta_j^M), \quad Z_{2ij} \sim \text{Beta}(\alpha_j^F, \beta_j^F)
\end{equation} 
The average methylation level $O_j, M_j, F_j$ are thus simple functions of $\alpha$'s and $\beta$'s. The model is fit by EM. 

\item Remark: it may make more sense to model data of individual families, in other words, we directly model $y_{ij}$ as function of $\gamma_{0j}$, $\gamma_{1j}$, $\gamma_{2j}$ and $Z_{1ij}$, $Z_{2ij}$ ($\gamma$'s are defined wrt. indiviudal families, not average). 

\item Lesson: clustering based on similar relationships among variables. 
\end{itemize}

%%%%%%%%%%%%%%%%%%%%%%%%%%%%%%%%%%%%%%%%%%%%%%%%%%%%%%%%%%%%
\section{Epigenetics in Human Diseases}

Epigenome-wide association studies for common human diseases [Rakyan, NRG, 2011]:
\begin{itemize}
	\item Goal: for any human complex disease, we remain unaware of the proportion of phenotypic variation that is attributable to inter-individual epigenomic variation. This problem can only be elucidated by large-scale, epigenome-wide association studies (EWASs). 
	
	\item Epigenetic information can be transtmitted via: DNA methylation, hmC, histone modification, ncRNA (miRNA, piRNA, lncRNA).
	
	\item Types of DNAm variations: 
	\begin{itemize}
		\item Methylation variable position (MVP). A CpG site that shows differential methylation between different disease states.
		\item Differentially methylated region (DMR). A region of the genome at which multiple adjacent CpG sites show differential methylation. they are typically $<1$ kb, but they can exceed 1 Mb.
		\item Variably methylated region (VMR). These regions are defined by increased variability rather than gain or loss of DNAm.
		\item Allele-specific methylation (ASM). These are positions or regions that vary in DNAm depending on the parent-of-origin, the presence of a polymorphism or as a result of a stochastic event
	\end{itemize}
	
	\item Evidence of epigenetic compoent in complex disease: 
	\begin{itemize}
		\item Monozygotic twin concordance for any complex disease is almost never 100\%. 
		\item The incidence of several complex diseases - such as T1D - rising in the general population and is frequently altered in migrant populations, suggesting a role for non-genetic factors. 
		\item Epidemiological evidence suggests that a suboptimal in utero or early childhood environment can have an impact on disease outcomes (such as type 2 diabetes) in adulthood. 
		\item In cancer: gain of methylation in CGI, loss-of-imprinting, loss of DNAm at repeat, esp. satellite DNA (main structural component of heterochromatin).
	\end{itemize}
	
	\item Causality problem: epigenetic variation can be causal for disease or can arise as a consequence of disease. 
	\begin{itemize}
		\item A key step towards achieving this goal is to determine whether the variation is present prior to any overt signs of disease. However, this does not guarantee causality.  
		\item It is also possible that the underlying genotype influences epigenetic variation, e.g. from methQTL stuides. Some evidence of trans-meQTL, but not as prevalent as cis-effects. 
	\end{itemize}
	
	\item Sources of DNAm variation: (1) inherited from parents (transgenerational inheritance); (2) environmentally-induced, including in utero effect: developmental reprogramming; (3) stochastic, could be present in many tissue if early in development; (4) genetic variation. Disease state could affect DNAm (source 2): e.g. changes of immune cell DNAm from autoimmune diseases.
	
	\item EWAS study designs: 
	\begin{itemize}
		\item Retrospective (case-control). However, a retrospective study cannot determine whether the identified epigenetic variants are due to disease-associated genetic differences, post-disease processes. 
		
		\item Parent-offspring pairs. These could be useful in EWASs that aim to identify transgenerational transmission of epigenetic marks. The genetic information could then be used to eliminate the possibility that genetic modifiers are causing the epigenetic variation. 
		
		\item Monozygotic twins. Monozygotic twins who are discordant for a disease of interest represent a useful resource for EWASs. However, these studies cannot be used to distinguish between cause and consequence.
		
		\item Longitudinal cohorts: can be invaluable for establishing the temporal origins and stability of disease-associated epigenetic variation, thereby helping to distinguish causal epigenetic variants from consequential ones.
		
		\item Example design: Start with genome-wide DNAm analysis of monozygotic twins who are discordant for the disease to identify disease-associated MVPs in immune-effector cells. Then take these MVPs and assay them in the same type of immune-effector cells from a prospective cohort to look at DNAm at these sites in unrelated individuals who were sampled both before and after disease onset. Any MVPs that can be validated prior to disease onset are then candidate causal variations. 
	\end{itemize}
	
	\item Environmental effects: unlike GWASs, environmental factors can also directly confound an EWAS by affecting both epigenotype and phenotype. Indeed, if GWAS data are also available on the EWAS individuals, it may be appropriate to adjust for leading principle coordinates of both genetic and epigenetic states. 
	
\end{itemize}

Epigenome-wide Association Studies and the Interpretation of Disease-Omics [Birney, PLG, 2016]
\begin{itemize}
	\item Specific challenges of EWAS:
	\begin{itemize}
		\item Cell subtype heterogeneity. Even present in ‘’purified’’ cell types (subtypes exist).
		\item Cellular mosaicism: typically DNAm in most CpGs are 0 or 100\%, so proportional changes in EWAS represent change of proportions.
		\item DNAm variation can result from transcriptional variation?
		\item Genetic variation between individuals: powerful influence, about 20-80\% of DNAm variability. Typically in EWAS, not do things like population stratification.
	\end{itemize}
	
	\item Advices on EWAS: (1) longitudal cohort if possible. (2) Address cellular heterogeneity: pure cell types, better methods (e.g. CellMix). (3) Correcting for transcriptional and genetic difference.
\end{itemize}

Epigenome-wide association study of body mass index and the adverse outcomes of aiposity [Nature, 2017]
\begin{itemize}
	\item Background: methylation array: based on BiS conversion, or each CpG site has two bead types, recognizing different methylation status. 
	\item Method: causal vs. consequential analysis of relationship among SNP, CpG and BMI. Causal model: SNP $\rightarrow$ CpG $\rightarrow$ BMI, use the strongest cis-SNP of CpG as IV, and the predicted SNP to BMI effect is the product of SNP to CpG effect and CpG-BMI correlation. Consequential model: similar, but use the polygenic score of BMI as IV. 
	
	\item Data: EWAS on 5K samples. DNA methylation in blood (450K array). EWAS: 187 significant CpG loci associated with BMI. Covariates in EWAS: technical factors, SNP PCs, methylation PCs. 
	
	\item Analysis of epigenetic heterogeneity across cell types: methylation in isolated CD4 and CD8 T-cells. Compare methylation levels of 187 loci in blood vs. fat, liver, muscle etc (21 tissues): $R = 0.37-0.93$. 
	
	\item Causal and consequential analysis (Figure 2): most CpG sites are consequence of BMI, rather than the causes. Likely explanation: changes in lipid and glucose metabolism associated with BMI. 
	
	\item Methylation as marks/predictors of T2D: independent of BMI. 
	
	\item \textbf{Lesson}: DNA methylation of strong loci (associated with trait) may not be highly cell-type specific. However, most of them may not be causal loci. 
\end{itemize}
%%%%%%%%%%%%%%%%%%%%%%%%%%%%%%%%%%%%%%%%%%%%%%%%%%%%%%%%%%%%
%%%%%%%%%%%%%%%%%%%%%%%%%%%%%%%%%%%%%%%%%%%%%%%%%%%%%%%%%%%%
\chapter{Systems Genetics}
\section{Methods for Molecular-QTL Analysis}

Unique problems/opportunites of eQTL analysis: 
\begin{itemize}
	\item A large number of traits are analyzed simultaneously: this raises challenges of analysis (multiple testing correction); also makes it possible to develop new methods that exploits the correlation among traits. 
	\item Mechanisms of eQTL: easier to study than complex traits. What the studies reveal about gene regulation. 
	\item Road to phenotype: natural bridge between genotypes and compelx phenotypes. 
\end{itemize}

Methods for detecting eQTLs [Personal notes]: 
\begin{itemize}
	\item Experiment design: linkage mapping in families or from crosses between two parental strains; association mapping using samples from unrelated individuals in a population. Association mapping has lower statistical power, but will be the method of choice in the future because of more genetic variations. 
	
	\item Correction of confounders: the challenge is latent confounders for expression data: PCA, SVA, PEER. Also may need to correct for population ancestry (PC).   
	
	\item eQTL method: the general idea is: divide the samples according to the marker alleles and test if the groups differ signficantly in expression. Could be enhanced by modeling genetic interactions among multiple loci. Common methods: linear model, non-parametric test (e.g. Spearman correlation). 
	
	\item Dimensionality reduction: combine multiple transcripts that behave similarly into single traits. 
	
	\item Meta-analysis: specific eQTLs are not generally replicated across studies. In model organisms, this may be due to the difference of strains used in different studies. In human populations, this could be due to various artifacts such as: the use of different $p$-value cutoffs, of different distance threshold of defining distal eQTLs. A meta-analysis using data from multiple studies, using a single consistent method is necessary for meaningful comparison. 
\end{itemize}

Significance and multiple testing correction in QTL studies [personal notes]:
\begin{itemize}
	\item Background: BH correction vs. Storey's q-value. The former is more conservative since it implicitly assumes that $\pi = 0$, while $q$-value method estimate $\pi_0$ from data (though the estimate is conservative). 
	
	\item Background: is FDR correction valid when the tests are correlated? On average the FDR is correct, but at a specific study, FDR may be over- or under-estimated. If correlations are only local (ie. a test is correlated with only a small subset of tests), then FDR is OK. 
	
	\item Challenge: why cannot we pool results of all genes? If we pool all tests, and do single multiple testing correction, we may not be able to control FDR for each phenotypes. Ex. one gene has many associated SNPs (tight LD): adding this signal gene will introduce many strongly associated SNPs. This leads to inflation of false positives in other genes because FDR measures global false discoveries. 
	
	\item The importance of adjusting for each gene: if we know the effective number of independent tests per gene, we should adjust for significant using different thresholds for different genes. 
	
	\item Calibration problem: FDR correction methods and Beta approximation (FastQTL) all require p-values to be calibrated. If not, need to calibrate e.g. by permutations. This happens when we test top SNPs, and min-p is not uniform, so we use permutation to get empirical p-values for min-p. 
		
	\item Permutation-based strategy to control QTL discovery by eGenes: a general strategy. We obtain min-p (lead SNP) per gene; then control FDR at the level of all genes (but each gene contributes only one SNP). Typically, the min-p is not calibrated, so performe permutations to obtain null distribution of min-p for each gene. This adjusts for local LD. 

	\item Adjusting for covariates in permutation: suppose we have data $G_i$ (genotype) and $Y_i$ (phenotype). We should keep $Y_i$, while permuting $G_i$, i.e. replace $G_i$ by $G_j$ for some $j$. When our causal diagram is $Z \rightarrow Y$, but no arrow from $Z$ to $X$, then this permutation preserves the $Z$ to $Y$ effect. However, when $Z$ is a confounder, is this permutation effective? 
	\begin{itemize}
		\item Remark: in general, permutation of a variable will destroy all the edges in the causal diagram of that variable. 
	\end{itemize} 
	
	\item Analysis: when pooling is effective? If each gene has a similar number of tests, or we have already adjusted for the difference across genes, then it's OK to pool. Examples in literature: 
	\begin{itemize}
		\item McVicker, Degner: tests are mostly local, so a gene/peak has a small number of tests. McVicker: only 2kb around each peak. 
		\item Rasqual and Battle et al: adjust for the number of SNPs/tests per gene, by first performing Bonferroni correction for top SNP at each gene, then FDR on the p-values of top SNPs of all genes. 
	\end{itemize}
		
	\item Power problem: different genes/tests may have different power. Using a uniform cutoff thus may not be optimal. [Degner, Nature, 2012] do FDR correction, using different thresholds, for different bins of peaks. 
\end{itemize}

How to detect QTL in the presence of hidden confounders? [M6A QTL meeting with M. Stephens, 2018]
\begin{itemize}
	\item PCA: R has several versions of PCA. They may be different in normalizing/standardizing variables. E.g. prcomp() center, but does not standardize the variables. In general, recommend to standardize s.t. the variables contribute equally to PCA (otherwise, variable with large variance will dominate the reconstruction error).
	
	\item Null distribution in regression analysis (Beta-Binomial regression in our case): typically use asymptotical distribution to get p-values. This may break down, e.g. when the number of samples is not large relative to the degree of freedom of test (number of covariates in regression).
	
	\item How dependency changes null distribution? When the test statistics are correlated, it will change the null distribution: even though FDR is still correct on average, it can be inflated or deflated in a specific study. What matters is the “global pairwise correlation”. In the QTL case, even if $Y$’s are correlated, $X$’s are most not, so $Z$-scores are largely independent. The null distribution should be OK.
	
	\item Permutation: one way to obtain true null distribution. Or we can use it to test if this poses a problem.
	
	\item Quantile normalization: when the outcome in regression may not follow normal, and there are outliers, we can quantile normalize the outcome.
	
	\item Correct for confounders: PCA vs. PEER vs. SVA. PEER differs from PCA in that it has normal prior to shrink coefficients. SVA solves the problem, where the variable of interest correlates with the confounder(s), e.g. PCs. If we regress out PC(s), we remove some of the signal. So ideally, we will only regress out the part of PCs that are uncorrelated with the variable of interest.
	
	\item Stabilization: sometimes, we may not have a lot of data for the parameters (interested ones or nuisance). Stabilizing the estimate via hierarchical Bayes or ASH would help. Ex. in paired peak calling problem, the read count in controls may be low - we can stabilize the estimate of background rate, e.g. by using a spatially smoothed prior.
\end{itemize}
	
Model of QTL mapping with read counts [personal notes; WASP; RASQUAL]
\begin{itemize}
	\item Basic model: let $Y_{ij}$ be the read count of feature (expression, peak, etc.) $j$ of sample $i$, and $G_i$ the genotype, we have: 
	\begin{equation}
	Y_{ij} | G_i \sim NB(\lambda_{ij} K_i, \theta_j) \qquad \log \lambda_{ij} = \beta_0 + G_i \beta
 	\end{equation}
 	where $\lambda_{ij}$ is the relative mean (expression level expressed as fraction of reads), $K_i$ the library size (or size factor) of sample $i$, and $\theta_j$ the overdispersion parameter. 
 	
	\item Normalization: the expected read count depends also on other factors, e.g. GC content and IP efficiency of a sample. So we should correct for these factors by replacing $K_i$ with $K_{ij}$, which is the expected read counts in feature $j$ if there is no such difference: library size, GC content, IP efficiency, PCs.  
\end{itemize}

Research directions and statistical challenges [Gilad \& Pritchard, TIG08; Kendziorski \& Wang, Mamm Genome, 2006]; Goring, Tissue specificity of genetic regulation of gene expression, [NG, 2012]:  
\begin{itemize}
	\item Multiple hypothesis testing correction: especially important for eQTL studies. Ex. with multiple expression traits, often only the strongest eQTLs are considered, and the $p$-values are transformed to $q$ values for correction.
	\item Rare eQTL: sequence information, perhaps coupled with different study designs, such as those based on large families, will be required to detect rare eQTLs of strong effect.
	\item Mapping the functional eQTL sites. 
	\item Identifying eQTL hotspots: the simple method: count eQTLs. A better alternative is to sum the evidence of all transcripts (weighted) in a region, and this shows improvement over simple counting [Kendziorski \& Attie, Biometrics, 2006]. 
	\item Reconstruction of GRN: e.g. identify local eQTLs for genes that also mapped as distal eQTLs for other genes. Correlation of transcripts is used for identifying functional modules. 
	\item eQTL across different tissues: how are they shared. One strategy is to have large sample eQTL on a few accessible tissues and use the results as a reference for other tissues. 
	\item Use eQTLs for linking genotypes to complex diseases, and cell line phenotypes (e.g. sensitivity to chemotherapeutic agents). 
\end{itemize}

Reference: [Rockman \& Kruglyak, Nature, 2006], [Revealing the architecture of gene regulation: the promise of eQTL studies, Gilad \& Pritchard, TIG, 2008], [Mapping complex disease traits with global gene expression, Cookson \& Lathrop, NRG, 2009], [From expression QTLs to personalized transcriptomics, Montgomery \& Dermitzakis, NRG, 2011].  	 
 	  
High-Resolution Mapping of Expression-QTLs Yields Insight into Human Gene Regulation [Veyrieras \& Pritchard, PG, 2008]: 
\begin{itemize}
	\item Problem: find the causal variant (eQTN) of the expression traits.  
	\item Methods: 
	\begin{itemize}
		\item Data: 210 unreleated individuals from HapMap project (immortalied B cells). A core dataset of 11,446 genes. The analysis of eQTLs is limited to cis-eQTLs, defined as 500 kb upstream or 500 kb downstream of the genes.  
		\item Method: hierarchical model: let $Z_{jk}$ be the indicator variable of whether SNP $j$ is an eQTN of the gene $k$, the prior of $Z_{jk}$ is modeled as logistic regression of the functional annotations/features of SNP $j$: its distance to TSS or TES, its conservation, etc. The expression of individual genes is given by: 
		\begin{equation}
			P(E_k) = \sum_j P(Z_{jk} = 1) P(E_k | Z_{jk} = 1)	
		\end{equation}
		The coefficient of the second term is expression-trait specific and the regression coefficients of the first term (prior) are shared by all genes. 
	\end{itemize}
	
	\item Results: 
	\begin{itemize}
		\item Distribution of eQTNs: strongly enriched near TSS and TES. Most of the background signals in simple eQTL analysis (linear regression) were removed. And more eQTNs were found: 1586 vs 744 (from simple analysis), from higher sensitivity of the hierarchical model to signals in locations that are likely a priori. 
		\item Functional annotation of eQTNs: the TSS and TES peaks tend to be highly conserved across mammals. Internal introns have a deficit of eQTNs compared to exons and the first introns. 
	\end{itemize}
	
\end{itemize} 	 
 	  	 
Epistasis in yeast eQTL data [Hannum \& Ideker, PG, 2009]: 
\begin{itemize}
\item Motivation: in eQTL studies, one could identify genetic interactions among markers, what are the interpretations of such interactions? Are they enriched with PPIs? 

\item Methods: 
\begin{itemize}
	\item Data: [Brem05] 
	\item Natural genetic interaction network: first identify markers that genetically interact [Storey, PB, 2005]; then bicluster markers: an exhaustive genome-wide scan is performed to identify interacting interval pairs, i.e. those that are enriched for marker-marker interactions. 
\end{itemize}

Results: 
\begin{itemize}
	\item A network of 2,023 interactions between 1,977 genomic intervals. 
	\item Interacting intervals are enriched with protein complexes. 
\end{itemize}
\end{itemize}
 	  	

Using probabilistic estimation of expression residuals (PEER) to obtain increased power and interpretability of gene expression analyses [Stegle \& Durbin, Nature Protocol, 2012]
\begin{itemize}
	\item PCA: factors that explain total variation of expression. Compliexity is controlled by specificying the number of PCs. 
	
	\item PCASig: similar to PCA, but the selection of PCs is determined by significance analysis. 

	\item Surrogate variable analysis (SVA): accounting for fixed effect, correction for latent factors, using orthogonal vectors. Similar to PCAsig, but allow sparse non-orthogonal components. 

	\item VBQTL (Stegle, PLCB, 2010): ARD prior to provide shrinkage. Interpretation of hidden factors: eg. cell growth (highly correlated, $r^2 = 0.96$). The global factors identified can be further analysed for biological signals, looking for GO term over-representation in the genes that they affect.
	
	\item PEER: hidden confounders can be explained as TFs and hot-spots. Use of PEER: Figure 1. A-B, eQTL mapping correcting for hidden confounders. C. association with hidden factors. 
\end{itemize}
 	  	 
A statistical framework for joint eQTL analysis in multiple tissues [Flutre \& Stephens, PLG, 2012]
\begin{itemize}
\item Background: weighted Z score method for multi-tissue analysis - combine p-values of multiple tissues. It does not so easily allow for investigation of heterogeneity. Problem: heterogeneity means the effect may be different in different tissues, thus combine them using weighted-Z is not optimal. 

\item Model: given a SNP and gene pair, suppose there are $S$ tissues, let the effect size (linear model) in the $s$-th tissue be $\beta_s$. The problem is to define a prior on $\beta_s$ so that information can be shared across multiple tissues. Define $\gamma$, a binary vector, as the ``configuration'' of the SNP across $S$ tissues: 1 if active. Several models of the prior on configurations: 
\begin{itemize}
\item Default choice (BMA model): uniform prior on the number of active tissues (from 1 to $S$), then within each value, uniform prior on the configurations. 
\item BMA.lite model: only two configurations, a single active tissue or all 1's with weight 0.5 each. 
\item Hierarchical model (HM): over all genes, estimate the weight of each configuration by borrowing information across all genes.
\end{itemize}
The prior of the effect sizes given the configurations: obviously if $\gamma_s = 0$, then $\beta_s = 0$. In general, we have: 
\begin{equation}
\beta_s | \bar{\beta}, \gamma_s = 1 \sim N(\bar{\beta}, \phi^2)	
\end{equation}
Furthermore, a prior of mean effect size (across genes or SNPs?): $\bar{\beta} \sim N(0,\omega^2)$. 
\begin{itemize}
	\item Why the model works? The prior of $\gamma$ favors multi-tissue eQTL, in fact the prior of tissue-specific eQTL is only $1/S$. 
\end{itemize}

\item Statistical test: (1) Combining SNP information of a gene: for our Bayesian approach the test statistic is the average value of BFs over all SNPs in the cis candidate region of that gene; (2) Using BF as test statistic, and compute $p$-values by permutation (permutation of individual labels). 

\item Data: Dimas et al, Science, 2009, LCL, T-cells and fibroblast in 75 individuals. A subset of 5012 genes robustly expressed in all three cell-types, and cis-eSNPs only (within 1Mb of TSS)
Joint mapping increases power: at FDR $< 0.1$, 1321 genes vs. 811 genes (tissue-by-tissue analysis)

\item Application of hierarchical model: an estimated 88\% of eQTLs being common to all three tissues. Caution: the estimates necessarily reflect patterns of sharing only for moderately strong eQTLs (strong enough to be detected): patterns of sharing could be different among weaker eQTLs. 

\item Remark:
\begin{itemize}
\item RNA-seq data: count data, better to use a different model than normal. 
\item When $S$ is large, need a better model of prior. In particular, having a separate parameter for each possible configuration is unattractive (both statistically and computationally) for large $S$. Idea: use a tree model (similar to phylogenetic model of lineage-specific events); or an Ising model that favors related tissues. 
\item Questions: what are biological influences of general or tissue-specific eQTL? eQTL location, type of genes (housekeeping vs. specific)? 
\end{itemize}

\end{itemize}
 	  	 
WaveQTL [Heejung Shim \& Stephens, AoAS, 2014] 	
\begin{itemize}
\item Background: wavelet transform method. 
\begin{itemize}
\item Fourier transform: a function is decomposed into a sum of multiple basis functions (trigonometric). This leads to a compact representation: e.g. a function that looks complex may become a simple sum of a few basis functions. In practice, this often means removing noises in a function (the higher order terms). 
\item Wavelet transform: the idea of Fourier transform can be generalized. Instead of having basis functions of specific forms, we focus on certain ``properties'' of the function (mean, spatial asymmetry at different scales, etc.), and a function is represented as a set of properties. This can acheive similar goal of denoising by focusing on the top few components. 
\end{itemize}

\item Discrete Wavelet Transform (DWT) applied to spatial genomic data: suppose we have data $d = (d_1, \cdots, d_B)$ where $d_i$ is the value at the $i$-th position in a region of size $B$. We can extract these features from the data, such as: 
\begin{equation}
y_{01} = \sum_b d_b
\end{equation} 
the total count, 
\begin{equation}
y_{11} = \sum_{b \leq B/2} d_b - \sum_{b > B/2}d_b
\end{equation}
the difference of counts between the first and second halves; and similarly, $y_{21}$ and $y_{22}$, the difference between the first and second quarters; and the third and fourth quarters; and so on. So the vector $d$ can be transformed to a vector $y$ (linear transform), which capture the same data. The advantage is that $y$ has the denoising property s.t. we can focus on the first few components. 

\item Model of waveQTL: consider the sequencing data (depth of coverage) at a region, let it be $d$. We use DWT to get $y$. Let $y_{sl}$ be the value at the scale $s$ and location $l$. We perform association of genotype $g$ and $y_{sl}$ separately under a linear model - let $\gamma_{sl}$ be the underlying binary indicator. To increase the power, we combine information across all $y_{sl}$ of the same scale $s$ by assuming a model: 
\begin{equation}
P(\gamma_{sl}=1|\pi) = \pi_s
\end{equation} 
And we test if $\pi_s = 0, \forall s$ using LRT.  

\item Extenstion: multi-seq. The idea is similar, but we model the count data directly. Suppose the data is generated from multiple Poisson processes, let $p_{sl}$ be the rate of the process of scale $s$ and location $l$, and the observed count is the sum of the counts from all these processes. In other words, if we know the total count in a region, the difference of counts between the first and second halves, the difference of counts among quarters, and so on, we can recover the original count at each position. Then we perform the association analysis of genotype and $p_{sl}$.   

\item Question: the wavelet coefficients $y_{sl}$ are not independent across scales, for example, if $y_{sl}$ is large at some high resolution (e.g. at 3rd quarter vs. 4th quarter - a peak at 3rd quarter), then $y_{sl}$ is likely non-zero at a lower resolution (1st half vs. 2nd half, since there is a peak in the second half). Can the model combine information across multiple scales?  

\item Lesson: for spatial/functional data, extract features/properties that summarizes the data, and this can be achieved through Fourier transforms, wavelet transforms, etc. More generally, we may work only on the extracted features even if we cannot completely recover the data. 
\end{itemize}

Characterizing the genetic basis of transcriptome diversity through RNA-sequencing of 922 individuals [Battle et al, GR, 2014]: 
\begin{itemize}
	\item Expression quantification: use HTSeq for gene expression, BEDTools for exon expression and cufflinks for isoform expression. 
	
	\item Correction for latent confounders: HCP method [Mostafavi, 2013], correct for technical and biological factors, including blood cell-type frequencies and the time of the blood draw. Background: correct for hidden confounders greatly increase power of cis-eQTL. HCP: similar to factor analysis. 35 known confounders (Table S1): sequencing depth, cell type frequency, etc. 

	\item Testing eQTL: Spearman rank correlation. For cis-eQTL, only SNPs within 1Mb, and Bonferroni correction within a gene. eQTL using gene-level significance at FDR 0.05. 
\end{itemize}
	
GTEx [Science, 2015]: 
\begin{itemize}
	\item eQTL mapping: using Matrix eQTL. (1) Expression data: RPKM. Quantile normalization across genes in a given tissue. The expression values for each gene were transformed into a standard normal based on rank. (2) Correction of confounders: 15 PEER factors, gender, first three genotype PCs. (3) FDR control: based on Matrix eQTL, which uses BH corrections. Separate p-value thresholds and FDR calculation in cis and trans- analysis.  
	
	\item Identifying eGene (eQTL containing gene): use minP as test statistic for each gene. Then permutation: swap sample label and expression data. For each gene, then obtain empirical p-value (each gene has its own null distribution from permutation), and do FDR correction, using Storey approach. 
\end{itemize} 

Fast and efficient QTL mapper for thousands of molecular phenotypes (fastQTL) [Ongen and Delaneau, Bioinfo, 2016]
\begin{itemize}
	\item Background: permutation of a large number of times for each molecular phenotype. However, to get accurate statistical significance of the most associated QTLs may require a large number of permutations. 
	
	\item Direct permutation: permutation that leaves genotypes unchanged (preserving LD). See Equation (1), usually add pseudocount of 1 in calculation of p-values (otherwise, p-value may be equal to 0). Do this for each gene separately. 
	
	\item Adaptive permutation: for each gene, suppose the strongest p-value among all its SNPs is $p$. Permute a certain number of times s.t. at least $B$ (e.g. 100) null p-values are smaller than $p$. 
	
	\item Background: suppose we draw from uniform distribution $n$ times, then the $k$-th smallest value follows $U \sim \text{Beta}(k,n)$.  
	
	\item Beta approximation: model the null distribution of the smallest $p$-value in a gene as Beta distribution. In one phenotype, permute $R$ times, and let $p_i, 1 \leq i \leq R$ be the smallest p-value in the $i$-th permutation. We can then fit $\text{Beta}(k,n)$ to $p_i$'s. This allows us to obtain adjusted $p$-value for the min. $p$-value of a gene, and hence FDR control at the gene level. 
	
	\item Results: the parameter $k$ is close to 1 for most genes, and $n$ is between 1000-4000, a few times lower than the number of tested SNPs. 
	
	\item Beta approximation is much faster than direct permutation to get accurate p-values (Figure 2a): it requires 100-1000 permutations vs. 100K for direct permutations. 
	
	\item Remark: the Beta approximation approach controls FDR at the gene level (eGenes). 
\end{itemize}

Multi-tissue eQTL (MASH) [Sarah Urbut, May, 2015; Apr, 2017]
\begin{itemize}
	\item Motivation: in the published muti-tissue eQTL paper, the effects are discretized (on or off in a tissue). However, we want the effects to be contiuous: e.g. possible that an eQTL is active in two tissues, but effect sizes very different. 
	
	\item Idea: a prior covariance matrix of $\beta$ (effect size in each tissue), the covariance terms encode both the effect sizes and the correlation of effect sizes of an eQTL across tissues. Use a mixture model for the prior covariance: for some eQTL/gene, it is correlated in one set of tissues; for another eQTL/gene, correlated in another set of tissues. 
	
	\item Model: the effect sizes are standardized, i.e. $Z$-scores, defined as $Z = \beta / se(\beta)$. The likelihood is: 
	\begin{equation}
	\hat{\beta}_j | \beta_j \sim N(\beta_j, \hat{V}_j)
	\end{equation}
	where $\beta_j$ is the effect size of the $j$-th SNP-gene pair. The prior distribution:
	\begin{equation}
	\beta_j |\pi, U \sim \sum_{k,l} \pi_{k,l} N_k(0, \omega_l U_k)
	\end{equation}
	where $U_k$ is the $K$ components of the MVN and $\pi$ the mixture proportions. $\omega_l$ are the stretch factors (to scale effect sizes). We assume that $U$ and $\omega$ are pre-specified, and the problem is to estimate $\pi_{k,l}$. 
	\begin{itemize}
		\item Remark: $U_k$ represents both correlations and effect sizes. Ex. large effects in both tissues and (large, small) in the two tissues (both are independent) are represented by two different $U_k$'s. 
		
		\item Remark: (Dan's comments) the model is based on sharing of $Z$-scores, instead of actual effect sizes. But $Z$ scores are  affected by sample size (s.e. of $\beta$), which may differ substantially across studies. 
		
		\item Remark: we need $\omega_l$ because for each specific variant $j$, even if the effect size pattern is given, we still need to know/model the actual effect size. So $\omega_l$ captures the actual effect size for each SNP. 
	\end{itemize}
	
	\item Inference of $\pi$: use the fact that mixture of normal prior with normal likelihood leads to mixture of normal posterior. The reason of using fixed $U$ is that EM algorithm on mixture of normal does not work if the normal covariance matrix is different.  
	
	\item Specifying $\omega$ and $U$: we simply use a large number of $\omega_l$'s (the large $\omega_l$'s will be discarded by the data). For $U$, the idea is to approximate them using the observed covariance of effect sizes. Suppose we obtain the $t$-statistic (measured effect size) of SNP-gene pairs - for each gene, we choose the strongest SNP. Then the sample covariance matrix of the $t$-statistic is our starting point for $U$. Other choices of $U_k$ are obtained through Sparse Factor Analysis (SFA): do SFA on the covariance matrix: 
	\begin{equation}
	X = \Lambda F + E
	\end{equation} 
	where $X$ is the observed effect sizes, $F$ the factors ($K \times R$ matrix, where $R$ is the number of tissues), and $\Lambda$ the loading of $X$ on $F$. Choose the top $q$ latent factors in $U_k$: $[(\Lambda F)^T (\Lambda F)]_q$. In GTEx analysis, use dimension reduction to learn 6 different $U$'s, plus 44 rank-1 matrices (one for each tissue).  
	\begin{itemize}
		\item Remark: $U_k$ represents the covariance structure given by the $k$-th latent factor: it is covariance of the reconstructed $X$ using only the $k$-th factor. Note: use only the samples where the factor is non-zero? 
		
		\item Remark: factor analysis is based on the idea that effects can be deconvulated into sum of effects. How would this reconcile with mixture model (discrete structure)? Intuition: when latent variables are sparse, then for each sample, often only one latent variable is non-zero, so this reduces to mixture of normal. 
	\end{itemize}
	
	\item Opposite signs of effects in different tissues: observed in the data. However, this is likely due to SNPs in LD (two different SNPs have two different signs in two tissues). The evidence: two-SNP model provides a better fit of the data. 
	
	\item Results of applying the analysis to GTEx data: observe the cases where additional tissues may (or may not) change the estimated effect in one tissue. In the negative example, the effect in brain is not changed. Overall tissue similarlity: brain is different from the rest. 
	
	\item Effect sample size gain: relatively large from a few hundred or even dozen to $>1000$. Q: vary across SNPs/genes? Also loss of power for tissue-specific eQTL? 
	
	\item Remark: this is a generl statistical problem of inferring MVN. Need to specify the prior covariance matrix. 
	
	\item Remark: how does the sparse factor approach encodes subtle configurations, e.g. a small set of eQTL have effects in a particular configuration $(1,0,1,0)$ (suppose $R=4$)? The intuition is that the model will learn a sparse factor that is active only in the tissues 1 and 3; then this subset of eQTL have higher loading in this factor. 
	
	\item Lesson: factor model can explain the mixture scenario: if a subset of variables display certain covariance, then we create a factor that explains this covariance. 
	
	\item Questions:
	\begin{itemize}
		\item A simple strategy: for the prior of $\beta$, using binary configurations (could use mixed membership model so that only a small number of configurations will be actually used), and for each tissue, add a scaling parameter - e.g. a mixture of Gaussian. What's the disadvantage of this method? 
		
		\item Learning the covariance matrix: use the strongest SNP per gene. Does this create some kind of bias? Ex. when testing any gene-SNP pairs, most often the effect is small across all tissues. 
		
		\item Inference of $U$: sparse factor analysis, what assumptions? Ex. sparsity of loading. 
		
		\item The assumption of one eQTL per gene: how do we relax this assumption? 
	\end{itemize}
	
	\item Mash-Common-Baseline: Gene expression data across time series: time 0 vs. 1 to $t$. Let $C$ be true expression (vector across all conditions): assume $C$ follows MASH prior, and the error term is correlated (across conditions) because of shared control (time 0). Use MASH prior for true effects. Without correcting for shared control: much higher type I error. Model: let $C_j$ be expression of $j$ ($R$-dim. vector where $R$ is number of conditions), we have: 
	\begin{equation}
	\hat{C}_j | C_j \sim N(C_j, \hat{V})
	\end{equation}
	We consider the difference of expression vs. common control, denoted as $\delta_j = L C_j$, where $L$ is $(R-1) \times R$ matrix (subtracting $C_j$ for the common control). Then the observed differential expression, relative to common control, is:
	\begin{equation}
	\hat{\delta}_j = L \hat{C}_j = L C_j + LE = \delta_j + E^*
	\end{equation}
	where $E^* \sim N(0, L \hat{V} L^T)$. We then model $\delta_j$ using MASH. 
	
	\item Application of MASH to GWAS: 16 traits - summary statistics. Use top 1000 to initialize covariance matrices. Training: use 100K SNPs to learn the covariance matrices $U_k$. Then use MLE to estimate the parameters $\pi$ with 50,000 random SNPs. Found 300K associations. Q: independent samples. Q: LD leads to shared effects.
	
	\item Questions/Remark: how to apply it to eQTL data, joint mapping across genes.   
\end{itemize}

Flexible statistical methods for estimating and testing effects in genomic studies with multiple conditions (MASH) [Urbut and Stephens, NG, 2019]
\begin{itemize}
	\item Learning $U_k$ matrices: input is $J \times R$ matrix, where $J$ is independent SNP-gene pairs (best SNP per gene). Two sources of $U_k$'s: (1) Data driven covariance matrix: first obtain z-score matrix. Then use correlation matrix $Z^T Z$; PCA and SFA, and obtain low-rank (3-5) matrix approximations; add rank-1 matrices capturing the effect of a single factor. (2) Canonical correlation matrices: single effect, shared effect, etc. After obtaining initial $U_k$'s, fit a mixture model (similar to MASH), that refines the estimates of $U_k$'s. Note: $U$ needs to be standardized before fitting MASH.  
	
	\item Accounting for correlation due to sample overlap: for row $j$, $V_j = S_j C S_j$, where $S_j$ is the diagonal matrix of standard errors, and $C$ is the average correlation matrix of null SNPs. 
	
	\item Model fitting and posterior inference: only performed on the input matrix data. Estimation of $\pi$'s. Results are summarized: (1) lfsr for each row; (2) posterior effect estimates; (3) BF testing global null. 
	
	\item Application to GTEx data: 16K genes, for each gene, choose the top SNP, defined by maximum $Z$ scores across all tissues. 
	
	\item Pattern of sharing and visualization (Figure 3): to visualize $U_k$ (44 $\times$ 44), obtain the first few eigenvectors, which shows the loading to each tissue. In GTEx, the largest pattern (34\% weight) show shared effects, with stronger effect sharing in brain. 
	
	\item Examples (Figure 4): MASH can shrink the effect when there is little signal - shown by smaller posterior intervals. 
	
	\item Gain of power by MASH: comparison with MASH-bmalite (a simpler version models only effect sharing - Flute et al.) and ASH. ASH: 13\% with lfdr $< 0.05$ and MASH 47\%. 
	
	\item Sharing by sign and sharing by magnitude: sharing by sign is common, about 85\%; but sharing by magnitude (less than 2 fold difference of effects) is only 30\%. 
		
\end{itemize}

\subsection{Context-Specific eQTLs}

Determining the loci of responses (response-QTL) [personal notes]: 
\begin{itemize}
\item Problem: suppose we measure gene expression in two conditions, one untreated, the other some treatment. And we also have genotypes, we want to determine the loci that modify how cells respond to treatment (in terms of transcriptional change). 

\item Three strategies: 
\begin{itemize}
	\item Strategy 1: differential eQTL. eQTL in one condition, but not the other; or different effect sizes.  
	\item Strategy 2: response eQTL. Treat the change of expression as new trait, and do QTL. 
	\item Strategy 3: gene-environment interactions, where environment is the treatment. 
\end{itemize}

\item Equivalence of the three approaches. We start with differential eQTL: 
\begin{equation}
y_1 = \beta_1 G + \epsilon \quad y_2 = \beta_2 G + \epsilon
\end{equation}
So the difference $\Delta y = y_2 - y_1 = (\beta_2 - \beta_1)G + \epsilon$, a locus is differential eQTL ($\beta_1 \neq \beta_2$) if and only if it is associated with the response $\Delta y$. For strategy 3, we write it as: 
\begin{equation}
y_1 = \beta G + \epsilon \quad y_2 = \beta G + \gamma G \times E + \epsilon
\end{equation}
So $G$ is a differential eQTL if and only if $\gamma \neq 0$. 

\item The difference of these strategies lie in: (1) whether samples need to be matched: for the response eQTL approach, the samples need to be matched (no treatment vs. treatment). When this is the case, response eQTL is preferred, since it removes other sources of variations. (2) Whether include treatment as a covarite in the regression model. 
\end{itemize}

Accounting for sample relatedness in response eQTL mapping [personal notes]: 
\begin{itemize}
\item Ref: [Knowles and Gilad, eLife, 2018]

\item Model: it is common in such experiments that the same individual is sampled multiple times (over treatment/time points). This leads to individual effects, which should be treated as random effects, and accounted for. Let $y_{ij}$ be the expression (of a gene) of individual $i$, $1 \leq i \leq m$ in condition $j$, $1 \leq j \leq J$. It has fixed effects from genotypes (ignoring other fixed effects), which may vary across conditions, random effects from genetic background (which vary across conditions), and from individual non-genetic effect (a single effect shared across conditions). Our model is:
\begin{equation}
y_{ij} = v_j + \beta_j x_i + u_i + \xi_{ij} + \epsilon_{ij}
\end{equation}
where $v_j$ is the mean expression across all individuals in condition $j$, $x_i$ is genotype and $\beta$ genetic effect. $u_i \sim N(0, \sigma_u^2)$ is the random, non-genetic effect from individual $i$, and $\xi \sim N(0, \sigma_{\xi}^2 K)$ is from genetic background (i.e. all other SNPs) with $K$ being GRM, and $\epsilon_{ij} \sim N(0, \sigma_e^2)$. We can think of $u_i$ as some special property of individual $i$, which is not genetically determined. Normally, in eQTL mapping, it would be treated as noised, but in experiments with repeated measurements of the same individual, it can be learned. 

\item Analysis: why it is important to model random effects $u_i$? Expression of gene in an individual $i$ may be high and has nothing to do with genetics (e.g. epigenetic, stress, etc.). Not including this effect will not affect FP rates, since it is not correlated with genotypes. However, it adds substantial noise. With repeated experiments, it is possible to learn such individual effects by using multiple measurements. Accounting for such effects is similar to consider only responses in eQTL analysis.  

\item Model simplification: in practice, we are testing many SNPs for each gene. The $u_i$ term is shared across all SNPs tested. So a practical strategy is to learn the random effects $u_i$, or $\sigma_u^2$ using all SNPs (adding genetic effects and kinship) only once, ignoring $\beta_j x_i$ term above. We can then write the model as $y_{ij} = v_j + \epsilon_{ij}^*$, with $\epsilon^* \sim N(0, V)$. Note $V$ is $N \times N$ matrix, where $N = m J$ is number of samples. Then in testing individual SNPs, account for the $u_i$ and $\xi_{ij}$ term:
\begin{equation}
y_{ij} = v_j + \beta_j x_i + \epsilon_{ij}^*
\end{equation}
where $\epsilon_{ij}^* \sim N(0, V)$. 

\end{itemize}

A dynamic model for genome-wide association studies [Das \& Wu, Hum Genet, 2011]: 
\begin{itemize}
	\item Motivation: suppose we are testing the time-series of a complex trait. The naive approach is, for each SNP, we test its association with the trait at each time point. However, this may lose power, as a SNP may influence one or more time points. 
	
	\item Varying coefficient model: given a SNP to test, let $a(t)$ and $d(t)$ be the effect of the SNP (additive and dominant). The null hypothesis is $\forall t, a(t) = d(t) = 0$. Assuming $a(t)$ and $d(t)$ can be modeled as sum of polynomial functions (splines), fit the coefficients of the splines. 
	
	\item Remark: 
	\begin{itemize}
		\item The idea of varying coefficient model: some coefficents of a complex model may be related in some way, in particular, some kind of continuity. Using non-parameteric model or splines to model the coefficients. 
		\item Comparison with Lasso: e.g. something like fusion penalty on the coefficients of the same SNP on the trait and different time points. Both approaches could acheive some type of ``smoothness'' (i.e. the coefficients of adjacent time points are similar). However, there are differences, e.g. Lasso forces sparsity but varying coefficient model not; Lasso, the ``smoothness'' is local, without any global trend; varying coefficient model with splines may not handle the case where the SNP affects a few unrelated time points, but not the others, etc. 
	\end{itemize}
\end{itemize}

Temporal Genetic Association and Temporal Genetic Causality Methods for Dissecting Complex Networks [Lin and Zhu, review for NC, 2017]
\begin{itemize}
	\item Background: Granger causality, to infer $X \rightarrow Y$, show that prediction of $Y_t$ can be improved by using $X$ at earlier time points. 
	
	\item Mapping dynamic eQTL (MPTGA): fit time-series expression data with a third-degree polynomial. A SNP is eQTL if the coefficients under the major allele is different from the coefficients under the minor allele. Do LRT: three coefficients (null model) vs. 6 coefficients (full model). Further extension: autocorrelation of expression across adjacent time points, $\rho$ for the nearest time point, and $\rho^2$ for time difference of two, etc.  
	
	\item Inferring causal model with Granger causality (TGCT): extend LCMS. To assess the model evidence of $M \rightarrow X \rightarrow Y$, use auto-correlation model for $X$ and Granger model for $Y$: 
	\begin{equation}
	X_{i,t} = \alpha_{0} + \alpha_1 X_{i,t-1} \qquad Y_{i,t} = \beta_0 + \beta_1 Y_{i, t-1} + \beta_2 X_{i, t-1} 
	\end{equation} 
	The paper allows the coefficients $\alpha_0$ and $\alpha_1$ to be different under different genotypes. Apply the model to cis-trans gene pairs. 
	
	\item Detecting cis-gene of teQTL hotspots: for a candidate gene, test only three models, all sharing $M \rightarrow X$. For these three models, the only difference lies in the distribution of $Y$, which depends on $X$ on the previous time points. The candidate gene is assessed by the number of targets it causally regulates.  
	
	\item Remark: MPTGA test suffers from potentially high d.o.f. TGCT: does not account for possible confounders (e.g. PCs). 
\end{itemize}

Dynamic regulatory QTL mapping during differentiation [Ben Strober, NHS, 2017]
\begin{itemize}
	\item Experiment: iPSC $>$ cardiomycote, 14 lines, 16 time points (15 days). K-means clustering: 4 clusters of genes by expression dynamics.
	
	\item Per-time cis-eQTL mapping. WASP combined haplotype test. Empirical FDR by permutation (5) - variant-gene pair level. Replication in iPSC-eQTL: choose iPSC-eQTL from larger samples, and assess their p-values in differentiated cells.
	
	\item Correlation of eQTL effect sizes across time. This motivates a model with a small number of factors that vary over time.
	
	\item Factor analysis on eQTL effect sizes across time points: SNP on factors; factors vary over time (effect size of a SNP is linear combination of effect sizes of the SNP on factors). Possible interpretation: SNP effects on TFs, and expression is a linear combination of TFs (gene-specific).
	
	\item Mapping dynamic eQTL by GLM: Genotype, Time (encoded as continuous variable) and interaction term. Library size as covariates. Effect sizes may differ between different time points (interaction term). Problem: time effect is linear, and interaction is also linear.
	
	\item Cell line specific confounder: PCA on cell line x genes (and time) expression matrix.
	
	\item Permutation for controlling FDR: simulation under the null (no effect size difference across time). Concern: simulation does not capture non-linear time effects.
	
	\item Remark: possible explanation of dynamic eQTL, e.g. [TF] changes, their binding sites can be dynamic eQTL.
	
	\item HMM to infer differentiation states, then control for HMM states. Discussion: HMM not necessarily better than PC.
	
	\item Discussion: QTL mapping of shape, e.g. wavelet coefficients.
	
	\item Remark: cell heterogeneity. About 40\% are cardiomyctes, the rest are other cell types.
	
	\item Lesson: challenges of context-specific QTL analysis in iPSC lines: (1) eQTL effects of a variant may change over time point, use interact terms. However, with many time points, this may lead to over-parameterization. (2) Different lines may differentiate at different rates; and cell type heterogeneity may differ between lines. Adjust for the difference using latent factors; using HMM states lead to similar results.
	
	\item Remark: Problem of dynamic eQTL mapping: one parameter for each time point - possible over-parameterization. May use factor-based eQTL to address the problem: learn eQTL effects on factors that represent a ``pattern'' of effects.
\end{itemize}

Response eQTL mapping and application in drug response [David Knowles, Stats seminar, 2017]; Determining the genetic basis of anthracycline-cardiotoxicity by molecular response QTL mapping in induced cardiomyocytes [Knowles and Gilad, eLife, 2018]
\begin{itemize}
	\item Background: Anthracycline cardiotoxicity (ACT) of doxorubicin: common chemotherapy drug. GWAS of ACT response: $<$1000 subjects, one SNP.
	
	\item Experiment: 45 LCL-derived iPSCs, then cardiomycytes. Treatment with ACT at 5 different concentrations and do RNA-seq.
	
	\item Expression changes: 98\% genes show DE. So fit a K-component mixture model for genes with Dirichlet prior. Learn dose-response curves of genes: different shapes.
	
	\item Likely model: initially drug induces (DNA) damage - damage response; then at high dosage, apoptosis.
	
	\item Response eQTL mapping: (1) Learn random effects: each sample is used multiple times (multiple concentrations), and this effect needs to be captured by random effects. Let $y_{ncg}$ be expression of gene $g$ of individual $n$ with concentration $c$, it has several parts: treatment effect (fixed), latent factors, individual random effect, genetic effects (treated as random effects)
	\begin{equation}
	y_{ncg} = v_{cg} + \sum_k W_{kg} x_{nck} + u_{ng} + \xi_{ncg} + \epsilon_{ncg}
	\end{equation}
	where $u_{ng} \sim N(0, \sigma_u^2)$ is the individual random effect, and $\xi \sim N(0, \sigma_{\xi}^2 \Sigma)$ is the genetic effects and $\Sigma$ is the kinship matrix, and $\epsilon \sim N(0, \sigma_e^2)$. To solve this model, integrate over $W, u, \xi, \epsilon$ and infer $x$ and $v$ and other parameters. This leads to MVN of $y_{:g}$ (data of a gene). \\
	(2) Testing individual SNP-gene pairs: test SNP effect on a gene while accounting for confounding by using the covariance matrix:
	\begin{equation}
	\Sigma_{\pi} = \Sigma_k \sigma_k^2 x_{:k} x_{k:}^T + \sigma_u^2 U + \sigma_{\xi}^2 \Sigma
	\end{equation}
	where $U$ is the a matrix of which samples are for the same individual. Testing is done by comparing three models via LRT: $\E(y_{ncg}) = v_{cg}$ vs. $\E(y_{ncg}) = v_{cg} + \beta d_n$ vs. $\E(y_{ncg}) = v_{cg} + \beta_c d_n$, where $d_n$ is the genotype of indivdiual $n$ and $\beta, \beta_c$ its effects. 
	
	\item Inference: MOM, the covariance of $y$ related to the parameters. Do EVD of covariance $\Sigma_{\pi}$, computationally efficient.
	
	\item Find 400 response eQTLs. Clustering dose-effect profiles of the SNPs.
	
	\item Higher enrichment of GWAS loci in response eQTL vs. eQTL only.
	
	\item Response ASE: ASE in resting and ASE at differenent concentrations.
	
	\item Model ASE: minor allele from Beta-Binomial, with mean logistic regression of $\beta_c$. Q: What’s $\beta_c$? Intuition: allele imbalance change with concentration. But why logistic regression, perhaps number of alternative alleles (in phased position)?
	
	\item Model of reQTL: TF binding? See enrichment of open chromatin.
	
	\item Remark: Testing response eQTL: different effects at different concentrations. High d.f. Alexis Battle: fit spline functions.
\end{itemize}

Shared Genetic Effects on Chromatin and Gene Expression Indicate a Role for Enhancer Priming in Immune Response [Alasoo and Gaffney, NG, 2018]
\begin{itemize}
	\item Motivation: study response-eQTL, under their mechanisms. In particular, are response-eQTLs also response ca-QTLs? What are specific TFs driving response-eQTLs? 
	
	\item Background: stimulation (IFN-gamma and pathogen) leads to signaling pathways and TFs: NF-kB, STAT2, IRF1.
	
	\item Models: how to explain stimulation-specific (response) eQTL? Figure 1a. chromatin effects are response-specific. Figure 1b. chromatin  effects happen before stimulation, mediated by a pioneering factor, e.g. PU.1. In stimulation, TF such as IRF1, leads to eQTL.
	
	\item Experiment: macrophages, treated with pathogen, IFN-gamma and both. RNA-seq in 80 lines, and ATAC in 40 lines.
	
	\item Mapping eQTL and caQTL: on each of the four conditions separately. About 3K eQTLs and 20K caQTL regions. Enrichment of TF disruption: comparison with ASTB data of NFKB and STAT2 (Figure S5).
	
	\item Detecting condition-specific QTLs: $Y_i = G_i + E_i + G_i \times E_i$, where $G_i$ is genotype and $E_i$ treatment (four conditions). Also, for the same cell line, measured in four conditions, the errors are correlated across four conditions, so introduce a random effect term for each cell line. Results: 387 response eQTL in at least one condition. Cluster by eQTL patterns across four conditions (Figure 2a).
	
	\item Enhancer priming: focus on 145 caQTL-eQTL pairs that are likely driven by the same causal variants (lead variants LD $> 0.8 $). For approximately half of the response eQTLs with a linked caQTL, the caQTL was present in naive cells before stimulation  (Figure 2c). Most cases: same directions of effects. The reverse direction (eQTL before response caQTL) is much less, 15\%. Enrichment of PU.1, CEBPa/b motif disruption in these cases.
	
	\item Regulation of multiple peaks by a single caQTL: use colocalization analysis, the fine-mapped SNPs same for multiple peaks. About 20\% have master regions and dependent regions.
	
	\item Disease colocalization: 22 eQTLs, about half are in stimulated condition. 24 caQTL, most detected in naive cells. Most of these caQTLs are not eQTLs: explanation, they are often secondary eQTLs that are missed in the caQTL-eQTL colocalization analysis. 
	
	\item Discussion: the implication of the study is chromatin + GWAS analysis may not find relevant cell states/types.
	
	\item Remark: under the priming model, gene expression is not activated, so the response eQTL is largely driven by expression changes (rather than change of genetic effects across conditions). This is not tested.
	
	\item Remark: the roles of stimulation-specific TFs, e.g. NF-kB and STAT2, are not investigated.
	
	\item Remark: not much joint analysis of caQTL and eQTL. In particular, the co-localization analysis (matching to the same SNP) considers only the strongest eQTL. This could miss other caQTLs that are weaker eQTLs. 
\end{itemize}
 	 
\subsection{Gene Networks and Trans-eQTL}

Summary: methods and strategy for finding trans-regulatory relationship
\begin{itemize}
	\item Joint learning of modules and genetic regulators: use regulators/genetic markers to guide the search of modules. Geronomo [Lee and Kohler, PNAS, 2006]. Bayesian Partition: expression of a gene is determined only by the genetic markers associated with that module. 
	
	\item Association of genetic markers with factors: ICA paper [PLG, 2011].  
	
	\item Searching for eQTL hot-spots: HESS. 
\end{itemize}

Regulator finding in yeast [Bing \& Hoeschele, Genetics, 2005]: 
\begin{itemize}
\item Methods: 
\begin{itemize}
	\item Data: [Brem, Science02]. 
	\item eQTL confidence interval (CI): for each eQTL, need to determine the interval flanking the eQTL. Bootstrap resampling method. 
	\item To narrow down the genes in the eQTL CI: (1) fine mapping: if there are multiple markers in the CI, narrow down to significant pairs of markers; (2) gene selection using co-expression with the target gene. 
\end{itemize}

\item Results: 
\begin{itemize}
	\item eQTL and CI: total of 570 (from [Brem02]) and 11 additional QTLs. The length of CIs: median 93kb, with 49 genes (from 0 to 717). 
	\item Candidate genes: in 65\% eQTL regions, a single gene was retained as the candidate gene, for other regions, 0 - 6 genes (zero in less than 10\% eQTLs). 45\% are cis-regulation, and 55\% are trans-regulation. 
	\item Transcriptional network: (Figure 4) overrepresented genes include: protein synthesis, aerobic respiration, transporter activity, lipid metabolism (Ura3, enzyme), pheromene response (MAT-$\alpha$), cytokinesis during cell separation (AMN1, protein). Only 26 TFs in the network, and most of them have a regulatory role in just one other gene. 
\end{itemize}
\end{itemize}

Causal genes and pathways [Tu \& Sun, Bioinfo, 2006]:
\begin{itemize}
\item Problem: given eQTL data, find the causal genes (in eQTL region), and the pathways linking the causal genes to the target through TFs. 

\item Method:
\begin{itemize}
	\item Network: available from PPI, protein phosphorylation and TF-DNA interaction.
	\item Search for the pathway from causal gene (unknown) to some TF (through intermediates) and to the target gene (TF-gene link), according to: 1) the causal gene should be in eQTL regions (i.e. a candidate list of causal genes is available); 2) the genes within the pathway should have correlated expression with each other. 
	\item Search algorithm: stochastic search, start from one TF, and at each step, move to an edge, according to the correlation coefficient at that edge. The algorithm stops if one candidate gene is reached. 
\end{itemize}
\end{itemize}

Structural model analysis of multiple quantitative traits [Li \& Churchill, PG, 2006]: 
\begin{itemize}
	\item Motivation: in QTL mapping, the traits are often correlated, and want to infer the genetic architecture of all traits: the QTLs common to all traits, and QTLs specific to individual traits, while taking into account the trait correlations. 
	
	\item Background: SEM is a hierarchy of regression relationship among variables, similar to Bayesian networks. 
	
	\item Methods: 
	\begin{itemize}
		\item Model selection problem: suppose $Q$ is one locus, $A$ and $B$ are two traits, the possible models include, e.g. $Q \rightarrow A \rightarrow B$; $Q \rightarrow A, Q \rightarrow B$; etc. The goal is to choose the best model. 
		\item Initial model construction: first do single locus analysis on each individual traits: 
		\begin{equation}
		Y = \beta_0 + \beta_1 Q + \epsilon	
		\end{equation}
		where $Y$ is the trait vector, $Q$ genotype vector, $\beta_0$ population mean, $\beta_1$ the effect of Q. Then identify pleotropic QTLs: suppose $Q$ is a QTL of $X$, and we want to know if $Q$ is also a QTL of $Y$, aware that $Y$ may be correlated to $X$. The idea is: conditioned on $X$, can the rest of variations of $Y$ explained by $Q$. This is done through a regression with additional variable $X$: 
		\begin{equation}
		Y = \beta_0 + \beta_1 Q + \beta_2 X + \epsilon	
		\end{equation}
		If this results in large change of LOD (the effect of $Q$ on $Y$), then $X$ is causally connected to $Q$ and $Y$ (we need $X$ to explain $Y$ variation). 
		\item Model refinement: testing significance of coefficients. 
	\end{itemize}
	
	\item Question: 
	\begin{itemize}
		\item The effect of multiple QTLs on the same trait? 
		\item The relationship between two traits will not change when analyzing different loci, how is this modeled? 
	\end{itemize}
\end{itemize}

eQED: interpreting eQTL associations using protein networks [Suthram \& Ideker, MSB, 2008]: 
\begin{itemize}
\item Motivation: from eQTL data, find the related genes (the loci often contain multiple genes), and identify the regulatory pathways. The method by [Tu \& Sun, Bioinfo06] suffers from the ``dead ends'', i.e. the search often cannot reach the candidate genes. 

\item Model: from a physical network, find the path from the source gene (reside in eQTL) to the target gene. Formulate as the electric circuit problem. 
\begin{itemize}
	\item Sinlge locus model: find the causal genes within a eQTL region (muliple ones). Apply voltage at the eQTL, determined by the strength of that eQTL, (connected to all candidate genes) (the voltage at the target is supposed to be 0), and then currents flow along the network according to the conductance of edges (reliability). The problem is to determine the flow across each candidate gene, and choose the maximal one. Equivalent to random walk model, where source is one candidate and sink is the target. 
	\item Multi-locus model: apply voltages at multiple eQTLs, and calculate currents flowing across each candidate gene. 
\end{itemize}

\item Results: 
\begin{itemize}
	\item Assess the prediction of casual-target associations: using genetic perturbation data to create a ``gold standard''. 
	\item Assess the direction of the edges: whether the current predominantly flows in one direction. 
	\item Prediction of regulatory pathways: the shortest route with highest total sum of currents across its interactions. 
\end{itemize}
\end{itemize}

Mouse liver eQTL analysis with modules [Lan \& Attie, PG, 2006]: 
\begin{itemize}
\item Methods: 
\begin{itemize}
	\item Data and analysis: 60 $F_2$ mice segregating for obesity and diabetes. 45,000 expression traits in liver were measured. Analysis with the standard interval mapping. The significance threshold is chosen as LOD of 3.4 or higher, which corresponds to FDR of 0.48 (weak threshold used). 
	\item Module identification: through both correlation across 60 individuals (in genetic dimension) and GO enrichment. From the list of genes with significant eQTLs (seed genes), find transcripts with Pearson CC of 0.7 or higher, then test for GO enrichment. A list of genes, starting from some seed transcripts, with strong GO enrichment will be chosen for analysis. 
\end{itemize}

\item Results: 
\begin{itemize}
	\item QTL detection: with LOD of 3.4 or higher, found 6,016 transcripts with at least one eQTL. Among these, only 723 (best eQTL) were classified as cis- and the rest trans-. There are 15 regions (hotspots). When all eQTLs that have maximum LOD positions (significant or not) were considered, the regions clearly have GO enrichment.
	\item Module analysis: from 6,016 seed transcripts, found 1,341 lists enriched for at least one GO term. The lists are combined to form 862 unique non-redundant lists each corresponding to one GO term. 
	\item Electron transport chain: none of 24 expression traits of ETC exceed LOD of 3.4, but many of them shared the linkage peak on chr. 2. 
	\item GPCR module: using 38 seeds identifies 174 genes correlated with these seeds, and they are all related to GPCR signaling. The co-regulation of these 174 genes can be verified. The eQTLs of these genes are clustered in 3 regions: Chr. 2, 10 or 13. Especially in the region of 10cm in Chr. 2: 50 genes of the module had major eQTL peak, and 81 had a sceondary peak. 
	\item Scd1 module: genes correlated with Scd1 (choose top 20), highly enriched for lipid metablism genes, and map to the same locations. 
\end{itemize}
\end{itemize}

Identifying regulatory mechanisms using individual variation reveals key role for chromatin modification (Geronemo) [Lee and Koller, PNAS, 2006]: 
\begin{itemize}
\item Motivation: two ideas that could improve the single locus-single gene eQTL analysis:  
\begin{itemize}
	\item Grouping genes to enhance the signal, assuming a gene group is affected by common SNPs. 
	\item Correlations among gene expressions, in particular, correlations between regulators and other genes. 
\end{itemize}

\item Geronemo begins by partitioning genes into modules with similar expression profiles. It then iterates over two steps: (1) learning a regulatory program (trans-G and trans-E regulators) for each module and (2) reassigning each gene to the module whose regulation program provides the best prediction for the gene's expression profile. The expression level (average of the model) is a function of all regulators (regression tree).

\item Methods: 
\begin{itemize}
	\item Data: [Brem, PNAS05]. Candidate regulators: TFs, kinases and phosphatases, chromatin modification factors, RNA factors (degradation and RNA processing). 

	\item Extensions of Module Networks: the number of modules is not fixed, and a single gene can be ``broken off'' from the module; a FDR permutation test when determining a split in learning the regression tree, a prior distribution that impose sparsity on the number of regulators and the number of targets per regulator. 
\end{itemize}

\item Results: 
\begin{itemize}
	\item 79 modules containing at least 3 genes, spanning both trans-E (71 of 79) and trans-G (45 of 79) regulation. Adding trans-E regulation significant improves the proportation of genetic variance (PGV) of expression explained: explaining $>50\%$ PGV for 828 genes, vs 238 in the original paper. 
	\item Zap1 module: a regulatory program of 10 genes. The regulators include: Zap1-E, Zap1-G (SNP), Gcr1-E and Puf4-E. With the standard eQTL analysis: only two genes of the module were found to be regulated by Zap1-G. 
	\item A large number of modules are probably controlled by chromatin regulation: (1) consective genes along the chromosomes; (2) in particular chromatin domains, e.g. telomeres; (3) enriched for targets of chromatin modifiying proteins. Ex. a telomere module (42 genes of which 40 are in the telomeres): the top regulator is a locus containing RIF2, and the transcriptional regulator Swi.  
\end{itemize}

\item Remark: the method is not designed for identifying trans- eQTLs. The trans- regulators may dominate the signals, and the effects of cis-eQTL on individual genes were ignored (which may be important for mapping trans- eQTLs, as they tend to have much smaller effect than cis-eQTLs). Also, the statistical test of the significance of individual regulators/eQTLs is not provided. 
\end{itemize}

eQTL module mediated by TF activities [Sun \& Li, Bioinfo, 2007]: 
\begin{itemize}
\item Motivation: what is the mechanism of eQTL hotspots that are linked to multiple genes? One hypothesis is that some genes in the hotspot affects activities of TFs, which affect the target genes. 

\item Model: let $M$ denote the DNA polymorphism, $GC$ be the expression of gene cis-linked to the eQTL hotspot, $GT$ be the expression of other genes linked to the hotspot, and $TA$ be activity of some TF (estimated from the expression level of the known target genes of this TF). The goal is to compare two hypothesis: 
\begin{itemize}
	\item Causal model: $(M \rightarrow GC) \rightarrow TA \rightarrow GT$. 
	\item Reactive model: $(M \rightarrow GC) \rightarrow GT \rightarrow TA$. 
\end{itemize}
\end{itemize}

Harnessing naturally randomized transcription to infer regulatory relationships among genes [Chen \& Storey, GB, 2007]: 
\begin{itemize} 
\item Motivation: 
\begin{itemize}
	\item Biologically, we desire causal networks, where changing of a regulator leads to changing of its target. However, in the case of environmental perturbations (or sinlge gene knockouts): the expression of regulators and targets are often correlated, instead of causal. E.g. all genes in a module will be co-expressed under stimulations. 
	\item Possible to resolve the causality in eQTL data: genetic perturbations may randomize the expression level of one gene, and the effect on another gene can be tested. 
\end{itemize}

\item Methods: 
\begin{itemize}
	\item Idea: view expression of a transcript as a random variable, determine if the transcript $T_i$ causally influences $T_j$ through random perturbation at the eQTL of $T_i$. Basically, if $T_i \rightarrow T_j$, then the locus of $T_i$ should also be linked to $T_j$, and the causal influence of $T_i$ on $T_j$ can be tested. 
	\item Theorem: the causal relationship $L \rightarrow T_i \rightarrow T_j$ exists and there are no hidden variables causal for both $T_i$ and $T_j$ if and only if the following three conditions hold: $L \rightarrow T_i$, $L \rightarrow T_j$, and $L \bot T_j | T_i$. The last condition expresses: the causal effect from $L$ on $T_j$ can entirely be captured by $T_i$, thus no hidden variable. 
	\item Probability computation: apply the three conditions
\begin{equation}
P(L \rightarrow T_i \rightarrow T_j) = P(L \rightarrow T_i) P(L \rightarrow T_j|L \rightarrow T_i) P(L_i \bot T_j | L \rightarrow T_i \text{ and } L \rightarrow T_j)
\end{equation}
	Each probability is computed by: LRT on the corresponding hypothesis, and then apply FDR to make the null statistics probabilities. 
	\item Data: [Brem05] data, and for simplicity, only consider cis-linkage in regulators, i.e. $L_i \rightarrow T_i \rightarrow T_j$, where $T_i$ is cis-linked to $L_i$. 
\end{itemize}

\item Results: 
\begin{itemize}
	\item At probability threshold 90\%, found 4,394 significant regulatory relationship among 2,145 genes where 127 are regulators. 
	\item Four regulators: (1) NAM9 (a mitochordrial robosomal component) $\rightarrow$ same or similar pathway; (2) CNS1 (co-chaperon) $\rightarrow$ transferase, and ribosome biogenesis; (3) ILV6 (enzyme in AA biosynthesis) $\rightarrow$ AA biosynthesis pathways; (4) SAL1 (mitochondrial transporter) $\rightarrow$ mitochondrial and member genes. 
\end{itemize}

\item Remark: 
\begin{itemize}
	\item The regulatory links inferred in this manner reflects influences, not necessarily correspond to regulatory mechanism. E.g. changing expression of one gene, $X$, involved in RNA metabolism, the transcripts of many genes will be changed, but we would not think $X$ as a regulator of the affected genes. In general, the transcriptional networks, signaling networks and metabolic networks are integrated, thus metabolites, signal sensing etc. all could have influences on gene expression: regulatory networks function to serve the stability of the metabolic networks, thus changing metabolites/external signals will induce regulatory networks, which change expression of relevant genes. 
	\item The QTL structure already implies causality: if $T_i$ is cis-linked, and $L_i \rightarrow T_j$, then it must be $T_i \rightarrow T_j$. 
	\item The assumption that one locus control the transcript level is unrealistic: (1) there could be multiple loci controlling $T_i$, and (2) the dependence among loci may be important. E.g. suppose $L_i \rightarrow T_i \rightarrow T_j$, and $L_k \rightarrow T_j$, but $L_k$ does not control $T_i$, then $T_j$ will not be independent from $L_i$ conditioned on $T_i$ (as $L_i$ will have information of $L_k$ and $L_k$ may influence $T_j$). 
	\item Two extra causal influences overlooked in this study: (1) $L_i \rightarrow T_j$: direct influence of $L_i$ on $T_j$, e.x. through other genes (especially if $L_i$ is an eQTL hotspot). With this link, the CI test does not hold. (2) $L_j \rightarrow T_j$: the QTL of $T_j$ itself, modeling this should increase the power (instead of treating $T_j$ as completely random once conditioned on $T_i$). 
\end{itemize}

\end{itemize}

Using genetic markers to orient the edges in quantitative trait networks: the NEO software [Aten \& Horvath, BMC Sys Biol, 2008]
\begin{itemize}
	\item Common pleiotropic causal anchor (CPA) model: we want to test the causal relation between two correlated traits $A$ and $B$. Suppose $M_A$ represents the QTL of the trait $A$, five possible models: $M_1: M_A \rightarrow A \rightarrow B$, $M_2: M_A \rightarrow B \rightarrow A$, $M_3: A \leftarrow M_A \rightarrow B$, $M_4: M_A \rightarrow A \leftarrow B$, $M_5: M_A \rightarrow B \leftarrow A$. The model $M_1$ has the following characteristics: 
	\begin{itemize}
		\item $M_A \rightarrow B$: i.e. $M_A$ are common pleiotropic markers of both $A$ and $B$. This is different from models $M_2$, $M_4$ and $M_5$. 
		\item $M_A \perp B|A$: this is different from $M_2$ and $M_3$. 
		\item $A \perp B|M_A$: this is different from $M_3$. 
	\end{itemize}
	
	\item Orthogonal causal anchor (OCA) model: utilizing the markers of $B$, $M_B$. Four possible models: $M_1: M_A \rightarrow A \rightarrow B \leftarrow M_B$, $M_2: M_A \rightarrow A \leftarrow B \leftarrow M_B$, $M_3: M_A \rightarrow (A,B), M_B \rightarrow (A,B)$, $M_4: M_A \rightarrow A \leftarrow C \rightarrow B \leftarrow M_B$. The model $M_1$ has the following characteristics: 
	\begin{itemize}
		\item Each of $M_A$ has a pleiotropic effect on both $A$ and $B$. Different from $M_2$ and $M_4$. 
		\item $A \perp M_B$: thus the name orthogonal causal anchor. Different from $M_2$ and $M_3$. 
		\item $M_A \perp B |A$: different from $M_3$. 
	\end{itemize}
	
	\item Correlation-based tests: formulate the intuitions above using correlation and partial correlation. Ex. $M_A \rightarrow B$ is equivalent to $\text{Cor}(M_A,B) \neq 0$, $A \perp B|M_A$ is equivalent to $\text{Cor}(A,B|M_A) = 0$. To test if correlation is equal to 0, convert to $Z$ scores, which should follow normal distribution if the correlation is indeed 0. In general, any causal model implies some relations of correlations/partial correlations, e.g. for $M_A \rightarrow A \rightarrow B$, we have: 
	\begin{equation}
	\text{Cor}(M_A,B) = \text{Cor}(M_A,A)	\text{Cor}(A,B)
	\end{equation}
	Thus if $M_A$ is correlated to $A$, and $A$ to $B$, then $M_A$ to $B$; furthermore, $\text{Cor}(M_A,B)$ would be smaller than $\text{Cor}(M_A,A)$. 
	
	\item SEM tests: combine all the evidence of a model in a SEM framework. Formalize the idea that a causal implies a correlation structure. 
	\begin{itemize}
		\item For each model, test a goodness-of-fit between the sample covariance matrix $S$, and the predicted covariance matrix based on the model, $\Sigma(\theta)$, where $\theta$ is model parameters. Since $\theta$ is unknown, need to replace with its MLE. 
		\item To establish a model, say $M_1$, compute the $p$-value of $M_1$ and all alternative models, the LEO score (local edge orienting) is based on the ratio of the $p$-value of $M_1$ and the $p$-value of the next best model. 
	\end{itemize}
	
	\item Remark: the selection of markers $M_A$ and $M_B$ may be important. Ex. under OCA, $M_A \rightarrow B$ if $A \rightarrow B$, but not under $B \rightarrow A$. However, if the later is true, $M_B \rightarrow A$, and one may select a marker of $B$ as a marker of $A$, and it will influence both $A$ and $B$. 
	
\end{itemize}

Integrating large-scale functional genomic data to dissect the complexity of yeast regulatory networks [Zhu and Schadt, NG, 2008]: 
\begin{itemize}
\item Methods: 
\begin{itemize}
	\item Use eQTL data to identify casual regulatory relationship: consider two expression traits, $T_1$ and $T_2$, if both map to the same QTL $L$, then determine if $L \rightarrow T_1 \rightarrow T_2$ or $L \rightarrow T_2 \rightarrow T_1$, similar to [Schadt and Lusis, NG, 2005]. 
	\item Integrative reconstruction: use TFBS data (ChIP-chip plus conservation), eQTL data and PPI data (if at least half of gene in the complex contains a given TFBS, then all genes will be considered under control of that TF) as prior to construct Bayesian Networks (using expression data, similar to [Friedman, JCB, 2000]). 
\end{itemize}
\end{itemize}

Lirnet: Learning a prior on regulatory potential from eQTL data [Lee \& Koller, PG, 2009]:
\begin{itemize}
\item Methods: 
\begin{itemize}
	\item Regression: (module network) suppose there are $n$ regulators (both markers, or G-regulators, and trans- regulatory proteins, or E-regulators), and for any module $m$, the expression of its member gene $g$ is:
\begin{equation}
y_{mg} = w_{m1}x_1 + w_{m2} x_2 + \cdots + w_{mn}x_n + \epsilon	
\end{equation}
		
	\item Prior: (1) the G-regulators, its regulatory potential depends on the chromosome region, distance to gene, conservation, etc.; (2) E-regulators: the function of genes, etc. The prior is a sigmoid function of all features. 
	
	\item Inference: Lasso-type of regression, with SSE (for any module, sum over all genes: since the parameters only depend on the module and regulators, thus all genes would have the same values), $L_2$ regularization, prior distribution. 
\end{itemize}

\item Remark: 
\begin{itemize}
	\item The model multiplicity problem is addressed by: (1) prior of the features: regulatory potential; (2) clustering of genes (response variables); (3) Lasso regression to reduce number of parameters. 
	\item Module-level regression: all genes share the same regulators (and parameters), thus there is only one predicted expression of any gene in a module (understood as module average). Thus SSE effectively measures the total distance from all genes to the module mean, summing over all modules. 
	\item Question: with prior distribution already specified, what is the interpretation of $L_2$ regularization (which is normally a prior)?
\end{itemize}
\end{itemize}

GFlasso: association analysis of a quantitative trait network [Kim \& Xing, Bioinfo, 2009]: 
\begin{itemize}
	\item Motivation: some loci may affect multiple phenotypic traits simultaneously (e.g. a eQTL hotspot). Thus we would expect the same loci controlling multiple correlated traits. 
	
	\item Background: the methods for multiple trait analysis: 
	\begin{itemize}
		\item Find loci that influence all phenotypes jointly. 
		\item Dimensionality reduction (e.g. PCA) on the traits, and apply linkage/association analysis on the new trait. 
	\end{itemize}
	
	\item Methods: 
	\begin{itemize}
		\item Data: $N$ individuals, each individual has $J$ SNPs, with value 0, 1 or 2 (the number of minor allels), denoted as $x_{ij}, 1 \leq i \leq N; 1 \leq j \leq J$; and has $K$ phenotypes, denoted as $y_{ik}, 1 \leq k \leq K$. Also assume the correlation graph of the phenotypes is available, where $r_{ml}$ is the correlation coefficient of two nodes (phenotypes) $m$ and $l$.  
		\item Model: let $\beta_{jk}$ be the regression coeffcient of the $j$-th predictor (SNP) on the $k$-th trait. The idea is: $\beta_{jk}$ should be sparse; and the correlated traits should have similar values of $\beta$. The first term is encoded as a lasso penalty, and the second a fusion penalty for not having the same weights ($\beta$) on two correlated traits. Formally, the penalties are: 
		\begin{equation}
		\lambda \sum_k \sum_j |\beta_{jk}| + \gamma \sum_{(m,l) \in E} f(r_{ml}) \sum_j |\beta_{jm} - \text{sign}(r_{ml})\beta_{jl}|	
		\end{equation}
		where $(m,l)$ denotes an edge in the correlation graph, and $r_{ml}$ is the correlation and $f(r_{ml})$ is some monotic function of $r_{ml}$, e.g. $f(r) = 1$ (unweighted) or $f(r) = |r|$. 
		\item Model fitting: the objective function and constrains are convex, could use quadratic programming. 
	\end{itemize}
\end{itemize}

QTLnet: Causal graphical models in systems genetics: A unified framework for joint inference of causal network and genetic architecture for correlated phenotypes [Chaibub-Neto \& Yandell, Annals of Applied Stat, 2009]
\begin{itemize}
	\item Model selection by conditional independence (CI): establish the model $Q \rightarrow X \rightarrow Y$ through the score:
	\begin{equation}
	\text{LOD}(y,q|x) = \text{LOD}(y,q,x) - \text{LOD}(y,x) = \log_{10} \frac{f(y|q,x)}{f(y)} - \log_{10} \frac{f(y|x)}{f(y)}
	\end{equation}
	i.e. the additional explanatory power of $Q$ on $Y$, beyond that of $X$. 
	\item Bayesian network: search for models that explain the trait network. Model averaging to obtain the posterior probability of each edge. 
\end{itemize}

ReL (Regulatory Linkage) analysis [Gat-Viks \& Shamir, PG, 2010]: 
\begin{itemize}
\item Motivation: identify groups of genes regulated by the same eQTLs. These groups should have similar expression patterns (in other conditions). 

\item Idea: the groups of gene should: (1) have similar expression profiles; and (2) linked to the same eQTL interval. Multiplying the conditions (1) and (2), we have the eQTLs should be linked to conditions where the group show characteristic patterns. (Thus avoiding hard-grouping of genes before the analysis).

\item Methods: 
\begin{itemize}
	\item Data: 112 segregants, gene expression profiles and a compendium of expression profiles from perturbations of regulators (including TFs). 
	\item ReL score: linkage between a eQTL and a regulatory signature (the expression profile under one condition: up- and down- patterns). A high ReL score if the genes linked to the eQTL have different expression than the rest of genes. 
	\item Grouping regulatory signatures, corresponding to eQTL intervals: biclustering of eQTL-signature matrix. Then identify: (1) the target genes, those linked to the eQTL interval; (2) causal regulators in the eQTL interval by other information: PPI, protein binding to promoter and same process. In addition, the regulatory proteins of the ReL modules can be identified as the source of perturbations.   
\end{itemize}

\item Results: 
\begin{itemize}
	\item Found 13 high-scoring ReL modules, covering 281 genes, 311 genetic markers and 82 regulatory proteins. 
	\item Uracil biosynthesis module: UAR3 (causal regulator), Ppr1 (regulatory protein). Possible mechanism: UAR3 mutation affects the uracil production rate, which negatively regulates the TF Ppr1. 
	\item Middle sporulation module: RFM1 (causal regulator), Hst1/Sum1 (regulatory proteins). Possible mechanism: RFM1 is a specificity factor that directs Hst1 (HDAC) to some of the promoters regulated by Sum1p. 
	\item Oxidative phosphorylation module: Cat5 and Crd1 (causal regulators), Swi3 (regulatory protein). Cat5 is required for ubiquinono biosynthesis and Crd1 for some lipid in the mitochondrial memberane; Swi3 is a subunit of the SWI/SNF chromatin remodeling complex. 
\end{itemize}
\end{itemize}
 	  
A Bayesian Partition Method for Detecting Pleiotropic and Epistatic eQTL Modules [Zhang \& Liu, PLCB, 2010]: 
\begin{itemize}
\item Idea: partition genes into modules s.t. expression of each module can be explained by a small set of markers. 

\item Methods: 
\begin{itemize}
\item Model: partition of the genes and the markers, given the partition, the expression of a gene, $g$, in the $i$-th sample, in the module $d$, is given by: 
\begin{equation}
y_{ig} = \delta_d(x_i) + r_i + \alpha_g + \epsilon_{ig}
\end{equation}
where $r_i$ and $\alpha_g$ are sample and gene effects, respectively, and $\delta_d(x_i)$ is the function of the genotype of the $i$-th individual of the markers associated with $d$. This term is the genotype combination of the markers, e.g., if $d$ is associated with 1 markers, it has three possible values; and if $d$ is associated with 2 markers, it has 9 possible values; etc (possible epistasis is thus modeled). 
\end{itemize}

\item Results: identified 29 modules, 20 are linked to a single marker, and the rest to two markers, three of which have significant epistatic interactions. The modules are generally reasonable, most enriched with relatively specific GO terms. 

\item Remark: 
\begin{itemize}
	\item An example of unsupervised learning, where the partition is helped by the demand that genes in the same module should be explained by the corresponding markers. Similar to Module Networks, where partition is helped by: genes in the same module explained by the regulators. The genes are clustered not according to their expression pattern directly, but by whether they are correlated with a common set of markers. 
	\item Limitations: the QTL effect on individual genes are not modeled (in particular, cis-eQTLs). 
\end{itemize}
\end{itemize}

Cancer TRN by copy number and expression variation [Nordlander \& Nelander, MSB-manuscript, 2010]:
\begin{itemize}
\item Idea: gene copy number variation provides natural variations of genotypes, and can be used to infer TRN. 

\item Model: gene-dosage model: the expression of a gene is determined by both the copy number of itself and of its regulators. Let $\Delta U$ be the copy numbers of a patient (for each gene), relative to some reference patient, and similarly, $\Delta Y$ be the gene expression vector, relative the reference patient, we have:
\begin{equation} 
S \Delta Y + \Delta U = 0
\end{equation}
where $S = W - V$, and $W = (w_{ij})$ is the gene-gene ineraction in mRNA synthesis and $V = (v_{ij})$ is the gene-gene interaction in mRNA degradation. 

\item Methods: 
\begin{itemize}
	\item Model inference: the matrix $S$ encodes the gene-gene interaction. It can be inferred by solving the optimization problem: 
\begin{equation}
\min ||\Delta Y + S^{-1} \Delta U||^2_{Frobenius} + \lambda ||S||	
\end{equation}
The first term is a quadratic error term, and the second is the number of non-zero, non-diagonal elements in $S$ (diagoal terms reflect the direct effect - more gene copy, higher expression, so will not be penalized), and $\lambda$ is a tuning parameter favoring compact solutions. 

	\item Statistical confidence: 100 boostrapping - randomly select patients and infer $S$. Interactions present in 95\% of simulations are reported as consensus network structure. (Because of $\lambda$, most terms in $S$ will be zero, thus any non-zero term is considered a link in the network.)
\end{itemize}

\item Results: 
\begin{itemize}
	\item Construction of TRN in glioblastoma: first use correlation threshold (between copy number and expression) to filter genes, leading to 191 genes. Next construct the TRN, found 122 genes in the final network. A few pleiotropic regulators, EGFP and PDGFRA (well-known), and NDN, etc. 
	\item Validation of the subnetwork controlled by PDGFRA and NDN (role in glioblastoma not clearly established): effect of NDN and PDGFRA perturbation in U343 glioblastoma-derived cell line. 
\begin{itemize}
	\item NDN over-expression reduce the grow rate of U343 cells. 
	\item Testing specific predictions of the model: CPNE8 induction by NDN, KCNH8 induction by PDGFRA, FGF9 induction by PDGFRA (but not in the presence of over-expression of NDN). 
\end{itemize}
\end{itemize}

\item Remark: 
\begin{itemize}
	\item The main equation relies on the assumption that (on average) patients do not have specific transcription or degradation parameters, i.e. no other sources of variation except gene copy number difference (if $\Delta U = 0$ - no copy number variation, $\Delta Y$ must be 0). 
\end{itemize}
\end{itemize}

Bayesian detection of expression quantitative trait loci hot spots (HESS) [Bottolo and Richardson, Genetics, 2011]
\begin{itemize}
	\item MOM (mixture over markers): each response is associated with 0 or 1 marker with probability $p_j$ for marker $j$.  
	
	\item BAYES: for response $k$ we have $\E(y_k) = G \beta_k$, where $\beta_{kj}$ for marker $j$ follows spike-and-slab: 
	\begin{equation}
		\beta_{kj} \sim (1-\omega_j) \delta_0 + \omega_j N(0, \sigma_j^2)
	\end{equation}
	Different markers may have different $\omega_j$ and $\sigma_j$. The ones with large $\omega_j$ are ``hot-spots''.  
	
	\item Regression model: $k$-th response $y_k$, and genotype $X$, we have: $y_k = X \beta_k + \epsilon_k$, where $\epsilon_k \sim N(0, \sigma_k^2 I)$. Let $\Gamma = (\gamma_{kj})$ be the indicator of whether marker $j$ affects response $k$. $\beta_{kj} = 0$ if $\gamma_{kj} = 0$. Once $\Gamma$ is given, non-zero coefficients $\beta_k$ follows G-prior:
	\begin{equation}
		\beta_k | \gamma_k, \tau, \sigma_k^2 \sim N(0, \sigma_k^2 \tau (X_{\gamma_k}^T X_{\gamma_k})^{-1})
	\end{equation}
	Background: G-prior is the conjugate prior of Bayesian regression. So the variance of the effect is scaled by error variance $\sigma_k^2$ and covariance structure determined by data, and depends on one parameter $\tau$. 
	\begin{itemize}
		\item Remark: the prior is shared among SNPs acting on the same response $k$. In contrast, in PEER, the prior is shared for the impact of the same factor (reflecting importance of factor). 
	\end{itemize}
	
	\item Prior of indicators/configuration: 
	\begin{equation}
		\gamma_{kj} \sim \text{Bern}(\omega_{jk}) \qquad \omega_{jk} = \omega_k \times \rho_j
	\end{equation}
	So $\rho_j$ is an indication of whether $j$ is a hotspot, and $\omega_k$ is an indicator of heritability of gene $k$. 
	
	\item Inference: the joint likelihood: 
	\begin{equation}
		P(Y, \Gamma, \Omega, \tau | X) = P(Y|X, \Gamma, \tau) P(\Gamma | \Omega) P(\Omega) P(\tau)
	\end{equation}
	where $\Omega = \{\omega_{kj}\}$. Inference by Gibbs sampling, with MCMC on each part. MCMC on $\Gamma$ (configurations): use evolutionary stochastic search: run with several chains at different temperature. So moves across chains: global move; and within chains: local moves to explore neighborhood. MCMC on parameters: adaptive proposal (tune step size) and fixed proposal.
	
	\item Application to human monocyte eQTL data ($n = $1400): IDIN subnetwork of 600 genes, run HESS with 200 SNPs spanning 1Mb in a chromosome region.
	
	\item Lesson: Bayesian hotspot model, G-prior for effect sizes; factorized prior for inclusion probability. 
	
	\item \textbf{Remark}: the model relies on a small input set of genes. In applications, one first finds a gene group whose factor(s) is associated with a SNP (locus), then use HESS on this group of genes and SNPs in the locus.
\end{itemize}

Integrating genome-wide genetic variations and monocyte expression data reveals trans-regulated gene modules in humans. [Rotival and Blankenberg, PLG, 2011]
\begin{itemize}
	\item Data: monocyte gene expression in 1,400 subjects.
	
	\item Preprocessing (Figure 1): (1) MDS to filter outlier samples. (2) SVD: determine the number of components. Plot PVE vs. PCs, also do the same plot of randomized data. Find the number of components beyond which PCs explain only noise (Figure S5).
	
	\item ICA: 112 modules. Filtering: 21 components attributable to single individuals; 27 components with kurtosis $< 3$. Final results: 64 components.
	
	\item Annotating signatures: define genes associated with each signature as those in extreme distribution, control FDR. $>60\%$ modules show enriched GO terms.
	
	\item Association test of patterns and SNPs: do association test at $P < 10^{-7}$, then do enrichment test: SNPs need to be associated with a significant number of genes (in trans) at relaxed threshold $p < 10^{-5}$.
	
	\item Results: 11 associations. 4/11 are driven by cell type composition difference across samples. Some locus: associated with AIDs.
	
	\item Issues: cross-hybridization can lead to false trans-eQTL, cell type composition variation across subjects is a confounder (some are genetic).
	
	\item Comparison with WGCNA: (1) WGCNA founds about 20 modules, most are correlated with ICA signatures, however only 31\% ICA signatures are correlated with WGCNA. (2) ICA much better in association tests (Figure 5).
\end{itemize}

Pathway-Based Factor Analysis of Gene Expression Data Produces Highly Heritable Phenotypes That Associate with Age [Brown and Durbin, G3, 2015]
\begin{itemize}
	\item Data: skin eQTL data of MuTHER (twins, $n = 657$).
	
	\item Define pathway phenotypes: (1) Regress out global covariates using PEER: they explain 37\% of variation. (2) For each of 186 KEGG pathways, do PEER analysis on residual expression, and select 5 factors. They explain a median of 17\% of variance.
	
	\item Association of pathway phenotypes with age: 62 significant associations, comparing with 7 if using single gene association followed by DAVID enrichment test. Comparison with single-gene test: much more significant results.
	
	\item Top pathways: insulin signaling, fatty acid metabolism, xenibiotic metabolism, cancer-related pathways.
	
	\item Contribution of genetics and environment: single gene h2 about 0.13, pathway phenotypes 0.18, age-associated pathways 0.32, global factor 0.18. However, not found significant SNP association with pathway phenotypes.
	
	\item Discussion: high heritability of pathways, averaging out noise at single gene level. For global factors: genetics unlikely to have coordinated effects on a large number of genes, so global factors mostly like represent technical factors.
\end{itemize}

Tensor decomposition for multiple-tissue gene expression experiments [Hore and Marchini, NG, 2016]
\begin{itemize}
	\item Idea: explain the variation of gene expression across individuals and tissues using hidden factors. Then GWAS of hidden factors to identify eQTL. 
	
	\item Basic model for one tissue: let $Y_{nl}$ be the expression of gene $l$ on sample $n$. Suppose we have two components, e.g. [TF] or growth rate. Let $A_{n1}, A_{n2}$ be the activities of these two component in sample $n$, we have: 
	\begin{equation}
		Y_{nl} = A_{n1} X_{1l} + A_{n2} X_{2l} + \epsilon_{nl} 
	\end{equation}
	where $X_{1l}, X_{2l}$ are the effects of the two components on expression of $l$. In general, we define $X_{cl}$ be the \textit{gene loading matrix}, which describes the effect of a component $c$ on genes. The model assumes a sparse prior for loading matrix: 
	\begin{equation}
		X_{cl} \sim p_{cl} N(0, \beta_c^{-1}) + (1- p_{cl}) \delta_0
	\end{equation}
	where $p_{cl}$ is the prior probability of gene $l$ loading in $C$. We further use a Beat prior for $p_{cl}$. 
	
	\item Multi-tissue model: more generally, we have $A_{ntc}$ as the activity of $c$ in tissue $t$ of sample $n$. Then we can express the expression of gene $l$ in tissue $t$ of sample $n$ as: 
	\begin{equation}
		Y_{nlt} = \sum_c A_{ntc} X_{cl} + \epsilon_{nlt} 
	\end{equation}
	We further assume that the activity tensor can be factorized as: $A_{ntc} = A_{nc} B_{tc}$. The intuition is that: the activities of components are mostly determined by tissue profiles, but modified by indivduals. The multi-tissue model is better than analysis of all tissues separately because $X_{cl}$ is shared across tissues. 
	
	\item Finding eQTL: we apply the model to all tissue expression data. And then we have $A_{nc}$ for each sample, and can do GWAS on each component. If a SNP is associated with $c$, and the gene loading for $c$ contains two genes, then the SNP is eQTL of the two genes. 
	
	\item Data: MuTHER eQTL data, LCL, skip and adipose, 845 related individuals. 
	
	\item Results: $>200$ components in total, most active in single tissues. 26 components have genetic basis. But 20 have very sparse gene loading and correspond to mainly cis-eQTL. The six trans-component identifies some trans-eQTL: none was found by ICA or PCA (Figure 2-6).
	
	\item Example, MHC Class II (Figure 2): factor is strongly associated with a locus, CIITA ($p < 10^{-10}$). Gene loading: top genes are HLA. Highly active in LCL. 
	
	\item Other examples: MHC Class I (Figure 3), Histone genes (Figure 4), Type I interferon response (Figure 5), Zinc-finger gene network (Figure 6). Overall: most cases have 1 strong locus ($P < 10^{-8}$) except Zinc-finger, and usually no other loci $P < 10^{-6}$.  
	
	\item Lessons: 
	\begin{itemize}
		\item The advantage of tensor formulation: if one tissue only, the components can be arbitrarily rotated (identifiability issue), but this can be solved with tensor. 
		
		\item Sparse gene loading allows easy interpretation of eQTL. 
		
		\item Most expression variations have no genetic basis. 
	\end{itemize} 
	
	\item Remark/Questions: 
	\begin{itemize}
		\item The key assumption is that gene loading is shared across tissues. This is certainly not true, but somewhat reasonable: e.g. we can think about the TF-gene network topology is fixed (gene loading), but [TF] may vary across tissues (activity).
		
		\item The trans-eQTL discovered due to shared information across genes or across tissues? Probably both. But a general question is that if trans-eQTL are not often shared across tissues, does it have any advantage of using multi-tissue model? 
		
		\item Some components are mostly cis, affecting a very small number of genes. How are they chosen by the model?  
		
		\item Can we integrate multiple components to discover eQTL of a gene? If a gene appears in multiple components, we may gain power in this way. 
	\end{itemize}
\end{itemize}

Identifying cis-mediators for trans-eQTLs across many human tissues using genomic mediation analysis (GMAC) [Fan Yang and Lin Chen, GR, 2016]
\begin{itemize}
	\item Background: mediation analysis. Suppose we want to show that $M$ is a mediator, $X \rightarrow M \rightarrow Y$. We need establish: (1) $X \rightarrow Y$; (2) $X \rightarrow M$; (3) $Y = \beta_1 X + \beta_2 M$, test if $\beta_2 = 0$. Note that conditions 1 and 2 are necessary, otherwise, we may have the model, $X \rightarrow Y \leftarrow M$. Also if we already establishes (2), then Sobel test means we only need to test if $\beta_2 = 0$. 
	
	\item Model: suppose we test a trio $L_i \rightarrow C_i \rightarrow T_i$, where $C_i$ and $T_i$ are cis- and trans-genes. We may have confounders of $C_i$ and $T_i$, and it would be better to adjust for them. The key idea is to adjust for confounders for each trio. However, common children and/or intermediate variables between $C_i$ and $T_i$ are correlated with $C_i$ and $T_i$, and including them as confounders would create FPs (collider) or reduce power. Note that common children and intermediate variables are both associated with $L_i$, while confounders not. 
	
	\item Procedure: (1) For each trio $L_i, C_i, T_i$, first filter those associated with $L_i$. (2) Test mediation: Wald test of the regression coefficient of $C_i$ vs $T_i$ conditioned on $L_i$. (3) Control for FDR by permutation: permute cis-expression within genotype groups. 
	
	\item Application to adipose: start with 8K cis-eQTLs, then test trans-association of 8K SNPs vs. 27K transcripts. Found 3300 trans-associations at $p < 10^{-5}$ (expect about 1600). Then do mediation analysis on these 3300 trios: found 300 significant ones at FDR $< 0.05$. 
	
	\item Results: increase power to detect mediations, from 3K to 6K, comparing with PEER factor adjustment. Also, 20\% cis-genes mediate two or more trans-genes. 
	
	\item Cis-hub analysis: 615 cis-genes that mediate multiple trans-genes. Total of 20 cis-genes mediate $\geq 4$ trans-genes (Table 2). 
	
	\item Lesson: adaptive adjustment of possible confounders. Distinguish confounders, children (colliders) and intermediate variables. 
\end{itemize}

Large-Scale trans-eQTLs Affect Hundreds of Transcripts and Mediate Patterns of Transcriptional Co-regulation [Brynedal and Costapas, AJHG, 2017]
\begin{itemize}
	\item Data: About 300 LCL RNA-seq, in 3 populations.
	
	\item CMPA: cross-phenotype meta-analysis, applied to one SNP and all transcripts. Test on $-log10(p)$ (9000 p-values, one for each gene. Under H0: SNP not associated with any, it follows exponential distribution. Under H1: fit the exponential distribution with a free parameter $\lambda$. Then do MLE.
	
	\item Account for correlation using Z-scores: obtain the correlation of Z-scores of every two genes; then simulate Z-scores for all genes. Obtain empirical $p$-values for each SNP.
	
	\item CPMA power analysis: even though we test the distribution of all $p$-values, the power is OK if there are 50 genes associated with the SNP.
	
	\item Found no genome-wide significant SNPs. But show that the number of SNPs with low FDR is higher than expected.
	
	\item Test trans-eQTL associate to the same probes across populations. Top SNPs from CPMA: (1) Replication across populations. (2) Consistency of effect direction: across popluations.
	
	\item 62 SNPs have significant overlaps between all three samples. 1000 SNPs consistent in directions. 8 in common: super trans-eQTL.
	
	\item Functional enrichment: of genes of the same trans-eQTL. The 8 top trans-eQTL: each regulates hundreds of transcripts, 77-800 (Table 2). Histogram of p-values of top eQTL. Enrichment of TF targets (ENCODE ChIP-seq) in about 4/8 trans-eQTL.
	
	\item Trans-eQTL: the 8 trans-eQTL are correlated with PEER factors. If using SVA to remove confounding variables, some signal can be removed.
\end{itemize}

An independent component analysis confounding factor correction framework for identifying broad impact expression quantitative trait loci (CONFETI) [Ju and Mezey, PLCB, 2017]
\begin{itemize}
	\item Background: main challenge is that some confounding factors are heritable. Current strategies: (1) Joint estimation of SNP effects on gene expression and confounding factors [PANAMA, Fusi, PLCB, 2012]. (2) Use a subset of genes to estimate non-genetic confounding factors [Joo and Eskin, GB, 2014].
	
	\item Identifying heritable ICs: $Y = AS$ where $A$ is IC values of $n$ samples and $S$ is $k \times g$ matrix, weighting matrix. Once ICs are identified, do association analysis of ICs, and find genetic ICs with Bonf. correction.
	
	\item Analysis: factors identified by ICA may be more distinct, and thus facilitate downstream analysis, e.g. genetic association. Show an example: best factors associated with sex, ICA much better than PCA (Figure S1).
	
	\item Correcting for non-heritable ICs in testing eQTLs: LMM approach. Use the remaining ICs to compute sample correlation, then fit LMM (Figure 1).
	
	\item Analysis: why use LMM instead of fixed covariates? Number of parameters may be large using fixed covariates.
	
	\item Simulation: real genotype data from yeast. Broad-impact eQTL: SNP affecting 10\% of genes. Covariates/ICs: sparse (30\%) or dense (all).
	
	\item Results from simulation: For eQTL mapping, CONFETI similar to PANAMA, which estimate genetic effects (eQTL) and factors simultaneously. For broad-effect QTL: CONFETI significantly better.
	
	\item Human data: (1) Cis-eQTLs: CONFETI modest gain. (2) Trans-eQTLs: CONFETI modest gain. PCA and PEER do very poorly, even worse than linear model.
	
	\item Replication of broad-impact eQTLs: only in MuTHER LCL (twin pairs), but not between blood dataset (DGN and NESDA). Figure 5: a small number of such eQTLs, only a few targets.
	
	\item Remark: CONFETI has some advantage over factor models assuming normality: the effects of factors on genes are likely non-normal. However, this may be captured by sparse models.
\end{itemize}

Moving beyond cis-regulation [Xuanyao Liu, NHS, 2019]
\begin{itemize}
	\item Ref: GBAT: a gene-based association method for robust trans-gene regulation detection [2018]. 
	
	\item I. Omni-genic model. Background: Serum urate levels: 180 loci. top 28 loci explain 5\% h2. 183 explain 7.7\%. Common variants: 23\%. Pedigree h2: 38\%.
	
	\item Variance partition for cis- and trans- effects: (1) core genes: cis-QTL. (2) core genes: trans-QTL. (3) Covariance of core genes. Estimates of contribution of each part depends on genetic correlation of gene expression: if it is high for cis-genes, trans-genes can explain $>70\%$ heritability.
	
	\item II. Trans-eQTL mapping. Ex. in NTR data, average cis-heritability is 0.02 and trans is 0.045. Across studies, trans-eQTL explain 2/3 heritability. 
	
	\item Challenges of trans-eQTL: (1) correct for confounders in trans-eQTL can lead to reduced power and FPs due to collider effects (both cis and trans- genes affect the confounder). (2) RNA-seq read mis-mapping to homologous regions: 75\% FP rates.
	
	\item GBAT: correct bias in read mapping (30-40\% reads). Predicting gene expression: BLUP, leave-one-sample-out cross validation (train weights in training and evaluate in leave-one-out sample).
	
	\item BLUP: BLUP predictions are correlated with errors. This lead to false correlation of predicted cis-gene and trans-gene. This is commonly addressed by cross-validated prediction model: i.e. train the BLUP estimator (BLUE) in all dataset except one, then make prediction in the remaining data. Use cvBLUP to implement a fast way of getting leave-one-out prediction.
	
	\item Association test: $Y \sim \tilde{X} + R$, where $\tilde{X}$ is the predicted cis-expression, and $R$ is the surrogate variable (SV) residual (after removing $X$).
	
	\item Quantile normalization: $y = C + $ error, C is t-distribution. QN(y) leads to increase FP.  Solution: regress out C first in y, then QN(y.residual).
	
	\item DGN: 400 trans associations in 157 unique regulators. Enrichment of TFs in regulators. Some control more than 3 genes, e.g. NFKBIA regulates 4 other genes, SRCAP regulates 88 genes. 
	
	\item Disease-specific trans-eQTL: SCZ vs. control, trans- genes in SCZ are three times enriched with TWAS hits, not in control. Discussion: controls may have different cell compositions (consequence of disease states). 
	
	\item III. Interaction testing. Crohn’s disease: common variants explain 20\% heritability. Interaction can increase it to 60%. 
	Epistasis: gene-based interactions (predicted gene expression). 
	
	\item Remark: why need cross-validation? Unlike TWAS, here prediction model of gene expression, and its application are performed in the same dataset.
	
	\item Lesson: what variables to control for in testing trans-eQTL? Account for SVs can lead to collider bias, so obtain SVs, then regress out $X$ (genotype or imputed expression), then adjust the residual.
	
	\item Lesson: when $X$ is t-distribution, do quantile normalization can lead to increased FPs.
	
	\item Q: how to deal with LD and pleiotropy? The general challenge of TWAS.
\end{itemize}

Pathway-level information extractor (PLIER) for gene expression data [Mao and Chikina, NM, 2019]
\begin{itemize}
	\item Model: let $Y$ be expression matrix, $Y = Z B$, where $Z$ is the gene to factor loading. The idea is to choose $Z$ to align with known pathways/gene sets, this is done by another decomposition $Z = C U$. This can be interpreted as a model: factor $\rightarrow$ gene set $\rightarrow$ genes.
	
	\item Inference: constraint rank$(Z) = k$ and rank$(B) = k$, where $k$ is number of factors. Optimization: $\norm{Y - Z B}_F^2 + \lambda_1 \norm{Z - CU}^2 + \lambda_2 \norm{B}^2 + \lambda_3 \norm{U}_1^2$, with $L_1$ penalty for $U$.
	
	\item Hyperparameters: (1) $k$: significant number of PCs. (2) $\lambda_1$: cannot use cross-validation as reconstruction error is always minimized at $\lambda_1 = 0$. Default setting of $\lambda_1$ and $\lambda_2$. (3) $\lambda_3$ controls sparsity of $U$ (how much LV is associated with given gene sets), optimize $\lambda$ s.t. 70\% of LVs are associated with gene sets.
	
	\item Evaluation of LVs: compute AUC and FDR. A LV is considered high confidence, if AUC $> 0.7$ and FDR $< 0.05$. AUC: by cross-validation, for a given pathway, remove a certain subset, and use the rest of genes for training. Then test if the resulting $Z$ matrix (gene-LV loading) can recover the left-out genes vs. random genes. 
	
	\item Results in cell type decomposition analysis: works as well as CIBERSORT, which is based on reference transcriptome data.
	
	\item Trans-eQTL analysis: all LVs vs. all SNPs, do BH FDR corrections. For the remaining ones at FDR $< 0.05$, Then do gene level association test, and filter pathways with low gene-level support. 
	
	\item Results in DGN: found 86 LVs assocated with 300 pathways. 12 LVs associated with 10 unique SNPs. Some SNP is associated with LV enriched in genes related to megakaryocyte/platelet lineage. However, the SNP is not found in GWAS catalog. 
	
	\item Q: how FDR of LVs is controlled?
\end{itemize}

\subsection{Allele-Specific Expression and Epigenomics}

Background: allelic imbalance and allele-specific expression (ASE) [personal notes]
\begin{itemize}
	\item ASE due to cis-regulatory polymorphism: suppose we have a SNP in a regulatory sequence (called test SNP), and the allele A is associated with higher TF binding and the allele a lower binding. Now consider a region targeted by this regulatory element (called linked region)in one heterozygous individual (Aa), the haplotype of the linked region lined to A is expressed highly and the haplotype linked to a is expressed at a lower level. This leads to ASE in this individual.
	 
	\item Individual-level ASE and cohort-level ASE: 
	\begin{itemize}
		\item At the individual level, any cis-regulatory polymorphism always leads to some kind of ASE in the gene it controls (it may not be detectable, e.g. when there is no exonic SNP). Note that at the individual level, the rQTL and the gene can be far, but the haplotype can potentially be resolved (for an individual, a haplotype is an entire chromosome). 
		\item At the cohort level, to have a consistent ASE (e.g. exonic allele B always expressed highly), the exonic locus should be in good LD with the cis-eQTL. 
	\end{itemize}
	
	\item Relating ASE to regulatory effects of SNPs: suppose we have one individual heterozygous of the regulatory SNP of interest, Aa. This SNP creates differential expression of the linked transcript. All exonic SNPs in this transcript would thus show ASE. Consider SNP $j$, suppose we know the exonic SNP linked to A (phasing, which is possible because of LD), the reads mapped to this allele $x_j$ follows $x_j \sim \text{Bin}(n_j, p)$, where $n_j$ is the read depth at $j$, and $p$ the extent of allelic imbalance. Let $\beta$ be the effect size of A: defined as log fold change, then we have: 
	\begin{equation}
	p = e^{\beta} / (e^{\beta} + 1)
	\end{equation}  
	Note that: the ASE of the transcripts is shared across the entire length, thus all exonic SNPs. SNPs are generally sparse, and much larger than read length, thus the read depth of each SNP can be considered independent ``reading'' of the underlying ASE. 
	
	\item Using ASE for mapping cis-eQTL: 
	\begin{itemize}
		\item Even though ASE can be detected with a single individual, we cannot resolve cis-eQTL in one individual because of many possible SNPs in the cis-region of a gene. We need to study multiple indidivuals to find the cis-SNP(s) with consistent ASE. 
		
		\item We don't have to require the LD between cis-eQTL and ASE: we only need to test, if one allele in cis-eQTL is always associated with higher expression in the linked region (it doesn't matter if the highly expressed haplotype is consistent across multiple ndividuals) [McVicker \& Pritchard, Science, 2013]. 
		\item Alternatively, we treat ASE as a quantitative trait, and test association of an allele in the test SNP with ASE level. It is called aseQTL [Battle et al, GR, 2013]
		\item Advantage: the allelic imbalance is entirely due to difference in cis (thus we control all the other covariates that would be required to compare expression level across individuals). Furthermore, allele imbalance test can be combined with the usual way of mapping cis-rQTL. 
	\end{itemize}
	
	\item Using ASE or ASB/ASM/ASHM (allelic-specific binding/methylation/histone modification) for disease studies: study the effect of one SNP on the regulatory activity/gene expression. 
	\begin{itemize}
		\item Utilizing ASE at the individual level: A disease locus may be a cis-rQTL, thus creating ASE in the target gene. We can use the ASE to map the target gene of the cis-rQTL: test in heterozygous individual, if a test SNP (potential cis-rQTL) is always associated with higher expression of the test gene (at the individual level).  
		\item Utilizing ASE at the cohort level: a disease locus may be a cis-rQTL, test if there is a cohort-level ASE in LD with this rQTL (if so, likely the gene is the causal gene behind the cis-rQTL).  
		\item A cis-rQTL may create ASB/ASHM/ASM (or allele-specific chromatin accessibility) in that site: this effect can be mapped using allelic imbalance in the site. Ex. in ChIP-seq experiments, if one allele is associated with a higher level of histone modification, then this allele will be enriched in all reads mapped to this region. 
		\item Some example studies: open chromatin in islet cells [Gaulton \& Ferrer, NG, 2010]
	\end{itemize}
\end{itemize}

ASE lecture: \url{https://www.ebi.ac.uk/training/online/course/embo-practical-course-analysis-high-throughput-seq/allele-specific-expression-and-eqtl}
\begin{itemize}
	\item Basic model: binomial test, $x \sim \text{Bin}(n, p)$, where $p = 0.5$ if no ASE, and $n$ the number of reads, and $x$ the reads mapped to the reference allele. 
	
	\item Read mapping bias: a study [Li et al, GR, 2008], QQ plot, most loci show ASE comes from reference alleles. Summary: 90 ASE, 61 show over-representation in reference alleles. 
	
	\item Addresing read mapping bias: do simulation under null (no ASE). Found 1\% of SNPs show ASE (75\% reads mapped to reference allele). Simple solution: masking the alleles during read mapping. 
	
	\item eQTL mapping using RNA-seq [Li et al, GR, 2008]: Negative correlation between GC content and expression. Solution: regress out GC content. Procedure for eQTL: adjusting for GC (regression out), then quantile normalization. Then correct for latent confounders using fist 16 PCs
\end{itemize}

Research progress in allele-specific expression and its regulatory mechanisms [Gaur \& Liu, J Applied Genet, 2013]
\begin{itemize}
	\item ASE studies in human: how broad ASE is
	\begin{itemize}
		\item In [Vidal, 2011] estimate that at least 25\% of the human genes display ASE. 
		\item In [Serre, 2008], 4.6\% of heterozygous single-nucleotide polymorphism (SNP)–sample pairs have evidence of ASE. 
		\item In [Lee, 2013]: using exonic SNP chip in colecteral cell lines, found two monoallelically expressed genes (ERAP2 and MYLK4), 32 genes with an allelic imbalance in their expression, and 13 genes showing allele substitution by RNA editing. Among a total of 34 allelically expressed genes, 15 genes (44\%) were associated with cis-acting eQTL. 
	\end{itemize}
	\item cis-regulatory mechanism of ASE: 
	\begin{itemize}
		\item Cis-regulatory polymorphism (e.g. change TF binding) in strong linkage disequilibrium with variants within a gene. 
		\item Possible interaction between cis- and trans-regulatory polymorphism: the exonic SNP allele expressed highly might be different in different individuals. 
	\end{itemize}
	\item Epigenetic regulation of ASE: more general than imprinting. 
	\begin{itemize}
		\item Allele-specific methylation (ASM): ASM is relatively widespread across the mammalian genome, both cis- and parent of origin (POD) in nature, and often heterogeneous across tissues and individuals. 
		\item Allele-specific histone modification (ASHM): ASHM are associated with various disorders, such as diabetes. The loci that display ASE are substantially enriched with ASHM. Important findings: allele-specific differences exist in TF binding and open chromatin, they have consequences on downstream events such as expression, and at least some proportion of these differences is due to heritable genetic variation. 
	\end{itemize}
	\item Non-coding RNA regulation of ASE: 3′-UTR-SNPs, targeted by miRNA and associate with mRNA stability. 
	\item Method for detecting ASE: 
	\begin{itemize}
		\item Allelic bias in read mapping: the reference genome contains only one of the possible alleles. Idea: use enhanced reference genome that has information of alternative alleles. 
		\item A new method: construction of two haplotypes and remapping of the reads against the diploid transcriptome. 
	\end{itemize}
\end{itemize}

A powerful and flexible statistical framework for testing hypotheses of allele-specific gene expression from RNA-seq data [Skelly \& Akey, GR, 2011]
\begin{itemize}
	\item Background: the standard test of ASE, at a SNP, suppose there are $n$ reads, and $x$ is mapped to one allele, then ASE is detected by the binomial test of $p = .5$. 
	
	\item Idea: combine information of all exonic SNPs of a gene to test if the gene has ASE or not. 
	
	\item Model: for the gene $i$ and SNP $j$, let $x_{ij}$ be the number of reads mapped to the allele and $p_{ij}$ be a measure of imbalance. Then $x_{ij}$ follows binomial distribution with parameter $p_{ij}$. Furthermore, $p_{ij}$ follows Beta distribution $\alpha_i, \beta_i$. Let $p_i = \alpha_i /(\alpha_i + \beta_i)$ be the prior mean, then if the gene has no ASE, $p_i = .5$; and if it has ASE, $p_i$ is different. We use a mixture distribution, $\pi_0$ genes have no ASE and the rest have ASE. 
\end{itemize}

Identification of Genetic Variants That Affect Histone Modifications in Human Cells [McVicker and Pritchard, Science, 2013]
\begin{itemize}
\item Background: Beta Negative-binomial model. If $X | p \sim \text{NB}(r, p)$, and $p \sim \text{Beta}(\alpha, \beta)$, then $X \sim BNB(r, \alpha, \beta)$.  

\item Method: uses both read depth and allelic imbalance to map cis-quantitative trait loci (QTLs) with small sample sizes. Consider a test SNP, and a linked region (the target of the SNP, not necessarily in LD because we are using individual level ASE data). To see if the test SNP is a rQTL, we can use: (1) Variation of expression across individuals: the genotype of the rQTL (one of three possible genotypes) is associated with the expression level (total) of the linked region - this is measured by the total read depth. (2) Variation of expression between two alleles within a heterozygous individual (at the rQTL): it will lead to ASE in the linked region (allele imbalance in reads). Ex. the alternative allele is always associated with higher reads in the linked region (does not matter what are the genotypes in the hetero SNPs in the linked region). 

\item Total read depth (total level of the linked region): Let $G_i$ be the genotype at the test SNP of individual $i$, let $X_i$ be the total read count in the linked region, then $X_i$ depends on $G_i$. Specifically $X_i$ follows Poisson distribution with rate $\lambda_i$, and $\lambda_i$ depends on the total read depth $T_i$ of individual $i$, and the genotype $G_i$: $\lambda_i = 2\alpha T_i$ if $G_i = 0$; $\lambda_i = (\alpha+\beta) T_i$ if $G_i = 1$; $\lambda_i = 2\beta T_i$ if $G_i = 2$. 

\item Overdispersion problem for read depth: compare the mean and variance of read counts, find overdispersion across individuals. Even if Negative binomial cannot correct for it. Use Beta Negative Binomial distribution. The assumption is that $\lambda_i \sim T_i \beta(G_i) u_i$ where $\beta(G_i)$ is the expected effect given genotype $G_i$, and $u_i$ an individual-specific factor. The model amounts to a distribution of $u_i$ (instead of constant).  

\item Allelic imblance: if the test SNP is rQTL and is heterozygous in an individual, it will create allelic imbalance in the test region. Suppose the alternative allele increases the binding/expression, then in the test region of this individual linked to this allele, we expect more reads in a heterozygous SNP (whether the genotype of that SNP is a reference or alternative allele for that SNP is irrelevant). Let $Y_{ij}$ be the reads of individual $i$ at SNP $j$ (in linked region to the test SNP). Then $Y_{ij}$ follows binomial distribution with probability $p$, which is .5 under the null model, but not under allelic imbalance:
\begin{equation}
Y_{ij} \sim \text{Binom}(n_{ij}, p)
\end{equation}
To test allelic imbalance, we use all SNPs of the linked region:
\begin{equation}
L(D|p) = \prod_{i,j} P(y_{ij}|n_{ij}, p)
\end{equation}
and do LRT. 

\item Overdispersion problem for allele imbalance: permutation shows that the model cannot fully account for individual variation. Not clear how permutation test is performed: likely permute the test SNPs vs. linked regions. Use Beta-Binomial distribution: the model is $Y_i \sim \text{BetaBin}(N_i, p, \theta)$, where $\theta$ is overdispersion parameter (to be estimated) and $p = \alpha / (\alpha+\beta)$. We test if $\alpha / \beta = 1$. 

\item Combined haplotype test: To relate to the total read depth model, we assume $p = \alpha / (\alpha + \beta)$, and have a combined likelihood to test. 

\item Motif break analysis: take SNPs within TFBSs (Centipede, Posterior $> 0.99$), then correlation of PWM changes with allelic imbalance in various histone marks. Merge similar motifs: clustering analysis, distance is defined by overlap of predicted TFBSs from Centipede. Testing association of TF binding with allele imbalance: LRT, the allele imbalance (2kb ChIP-seq reads) is a function of change of PWM scores. 

\item Results of motif break analysis (Figure 3): Overall positive correlation with H3K27ac, and negative with H3K4me1. Specific TFs: Found 11 TFs at FDR $< 0.1$. Most are activators, but NRSF is a repressor. Validation: using allele-specific TF binding/ChIP-seq data (LCL). Confirm most of the 11 TFs (Figure S11). 

\item Remark: it is unclear how the method models multiple linked SNPs in the allelic imbalance test. The likelihood function suggests that they are independent (but later dropped index $j$). However, many of these SNPs must be in close LD, and thus not independent. Ex. any SNPs in the same haplotype of an individual will have the same reads (or similar), so the information is completely redundant. 

\item \textbf{Analysis}: Why individual variation in allele imbalance? Even if the test SNP has no effect, we still need to model individual variation of $p$ (expected value is .5). Possible confounder: in an individual, the allele ratio may be affected by other SNPs (not the test SNP) or other sources (e.g. imprinting).  
\end{itemize}

QuASAR: Quantitative Allele Specific Analysis of Reads, [Harvey,Bioinformatics, 2014]
\begin{itemize}
	\item Motivation: detecting ASE using only RNA-seq data (no genotype calls). Joint inference of genotypes and ASE. 
	
	\item Notations: we have RNA-seq data of one individual over many SNPs/sites, but multiple samples (e.g. different conditions). Note that in different samples, genotypes are shared but ASE may be different. Our data are $R_{sl}$ the reads of sample $s$ and SNP $l$, mapped to reference allele, and $A_{sl}$ those to alternative allele. The data depends on the underlying genotype. When it is heterozygous, it also depends on ASE, let $\rho_{sl}$ be the measure of ASE (proportion of reference allele). 
	
	\item Allele-specific model: let $g_l$ be the genotype of SNP $l$ (unknown). When it is homozygous, the read counts follow simple binomial distribution, allowing sequencing errors. When it is heterzygous, we use Beta-Binomial distribution. Let $N_{sl} = R_{sl} + A_{sl}$ be the total read counts
	\begin{equation}
	R_{sl} | N_{sl} \sim \text{Beta-Bin}(N_{sl}, \psi(\rho_{sl}, \epsilon_s), M_s)
	\end{equation}
	where $\epsilon_s$ is the error rate, $\psi(\rho, \epsilon) = \rho(1-\epsilon) + (1-\rho) \epsilon$, and $M_s$ is the overdispersion parameter. Note that: $\epsilon_s$ and $M_s$ are shared across all SNPs for a sample. The likelihood function marginalize $g_l$, with a given prior (default: 1KG, but can be provided by users). 
	
	\item Inference: EM to estimate genotype for each SNP $l$ and error rates $\epsilon_s$ for each sample, using all data. Genotyping: fit the mixture model (3 possible genotypes) to infer $g_l$ and estimate $\epsilon_s$. In this step, assume $\rho = 0.5$. 
	
	\item ASE inference: Consider only heterozygous sites with large MAP of genotype. (1) Estimate overdispersion parameter $M_s$ using all sites of sample $s$: fix $\rho = 0.5$, and use grid search. (2) For each site $l$, test if $\rho_{sl} = 0.5$ using LRT, with fixed $\hat{M_s}$. Also determine confidence interval using asymptotic distribution. 
	
	\item Analysis: problem occurs to decide between i) an heterozygous genotype call with extreme allelic imbalance, or ii) an homozygous genotype call with base call errors; our model will favor the latter ii) because it allows genotype uncertainty and base calling errors. Had genotypes been given, the only way to explain alt. allele reads is ASE. 
	
	\item Comparison with other tests: calibration of $p$-values (Figure 4). Binomial test: inflated. Beta-Binomial test: better, still many small p-values (FPs), possibly because homozygote genotypes called falsely as heterozygotes. Comparison of $p$-values: Beta-Binomial test similar $p$-values, but ``SNPs with more uncertainty on being heterozygous are corrected in a higher degree towards a less significant p-value''. 
	
	\item Q: why genotype uncertainty can influence $p$-values? If a SNP passes the MAP threshold, $p$-value of the test should no longer be affected by genotype uncertainty. 
\end{itemize}

WASP: allele-specific software for robust molecular quantitative trait locus discovery [van de Geijn and Pritchard, NM, 2015]
\begin{itemize}
	\item Addressing mapping bias (Figure 1): mapping to personalized genome does not addressing ref. bias, as the uniquely mappable reads of two alleles may differ. WASP: for reads that map to a polymorphic site, replace with the alt. allele and remap the reads - if mapped not to the exactly same location, discard the reads. Problem: underestimation of expression level. 
	
	\item Filtering duplicate reads: this step can introduce ref. bias, as the highest scoring reads will usually map to ref. allele. WASP: filter duplicate reads randomly. 
	
	\item Adjusting for total read depth (Figure SN1): samples could differ in their amplification efficiency, and hence the number of reads mapped to peaks vs. background. This changes the relationship of read depth vs. true expression, e.g. in low efficiency experiment, reads are not concentrated on high expression features. To adjust for this, we need: for a particular sample, given the expression level of a gene, how many reads (fraction of reads) we expect. So for each sample, we fit feature read count vs. total read count across all samples (proxy of true expression), and obtain sample-specific quadratic model. This adjust for sample difference in efficiency: e.g. a high-expression gene will have a lower fraction of reads (over the library) in low efficiency samples, comparing with high-efficiency samples. 
	
	\item Adjusting for GC correction: similar to RASQUAL. 
	
\end{itemize}

High-throughput allele-specific expression across 250 environmental conditions [Moyerbrailean \& Luca, GR, 2016]
\begin{itemize}
	\item Principle: GWAS SNPs may manifest as condition-specific eQTL (or ASE). 
	
	\item Experiment: 50 treatments, 5 cell types. Two pass: first shallow sequencing $<10$M, to find DE genes and only follow up experiments with large DE; then deep sequencing (130M). 
	
	\item Detecting ASE: QuASAR on each condition separately. 
	
	\item Detecting conditional ASE (cASE): two approaches 
	\begin{itemize}
		\item MeSH [Wen \& Stephens]: test four models, not ASE in treatment and control; ASE only in treatment or control; ASE in both. Obtain BF for each of the models. Note: not capture quantitative changes of ASE magnitude across conditions. 
		\item $\Delta$AST (differential allele-specific test): $Z$ score from the difference of the estimated effect sizes in treatment vs. control. 
	\end{itemize}
	
	\item Detecting induced ASE: ASE in one condition (e.g. treatment) and gene not expressed in the other condition. 
	
	\item Validation of ASE genes using GWAS: gene set enrichment. Per SNP heritability enriched in ASE, cASE and iASE genes. Example: genes of T2D show differential ASE in the caffeine treatment condition. 
	
	\item Remark: detection of ASE jointly across multiple conditions may increase the power. Could use CorMotif or MASH. 
\end{itemize}

Fine-mapping cellular QTLs with RASQUAL and ATAC-seq [Kumasaka and Gaffney, NG, 2016]
\begin{itemize}
	\item Idea: we test association of a putative QTL (rSNP) with a feature (gene expression or peak). If the rSNP is real, it will also generate allele imbalance of the SNPs in the feature (fSNP), assuming rSNP and fSNPs are linked.
	
	\item Notations: Figures S26. We study one rSNP a time, and let $i$ be index of sample. The genotype of sample $i$, $G_i$ is linked to a feature (peak or expression), but there could be multiple variants in that feature. Ex. two fSNPs, our data are $Y_{i1} = (Y_{i1}^{0}, Y_{i1}^1)$ and $Y_{i2} = (Y_{i2}^{0}, Y_{i2}^1)$. Note: QTL can be located in the feature and QTL can be one of the feature variants. Denote $G_{il}$ as the genotype of the $l$-th fSNP in sample $i$. We can consider the ``diplotype'', i.e. two haplotypes, at $i$ and $l$: e.g. $D_{il} = 00/11$. The uncertainty of $G_i$ and $D_{il}|G_i$ is modeled (below). For read counts, let $Y_i$ be the total read counts in the feature, and $Y_{il} = (Y_{il}^{(0)}, Y_{il}^{(1)})$ the count of ref. and alt. allele of the fSNP $l$. Also denote $Y_{i0}$ as the count of all reads not overlapping with any fSNP. 
	
	\item Model overview: let $1-\pi, \pi$ be the allelic effect of QTL (allele bias). Then the expected rate (of read counts) would be $(1-\pi) \lambda$ for ref. allele and $\pi \lambda$ for the alt. allele, where $\lambda$ is the relative expression level (proportion of reads of the feature in the library). (1) Between individual variation: $Y_i$ depends on $G_i$ using NB model, with parameters $\pi, \lambda$ and $\theta$ (overdispersion). (2) Allele-specific (AS) model: Beta-Binomial model with the same overdispersion parameter as between-individual model, and error parameters $\delta$ (mapping and sequencing error) and $\phi$ (reference bias). So the overall model can be written as: 
	\begin{equation}
	P(Y|G, \pi, \lambda, \theta, \delta, \phi) = \prod_i P(Y_i|G_i, \pi, \lambda, \theta) \prod_i \prod_l P(Y_{il}^{(1)}|Y_{il}, D_{il}, \pi, \delta, \phi, \theta)
	\end{equation}
	The validity of this can be seen in ``Probability Decomposition'' in Supplement. 
	
	\item Total count model: let cis-effect of the QTL $G_i$ as $f(G_i)$, which is $2(1-\pi)$ if $G_i = 0$, and 1 if $G_i = 1$ and $2\pi$ if $G_i = 2$. Then our model is: 
	\begin{equation}
	Y_i | G_i \sim NB(\lambda K_i f(G_i), \theta K_i f(G_i))
	\end{equation}
	where $K_i$ is the size factor of sample $i$, $\lambda$ is the average level of feature and $\theta$ overdispersion parameter. Under this model, $Y_i$ depends only on the genotype of rSNP. 
	
	\item NB and Beta-Binomial distributions: see the section ``Negative binomial and beta binomial distributions'' in Suppl. If we have two NB distributions with the same index of overdispersion, then the conditional distribution is Beta-Binomial and the sum is still NB. Let 
	\begin{equation}
	X \sim NB(\lambda, \beta) 
	\end{equation}
	where we parameter it s.t. $\lambda$ is mean and $\Var(Y) = \lambda + \frac{1}{\beta} \lambda^2$. Suppose
	\begin{equation}
	K \sim NB\left(\frac{\alpha \lambda}{\beta}, \alpha\right)
	\end{equation}
	It is easy to check $X$ and $K$ have the same index of overdispersion. Then the conditional distribution is: 
	\begin{equation}
	Y | N = Y+K \sim \text{BetaBinom}\left(N; \frac{\beta}{\alpha+\beta}, \alpha+\beta\right)
	\end{equation} 
	
	\item AS Model: see ``Fitting specific over-dispersed count distributions''. This leads to the conditional distribution of alt. allele reads for each feature variant: 
	\begin{equation}
	Y_{il}^{(1)} \sim \text{Beta-Binom}\left(Y_{il}, \frac{f_1(D_{il})}{f_1(D_{il}) + f_0(D_{il})}, \theta(f_1(D_{il}) + f_0(D_{il})) \right)
	\end{equation}
	where $f_1(D_{il})$ and $f_0(D_{il})$ describe the relative level of ref. and alt. alleles at fSNP $l$ when the diplotype is $D_{il}$, and is given by Table S4. Ex. when $D_{il} = 11/00$, we have $(f_0(D_{il}), f_1(D_{il})) = (1-\pi, \pi)$. 
	
	\item Remark: the derivation of AS model comes from the general property of NB distributions. However, in our case, the overdispersion of NB results from individual random effects, which should be controlled in two alleles of the same individual. To understand why we still need BetaBinomial to account for overdispersion, we can think of ``haplotype random effects'': the means of the two haplotypes are equal, but they each have an additional random effect. 
	
	\item Modeling errors and bias: the fraction of alleles change from above due to mapping errors (e.g. repeats) and reference bias. Let $\delta$ be the error of a read (allele switch), and $\phi$ be the reference bias ($\phi = 0.5$ no bias), then we modify $f_1(D_{il})$ and $f_0(D_{il})$ as: 
	\begin{equation}
	\begin{array}{lll}
	\tilde{f}_0(D_{il}) & = & 2 (1-\phi) [(1-\delta) f_0(D_{il}) + \delta f_1(D_{il})] \\
	\tilde{f}_1(D_{il}) & = & 2 \phi [\delta f_0(D_{il}) + (1- \delta) f_1(D_{il})] 
	\end{array}
	\end{equation}
	
	\item Genotype and haplotype uncertainty: consider errors in imputation and phasing. Also haplotype switching between $i$ and $l$ can happen. Then $G_i$ and $D_{il}$ need to be marginalized in the likelihood (main equation in text). Use EM to estimate parameters, and marginalize genotypes/diplotypes. 
	
	\item Allowing total counts to depend on $\phi$ and $\delta$ (Page S48): the current total count model does not consider reference bias. Ex. if many fSNPs in a sample are alt. alleles, then the observed total count may be lower than expected. To address this, we consider two types of reads: those not overlapping fSNPs, and those overlapping fSNPs. For the former, its expected read count depends only on rSNP, but for the latter, it depends on both rSNP (effect) and fSNP (mapping bias), as described in the AS model. We have: 
	\begin{equation}
	\E(Y_i) = \lambda \left[ (1-hL)K_i f(G_i) + hK_i \sum_l (\tilde{f}_0(D_{il}) + \tilde{f}_1(D_{il}))\right]
	\end{equation}
	where $h$ is the fraction of reads contributed by one fSNP. Inference of this model is difficult, since now $Y_i$ model depends on a large number of hidden genotypes $D_{il}$. The model approximates by, for each $G_i$, averaging over all possible $D_{il}$'s. It is shown that the model makes relatively little difference comparing with the simpler model. 
	
	\item Normalization problems and strategy: 
	\begin{itemize}
		\item GC content: amplification efficiency depends on GC content, and the relationship can vary across samples. This leads to noises: e.g. in one sample, amplification is high for high GC peaks, then those peaks will have more reads, not due to genotypes, but due to amplification bias.
		\item Hidden confounders: e.g. some samples are subject to higher stress, which activates stress-related peaks. Then in these samples, the stress peaks have higher read counts, not due to genotypes. 
	\end{itemize}
	In general, GC and confounders reduces power, but do not create FPs. Two general strategies to address this are: (1) normalize RPKM/FPKM by multiplying a correction factor so that peaks with different GCs or confounders are comparable. Let $Y_{ij}$ be the read count of peak $j$ in sample $i$, then the log2-RPKM/FPKM is defined as: 
	\begin{equation}
	y_{ij} = \log_2 \frac{Y_{ij}+1}{l_j Y_i}
	\end{equation}
	where $l_j$ is the length of feature $j$ and $Y_i = \sum_j Y_{ij}/10^6$ is the library size. We will then adjust for $y_{ij}$. (2) Normalize the expected rates in a model (instead of the data): this is the size factor $K_i$ in the total count model, and is simply defined as (for now, and to be adjusted later): $K_i = Y_{i \cdot} / Y_{\cdot \cdot}$, where $Y_{i \cdot} = \sum_j Y_{ij}/J$ and $Y_{\cdot \cdot} = \sum_{i,j} Y_{ij} / (NJ)$.  
	
	\item Normalization by GC content (``GC correction for fragment counts and FPKMs''): we obtain a sample-specific profile of read enrichment vs. GC content. We first bin all features by their GC contents. Let $S_{il} = \sum_{j \in B_l} Y_{ij}$ be the total reads in bin $l$ of sample $i$. We can then define enrichment of reads in bin $l$ for sample $i$ as: 
	\begin{equation}
	F_{il} = \log_2 \frac{S_{il}/S_{\cdot l}}{S_{i \cdot} /S_{\cdot \cdot}}
	\end{equation}
	We can then fit a spline function of $F_{il}$ vs. GC content of bin $l$, and let $c_{ij} = \hat{F}_{ij}$ be the expected log2 enrichment of feature $j$ in sample $i$. This can be used to normalize RPKM/FPKM or the size factor: 
	\begin{equation}
	\tilde{y}_{ij} = y_{ij} - c_{ij} \qquad K_{ij} = K_i e^{c_{ij}}
	\end{equation}
	
	\item Normalization by PCs (``Principal component correction''): the strategy is to first learn hidden confounders (e.g. stress level) for each sample, then for each feature, we adjust for the effect of confounders (the effects may differ for different features). To learn hidden confounders, we do PCA on $y_{ij}$'s, and choose PCs based on permutation (only PCs whose contributions are greater than permutation). To adjust for PCs with RPKM/FPKM: we regress $y_{ij}$ vs. $x_i$ (PCs), and obtain $\tilde{y}_{ij}$ as residuals. For the count data, we fit NB regression model (ignoring genetic effects), and assume the mean for peak $j$ of sample $i$ depends on PCs: 
	\begin{equation}
	Y_{ij} \sim NB(\lambda_{ij} K_i, \theta_j) \qquad \log \lambda_{ij} = \alpha_j + x_i^T \beta_j
	\end{equation}
	The estimated parameters can be then used to adjust for size factor as: $K_{ij} = K_i \exp (x_i^T \beta_j)$. 
			
	\item Multiple testing correction: (1) Correct for number of SNPs per feature: from the $p$-value of lead SNP, multiply by the number of SNPs tested (Bonferroni correction). (2) Calibration of genomewide threshold: permutation (swap genotypes and read counts of multiple samples), and run the same procedure on permuted data. From the list of $p$-values from permuted data, do FDR estimation: expected number of features passing threshold under null vs. observed number. 
	\begin{itemize}
		\item Remark: the null distribution is obtained from all features in permutations. However, we expect the null to be difference. To address this, first use Bonferroni correction as the first step to adjust for number of tested SNPs per features. 
	\end{itemize}
	
	\item Remark/Question: only consider lead SNPs in multiple testing? How many permutations are required? 
		
	\item Comparison with Combined Haplotype Test (CHT): the last part of Suppl. and Table S2. The main differences: 
	\begin{itemize}
		\item Overdispersion: shared with between-individual variation and is feature-specific under RASQUAL. Sample-specific under CHT. 	
		\item Reference bias: estimate for each feature under RASQUAL. Not estimated, but addressed by randomization under CHT. 
		\item Additional sequencing/mapping error: estimated for each feature under RASQUAL. Fixed error rate under CHT. 
		\item Genotype uncertainty and haplotype switching: not modeled by CHT. 
	\end{itemize} 
		
	\item Assessing performance: use QTL of CTCF peaks, and enrichment of true causal variants (CTCF motif changing SNPs). 
	
	\item ATAC-seq of 24 LCL: 2000 caQTL, and 800 of them are in the peaks, and some in perfect LD. Among 900 lead SNPs (peaks or in perfect LD), more than 600 change a motif. Also examples of a QTL affecting multiple peaks: particularly, a SNP is caQTL of an enhancer and caQTL of nearby promoter. Among 173 multi-peak caQTL: most often the effect of caQTL on master and dependent peaks are consistent. However, not confident for interactions more than 100kb away. 
	
	\item Lesson: if we can detect true causal variants for regulatory SNPs, many of them act by changing TF binding motifs. 
	
	\item Lesson: regulatory SNPs can be used as IVs to study relationship among molecular traits.
\end{itemize}

Leveraging allelic imbalance to refine fine-mapping for eQTL studies [Zou and Eskin, PLG, 2019]
\begin{itemize}
	\item Limitations (from PLASMA paper): treat AS as binary. No phasing.
\end{itemize}

Allele-Specific QTL Fine Mapping with PLASMA [Wang and Gusev, AJHG, 2020]
\begin{itemize}
	\item Background: QTL effect size vs. AS effect size, nonlinear relationship. To understand why: QTL effect sizes depend on AF, while AS effect size not.
	
	\item Model: let $y_i$ be normalized expression, $y_i = x_i \beta + \epsilon_i$, where $x_i$ is genotype. For AS signal, let $w_i$ be the log. allelic ratio, and $v_i$ be AS phasing, coded as 1 if alt. haplotype and -1 if ref. haplotype and 0 if heterozygotes. We have $w_i = v_i \phi + \xi_i$, where $\xi_i$ models the error. Note that $\xi_i$ is not the same for all, derived it from Beta-Binomial distribution to capture over-dispersion. The model is equivalent to weighted linear regression.
	
	\item Modeling assumption and fine-mapping: allow correlation of QTL and AS effect sizes. However, found that in practice, this is not helpful, so the default is 0. Shared signal: same causal variants for QTL and AS signals.
	
	\item Methods compared: RASQUAL+, convert $\chi^2$ to $Z$ scores, then do FINEMAP. Also AS-Meta [Zou and Eskin, PLG, 2019]. QTL only. PLASMA-J (use both QTL and AS) and PLASMA-AS (only use AS signal).
	
	\item Simulation: for QTL signal, simulation of data similar to quantile normalization. Simulation under low AS variance and high AS variance setting. PLASMA-J and PLASMA-AS best. RASQUAL+ works well in high AS variance setting, but poorly under low AS variance.
	
	\item Results in TCGA: PLASMA-J slightly better than PLASMA-AS, much better than QTL only and FINEMAP (Figure 5): median CS size of 32 vs. 167.
	
	\item Lesson: AS signal is much stronger than QTL signal. Quantile normalization procedure in QTL study probably significantly reduces the power.
	
	\item Q: How is the method applied to real data? QTL discovery step? Only applied to significant QTLs?
	
	\item Q: how is QTL model used in real data? Does it adjust for covariates/hidden factors?
	
	\item Remark: credible set size is very large for FINEMAP, etc. in real data. Ex. in tumor data, CS is bigger than 100 most of time.
\end{itemize}


%%%%%%%%%%%%%%%%%%%%%%%%%%%%%%%%%%%%%%%%%%%%%%%%%%%%%%%%%%%%
\subsection{Single-cell QTL}

eQTL mapping in single cell experiments: [personal notes]
\begin{itemize}
	\item Consider an scRNA-seq experiment where we collect data across multiple time points/conditions. Additionally, samples are multiplexed, so each batch may contain a mix of multiple samples. 
	
	\item Creating pseudo-bulk data: first need to define cell types/populations. This is usually done by joint clustering analysis of all cells from all conditions. Downstreaming analysis is usually done on each cell type and each condition, separately. Sometimes, it may be better to use cell clusters defined from the joint clustering analysis, e.g. time point may not be a good marker of cell state [Cuomo, NC, 2020]. 
	
	\item Mapping eQTL by Linear mixed model (LMM): each sample is defined by genetic background (donor) and experimental batch (in multiplexed design). Both of which may affect gene expression. For a given gene, let $y_{ij}$ be its expression at donor $i$ in batch $j$. Our model will included fixed effects from eQTL $X_i$ and latent variables (e.g. PCs) $Z_{ij}$, and random effects from genetic background (which is shared across multiple batches for the same donor) and batches. 
	\begin{equation}
	y_{ij} = X_i \beta + Z_{ij} \gamma + g_i + b_j + \epsilon_{ij}
	\end{equation}
	where $g_i$ and $b_j$ are random effects. Their correlation structure is given by GRM and sample-repeat structure.
	
	\item Mapping response/condition-specific eQTL: [Cuomo, NC, 2020] uses ASE interaction test on cis-eQTL already mapped. This removes the batch effects. 
\end{itemize}

Science Forum: The single-cell eQTLGen consortium (sc-eQTLGen) [eLife, 2020]
\begin{itemize}
	\item Advantages of single-cell eQTL mapping vs. bulk eQTL: mapping genetic loci of cell types and states; trajectory; variability of genes across cells; coexpression network. Possible cost advantage: easy to multiplex to collect data of multiple samples in one experiment.
	
	\item Mapping cell-type specific eQTL: (1) define cell types, better to use supervised approach, e.g. use HCA as reference. Linear model is found to work as well as more complex models (e.g. DNN). Important to have a rejection option: i.e. a cell not belong to any known cell type/state. (2) Unsupervised clustering still important: to define un-annotated cells.
	
	\item Personalized GRN: to define co-expression relationship/network for an individual: technical challenge because of read sparsity.  (1) gene expression imputation (currently limited)  (2) Use prior GRN (e.g. TF to target), and combine with imputation. (3) Use temporal information and/or pseudo-time, RNA velocity.
	
	\item Challenge with trans-eQTL and co-expression QTL: sharing summary statistics is difficult because of large size.
	
	\item GRNs from scRNA-seq and association study: personalized GRNs can be used in multiple ways. (1) Association test of GRN topology. (2) Improve eQTL detection: use use network to define prior.
	
	\item Application to phenotypes: assess convergence of effects on genes or processes.
	
	\item Remark: problems/challenges: how to perform association analysis on trajectory?
	
	\item Remark: association of genotypes with features from GRN (e.g. co-expression or regulatory strength of TF-gene). Opportunities for method development: (1) Use all samples together to do GRN reconstruction for each sample. (2) Model uncertainty of the GRN features.
	
	\item Remark: sc-eQTL data would support MR/mediation analysis of gene > GRN features/cell composition/trajectory, because gene and cellular level traits (e.g. cell composition) are measured in the same samples.
\end{itemize}

Single-cell RNA sequencing identifies cell type specificcis-eQTLs and co-expression QTLs [van der Wijst and Franke, NG, 2018]
\begin{itemize}
	\item Background: Controlling for batch effects in scRNA-seq using PCA. The top PCs capture mostly technical artifacts, they correlate to the proportion of non-zero reads in cells. This could create problem for clustering: e.g. PC1 correlated with droput rate (batch effect) and the rates differ in different batches, so the samples are clustered by batches.
	
	\item Background: Imputation for scRNA-seq. In Fludigm - borrow information from cells, good results. Drop-seq/10x: no current imputation method works.
	
	\item Experiment: 45 individuals in 8 batches. 10x Genomics for scRNA-seq, total of 25K cells. In pooled individuals in one batch: identify individuals using SNPs from reads.
	
	\item Mapping eQTL with pseudo-bulk data (pool all cells of one sample): log-transformed read counts and quantile normalization.
	
	\item Validation: BIOS 2K samples and DeepSAGE, bulk-RNA seq. Found 8\% and 1\% of the cis-eQTL from these two studies using pseudo-bulk.
	
	\item Create clusters of cell types: cluster cells by PCs, obtain 11 cell types. Then manually annotate the cell types.
	
	\item Cell-type specific eQTL: union of all cell types: 379 cis-eQTL, most are found in bulk-like analysis. Among 48 new cis-eQTL, 29 are found in bulk of bigger samples. For 19 new cis-eQTL: 3 are validated using cis-eQTL of purified cells.
	
	\item Possible explanation of cell-type specific eQTL that are missed by bulk study: (1) Rare cell types. (2) Could also happen in common cell types: eg. Figure 1b, cis-eQTL in CD4 T cell (common), but the expression level of the gene is low, comparing with other common cells. This masks the signal. 
	
	\item Mapping co-expression QTL: choose CD4 T cells, 100 genes with cis-eQTL. Consider pairs these genes with other genes that are also significantly expressed. Estimate correlation of gene pairs in each sample, then co-expression coefficient vs. genotypes (limited to cis-eQTL). Found 2 cis-genes with co-expression QTL, Figure 2. 
	
	\item Q: what’s the mechanism of co-expression QTL? Ex. could cis-eQTL of one gene lead to co-expression QTL?
	
	\item Question: What’s the nature of the problem with PCA? If we properly normalize our data, do we get PCs that are uncorrelated with technical variables?
	
	\item Remark: dropout is also a problem for computing correlation.
\end{itemize}

Single-cell RNA-sequencing of differentiating iPS cells reveals dynamic genetic effects on gene expression [Cuomo and Stegle, NC, 2020]
\begin{itemize}
	\item Experiment: endoderm differentiation in 125 iPSC lines, SMART-seq and FACS for each cell. Pooled differentiation assay: 4-6 lines per experiment, then differentiation. Sample at Day 0, 1, 2, 3. Total: 36K cells, and 11K genes.
	
	\item Understanding sources of variation: LMM, time point as main source of variation, then cell line, then batch.
	
	\item PCA: using 500 most variable genes, PC1 (19\%) corresponds to time points (Figure 1c). Assign psuedotime: PC1, scaled to [0,1]. Alternative: diffusion map, principal curve form top 2 PCs.
	
	\item Assign cells to stages: use canonical marks of stages, which correlate with psuedotime (Figure 1d). About 20K cells can be assigned, and 7K not.
	
	\item Stage-specific cis-eQTL mapping procedure: group samples by individuals, days, and experiment (batch). Use LMM to map: treat individual genotype and batch (sample repeat structure) as random effects. Let $y_{ij}$ be expression of line $i$ from batch $j$:
	\begin{equation}
	y_{ij} = G_i \beta + g_i + b_j + \epsilon_i
	\end{equation}
	where $g_i$ is the random effect due to genetic background and $b_j$ is the random effect due to batch. The random effect terms are correlated across samples, given by GRM and sample-repeat structure (which donors occur in the same batch). Also adjust for 10 PCs of expression data. Response: log2-CPM. Note: because the same sample is used in multiple experiments, it’s important to correct for individuals as random effects.
	
	\item Results of cis-eQTL mapping: (1) about 1K genes in each stage. Importantly, the power is substantially higher than using days. (2) Stage-speicficity: abou 30\% are stage-specific. (3) 155 switching events: top eQTL change across stages. In 22 cases, see corresponding chromatin changes.

	\item Quantification of ASE (for dynamic and interaction eQTL): only performed for cis-eQTL. For a given eQTL, ASE was only quantified across cells from donors heterozygous for that eQTL variant. Donors that are not heterozygous at the eQTL variant of interest were not used for quantificatio. For each cell, quantification of ASE: (1) for each heterozygous exonic SNP, obtain ref. and alt. allele reads. (2) Aggregate SNP level to gene level alt. vs. ref. alleles, using phasing information. Then conversion to allelic fractions. 
	
	\item Mapping of dynamic and interaction eQTL using ASE: Figure 4a. Identify 60 clusters of genes. Use 4 clusters to represent cell states, respiration and G2/M transition. Use ASE, which is free from batch effects. (1) Dynamic eQTL: ASE effect is a function of pseudo time, modeled as: $\text{ASE} = \text{pseudo} + \text{pseudo}^2 + \epsilon$. (2) Cellular factors: $\text{ASE} = \text{pseudo} + \text{pseudo}^2 + \text{factor} + \epsilon$. (3) Pseudotime-factor interaction test: $\text{ASE} = \text{pseudo} + \text{pseudo}^2 + \text{factor} + (\text{pseudo} \times \text{factor}) + \epsilon$. Tests were only performed for eQTL for which ASE was quantified in at least 50 cells.
		
	\item Dynamic eQTL: use sliding window (25\% cells), and compute eQTL and ASE in each window. Then do regression of effects vs. pseudotime (linear or square). About 800 eQTLs show dynamic effects. Show several clusters of dynamic eQTL patterns.
	
	\item Possible mechanisms of dynamic eQTL: only partially correlated with gene expression changes over pseudotime.
	
	\item Cellular states modify eQTL effects: 668 eQTL that had an interaction effect with at least one factor (Fig. 4b). Ex 1: interaction of eQTL of RNASET2 with cellular respiration. Ex. 2: interaction of eQTL of SNRPC with G2/M transition. The patters are different: in SNRPC, G2/M has not much effect on expression on ref. allele. In RNASET2, respiration has a large effect on expression on both ref. and alt. alleles. 
	
	\item Biomarkers of differentiation: genetic effects not associated later trajectory, some genes are.
	
	\item Lesson: use LMM to study variance components of gene expression in such experimental design.
	
	\item Lesson: using data-derived stages may be better than actual experimental time in cell differentiation.
	
	\item Remark: interaction test is based on quantification of ASE in each cell. This is difficult to generalize to caQTL and may miss many interactions. 
	
	\item Remark: interaction eQTLs can result from different patterns. Ex. it may be driven by change of expression by conditions; or change of genetic effects. 
\end{itemize}

Population-scale single-cell RNA-seq profiling across dopaminergic neuron differentiation [Gaffney, biorxiv, 2020]
\begin{itemize}
	\item Experiment: 215 iPSC lines, differentiation, D11, D30 and D52, then oxidative stress. ScRNA-seq. Multiplexed design: 7 and 24 cell lines per experiment.
	
	\item Clustering: joint analysis of all cells from all conditions. 14 cell populations, including 6 dominant ones. Some are non-neuronal cells.
	
	\item iPSC line differentiation efficiencies differ: large variations of cell type composition at D52. Some lines are more efficient, and more importantly, the difference is reproducible and extend to other neuron differentiation protocols.
	
	\item What determines differentiation efficiency? Not chr. X inactivation, genetic background (multiple lines of the same individual show large difference). Gene expression signatures: explain difference, and in particular a cluster of genes are predictive of poor differentiation.
	
	\item cis-eQTL mapping: on 14 populations and conditions separately. LMM, include a random effect term to capture differentiation efficiency difference across donors (``variance term $1/n$’'), which influence the number of cells in a population (and accuracy of estimating expression).
	
	\item Results of cis-eQTL mapping: $>1500$ eQTLs for D11 and D30 in some common cell types, and 500-1000 for D52 and stress in common cells. For astrocytes, $<100$ eQTLs across conditions. Some examples of time point and condition-specific eQTLs (Figure 4b): effects much larger in one time point/condition than another (though some effects). Comparison with GTEx: about half are new.
	
	\item Colocalization with GWAS: coloc with 25 neurological traits. About half are detected in D52 and stress conditions. Two examples: (1) With SCZ: eQTL in stress, but not found in GTEx. (2) With SCZ: eQTL only in D11, but not in GTEx.
	
	\item Q: How does LMM work?
	
	\item Remark: (1) Gene expression signature of differentiation efficiency: enriched with neuronal lineage genes? (2) Differentiation efficiency may be explained by epigenetic states - the developmental competency model [Wang and Sander, CSC, 2015]. Hypothesis: if neuronal lineage enhancers are primed, then more likely to differentiate to neurons.
\end{itemize}

%%%%%%%%%%%%%%%%%%%%%%%%%%%%%%%%%%%%%%%%%%%%%%%%%%%%%%%%%%%%
\section{eQTL and regulatory QTL (rQTL) Studies}

Research problems of eQTL and regulatory QTL: 
\begin{itemize}
\item Genetic architecture of eQTL: how much is cis vs. trans? Spatial distributions? Tissue-specificity? Do trans-eQTL often cluster in co-regulated genes? 
\item Mechanisms of eQTL: TF-binding, chromatin structure, etc.  
\item Evolutionary explanation of patterns/genetic architecture of eQTL. Ex. why some genes have a lot more eQTL than other genes. 
\item Connection to complex diseases: phentypic consequences of eQTL.
\end{itemize}

Challenges of eQTL and regulatory genetics: 
\begin{itemize}
\item Variation of gene expression: the relative role of pioneering TFs and secondary TFs? How to identify pioneering TFs? 

\item Why variation of TF binding often do not lead to change of gene expression? 
\end{itemize}

Genetics of human gene expression: mapping DNA variants that influence gene expression. [Cheung \& Spielman, NRG, 2009]: 
\begin{itemize}
\item Tissue types: most are lymphoblastoid cell lines, or LCL (B cell lineage, or immortalized B cells), the other ones include: blood and adipose tissues, brain, lymphocytes, liver samples from surgical or cadavers. 
\item Comparison of primary cells vs cell lines: cell lines are relatively free of environmental influences. Ex. in human, the blood cell count, diet, medication, etc. can all influence gene expression in blood cells. 
\end{itemize}

The Genetic and Mechanistic Basis for Variation in Gene Regulation [Pai and Gilad, PLG, 2015]
\begin{itemize}
	\item eQTL mapping studies: 
	\begin{itemize}
		\item In [Battle et al], eQTL in 962 individuals, 78\% of genes were linked to at least one eQTL. Most of them are in 5' end, suggesting transcriptional regulation (rather than RNA decay) might be more important. 
		\item Possible to map trans-eQTL reliably. Estimate that more 50\% of heritability is from trans-eQTL. 
	\end{itemize}
	\item Approach: regQTL, find mechanisms underying eQTL by studying which part of the regulatory process is affected. 
	\begin{itemize}
		\item In LCLs, an estimated 10-20\% of eQTLs are also classified as methylation QTLs (meQTLs), suggesting a small proportion of loci that affect gene expression do so by perturbing DNA methylation levels.
		\item The problem of causality: meQTL may not explain the effect of eQTL, in other words, we may have two models, A) eQTL $\rightarrow$ DNA methylation $\rightarrow$ expression, or B) eQTL $\rightarrow$ DNA methylation and eQTL $\rightarrow$ expression. The two models can be distinguished by partial correlation analysis: under model B), DNA methylation and expression are uncorrelated if regress out eQTL. The data supports model B). 
	\end{itemize}
	\item Findings from regQTL studies: transcriptional regulation
	\begin{itemize}
		\item Chromatin accessibility (TF binding) seem to play a major role: 55\% of eQTL are found to be dsQTL. 
		\item Histone modification-QTL [Kasowski \& Synder, Science, 2013]: QTL of enhancer states (defined primarily by H3K27ac and H3K4me1) are often not eQTL. Explanatios: many enhancers defined in this way are not functional, compensatory change (including enhancer redundancy) or other means of buffering. 
		\item Histone modification-QTL [Kilpinen, Science, 2013; McVicker, Science, 2013]: changes in sequence-based affinity for TF
	\end{itemize}
	\item binding underlie a subset of observed histone-mark QTL (hmQTL). Also the effect is stronger on nascent RNA. 
	\begin{itemize}
		\item Additional support of TF binding as the driving event: [White \& Cohen, PNAS, 2013], enhancer assay to test activity of a large number of enhancers. The test sequences include: 1,300 sequences bound by Crx from ChIP-seq and 3,000 control sequences containing the motif, but not bound. Results: the Crx-bound sequences drive expression, while unbound sequences do not. In the experiment, the chromatin context has been removed, so the difference of activities must be due to local sequence features. 
	\end{itemize}
	\item Findings from regQTL: post-transcriptional regulation
	\begin{itemize}
		\item Splicing QTL (sQTL): many sQTLs fall directly within primary splice sites. However, also show an enrichment near TSS and 5'-UTR and TFBS. 
		\item RNA decay QTL (rdQTL): almost half of the nearly 200 rdQTLs identified showing counterintuitive associations with mRNA expression levels (higher rate of decay, higher expression). Overall, it was estimated that as many as 19\% of eQTLs might be driven by differences in mRNA decay. 
		\item protein QTL (pQTL): [Wu \& Synder, Nature, 2013] Only about half of these pQTLs were also found to be affecting transcript levels, suggesting many affecting regulation of translation or protein stability. Most of (80\%) the eQTL SNPs found in these
	\end{itemize}
	\item LCLs were not also associated with variation in protein levels: suggesting possible buffering mechanism. 
	\item Predict variation in gene expression levels based on the DNA sequence: 
	\begin{itemize}
		\item Challenge: many changes in TF binding do not seem to result in measurable changes in gene expression levels. [Cusanovich \& Gilad, PLG, 2014] knockdown expression of many TFs, only a small subset of genes inferred to be bound by a TF (using DHS or ChIP-seq data) were differentially expressed. 
	\end{itemize}
	\item Model of transcriptional variation: 
	\begin{itemize}
		\item The model that histone modifications regulate chromatin state, which in turn determine whether factors can bind to different sites need to be revised. Instead, TF binding seems to be the central event. 
		\item Possible model: pioneering TF binding concerted changes in histone marks, DNA methylation and nucleosome binding. Then chromatin areas that are accessible because of pioneer activity are available for binding by secondary factors. 
	\end{itemize}
	\item Lessons:
	\begin{itemize}
		\item regQTL paradigm to understanding mechanism of transcriptional variation, and the difficulty of establishing causality. 
		\item TF binding may be the driving event; however variation of TF binding often do not correspond to expression change. 
	\end{itemize}
	\item Remark: 
	\begin{itemize}
		\item The issue of power: only a small fraction of eQTL are meQTL, but the power of detecting meQTL might be low, and this should be accounted for.  
	\end{itemize}
\end{itemize}

\subsection{eQTL Studies in Human}

Genetic architecture of gene expression: 
\begin{itemize}
	\item Heritability of gene expression: 
	\begin{itemize}
		\item Estimation from families [Goring, NG, 2007]: 1240 individuals in 30 large families. Most genes (86\%) show significant heritability, but the level of heritability is moderate: 41\% had heriability $> 0.3$ but just 5\% had heritability $> 0.5$ . Thus non-genetic factors are also important. 
		\item Mean expression level difference across human populations: 17-29\% genes have significant difference in mean expression levels between pairs of HapMap populations [Stranger, NG, 2007]. It is likely that this is due to environmental factors: few SNPs have large frequency difference across populations, and the comparison of genetically similar groups in different environments show 37\% of expressed genes show significant difference [Idaghdour Y, PG, 2008]. 
	\end{itemize}
	\item Detection rates of eQTL: in a mouse study, QTLs were detected for only 27\% of genes with significant genetic differences in expression. 
	\item Effect size distributions: 
	\begin{itemize}
		\item Number of QTLs: most studies detected only a single locus for most expression traits. However, most expression traits should involve multiple QTLs because no single QTLs can explain most of the genetic variation. Ex. it is estimated that in yeast, only 3\% of expression traits are consistent with single-locus inheritance. 
		\item Effect size: consider only the most significant QTLs for the expression traits. In yeast, the median phenotypic effect of a detected QTL was 27\% of genetic (heritable) variance of expression; in mice, average 25\%, in human, 27-29\%.  
	\end{itemize}
	
	\item Type of complex inheritance: a large fraction - transgressive segregation (segragants fall outside parent means); and a small faction - directional genetics (segregants fall between parent means). 
	\item Cell-type dependent expression: a special form of gene-environment interactions. In mouses and rats, the genetic basis of variation of a gene's expression is sometimes shared between different tissues, but is often unique to each tissue [Cotsapas, Mamm Genome, 2006]. 
\end{itemize}

Local and distal eQTLs: 
\begin{itemize}
	\item Mechanisms of local eQTLs: polymorphism at (1) cis-regulatory sequences, (2) neighboring genes that control the expression, (3) coding sequences of auto-regulatory genes. 
	\item Mechanisms of distal eQTLs: polymorphism at (1) coding or cis-regulatory sequences of regulators, including TFs and indirect regulators; (2) distal enhancers. 
	
	\item Local vs. distal eQTLs: 
	\begin{itemize}
		\item Importance of local eQTLs: (1) In human, proximal eQTLs (within 2Mb) are much more common and most proximal eQTLs are close to the gene (within 100 kb), and distal eQTLs have much smaller effect sizes [Dixon, NG, 2007]. (2) Similar pattern in model organisms, eg. as many as 25\% of all expression traits in yeast are affected by local QTL. 
		\item Bias of detection: often difficult to detect distal eQTLs (especially in low-power studies) because of multiple hypothesis testing. Evidence: (1) in studies with large sample size, most transcripts are linked to distal eQTLs. Ex. [Yvert, NG03], among 2,294 expression traits, 578 show local linkage while 1,716 show distal linkage. (2) In human studies, no eQTL hot-spots (distal) are found, suggesting many distal eQTLs may be missing. 
		\item What are genes in distal eQTLs? From the existing studies, TFs were not overrepresented near the distal QTLs for 1,716 linkages in yeast [Yvert, NG03].
		\item A consistent result from most studies is that trans-eQTLs have weaker effects than cis-eQTLs. This view is contested by more recent studies that suggest that, despite lower effect-sizes, trans-eQTLs cumulatively explain more of the heritability of expression [Montgomery \& Dermitzakis, NRG, 2011]. 
		\item Within species, cis-eQTL explains about 35\%; across species, about 64\% [Regulatory changes underlying expression differences within and between Drosophila species, Wittkopp \& Clark, NG, 2008]
	\end{itemize}
	
	\item eQTL hotspots: a loci that affects expression of many genes. Not necessarily ``master regulators'', some of them have pleiotropic effects: e.g. a structural gene that significantly affects that phenotype. 
\end{itemize}

Tissue specificity of eQTL: 
\begin{itemize}
	\item Reference: [Tissue specificity of genetic regulation of gene expression, Goring, NG, 2012]
	
	\item MuTHER eQTL data: skin, subcutaneous fat and peripheral blood (LCL). 
	\begin{itemize}
		\item cis-eQTL: Overall, 50-80\% of the loci identified in one tissue were estimated to have gene regulatory effects in a comparison tissue. Effect sizes were often also comparable between tissues. 
		\item Trans-eQTL: largely tissue specific. However, this inference is less convincing than those made for cis eQTLs, because the power of detecting trans-eQTL is low in the first place. 
	\end{itemize}
	
	\item Proxy tissue problem: the question of whether or not eQTL information from one tissue is relevant for another is hotly debated. When eQTLs have not been reliably cataloged for the tissue affected by a disease, it seems entirely rational to try to use a proxy tissue. 
	
\end{itemize}

Methods of identifying gene regulatory networks in eQTL data: 
\begin{itemize}
	\item Module identification and module QTL (mQTL): similar to the usual analysis, modules can be found by clustering of expression patterns across different genetic pertubrations/backgrounds. The module QTLs can be identified via linkage/association of genetic markers with module expression. 
	\item Regulator finding of transcripts or modules: treating the transcripts or modules as complex traits, and apply similar strategy for integrating complex trait and eQTL data (the section ``Systems Genetics''). 
\end{itemize}

Small-scale study of human eQTL [Cheung \& Burdick, Nature, 2005]: 
\begin{itemize}
\item Data: 57 CEPH individuals, lymphoblastoid cells (immortalized B cells), expression of genes that show significant linkage in previous experiment.
\item Analysis: $\log_2$ transformed gene expression, regression on SNP genotypes (coded 0, 1, 2). The effect size is measured by $R^2$. 
\item Results: 
\begin{itemize}
	\item 374 expression traits: with evidence of previous linkage, association test with markers in the region with linkage in the previous experiment. There are 17\% expression traits with at least one marker that show evidience of association at nominal $P < 0.001$. 
	\item 27 expression traits: also with evidence of previous linkage, genome-wide test ($700,000$ markers). Out of 14 traits, significant association at nomimal $P < 6.7 \times 10^{-8}$. Most significant associations occur in cis- (within 50kb of 5' and 3'), only one trait has significant association in both cis- and in trans-. For non-significant traits, most have trans- association. 
	\item Experimental validation of one SNP (probably causal): about 2-fold didfference of expression with alternative alleles. 
\end{itemize}
\item Discussion: sample size estimation. In the ideal case where a single causal variant determines expression variation, to achieve a probability of 0.8 of detecting effect size $R^2$ of 0.1, the sample size of 500 would be needed. 
\end{itemize}

Human eQTL in asthma dataset [Dixon \& Cookson, NG, 2007]: 
\begin{itemize}
\item Methods: 
\begin{itemize}
	\item Data: 400 children from families recruited through a proband with asthma. 400,000 SNPs, measurement of 54,675 transcript (20,599 genes) in lymphoblastoid cells. 
	\item Analysis: FASTASSOC component of MERLIN, including sex (probably include family structure). LOD score 6.0 as threshold for significance, corresponding to FDR 0.05. 
\end{itemize}

\item Results: 
\begin{itemize}
	\item Heritability and eQTL: 14,819 traits have $H^2 > 0.3$, the peak LOD score for association: 3.7 to 59.1. 1,989 traits have peak SNP LOD score $> 6.0$, and about 33\% of $H^2$ in these traits can be explained by the peak SNPs. The GO category of most heritable genes: UPR, genes regulating cell cycle DNP repair, immune response, etc. Only 88 genes areassociated with three or more SNPs 
	\item Trans- and cis- ($>100$ kb) associations: (1) 13 SNPs showed association with ten or more heritable expression traits with lod scores $>6$; however, if limited to $H^2 > 0.3$ and remove MHC, only 3 SNPs are associated with five or more transcripts. (2) Trans-effects are weaker than cis-effects, and most LOD $>9$ were in cis-. (3) However, numerous distant associations were found: the peak of association for 698 transcripts was on the same chromosome but $>100$ kb away, and for 10,382 transcripts, the peak was on a different chromosome. 
	\item Application: a SNP is eQTL of the gene ORMDL3, and also a locus of childhood asthma, suggesting ORMDL3 is a candidate gene of the disease. 
\end{itemize}
\end{itemize}
 
Population genomics of gene expression [Stranger \& Dermitzakis, NG, 2007]
\begin{itemize}
\item Methods: 
\begin{itemize}
	\item Data: expression of Epstein-Barr virus-transformed lymphoblastoid cell lines of 270 HapMap individuals: 30 Caucasian trios, 45 unrelated Chineses, 45 unrelated Japanese, and 30 Yoruba trios. Choose 13,643 distinct genes for final analysis. 
	\item Association analysis across populations: either linear regression (LR) or Spearman rank correlation (SRC) at FDR 4-5\% - no significantly different results found. Either do association test within each population, or do pooled test in the whole with conditional permutation for correcting $P$ values ($P$ values will be inflated under conditional permutation). 
	\item Separate cis- and trans- test: For cis- analysis: permutation test within a region of 1 Mb. For trans- analysis: permutation test is too expensive, limit to 4 cateogires of SNPs: shown cis- effect, nonsynonymous SNPs, SNPs influencing splicing and SNPs within microRNAs, leading to about 25,000 SNPs. 
\end{itemize}
\item Results:
\begin{itemize}
	\item Population difference: 17 - 29\% expression traits show significant difference across pairs of populations. However, caution: may be due to different ages of cell lines from different populations. 
	\item cis- association: a total of 831 genes show association in at least one population, the pooled association test gives similar results. 
	\item The functional analysis of cis-associated SNPs: many SNPs may actually be causal, rather than markers because of the high density of HapMap markers. They are very close to TSS (within a few hundred kb), and symmetric in 5' and 3'. 
	\item Trans- associations: only 108 genes show significant association in at least one population. 
\end{itemize}
\end{itemize}

eQTL of human liver [Schadt \& Ulrich, PLoS Biol, 2008]: 
\begin{itemize}
\item Methods:
\begin{itemize}
\item Data: human liver sample from 427 Caucasian subjects. 
\item eQTL identification: association test using Kruskal-Wallis test. $P$ value threshold is determined by FDR: at any given $P$ value cutoff, do permutation (of sample labels) and count the number of predicted eQTLs, and compare this with the actual number of eQTLs. The $P$ value thresholds corresponding to FDR threshold 10\% are $5.0 \cdot 10^{-5}$ and $1.0 \cdot 10^{-8}$ for cis- and trans-eQTLs respectively. cis-eQTL defined as 1Mb of TSS or TES.
\end{itemize}

\item Results: 
\begin{itemize}
	\item cis-eQTL: about 3,000 genes significantly associated with at least one cis-eQTL. More than 30\% of all cis-eQTL are more than 100 kb away from TSS or TES of the corresponding gene (suggesting nearest genes may not always be the true target of a SNP). Comparison btween blood, adipose and liver cis-eQTL: about 30\% overlap, but the majority of cis-eQTLs may be tissue-specific. 
	\item trans-eQTL: 474 genes were found to have at least one trans-eQTL. 
	\item A more extensive eQTL set: use all significant $\sim 3,700$ SNPs from the two previous steps, and identified additional expression traits at FDR 10\%: 3,053 more expression traits or 2,838 genes (a total of about 6,000 genes). A number of eQTL hot spots (defined as more than 20 expression traits) emerged in this full set: highly significant. The total results are found in Table S2.  
\end{itemize}
\end{itemize}

Human cortical eQTL [Myers \& Hardy, NG, 2007]:
\begin{itemize}
\item Methods: 
\begin{itemize}
	\item Data: 193 neuropathologically normal human brain samples (postmortem). Correlations among 366,140 SNPs on the Affymetrix platform and the expression of the 14,078 detected transcripts (detected in at least 5\% of 193 samples).
	\item Statistical analysis: linear regression with additive model. Outliers due to genetic relatedness and ethnic bias were excluded. Corrected for several biological covariates (gender, age at death and cortical region) and several methodological covariates (day of expression hybridization, institute source of sample, postmortem interval and a covariate based on the total number of transcripts detected in each sample)
	\item Statistical significance: report both uncorrected Wald P values and empirical $P$ values from 1,000 permutations. 
\end{itemize}

\item Results: 
\begin{itemize}
	\item Sigificant associations: at empirical $P$-value 0.05, 433 SNP-transcript pairs (99 transcripts) show cis-association (defined as 1Mb from either end of the gene) and 16,701 SNP-transcript pairs (2,876 transcripts) show trans association. 
	\item Strigent associations: no variation located within the transcript probe, gene expression detection rate in samples greater than 99\%. 26 cis-associations (8 transcripts) and 336 trans-pairs (161 transcripts). 
	\item Positive control: MAPT expression and MAPT haplotype. 
	\item Comparison with LCL results: very few, two common cis-associations, and for stringent trans-associations, only one common transcript (but different SNPs). 
\end{itemize}
\end{itemize}

Human eQTL in LCL of HapMap individuals [Duan \& Dolan, AJHG, 2008]: 
\begin{itemize}
\item Methods: 
\begin{itemize}
	\item Data: HapMap lymphoblastoid cell lines from 30 CEU trios and 30 YRI trios, 12,747 transcript clusters (TCs) covering all exonic regions. 2 million SNPs.  
	\item Analysis: QTDT. FDR threshold 0.1 or $p$ value $2 \times 10^{-8}$. 
	\item eQTN blocks and eQTN hotspots: eQTL blocks defined as a region ocontaining one or more eQTNs associated with the same TC, where betwen eQTN interval $< 500$ kb. eQTN hotspots: a region asscoiated with more than one non-redundant TCs, where the bin size is 500 kb. 
\end{itemize}

\item Results: 
\begin{itemize}
	\item eQTL results: 4,677 significant TC-eQTN associations in CEU and 5,125 in YRI. In terms of TC and eQTN blocks (CEU): 741 unique TCs, out of which 67 TCs associated with 67 local (4Mb) eQTN blocks, and 691 TCs associated with 1,074 ditant eQTN blocks. 23 local TCs share eQTN blocks across populations (CEU and YRI), but none of 143 distnct TCs share the same eQTN blocks. 
	\item eQTN hotspots: 14 (CEU) and 38 (YRI) distant eQTN hotspots. eQTN harboring genes are enriched with nucleosome assembly genes, and membrane signaling. 
\end{itemize}
\end{itemize}

Polymorphic cis- and trans- regulation of human gene expression [Cheung \& Spielman, PB, 2010]: 
\begin{itemize}
\item Methods: 
\begin{itemize}
	\item Data: 45 CEPH families (3-generation, the family size could be big, e.g. 13 members in 13291), LCL. 
	\item Linkage analysis: regression of the phoneoytpe difference between siblings on the estimated proportion of marker alleles shared IBD between siblings. Only children data are used. 
	\item Association analysis: QTDT, using data of all members of CEPH families. 
\end{itemize}

\item Results: 
\begin{itemize}
	\item Linkage results: 70 expression traits have proximal regulators, 1,574 have distal regulators, and 37 have both. 94\% of distal regulators are in a different chromosome. 
	\item Association analysis of cis- linkage peaks: 63 of 100 expression traits with local eQTL show significant evidence - differential allelic expression by RNA-Seq. 
	\item Association analysis of trans- linkage peaks: for 1,611 (1,574 + 37) expression traits, define the trans-linkage peaks, then for each peak region, choose the SNPs of the candidate trans-regulators (coding and up/down-stream 5kb), and do association. at FDR 0.08, 917 expression traits have significant association with 742 trans-regulators, out of which 161 influence the expression of two or more genes. 
	\item Molecular validation of regulatory relationship: some pairs are known in literature (e.g. myoctye enhancing factor 2A, MEF2A and myosine regulatory light chain 2, MRLC2). Choose 25 regulators with modest evdience of linkage and association (QTDT $P$-value from $10^{-5}$ to $10^{-2}$), and do knockdown (only successful in 18 regulators). 13 target genes show significant change of expression after knockdown of regulators. Also verify INSR as the regulator of expression of 4 other genes in primary fibroblast. 
	\item Trans-regulator analysis: among 742 trans-regulators, 15\% are TFs, 19\% play signaling roles, the rest in metabolism, protein transport or modification in ER, etc. Also the target genes and trans-regulators are often in the same functional pathway/GO. 
\end{itemize}
\end{itemize}

Monocyte eQTL [Zeller \& Cambien, PLos ONE, 2010]: 
\begin{itemize}
\item Methods: monocytes of 1,490 unrelated individuals in GHS, 12,808 expression traits. 675K SNPs.  

\item Results: 
\begin{itemize}
	\item eQTLs: at $P < 5.78 \times 10^{-12}$, 37,403 associtions between SNPs and expression, invovling 29K SNPs and 2,745 expression traits. The number of cis- and trans- regulated expression traits are 2,477 and 349, respectively. Changing the threshold has a strong effect on cis/trans ratio (Figure 1). 
	\item Comparison with previous eQTLs:  $> 50\%$ of previousely identified cis-eQTLs ([Stranger07, Dixon07, Schadt08]) were replicated in GHS. However, trans-eQTLs were hardly replicated. 
	\item Application to GWAS results: check if GWAS SNPs are associated with any expression traits in GHS and if yes, test if the expression traits are associated with risk factors (BMI, blood pressure, etc.). Very few GWAS results were compatibile with an effect mediated by gene expression at the locus. It's possible that monocyte is not the relevant tissue. 
\end{itemize}
\end{itemize}

QTL of DNA methylation and expression in brain [Gibbs \& Singleton, PG, 2010]: 
\begin{itemize}
\item Methods: 
\begin{itemize}
	\item Data: cerebellum, frontal cortex, temporal cortex, and pons regions of 150 individuals (600 tissue samples). Test associations of 22,184 genes and 1,629,853 SNPs (after imputation and quality filtering). 
	\item Statistical analysis: linear regression with additive model (regression of allele dosage and trait). Report both nominal $P$-values and empirical $P$-values (1000 permutations).  
\end{itemize}

\item Results: 
\begin{itemize}
	\item Comparison of expression across brain regions: broadly similar (Figure 1D). Measures within frontal and temporal cortices were consistently the most alike and cerebellar tissue provided the most distinct profile of the four regions.
	\item Significant associations: using FDR threshold based on empirical $P$-values (perhaps $0.05$, a conservative threshold), about 5,000 associations per brain region, out of which about 75\% are cis-associations (Table S4). Number of transcripts from this analysis: ranging from 280 (3.2\%) in the pons to 391 (4.2\%) in the temporal cortex. 
	\item Comparison of eQTL across regions: The majority of large effect and many moderate effect QTLs were shared across the four brain regions (comparing the $R^2$ value across tissues). 
\end{itemize}
\end{itemize}

The architecture of gene regulatory variation across multiple human tissues: the MuTHER study [Nica \& Spector, PG, 2011]: 
\begin{itemize}
\item Methods: 
\begin{itemize}
	\item Data:  female Caucasian twins aged between 40 and 87 years old (mean 62 years) from the UK Adult Twin registry, 156 LCL, 160 skin, 166 fat samples. 865,544 SNPs with MAF $>1\%$ passed quality check (QC), on 18,170 genes. 
	\item Statistical analysis: Spearman rank correlation (SRC) on cis-associations (1Mb) for each tissue separately, $P$-value based on 10K permutations (permutation threshold, or PT, at $10^{-3}$). 
\end{itemize}

\item Results: 
\begin{itemize}
	\item Validation of cis-eQTLs in twins: use the twin data for validation (one analysis would use only one of a twin), the estimated proportion of true associations is very high: 0.93 in skin and 0.98 in LCL and fat. Even with eQTLs not replicated in twins, the estimated proportion of true associations is high: 0.84 for skin and 0.94 for LCL and fat. 
	\item Validation of cis-eQTLs with earlier datasets: 40\% of the genes for which we detect LCL eQTLs overlap with eQTLs detected in HapMap individuals. Likewise, 36\% of the cis associations detected by Gibson et. al. in LCL derived from 194 southern Moroccan individuals overlap with genes reported in our study. 
	\item Significant associations and tissue specificity: at PT 0.0001, 106 genes (12.35\%) are shared across all tissues, 139 (16.2\%) are shared in at least two tissues and 613 genes (71.44\%) are detected in only one tissue. SRC-FA (factor analysis) results confirm the estimated about 30\% of eQTLs to be shared in at least two tissues based on threshold eQTL discovery. Tissue-specific effects are largely not due to tissue-specific expression of the underlying transcripts, but the specific effects of SNPs. Even statistically tissue-shared eQTLs have additional dimensions of tissue-specificity and their mere discovery in multiple tissues does not guarantee similar magnitude of consequences.
	\item Genetic architecture: 7\% of the genes tested are regulated by more than one independent cis eQTL. 
\end{itemize}

\item Remark: the replication of LCL cis-eQTL from earlier studies: only genes overlap or the actual associations? 
\end{itemize}

Gene Expression in Skin and Lymphoblastoid Cells: Refined Statistical Method Reveals Extensive Overlap in cis-eQTL Signals [Ding \& Abecasis, AJHG, 2010]
\begin{itemize}
\item Background: (1) Psoriasis, an immune-mediated, inflammatory disease of the skin and joints. (2) Dimas09 estimated 69 to 80\% of cis-eQTLs are cell-type-specific (LCL, T cell, fibroblast). But the overlap may be underestimated because of the low power of detecting eQTL. 

\item eQTL mapping: SNP-gene expression associations separately in normal skin ($n = 57$), in uninvolved skin of patients ($n = 53$), and in lesional skin of patients ($n = 53$). The score test in Merlin (fastassoc option), limited to cis (1M) and one best eSNP per gene. $p$-value threshold $9E-7$, with FDR 0.01 in the three skin types.  

\item eQTL overlap across three skin types: identified 331, 275, and 235 independent cis-associations in normal, uninvolved, and lesional skin, respectively. 95.1\%, 96.7\%, and 98.7\% of the significant cis-eQTLs in normal, uninvolved, and lesional skin, respectively, were detected in the other two skin types (at $p < 0.05$).

\item Method of adjusting for eQTL overlap in two studies: given two studies, suppose in Study 1, at $p$-value threshold $\alpha_1$, the FDR is $FDR_1$. Among all eQTL passing the threshold in Study 1, the fraction $\pi_{\text{raw}}$ is replicated in Study 2 at $p$-value $\alpha_2$. Since the Study 2 has relatively low power, clearly, some of the eQTL in Study 1 may be true, but may fail to be replicated, leading to an underestimate of eQTL overlap. To correct for that, note in all significant eQTL in Study 1: 
\begin{itemize}
\item Fraction $FDR_1$: are false positives. Among these eQTL, $\alpha_2$ will be replicated in Study 2. 
\item Fraction $1 - FDR_1$: are true ones. Among them, fraction $\pi$ are also eQTL in Study 2 with probability $power_2$ to be replicated; and fraction $1 - \pi$ are not eQTL in Study with probability $\alpha_2$ to be replicated.
\end{itemize}
Thus we have: 
\begin{equation}
FDR_1 \cdot \alpha_2 + (1 - FDR_1) [\pi \cdot power_2 + (1 - \pi) \cdot \alpha_2]	= \pi_{\text{raw}}
\end{equation}
This allows one to compute $\pi$. Note that to estimate $power_2$, we first obtain $power_{2\text{raw}}$: suppose we estimate the effect sizes of all significant eQTL in Study 1, and estimate the power under this effect size distribution and the sample size of Study 2. Two corrections are needed: 
\begin{itemize}
\item False positives: among all significant hits in Study 1, not all of them are true. We have this equation: 
\begin{equation}
power_{2\text{raw}} = (1 - FDR_1) \cdot power_{2\text{approx}} + FDR_1 \cdot \alpha_2	
\end{equation}
Solving this to obtain $power_{2\text{approx}}$. 

\item Winner's curse: the effect sizes of the top hits in Study 1 are probably overestimated, so need to correct it too. The idea is to split the Study 1 data into two parts, one for identifying the significant hits, the other for estimating effect sizes. 
\end{itemize}

\item eQTL overlap across LCL and skin: at various $FDR_1$ and $\alpha_2$ thresholds, about 70\% LCL eQTL are also eQTL of skin (raw estimate without correction is about 30 to 40\%). 

\item Comparison of GWAS $p$-values of eQTL: 
\begin{itemize}
\item Extract independent eSNPs: from 9462 eQTL at $p < 9E-7$ (FDR 0.01), find 389 independent eSNPs, using linkage disequilibrium (LD) while favoring SNPs with stronger cis-association $p$ values. 
\item Compare the distribution of disease-association $p$ values of eSNPs and non-eSNPs (randomly sampled SNPs, excluding those within 1 Mb of regions known to be associated with psoriasis): QQ plot shows a trend for eQTL SNPs to be more strongly associated with psoriasis than non-eQTL SNPs. 
\item The overlap between eQTL signals and psoriasis associations: top eight in the QQ plot (Most Significant Psoriasis Association). 
\end{itemize}

\item Additional eQTL overlap analysis: the LCL cis-eQTLs in our analysis with cis-eQTLs identified in fibroblasts and T cells generated by Dimas. 65\%-70\% of significant LCL cis-eQTLs were also present in fibroblasts and T cells. 

\item Lessons:
\begin{itemize}
\item Measuring overlap of associations (or in general true hypothesis) across multiple studies: must take power into account. 
\item eQTL similarlity in normal and controls: the disease state may not significantly alter the eQTL. 
\end{itemize}
\end{itemize}

Single-Tissue and Cross-Tissue Heritability of Gene Expression Via Identity-by-Descent in Related or Unrelated Individuals [Price, PLG, 2011]
\begin{itemize}
	\item Estimation of $h^2$: use Icelandic data, about 700. LMM using IBD as relationship matrix. 
	
	\item Total heritability $h^2$ and contribution from cis $\pi_{\text{cis}}$: in blood, 42\% of genes have $h^2 > 0$ and in adipose, 63\%. In blood, $\pi_{\text{cis}}$ about 37\%, and in adipose, 24\%. In previous literature, 12\% in LCL. 
	
	\item Small s.e. of average $h^2$ in cis and large s.e. of average $h^2$ in trans (many genes, expression are affected by common factors, e.g. environment or biological networks, simultaneously). 
	
	\item Contribution of transgenerational epigenetic inheritance: compare $h^2$ in cis from IBD and unrelated individuals, very similar, suggesting that transgenerational inheritance makes a small contribution. 
	
	\item Cross-tissue comparison: 50\% shared cis-heritability, and close to 0 for trans.
	
	\item Lessons: 
	\begin{itemize}
		\item Generally higher standard error in estimation of $h^2$ in trans, while $h^2$ in cis is relatively robust. 
		
		\item Cis-heritability tends to be shared across cell types, while trans not. Thus tissues with mixed cell types have lower contribution from trans and higher from cis (blood $>$ adipose $>$ LCL). 
	\end{itemize}
\end{itemize}

Dissecting the regulatory architecture of gene expression QTLs [Gaffney \& Pritchard, Genome biology, 2012]
\begin{itemize}
	\item Data: LCL eQTL, for cis-eQTL, defined as upstream 100kb or downstream 100kb. Imputation using 1000GP data, 13.6M SNPs per individual. eQTL mapping shows that imputation helps with finding the causal SNPs (higher significance from imputed SNPs). 
	\item Model challange: the causal SNPs may not be known in each candidate region, so use Bayesian posterior prob. to account for the uncertainty. 
	\item Bayesian hierarchical model: 
	\begin{itemize}
		\item For each SNP, we use BIMBAM to relate its genotype with expression. The prior of effect size (given that the SNP is causal) is a mixture of five normal distributions. 
		\item The prior of SNP (binary indicator, whether it is causal or not): logistic function of multiple annotaitons (DHS, histone marks, etc.). For each annotation, the parameter $\lambda_i$ represents the enrichment of eSNPs in that annotaiton (interpretation is odds ratio of being an eSNP given that it has an annotation). 
		\item Empirical Bayes to estimate the parameters $\lambda_i$ combining all genes. 
		\item One additional difficulty is that many annotations correlate with spatial distributions, and eQTL also tend to occur in certain distribution, e.g. close to TSS. So define a background model that accounts for this spatial distribution: compare the model using annotations with the background model (which favors close regions). 
	\end{itemize}
	\item Enrichment of annotations in eQTL: LCL eQTL data and ENCODE annotations
	\begin{itemize}
		\item SNPs located within open chromatin are 4-times more likely to be an eQTL. Enrichment of other marks: approximately three-fold enrichment in H3K27ac; H3K4me1: about 2-fold. If limit to regions upstream more than 5kb of TSS: each of the three leads to 4-7 times enrichment (strongest for H3K27a). No enrichment of eQTL in regions with repressive marks.
		\item Total: 20\% of all eQTNs occur within DNaseI hypersensitive sites. Over 40\% of all eQTNs occur within either a DNaseI hypersensitive site or within a histone-modified region: even though the regions cover only 4.5\% of the genome. 
		\item Enrichment in TFBSs: strong enrichment in c-Jun and NK-kb binding sites (ChIP-seq)
		\item A large fraction of eQTNs occur very close to the TSS, and presumably affect the core and distal promoter architecture.
	\end{itemize}
	\item Sequence conservation: using PhastCons, PhyloP and conserved TFBS, surprisingly little enrichment (no enrichment at larger distance from TSS). Likely due to correction for distance in the background model. 
	\item Combined model: 
	\begin{itemize}
		\item Include all annoations including DHS, histone marks, motifs, TF ChIP-seq. Use AIC to select model. 10-fold cross validation shows that the combined model is better than the single or background model (higher likelihood). 
		\item Selected annotations: in regions $>5$ kb from TSS, H3K27ac is the dominate one, other features add relatively little. 
		\item The combined model aids the identification of causal eSNPs. 
	\end{itemize}
	\item Lessons:
	\begin{itemize}
		\item When there is a significant uncertainty of assigning to categories, frame enrichment testing problem as inference of parameters. 
		\item For distal enhancers, H3K27ac captures most of the information (comparing with DHS and other histone marks). 
	\end{itemize}
	\item Remark: 
	\begin{itemize}
		\item Lack of signal in conserved regions: could be due to evolutionary selection against variation of sequences in these regions, thus eQTL (which are common SNPs) are depleted. 
	\end{itemize}
\end{itemize}

Mapping cis- and trans-regulatory effects across multiple tissues in twins, [Grundberg, NG, 2012]
\begin{itemize}
	\item Data: Adipose, LCL, skin. 856 twins (1/3 MZ, rest DZ) 

	\item Distribution of h2: average 0.26 (adipose) – 0.16 (skin). Figure 1A. 
	
	\item Cis-eQTL mapping: linear mixed model. High replication rate $>0.7$.
	
	\item Estimated degree of overlap between tissues: Table 1. Two approaches: threshold and estimated proportion. Estimate that $>60\%$ of cis-eQTL have effects in multiple tissues. 
	
	\item Estimated heritability explained by cis-eQTL: $<15\%$ from common SNPs. Increase to $30\%$ or so with linkage: rare variants. 

	\item Trans-eQTL: $>60\%$ likely due to trans-eQTL. Found 500-1000 trans-eQTL in three tissues at $P<5E-8$. Many of these trans-eQTL are associated with multiple genes. 
\end{itemize}

Integrative Modeling of eQTLs and Cis-Regulatory Elements Suggests Mechanisms Underlying Cell Type Specificity of eQTLs, [Brown \& Engelhardt, PLG, 2013]
\begin{itemize}
	\item Data: eleven eQTL studies from seven unique cell types, LCL, brain, liver, blood fibroblast and T cells. 
	\item eQTL mapping: uniform pipeline. 
	\begin{itemize}
		\item Gene expression array: uniformly processed. Imputation. 
		\item Control for the confounding effects of both known covariates and unknown factors by removing the effects of principal components. For each dataset, do the regression analysis, controlling for covariates and PCs. Then project residual expression variation to the quantiles of a standard normal distribution to control for outliers, and used these projected values as the quantitative traits for association mapping. 
		\item Results represented by BF using BIMBAM. Assess FDR by permutation. 
	\end{itemize}
	\item eQTL results and replication studies across cell types: 
	\begin{itemize}
		\item Across all 11 studies, 29\% of eQTL associated genes are independently associated with at least two SNPs in at least one study. In one study, such fraction is 3-22\%. 
		\item Replication experiment: create trios, e.g. two LCL studies plus a liver study. Replication frequency is higher between the same cell types than between different cell types, and depends on a number of factors: log-BF, distance to TSS, etc. At high BF (log10-BF $>10$), high replication between the same cell types (50\% or higher), but lower between different cell types (20-40\%). This suggests that cell-type specific eQTL tend to have smaller effects. 
	\end{itemize}
	\item Intersection with CRE annotations to understand cell-type specificity. 
	\begin{itemize}
		\item Define activating CREs and repressive CREs, and intersect eQTL with annotations from LCL and liver. 
		\item cis-eQTL enriched for overlaps with several classes of CREs, including DHS sites, and depleted within regions in which a CTCF binding site lies between the eQTL SNP and the target gene TSS. Almost universally, QTL SNPs are enriched within regions of activating CREs and depleted within repressive CREs.  eQTL-CRE enrichment peaks immediately adjacent to the TSS for several classes of activating CREs, including H3K4me3 and H2A.Z.
		\item Intersected eSNPs tend to be cell-type specific: significantly more overlap between eQTL and CREs (DHSs) derived from the same type than from different cell types. Ex. SORT1 eQTL overlaps with a cluster of liver enhancers (but not in LCL). 
		\item The proportion of eQTL SNP - TSS pairs with intervening insulators is remarkably consistent across cell types, suggesting that CTCF binding sites do not substantially affect cell-specific eQTL function. 
		\item Random forest classifer to predict if an eQTL is likely to be active in second cell type, using CRE data from the second cell type. 
	\end{itemize}
	\item Lessons: 
	\begin{itemize}
		\item eQTL mapping across multiple datasets: control for confounding and hidden confounding variables. 
		\item Cell-type specificity: large-effect, proximal eQTL are more likely to be common across multiple cell types. A significant fraction of cell-type specific eQTL (at least 50\%) and they tend to overlap with CREs.
	\end{itemize}
\end{itemize}

Innate Immune Activity Conditions the Effect of Regulatory Variants upon Monocyte Gene Expression [Fairfax and Knight, Science, 2014]
\begin{itemize}
	\item Hypothesis: many SNPs have effects only under appropriate stimulations. So if we map eQTL in stimulated conditions, we will find identify more eQTL. 
	
	\item Data: 228 individuals, treatment with LPS and IFN-$\gamma$, a total of four conditions (treated or resting). 
	
	\item DE gene analysis: DE of canonical pathways, consistent with expectation of LPS effect. 
	
	\item Condition-specific eQTL: map cis-eQTL in each condition separately, and use surrograte variable analysis to increase the power of eQTL mapping. The majority of cis-eQTL were observed only after stimulation. And 54\% of resting eQTL were absent in stimulated condition. Limit to genes expressed in most samples: 33\% show eQTL after treatment. The stimulated eQTL include many immune-related genes such as TFs, key cytokines and receptors.  
	
	\item Trans-eQTL: some cis-eQTL are also trans-eQTL, possibly at a later time point. Ex. after 2-hr LPS treatment, cis-eQTL of IFNB1 was associated with expression of several genes at 24-hr in trans. 
	
	\item IRF2 cis-eQTL and direct targets: IRF2 cis-eQTL associated with 300 trans-genes and located in DHS. ChIP-seq targets of IRF2 enriched with trans-associated genes. 
	
	\item Utility for GWAS: overlap of cis-eQTL with GWAS, enriched for phenotypes such as bacterial infection, inflammation. Many GWAS loci overlap with eQTL specific to induced cells. 
	
	\item \textbf{Lesson}: eQTL and GWAS loci may become effective only under certain conditions/stimulations. 
\end{itemize}

Characterizing the genetic basis of transcriptome diversity through RNA-sequencing of 922 individuals [Battle \& Koller, GR, 2014]
\begin{itemize}
	\item Background: ASE is defined on individual data. It was found that imprinting or other non-genetic factors explain only a small fraction of ASE. Also ASE is generally not caused by the trans-acting factors, so an observed ASE is likely due to some genetic difference between alleles, or cis-mutations that cause different expression (cis-eQTL).  
	\item Data: RNA-seq from whole blood in 922 genotyped individuals from the Depression Genes and Networks cohort. 
	\item eQTL mapping: found eQTL in the large majority (78.8\% at FDR 0.05) of genes with quantifiable expression. Nearly half of SNPs in GWAS catalog are associated with some expression. 
	\item Proximal regulatory variation: 
	\begin{itemize}
		\item cis-eQTLs explaining a median of 3.3\% of expression variance (median 7.7\% among genes with an eQTL), compared to 0.7\% explained by age and sex combined. 
		\item Comparison with earlier studies of LCL: high replication rate ranging from 51\% to 89\%. 
		\item sQTL: using isoform ratio as a quantitative trait, found 2851 transcripts from 1370 unique genes with sQTL at FDR 0.05. Example: a SNP associated with LOAD is a sQTL, much stronger than its effect as eQTL.  
		\item ASE: found a set of regulatory variants consistently associated with allelic imbalance in nearby genes. Most ASEs in individuals can be explained by eQTL: 74\% of individual ASE events co-occurring with heterozygous status for the single best cis-eQTL SNP of the same gene. The remaining cases are candidates for rare regulatory variation. 
		\item aseQTLs: associations between heterozygous status at individual regulatory variants and allelic imbalance at nearby expressed coding loci. Confirm that 641 of our cis-eQTL SNPs are also associated with changes in allele-specific expression in the corresponding gene at FDR 0.05. 
	\end{itemize}
	\item Distal regulatory variation:
	\begin{itemize}
		\item Intra-chromosome eQTL: 381 genes affected by SNPs $>500$ kb away from TSS, including 269 genes affected by SNPs $>1$ Mb away. Find modules of coregulated genes, with 803 eQTL variants affecting two or more genes and 106 variants affecting three or more. Ex. (Figure 2B) rs11644386 affects a discontinuous group of genes, with the farthest association (CYLD) being $>400$ kb away, and does not have significant associations with two intermediate genes SNX20 and NOD2. 
		\item Trans-associations: 138 genes whose expression is associated with a distant SNP; and 5 trans-sQTL. Evidence of modularity: 20\% of associated SNPs affecting two or more genes. The largest module is a set of 57 genes, enriched for platelet aggregation function ($P < 10^{-7}$), all affected by the SNP rs1354034, previously associated with mean platelet volume.
		\item The majority of trans-eQTLs SNPs (76 of 138) also have cis-regulatory effects. Example: rs10251980, is a cis-eQTL for IKZF1, whose loss of function has been linked to leukemia (Mullighan et al. 2009). The SNP affects eight distant genes, five of which are up-regulated in response to tretinoin treatment in leukemia. 
		\item For trans-eQTL, the expression level of nearby genes mediate the trans effect 85\% of the time. 
	\end{itemize}
	\item Natural selection on eQTL: 
	\begin{itemize}
		\item Negative correlation between effect size of cis-eQTL and MAF. 
		\item Depletion of cis-eQTLs among genes with annotations suggesting critical roles in cellular functioning: highly conserved and hubs in PPI network, and TFs. May explain the scarcity of trans-eQTL. 
	\end{itemize}
	\item Genomic properties/mechanisms of eQTL: 
	\begin{itemize}
		\item Position: QTL enrichment near TSS. sQTLs are concentrated among exonic and intronic loci. Splice site, essential splice site, and stop gained functional annotations are particularly enriched for sQTLs, beyond the effects of position. 
		\item Regulatory annotations: both eQTL and sQTL are enriched in TFBS ChIP and DHS. Role of TF binding in splicing: could be regulation of expression of particular isoforms under different conditions (mechanisms such as cotranscriptional splicing)
	\end{itemize}
	\item Using genomic annotations to predict regulatory SNPs: LRVM. Given a gene and its adjacant SNPs (20kb near TSS), we define $a_i$ to be indicator of whether SNP $i$ is associated with the gene. For each SNP, we also define $d_i$ as its intristic regulatory potential (binary). The variable $d_i$ is related to annotation/feature vector of $i$. We assume that $a_i$ is related to $d$ of all SNPs, accounting for LD and MAF. Use the observed $a_i$ for each gene to learn the importance of features for regulatory potential $d_i$. 
	\item Remark/Questions:
	\begin{itemize}
		\item For eQTL of co-regulated genes: control for correlated expression, or independent association of SNPs with expression of multiple genes? 
	\end{itemize}
\end{itemize}

Heritability and genomics of gene expression in peripheral blood [Wright, NG, 2014]
\begin{itemize}
	\item Estimation of $h^2$: three strategies: 
	\begin{itemize}
		\item ACE model: let $y_i$ be phenotype, $x_i$ covariates, $a_i, c_i, e_i$ as additive, common environment and unique environmental contributions: 
		\begin{equation}
		y_i = \mu + x_i \beta + a_i + c_i + e_i
		\end{equation}
		The difference in MZ and DZ twins is the distribution of $a_i$: $a \sim N(0, \sigma_a^2 A)$, where $A$ is the genetic relationship matrix using the expected value. 
		
		\item IBD model: similar to ACE, except that $A$ is defined by the actual IBD between DZ twins. 
		
		\item GCTA model: use unrelated individuals only in the analysis. 
	\end{itemize}
	
	\item Data: Netherlands Twin Registry (NTR), 1,308 pairs including 690 MZ pairs and 618 DZ pairs. 
	
	\item Distribution of $h^2$: mean 0.10, $+/-$ 0.142. 777 genes with $h^2$ significantly greater than 0 at $q < 0.05$. $h^2$ correlated with expression mean and variance. Stabilizing estimation of $h^2$: use herirachical model, assume a Gamma prior. With this model, the proportion of genes with $h^2 > 0.3$ is 7.9\%. 
	
	\item Local genetic contribution: use GCTA to estimate contribution of cis-eQTL (use cis-SNPs for GRM). This ratio is only 0.04 (mean) or 0.09 (median). Use local IBD approach, 0.11 (median) or 0.3 (mean). 
	
	\item Mapping and validation of trans-eQTL: at $q < 0.001$, found about 600 eQTL, then apply additional QC, 348 robust trans-eQTL. 
	\begin{itemize}
		\item Additional QC: SNP and genes in LD; SNP alone (likely genotype quality problem), probe cross-hybridization and adjusting for local SNP. 
		
		\item Comparison with Westra data: in the eQTL found in Westra, estimated true discovery rate in NTR is about 23\%. 
		
		\item Pleiotropicity of trans-eQTL: on average, $\pi_1$ (fraction of associated transcripts) is about 0.001 to 0.008. 
	\end{itemize}
	
	\item Remark/Question: 
	\begin{itemize}
		\item Lack of correlation between $h^2$ from GCTA and from twins? 
		\item Replication of trans-eQTL: how is it calculated for Westra data? 
	\end{itemize} 
\end{itemize}

The human transcriptome across tissues and individuals [Mele, GTEx, Science, 2015]
\begin{itemize}
\item Tissue specificity of genes: use RPKM $>0.1$ as threshold (80M reads). 88\% of protein-coding genes and 71\% of lncRNAs are expressed in at least one sample. For many tissues, expression are dominated by $<100$ genes, e.g. hemoglobin in blood. 200 genes are exclusively expressed in one tissue, most in testis.  

\item Variation of gene expression across tissues and individuals: use LMM, 47\% variation from tissue and 4\$ from individuals. Genes with large individual variation: many in sex chromosome. 

\item Correlation of expression with age: 2,000 genes correlate with aging. Top one: EDA2R, and EDA is associated with age-related phenotype. 

\item Sex-specific expression: mostly sex-chromosome (X), suggesting escape of X-inactivation. 

\item Alternative splicing: the variation of isoform levels across tissues, largely explained by the gene expression (total isoform). 
\end{itemize}

The Genotype-Tissue Expression (GTEx) pilot analysis: Multitissue gene regulation in humans [GTEx, Science, 2015]
\begin{itemize}
\item Data: 237 donors, 28 tissues per sample, 6.8M SNPs, with MAF greater than 5\%. RNA-seq: 76bp PE, 82.1M reads per sample, use RPKM $>0.1$ as threshold. Tissues: 29 solid, 11 brain regions, whole blood, LCL, skin. 

\item Single tissue eQTL: cis-eQTL only, use Matrix eQTL. To correct for multiple testing, permutation for the most significant SNP per gene. The number of eGenes range from 900 in heart to 2200 in thyroid, with a total of 6K eGenes in 9 tissues. The majority of cis-eQTL cluster around TSS. 

\item Multi-tissue eQTL: 
\begin{itemize}
	\item Pair wise analysis to assess eQTL sharing: first find significant eSNPs in one tissue, then use the distribution of $p$-values of these pairs in the second tissue to estimate $\pi_1$, the proportation of non-null pairs. $\pi_1$ ranges from 0.54 to 0.9. 
	\item More than 50\% of all detected eQTL are common to all nine tissues. 
	\item UNC approach: use minP across tissues as test statistic, to permutation. Found 7425 eGenes with FDR $< 0.05$ - 3 fold increase relative to the number for single tissue. 
	\item Chicago approach: 10K genes show a signicant eQTL at the same FDR. 
\end{itemize}

\item ASE: 
\begin{itemize}
	\item Fraction of ASE: median of 6K sites that are heterozygous on one sample. About 2-3\% are ASE. Brain shows the lowest ASE. 
	\item What drives ASE (tissue vs. individual): average correlation of total reads (expression level) or allelic ratio between samples, either from different individuals, or from different tissues (three cases). For total count, higher correlation between samples than between individuals. For allelic ratio, higher correlation between tissues of the same individuals, suggesting ASE is primarily determined by the genome. 
	\item ASE validation of cis-eQTL: NDRG4 example, allelic ratio in the heterozygous (for the cis-eQTL) higher than 50\%. 
\end{itemize}

\item Splicing QTL:
\begin{itemize}
	\item Methods:  Altrans - association with expression levels of exon junction, both novel and annotated splicing. sQTLSeekerR - association with isoform ratios, only annotated splicing forms. 
	\item Average of 1900 genes with sQTL using Altrans, and 250 with sQTLseekerR. The sQTL detected by two methods follow different patterns, in particular not all sQTL involve different exon usage, e.g. complex 3' event. Most are not tissue-specific. 
\end{itemize}

\item Gene co-expression network: for any single tissue. The network overlap substantially across tissues: about 0.3 to 0.58 correlated genes are still correlated in a second tissue. Use WGCNA to find the modules: enriched with GO and ENCODE TF binding. 

\item Tissue-specific expression profiles: for any individual, we can group all genes by their tissue-expression profiles, then we merge such clusters from all individuals. The individual variation of clusters is small. But there are genes (21\%) that change modules across individuals. Identify SNPs correlated with module membership scores: modQTL, 58\% of which are not eQTL. 

\item GWAS and eQTL: enrichment of GWAS signal in eQTL from specific tissues. In 34\% of cases, eQTL-gene are not the nearest gene to SNP. 

\item Remark/Questions:
\begin{itemize}
	\item Most eQTL are not tissue-specific, but eQTL-GWAS enrichment analysis tend to be quite tissue-specific. And similarly enhancers are often tissue-specific. How to reconcile? Probably enhancer-eQTL are more tissue-specific than promoter-eQTL, and are more relevant to diseases? 
	\item What drives ASE? Only from cis-eQTL? The contribution from exonic SNPs (ex. NMD)? 
\end{itemize}
\end{itemize}

Effect of predicted protein-truncating genetic variants on the human transcriptome [Rivas \& MacArthur, Science, 2015]
\begin{itemize}
	\item Goal: protein-truncating variants (PTV) on transcription.
	\item Data: GTEx, Geuvadis, 462 individuals with WGS and LCL RNA-seq. Variant discovery: frameshift indels disagree with two datasets (100 vs. 16 per subject)
	 \item Transcripts with PTVs are expressed at lower levels and more tissue-specific (more tolerant). 
	 \item Splice junctions are less often used. 
	 \item Nonsense-mediated decay (NMD) pathway: exon juction (EJC). If a transcript has a premature stop-codon before EC, then SURF complex recognizes the stop codon and trigger NMD. Frameshift indels can cause NMDs: will always hit a stop codon prematurally. 
	 \item Calling indels: create reference genome that contains indels (heterozygotes)
	 \item Carole: easier to call deletions than insertions. 
	 \item mmPCR-seq: microfludic PCR-seq, targeted RNA-seq. Validation of allelic ratio. 
	 \item NMD-inducing variants tend to create more ASEs.
	 \item 50bp rule trigger NMD: only termination codons located more than 55bp upstream of 3'-most exon-exon juction trigger NMD. 
	 \item Validation of the rule: normal < NMD-escaping SNVs < NMD-triggerin SNVs. 
	 \item Of the ones triggering NMD: about 30\% show no ASE. 
	 \item Predictive model of NMD: 38 features, 50bp rule, distance to start/stop codon, etc, by Random Forest. Important features: number of downtream exons, distance to donor site. Implication: genes with large number of exons more likely to have NMD. 
	 \item Dosage compensation of PTVs: rare. 
	 \item Splicing-disrupting variants have signiature outside essential sites. 
	 \item Question: fraction of nonsense that trigger NMD? What explains tissue-specificity? 158 * 40% + 87 * 70
\end{itemize}

A systematic heritability analysis of the human whole blood transcriptome [Huan \& Levy, Human Genetics, 2015]
\begin{itemize}
	\item Data: FHS, known pedigrees, about 5000 in 700 families and 400 unrelated. Whole blood. 
	
	\item Method: (1) Estimation of $h^2$: variance component using pedigree. (2) Mapping eQTL: LMM, adjusting for relatedness, PC, cell type, etc. 
	
	\item Overall heritability: mean 0.07, 40\% genes have $h^2 > 0$  and 10\% genes have $h^2 > 0.2$. 
	
	\item Cis- vs. trans-eQTL: at FDR $<0.001$ threshold, (1) High $h^2$ genes have more cis-eQTL; (2) High $h^2$ genes have fewer trans-eQTL. In total, 3\% genes have trans-eQTL, among genes with $h^2 > 0.2$, 21\% has trans-eQTL. 
	
	\item Overlap of eQTL with GWAS loci of metabolic trait: a number of examples of trans-eQTL that are also GWAS, highly significant in both. Figure 4: a SNP in a regulatory gene is trans-eQTL of three other genes, and the SNP has $p < 10^{-9}$. 
\end{itemize}

Distant Regulatory effects of genetic variation in multiple tissues [Jo and Battle, bioRxiv, 2016]
\begin{itemize}
	\item Trans-eQTL mapping in GTEx data: 
	\begin{itemize}
		\item Found about 100 eGenes in 44 tissues, most tissues have $<5$ eGenes. Testis has 28. 
		
		\item If limit to cis-eQTL: add 14 new eGenes. 
	\end{itemize}
	
	\item Properties of trans-eQTL: (1)Tissue specificity: higher than cis-eQTL; (2) Higher enrichment in enhancers than cis-eQTL.  
	
	\item Pleiotropic effects of trans-eQTL: estimate $(1-\pi_0)^{27}$ (effect in at least one more tissue), about 3\%, much higher than cis-eQTL (close to 0). 
	
	\item Replication: a separate dataset, substantial enrichment of low p-values. 
	
	\item Examples of trans-eQTL: (1) FOXE1 locus in thyroid, broad transcriptional effect, corrected by PEER factors. Found 1085 uqniue trans-eGenes (FDR $<0.1$). (2) A SNP at KLF14 locus, also cis-eQTL, associated with many genes in trans in adipose. 
\end{itemize}

Systematic evaluation of genetic correlations between expressed transcripts in peripheral blood [Lukowski, review for NC, 2016]
\begin{itemize}
	\item Data: CAGE data, 1,748 unrelated individuals, peripheral blood. Limit the analysis to 2,469 transcripts with $h^2 > 0.25$.  
	
	\item Predicting pairs of genetically correlated transcripts: GREML, estimate the correlation of genetic (random) effects between two traits from all SNPs. Let $r_P, r_G$ be phenotypic and genetic correlation, $h_i, h_j$ be the heritability of two transcripts and $e_i, e_j$ be the environmental contribution, we have
	\begin{equation}
	r_P = h_i h_j r_G + e_i e_j r_E	
	\end{equation}
	where $r_E$ is the environmental correlation. 
	
	\item Results of genetic correlation analysis: among 2M pairs tested, 556 pairs with Bonferroni threshold, and 15K pairs with FDR $< 0.05$. For the strongest pairs (former), about 1/2 are trans-pairs; and for the latter, 94\% are trans. 
	
	\item Identifying shared eSNPs: to show that the results are replicated, identify shared eSNPs in a different study (2000 unrelated individuals). Method: for each pair, find the strong SNP in one, then test if it is associated with the other. Found 934 eSNPs with a significant effect in the second at Bonf. threshold. Next show that among 934 eSNPs (half trans), 100\% replicated in CAGE. 
	
	\item General issue of trans-eQTL replication: use Westra trans-eQTL, found large effect eQTL are often replicated, and the percent of replication correlates with effect sizes. Ex. at $z > 10$, most eQTL are replicated in at least one eQTL dataset. 
		
	\item Genetically correlated transcripts are enriched in chromatin-interaction regions. In Bonferroni pairs of transcripts, many are located in interacting loci from Hi-C, representing 25-90 times enrichment. Interpretation: these likely represent pairs of transcripts that are regulated by the same sequences (or eQTL). 
	
	\item Connectivity of genes in genetic correlation network: for each transcript, count the number of connections with $r_G >$ a threshold (2 or 3 $\sigma$, where $\sigma$ is the sd of $r_G$). At $r_G>2\sigma$, expect a gene to have 113 connections, and 7 for $3\sigma$. Found 2,317/2,468 transcripts with more than 113 connections and 2,397/2,468 with more than 7 connections. 
	
	\item Example from network analysis: two transcripts of the same gene, have different connections: some shared ones, but most are unique. 
	
	\item Hubs are enriched with TFs. Also evidence that pairs tend to share GO terms.  
	
	\item Remark/Lessons: 
	\begin{itemize}
		\item Trans-eQTL can drive the discovery of correlated genes; and can be replicated. 
		
		\item Significant number of co-regulation are local (perhaps due to common CREs). 
		
		\item High level of co-regulation among genes. 
		
		\item Need more analysis on the biological significance of co-regulation. 
	\end{itemize}
\end{itemize}

Heritability and Sparse Architecture of Gene Expression Traits [Wheeler, PLG, 2016]
\begin{itemize}
	\item Comparison of Polygenic and BVSR priors; also Elastic net.
	
	\item Fit each gene separately: adjust for cross-tissue term for each individual (shared across tissues of the same individual)
	
	\item Cis-h2 analysis: genes with high h2 tend to be less constrained (measured by pLI).
	
	\item Method to study sparsity: estimate PGE of sparse and polygenic components using BSLMM.
	
	\item Sparsity estimates using BSLMM in DGN data: (1) high h2 genes has high PGE (proportion of variance explained by sparse component); (2) low h2 genes: cannot estimate well the PGE.
	
	\item Lasso predicts better than Elastic Net: therefore more sparsity.
	
	\item Orthogonal tissue decomposition: identify cross-tissue components.
	
	\item Remark: the  analysis using EN doesn’t take power into account. With more samples, the polygenic component could benefit more, and higher PVE.
	
	\item Remark: the Cross-Tissue component captures shared expression, but could just reflect individual covarites (hidden). Not represent the shared eQTL effects.
\end{itemize}

Genetic mapping of the plasma proteome to inform drug development [Joseph Maranville, HG seminar, 2017]
\begin{itemize}
	\item SomaLogic platform for measuring 3K proteins in 3K individuals. DNA-based aptamers: barcoded, bind to proteins. Find the aptomaer that bind to protein. Potential problem: protein complexes may bind to aptamers together. But this is likely rare.
	
	\item pQTL mapping: Found 2K pQTL in 1.5K proteins, both cis and trans. Caution: cis-pQTL in coding sequences can alter aptamer binding.
	
	\item Intersection with disease GWAS: 88 pQTL, 253 GWAS signals. Do coloalization, 61 pQTL, 11 cis and 50 trans.
	
	\item Example: cis-pQTL of IL23 and IBD GWAS. Cis-pQTL of MMP12: disprove the gene of CVD gene by MR.
	
	\item Drug target analysis: among 600 targets, 111 have cis-pQTL. Could use them as IVs to study their impact on GWAS.
	
	\item A cis-pQTL of MST1: the gene binds to RON receptor, and RON KO has IBD phenotype.
	
	\item The cis-pQTL is also trans-pQTL of 8 other proteins. Near 8 trans target genes: 3 independent GWAS signals.
	
	\item Model: MST1 affects protein, then ROS signaling, which affect several trans-targets, some of which influence IBD.
	
	\item Comparison with eQTL: 60\% cis-pQTL has no cis-eQTL (could be lower power).
	
	\item Remark: in the case of colocalization with trans-pQTL, it is hard to find the causal gene.
	
	\item Remark: a better analysis about the relationship of cis-eQTL and cis-pQTL is: heritability of pQTLs mediated through eQTLs.
\end{itemize}

Integrated genome-wide analysis of expression quantitative trait loci aids interpretation of genomic association studies [Joehanes and Munson, GB, 2017]
\begin{itemize}
	\item Framingham Heart study (FHS): 5K samples, whole blood eQTL. QC: polymorphism-in-probe effect is likely minor. 
	
	\item Mapping eQTL: (1) Adjusting for observed covariates and family relationship with LMM: obtain residuals. (2) Adjusting for hidden covariates using PEER factors, and obtain p-values. (3) Multiple testing correction: for all associations with $p < 10^{-4}$, do BH correction to obtain FDR. Also adjusting for p-value inflation by Genomic Control, however, the results are only slightly affected. Note: correction is done separately for cis- and trans-eQTLs. 
	
	\item Obtaining independent eQTL: step-wise regression to find independent eQTL. Found 19K independent cis-eQTL and 6K trans-eQTL. Sample size is important for the power: it scales linearly with cis-eQTL, but more with trans-eQTL. Double the size from 2500 to 5000 increase the trans-eQTL by 3-4 fold.
	
	\item Validation of eQTL: (1) Internal: 75\% cis and 41\% trans-eQTL are validated. (2) With previous studies: 50-70\% previous cis-eQTL and 30-60\% trans-eQTL are replicated. The replication in the other way is low due to lower power in previous work and different sequencing platforms, etc; but still 90\% of cases the directions are consistent. 
	
	\item Distribution of eQTL: highly enriched in transcribed regions, especially first exons and 5’ UTRs (45 fold). Modest enrichment (2-fold) of trans-eQTL in regulatory regions. 
	
	\item Clusters of trans-eQTL: 59 clusters with 6-200 genes. Some are due to genetic effect on cell type composition. In some clusters, found enrichment of TF motifs in promoters and miRNA targets. The majority 90\% of trans-eQTL are not in any of the clusters. 
	
	\item GWAS analysis: with CAD/MI 58 loci, 21 loci or 36\% are lead cis-eQTL. Also an example where a SNP is the trans-eQTL of a cluster of genes (SH2B3 locus).  
\end{itemize}

Functional Architectures of Local and Distal Regulation of Gene Expression in Multiple Human Tissues [Xuanyao Liu and Price, AJHG, 2017]
\begin{itemize}
	\item Using S-LDSC for estimating enrichment and heritability: let $C$ be an annotation (possibly overlapping), and $\tau_C$ be per SNP heritability of SNPs in $C$. Then $h^2$ of $C$ is the sum of SNP heritability of all SNPs in $C$. This allows us to define: (1) Proportion of $h^2$ by $C$: it is $h^2$ in $C$ divided by $h^2$ summing over all categories. (2) Enrichment of $C$: defined as (1) divided by the percent of SNPs in $C$. Consider an example with two categories $C_1$ and $C_2$: let $M_1, M_2$ be numbers of SNPs in the two categories, and $M_{12}$ be the SNPs in both. Let $\tau_1, \tau_2$ be per SNP heritability of $C_1, C_2$, then we have: 
	\begin{equation}
	h^2(C_1) = \tau_1 M_1 + \tau_2 M_{12} \qquad h^2(\text{total}) = \tau_1 M_1 + \tau_2 M_2
	\end{equation}
	where the second term of $h^2(C_1)$ is from SNPs shared with $C_2$. 
	
	\item Applying S-LDSC to eQTL: apply it to all genes with $>0$ heritability. Then average $h^2$ over all genes for each category separately. To obtain SE of estimates, using block jackknife: 200 genomic blocks (each block preserve cis-eQTL dependency).  
	
	\item Estimation of enrichment in cis-eQTLs: include 50 annotations for joint estimation. Strongest: 5'UTR, TSS, conservation (7-10), promoter, enhancer (5). Estimates are generally consistent across different tissues, and different sample sizes. 
	
	\item Genetic correlation between tissues: generally high between tissues in cis-eQTL, but very low (around 0.1) in trans-eQTLs.
	
	\item Remark: heritability explained by a category $C$ is somewhat inflated, as it includes contribution of all SNPs in $C$, but some SNPs obtain bigger effects from other annotations. 
\end{itemize}

Genetic effects on gene expression across human tissues [GTEx, Nature, 2017]
\begin{itemize}
	\item Procedure for multiple testing correction: control number of eGenes at a given FDR. Use fastQTL: fit the null distribution of min. p-value per gene using Beta distribution. Then control FDR using q-values.
	
	\item Defining eVariants: (1) Determine the global threshold for min. p-values $p_t$: find a gene closest to FDR 0.05, and the empirical p-value for that gene. (2) Need to determine the threshold for each gene: from the null distribution of min. p-values, determine the threshold for $p_t$ (convert $p_t$ to the threshold using CDF). Then any SNPs below this threshold will be selected as eVariants for that gene.
	
	\item Identification of additional eQTL per gene: using forward-backward stepwise regression.
	
	\item Results of cis-eQTL per tissue: generally a few thousand eGenes at FDR $< 0.05$ per tissue. Replication by ASE: among ascertained eVariants, how often they are replicated in ASE (at nominal p $<0.01$). The replication rate drops with distance to TSS, but saturated at 1.3Mb.
	
	\item Enrichment of cis-eQTLs in functional annotations: Enrichment test: compare percent of eQTLs in an annotation vs. control SNPs to estimate log-OR. Do this for Roadmap annotations. Enrichment is higher (could be $>10$) for matched tissues (Figure 3a). Also do this for multiple variants per gene: generally higher enrichment in promoters than enhancers; but for secondary SNPs, the gap is smaller (Figure 3c).
	
	\item Shared eQTL across tissues are more likely to share CREs (Figure 3b).
	
	\item Fine-mapping and validation: use CAVIAR to fine-SNPs, and obtain PIPs of all in the credible sets. Correlation of PIPs with the proportion of eQTLs localized in DHS (Figure 3d).
	
	\item Characterizing contributions of different functional elements to cis-eQTLs: compare effect sizes of eQTLs in 3’ UTRs, exons, splice sites, noncoding, etc. Smaller effects in untrasnlated regions comparing with upstream regulation (Figure 3e), except canonical splice sites.
	
	\item Replication of effect sizes in ASE: highly correlated effect sizes (Extended Data Figure 6)
	
	\item Trans-eQTL: 673 trans-eQTLs found, 113 have cis-associations. Removing PEER factors (15-35 factors) have major effects on trans-eQTL detection: Figure S13, in testis and stomach, much fewer trans-eQTL after removal. Testis has a large number of trans-eQTLs: about 28\% (12 independent trans-loci) overlap with piRNAs. 
	
	\item Enrichment of GWAS variants in eQTL: (1) eGenes: tissue-shared eGenes are less likely to be in OMIM or LoF intolerant. (2) 50\% of GWAS loci are associated some gene expression: at $p < 0.05/44$ (44 is number of tissues). (3) In these variants, even with very strong cutoff, 10-20\% cases the top genes vary across tissues (Figure 5c)
	
	\item Co-localization analysis of GWAS and eQTL: Figure 5d, 21 traits, about 50\% of GWAS loci show colocalization; about half of cases are in nearest genes.
	
	\item \textbf{Lesson}: strategies for validation of cis-eQTLs (1) Use an independent dataset, e.g. ASE. Higher pi1, or replication rate; effect size correlation. (2) Spatial distribution: highest near TSS, but also saturate as one move to large distance. (3) Enrichment of functional annotations. This analysis can use different tissue-specific annotations (irrelevant tissues serve as negative control).
	
	\item Lesson: validation of fine-mapped SNPs, correlation of PIPs with functional features.
\end{itemize}

Impact of Genetic Polymorphisms on Human Immune Cell Gene Expression (DICE) [Schmiedel and Vijayanand, Cell, 2018]
\begin{itemize}
	\item Data: 91 samples, three innate immune cells, classical and non-classical monocytes and NK cells. B cells, naive CD4, CD8 and T-reg cells. Also six memory and differentiated T cells and two activated T cells.
	
	\item Results of eQTL mapping: about 2k-3k eGenes per cell type.
	
	\item Cell type specificity of eQTLs: many cell type specific effects, plot Z-scores across cell types (large blocks); in contrast, cell type specificity of RNA levels are lower (Figure 3F). Some examples: Figure 3G, GAB2, eQTL has opposite effects in B and T cells. Many eGenes are found only in activated T cells.
	
	\item Integration with GWAS: some example, LACC1 is involved in controlling T cell response. Found eQTL of LACC1 (only in T cells), also GWAS SNP of AID. The eQTL reduces expression of LACC1 in activated T cells: confirm that K.D. of the gene reduces T cell production of cytokines after stimulation.
	
	\item Lesson: cell-type specificity of genetic effects are probably higher than cell-type specificity of gene expression (or any molecular traits).
	
	\item Remark: missing the global pattern of cell-type specificity of eQTLs.
\end{itemize}

\subsection{Regulatory QTL studies}

Variation in Transcription Factor Binding Among Humans [Kasowski \& Snyder, Science, 2010]
\begin{itemize}
	\item ChIP-seq data of NF-kappB and RNA Pol 2 in 10 human and 1 chimp. 
	\item Calling peaks from ChIP-seq: 
	\begin{itemize}
		\item Calling peaks in each sample with PeakSeq at $p < .001$. 
		\item Cluster peaks from different replicates of the same individual into BRs. Peaks not replicable in at least two were discarded. 
		\item Join BRs across individuals by combining intersecting events 
	\end{itemize}
	\item Test differential binding: 
	\begin{itemize}
		\item Normalization: normalize numbers of reads in a BR across all replicates and samples using quantile normalization.
		\item Comparison between individuals: ANOVA, using a generalized linear model with Poisson error distribution, with Bonferroni correction (N being the number of pair-wise comparisons made). 
	\end{itemize}
	\item Spearman correlation of binding across replicates: median 0.95; correlation of binding across individuals: median 0.90. 
	\item Difference between individuals: 7.5\% and 25\% of the NFκB and PolII binding regions, respectively, differed significantly between two individuals (Poisson ANOVA comparison). The signal intensity in ``lost'' peaks are similar to background level, suggesting that the peak is completely absent, instead of the threshold effect. 
	\item Variability of BRs: BRs near TSS show less variability than intergenic ones. 
	\item Human-Chimp comparison of Pol 2 binding: analysis of human BRs with syntentic regions in chimp (81\%). Binding differences between the chimpanzee cells and each of the ten human samples were identified: on average 32\% of BRs show significant differences in binding (corrected $P < .05$). 
\end{itemize}

DNase I sensitivity QTLs are a major determinant of human expression variation [Degner and Pritchard, Nature, 2012]
\begin{itemize}
	\item dsQTL mapping: use non-overlapping 100bp windows as test units, and top 5\% of windows in DNase reads as phenotypes. For each peak, test association of normalized read counts with SNPs within 40kb windows. 
	
	\item Read count normalization: correct for library size and GC content first. Then standardized and quantile normalization. Finally, adjust for PCs: varying the number of PCs. 
	
	\item Allele specific analysis: for only significant dsQTLs. Only consider those with 90 reads or more, then estimate the proportion of major alleles. To compare with dsQTL: now use raw read counts as the dependent variable and do linear regression. Use the estimated parameters to estimate the proportion of major alleles. 
	
	\item Calibration of null distribution of $p$-values: Figure S13. $P$-values are uniform under two types of permutations: for each test window $w$, either permute genotype labels of samples; or test association of $w$ to the SNPs in a random genomic location. Note: in sample permutation, pheotypic relations (among peaks, and PCs to peaks) are preserved. 
		
	\item Determining statistical significance: significant associations found at FDR $10\%$ using q-values. Correct for windows of different intensities (10 bins) separately, using different significance threshold varies (because of power difference). Top windows: many more significant associations at the same FDR (Figure S12). Most associations are found for top 1\% of windows.   
	
	\item Validation of dsQTLs: QQ plot of dsQTLs 2kb (stronger signal) vs. 40kb (Figure 1a). Correlation of effect size of dsQTL and allele-specific effects, $r = 0.72$ (Figure 1b). 
	
	\item Distribution of dsQTLs in functional sequences: 41\%in predicted enhancers, 26\%in promoters, and 10\% in insulators, even though those chromatin states together cover only 6.7\% of the genome overall. 
	
	\item dsQTLs and TF binding: 3.6-fold enrichment of dsQTLs in TF footprints (controlling for overall enrichment in DHS). In dsQTLs, higher accessibility alleles show higher PWM scores and ChIP-seq reads. Also correlation of chr. accessibility and TF binding (Figure 2e): all positive including CTCF. Note: use ChIP-seq data from LCL. If using other tissues, correlations much weaker (Figure S14).
	
	\item Overlap of dsQTLs and eQTLs (Table S4): (1) for each dsQTL, consider a window and all expressed genes in that window. For all dsQTL-gene pairs, vary window size and estimate the proportion of eQTLs via $\pi_1$ analysis. Results: 41\% at 10kb and 16\% at 100kb. (2) Estimate the percent of eQTLs that are significantly associated with a gene (all pairs): for each window, use FDR correction for all pairs (p-value threshold differs between window sizes). Found 1027 significant dsQTL-eQTL pairs at FDR $<10\%$ for 100kb, or 5.3\%. This is the maximum number of pairs, so use 100kb for analysis. 
	
	\item Estimating proportion of dsQTLs that are eQTL of at least one nearby gene (Suppl 19.1): consider all genes within 100kb of a dsQTL, obtain its minimum p-value, and adjust for multiple genes using permutation - so each dsQTL has a single p-value. Then do $\pi_1$ analysis: about 39\% dsQTLs are eQTLs. Similarly, do FDR correction on p-values (one value per dsQTL), at FDR $< 0.1$, found 809, or 16\% dsQTLs that are eQTLs of at least one nearby gene. 
	
	\item Estimating proportion of eQTLs that are dsQTLs: (1) Calling eGenes: 1200. (2) For each eQTL (strongest per eGene): find strongest dsQTL p-value within 100kb, then use permutation to adjust p-values. Estimate percent of dsQTLs using Storey's method: about 55\% eQTLs are dsQTLs.
	
	\item Example: dsQTL in a gene intron that disrupts a CRE, and show corresponding gene expression variation (Figure 3ab). 
	
	\item Joint dsQTL-eQTL analysis: (1) Directional comparison: for 70\%, directions are consistent. The signal is especially strong if limiting to DHS within 1kb (Figure 3b). (2) Motif enrichment in enhancers (same direction) and repressors (opposite direction): different motifs. (3) Spatial distribution: 23\% are within 1kb of TSS of the gene, and 39\% within 10kb (Figure 4a). 
	
	\item Possible mechanisms that control whether dsQTL is also an eQTL (Figure 4b): distance to TSS, trascribed region, CTCF between DHS and TSS. 
\end{itemize}

Identification of Genetic Variants That Affect Histone Modifications in Human Cells [McVicker \& Pritchard, Science, 2013]
\begin{itemize}
	\item Histone and Pol II-QTL data: 10 unrelated Yoruba, 4 histone marks (H3K4me1, H3K4me3, H3K27ac, and H3K27me3) and Pol II binding. 
	
	\item QTL mapping and calibration: define testable SNPs (enough reads) and test association with reads in 2kb windows around the tested SNPs. Test results are calibrated (Figure S4): permutation of haplotypes and/or two alleles (flipping alleles randomly). Then control FDR by $q$ values, and merge overlapping windows after FDR correction. 
	
	\item At an FDR threshold of 10\%, we identified 582 distinct histone mark and Pol II QTLs. 
	
	\item Correlation between histone QTL, dsQTL and eQTL: the effect of QTL on multiple molecular phenotypes
	\begin{itemize}
		\item Most histone mark and Pol II QTLs are within 1 kb of a DHS, but many are far from known dsQTLs. 
		\item At dsQTL or eQTL sites, the histone marks are often different at different alleles. Individuals who are homozygous for the high-expression genotype generally have higher levels of DNase I sensitivity, H3K4me3, H3K27ac, and Pol II occupancy at transcription start sites (TSSs). 
	\end{itemize}
	\item TF binding sites in histone-QTL and PolII-QTL: 
	\begin{itemize}
		\item Method: evaluate whether polymorphisms in TFBSs are associated with allelic imbalance in histone marks or Pol II. Map all TFBSs in 10 individuals containing polymorphism, than test if changes of PWM scores correlate with allelic imbalance in histone marks or Pol II. 
		\item Increased transcription factor occupancy generally increases levels of nearby activating histone marks and lowers the levels of H3K27me3.
		\item Identification of specific transcription factors that direct histone marking:  Out of the 39 clusters (similar TFs) that have a sufficient number of polymorphic TFBSs to be testable, 11 have a significant association with at least one histone mark. 
	\end{itemize}
	
	\item Lessons: Non-coding variation (often TFBS) can affect multiple molecular traits simultaneously: TF binding, histone marks, DNase HS, PolII occupancy and gene expression. 
\end{itemize}

Methylation QTLs Are Associated with Coordinated Changes in Transcription Factor Binding, Histone Modifications, and Gene Expression Levels [Banovich and Gilad, PLG, 2013]
\begin{itemize}
	\item Mapping meQTLs: 450K CpGs, enriched in promoters. Quantile normalization. Cis-meQTL: within 6kb of SNPs to max. power. Found 13K CpG sites with QTL, mostly independent. Only 13\% are outside 3kb.
	
	\item Overlap with eQTLs: given eQTL, for those within 3kb of CpG, 25\% are also meQTLs. However, for those both meQTL and eQTL (150), half have positive correlation.
	
	\item Overlap with regulatory QTLs (DHS, histone): often see coordiated changes of multiple molecular traits (Figure 3). Also Table 1: 30-40\% regulatory QTLs (K4me1/3, K27ac, DHS) are also meQTL (vs. 5-10\% by chance). More common to have negative correlation, but a substantial fraction have positive correlation.
	
	\item TF binding may drive meQTLs: SNPs disrupting TF binding in DHS (using CENTIPEDE), more likely are meQTLs 10-15\% vs. 2-3\% by chance (Table 2).
	
	\item Lesson: effect direction of methylation on expression is uncertain.
\end{itemize}

RNA splicing is a primary link between genetic variation and disease [Yang Li and Pritchard, Science, 2016]
\begin{itemize}
	
	\item Data: 8 molecular phenotypes on about 60 LCL lines. One new phenotype is 4sU-seq, which measures transcription (initiation) rate - how labeled uridine is incorporated into mRNA. 
	
	\item Information flow from chromatin to protein: (1) Correlation between molecular traits: relatively low correlation ($<0.5$) for H3k27ac and the rest; high correlation (above 0.8) between transcription-related traits (including RP); moderate correlation with protein. (2) comparison of effect sizes of QTL, H3K27ac-QTL effects correlated with other traits, r about 0.5, and the Txn rate, RNA and RP-QTL all highly correlated, r about 0.9. Transcription-QTL and protein-QTL: r about 0.7-0.8. 
	
	\item Sharing of QTL across phenotypes: $\pi_1$ is about 0.25 to 0.5 for enhancer-QTL with eQTL, 60\% promoter-QTL are eQTL. Other pairs highly shared: e.g. txp rate-QTL, more than 70\% are eQTL. 
	
	\item What explain eQTL? About 60\% eQTL are chromatin-QTL. Unexplained QTL are enriched in txn elongation, exons, introns, etc. 
	
	\item Splicing QTL (sQTL): detection by LeafCutter. Found nearly 3K new sQTL. Mostly independent of eQTL: (1) Different spatial distribution. (2) In cases where eQTL and sQTL are in the same genes: lead SNPs often $>10$ kb apart. (3) sQTL are enriched in splice regions, coding, introns while eQTL enriched in chromatin features. 
	
	\item Splicing can be affected by chromatin features: show that chr. annotations are enriched in sQTL (vs. control). An example CTCF binding site. 
	
	\item Enrichment of sQTL in GWAS loci: using fgwas or PolyTest. In autoimmune diseases, "sQTLs appear to have effects of similar or even larger magnitude than eQTLs”. Remark: this does not mean that sQTL explain more disease risk, since the number of sQTL vs. eQTL is not taken into account. 
	
	\item Pipeline for peaking calling: for H3K27ac, MACS peak calling for each sample separately, then merge overlapping peaks. The MACS windows then split into 1kb segments (if longer).  
	
	\item Pipeline for QTL mapping: (1) Use WASP to adjust for differences in sequencing depth and GC (2) standardize all measurements by gene and then quantile-normalize by individual (3) Regress out PCs, the number of PCs (0 to 15) is chosen to maximize the power. 
	
	\item Estimation of sharing QTL between phenotypes: e.g. eQTL vs. pQTL. Choose top SNP-gene eQTL pairs, e.g. p $<$ 1E-4, then obtain the p-values of these pairs in pQTL, estimation of $\pi_1$ using Storey’s method, R qvalue(). Use bootstrap to compute confidence interval for $\pi_1$. 
	
	\item Estimation of QTL sharing using Bayesian approach: joint model of QTL effect sizes across molecular phenotypes. For each SNP-gene pair, model the effect size (vector) as a mixture of 0 and MVN. Combine all pairs to estimate the MVN covariance matrix and mixture parameters. However, it’s not clear how it is used to estimate QTL sharing. 
	
	\item Estimation of fraction of eQTL that are chromatin QTL (partition of eQTL): choose a set of genes, their top eQTL, and then ask how often they are chromatin QTL by p-values. Comparison with control SNPs. 
	
	\item Testing enrichment of annotations: comparison of sQTL (or unexplained eQTL) vs. chromatin eQTL. For these eQTL, first obtain the posterior causal probability (PCP) for each SNP. Then for each annotation, sum PCP over all SNPs, which is the expected number of causal SNPs fallen into the annotation. Compute the fold difference between two types of eQTL, and obtain confidence interval by bootstrap. 
	
	\item Q: how the Bayesian model of QTL sharing is used to estimate proportion (Figure S3)? 
	
	\item Q: Computing PCP: not handle LD? 
\end{itemize}

Disease variants alter transcription factor levels and methylation of their binding sites [Bonder and Teijmans, NG, 2017]
\begin{itemize}
	\item Data: 3,800 whole blood, methylation and expression.
	
	\item cis-meQTL: most are within 10kb of CpG sites. Modest enrichment in active TSSs and enhancers.
	
	\item cis-eQTM: expression Quantitative Trait Methylation. Most negative effects (69\%), but some positive effects. Show different enrichment patterns: e.g. negative enriched in TSS and active enriched in polycomb. A decision tree that classifiers the two using histone marks and distance to TSS.
	
	\item Trans-meQTL: using 6000 SNPs previously associated with disease traits. 1/3 of them have trans-meQTL effects.
	
	\item Possible mechanism of trans-meQTL: change of [TF] in cis. Figure 4a: often trans-meQTL of multiple CpGs show consistent effect directions. Ex. NFkB cis-eQTL, changes methylation of many CpG sites, enriched with NFkB targets. Similar pattern with CTCF.
\end{itemize}

Genetic determinants of co-accessible chromatin regions in activated T cells across humans [Gate and Regev, NG, 2018]
\begin{itemize}
	\item Data: CD4 T cells from 100 healthy donors, stimulation (CD3, CD28) and ATAC-seq profiling. Also RNA-seq. Also in situ Hi-C: 3.5B reads for CD4+ T cells. 
	
	\item Chr. accessible peaks: difference due to stimulation. To characterize peaks: enrichment of known annotations, e.g. Th1, Th2, Th17, T-reg. Change of enrichment patterns for different sets: T-rest, or T-stim or shared. 
	
	\item Enrichment of GWAS variants: three groups of peaks, however, the enrichment pattern is similar, mostly immune phenotypes, but also fatty acid levels high enrichment. Remark: no comparison with whole blood, or other cell types. 
	
	\item Analysis of differential TFs in cell-type specific and shared peaks (Figure 2): e.g. in peaks specific to stimulated T cells, find TFs (via footprint analysis) important for CD4+ T-cell activation or differentiation. In shared peaks, found CTCF. Show footprint difference in different conditions (footprints changed after stimulation). 
	
	\item Co-accessibility of peaks across individuals: (1) both 1Mb and 100kb resolution (Figure 3c), matching Hi-C interactions. (2) Between peaks within 1.5Mb bins, find 2000 pairs, often involving enhancers. Co-accessible peaks are closer than expected: median distance 380kb. Discussion: perhaps peak calling bias, e.g. call one peak into two adjacent peaks (actually one peak). 
	
	\item Local caQTL mapping: only analyze 60K SNP-containing peaks, and SNPs within peaks - total of 150K SNP-peak pairs tested using Rasqual. Use permutation: ``-r'' option of Rasqual, 10 permutations. Found 3K local caQTLs (SNPs within peaks) at FDR $< 0.05$. These QTL explain most of cis-heritability. 
	
	\item Mechanism of local caQTL: enrichment near TSS and TES. (1) 900 local ATAC-QTLs often disrupt TF binding: use DeltaSVM to predict the effects of SNPs on TF binding (pre-trained model), about half local ATQC-QTLs disrupt six TFs (Figure 4d). (2) Local ATAC-peaks that overlap BATF, ETS1 and CTCF binding sites: different chr. accessibility along TFBS footprints (aggregate over all targets of the TFs, Figure 4f). However, only 5\% of the corresponding local-ATAC-QTLs directly alter the core motif sequences.
	
	\item Prediction of ATAC-QTL effects with delta-SVM: train gkm-SVM using the peaks, then predict SNP effects using delta-SVM. Correlation of deltaSVM scores with ATAC-QTL effects: R = 0.6 (Figure 4e).
	
	\item Enriched heritability of local caQTL in GWAS: 5-8 fold enrichment of local caQTL in AID associated loci. LDSC analysis: 50-60 fold enrichment of AID heritability, however, not significant after adjusting for multiple testing (?)

	\item Example: a local ATAC-QTL, associated with several AIDs, Disruption of BATF motif. 
	
	\item Genetic determinants of co-accessibility. 

	\item Expression effects of caQTLs: eQTL mapping using RASQUAL. Found 424 genes to be associated with at least one local caQTLs within 500kb. Estimate 30\% of caQTLs are eQTLs. 
\end{itemize}

Impact of regulatory variation across human iPSCs and differentiated cells [Banovich and Gilad, GR, 2018]
\begin{itemize}
	\item Experiment: RNA-seq and ATAC-seq in LCL, iPSC (from LCL), about 60 lines. And 14 lines of Cardiomycytes from iPSC (CMs).
	
	\item Calling caQTL and eQTLs in CMs: even with only 14 lines, WASP is able to call 500 eQTLs and 4000 caQTLs.
	
	\item Cell type specific caQTLs explain cell-type specific eQTLs: LCL caQTLs better explain LCL eQTLs (Figure 2A: QQ plot). Note: see Supplementary Materials, Section 10 about definition of cell-type specific QTLs: eQTL, FDR $< 0.1$ in one and p value $> 0.05$ in the other. caQTL, $p < 5 \times 10^{-4}$ in one and $p > 0.05$ in the other. 
	
	\item Cell-type specific caQTLs tend to be more open in that cell type: Figure 2B: heat map, e.g. LCL-specific caQTL are much more open in LCL than iPSC and iPSC-CM. Note that the percent depends on cutoffs to define caQTLs. 
	
	\item What drives cell type specific caQTLs? Comparison of caQTLs between iPSC vs. LCL. Largely driven by chromatin accessibility across cell types, but about 20\% of iPSC-specific caQTLs are located in LCL accessible regions (Figure 2E: for iPSC caQTLs, plot of ATAC-seq signal in two cell types).
	
	\item Possible mechanism of cell-type specific caQTLs: build DNN to predict chromatin states, high AUC. Show that predicted effects correlate with measured caQTL effects (R = 0.5), Figure 3E. Example of SNP showing caQTL effect only in iPSC, and motif change.
	
	\item Remark: the proportion of cell-types specific caQTL that are due to chromatin accessible change, may vary depending on comparison. For distant cell types, most are likely driven by chromatin state changes.
\end{itemize}

Co-occurring expression and methylation QTLs allow detection of common causal variants and shared biological mechanisms [Pierce and Ahsan, NC, 2018]
\begin{itemize}
	\item Data: 900 subjects for eQTL and 300 for meQTL (300 with both). Find 5000 cis-SNP-expression-methylation.
	
	\item Coloc. analysis: results are sensitive to the prior of coloc, $p_{12}$. The number of pairs passing threshold vary greatly (nearly 10 times) with 10-fold change of the prior. Use “internal empirical calibration” (similar to EB) to choose the prior.
	
	\item Results of coloc: about 2700 triplets. Results strongly depend on LD between the top cis-eQTL and top cis-meQTL. Also depend on LD score.
	
	\item Partial correlation analysis: reject pleiotropy, if $G$ affects $M$ methylation and $E$ expression independently, then regressing out $G$ in $M$ and $E$, the residuals should be uncorrelated. About 10-20\% pass threshold.
	
	\item Mediation analysis: Sorbel test in either M to E, or E to M. About 10-20\% pass threshold in either direction. Note that mediation analysis cannot distinguish causality.
	
	\item Inferring causal relation between methylation and expression: (1) Direction of effects: eQTL and meQTL often have opposite effect direction, about 58\%. This increases to 70-80\% in loci with some causal evidence found from partial correlation and mediation analysis. (2) Bayes network analysis: most often M $>$ E is selected.
	
	\item Lesson: four types of analysis in the paper to learn the model of $M$ and $E$. Each has limitations:
	\begin{itemize}
		\item Partial correlation analysis: test $H_0: G \rightarrow E, G \rightarrow M$. Test: regress out $G$, and correlate residuals of M and E. It does not allow confounder, so it is a weak null model. Most often, there is a causal effect or some confounders acting on M and E. 
		
		\item Mediation analysis: cannot distinguish if there is a causal effect or causal direction. Even if we reject null from partial correlation analysis, and find a significant mediation, we still cannot say if there is a causal effect: could be a shared confounder. 
		
		\item MR: similar to co-localization. Could be due to LD between eQTL and meQTL. 
				
		\item Bayes network: to explicit model the $(G, M, E)$ data, need to assume that there is no confounder. 
	\end{itemize}
\end{itemize}

Genome-wide identification of DNA methylation QTLs in whole blood highlights pathways for cardiovascular disease [Huan and Levy, NC, 2019]
\begin{itemize}
	\item Data: 415K CpG sites in 4000 whole blood samples (FHS). Samples: 456 unrelated, and the rest from 500 families.
	
	\item MeQTL mapping and genetic architecture: (1) 100K (25\%) CpG have $h^2 > 0.1$. Most often, single best cis-eQTLs do not explain all h2: mean 0.07 vs. 0.18. (2) Total: 394K independent cis-meQTL (within 1Mb) and 21K trans-meQTLs, where independent are defined by $r^2 > 0.2$.
	
	\item Features of me-QTLs: enrichment in promoters, TSS, enhancers.
	
	\item Causal CpGs of CVD and related traits using MR: merge nearby CpGs with high LD (CpGs within 2Mb and highly correlated with each other, often share meQTLs). Find 14K CpGs with 3 independent cis-meQTLs, do MR with CVD traits: found 92 CpGs.
	
	\item Finding target genes of CpGs: cis-association of CpGs and genes, then do co-localization analysis of eQTL-meQTL. Found 8 genes: test their roles in traits using MR, with cis-eQTLs as IVs.
	
	\item Trans-meQTL hotspots: 22 hotspots, at least 30 CpGs. In 74 genes near these hotspots: enrichment of TFs (ZNF genes).
	
	\item Remark: CpG-gene causal test to link CpGs to genes.
\end{itemize}

Common DNA sequence variation influences 3-dimensional conformation of the human genome [Gorkin and Bing Ren, GB, 2019]
\begin{itemize}
	\item Experiment: dilution Hi-C in 20 LCL samples.
	
	\item Derive features: 40kb contact matrix, compartmentalization (PC1 from Hi-C), directional index (DI), insulation scores (INS) - measure abundance of interactions spanning a region (strong insulation, low INS). TAD boundaries are associated with low INS and high DI, but high DI can also occur elsewhere. FIRE scores: frequency of contact of a region with nearby regions (15-200kb).
	
	\item Remark: is INS score properly normalized? Low INS could just mean that the region doesn’t have many interactions, e.g. gene desert.
	
	\item Inter-individual variation of chromatin structure features: generally correlations within an individual (replicates) are higher than between individuals.
	
	\item Detecting variable regions across samples: use limma, find regions whose variations across individuals are higher than within individuals. Found several thousand in each features. An example in Fig. 2A: different DIs.
	
	\item Covariation of variable regions with other epigenomic features: FIRE regions correlate with H3K27ac and H3K4me1. DI and INS regions correlate with histone modification levels as well as CTCF and Cohesin binding.
	
	\item CTCF-changing SNPs and chromatin loops: (1) Define chromatin loops: from external data. (2) Consider SNPs that change CTCF motifs in chromatin loop anchors. Found overall association with strength of chromatin loops.
	
	\item Q: How does one infer the strength of individual chromatin loops from Hi-C data?
	
	\item Mapping Hi-C QTLs: use hi-C derived features, and use LMM to include biological replicates, use 11 YRI samples for discovery. Features: FIRE, DI, INS, and contact frequency, but not PC1. For DI and INS: use window size of 200 Kb up- stream and downstream of the target bin, instead of larger bins. Choose test bins (40kb), and test all SNPs within that bin and SNPs in perfect LD. Found a few hundred SNPs per feature at FDR $<$ 0.2.
	
	\item Effects of HiC-QTLs: often have effects on other epigenomic features. Ex. FIRE-QTLs, the high-FIRE allele is also associated with higher levels of active histone modifications and chromatin accessibility.
	
	\item Enrichment of HiC-QTLs in GWAS variants of AIDs: 1.6 - 1.7 fold enriched of nominal GWAS variants. Enrichment higher than the test bins.
	
	\item Remark: HiC-QTL mapping only use derived features, but not individual contact pairs.
	
	\item Remark: no discussion of the mechanisms of HiC-QTLs, enriched with CTCF motifs/targets? Or enriched in TAD boundaries?
	
	\item Remark: consequence of HiC-QTLs, do they tend to be eQTLs of multiple genes?
\end{itemize}

Genetic drivers of m6A methylation in human brain, heart, muscle, and lung [Xiong and Kellis, review for NG, 2020]
\begin{itemize}
	\item Data: 102 m6A MeRIP profiles, 4 tissues, 76 samples. Average 42K sites per sample.
	
	\item Tissue-specific patterns of m6A: Samples clustered by tissues: with brain most different, and heart/muscle cluster together, closer to lung. Overall, about 45\% m6A peaks are shared. For most tissue-specific m6As: genes are universally expressed.
	
	\item M6A QTL discovery: m6A normalization: only m6A levels, not mRNAs. Use fastQTL, search for SNPs in promoters, introns and exons. Found 400 genes in brain, 400 in lung and 200 in heart/muscle (merged), at empirical p-value $<$ 0.005. This “empirical” p-values are defined using FastQTL: for min-p, do permutation to obtain its empirical p-value. Then use beta-binomial approximation to get the p-values. Note: no FDR correction.
	
	\item M6A-QTL validation: use only two samples, check effect size direction, whether consistent with discovery cohort. 82\% and 62\% respectively. Effect direction in the two samples: compare two genotypes (only SNPs that have different genotypes are chosen), and see if difference is consistent. Remark: let $\pi$ be the TP proportion, for TP, expect 100\% agreement, and for FP: 50\%. Then we have 1*pi + 0.5 * (1-pi) = 0.72, so we have pi = 0.44.
	
	\item Tissue-specificity of m6A-QTL: $>$90\% are tissue-specific (Fig. 3A), only 1-2\% are significant in multiple tissues. Within a tissue: very small percent shared m6A-QTL; 70\% are m6A QTL in one tissue but m6A peaks are also in another tissue; 24\% are m6A-QTL in one tissue and also peak in one tissue (Fig. 3B). Using a more relaxed threshold of p = 0.05 in a second tissue, still 94\% are tissue-specific. Plotting: Fig. 3D, show m6A-QTLs in tissue 2, but different shapes for m6A-QTLs found in tissue 1 - show that most tissue 1 m6A-QTLs have insignificant p-values, and low effect sizes.
	
	\item Comparison with m6A-QTLs in LCL: little agreement in effect directions.
	
	\item M6A-QTLs are modestly enriched with eQTL GTEx ($<$2 fold), but effect directions not consistent for shared m6A-QTL and eQTLs (Fig. 4b).
	
	\item GWAS enrichment: 55 GWAS traits, m6A-QTLs at p-value 1E-3 or 1E-4 (Fig. 5A). Use S-LDSC, remove m6A-QTLs overlapping with enhancers. See enrichment in tissue-specific fashion: e.g. lung m6A-QTLs in lung-related traits; muscle/heart not in NPDs. 13 traits show enrichment in multiple tissues, including height, BMI, and blood pressure. Remark: some not very tissue-specific patterns, brain QTLs enriched most strongly in height and BP; brain QTLs in lung traits.
	
	\item Some examples of GWAS SNPs that are m6A-QTLs: Fig. 5 de. In both cases, close to top GWAS, and top in m6A-QTL, not eQTL.
	
	\item Putative m6A regulators: enrichment of RBPs sites in m6A-QTLs using GARFIELD. Found 27 regulators in total (at p $<$ 0.05, no multiple testing adjustment), including known readers DF2 and FMR1. Show PPI with known M6A regulators, writers, erases (source: stringdb R package).
	
	\item Correlation of regulator expression with m6A targets (Fig. 6c): some additional genes at p $<$ 0.1. Correlation of [RBP] with m6A targets (require m6A-QTLs to be inside RBP binding sites). Remark: correlation extremely weak.
\end{itemize}

%%%%%%%%%%%%%%%%%%%%%%%%%%%%%%%%%%%%%%%%%%%%%%%%%%%%%%%%%%%%
\subsection{QTL Studies in Model Organisms}

eQTLs in yeast: [Brem \& Kruglyak, Science, 2002]
\begin{itemize}
	\item Methods: 
	\begin{itemize}
		\item Data: expression of 6,215 genes are measured in 40 segregants, genotyped at 3312 markers. 
		\item Test linkage between a marker and an expression trait: partition the segregants into two groups, and compare the expression levels between the groups with Mann-Whitney test. 
	\end{itemize}
	
	\item Results: 
	\begin{itemize}
		\item Detection rate: 1528 messages show differential expression in parental strains. Linkage analysis: 570 messages show linkage to at least one locus (308 among 1528 parental different messages). 
		\item Local eQTLs (within 10kb): 185 out of 570 messages show linkage in this category. 
		\item Distal eQTLs: eight hotspots (Figure 3) - 40\% linkages fell into one of the eight groups of genes controlled by these hotspots. The group size: 7 to 94 genes. Examples: leucine biosynthesis - linked to Leu2, Ura3 (enzymes); fatty acid metabolism - linked to TF Hap1; daughter-cell specific genes - linked to AMN1 (protein required for daughter cell separation, not TF); Msn2/4-dependent genes - show Msn2/4 consensus sites.  
	\end{itemize}
\end{itemize}

Trans-acting regulatory variation in yeast [Yvert \& Kruglyak, NG, 2003]: 
\begin{itemize}
	\item Methods: 
	\begin{itemize}
		\item Data: 86 segregants from a cross between BY and RM strains. 
		\item Module QTL identification: treat the mean expression of a module as a quantitative trait, and do QTL analysis. 
	\end{itemize}
	
	\item Results: 
	\begin{itemize}
		\item Gene clustering: by pairwise correlation of expression patterns exceeding 0.725 (threshold determined by permutation test). Found 593 clusters of at least two genes and 205 clusters of more than two. Clusters often are enriched with specific processes: e.g. hexose transport, pheromone response, daughter-cell specificity. 
		\item Trans-acting loci: a total of 304 clusters show linkage to at least one position. Overall, 75\% of genes and 80\% of clusters did not show self-linkage: most genetic variation is expression is due to trans-acting factors.
		\item Case studies: GPA1 is a causal locus of pheromone response cluster with nonsynonymous substitutions in the protein sequence; and AMN1 a causal locus of daughter cell separation cluster. 
		\item Most trans-variations do not map to TFs: the regulatory TFs of the gene clusters (according to ChIP-chip data) are not overrepresented in the module-QTL of these clusters. In fact, most trans-eQTLs are not mapped to TFs. Many classes of genes were found in trans-eQTLs.  
	\end{itemize}
\end{itemize}

Genetic landscape of gene expression in yeast [Brem \& Kruglyak, PNAS, 2005]: 
\begin{itemize}
	\item Problem: estimate the genetic models of transcripts, e.g. the number of loci. 
	
	\item Methods: 
	\begin{itemize}
		\item Data and analysis: similar to [Brem02], but with 112 segregants, $2,957$ markers. 
		\item Power of study: suppose the true model is a main locus plus many with infinitesimal effects, then the study has $>90\%$ power of detect genes with a main locus of effect size $25\%$. 
		\item Estimating genetic models: e.g. the fraction of transcripts with single QTL. The idea is: e.g. suppose all genes are controlled by a single locus, then in simulation (with this genetic model for all transcripts, and apply the same QTL analysis and threshold), most genes will be found with one QTL with large effect. The actual distribution of effect size is different, and the ratio of large-effect transcripts can be used to estimate the fraction of genes with single QTL. The same idea can be applied to estimate the fraction of genes with $n$ equal-effect QTLs, $n = 1, \cdots, 10$. 
		\item Epistasis test: tests for a difference between the mean expression levels of segregants and parents, because the means are equal for any additive inheritance pattern. 
	\end{itemize}
	
	\item Results: 
	\begin{itemize}
		\item QTLs: 3,546 transcripts have strong heritability $H^2 > 0.69$. Among these, 2,091 ($59\%$) showed linkage to at least one QTL (FDR at 0.05). Linkage results were robust to different normalization procedures and linkage tests.
		\item Effect size distribution: (effect sizes are estimated from independent test set, half of the data) for all transcripts with at least one QTL, choose the most significant QTL, the median QTL effect is 0.27, with 23\% of QTLs have effect $> 0.5$. 
		\item Genetic model estimation:  only 3\% of highly heritable transcripts are consistent with single-locus model; 17-18\% can be explained by models with one or two loci, and half of the transcripts requre models with $n > 5$. 
		\item Epistasis test: for 3,546 highly heritable transcripts, 583 (16\%) passed the epistasis test. 
	\end{itemize}
\end{itemize}

Mouse liver eQTL [Schadt \& Friend, Nature, 2003]: 
\begin{itemize}
	\item Methods: liver tissues from 111 $F_2$ mice from two standard inbred strains, 23,574 genes. Standard interval mapping. 
	
	\item Results: 	
	\begin{itemize}
		\item Detection rates: 7,861 genes were differentially expressed in two parental strains, among these, 2,213 gnes have at least one eQTL with LOD score $> 4.3$ ($P < 0.00005$). Without filtering based on DE, found 4,339 eQTLs over 3,701 genes with LOD greater than 4.3. 
		\item Effect size and number of eQTLs: on average eQTLs with LOD greater than 4.3 explained 25\% of the variance in $F_2$. 40\% of genes with at least one eQTL with LOD greater than 3.0 had more than one eQTL, and close to 4\% of these genes had more than 3 eQTLs. 
		\item Local and distal eQTLs: about 34\% of eQTLs are local (LOD $> 4.3$), however, 71\% of eQTLs are local if LOD threshold is 7.0. Thus eQTLs with high LOD tend to be cis-acting, while moderate eQTLs act in trans- in most cases. Also several eQTL hot-spots were found. 
	\end{itemize}
\end{itemize}

Mouse eQTL in hematopoietic stem cells (HSC) [Bystrykh \& Haan, NG, 2005]: 
\begin{itemize}
	\item Data: HSCs isolated from the bone marrow of D2 and B6 mice, 30 strains. 
	\item eQTL pattern: 478 transcripts were associated to a QTL within 20 Mb of the gene itself. Also identified multiple ``vertical bands'': the QTLs that modulate expression of a large number of transcripts. 
	\item Co-regulated genes: for four strongly cis-regulated transcripts (TF, receptors, etc), find the co-regulated genes through correlation. Many of these genes are downstream targets of the four chosen genes.  
\end{itemize}

Mouse eQTL in brain [Chesler \& Williams, NG, 2005]: 
\begin{itemize}
	\item Data: 80 recombinant inbred (RI) strains from BXD. 
	\item eQTL: at FDR 10\%, found 88 significant transcripts, most (83) were cis-linked; at FDR 25\%, found 101 significant transcripts. Seven trans-regulatory QTL bands were identified: one band regulates 1,650 transcripts. 
	\item Tissue specificity of expression regulation: the comparison with HSC eQTL, most global regulators (trans-) are tissue-specific. The cis-regulatory QTLs are often shared between brain and HSC. 
	\item Synaptic vesicle-related module: most of the trans-acting bands also regulate this module [Chesler05-Figure 6]. Some of the transcripts are also cis-regulated in these lcoi, making them candidate modifiers of the synapse-related module. 
\end{itemize}
%%%%%%%%%%%%%%%%%%%%%%%%%%%%%%%%%%%%%%%%%%%%%%%%%%%%%%%%%%%%
\section{Systems Genetics Methods}

Systems genetics paradigm: 
\begin{itemize}
	\item Challenge: the central goal is to characterize the candidate genes identified through linkage or association studies, but whose function/mechanism remain unknown. 
	
	\item \textbf{Reduction of a complex phenotype to molecular traits}: identify the intermediate molecular traits between genetic variations and phenotypes, then one can study these molecular traits: how are they influenced by genetic variations and how they influence the phenotypes. Each problem can then be studied separately, e.g. through an animal model. \\
	Example. atherosclerosis involves many aspects, in particluar, the response of endothelial cells to oxidized lipids. Study this system in vitro: how the gene expression of endotehlial cells change in response to oxidzied lipids. [Lusis at UCLA lecture]
	
	\item \textbf{Understand the links from DNA to molecular traits}: for cis-eQTL, this is about understanding the gene regulation. For trans-eQTL, this involves indirect effects (regulatory genes, but also compensatory effect, etc.). 
	
	\item \textbf{Understand the links from genes to phenotype through biological processes and gene networks}: generally a gene influences some biological processes, and cellular phenotypes, then organismic level phenotypes. The knowledge of processes a gene is involved and of gene networks can help understand the causal consequence of a gene (including its possible global effects). \\
	Example: PPAR-gamma: predict the effect of deletion using network, the change of expression of insulin-related genes (good) and lipid-metabolism related genes (bad). Similar analysis on GPR105 (p2ry14), good effects on both T2D and fat/heart disease. [Schadt, TED2011 talk]
\end{itemize}

\textbf{Systems genetics approach to complex diseases} [personal notes]
\begin{itemize}
\item Problem: given multiple types of data related to a disease, including GWAS, transcriptome (in patients and controls), eQTL (in patients or independent cohorts) and other network information, how do we better understand the genetics of disease? 

\item Summary: key considerations to study a candidate module are: (1) Connection with phenotypes: association with traits, eQTL as IV. (2) Upstream regulation: TFs, RBPs, mediator of trans-eQTL. (3) Downstream effects: functions of genes, disease relevance.
\begin{itemize}
	\item Example: KLF14 study [Small and McCarthy, NG, 2018]: (1) trans-gene network is a target of T2D associated locus. (2) KLF14 expression as mediator, also enrichment of KLF14 ChIP-seq and motif. (3) Some T2D related genes, e.g. GLUT4, IDN (insulin degradation). 
\end{itemize}

\item Cellular phenotypes: a key aspect of systems genetics is to use intermediate phenotypes, particularly cellular phenotypes. The key is to find some metrics of cells that reflect phenotypes, e.g. cell proliferation and cytokine production for immune diseases. In the absence of direct measurement, we can infer them from transcriptome signatures.
\begin{itemize}
	\item Remark: often factor analysis assumes independence of factors. In the case of cellular phenotypes, we may need to consider their interactions/dependence. Ex. in yeast, cell growth pathways and stress response are negatively correlated. 
\end{itemize} 

\item SNP-centered analysis: understand the mechanism of disease SNPs in terms of their effects on molecular and cellular phenotypes. Ex. KLF14 study, the SNP is a trans-eQTL of multiple genes, enriched with T2D risk genes. The limitation is that trans-effects of disease SNPs are difficult to study. Possible solution: combine many disease SNPs with PRS. 

\item Gene and pathway-centered analysis: identifying disease genes and pathways/modules. 
\begin{itemize}
	\item Differential expression and co-expression analysis: genes and modules discovered in this way may be candidates for causal genes/pathways; and may be also biomarkers of disease subtypes. For modules: use association of eigen-genes with traits. 
	
	\item Finding causal genes: MR analysis, colocalization. 
	
	\item Finding causal pathways/modules by enrichment: (1) Enrichment of disease associated variants in cis of the genes. Could use cis-eQTL of genes. (2) Enrichment of GWAS variants in eQTL (both cis and trans) of the genes. 
	
	\item Finding causal pathways/modules by IVs: use eQTLs of pathways as IVs to test causality. Use latent factors to represent pathway activity. Or mixture of MR: some genes are causal to phenotypes, some not.  
\end{itemize}

\item Identification of key drivers of phenotypes: manipulation of drivers may cause or restore diseases. 
\begin{itemize}
	\item Find regulatory genes of disease modules: TFs, RBPs.
	
	\item Find master genetic regulators: eQTLs of disease modules. Ex. SESN3 study: find epilepsy related genes first, then search for eQTL. Challenge is that this study may be under-powered. Hypothesis: genes within modules are candidates for master regulators (e.g. EndoG in T1D), so we can search for SNPs/eQTL of genes in modules.  
\end{itemize}

\item Disease subtypes: it is important to know that a disease may have multiple subtypes. Keep this in mind when developing specific analysis. 
\begin{itemize}
	\item Differential expression/co-expression analysis may be more powerful if we can segregate by subtypes (e.g. through PRS). 
	
	\item Possible strategy for disease subtype discovery: identify disease modules, and then define PRS based on SNP effects on the modules. Segregate patients by these PRSs. 
	
	\item How to confirm subtypes? Use effect size dependency on subtypes. 
\end{itemize} 

\item Context specificity of molecular and cellular phenotypes: contexts are tissue type or proper stimuli (e.g. LPS for immune cells) that are relevant to the phenotypes. This would be in part of experimental design: how to choose the relevant cell types and stimulate cells in relevant ways. Also the anaysis part when considering stimuli: either response QTL, or QTL whose effects depend on stimuli (similar to GxE interactions). It may be more powerful to combine the conditions as many eQTL would be shared, in a hierarchical way.

\end{itemize}

Reference: [Molecular networks as sensors and drivers of common human diseases, Shadt, Nature, 2009; Schadt \& Shaywitz, NRDD, 2009; Cookson \& Lathrop, NRG, 2009; Rockman, Nature, 2008]

%Strategies of integrating complex trait and eQTL studies: 
%\begin{itemize}
%	\item Idea: suppose the locus $L$ influences the trait $T$ through influencing the expression of some gene, $E$, i.e. $L \rightarrow E \rightarrow T$, then we would have three relations: $L \rightarrow E$, $L \rightarrow T$, and $E \sim T$ (correlation). Thus the strategy is to search for patterns that match these three (all or some of) relations. 
%	
%	\item eQTL of candidate genes: 
%	\begin{itemize}
%		\item Strategy: from the correlation between gene expression and trait, identify the candidate genes/modules; then the eQTL of these candidate genes may cause the trait variation: this can be verified by association of these eQTLs and traits [Schadt, Nature, 2003]
%		\item Limitations: the expression difference of the causal gene in case and control may be small (averaged over all genotypes). 
%	\end{itemize}
%	
%	\item Co-mapping of eQTL and complex trait QTL: 
%	\begin{itemize}
%		\item Strategy: if a GWAS locus is also an eQTL of some gene $X$, then $X$ is likely a candidate gene of the disease. Typically, search the eQTL database, from a few tissues (mostly lymphoblastoid cell lines, LCLs).
%		\item Conditional test of co-mapping: test whether a trait signal and a detected cis-eQTL association reflect the same underlying association. Suppose $L$ is the best eQTL of gene $g$ (usually cis-), and is also significant to the complex trait and let $L'$ be the best SNP of the complex trait. COnditional test would be test if $L$ is significant to the trait conditional on $L'$; the opposite conditional test may also be used. See [Nica \& Dermitzakis, PG, 2010; Voight, T2D meta-analysis, NG, 2010]. 
%	\end{itemize}
%	
%	\item Remark: 
%	\begin{itemize}
%		\item Application to expression analysis: the trait $T$ here could also be an expression trait (of a single gene, or a module), and the same strategy can be applied. This leads to the methods of finding the regulatory networks from eQTL data. 
%		\item Causal gene finding/fine mapping: in GWAS or QTL studies, the interval identified often contain multiple genes. eQTL can be used to narrow down the genes. In particular, we often limit $L \rightarrow E$ to cis-eQTL, then we have this method: the gene with cis-eQTL in the candidate region $L$ may be the causal locus of the trait $T$. However, the co-mapping test only tests whether the same SNP is the locus for expression and complex trait (coincidence), but this doesn't mean the gene expression trait is causal to the complex trait. 
%	\end{itemize}
%\end{itemize}

Case studies of GWAS-eQTL: 
\begin{itemize}
	\item Asthma study: a series of SNPs in strong LD spanning 200 kb containing 19 genes, strong effect in cis on the expression of one gene, ORMDL3.
	\item Crohn's disease (CD): the associated SNPs are located in a gene desert. Examination of LCL eQTL database showed that one SNP acts as a long-range cis-acting factor influencing expression of PTGER4. 
	\item Body mass index (BMI): a missense SNP in SH2B1 assocated with BMI, but also with expression of EIF3C and TUFM. Not clear the causal relationship. 
\end{itemize}

Difficulties/challenges of GWAS-eQTL studies: 
\begin{itemize}
	\item Expression measurement: many other sources of variations of expression, including batch effect, systematic bias in sample preparation, etc. And also results from different microarray platforms cannot be compared. 
	\item Detecting loci with small effects: particularly hard with small sample size. Generally trans-eQTLs are hard to found.  
	\item Missing heritability of expression: other variants include: (1) CNV: estimated that SNP and CNV captured 84\% and 16\% genetic variation in gene expression, but the signals have little overlap; (2) epigenetic factors: DNA methylation and histone modification. 
	\item Gene expression in tissues: the current eQTL data often from a few tissues. However, want expression in tissues involved in diseases. The tissues are generally hard to access, or if from post-mortem sample, often have changes accompy death or surgery. The ideas: ``exercise the genome'', or treating cells with different ways: pro-inflamatory stress, metabolic stress, response to signaling molecules, therapeutic agents, etc. 
\end{itemize}

Leveraging gene co-expression networks to pinpoint the regulation of complex traits and disease, with a focus on cardiovascular traits [BIFG, 2013]
\begin{itemize}
	\item Factor analysis: ICA, PCA and NMF. NMF is best for decomposition into cell types or states.
	
	\item Co-expression network reconstruction (Figure 2): the metric for defining the adjacency matrix, such as Spearman correlation and Mutual information (MI). Pruning edges: clustering (WGCNA), ARACne prune weakest for every three-edge triplets, Gaussian graphical model, GENIE3 that uses tree to regress expression with predictors (TFs).
	
	\item Analysis: why Gaussian graphical model is not good for network reconstruction? It tends to find edges with low degrees, and miss high-degree edges. Ex. suppose we have a cluster of genes, when y can be predicted from x, then any other $z \rightarrow y$ will be missed.
	
	\item Understanding the regulators and functional context of modules (Figure 2): TFBS enrichment, pathway enrichment.
	
	\item Connecting co-expression networks with phenotypes: define candidate modules by enrichment of disease genes (GWAS pathway analysis). Two additional analysis: (1) Master genetic regulators: genetic variants associated with expression of multiple genes in a module. Bayesian multivariate eQTL mapping (network QTL). (2) Differential co-expression network analysis: co-factorization analysis. Note that the modules correlated with phenotypes may be used to construct disease subtypes.
	
	\item Using co-expression networks to study cardiovascular traits: (1) EndoG (Figure 3): a WGCNA module in human heart, with ENDOG being a hub and many mito. genes. Confirm the function of ENDOG in heart function in mouse model. (2) Ebi2 (Figure 4): a module in rat (7 tissues), IRF7 regulated inflammation network (motif/ChIP-seq analysis). Found this module is enriched with T1D genes, and Ebi2 a genetic regulator of this module.
	
	\item Lesson: best network construction methods use prior knowledge in the form of TF-gene interactions.
	
	\item \textbf{Lesson}: the general strategy is to identify co-expression modules related to diseases, and identify TF or genetic regulators of the modules. Finding hub genes may help find key regulators of the modules.
	
	\item Remark: the factor analysis models are all based on linear effects. Can we expand to nonlinear cases such as boosting?
\end{itemize}

Systems Genetics as a Tool to Identify Master Genetic Regulators in Complex Disease [Chapter 16 of Systems Genetics book, 2016]
\begin{itemize}
	\item Strategies of mapping genetic regulators of modules: two step strategy (Figure 2). (1) Dimensionality reduction (e.g. eigengenes of modules), and do QTL mapping; (2) multivariate genetic mapping of genes in the modules.
	
	\item Example: Asxl2. From GWAS of bone marrow density, find Asxl2 as a candidate gene. Do WGCNA on rat expression data: find a module with Asxl2. Use the module to annotate the functions/mechanisms of Asxl2.
	
	\item Example: \textbf{KLF14}. Start with KLF14 as a risk gene of T2D, find it is a trans-eQTL of multiple genes (MuTHER data). Furthermore, for these genes, their cis-eQTLs are enriched with T2D variants.
	
	\item Example: Trem2. Using eQTL of 200 rat samples, find Trem2 a trans-eQTL of 190 genes.
	
	\item Example: \textbf{SESN3}. From gene expression data of 129 hippocampus brain samples, find a module enriched with epilepsy genes. Then do eQTL hotspot mapping (two-step strategy), identify SENSE3 as a trans-regulator.
	
	\item Lesson: co-expression networks and elucidate functions of disease genes.
	
	\item Lesson: if we start with known trait variants, finding that it is a trans-eQTL of other genes is not enough, as this can be due to pleiotropic effect of the trait variants. So need evidence that trans-associated genes are also disease related.
\end{itemize}

Multi-omics approaches to disease [Hasin and Lusis, GB, 2017]
\begin{itemize}
	\item Genotype-first approach: start with GWAS variants, identify causal genes and targets. FTO study [NEJM paper], allele-specific enhancer activity, then confirm by ASE and eQTL. Use trans-eQTL and gene correlation to find the functions of target genes of IRX3/5: adipocyte differentiation.
	
	\item Phenotype-first approach: start with gene networks of diseases, then find causal genes. AD study: gene sets changed during progression of AD, find immune genes. Then GWAS variants are enriched in enhancers of immune-related genes, but not neuronal function related enhancers and promoters.
	
	\item Problem: protein levels are not reflected by mRNA levels. Only a subset of genes show good correlation.
	
	\item Lesson: increase GWAS signals by stratification of variants related to certain biological functions/process.
\end{itemize}

Camelot: predict phenotypes from linkage and expression [Chen \& Pe'er, MSB, 2009]:
\begin{itemize}
	\item Motivation: linkage analysis can reveal candidate genes of a trait; in addition, expression profile is also predictive. Combine the two dataset to better identify the genes underlying a trait. 
	
	\item Methods: 
	\begin{itemize}
		\item Input data: 104 individual strains with 94 drug responses each [Perlstein, NG07]. For each strain wrt. one drug, the data: $D$ - drug response, $\mathbf{L}$ - the genotype, i.e. 526 markers. And $\mathbf{E}$ - the expression profile (under drug-free condition) of 6,189 genes [Brem, PNAS05]. 
		\item Pre-processing: expression features are limited to regulators (TFs, signaling molecules, chromatin factors, RNA factors), and multi-drug resistence genes (endosome transport, etc.). 
		\item Predictive model of phenotype: linear regression of $D$ on $\mathbf{L}$ and $\mathbf{E}$. Use elastic net, non-parameter bootstrap to select features for prediction. 
		\item Causality test: suppose one transcript is identified as predictive, it might be that its effect has already been incorporated by the genotype $\mathbf{L}$, thus design a triangle-test, does the effect of this gene go beyond what is explained by the genotype? The idea is to control $\mathbf{L}$. 
		\item Zoom-in score: for an important feature in $\mathbf{L}$, still need to choose genes (often tens of). The test is based on: the expression level is also correlated with the trait; plus two priors: 1) conservation (deviations from the consensus sequence); 2) cis-linkage.  
	\end{itemize}
	
	\item Results: 
	\begin{itemize}
		\item Evaluation by predictability: compare predictions of phenotypes with 1) no expression data used; 2) linkdage analysis (only QTLs, and predict trait by linear regression). 
		\item Verify the role of selected genes in the relevant phenotypes: e.g. DHH1 in \ce{H2O2} response. The model: region in Chr XIV $\rightarrow$ DHH1 expression $\rightarrow$ mitochondrial biogenesis $\rightarrow$ oxidative stress response. 
	\end{itemize}
\end{itemize}

Twelve type 2 diabetes susceptibility loci identified through large-scale association analysis [Voight, NG, 2010]
\begin{itemize}
	\item Method (Supplement, page 36). 
	
	\item Motivation: find the causal genes of GWAS-SNPs. The idea is to find the true cis-eQTL effect of the GWAS-SNP, but the problem is: disease SNPs can have significant cis-eQTL effect, even if they are not coincident. 
	
	\item Idea: find the candidate target gene of the GWAS-SNP of interest, and then assess if the GWAS-SNP explains expression variation of that gene (ie. see if GWAS-SNP explains the effect of the best cis-eQTL).  
	
	\item Step 1: for lead T2D SNPs, find its cis-effect on all nearby genes (2Mb) at $p < 0.001$. In some examples, multiple associations were found. E.g. HNF1 locus, CAMKK2 is the strongest.
	
	\item Step 2: for that gene, find the strongest cis-eQTL. Compare the cis-effect of the two SNPs: GWAS lead SNP and best cis-eSNP. In CAMKK2 example, best cis-eSNP has much larger effect than GWAS SNP, thus likely not coincident.  
	
	\item Step 3: Conditional analysis, test if GWAS-SNP is sufficient to explain the signal at cis-eQTL. Regression of expression on SNPs: use GWAS-SNP as explanatory variable, but conditioned on best cis-eSNP. If GWAS-SNP evaporates, it suggests that the signal is entirely driven by best cis-eSNP. Alternatively, we view best cis-eSNP as explanatory variable and conditioned on GWAS-SNP. If no effect, it suggests that GWAS-SNP does not make much contribution. 
\end{itemize} 

Bayesian Test for Colocalisation between Pairs of Genetic Association Studies Using Summary Statistics (COLOC) [Giambartolomei \& Plagnol, PLG, 2014]
\begin{itemize}
	\item Problem: given a locus, test if GWAS and eQTL signal colocalize. Compare 5 hypothesis (Figure 1): $H_0, H_1, H_2$ for no association, and association with only one trait; $H_3$ association with two traits on two independent SNPs; $H_4$ association with two traits on a single SNP. 
	
	\item Model: Define a configuration $D$ as the true association status of all SNPs in a locus wrt. the two traits ($2 \times n$ matrix, where $n$ is the number of SNPs). The BF of a hypothesis depends on the likelihood under any configuration that is consistent with the hypothesis: 
	\begin{equation}
	\frac{P(H_h|D)}{P(H_0|D)} = \sum_{S \in S_h} \frac{P(D|S)}{P(D|S_0)} \times \frac{P(S)}{P(S_0)}. 
	\end{equation}
	For $H_0$, only one $S$. For $H_1, H_2, H_4$, $n$ possible $S$ and for $H_3$, ${n \choose 2} - n$ possible $S$. We define the approximate BF of a SNP $j$ as $ABF_j$ (two values for two traits). Also for each SNP, define prior probability of associated with trait 1, trait 2 and both traits as $p_1, p_2$ and $p_{12}$. Then we have: 
	\begin{equation}
	\frac{P(H_1|D)}{P(H_0|D)} = p_1 \sum_j ABF_j^1
	\end{equation}
	Similarly we can obtain the BFs for other hypothesis (see Supplements). Note that ABF of SNPs for trait 2 will appear in the BF of $H_2$, and so on. Choose $p_1 = p_2 = 10^{-4}$ and The prior $p_{12}$ for $H_4$, $p_{12} = 10^{-6}$. The evidence of a hypothesis is summarized as posterior probability (PP), which sum to 1 over 5 hypothesis. 
	
	\item Examples (Figure 2): (A-B) PP3 (posterior prob of $H_3$) is large (C-D) PP4 is 82\%: the same top SNP in both traits. 
	
	\item Remark: the model uses the fact that when $S$ has a single causal variant, then $P(D|S)/P(D|S_0)$ would be just ABF of the causal variant. Also the model makes the assumption that data of two traits are independent.  
\end{itemize}

A gene-based association method for mapping traits using reference transcriptome data (PrediXcan) [Gamazon, NG, 2015]
\begin{itemize}
	\item Data: DGN (Depression Genes and Networks), 922 whole-blood samples eQTL. WTCCC GWAS data. 
	
	\item Prediction of gene expression: top SNP, polygenic score and elastic net (Lasso is similar). Only on cis-heritable genes. The mean $h^2$ is 0.153, with the predicted $R^2 = 0.114$ for top eQTL, 0.099 for polygenic score and 0.137 for elastic net. 
	\begin{itemize}
		\item Application to other dataset: on LCL, $R^2 = 0.0197$, 0.0367 for adipose, 0.0359 for lung and 0.0458 for whole blood. 
		\item Including trans-eQTL: results worse. 
	\end{itemize}
	
	\item Application to WTCCC: most findings are in autoimmune disease genes, and are either reported or located near the reported genes. The only exception: PTPRE (BD) and KCNN4 (HT). 
	
	\item Remark: prediction error is not modeled (attenuation bias).
	
	\item \textbf{Remark}: correlation, not causality. Consider an example of a non-causal gene with 2 eSNPs, one of them is associated with phenotype. Under an extremely simple model, we have: at genotype (0,0), expression and phenotype are 0 and 0; at genotype (0,1), expression and phenotype 1 and 0; at genotype (1,0), expression and phenotype 1 and 1; and at genotype (1,1), expression and phenotype 2 and 1. There is a good correlation of expression and phenotype, $r = 0.5$, and it could be highly significant with large samples. 
\end{itemize}

Integrative approaches for large-scale transcriptome-wide association studies (TWAS) [Gusev and Pasaniuc, NG, 2016]
\begin{itemize}
	\item eQTL data: 3000 individuals for blood adipose eQTL, cis heritability of gene expression from 0.01 to 0.07 and trans $h^2$ 0.04 to 0.06. Focus on 6000 cis-heritable genes. 
	
	\item Prediction of gene expression: compare top eQTL, BSLMM and BLUP. Found that BSLMM performs the best. 
	
	\item Summary statistics based TWAS: let $W$ be the effect size in eQTL and $Z$ be the standardized effect size in GWAS. The test statistic is $\sum_i W_i Z_i = W^T Z = Z^T W$. Its variance is given by (treating $Z$ as random and $W$ fixed):
	\begin{equation}
	\Var(W^T Z) = (Z^T W)^T (Z^T W) = W^T Z Z^T W = W^T \Sigma_{s,s} W
	\end{equation}
	where $\Sigma_{s,s}$ is the LD matrix of SNPs. We use here the fact that for standardized effect sizes, covariance between $Z_i$ and $Z_j$ is just the LD between two SNPs.  
	
	\item Comparison with COLOC: similar when there is a single causal variant, but better when the causal variant untyped or allelic heterogenity. Probably due to LD modeling. 
	
	\item Application to lipid GWAS: found 25 new associations, 19 of which replicated in later studies. In GWAS of obesity related trait, found 69 new genes. Note: in choosing what genes to test for TWAS, use $p$-value from heritability analysis, $p < 0.01$ by default. 
	
	\item \textbf{Remark}: causality analysis, if an eSNP has no signal in GWAS, we have $z_i = 0$, thus $w_i z_i = 0$, so it does not contribute to the correlation. But if a gene is true causal gene, this SNP should penalize. 
	
	\item \textbf{Remark}: summary statistics modeling not proper modeling of LD, because the statistics is based on observed effect sizes, not true effect sizes. Because of LD, they could differ significantly. 
\end{itemize}

Integration of summary data from GWAS and eQTL studies predicts complex trait gene targets (SMR) [Zhu and Yang, NG, 2016]
\begin{itemize}
	\item Problem: detect ``pleiotropic association'' between genes and phenotype, which is either due to pleiotropy or causality. ``The MR approach using a single genetic variant is unable to distinguish between pleiotropy and causality''.
	
	\item SMR method: let $\hat{b}_{zx}$ be the effect of SNP ($z$) on $x$ and $\hat{b}_{zy}$ be the effect on $y$, then the estimated causal effect is $\hat{b}_{xy} = \hat{b}_{zy}/\hat{b}_{zx}$. Its variance is given by Equation (2). The MR statistic is $\hat{b}_{xy}^2 / \Var(\hat{b}_{xy})$. With summary data, we can show that the SMR statistic is given by z-scores of $Z$ to $X$ and $Z$ to $Y$: 
	\begin{equation}
	T_{SMR} = \frac{z^2_{zy}z^2_{zx}}{z^2_{zy}+z^2_{zx}}
	\end{equation} 
	When applying the method, we could obtain effect sizes $\hat{b}_{zx}$ and $\hat{b}_{zy}$ using z-scores and AF of SNPs. 
	
	\item HEIDI: detect heterogeneity of effects. The intuition is if the signal is due to linkage, then there are two causal variants (one for each trait), then the SMR estimated effects would be different between two SNPs. Suppose we choose a top SNP, and we assess if the SMR estimator at another SNP is different, if so, there is evidence of linkage. $H_0$: same effects in all SNPs (co-localization). The test statistic is $d_i = \hat{b}_{xy(i)} - \hat{b}_{xy(top)}$, and we can account for LD in the distribution of $d_i$.  
	
	\item Results of GWAS-eQTL analysis: Westra eQTL data vs. 5 phenotypes. 68 genes for height, 9 for BMI, 2 for WHRadjBMI, 9 for rheumatoid arthritis and 16 for SCZ. 
	
	\item Casuality analysis of genes: most genes have no trans-eQTL. For the 4 genes with trans-eQTL, none of them have consistent effects in GWAS. 
	
	\item Discussion: we use HEIDI to filter out heterogeneous effects, so generally we favor HEIDI results with \textit{large} $p$-values. But when SNPs are in high LD (two causal variants), the $p$-values will be large (because we do not have power to detect heterogeneous effects), so HEIDI may miss many cases due to linkage, but treat them as causal or pleiotropy. 
	
	\item Note: see Supplementary Note 2 of [Integrative analysis of omics summary data reveals putative mechanisms underlying complex traits, NC, 2018] for discussion of HEIDI. (1) Analogy with HWE in QC: in HWE test, we want to filter out SNPs showing deviation from HWE ($H_1$). (2) Threshold: Using $p < 0.05$ without multiple testing correction may be too conservative: especially when this is used in multiple steps, e.g. lose 15\% of true pleiotropic effects (3 steps). So the paper filters out all with $p < 0.01$. (3) Inclusion of weak cis-eQTL: inflation of $p$-values, and reduce the power of detecting true pleiotropic associations. Suggest to remove SNPs not or in weak LD with top cis-eQTL. 
	
	%Meanwhile, when there are two causal variants in high LD, HEIDI cannot distinguish causality/pleiotropy (no heterogeous effect) from linkage.  
\end{itemize}

Integrating molecular QTL data into genome-wide genetic association analysis: Probabilistic assessment of enrichment and colocalization (Enloc) [Wen, PLG, 2017]
\begin{itemize}	
	\item Motivation: to assess co-localization, we need to know the prior probability of an eQTL being a GWAS variant. This step would benefit from Empirical Bayes estimate, which is not done in previous methods, eCAVIAR and coloc. 
	
	\item Method outline: let $d$ be whether a SNP is eQTL, treat $d$ as annotations of SNPs in GWAS data for fine-mapping. The issue is that $d$ is not observed, so we need to marginalize the missing $d$. 
	
	\item Estimation of $\alpha$ (enrichment of eQTL in GWAS loci, prior in DAP): to deal with missing $d$, use DAP to sample the posterior of $d$ 20-30 times: $p(d|Y_{qtl}, G_{qtl})$. Let $\hat{\alpha_1}^{i}$ be the estimate of the $i$-th imputed data, then the final estimate of $\alpha_1$ is the mean of all estimates, and the variance can also be determined (Text S1). Note: this procedure underestimates $\alpha_1$, as $d$ posterior is from only eQTL data, while it should be conditioned on both eQTL and GWAS data. 
	
	\item Fine-mapping of GWAS loci: let $\gamma$ be SNP configuration, we need $P(\gamma_i = 1 |D)$. DAP-1 is found sufficient in most cases. DAP-1 plus conditional analysis can find additional loci. 
	
	\item Assessing co-localization:  SNP colocalization probability (SCP) $P(\gamma_i = 1, d_i = 1| D, D_{qtl})$ given by Equation (8), where $D$ is GWAS data. Note in LHS of the equation, should be $d_i =1$. To see this, we note:
	\begin{equation}
	P(\gamma_i, d_i | D, D_{qtl}) = \frac{P(\gamma_i, d_i, D, D_{qtl})}{P(D, D_{qtl})} = \frac{P(d_i)P(D_{qtl} |d_i) P(\gamma_i|d_i) P(D|\gamma_i)}{P(D, D_{qtl})}
	\end{equation}
	Note that $P(d_i) P(D_{qtl}|d_i) \propto P(d_i|D_{qtl})$, the PIP of eQTN. Full derivation in Text S1: in Eq. (7), the ratio is the product of posterior ratio of eQTN for $P(d_i=1|D_{qtl})$; and the prior ratio of $P(\gamma_i=1|d_i=1)$. Note: $P(D|\gamma_i)$ term cancels out. Regional colocalization probability (RCP): is the sum of SCP for all SNPs. 
	
	\item Analysis: with large $\hat{\alpha}_1$ (enrichment parameter for eQTNs), it is possible that a SNP with low eQTN PIP has a high SCP. Intuitively, SCP depends on the product of eQTL PIP and $P(\gamma_i | d_i)$, so a large $\hat{\alpha}_1$ can overcome small eQTL PIP. In other words, we can think of PIP from eQTL as the prior of a SNP being eQTN, and GWAS result as the data of this SNP, then the posterior of this SNP can be large if the SNP has high PIP in GWAS.  
	
	\item Impact of the enrichment parameter $\alpha_1$ (Figure 1): e.g. two SNPs in perfect LD, RCP is 0.5 with $\alpha_1 = 0$. But it increases significantly with large $\alpha_1$. When it is 4, RCP is very close to 1. 
	
	\item Connection with eCAVIAR and coloc: (1) eCAVIAR: simply fine-mapping on both traits, so it's equivalent to $\alpha_1 = 0$. (2) coloc: correspond to relatively large values of $\alpha_1$, Equation (10), for default values of coloc, this is $\alpha_1 = 4.6$, or 100 fold enrichment. Intuitively, two independent association by chance are very unlikely to hit the same SNP $p_1 \times p_2 =10^{-8}$; however, colocalization has a relatively large prior chance $p_{12} = 10^{-6}$. 
	
	\item Simulation: use 1000GP real genotype data. Choose $\alpha_0$ and $\phi$ (standard deviation of causal SNP effect size) s.t. the GWAS Z-score follows real distribution. Using height GWAS, $\alpha_0 = -8.4$ (or $2 \times 10^{-4}$) and $\phi = 0.4$. 
	
	\item Accuracy of estimation of $\alpha_1$ (Table 1): when it is $<2$, often have large confidence interval, so statistically not significant. Comparison with alternative approaches (Figure S2): (1) Best SNP per gene is a reasonably good approximation: it slightly underestimates $\alpha_1$ when true value is large (4 or 5)  (2) Mean imputation (using PIP of eQTL): at $\alpha_1 \leq 2$, it has large standard error; at $\alpha_1 \geq 3$, mean imputation has similar standard error as multiple imputation, and slightly overestiamtes $\alpha_1$ (while best SNP and multiple imputation underestimate). This can be due to the scaling effect (Text S1). Also shrinkage has small effect when $\alpha_1 > 3$. 
	
	\item Power of colocalization (Figure 3): when $\alpha_1 = 0$, power is low, $<0.1$. With larger $\alpha_1$, e.g. 4, could reach 40\%. 
	
	\item Blood eQTL vs. lipid GWAS: $\alpha_1$ varies from 0.4 to 5 for several lipid traits (Figure 5). Estimated number of eQTL 8.9K. The number of GWAS loci is around 50. The number of co-localized loci: around 20. 
	
	\item Analysis: Enloc $\alpha_1$ estimation may be too conservative or lead to large confidence interval. Enloc performs eQTL fine-mapping only on selected eGenes (user-specified). Typically, the list is conservative, say, FDR $< 0.1$. When running Torus to estimate $\alpha_1$, still use all genes, but for genes not selected as eGenes, the PIPs of their SNPs would be 0. 
\end{itemize}

Exploring the phenotypic consequences of tissue specific gene expression variation inferred from GWAS summary statistics (S-PrediXcan) [Barbeira and Im, NC, 2018]
\begin{itemize}
	\item Test statistic: Let $T_g$ be expression of gene $g$, we know $T_g = X^T W$, where $X$ is genotype. Our problem is simple regression of $Y \sim T_g$. The estimated effect is: 
	\begin{equation}
	\hat{\gamma}_g = \frac{\Cov(T_g, Y)}{\Var(T_g)}
	\end{equation}
	Use $T_g = X^T W$, it is easy to see that
	\begin{equation}
	\Var(T_g) = \hat{\sigma}_g^2 = W^T \Gamma W
	\end{equation}
	where $\Gamma$ is the sample covariance matrix of $X$. Next we consider the covariance term:
	\begin{equation}
	\Cov(T_g, Y) = \Cov(\sum_l w_{lg} X_l, Y) = \sum_l w_{lg} \Cov(X_l, Y) = \sum_l w_{lg} \hat{\beta}_l \hat{\sigma}_l^2
	\end{equation}
	where we use the relation of $\Cov(X_l, Y)$ and the estimated effect size of $X_l$.
	
	\item Computing Z-score: next we need to compute the s.e. of the estimate $\hat{\gamma}_g$. From the simple regression model, is just $\sigma_Y^2 / (n \hat{\sigma}_g^2)$. We relate $\hat{\sigma}_l^2$ with $\hat{\sigma}_Y^2$. Putting all these together, we have the Z-score:
	\begin{equation}
	Z_g = \sum_l w_{lg} \frac{\hat{\sigma}_l}{\hat{\sigma}_g} \frac{\hat{\beta}_l}{se(\hat{\beta}_l)}  
	\end{equation}  
\end{itemize}

Probabilistic fine-mapping of transcriptome-wide association studies (FOCUS) [Mancuso and Pasaniuc, review for NG, 2018]
\begin{itemize}
	\item Background: credible set. Given a set of configurations $C$, define its posterior as: $P(C|D) = \sum_{\gamma \in C} P(\gamma|D)$ Choose a minimum $C$ to reach $P(C|D) \geq \rho$ for some threshold $\rho$. 
	
	\item Problem: in TWAS, z-scores of genes can be correlated. Ex. if each gene has a causal eQTL, but the eQTL of multiple genes are in LD, then an eQTL of one gene may be associated with another gene. More formally, let $\hat{G}_j = X \Omega_j$ be the predicted expression of gene $j$, and $\Omega_j$ be the effect size of SNPs on gene $j$, then we can analyze dependency of $\hat{G}_j$ via LD and $\Omega_j$'s. 
	
	\item Method overview: derive the distribution of TWAS Z-scores of all genes in LD regions, accounting for dependency of Z-scores due to LD.  
	
	\item Model: let $X$ be genotype vector, and $G$ be expression of $m$ genes, we write the true model of phenotype as: 
	\begin{equation}
	y = X \beta + G \alpha + \epsilon
	\end{equation}
	where $\alpha$ is the vector of gene effects. For gene expression, our model is: $G = X W + E$, where $W$ is eQTL effects. The imputed expression in TWAS: $\hat{G} = X \Omega$, where $\Omega$ is the estimated effects (accounting for LD). The TWAS Z-scores for gene $j$ is normalized $\hat{G}_j^T y$. Let $V$ be the LD matrix. This allows us to show that: 
	\begin{equation}
	Z_{\text{twas}} | \lambda_{\text{snp}}, \lambda_{pe}, \Omega, V \sim N(\Omega^T V \lambda_{\text{snp}} + \Gamma \lambda_{pe}, \Gamma)
	\end{equation} 
	where $\Gamma = \Omega^T V \Omega$ is the covariance of predicted expression of genes, $\lambda_{pe} \propto \alpha$ is the vector of true gene effects and $\lambda_{\text{SNP}}$ is the pleiotropic effect of SNPs (vector). Assume a simple model for $\lambda_{\text{SNP}}$: vector of common effect size (a parameter to be estimated). To fine-map: spike-and-slab prior of $\lambda_{pe}$. 
	
	\item Analysis: consider a region with two genes, several scenarios. (1) A single causal SNP has effects on both genes: only one is causal. Then this SNP has large GWAS effects. It is hard to distinguish the two genes. (2) Two causal SNPs, one for each gene: let SNP 1 be causal for gene 1, and SNP 2 for gene 2, and gene 1 is causal for trait. Then SNP 1 should have large GWAS effect, and SNP 2 smaller due to LD with SNP 1 (not because of effect on gene 2). Possible to fine map using either SNP level info. or gene-level information. 
	
	\item Simulation: 25 blocks, in each block, average of one causal gene. Each gene has 1 or 2 causal SNPs in 100kb regions for both eQTL and GWAS. $\Omega$ estimated using GBLUP. 
	
	\item Results: in simulation, type I error is controlled (Figure 2). When the causal gene is missing, most often choose null model in the credible set (Figure 3). For power, comparison with TWAS z-scores, higher AUC (Figure 4).  
	
	\item Application to lipid GWAS: reduce the size of TWAS genes: from 60 to 30-40. 
	
	\item Remark: the method may lose power. Intuitively, we should fine-map at the SNP level, as co-localization of eQTL and GWAS provides a good signal of causal gene. Also, the if we know causal variants of gene expression, $\Omega$, the problem is considerably easier, however, no effort is done to fine-map eQTL, instead, $\Omega$ is obtained from a polygenic model (GBLUP).  
\end{itemize}

Leveraging molecular quantitative trait loci to understand the genetic architecture of diseases and complex traits [Hormozdiari and Price, NG, 2018]
\begin{itemize}
	\item Creating PIP annotations: use CAVAIR to fine-map SNPs. Several options: MaxCPP - for each SNP, the top PIP among all cis-genes (SNPs not in any credible set assigned 0). All-cis-eQTL: all cis-eQTL passing threshold. Top-cis-eQTL: top eQTL per gene. Credible set: binary annotation of whether a SNP is in the credible set. 
	
	\item Estimation of enrichment and effects of annotations in S-LDSC: the model is for SNP $j$, we have the prior $\Var(\beta_j) = \sum_c \tau_c a_{cj}$, where $\tau_c$ is the effect of annotation $c$. To estimate enrichment: assuming $c$ is binary, then the heritability explained by $c$ is simply the sum of heritability of all SNPs belonging to $c$: $h_g^2(c) = \sum_j a_{jc} \Var(\beta_j)$. This relationship is generalized to continuous case. 
	
	\item Difference of effects and enrichment of an annotation: if two annotations are correlated, then one will learn independent effects (one annotation has effect 0), but both will have significant enrichment. Intuitively, the SNPs of the two annotations will overlap, thus enriched with heritability. 
	
	\item Simulation: Top-cis-eQTL show 3-4 times over-estimation of effects. Other annotations fine.  
	
	\item Fine-mapped SNPs are enriched with GWAS heritability (Figure 2): averaging over multiple GWAS traits, use blood eQTL or FE-meta analysis eQTL (similar results). Enrichment: all eQTL (6\%) - 2 fold; credible set (2\%) - 2.5 fold; MaxCPP (0.1\%, defined as the average value of annotations) - 5 fold. Effect: larger difference among the three groups.
	
	\item Lesson: to better use eQTL in studying traits, use MaxCPP or Credible set as annotations. 
\end{itemize}

Principled multi-omic analysis reveals gene regulatory mechanisms of phenotype variation (pGENMi) [Hanson and Sinha, GR, 2018]
\begin{itemize}
	\item Goal: identify TF-phenotype associations. Intuition: if TF is important, then its targets are likely enriched with GWAS signal. 
	
	\item Model: given a TF, consider all its targets. Suppose we have p-values of target association with phenotypes (e.g. from TWAS). Let $Z_g$ be an indicator of whether gene $g$ is associated with phenotype. When $Z_g = 0$, uniform distribution, and when $Z_g = 1$, Beta distribution. The prior of $Z_g$ depends on regulatory evidence of $g$: $r_{gm}$ for the $m$-th type of evidence of TF regulating $g$. Use eQTL or eQTM: whether the eQTL or eQTM of $g$ is a target of TF in ChIP-seq. 
	
	\item Analysis: difference from Torus/S-LDSC: test enrichment of h2 in TF targets (1) use gene level evidence for phenotype association. (2) Use cis-eQTL/eQTM to link TF to target genes.  
\end{itemize}

Quantifying genetic effects on disease mediated by assayed gene expression levels [Yao and Gusev, NG, 2020]
\begin{itemize}
	\item Model: let $w_j$ be the GWAS effect of SNP $j$, it may act on trait via gene expression with $\beta_{jk}$ the effect on gene $k$, or pleiotropic effect $\gamma_j$. This leads to the effect size equation:
	\begin{equation}
	w_j = \sum_k \beta_{jk} \alpha_k + \gamma_j
	\end{equation}
	With this, derive LDSC type of equation/MOM estimator. Let $\chi_k^2$ be the chi-square statistics of SNP $k$, and $d$ be index of gene category: 
	\begin{equation}
	\E(\chi_k^2) = N \sum_c \tau_c l_{k,c} + N \sum_d \pi_d L_{k,d} + 1
	\end{equation}
	where $c$ is the index of SNP category, and $\pi_d$ is per gene contribution, and $L_{k,d}$ is the expression score of SNP $k$. It is defined as:
	\begin{equation}
	L_{k,d} = \sum_{i \in D} \sum_j r_{jk}^2 \beta_{ij}^2
	\end{equation}
	So the expression score of a SNP wrt. genes in set $D$ measures its total LD to eQTLs of all genes in $D$, weighted by eQTL effect sizes. 
	
	\item Simulations under model assumptions: 10,000 GWAS samples, 100-1,000 eQTL samples. Only Lasso and REML correction gives unbiased estimates. BLUP and OLS: terrible results. Varying sparsity: e.g. 10\% of genes or 10\% of SNPs, overall robust, but at the setting where mediated $h^2$ is high, significant under-estimate with 10\% genes. 
	
	\item Simulations under violation of effect size assumption: effect size of genes depend on cis-heritability. Binning by cis-heritability is important. 
	
	\item Results: use GTEx and 42 traits, median explained h2 is 0.11. Also inverse relationship between cis-heritability of expression and disease heritability mediated by expression.
	
	\item Remark: it may be important to consider prediction errors, as the results are sensitive to prediction methods used.  
\end{itemize}

Bayesian Genome-wide TWAS Method to Leverage both cis- and trans-eQTL Information through Summary Statistics [Luningham and Jingjing Yang, AJHG, 2020]
\begin{itemize}
	\item Prediction model of expression: BVSR model, with different priors on cis-SNPs and trans-SNPs. Fit with EM-MCMC: parallel blocks, using summary statistics to speed up computation, pruning of blocks - only those with certain p value cutoff. Running time: a single gene, about 30 mins. 
	
	\item BGW-TWAS: same as S-PrediXcan, the Z-score of association is sum of $w_{lg}$, the weight of SNP $l$ on gene $g$ (posterior mean from BVSR), times GWAS Z-score, times $\hat{\sigma}_l / \hat{\sigma}_g$, where $\hat{\sigma}_l$ is the s.d of SNP $l$, and $\hat{\sigma}_g$ is the s.d. of imputed gene expression. 
	
	\item Genetic architecture of gene expression: Table 4, in brain eQTL data of 500 samples, generally the total PIP sums to 1-3, most in trans. 
	
	\item Results of BGW-TWAS in AD: use AD with individual level data. One gene with 5 supporting SNPs, all in trans-. However, 4 SNPs are in LD (but selected by BVSR). 
	
	\item Remark: no cross-validation of gene expression heritability.  Sign of potential problem: genes with smallest $R^2$ from BVSR have highest PIP sum (Table 4). 
\end{itemize}

%%%%%%%%%%%%%%%%%%%%%%%%%%%%%%%%%%%%%%%%%%%%%%%%%%%%%%%%%%%%
\section{System Genetics Studies}

Mouse obesity study with liver eQTL [Schadt \& Friend, Nature, 2003]
\begin{itemize}
\item Methods: 112 $F_2$ mice using liver sample, see notes before for details. 

\item Results: 
\begin{itemize}
	\item Candidate genes of obesity through expression data: 280 genes significantly DE in two groups of mice: high FPM and low FPM (related to obesity). The expression pattern of these gene can be used to cluster mice and the results correspond to the two groups. 
	\item Heterogeneity of the FPM trait: gene expression patterns ientify two subgroups of high FPM mice. Also verified by eQTL: only by using one subgroup of FPM mice (plus the low FPM group), some eQTLs could be identified (the chromosome 2 region). 
	\item eQTL of the candidate genes: five regions containing eQTL os more than 50\% of the genes in the FPM set. Analysis of the chromosome 2 region identifies two candidate genes: they are located within this region, and they have eQTLs in this region. One gene is a protein glycoylatransferase, and the other is a cation-transporting ATPase. 
\end{itemize}

\item Remark: in this case, if just use QTL of FPM trait, only four QTLs were found with LOD score greater than 2.0. 
\end{itemize}

Candidate genes in hematopoietic stem cells (HSC) turnover [Bystrykh \& Haan, NG, 2005]: 
\begin{itemize}
	\item Problem: HSC turnover is determined by the fraction of cells in the S phase in HSC. What genes may influence the HSC turnover? The strategy is to use genetic variations (of strains) to identify the causal genes. 
	\item Data: 30 BXD mouse strains (see before). 
	\item Co-mapping of eQTLs to trait QTL: a QTL of HSC turnover has been identified before, called stem-cell proliferation (Scp2). 8 cis-acting eQTL were mapped to this region. 
\end{itemize}

Candidate genes of weight in mouse through liver eQTL [Ghazalpour \& Horvath, PG, 2006]: 
\begin{itemize}
	\item Data: Liver expression data of 135 famale mice from a F2 inter-cross between two inbred strains. 
	\item Gene module identification: weighted co-expression network approach: the link between two genes is the power of the correlation coefficient (the power function makes the results less sensitive wherease unweighted networks display sensitivity to the choice of cutoff). 12 distince modules were found. 
	\item Association of modules and clinical traits (weight): defined through average correlation (absolute value) of the genes in the module and the trait. The most significantly associated module is enriched for genes in the EMC-receptor interaction and complement and coagulation cascades. 
	\item moduleQTL (mQTL) analysis: the locus with a significant enrichment for eQTL of the genes within the module, assessed by Fisher's exact test. 
\end{itemize}

Molecular networks underlying metabolic syndrome [Chen and Schadt, Nature, 2008]:
\begin{itemize}
\item Motivation: diseases are caused by change of states of molecular networks (a functional module), thus identyfy these disease-causing modules. 

\item Methods: 
\begin{itemize}
	\item 334 $F_2$ mouse from crossing two strains. Each mouse is: genotyped with 1,300 SNPs, expression-profiled (liver and adipose), and phenotyped (metabolic syndrome). 
	\item eQTL and cQTL (metabolic traits) identification. 
	\item Assess whether a expression trait is supported as having a causal relationship with metabolic traits [Schadt \& Lusis, NG, 2005]. 
	\item In co-expression networks, first identify modules and then test the modules where the causal genes are enriched, by Fisher Exact Test. 
\end{itemize}
\end{itemize}

Genetics of obesity through eQTL in adipose tissue [Emilsson \& Stefansson, Nature, 2008]: 
\begin{itemize}
\item Methods: 
\begin{itemize}
	\item Data: 2 cohorts (1,002 and 673 respectively) with expression profiling (23,720 transcripts) in blood and adipose tissues, and clinical traits including body mass index (BMI), percentage body fat (PBF), etc. 
	\item Identification of obesity-related gene modules: (1) select all genes that show differential expression in adipose tissue, then construct pairwise coexpression network (weighted) in human and search for co-regulated module; (2) the coexpression network in mouse and conserved modules; (3) test if the conserved module(s) is enriched with obesity-associated genes. 
\end{itemize}

\item Results: 
\begin{itemize}
	\item Gene-clinical trait association: in blood sample, less than 10\% genes show association with clinical traits; while in adipose tissue, 60-70\% genes show association. To narrow down genes, find conserved co-regulated modules between human and mouse. This reveals a single core module enriched with macrophage function and metabolic traits (MEMN). And 886 genes in MEME (98\%) are significantly correlated with BMI. 
	\item Expression QTLs: two methods are used (1) eQTLs using family information, with 1,732 microsatellites; (2) eSNPs using 317,503 SNPs on 150 unrelated individuals. The results between the two are highly consistent. At FDR 0.05, detected cis eSNPs for 2,417 (15\%) expression traits in blood and 3,048 (14.6\%) traits in adipose, and the two sets of cis eSNP highly overlap (50\%). 
	\item Trans-eQTLs and trans-eSNPs: at FDR 0.05, found 52 and 25 trans eQTL signals of genome-wide significance in blood and adipose tissues; and 67 trans eSNPs signals in blood and 59 in adipose tissue. Few eQTL hotspots were detected, beyond what was expected by chance (through simulation: note that due to correlation of transcripts, some false eQTL hotspots would be expected). One possible hotspot at 3q22. 
	\item eQTL of the MEMN genes: most MEMN genes have cis eSNPs, and confirmed that some cis eSNPs show association to BMI (genotyping of 768 cis eSNPs). 
\end{itemize}
\end{itemize}

Application of liver eQTL to GWAS of T1D and CAD [Schadt \& Ulrich, PLoS Biol, 2008]: 
\begin{itemize}
\item Methods: see the section ``Genetics of Gene Regulation'' for data and eQTL finding. To identify disease genes using eQTL, use co-mapping, i.e. find genes whose expression are mapped to the same SNP as the GWAS of the complex trait. The functional evidence of a candidate gene: construct the causal network of eQTL, and check if the linked genes in the network are enriched with known genes of these traits. 

\item Application to T1D: from T1D-associated SNPs, find the genes in the vincinity of these SNPs whose expressions are correlated with them (Table 1). The results may include eQTLs not passing the threshold: in fact, only 3 genes: RPS26, CLECL1 and HLA-DRB1 have eSNPs with $P < 10^{-5}$. Out of 13 genes found in this way, 9 are known T1D risk genes. 
\begin{itemize}
\item Rps26: the candidate gene for the cis-SNP (rs2292239) was thought to be Erbb3 in WTCCC studies. But the SNP is not associated with expression of Esrrb3, but Rps26 in liver eQTL data. The evidence that Rps26 is the true candidate gene: the causal network from the eQTL using mouse expression data Rps26, but not Erbb3 is related to a number of genes known to be important in T1D. 
\item CLECL1: the candidate gene of rs3764021 was thought to be CLEC2D in WTCCC studies. With eQTL data, it was found this SNP is associated with CLECL1, but not CLEC2D. 
\end{itemize}

\item Application to CAD: WTCCC identified 7 significant SNPs, check if any of them is a eQTL using threshold $P < 0.05 / (7 \cdot 40000) = 1.79 \cdot 10^{-7}$ (test 7 SNPs for 40,000 expression traits). Only one of them  (rs599839) is associated with any expression trait: PSRC1, CELSR2, and SORT1. This SNP is also associated with LDL in a GWAS of LDL. Relevant evidence of these genes: 
\begin{itemize}
	\item Psrc1, Sort1 were significantly associated with plasma LDL level in mouse, and  
	\item Psrc1 and Sort1 were part of the MEMN network in the causal network constructed from mouse eQTL data. 
\end{itemize}
\end{itemize}

Candidate causal regulatory effects by integration of expression QTLs with complex trait genetic associations [Nica \& Dermitzakis, PG, 2010]:
\begin{itemize}
\item Motivation: for GWAS-SNPs, find the expression trait that it influences (thus suggesting the mechanism of its action). The simplest method would be to test if a GWAS-SNP is co-localized with some eQTL, but it is possible that GWAS-SNP and an eQTL may tag different causal SNPs becuase of LD. 

\item Methods: 
\begin{itemize}
\item Data: 976 GWAS SNPs from NHGRI, into 784 intervals (defined by recombination hotspots). Genotype and expression data of lymphoblastoid cell lines (LDL) of 109 unrelated HapMap individuals. 
\item RTC: measure the association of GWAS-SNP with an expression trait. Suppose there is an eQTL in the interval of GWAS-SNP, the idea is: if GWAS-SNP has a large effect on expression, then the eQTL should have weak association with the expression trait after removing the effect of GWAS-SNP (i.e. association of eQTL and the residual of GWAS-SNP and expression trait). Since an interval may have multiple SNPs, using the rank of GWAS-SNP relative to all other SNPs in the interval. 
\end{itemize}

\item Results: 
\begin{itemize}
	\item GWAS SNPs are enriched for regulatory variants: for these SNPs, find its cis- and trans- regulatory effects. The distribution of regulatory effects in GWAS SNPs is higher than that of the random SNPs. 
	\item Comparison of RTC and correlation-based measures: RTC performs better than $r^2$ or $D'$ between GWAS-SNP and eQTL, in both simulation data and in real data. (1) RTC outperforms $r^2$ because it is able to recover causal effect even for low correlated pairs. (2) A high $D'$ is insufficient to predict causal effect because it is impossible to distinguish causal from coincidental effects given a strong historical correlation. 
	\item GWAS-SNPs with significant cis-regulatory effects: out of 976 GWAS SNPs, 130 have at least one cis-eQTL (with $P < 0.05$ after permutation) in the GWAS interval. Among these, 28 have RTC $> 0.9$. Many related genes are immunity related (consistent with the use of LDL expression data). 
	\item GWAS-SNPs with significant trans-regulatory effects (trans- is defined as 5 Mb from TSS or TES): top 50 GWAS intervals ordered by trans-eQTL significant. Among these, 24 have RTC $> 0.9$. 
\end{itemize}

\item Remark: 
\begin{itemize}
	\item The problem of using $r^2$ and $D'$ is lack of normalization wrt. the local LD patterns. Ex. in a region with low LD, even small value of $r^2$ may be very significant; alternatively, in a region with high historical correlation, even high $D'$ may not suggest causal effect. 
	\item The RTC method is essentially a way of assessing the significance of the feature (GWAS-SNP) in the presence of a correlated feature (eQTL). Thus using residual from using the eQTL feature, and regression (How is this method compared with e.g. feature selection by testing if $\beta = 0$ for the feature of interest?). Meanwhile, RTC performs normalization by ranking the GWAS-SNP in all SNPs of the same interval (thus normalize wrt. the SNP density, the LD of the region, etc.).  
\end{itemize}

\end{itemize}
 
Association of pathway expression and complex traits [Zhong \& Schadt, AJHG, 2010]: 
\begin{itemize}
\item Strategy: the hypothesis is: the perturbation of expression of a pathway leads to a complex disease. Thus, to test association of a pathway (its expression) and a trait, we first find the eQTLs of the genes belong to this pathway, then test the association of these eQTLs to the trait. 

\item Methods: 
\begin{itemize}
	\item Data: gene expression profiling of 707 liver samples, 916 omental adipose samples and 870 subcutaneous adipose samples. 
	\item Pathway expression-association test: for each gene, among all eSNPs (at FDR 10\%), choose the one with the strongest association with T2D (usually single independent eSNP). Then test the pathway association with T2D, similar to GSEA (the deviation of $P$ value distribution from the null distribution). Only 110 KEGG pathways are tested, with size of 20 to 200 genes. 
	\item Determine the FDR: any pathway has an enrichment score $NES$ (normalized s.t. it is comparable across pathways), meanwhile, obtain the $NES$ null distribution by permutation: switch case and control label and computer the $NES$ scores for every gene set. 
	\item Controls in pathway association test: (1) positive controls - one set of 18 genes replicated to be associated to T2D in WTCCC and another set consisting of these 18 and other random 22 genes; (2) negative controls - gene sets of size 20, 40, ..., 200, randomly selected from the set of genes associated with at least one eSNP. 
	\item Weight subtraction algorithm: to perform the analysis on a different dataset (WTCCC vs. non-WTCCC), need to get the association statistic of non-WTCCC, however, only $P$-values from DIAGRAM (meta-analysis of WTCCC and non-WTCCC) are available. Need to calculate $P$ value from non-WTCCC. This is done by computing based on $Z$-scores (DIAGAM used fixed-effect weighted average of the summary $Z$ scores): 
\begin{equation}
Z_{\text{DIAGRAM}} = w_{\text{WTCCC}} \cdot Z_{\text{WTCCC}} + w_{\text{non-WTCCC}} \cdot Z_{\text{non-WTCCC}}
\end{equation}
where the weights are proportional to the square root of the effective sample sizes. 
\end{itemize}

\item Results: 
\begin{itemize}
	\item eQTL results: identified a total of 20,563 eSNPs of 9,964 genes. On average, 30\% eSNPs are tissue-specific, the rest common to all three tissues. 
	\item Significant pathways with T2D using WTCCC: 23 out of 110 KEGG pathways have nominal $P$ value $< 0.05$ (16 with FDR 20\%). Correlation among KEGG pathways (shared genes) may explain the excess of significant pathways. Two positive controls are highly significant (due to 6 genes with eSNP, where 3 of them have small $P_{T2D}$ in WTCCC), and negative controls not. Note that individual eSNPs generally show subtle T2D association (most have $P$ value in the range of $10^{-3}$ to $10^{-4}$ in WTCCC data), see Figure 1. 
	\item Replication in DIAGRAM dataset (Table 1): 9 out of 23 pathways were replicated ($P < 0.05$). In a different replication, removing WTCCC statistics from the DIAGRM statistics, and 4 pathways were replicated. 
	\item Biological evidence of the pathways: some have known function in T2D, e.g. calcium signaling, PPAR signaling, TGF-$\beta$ signaling. The hematopoietic cell lineage pathway reflects mainly the immune response and the inflammatory pathway, which has been extensively linked to T2D. Novel pathways include: tight junction, adherent junction, complement and coagulation, and antigen processing and presentation. 
	\item False negatives: four known T2D pathways were missed. Two due to technical reasons (not selected for pathway association test), and the other two may be due to the incomplete coverage of eQTL:many genes may not affect expression; eQTL study power may be low; relevant tissue not profiled, etc.
\end{itemize}
\end{itemize}

A trans-acting locus regulates an anti-viral expression network and type 1 diabetes risk [Heinig and Cook, Nature, 2010]: 
\begin{itemize}
	\item Data: eQTL of 7 tissues (fat, kidney, heart, skeletal msucle, liver, etc.) in recombinant inbred rat strains. 
	
	\item Identifying IRF7 as a putative master regulator: (1) Found 147 TFs with at least one eQTL in 7 tissues, most ($>90\%$) under trans-regulation. (2) Co-localization analysis: for each TF eQTL, find all transcripts controlled by the eQTL. This gives a list of transcripts for each TF. (3) Enrichment of direct TF targets in the transcripts associated with a TF. Over all TFs: 13 show over-representation of TFBSs in target gene promoters. IRF7 is the strongest with 23 targets, all controlled by a single locus at the rat 15q25 locus. 
		
	\item Define TF-driven subnetworks (IDIN): Co-expression analysis: expand to genes co-expressed with IRF7 targets. This identified 247 genes. The targets are enriched with anti-viral genes. All the genes are in trans
	
	\item Causal gene regulating Irf7 and its targets: from the trans-eQTL locus, sequencing identified the gene Ebi2 (the other genes have no functional SNPs). 
	
	\item Association of IDIN in human: SNPs close to any IDIN genes were significantly more likely to associate with T1D in GWAS than SNPs close to genes not in the network (Figure 3). 
	
	\item Role of Ebi2: the minor C allele of SNP rs9585056 was associated with T1D risk, lower EBI2 expression (cis-eSNP) in both GHS (monocyte eQTL data) and Cardiogenics Study cohorts, and, on average, increased expression levels of IDIN genes in the Cardiogenics Study cohort.
\end{itemize}

Trans-eQTLs Reveal That Independent Genetic Variants Associated with a Complex Phenotype Converge on Intermediate Genes, with a Major Role for the HLA [Fehrmann and Lude Franke, PLG, 2011]: 
\begin{itemize}
\item eQTL data: a genome-wide eQTL analysis on 289,044 common SNPs, present on the Illumina HumanHap300 platform in peripheral blood expression data of 1,469 unrelated individuals. Permutation test, FDR $< 0.05$. 

\item eQTL results: non-parametric Spearman's rank correlation. (1) cis-eQTL $P < 1.73E-3$, number of unique eQTL genes $= 7,589$, number of unique eQTL SNPs $= 48,717$ (2) trans-eQTL $P < 3.6E-9$, number of unique eQTL genes $= 202$, number of unique eQTL SNPs $= 467$. HLA single-nucleotide polymorphisms (SNPs) were 10-fold enriched for trans-eQTLs: 48\% of the trans-acting SNPs map within the HLA.

\item Trans-effect of trait-associated SNPs: among 1,167 SNPs from GWAS catalog, 472 (40.4\%) of these SNPs were cis-eQTLs, affecting 538 genes; 67 (5.7\%) SNPs were trans-acting on 113 genes. 

\item Some examples of trans-eSNPs that also affect phenotypes: 
\begin{itemize}
	\item UC: rs2395185 is the strongest risk locus, and also the strongest SNP, trans-acting on AOAH, an enzyme that modulates host inflammatory responses to gram-negative bacterial invasion. AOAH is significantly co-expressed with colony stimulating factor 1 receptor (CSFR1) and HLA-DRA. Hyperstimulation of CSFR1 is implicated in UC. 
	\item T1D: 59\% (30/51) of the known and tested T1D associated SNPs are cis-acting (on in total 53 unique genes) and 17\% (9/50) are trans-acting on 22 unique genes. Potentially interesting trans-genes include CCL2, CFB, CLN1, KRT19, OSR1 and RARRES1, all strongly co-expressed with each other. CCL2 and CFB are known immune response genes and have been implicated in T1D before. 
	\item Breast cancer: rs3803662 trans-acting on origin recognition complex subunit 6 (ORC6L), involved in DNA replication and has been frequently used as part of prognostic profiles for predicting the clinical outcome in breast cancer. 
\end{itemize}

\item 7 unique pairs of unlinked SNPs that are associated with the same phenotype and that also affect the same downstream genes in trans or cis (at FDR 0.05), the expected number of pairs is only 0.15 (permutation test). Similar results at FDR = 0.50, 18 pairs, 21 times enrichment vs. random. 

\item Examples of unlinked SNPs that affect both expression and phenptypes:
\begin{itemize}
	\item Hemoglobin protein level: three independent loci on  hemoglobin gamma G (HBG2) expression 
	\item Mean corpuscular volume (MCV): two independent SNPs affect several genes, some of which are known blood coagulation genes
	\item mean platelet volume (MPV) : two SNPs, include one cis-, on tropomyosin 1 (TMP1). 
\end{itemize}
 
\item Phenotypic buffering: the effect of trans-eSNPs on genes are higher than the effect of these SNPs on phenotypes. This is a sign of causal influence. Ex. MPV- and MCV-associated SNPs explain between 1.41\% and 10.99\% of trans-expression variation while only explain 0.24\% and 1.12\% of the MPV and MCV phenotype variation.

\item Replication of trans-eQTL: 
\begin{itemize}
	\item Replication in monocyte expression data (Zeller10): 46 out of the 130 different trans-eQTLs, each has $P < 1E-5$ in the monocyte data. 
	\item Replicated 18 trait-associated trans-eQTLs in four different non-blood tissues (subcutaneous adipose, visceral adipose, liver and muscle). 
\end{itemize}

\end{itemize}

Genetic Mapping with Multiple Levels of Phenotypic Information Reveals Determinants of Lymphocyte Glucocorticoid Sensitivity. [Maranville \& Di Rienzo, AJHG, 2013]
\begin{itemize}
\item Concept: lymphocyte GC sensitivity (percent inhibition of cell proliferation due to GC) is correlated with clincal response to GC therapy in a wide range of diseases. So we profile GC sensivity as well transcriptome respose. 

\item Data: PBMC from 88 AAs. First treat with PHA (induce inflammatory response), then with GC. Do expression profiling and GC sensitivity of individuals. 

\item Finding candidate genes of GC sensitivity: genes whose expression correlated with GC sensitivity, found 27 genes (or 85 with a different metric). 

\item GWAS for GC sensitivity: one SNP in RBMS3 highly significant, explaining 26\% of variation in phenotype. This SNP also associates with 14/27 genes correlated with GC sensitivity. 

\item Experimental evidence of RBMS3, tumor suppressor gene: ASE of RBMS3 in this SNP, only in stimulated condition. Knockdown of RBMS3 induces PHA-mediated cell proliferation. 

\item \textbf{Lesson}: lymphocyte in vitro sensitivity as a surrogate of clincal response. Trans-acting effects of SNPs. 
\end{itemize}

Systematic identification of trans eQTLs as putative drivers of known disease associations [Westra and Franke, NG, 2013]
\begin{itemize}
	\item Identifying trans-eQTL: (1) eQTL data: 5000 peripheral blood samples. (2) Trans-eQTL analysis: focuse on 4,500 SNPs in GWAS catalogy, found 1,513 significant trans-eQTL affecting 400 genes. Trans-eQTL show similar effect sizes across various cohorts. (3) Replication of trans-eQTL: $>50\%$ replicated in independent studies of peripheral blood eQTL. 
	
	\item Tissue-specificity of trans-eQTL: most cannot be replicated in adipose. 
	
	\item Case study: trans-eQTL affecting expression of disease-related genes. IKZF1 locus: two SNPs, one associated with SLE, also trans-eQTL of 8 genes, including two groups, complement, and type I interferon response genes, both known to play a role in SLE. Also confirm the IKZF1 as likely cis- gene.  
	
	\item Convergence of multiple GWAS SNPs on the same downstream genes from trans-eQTL (Table 2): in 21 different traits, most are blood related, but also blood pressure, immune diseases and T1D. 
\end{itemize}

Integration of Transcriptomic Profiling, Genome-wide Association, and Network Biology Reveals Molecular Mechanisms Underlying Blood Pressure Regulation. [Huan and Levy, Molecular Systems Biology, 2015]
\begin{itemize}
	\item Strategy: (1) BP (blood pressure) - transcriptome analysis to identify DE genes and subnetworks correlated with BP. (2) To identify possible causal subnetworks, map eQTL, and use BP GWAS to prioritize subnetworks. (3) Upstream regulator analysis of candidate subnetworks. 
	
	\item BP subnetwork analysis: transcriptome data from 3,679 FHS subjects. DE gene analysis: 83 genes. Define co-expressed modules (coEMs) - WGCNA, then correlate eigen-genes with BP. Found 6 coEMs.  
	
	\item Identifying causal modules using SNP set enrichment analysis (SSEA): for each BP-correlated gene set, retrieve their eSNPs, and test overlap of these eSNPs with GWAS loci of BP. Use KS test and Fisher's exact test for assessing if the GWAS p-values of these eSNPs depart from null distribution. Results: 4 coEMs likely causal. 
	
	\item Identification of key drivers (KDs): use blood Bayesian networks (BNs) and PPI. KD is defined as a local network hub whose neighbors show enrichment for BP genes in the causal gene set. Found 545 KDs from PPI network and 131 from BN. KD ranking: association in BP GWAS, BP correlation in expression, their statistics in KD analysis. 
	
	\item Candidate KD gene: SH2B3, a missense SNP associated with BP and hypertension, and a trans-eSNP for 16 genes. Sh3B3 knockout mice show normal baseline BP but elevated BP in response to angiotensis II.  
\end{itemize}

Systems genetics identifies Sestrin 3 as a regulator of a proconvulsant gene network in human epileptic hippocampus [Johnson and Petretto, NC, 2015]
\begin{itemize}
	\item Data: 129 TLE (epilepsy) patients, brain eQTL.
	
	\item Discovery of modules: Gaussian Graphical model, 400 genes. Then refine with hier. clustering: two modules of size about 70 genes.
	
	\item Validation of modules: enrichment of PPI (DAPPLE), KEGG enrichment, conservation in mouse (highly co-expressed).
	
	\item Genetic regulator: two step strategy (Figure 3), genetic association analysis with PC1 of module 1 (Bayesian variable selection model for association test). Then refine the association analysis: using hotspot analysis algorithm (HESS), find several more SNPs. Most are associated with a large percent of genes in module 1.
	
	\item Discovery and validation of SESN3: among genes in the QTL hotspot, SESN3 expression shows the highest correlation with the module. Validation: K.D. of SESN3, and observe reduced expression of genes in the module in macrophages.
	
	\item Lesson: trans-eQTL hotspot may be best studied at the level of modules $<100$ genes. To narrow down to exact regulators of module: use co-expression of the putative regulator with genes in the module. Usually, genes in a QTL region are functionally unrelated, so this test may be able to discriminate genes in QTL regions.
\end{itemize}

Limited statistical evidence for shared genetic effects of eQTLs and autoimmune-disease-associated loci in three major immune-cell types [Chun and Costapas, NG, 2017]
\begin{itemize}
	\item Data: AID GWAS with 272 loci. CD4 T cells, monocytes and LCL eQTL data.
	
	\item Only in about 25\% of cases, GWAS and eQTL signals colocalize.
\end{itemize}

Shared Genetic Regulatory Networks for Cardiovascular Disease and Type 2 Diabetes in Multiple Populations of Diverse Ethnicities in the United States [Shu \& Yang, review for PLG, 2017]
\begin{itemize}
	\item Goal: infer the common biological processes disrupted in the two diseases, and the key driver (KD) genes. 
	
	\item Background: Patients of T2D are 2-6 times more likely to have CVD. 

	\item Construction of tissue-specific modules: use WGCNA on gene expression data in relevant tissues, including adipose, blood, liver, heart, islet, kidney, muscle and brain, from human and mouse studies. 2,600 modules and annotate the modules using Reactome and KEGG. 
	
	\item Relating co-expression modules to diseases: MSEA (market set enrichment analysis). (1) SNP-gene mapping: either eQTL; or SNPs within 50kb of genes that have annotations in Regulome. (2) For a gene set, test the enrichment of association in the mapped SNP set. Use chi-square like test, and obtain null from ``shuffling gene labels''. Results: 79 modules associated with CVD and 54 with T2D. 2 modules with both.  
	
	\item Annotating modules and assessing the relationship between functional categories and diseases: significant sharing of functional categories (KEGG, Reactome) between two diseases. 
	\begin{itemize}
		\item Associate a category with a disease: if it is annotated to at least one disease module.  
		\item Shared categories/pathway: to rank them, consider the pathway associated with both diseases (through some modules). A pathway that is associated with multiple modules in both diseases would be favored (Figure 3). 
	\end{itemize}
	Top shared pathways include well-established processes such as lipid metabolism, glucose metabolism, oxidation, cytokine signaling. One novel finding: BCAA pathway, the genes themselves show little association, but their network neighbors do. 

	\item Identifying key drivers: use GIANT networks and Bayesian networks from 25 CVD and T2D relevant tissues. KDs: genes whose local neighborhood neighborhoods show enrichment of genes from disease-associated modules. Results: 226 KDs. To find KDs associated with both CVD and T2D: significant in both phenotypes, replicated by GIANT and Bayesian networks, order by the strength of association between KD subnetworks and CVD/T2D Found 15 KDs at FDR $<0.1$. 
	
	\item Connection of KDs with known CVD/T2D genes: overrepresentation of these genes in KD neighborhoods in the networks. Remark: this may be circular. 
	
	\item Role of KDs in CVD and T2D: 
	\begin{itemize}
		\item Mouse transcriptome data: 100 mouse strains with both expression and relevant phenotypes such as lipid levels, fasting glucose. All 14 KDs have significant trait associations. 
		\item Transcriptome perturbation by KDs: CAV1 knockdown changes expression of neighbor genes in vitro and in mouse model. 
		\item Mouse GWAS data: association of cadiometabolic phenotypes with KD or KD subnetwork genes. 
	\end{itemize} 
	Note that KDs themselves are not top signals in GWAS. 
	
	\item Biological evidence of KDs. CAV1: KO mouse shows phenotype. HMGCR: target of statin. IGF1. Three ECM KDs: e.g. SPARC inhibits adipogenesis and is associated with insulin resistance. 
	
	\item Remark: the evidence of KDs do to fully establish their causality: (1) transcriptome correlation; (2) in mouse GWAS, association with KD neighbors, not KD genes. 
	
	\item \textbf{Lessons}:
	\begin{itemize}
		\item Gene pathway analysis: using co-expression modules in disease-related tissues can be a better srategy than using pre-defined GO/KEGG pathways. 
		\item Key driver gene identification: many neighbors (or targets) that are disease related. The KD genes may not have strong association in GWAS themselves. Also note that in this analysis, there are relatively large number of KDs in one disease.  
		\item Validation strategy: using mouse data, both GWAS and transcriptome association. 
	\end{itemize}
\end{itemize}

Unraveling the polygenic architecture of complex traits using blood eQTL meta-analysis (eQTLGen) [Vosa and Franke, Biorxiv, 2018]
\begin{itemize}
	\item Data set: 30K samples, meta-analysis.
	
	\item Cis-eQTL: most are within 100kb. Distal eQTL (more than 100kb): 37\% are in Hi-C contact.
	
	\item Gene prioritization of GWAS using cis-eQTL: SMR and DEPIT results do not agree.
	
	\item Trans-eQTL mapping: 10K trait-associated SNPs, about 1/3 are trans-eQTL, 6000 genes. Possible mechanisms: (1) TF - targets: 2.2 fold enrichment. (2) Colocalization with cis-eQTL: COLOC estimates 52\%. (3) Mediation by cis-genes: (4) Gene co-expression. Together, TF-targets, cis-mediation and co-expression explain 17\% of trans-eQTL. Also found interactions of cis-SNPs and cis-gene expression in determining trans-gene eQTL: 5 fold enrichment.
	
	\item \textbf{Analysis}: how to use trans-eQTLs to understand genetics of complex traits? Suppose we have a disease-associated SNP, and it is a trans-eQTL of some genes. The problems are: (1) What is the cis-mediator? (2) What are the trans-genes that mediate the effects of the SNP? We need additional evidence that the trans-genes are disease-related.
	
	\item Examples of how trans-eQTLs inform disease genetics: (1) GWAS SNP of age of menarche: cis-eQTL of ZNF131, which is involved in ER signaling. Trans-genes are enriched with ZNF131 targets by K.D. (2) T1D SNP: no cis-eQTL, but coding variant, trans-genes enriched with IFN response. (3) Asthma SNP: cis-gene involved in B cell proliferation, many trans-genes (cell type composition change).
	
	\item Convergence of multiple GWAS hits to the same genes (Figure 5): SNPs of SLE converge to a cluster of IFN response genes. Effect sizes of the SNPs on genes also show correlation: if one SNP shows a large effect in a a gene, other SNPs also show large effects.
	
	\item eQTS: expression quantitative trait scores, correlation of expression and PRS. Justification: let $G_s$ be SNP $s$, and $X_j$ be expression of gene $j$, and $Y$ phenotype. Our causal model for gene $j$ is: 
	\begin{equation}
	\{G_s\} \xrightarrow{\beta_{s,j}} \{X_j\} \xrightarrow{\gamma_j} Y
	\end{equation}
	The PRS of sample $i$ is basically $Y_i$ (expectation). So we have the correlation of expression and PRS: 
	\begin{equation}
	\Cov(X_j, \text{PRS}) = \gamma_j \sigma_j^2 + \sum_{k \neq j} \gamma_k C(k,j)
	\end{equation}
	where $\sigma_j^2$ is the genetic variance of gene $j$, and $C(j,k)$ genetic correlation of expression of gene $j$ and $k$. 
		
	\item eQTS results: analysis of 1,200 traits and 28K samples, found 18K eQTLs effects, representing 689 unique traits and 2,500 unique genes. Average 10-20 genes per trait (some traits have none). Ex. ASD: found 21 genes, 3 genes have known functions in synapse.  
	
	\item \textbf{Lesson}: prioritization by cis-eQTL is NOT easy, as shown by the disagreement of SMR and DEPICT.
	
	\item Remark: for majority of trans-eQTLs, we do not know their mechanisms. One mechanism is cell type composition change: could lead to many trans-eQTLs.
	
	\item \textbf{Lesson}: linking trans-eQTL and GWAS: (1) enrichment of some disease-related pathways; (2) plausible cis-genes with disease functions; (3) SNP leading to compositional change of disease-related cell type(s). 
	
	\item \textbf{Remark}: under what conditions, we can translate correlation of expression with PRS to causal relations? In general, if PRS and gene share genetic variants, we will see correlation. 
\end{itemize}

\subsection{TWAS and cis-QTL Assisted GWAS}

Trait-associated SNPs are more likely to be eQTLs: annotation to enhance discovery from GWAS. [Nicolae \& Cox, PG, 2010]: 
\begin{itemize}
	\item Motivation: the hypothesis is the expression changes can lead to complex diseases. This has two consequences: (1) a significant fraction of GWAS SNPs influence gene expression; (2) some eQTL genes likely increase disease risk, if expression perturbed. Testing these two hypothesis using GWAS and eQTL data. 
	\item eQTL data: HapMap LCL data in SCAN. 
	\item GWAS SNPs (1,598 from the GWAS catalog) are more likely to be eQTLs than MAF-matched random set of SNPs: at eQTL threshold $P < 10^{-4}$, found 625 GWAS SNPs (expected number about 600 at 5\%), at $P < 10^{-6}$, found 46 (expected about 30 at 5\%), and 17 (expected about 2 to 3 at 5\%). 
	\item Crohn's disease from WTCCC: (1) top 10,000 eQTLs: 357 SNPs associated with the trait (Crohn's disease) at $P < 0.01$, while the expected number was 117-178; (2) top 1000 GWAS-SNPs with Crohn's disease: enriched with eQTLs, e.g. at eQTL threshold 2, there are 324 SNPs (143.5 expected by chance), and at eQTL threshold 3, there are 172 SNPs (only 18.6 by chance). 
	\item Other WTCCC diseases: (1) top 10,000 eQTLs: T1D and RA have significantly more SNPs than expected with phenotype associations $P < 0.01$. (2) the enrichment of eSNPs (eQTL function scores larger than 3) among the SNPs with the strongest associations to CD, T1D, RA, hypertension and bipolar disorder. 
	\item Lesson: (1) Weak or intermediate SNPs may be eQTLs too; (2) eQTL data of LCL may be useful for complex diseases not directly related. 
	\item Remark: the possible bias in the study, MAF (it is easier to identify trait-associated SNPs with high MAF because of higher power), LD (it is easier to find SNPs with higher LD). In the study, MAF is controlled, but not local LD. 
\end{itemize}

Liver and adipose eQTLs are enriched for association to T2D [Zhong \& Schadt, PG, 2010]: 
\begin{itemize}
	\item Hypothesis: many GWAS SNPs may be (weak) causal variants that affect gene expression. If this hypothesis is true, then among all eSNPs, some may influence expression of genes important for T2D, thus the T2D-associated SNPs may be enriched. 
	
	\item eQTL Data: 427 liver cohort [Schadt08]; about 900 individuals with liver, subcutaneous adipose and omental adipose tissues. eQTL identification follows [Schadt08]. High overlap of eSNPs identified in three tissues in one cohort: about 70-80\% of eSNPs in one tissue are also found in the other two. Also among liver eSNPs from the first cohort, there was a 66\% overlap in eSNPs indentified between the two studies.
	
	\item The whole set of eSNPs: the distribution of $P_{T2D}$ ($p$ value in GWAS of T2D), is significantly different from random distribution (random SNP sets matching MAF). Ex. in DGI study, 6.2\% of the eSNPs (241 out of 3,888 total) had $P_{T2D} < 0.05$, compared to a mean of 5.2\% (202 out of 3,888) in random SNPs. The enrichment of SNPs with $P_{T2D} < 0.05$ is low, however, only 1.19 fold. 
	
	\item Increase of enrichment: (1) using eSNPs of adiptose tissue of the genes that show differential expression in mouse studies (top 25\%, i.e. 9,9000 genes). (2) further enrichment if using eSNPs of the causal disease subnetwork (MEMN1) of 159 genes: 37\% vs. 9\% (random) with $P_{T2D} < 0.05$. 
	
	\item T2D causal gene: of 158 genes in the adipose causal network, 117 have cis or trans- eQTL in human, however, only 8 were identified with strong adipose cis-eQTL. Among these, only ME1 was associated with at least one cis-eSNP that was also associated with T2D with $P_{T2D} = 0.002$. The function of ME1 was verified. 
\end{itemize}

Schizophrenia susceptibility alleles are enriched for alleles that affect gene expression in adult human brain [Richards \& O'Donovan, Molecular Psychiatry, 2011]: 
\begin{itemize}
	\item Hypothesis: polymorphisms that are associated with schizophrenia are enriched among brain eSNPs. If this hypothesis is true, then we classify all putative risk alleles of schizophrenia as ``top eQTLs'' and ``bottom eQTLs'', we would expect that the ``top eQTL'' risk alleles are better predictors of schizophrenia than ``bottom eQTL'' risk alleles. (The reason for testing predictability is the schizophrenia risk alleles are very weak.)
	
	\item Data: GWAS data from ISC and MGS. Brain eQTL data form [Myers07] and [Webster09]: analysis using linear regression controlling for many covirates, including a gene as a marker of number of neurons. Use cis-eQTLs in the analysis (trans-eQTLs did not show significant results). ``Top eQTLs'' are defined as top 5\% ($P < 0.02$) and ``bottom eQTLs'' as bottom 5\%. 
	
	\item Comparison of ``top eQTL'' and ``bottom eQTL'' risk alleles: schizophrenia risk alleles are defined as $P < 0.5$, and the two classes of risk alleles are assessed by: applying the risk alleles in an independent test group, and compare the ``polygenic score'' (predictive score, summing all risk alleles weighted by log-odds ratio in the training data) of the cases and controls. 
	
	\item The difference in the scores between the top and bottom cis-eQTLs was significantly greater in the cases than in the controls for all analyses. In other words, among the variants selected for marginal association to schizophrenia, those that additionally show evidence for being cis-eQTLs predict affection status better than those variants showing no evidence for being cis-eQTLs. 
\end{itemize}

Loci nominally associated with autism from genome-wide analysis show enrichment of brain expression quantitative trait loci but not lymphoblastoid cell line expression quantitative trait loci. [Davis \& Cox, Mol Autism, 2012]
\begin{itemize}
	\item Background: pathway analysis of autism GWAS reveals some intriguing pathways, e.g. ubiquitination, synthesis and degradation of ketone bodies. 
	\item Goal: enrichment of brain eQTL amongst top signals from the recent AGP GWAS. 
	\item Data: AGP GWAS (4 SNP lists at $p < 0.001$), the SNP and CNV Annotation Database (SCAN) and genome-wide expression datasets in brain.  
	\item eQTL mapping: Cerebellar (GSE35974) and parietal cortex (GSE35977) cis and trans eQTL were generated from 153 individuals of European ancestry. ComBat and Surrogate Variable Analysis were used to adjust for batch and both known and unknown covariate effects. Imputed genotype dosage data were analyzed for association with expression using PLINK. $P <0.0001$ for cis-eSNP, and Bonferroni correction for trans-eSNP (23K probes). 
	\item Enrichment analysis: among the four SNP list from ASP GWAS, count the number of significant eQTL. To test its significance, randomly sample the same number of SNPs as the GWAS SNP lists (matching MAF), and count significant eQTL. 
	\item Significant enrichment of parietal ($P<0.004$) and cerebellar ($P <0.003$) eQTL, but not LCL eQTL ($P=0.502$) among the top signals from the most broadly inclusive dataset of spectrum diagnosis including all ancestries. About 60 overlapped SNPs in parietal and cerebellar eQTL. SNP statistics in enrichment analysis: 
	\begin{itemize}
		\item A total number of 539 brain eQTL found in the primary AGP GWAS top signals (256 independent eSNPs). 
		\item Of the 214 SNPs that target only one gene, 124 (58\%) act in cis and 88 (42\%) act in trans. 
		\item 140 genes were uniquely implicated as eQTL targets. 62 (44\%) were cis implicated and 78 (56\%) implicated in trans. Only 18 (13\%) of the 140 genes were also targeted by eQTL found in LCLs, and only 10 genes (7\%) were found in both cerebellum and parietal tissues. In all pairwise comparisons of tissues, the overlap was not statistically significant. 
	\end{itemize}
	\item Implicated genes: 
	\begin{itemize}
		\item SLC25A12 was multiply-implicated by a unique set of 31 SNPs (in LD), all in cis (GWAS-p about 2E-5 to 1E-3).  
		\item PANX1 was targeted by nine cis eQTL and implicated in multiple tissues. 
		\item PANX2 was targeted by three trans eQTL, and the gene modulates the timing of neuroprogenitor commitment to a neuronal lineage in the hippocampus. 
		\item Comparison with autism DEX genes: four of the 140 brain eQTL target genes overlapped with the 1,153 differentially expressed genes identified by Voineagu et al, including SLC25A12  and PANX2. 
	\end{itemize}
	
	\item Lessons:
	\begin{itemize}
		\item Trans-eQTL overlap with GWAS associations may be significant (more than half in this dataset), and the genes implicated may be highly plausible: e.g. for PANX2, both DEX in ASD vs. controls, and supported by other functional evidence.
		\item eQTL overlap between multiple brain regions, and between tissues: may be small. 
	\end{itemize}
	
\end{itemize}

Targeted allelic expression profiling in human islets identifies cis-regulatory effects for multiple variants identified by type 2 diabetes genome-wide association studies. [Locke \& Harries, Diabetes. 2014]
\begin{itemize}
	\item Allele expression imbalance (AEI) measurement: amplification and measurement of mature (i.e. spliced) mRNA species and normalisation of allelic expression using genomic DNA from the same individual. 
	\begin{itemize}
		\item Genotyping of the genomic DNA: Sanger sequencing or TagMan genotyping assay to confirm that the lead SNP was heterozygous. 
		\item cDNA: treat cells with DNase to digest DNA, then reverse transcribed. PCR amplification. 
		\item Genomic DNA samples should show a 1:1 allelic ratio and thus any departure from 0 illustrates unequal amplification of alleles which must be corrected for.
		\item Mean average allelic expression measurements: determined from two independent cDNAs reverse transcribed and amplified on different days. 
		\item Paired two-tailed T-tests, comparing genomic DNA and cDNA values from the same donor, were used to determine statistical significance for allelic expression. 
	\end{itemize}
	\item Validation of the robust measurement of the AEI: 
	\begin{itemize}
		\item Correlation between allelic expression measurements determined from independent cDNAs reverse transcribed and amplified on different days
		\item Correlation between allelic expression measurements calculated from SNPs in high linkage disequilibrium with each other and residing within the same gene 
	\end{itemize}
	\item T2D candidate SNPs and genes: 65 lead SNPs from GWAS, 1525 proxy SNPs (r2>0.8, CEU, 1000 Genomes Phase 1) were found. 45/1590 (2.8\%) map to exons of 23 human RefSeq genes. 
	\item AEI results: For 18 of the 23 genes, the TaqMan SNP assay can be designed to map entirely to exonic sequence. After filerting (too few heterozygous samples, too low gene expression level), allelic expression could be determined for 14 genes in samples from 36 white, non-diabetic donors. 7/14 genes show AEI (Figure 1) - results of cDNA show significant depature from 0 (genomic DNA for the purpose of normalization). And five genes are validated with another exonic proxy SNP. The AEI values of the five genes are small: $< 1.25$. 
	\item Discussion:
	\begin{itemize}
		\item The study is limited to genes with exonic SNPs, while cis-eQTL can study any candidate genes in a locus. Future studies may consider using intronic SNPs to measure allelic expression. 
		\item Comparison with targeted approach vs. RNA-seq: using RNA-seq, on average, each exonic SNP has coverage about 25. To detect an AEI =1.25, $>$ 500 mapped reads ($>20$ samples) would be needed. 
	\end{itemize}
\end{itemize}

\subsection{Multi-omics QTL}

Metabolic and transcriptional profiling of liver metabolism [Ferrara \& Attie, PG, 2008]: 
\begin{itemize}
	\item Motivation: reconstruction the causal networks among transcripts and metabolites. 

\item Methods: 
\begin{itemize}
	\item Data: 60 F2 mouse, genotyped (293 markers), liver gene expression is profiled, and concentration of 67 liver metabolites are determined through MS/MS, including 15 AAs and urea cycle intermediates, 45 acyl-carnitines and 7 organic acids (TCA and related intermediates). 
	\item QTL identification: both eQTL and mQTL, use interval mapping method. 
	\item Network inference: construct glx (glutamine) network, first select 250 transcripts most correlated with glx level and in the category ``metabolism''. Then test causality of locus, metabolite and transcript and generate the network [Chaibub, Genetics, 2008]. 
\end{itemize}

\item Results: 
\begin{itemize}
	\item Metabolites of common functional group are highly correlated, e.g. among all AAs, most pairs have CC $> 0.5$. 
	\item Correlation among metabolites and transcripts: metabolites generally correlated with related processes. E.g. 15 AAs correlated with transcripts in protein metabolism, glycolysis, TCA cycle and lipid metabolism. 
	\item Glx network: predict that Glx causally control expression of genes Agxt, Arg1 and Pck1. In experiment, adding 10 mM glx to culture cells, and found expression change of a number of transcripts, including the model predicted ones. 
\end{itemize}
\end{itemize}

\subsection{Systems Genetics of Model Organisms}

Systems genetics of complex traits in Drosophila [Ayroles \& Mackay, NG, 2009]:
\begin{itemize}
\item Motivation: popoulation variation of gene expression traits, and phenotypes; genetic basis of complex traits through association with gene expression. 

\item Methods: 
\begin{itemize}
	\item Data: 40 fruit fly lines from natural popoulation, each is assayed with 18,800 transcripts (25 flys/sex/line, 3-5d young), and 6 traits: resistance to starvation stress, time to recover from a chill-induced coma, life span, a startle-induced locomotor response, mating speed and competitive fitness. 
	\item Analysis of expression variation: ANOVA to partition variation in expression betwen sexs, among lines, and the sex $\times$ line interactions.
	\item Transcriptional modules: genes with correlated expression across population, first use ANOVA to identify the line terms, then cluster genes (graph-based method). 
	\item Transcripts associated with quantiative traits: let $Y$ be transcript level, $S$ be the influence of sex, and $T$ be the influence of trait, then regression: $Y = \mu + S + T + S \times T + \epsilon$, to identify the genes significantly associated with phenotypic variations. 
	\item Modules in trait-associated genes: the goal is to find modules that are correlated (in addition to the common effect of the trait on all genes). Thus use residual terms after regression with traits to define correlation. 
\end{itemize}

\item Results: 
\begin{itemize}
	\item Genetic architecture of expression: nearly 80\% transcripts are expressed in adult flies, about 90\% of these have sex-biased expression. Two-thirds of the expressed trasncripts also show line variations. X-chromosomes is a hospitable spot for female-biased but not male-biased genes (mutations in X chromosomes tend to have large deleterious effect on males). 
	\item Correlation of transcripts: the genome as whole is highly correlated at the transcript level. Cluster genes into 241 modules, the biggest two coorespond to sex-biased genes. Genes within a module tend to show similar tissue-specific expression (FlyAtlas), involved in the same pathway, or enriched with TFBSs. 
	\item Association with phenotypes: a small amount of correlation among traits (trade-offs: e.g. resistence to starvation have longer life spans but reduced competitive fitness). Identify hundreds of genes for each trait and generally more than half are verified (mutation by $P$-element, and assess the trait). 
	\item Transcriptional modules associated with traits: e.g. fitness trait: immune response, visual perception, function of nevous system, chemosensation and sex-specific transcripts. 
\end{itemize}
\end{itemize}

Long-distance phenotypic logic chain enable precise inference of uncorrelated traits [Xionglei He, 2019]
\begin{itemize}
	\item Yeast data: 815 haploid segregants, and 405 quantitative traits, such as morphology during each phase of cell cycle. Estimation of heritability of traits: $h^2$ median 0.15 and $H^2$ median 0.42. 
	
	\item Uncorrelated traits can predict exemplar traits: e.g. Nuclear brightness in whole cell, 7 uncorrelated traits can predict well. Test these on all traits, find that the prediction performance using only uncorrelated traits can reach high levels, $\approx H^2$. In contrast, using SNPs or gene expression does much more poorly (Figure 2c,f). 
	
	\item Explanation of the observation by latent dimensions: use autoencoder to learn 20 latent variables (linear transfer function). Then show that many traits are explained by these 20 latent variables: explain 85\% variation. Example: nuclear brightness $Y$ is explained by 5 latent variables. The seven uncorrelated traits that predict $Y$ include 5 that are also explained by these 5 variables. Two other variables are needed to ``cancel out'' extra dimensions. 
	
	\item Geometric view: Figure 3. suppose we have three variables, $\eta, \alpha, \beta$. It's possible that $\eta$ is in the hyperplane defined by $\alpha$ and $\beta$, but $\eta$ is orthogonal to both. From latent variable perspective, it means that they all depend on the same set of latent variables (and 0 for other latent variables), thus can predict each other. 
	
	\item Analysis [personal notes]: is it possible that $Y$ is a linear function of other random variables, but $Y$ and each of them marginally uncorrelated? Let $Y = \sum_j X_j \beta_j$, then for any $i$, we have
	\begin{equation}
	\Cov(X_i, Y) = \sum_j \beta_j \Cov(X_i, X_j)
	\end{equation}
	If we define the pairwise covariance matrix of $X_i, X_j$ as: $\Sigma = [\Cov(X_i,X_j)]_{ij}$ and $\beta$ is the vector of $\beta_j$s, then the condition can be written in matrix form as: 
	\begin{equation}
	\Sigma \beta = 0
	\end{equation}
	When $\Sigma$ is not full ranked, which would be the case if there are linear dependency of $X_j$'s, then there are infinitely many solutions with $\beta \neq 0$. 
	
	\item Analysis: probabilistic PCA perspective. Suppose $z$ is latent random variable, with $z \sim N(0, I)$, and $x_1 = W_1 z$ and $x_2 = W_2 z$, where $W_1$ and $W_2$ are row vectors of the loading matrix. Then we have: 
	\begin{equation}
	\Cov(x_1, x_2) = \E(W_1 z (W_2 z)^T) - \E(W_1 z) \E(W_2 z)^T = W_1 \E(z z^T) W_2^T - W_1 \E(z) \E(z)^T W_2^T = W_1 W_2^T
	\end{equation}
	It is possible that this term is 0. If we have a relatively large number of latent variables, and their effects on observed variables are generally independent, then we may have this term often close to 0. \\
	However, even given marginally independent random variables, we can still predict one from others. Suppose $Y$ can be expressed as $y = \beta z$, and $x$'s can also be expressed as linear functions of $z$ as in PPCA: $x = Wz$. Then we can solve $z = W^{-1} x$, where $W$ is loading matrix. We then have: $y = \beta W^{-1} x$. 
	\begin{itemize}
		\item Remark: this analysis is based on PPCA. It is also possible to do the analysis with classical PCA. 
	\end{itemize}
	
\end{itemize}

%%%%%%%%%%%%%%%%%%%%%%%%%%%%%%%%%%%%%%%%%%%%%%%%%%%%%%%%%%%%
\section{Gene Regulation in Complex Traits}

Experimental studies of disease loci and non-coding sequences [personal notes]
\begin{itemize}
\item Reference: [Pancreatic islet enhancer clusters enriched in type 2 diabetes risk-associated variants, NG, 2014], [Obesity-associated variants within FTO form long-range functional connections with IRX3, Nature, 2014], [Leveraging cross-species transcription factor binding site patterns: From diabetes risk loci to disease mechanisms, Cell, 2014], [TCF7L2 loci from T2D GWAS, Marcola Nobrega lab, Dec, 2015]

\item Transcriptional activity of non-coding sequences and variants:
\begin{itemize}
	\item To show the non-coding sequences and variants are functional: control, wt. sequence and variant in luciferase assay, show w.t. sequences can drive expression, but it is abolished by mutation [Islet paper, Figure 6C]. 
	
	\item Allele-specific expression or eQTL: number of risk alleles associated with expression of target genes. Alternatively, ASE in heterozygous individuals. [FTO paper, Figure 2B]
\end{itemize}
 
\item TF binding and regulation of enhancers: 
\begin{itemize}
	\item EMSA to study protein binding of non-coding sequences. [Islet paper, Figure 6B]. 
	\item To study the role of TF in regulation: knockdown of TF by RNAi [Islet paper, Figure 3C]
\end{itemize}

\item In vivo expression pattern of enhancers:
\begin{itemize}
	\item In vivo transgenic reporter assays: zebrafish or mouse, establish the tissue-specific regulatory activities of enhancers. [FTO paper, Figure 1B]
\end{itemize} 

\item Enhancer-promoter interactions: 
\begin{itemize}
	\item 3C or 4C: profile interactions between a test loci/enhancer with all regions within a certain distance (e.g. 1Mb). [FTO paper, Figure 1A]. 
\end{itemize}

\item Tissues and endophenotypes affected by regulatory sequences: identify the relevant tissues and endophenotypes, experimentally study how they are affected by regulatory sequences or tissue-specific overexpression/deletion of the genes.  
\begin{itemize}
	\item Relevant tissues: activity pattern of enhancers or expression pattern of genes, eQTL. 
	
	\item Endophenotype study: e.g. number and size of adipocytes in T2D, a target of TCF7L2 gene [TCF7L2 work from Marcelo lab]. Use tissue-specific expression.
	
	\item Remark: the ability to detect relevant tissues (and stages) is a main advantage of studying non-coding sequences underlying diseases - similar to tissue-specific perturbation of gene expression.  
\end{itemize}
\end{itemize}

Approaches for establishing the function of regulatory genetic variants involved in disease [Knight, Genom Med, 2014]
\begin{itemize}
	\item Examples of regulatory variants important for diseases: 
	\begin{itemize}
		\item Variant in 3'UTR: Crohn’s-disease-associated variant in the 3’ UTR of IRGM that alters binding by the microRNA mir-196, enhancing mRNA transcript stability and altering the efficacy of autophagy. 
		\item Alternative splicing: a variant of TNFRSF1A associated with multiple sclerosis, which encodes a novel form of TNFR1 that can block tumor necrosis factor 
	\end{itemize}
	
	\item Regulatory epigenomic data (Table 1: comprehensive resources)
	\begin{itemize}
		\item FANTOM5: high-resolution context-specific maps of TSS and their usage for 432 different primary cell types, 135 tissues and 241 cell lines, enabling promoter-level characterization of gene expression. Also map active enhancers by eRNA. 
		\item UCSC Genome Browser: Variant Annotation Integrator. 
		\item Ensembl genome browser includes the Ensembl Variant Effect Predictor. 
		\item RegulomeDB: functional regions (ENCODE), eQTL, prediction of motif disruption. 
		\item Combined Annotation-Dependent Depletion method: 63 types of genomic annotation to establish deleteriousness for SNVs and small insertion-deletions [Kircher \& Shendure, NG, 2014]
		\item Conservation: 8.2\% of the human genome is subject to negative selection and is likely to be functional [Rands \& Lunter, PLG, 2014]
		\item SNPnexus: coding SNPs, Regulatory elements (conserved TFBS in human/mice/rat, vista enhancers, microRNA sites), conserved sites from PhastCons and GERP
		\item GWAS3D: [Junwen Wang lab] cell-type specific annotation, TFBS scanning, histone modifications, chromatin interactions
		\item MAPPER2: TFBSs located in the upstream sequences of genes from the human, mouse and D.melanogaster genomes, combines TRANSFAC and JASPAR data with the search power of profile hidden Markov models (HMMs)
		
	\end{itemize}
	
	\item Findings/insights from QTL studies: 
	\begin{itemize}
		\item Importance of trans-eQTL: may affect expression of other genes through transcriptional regulation or signaling (1) a cis-eQTL for the transcription factor KLF14: associated with T2D and HDL, act as a master trans regulator of adipose gene expression. (2) a cis-eQTL involving IFNB1: associated in trans with a downstream cytokine network, found in stimulated cells. 
		\item Importance of considering context/condition: eQTL analysis of the innate immune response transcriptome in monocytes defined associations involving canonical signaling pathways, key components of the inflammasome, downstream cytokines and receptors. Often disease-associated variants and were identified only in induced monocytes. 
		\item Coding SNVs: An estimated 15\% of codons [Stergachis, Science, 2013] specify both amino acids and transcription factor binding sites. 
	\end{itemize}
	\item Methods for functionally studying regulatory variants
	\begin{itemize}
		\item Allele-specific expression (ASE): early studies show that in addition to the small number of classical imprinted genes showing monoallelic expression, up to 15 to 20\% of autosomal genes show heritable allele-specific differences. From [Lappalainen, Nature, 2013],  LCLs from 462 individuals: almost all the identified ASE were driven by cis-regulatory variants rather than genotype-independent allele-specific epigenetic effects. 
		\item Allele-specific TF binding: use ChIP-seq, applied to heterozygous cell lines or individuals can provide direct evidence of relative occupancy by allele. 
		\item Chromatin interactions, in particular, Capture-C: cross-linking of chromatin interactions followed by capture of hundreds of target regions. 
		\item Genome editing by CRISPR/Cas9: (1) eQTL of SLFN5, used CRISPR-Cas9 to demonstrate loss of inducibility by IFN-beta on conversion from the heterozygous to homozygous state in a human embryonic kidney cell line. (2) T2D-associated variant in PPARG2: replaced the endogenous risk allele in a human pre-adipocyte cell strain with the non-risk allele and showed increased expression of the transcript.
	\end{itemize}
\end{itemize}

A map of open chromatin in human pancreatic islets [Gaulton \& Ferrer, NG, 2010]
\begin{itemize}
	\item Motivation: use open chromatin (CRE map) to study T2D genetics. 
	\item Open chromatin in pancreatic islets defined by FAIRE-seq. 
	\item Overlap with T2D loci: of 350 SNPs in strong LD with a reported T2D locus, 38 SNPs at 10 loci overlapped islet FAIRE regions. Verification of rs7903146 in TCF7L2: test the hypothesis that rs7903146 variant changes chromatin state (accessibility) and the enhancer activity (hence likely a causal variant of T2D): 
	\begin{itemize}
		\item Allele imbalance at chromatin state: identify 9 individuals with heterozygous rs7903146, then find the allelic imbalance (T:C ratio) in the open chromatin (FAIRE-isolated DNA). 
		\item Enhancer activity: luciferase reporter assay in islet cells, compare the two enhancers differing in rs7903146. Only one allele shows enhancer activity. No difference in a control cell type. 
	\end{itemize}
	\item Lesson: use allele-specific chromatin state (similar to ASHM) to demonstrate that a SNP has an effect on chromatin and regulatory activity. 
\end{itemize}

Systematic Localization of Common Disease-Associated Variation in Regulatory DNA [Maurano \& Stamatoyannopoulos, Science, 2012]: 
\begin{itemize}
\item DHS mapping: 349 cell and tissue samples, including 85 cell types studied under the ENCODE Project and 264 samples studied under the Roadmap Epigenomics Program. About 100 from cell lines, primary tissues, hematopoietic and differentiated cells, etc; and 233 diverse fetal tissue samples across days ~60 to 160 after conception. Average of 200K DHS per cell type, and a total of 3M DHS (42.2\% of the genome). 

\item Distribution of 5K non-coding SNPs in GWAS catalog (more than 600 studies/traits): 
\begin{itemize}
\item Hypothesis: non-coding SNPs involved in complex diseases are enriched in DHS. 
\item Fully 76.6\% of all noncoding GWAS SNPs either lie within a DHS (57.1\%, 2931 SNPs) or are in complete linkage disequilibrium (LD) with SNPs in a nearby DHS (19.5\%, 999 SNPs). 
\end{itemize}

\item Cell- and developmental stage specificity: 
\begin{itemize}
\item Hypothesis: DHS containing the non-coding SNPs are tissue/developmental stage-specific. 
\item Examples: for a number of diseases, the variants are located in DHS specific to disease-relevant tissues. Ex. for cardiovascular diseases, find a SNP in the DHS specific to fetal heart. 
\item Importance of early gestational exposures: 88.1\% (2583) lie within DHSs active in fetal cells and tissues. 
\item Enrichment or depletion of replicated disease-specific GWAS variants in fetal-stage DHSs: the most enriched traits are, menarche, cardiovascular disease, and body mass index (gestational exposures or growth trajectory known to play a role). Relative depletion in fetal DHSs of aging-related diseases, cancer, and inflammatory disorders with presumed (postnatal) environmental triggers.
\end{itemize}

\item Identification of target genes of DHS: 
\begin{itemize}
\item Method: correlate DNase sensitivity of DHS with DNase I sensitivity patterns at cis-linked promoters. Use $r > 0.7$ as a cutoff. Verification via  paired-end tag sequencing (ChIA-PET). 
\item Identified 419 DHS-gene pairs (within 500kb). Fully $40.8\%$ of correlated DHS-gene pairs span $>250$ kb and 79\% represent pairings with distant promoters versus those of the nearest gene. 
\item Examples of target genes that play plausible roles in diseases: a SNP associated with platelet count, located in a DHS that physically interacts with the 222-kb distant promoter of JAK2. 
\end{itemize}

\item Alternation of TFBSs within DHS: 
\begin{itemize}
\item Hypothesis: many non-coding variants alter TFBS binding and chromatin states (thus affecting phenotypes). 
\item Define TFBSs: scanning for known motif models at a stringency of $P < 1e-4$. 
\item Of GWAS SNPs in DHSs, 93.2\% (2874) overlap a transcription factor recognition sequence. 
\item Detection of altered chromatin structure in heterozygous SNPs (an imbalance in the fraction of reads obtained from each allele). Nearly 40\% of GWAS variants in similarly sequenced DHSs would be expected to show allelic imbalance. 
\item In general, TFBS binding removes nucleosome, making the site more accessible (sensitive). Figure 2C: the first and second examples (TFBS alleles have higher counts). 
\end{itemize}

\item Disease-associated variants in TFBSs of specific disease classes: 
\begin{itemize}
\item Hypothesis: for a TF related to a disease class, its TFBSs may be enriched in the disease-related DHS (i.e. the DHSs containing the noncoding variants associated with this disease). 
\item Disease-related TFs: using known TFs (e.g. HNF1a for MODY), and interacting TFs of the known TFs (from Ingenuity)
\item Autoimmune diseases: IRF9 and 15 interacting TFs, 24.4\% (64/262) of GWAS SNPs within DHSs of immune cells and associated with autoimmune disease alter one or more of the 15 transcription factor motifs from the IRF9-centered network. 
\item Multiple related diseases often share the same TFs: TF-disease networks (Figure 4) for autoimmune diseases and cancer. Also six neuropsychiatric disorders with 23 transcription factors. 
\end{itemize}

\item Identification of disease-related cell types: 
\begin{itemize}
\item Hypothesis: in a disease-related cell type, the DHS will be enriched with disease-associated variants. 
\item For Crohn, MS and QRS duration (heart): selective enrichment of SNPs associated with GWAS in relevant cell types. Furthermore, even at relatively low GWAS threshold (e.g. $p < 0.01$), the enrichment can be detected (higher enrichment at higher $p$ threshold). 
\end{itemize}

\item Lessons: 
\begin{itemize}
\item Target genes of non-coding disease SNPs: some experimental data can reveal the long-range interactions between non-coding sequences and promoters (ChIA-PET, 5C). Also the correlation between epigenetic states of enhancers and those of promoters or gene expression. 
\item TFs involved in complex diseases: enrichment of TFBSs in disease-associated non-coding SNPs 
\item Disease-related tissues: enrichment of disease-associated SNPs in tissue-specific DHS. 
\end{itemize}

\item Remark: 
\begin{itemize}
\item Target genes of DHS: through correlation of DHS and promoter sensitivity pattern across tissues. Problem: DHS may be tissue-specific (open in specific tissues) while promoters may be open in any tissue a gene may be expressed. 
\item Role of TFs in disease: through enrichment of TFBSs in disease-associated non-coding SNPs (located within DHS). Stringent p-value threshold. 
\item Common diseases, common networks: some TFs are found to be associated with multiple diseases. However, these diseases are known to share SNPs: suppose we have a single SNP common to multiple disorders, and this SNP is located in a DHS containing BS of some TF, then this TF would appear to be associated with mulitple disorders. 
\end{itemize}
\end{itemize}

Pancreatic islet enhancer clusters enriched in type 2 diabetes risk-associated variants [Pasquali \& Ferrer, NG, 2014]
\begin{itemize}
\item Hypothesis: dysregulation of TF target enhancers in pancretic islet cells increases the T2D risk.
\item Overview: ChIP-seq of pancretic islet TFs, and histone markers. Enrichment of GWAS-T2D SNPs in these CREs.

\item Regulatory map of pancretic islets:
\begin{itemize}
\item ChIP-seq of five TFs
\item Open chromatin states: FAIRE-seq and ChIP-seq of H2A.Z
\item Key histone modifications: H3K4me3, H3K4me1, H3K27ac and CTCF-binding. Clustering of these four states on open chromatin reveals five classes: promoters (C1), poised or inactive enhancers (C2), active enhancers (C3), CTCF-bound sites (C4) and the rest (C5).
\end{itemize}

\item Pattern of cis-regulatory map:
\begin{itemize}
\item Auto- and cross-regulatory relationship between five TFs.
\item Targets of the five TFs often overlap.
\item 92\% of TFBS mapped to open chromatin, and they bind to distinct chromatin states.
\end{itemize}

\item TF binding enhancers drive islet-specific transcription
\begin{itemize}
\item TF-bound C3 sites (active enhancers) are associated with islet-specific gene expression, while non-C3 sites do not show association. Shown by luciferase assay in beta cells vs. fibroblast cells (Figure 3A). 

\item Experimental validation of some of the TF-bound enhancers (TFBS clusters): knockdown of TF by RNAi reduces gene expression (Figure 3C). Also more likely to interact with promoters in 4C (Figure 3E).

\item Additional TF motifs were found in the enhancers.
\end{itemize}

\item Sequence variation in islet enhancers is associated with T2D:
\begin{itemize}
\item Enrichment of T2D and glycemia GWAS hits in clustered C3 sites, but not in orphan C3 sites.
\item Relaxing the threshold of GWAS p-values: fold enrichment of the sites in clustered C3 sites.
\item A catelog of causal cis-regulatory variants of T2D: intersect GWAS hits with clustered C3 sites and DNA-binding motif analysis.

\item Results of specific GWAS loci, T2D risk variant at ZFAND3: the variant disrupt TF binding, (EMSA experiment, Figure 6B) and change gene expression in vitro (luciferase in mouse MIN6 beta cells, Figure 6C). 
\end{itemize}

\item \textbf{Lessons}:
\begin{itemize}
\item TF binding and chromatin states: TF binding are often associated with open chromatin. Functional targets of TFs are often active enhancers (in open chromatin).
\item Active enhancers, but not other states (promoters or inactive enhancers) are associated with tissue-specific expression.
\end{itemize}
\item Remark: ideally, we will study how DNA variations  correspond to change of epigenetic states/expression.
\end{itemize}

Leveraging Cross-Species Transcription Factor Binding Site Patterns: From Diabetes Risk Loci to Disease Mechanisms [Claussnitzer \& Laumen, Cell, 2014]
\begin{itemize}
	\item Motivation: the causal SNPs should be in CREs, which has the feature of cluster of conserved TFBSs. 
	\item PMCA algorithm: 
	\begin{itemize}
		\item Start with a disease-related SNP, first find all non-coding SNPs (ncSNP) in high LD. Then find the region surrounding an nc SNP (60 bp) from the human genome, and the orthologous region in 15 vertebrate species. 
		\item Define the conserved TFBSs, conserved TFBS modules (oc-occuring TFBSs at the same order). The motifs are from Genomatix library (800 human TFs).  
		\item Scoring of the ncSNP (and the surround region): a significant enrichment of phylogenetically conserved TFBS modules. Basically counting the number of conserved TFBS, then evaluate its significance by randomization. 
	\end{itemize}
	\item Clusters of sites of homoebox TFs is a distinct feature of T2D loci: define positional bias as TFBS clustering relative to transcription start sites. From the T2D loci related to sequences (8 loci), find positional bias of Homeobox TFs, such as CART and PDX1.
	
	\item Remark: many functional TFBSs are not conserved, and this may significantly limit the power of PMCA. 
	  
	\item Lessons: 
	\begin{itemize}
		\item Cross-species conservation can be used to define CREs that are likely causal loci of diseases. The method may have high precision, even if the sensitivity is low. 
		\item A disease may be associated certain distinct TFs (or TF families), and the signature of the TFs can be found near the GWAS loci. 
	\end{itemize}
\end{itemize}

Genetic and epigenetic fine mapping of causal autoimmune disease variants [Farh and Bernstein, Nature, 2015]
\begin{itemize}
	\item Fine-mapping method PICS: first show that $\chi^2$ statistics decay with $r^2$ to causal SNP. Next, causal SNP may not be the strongest SNP due to statistical flucutations, infer the probability that a SNP is a lead SNP given that another SNP is causal, using permutation. 
	
	\item Results of fine-mapping: (1) GWAS catelog index SNPs: \textbf{only 5\% represent a causal SNP}. (2) Most GWAS signals cannot be resolved to a signal causal SNP. 
	
	\item Causal SNPs and immune enhancers: PICS SNPs are enriched in stimuls-dependent enhancers. About 60\% of SNPs are in immune-cell enhancers, many of which are induced by immune activation. 
	
	\item Cell-type signatures of complex diseases: causal SNPs of GWAS loci from 21 AIDs, and enrichment in enhancers from different tissues. Results better than the expression pattern of genes targeted by coding GWAS hits. Almost all AIDs: CD4 T cells. Some such as SLE preferentially map to B cells. T1D: also enriched in pancreatic islet. UC: gastrointestinal tract elements. 
	
	\item Causal SNPs and disruption of gene regulation: (1) TFBS (from ENCODE) enrichment: many TFs, top ones are NF-kB, IRF4 (Figure 5b). (2) 800 high confidence SNPs (average PIP 0.3): only 7\% change  motifs of over-represented TFs, this compares to about 1\% by control SNPs. 13\% change motifs of any TFs, similar to background. 26\% residing within 100 bp of motifs.  
\end{itemize}

The osteoarthritis and height GDF5 locus yields its secrets [NG, 2017]
\begin{itemize}
	\item GDF5 locus found in GWAS of height. The gene GDF5 is a member of BMP family and a good candidate. 
	
	\item Fine-mapping: transgenic mice carrying the upstream and downstream sequence of GDF5. Several experiments to identify the enhancer GROW1 (2.9kb): expression of reporter, rescue GDF5-null phenotype, deletion of the locus leads to phenotype.  
	
	\item Positive selection of the GROW1 haplotype: long haplotype blocks in high LD, a signature of positive selection. Comparison of haplotypes between Euroasian vs. African: the allele is more common in Eurasian than African, also found in Nean and Denosovan. 
	
	\item Unresolved issue: could be three independent SNPs in the locus. 
\end{itemize}
%%%%%%%%%%%%%%%%%%%%%%%%%%%%%%%%%%%%%%%%%%%%%%%%%%%%%%%%%%%%
\section{Network Genetics}

Problems of network genetics: 
\begin{itemize}
	\item \textbf{Reconstruction} of molecular networks. 
	\item \textbf{Organizational principles} of networks: biological networks are far from random, and may possess certain features including master regulators, convergent nodes, coordinated modules, and so on. Identify and rationale such features. Note that the definition of organization here refers not only to topology, but also the influences. 
	\item \textbf{Implications on phenotype}: what does the network say about the effect of changing one node on phenotypes?  
\end{itemize}

Principles of network analysis of complex traits: 
\begin{itemize}
	\item \textbf{Association of gene modules with traits}: this is based on the principles of modularity and guilt-by-association. It can be done in multiple ways. Gene modules can be identified via clustering of expression patterns across genetic perturbations. 
	\begin{itemize}
		\item Dysregulation of modules in diseases: correlation of gene expression with traits; or change of co-expression pattern in diseases. 
		\item Enrichment of disease-related genes in modules. 
	\end{itemize}
	
	\item \textbf{Causal gene network} from eQTL: it is possible to construct causal networks from eQTL. This serves as a reference to interpret the link between genes and phenotypes.
	
	\item \textbf{Understanding the function from genes to phenotypes}: We can study the causal chain of events from genes to phenotypes. Suppose we represent the state of system as $x$ (network state), and its disease state $y$, our goal is to learn the function $y = f(x)$. Without the network, we can only do linear modeling (sometimes allow interaction terms), but it is very limited. With the network, the function is more structured and greatly constrained.  
	
	\item \textbf{Master regulators}: the structure of biological networks is that some nodes play unusually large roles than other nodes. Identifying such ``master regulators/players'' can be helpful. 
	
	\item \textbf{Importance of biological contexts}: for many genes, their effect on phenotypes depend on biological contexts, both environmental (eg. smoking) and cellular conditions. 
	
	\item \textbf{Multi-layered structure of networks}: the networks are organized into multiple layers, e.g. SNVs influence expression of RNA, which affects translation and protein levels. The enzyme levels control metabolic states. At even higher level, some part of networks are cell-specific, some are active across-tissues.\\
	Remark: simlarity to multi-layered ANN for deep learning, we have some features at each layer that determine the next layer.     
\end{itemize} 

Using networks to study candidate genes of diseases [personal notes]:
\begin{itemize}
\item Intra-connectivity of candidate genes: average degree, clustering coefficient of a subnetwork. 

\item Inter-connectivity with known genes: 
\begin{itemize}
	\item Number of edges between two groups of genes. 
	\item Average distance between two groups of genes. 
	\item Some kind of weighted distance: a new gene highly connected to seed genes will receive higher score. Thus we can compare the scores of candidate genes with control genes. 
\end{itemize}

\item Subnetworks: in addition to statistical testing, often it is informative to visualize the subnetworks formed by the candidate and known genes. The subnetworks tend to be functionally coherent (enrichment analysis). 

\item Choice of networks: PPI network, gene co-expresison network. Ideally, use networks that are more relevant, e.g. synaptic PPI network for psychiatric diseases. 

\item Importance of control: to assess the significance, we need some control genes. This needs to be done carefully, e.g. candidate genes all have certain properties (e.g. brain expression), and the control genes should match these properties. 
\end{itemize}

Network overview: Dan Nicolae's talk at Complex trait journal club
\begin{itemize}
	\item Resources: Ingenuity, GeneWays (from text mining), STRING. 
	
	\item Clustering coefficient: defined on a node, how related the neighbors of this node are. 
	
	\item Network analysis for establishing the functional relationship among genes: when comparing two sets of genes and want to show one set is on average closer, need to be careful about the possible confounders. 
	\begin{itemize}
		\item Example: autism study, compare de novo genes vs. random genes. De novo genes are generally longer, and might have more PPIs. 
	\end{itemize}
	
	\item Building co-expression network (WGCNA): let $s_{ij}$ be correlation between two nodes, then the edge weight $a_{ij} = s_{ij}^{\beta}$. 	
\end{itemize}

Network medicine: a network-based approach to human disease [Barabasi \& Loscalzo, NRG, 2011]:
\begin{itemize}
	\item Motivation: only about 10\% of human genes have a known disease association. What are properties of disease networks, in terms of how disease genes are distributed? 
	
	
	\item Gene topology: hub genes tend to be essential, more conserved, and pleiotropic (deletion leads to more phenotypes). However, not all essential genes are disease genes in humans. Mutations in genes that are essential in early development lead to spontaneous abortions. Essential genes that are not associated with disease show a strong tendency to be associated with hubs and are expressed in multiple tissues.
	
	\item Modularity: proteins involved in the same disease have an increased tendency to interact with each other. For example, one group observed 290 physical interactions between the products of genes associated with the same disorder, representing a tenfold increase relative to random expectation. Thus, each disease can be linked to a well-defined neighbourhood of the interactome, often referred to as a ``disease module''. 
	
	\item The network parsimony principle: causal pathways are the shortest paths connecting the known disease components. 
	
	\item Methods that use the modularity principle for predicting disease genes: 
	\begin{itemize}
		\item Linkage methods: the direct interaction partners of a disease protein are likely to be associated with the same disease phenotype. Example: severe combined immunodeficiency syndrome, the set of genes within the locus whose products interacted with a known disease protein were shown to be tenfold enriched in true disease-causing genes. 
		
		\item Disease module-based methods: constructing the interactome in the tissue and cell line of interest and identifying a subnetwork, or disease module, that contains most of the disease-associated genes. Variants of this methodology have been applied to a wide range of diseases and pathophenotypes, including several different types of cancer, neurological diseases, cardiovascular diseases,  systemic inflammation, obesity, asthma, T2D, etc. 
		
		\item Diffusion-based methods: Proteins that interact with several disease proteins will gain a high probabilistic weight, as will those that may not directly interact with any disease proteins but are in close network proximity to them. 
	\end{itemize}
	
	\item Shared components hypothesis: Diseases that share disease-associated cellular components (genes, proteins, metabolites or microRNAs) show phenotypic similarity and comorbidity. In the human disease network (HDN) - two diseases are connected if sharing at least one gene, 867 of 1,284 diseases with an associated gene are connected to at least one other disease, and 516 of them belong to a single disease cluster. For example, cancers form a tightly interconnected and easily detectable cluster, which is held together by a small group of genes that are associated with multiple cancers.
	
	\item Same gene, different diseases: many disease pairs that share genes do not show significant comorbidity. One explanation is that different mutations in the same gene can have different effects on the gene product, and therefore different pathological consequences91 that are organ and context dependent. Such 'edgetic' alleles affect a specific subset of links in the interactome. 
	
	\item Metabolic diseases: links that are induced by shared metabolic pathways are expected to be more relevant than are links based on shared genes. For example, purine metabolism consists of 62 reactions associated with 33 diseases. Comorbidity analysis confirms the functional relevance of metabolic coupling: disease pairs that are linked in the MDN have a 1.8-fold increased comorbidity compared to disease pairs that are not linked metabolically. 
	
	\item Application: therapies that involve multiple targets, avoid side effects. Can one systematically identify multiple drug targets that have an optimal impact on the disease phenotype?
\end{itemize}

NEW: Network-Enabled Wisdom in Biology, Medicine, and Health Care [Schadt et al. Sci Transl Med, 2012]
\begin{itemize}
	\item Using gene networks (co-expression, causal) for association analysis: basic strategy is to find modules/pathways linked to diseases. 
	\begin{itemize}
		\item Procedure: (1) identify modules; (2) eSNPs associating the expression of genes are extracted from eQTL data; (3) enrichment/pathway association analysis of the eSNPs. 
		\item Other ideas, e.g. (1) use correlation of expression data and phenotypes (e.g. diff. expression) to identify candidate modules; (2) use genotypes to distinguish causal and correlative/responsive modules. 
	\end{itemize}
	\item Gene-environment interactions and the importance of context: ``increasing evidence suggests that most genetic risk variants are dependent on particular environmental contexts to effect risks for CCD''. Use networks to identify the combined risk-enrichment for groups of functionally associated genes. Examples:
	\begin{itemize}
		\item The effects of most DNA sequence variants linked to type 2 diabetes in Caucasians are manifested only in patients with a body mass index above 26. 
		\item DNA sequence variations linked to certain types of high blood pressure exert their negative effects only in the context of low physical activity. 
	\end{itemize}
	\item Decomposition of context: macroenvironment (smoking, exercise, toxicity, etc.) and microenvironment (cellular conditions). The context affects how DNA variants increase the disease risk. To understand DNA risk variants and context: need to use omics data to create molecular networks. 
	\item Multi-layered molecular networks: with omics data of RNA, proteins, metabolites, possible to construct multi-layered networks, e.g. DNA $\rightarrow$ RNA $\rightarrow$ protein $\rightarrow$ metabolite $\rightarrow$ phenotype. 
	\item Constructing tissue/organ specific networks are essential: they are likely more important in later phases of disease development, when pathological changes are spreading across the borders of individual organs. 
	\item Implications for health-care: 
	\begin{itemize}
		\item DNA profiles: risk assessment early in life. Prevention through controlling environment, etc. 
		\item Activity profiles: biomarkers (CRP, liver-enzymes, and future ones) from OMICS data can provide a snapshot of network states, thus reflecting any signs of molecular pathology (such as tumor growth, atherosclerosis, inflammation, or immune responses).
		\item Treatment: the DNA profiles and network states can guide the selection of treatment strategies. And the markers can monitor the disease progression/recovery. 
	\end{itemize}
\end{itemize}

Leveraging models of cell regulation and GWAS data in integrative network-based association studies [Califano \& Schadt, NG, 2012]	
\begin{itemize}
	\item Pathway-wide association study (PWAS) strategy: the difficulty is pathways are often poorly characterized. 
	\item Integrative network-based association studies (INAS): favored, the simultaneous reconstruction of context-specific gene regulatory networks. 
	\item Why INAS approach? Linear pathways are a poor representation. 
	\begin{itemize}
		\item Complexity of GRNs: a TF may be regulate hundreds of genes, and combinatorial regulation of multiple TFs. Each TF is further regulated by many signal transduction proteins. 
		\item Cellular context and higher-level interactions among cells. 
	\end{itemize}
	\item Reverse enginerring of networks: 
	\begin{itemize}
		\item High throughput experiments of PPI, kinase, TF regulation, etc. An initial, albeit sparse, snapshot of regulatory networks, especially when integrated with other types of data that can help contextualize individual interactions.
		\item Computational predictions: often rely on perturbations (internal or external) or temporal data, and often integrate multiple sources. 
		\item Functional interactions (GIs): another layer. 
	\end{itemize}
	\item Canonical pathway analysis: most successful examples in immunological pathways, NF-kappa B. 
	\item Dysregulation of subnetworks in diseases: 
	\begin{itemize}
		\item Dysregulated gene set analysis via subnetworks (DEGAS) and interactome dysregulation enrichment analysis (IDEA). Examples: Parkinson's disease and B-cell lymphoma. 
		\item Search for subnetworks enriched in linkage or association of diseases. Or use networks (complexes) to increase the power of detecting epistasis. 
	\end{itemize}
	\item Molecular phenotypes/eQTL: construct networks from eQTL data. 
	\item GRN analysis: identifying master regulators. Examples: human high-grade glioma, and normal physiological formation of germinal centers. 
	\item Diseasome approaches: exploits previous biological knowledge of gene similarities and dissimilarities across diseases. Ex. G allele of the rs2076530 in BTNL2 is more frequent among individuals with T1D and RA than in healthy controls, whereas the A allele was more frequent in SLE than controls. 
	\item Lessons: 
	\begin{itemize}
		\item A broad view of networks: physical interactions, functional, causal, correlated. 
		\item Importance of context-dependency of networks and dynamic nature. Diseases often involve multiple tissues. 
		\item Analysis of networks: master regulators, integrative analysis of multiple ones across diseases
	\end{itemize}
\end{itemize}

The human disease network [Goh \& Barbasi, PNAS, 2007]: 
\begin{itemize}
	\item Background: 
	\begin{itemize}
		\item Diseases are often caused by mutations of related genes: e.g. Zellweger syndrome is caused by mutations in any of at least 11 genes associated with peroxisome biogenesis. 
		\item Mutation of one gene can give rise to multiple disorders: e.g. mutations in TP53 have been linked to 11 cancer-related disorders. 
	\end{itemize}
	
	\item Data: gene-disease bipartite graph constructed from OMIM, which include data of both monogenetic disease and complex traits. 1,284 disorders and 1,777 disease genes. 
	
	\item Human disease networks: link two diseases if they share a gene. The network is clustered into many modules of major disease classes. Cancer is a large cluster; metabolic diseases have low genetic heterogeneity and are not very connected; neurological disorders show high locus heteogeneity and also represent the most connected disease classes. 
	
	\item Disease gene network (DGN): two genes are linked if they are associated with the same disease. several disease genes (e.g., TP53, PAX6) are involved in as many as 10 disorders, representing major hubs in the network.
	
	\item Disease associated gene modules: genes associated with the same disease tend to: interact with each other through PPI (10-fold enrichment); tend to be expressed in the same tissues; co-expressed; and in the same GO categories. 
	
\end{itemize}

Diverse types of genetic variation converge on functional gene networks involved in schizophrenia [Gilman \& Vitkup, Nature Neuro, 2012]
\begin{itemize}
\item Motivation: given a diverse set of genetic data (putative risk genes), find the subnetworks that are enriched with risk genes. 

\item Background network construction: first obtain gene features, which measures how related two genes are. These features include PPI, common annotations (a single feature based on GO), phylogenetic profile, and so on. To map these features to response (whether they have the same phenotype), use Naive Bayes. To train the model, use a known gene-disease network. The results are expressed as LR scores for two genes as edge weights (same phenotype vs. different phenotype). 

\item Genetic data: 159 de novo SNVs, 712 from de novo CNVs and 173 from 14 GWAS loci from SCZ data. 

\item Search for clusters: in a test cluster, each gene is from one of the three sources, and for CNVs and GWAS region, only one gene is allowed in a cluster. 
\begin{itemize}
	\item Cluster scores: the sum of LR scores of all edges in the cluster. Obviously, the scores are not normalized by cluster sizes. 
	\item Cluster significance: random selection of clusters (matching the node connectivity), the obtain $p$-values. 
\end{itemize}

\item Biological processes and validation of the clusters: top cluster about 30 genes, $p < 0.001$. Cluster I: enriched in neurodevelopmental processes such as axon guidance, neuron projection development. Also evidence that the genes are expressed early in development. 

\item Relation to other disorders: the Cluster genes are significantly more connected to autism gene sets. Also find evidence that the genes of autism and SCZ may affect dendritic growth differently. 

\item Remark: 
\begin{itemize}
	\item Search of clusters is challenging: esp. due to the constraints (one gene per CNV/GWAS region). 
	\item No validation of individual genes, e.g. the genes picked by the program from GWAS loci. 
\end{itemize}
\end{itemize}

Integrated Systems Approach Identifies Genetic Nodes and Networks in Late-Onset Alzheimer's Disease [Bin Zhang, Cell, 2013]	
\begin{itemize}
	\item Background: 
	\begin{itemize}
		\item Progress in LOAD research is fundamentally limited by our reliance on mouse models of severe familial/early-onset Alzheimer's disease. 
		\item Genetics of LOAD: APOE accounts for 30\% of genetic variance. GWAS implicates immunity (CLU, CR1, CD33, EPHA1, MS4A4A/MS4A6A), lipid processing (APOE, ABCA7), and endocytosis (PICALM, BIN1, CD2AP) as important causal biological processes. More recently, low-frequency missense variants in APP and TREM2 were found to confer strong protection or elevated risk of LOAD. 
	\end{itemize}
	\item Brain expression data: 1,647 autopsied tissues from dorsolateral prefrontal cortex (PFC), visual cortex (VC), and cerebellum (CB) in 549 brains of 376 LOAD patients and 173 nondemented healthy controls. Expression analysis: robust linear regression adjusting for covariates: age and sex, postmortem interval (PMI) in hours, and sample pH and RNA integrity number. 
	\item Analysis pipeline: (1) Network construction: co-expression network; use genetic markers to anchor the causal network. (2) Module ranking: differential connectivity in LOAD and normal networks, enrichment of brain eSNPs. 
	\item Module differential connectivity (MDC): comparison of the networks constructed from LOAD and normal people. Define MDC as the ratio of the average connectivity for any pair of module-sharing genes in LOAD compared to that of the same genes in the nondemented state. Detect a number of modules with high change of MDC (gain or loss of connectivity, GOC or LOC). This cannot be captured by the traditional diff. expression analysis.
	\item Functional categories of GOC or LOC modules: eg.  the immune module shows the statistically most significant functional enrichment of all modules. 
	\item Association of modules with LOAD pathophysiology: A covariance matrix of the average expression correlation (absolution value) between 49 modules (using PCA) and 25 LOAD-related traits is constructed. The immune/microglia showed correlation to the greatest number of LOAD-related neuropathology traits. 
	\item Brain eSNP mapping and modules enriched with brain eSNPs: 10K eSNPs identified at FDR 10\%. And cis-eSNPs are used to construct causal networks. 
	\item The immune/microglia module is highlighted: (1) significant differential connectivity in LOAD; (2) the most significant enrichment of functional categories; (3) the highest degree of gene-expression correlation to LOAD neuropathology; (4) the PFC version of the module was highly enriched for brain eSNPs.
	\item Ranking causal regulators: based on (1) regulatory strength: the number of downstream nodes of the immune module in the causal network; (2) diff. expression in LOAD patients. TYROBP scored the highest. 
	\item Convergent molecular pathway: TREM2 is known to associate and signal via TYROBP. The subnetwork also contains previous top GWAS risk loci including MS4A4A, MS4A6A, and CD33. TYROPB positions in several microglia activation-signaling cascades. Hypothesis: TYROBP may be associated with neuronal pruning activity of the complement system that may be reawakened in LOAD via amyloid-beta and tau aggregates. 
	\item Experimental validation of TYROBP: in mouse microglia cells, perturbation of TYROBP, and measure diff. expression. The DE genes are highly enriched with the immune module. Also observe a strong correlation between pathway distance to TYROBP in the network and the fraction of DE genes. 
	\item Lessons: 
	\begin{itemize}
		\item Change of connectivity in conditions: could capture more information than the standard DE. Example: suppose a pathway (complement system) is activated in pathological condition, a number of genes will be simultaneously up-regulated, creating significant coexpression. 
		\item Multiple measures to implicate disease-causing modules: change of expression/connectivity in disease; correlation with disease markers; functional coherence of the module. Could also add: enrichment of potential disease-related genes. 
		\item Identifying key regulators of disease module: by using the causal network. 
	\end{itemize}
\end{itemize}

Widespread macromocular interaction perturbations in human genetic disorders [Sahni \& Vidal, Cell, 2015]
\begin{itemize}
\item Data: about 3,000 mutations in 1,140 genes from HGMD, Mendelian diseases. For control, use common variants from 1000 GP. 

\item Different types of mutatations when describing the change of interaction profiles: quasi-wild type (mutation does not change), quasi-nul (abolish most of the interactions), edgetic (specific edges). 

\item Impact of mutations on protein stability: if a mutation reduces stability, the mutated protein will require more interactions with chaperons and quality control factors (QCF), so using these interactions to profile the stability change. Overall, relatively small effect on protein stability, but a small fraction of mutations lead to increased interactions (reduced stability) - about 28\% with at least one of the seven QCFs. 
\begin{itemize}
	\item For those with increased binding, the mutations are often located in the core of the protein (than surface or disordered regions).  
\end{itemize}

\item Impact of mutations on PPI: use Y2H to profile PPI. Out of 1,300 PPIs, found 521 perturbed interactions. Among all mutations, about 2/3 change interactions (edgetic or quasi-null), and about half of them are edgetic. 
\begin{itemize}
	\item Quasi-null mutations are often unstable or mis-folded and have lower expression level. 
	\item Comparing with controls (common variants): only 8\% change interaction profile. 
	\item Polyphen and conservation analysis could distinguish changes vs. nonchanges (quasi-wildtype), but not between edgetic and quasi-null. 
	\item Edgetic mutations are enriched in structurally exposed residues compared to quasi-null mutations. 
\end{itemize}

\item Impact of mutations on protein-DNA interactions (PDI): enhanced Y1H on 70 TFs and 152 enhancers. 38\% of mutations are quasi-null, 43\% edgetic and 19\% quasi-WT. Including both gain and loss of PDIs: likely due to loss of specificity from mutations. Quasi-null mutations are highly enriched in DBD regions.  

\item Remark/Questions: 
\begin{itemize}
	\item The claim that using change of PPI can achieve precision of 96\% and sensitivity 61\% to distingush disease and non-disease alleles is misleading: the data are highly enriched with disease mutation.  
	\item The causal link from PPI changes to phenotypic consequences: not studied much in the paper. 
	
	\item Can we \textbf{extrapolate} from the current findings for unseen mutations? 
\end{itemize}


\end{itemize}

%%%%%%%%%%%%%%%%%%%%%%%%%%%%%%%%%%%%%%%%%%%%%%%%%%%%%%%%%%%%
%%%%%%%%%%%%%%%%%%%%%%%%%%%%%%%%%%%%%%%%%%%%%%%%%%%%%%%%%%%%
\chapter{Genetics of Complex Traits}
\section{Overview of Genetics of Complex Traits} 

Mutations and functional changes of genes: including the change in the regulatory sequences (abolish expression of a gene is similar to the loss of activity of the protein product) [Human Molecular Genetics, 3rd Ed., Chapter 14]
\begin{itemize}
\item Two types of functional consequences of mutations: 
\begin{itemize}
	\item Loss of function mutations: often heterogeneous, as many different mutations can lead to loss of function. 
	\item Gain of function mutations: generally rare. Mutational homogeneity is a indicator of gain of function. 
\end{itemize}

\item Loss of function mutations: 
\begin{itemize}
	\item Small deletions and insertions, nonsense-mutations including premature termination, splicing mutations, frameshift, point mutation of essential AAs, etc. 
	\item Haploinsufficiency: some genes are dosage sensitive, thus 50\% reduction of activity may cause abnormal phenotype. Dosage-sensitive genes are few, including: genes needed in large quantiy (e.g. elastin), genes that interact in certain proportions (e.g. in signaling/metabolic switches, in protein assembly such as $\alpha$ and $\beta$ globins). 
	\item Dominant negative effect: a non-functional copy of the gene may interfere with the function of the normal copy. Ex. collagens - the assembly is disrupted by the non-functional copy; bHLH-ZIP family of TFs that bind in dimers - mutants can sequester functioning molecules into inactive dimers. 
\end{itemize}

\item Gain of function mutations: usually cause dominant phenotypes. 
\begin{itemize}
	\item Chromosome rearrangements that induce exon shuffling/fussion: common in cancer. 
	\item Overexpression: e.g. by transposition of a gene to a active chromatin environment.
	\item Mutations that make a protein insensitive to regulation: this would cause constitutively active proteins, e.g. constitutively active receptors in GPCR signaling. 
	\item Protein aggregation: often from unstable expanding repeats (which may leads to other effects such as reduction of transcription of nearby genes), notably CAG repeats that encode poly-Q tracts. Can be also due to chance events of protein misfolding. Particuarly important in neurodegenerative diseases.  
\end{itemize}
\end{itemize}

The pathways from genes to diseases: [Human Molecular Genetics, 3rd Ed., Chapter 16]
\begin{itemize}
	\item Multiple paths to deficiency of one protein: may not be the protein itself, could be any step that leads to the production of this protein. Ex. immunodecificiency (lack of immmuoglobins) can be caused by failures in: immunoglobin gene processing, $B$-cell maturation, or other steps in the development of the immune system. 
	\item Protein complexes and pathways: mutations of different members may lead to similar phenotypes. Ex. collagen of skin: mutations in COL1A1, COL1A2, type XI collagen all have similar phenotypes. 
	\item Mutations in different members of a gene family: could cause related or overlapping syndromes. Ex. fibroblast growth factor receptors (FGFRs): 1-4, the mutants may affect the balance of different forms. 	
	\item Mutations often affect only a subset of the tissuss in which the gene is expressed: e.g. HD gene (Huntington disease) is widely expressed, but the mutation mainly affects brain. 
	\item Dependence on genetic background and on environment: this would be expected for instance in dosage-sensitive genes. Ex. a common variant, R402Q, in tyrosinase gene (a key enzyme of melanoctyes) is normal, but can lead to ocular albinism in the presence of a mutation in MITF (a gene involved in differentiation of melanoctyes). 
\end{itemize}

Genetic architecture of complex traits/diseases: 
\begin{itemize}
	\item Genetic architecture is ultimately determined by the complex, often opposing effects of selection, population history, migration and mutation rates.
	\item Diversity of genetic architecture: different between, for example, autims and intelligence, height. E.g. pervasive epistatic effects have been documented in autoimmune conditions, morphology and susceptibility to cancer, but fear-related phenotypes consists almost entirely of multiple small additive effects.
	\item A fundamental problem is the risk model: what genetic factors cause the disease? This may include for example, a single-hit risk model vs. multi-hit risk model (a major mutation causes the disease in one individual or multiple ones)? What role does genetic background (epistatic effect) play? 
	\item Implication: understanding why genetic architecture differs for different traits could help when choosing the correct tools to find the underlying genes and deciding whether to look for common or rare variants. 
	\item Reference: [Eichler \& Nadeau, Nature, 2010]
\end{itemize}



Missing heritability of complex traits/diseases: [Eichler \& Nadeau, Nature, 2010] 
\begin{itemize}
	\item Rare variants with possibly large effects: rare variants are individually rare, but collectively frequent,
	\item Many common variants with small effects: with many tests performed, there is a high false-negative rate in GWAS, as true associations are hidden in the fog of random associations. 
	\item Genetic interactions (non-additive effects) such as dominance and epistasis: epistatis is found to be important in a number of diseases. 				\item Epigenetic effect: example, parent-of-origin effect (genetic interaction with parent). Epigenetic effects beyond imprinting that are sequence-independent and that might be environmentally induced but can be transmitted for one or more generations.   
	\item Structural variations (deletions, duplications and inversions) of genomes: (1) individually rare but collectively common variations in the human population. An estimated 8\% of the general population carry a large ($>500$ kb) deletion or duplication that occurs at an allele frequency of $<0.05\%$. (2) Copy number variations (CNV): some genes are highly variable among individuals, are enriched in genes associated with drug detoxification, immunity and environmental interaction. 
	\item Genotype-environment ($G \times E$) interaction: we are largely unable to identify what the relevant environments are. 
\end{itemize}

Estimating genetic architecture: 
\begin{itemize}
\item Motivation: even if we cannot detect all individual loci of a trait, we may still be able to estimate some key parameters of the genetic architecture such as the number of causal genes, the effect size distribution, the explained variance, etc. 

\item Strategy: model or simulate how the key parameters, $\pi$ - the fraction of causal genes, and $\gamma$ - effect size distribution, influence the key aspects of data, such as: the enrichment of low $p$-value genes relative to null distribution, the enrichment of de novo mutation events in probands vs. siblings, and other informative patterns: e.g. how often a gene has multiple de novo mutation events. 

\item Explained variance: suppose we want to estimate how much phenotypic variance can be explained by a type of genetic variance. For linear model, this is determined by the effect size and frequency of causal alleles; for binary trait, one can use liability threshold model. 

\item Reference: [Sanders, De novo mutations revealed by whole-exome sequencing are strongly associated with autism, Nature, 2012], [Iossifov, De Novo Gene Disruptions in Children on the Autistic Spectrum, Neuron, 2012]
\end{itemize}

Major diseases and traits studied by GWAS (\url{http://genome.gov/gwastudies/}):
\begin{itemize}
\item Metabolic diseases: obesity, T2D
\item Heart and circulation system diseases: coronary heart disease, hypertension, sudden cardiac arrest, myocardial infarction
\item Neurological and behavior diseases: Alzheimer's disease, Parkson's disease, schizophrenia, bipolar disorder, ADHD, autism
\item Auto-immune diseases: T1D, multiple sclerosis, asthma, Crohn's disease, rheumatoid arthritis
\item Cancer: breast cancer, prostate cancer, lung cancer, leukemia, colorectal cancer, ovarian cancer
\item Other diseases: kidney stones, gallstones, AMD, osteoporosis
\item Responses to treatment: radiation response, aromatase inhibitors (breast cancer), Warfarin (anti-coagulant), anti-depressant, antipsychotic therapy, hepatitis C treatment, interferon beta therapy, treatment for acute lymphoblastic leukemia
\item Phenotypic traits: height, body mass index, waist circumference, waist-hip ratio, longevity, eye color
\item Behavior traits: alcohol dependence, smoking behavior, Nicotine dependence, personality dimensions, heroine addiction, cognitive ability
\item Heart and blood function: pulmonary function, bone marrow density, blood pressure
\item Metabolites: LDL, HDL, triglyceride, serum urate, HDL cholesterol, LDL cholesterol, serum calcium, vitamin D, insulin-related traits
\item Proteins: tau (cerebrospinal fluid), adiponectin, glycated hemoglobin, immunoglobins, plasma level of liver enzymes
\item Ceullar properties: telemere length, erythrocyte phenotypes
\end{itemize}

How do we understand genetics of complex traits from many small effects? [Personal notes]
\begin{itemize}
	\item \textbf{Principle}: a complex trait may involve a set of interacting cell types, and disruption of normal functioning, maintenance of normal cellular states and differentiation, and cell interactions lead to phenotypic changes. 
	
	\item Example: known risk genes of psychiatric diseases can affect: 
	\begin{itemize}
		\item Cellular functions: SCN2A and SHANK, synaptic functions. 
		\item Cellular states: mTOR pathway affects cell proliferation (and brain size). Many regulatory genes may change cell states: FMRP, MeCP2, CHD8. 
		\item Cell-cell interactions: a complement gene changes how immune cells prune neurons (synaptic pruning). 
	\end{itemize}
	
	\item \textbf{Principle}: each of these steps, especially changes of cellular states (e.g. differentiation or proliferation), involves coordinated actions of many genes. Major genes and pathways exist to control these processes. These may be why complex diseases genes are regulatory genes. 
	
	\item Implications on genetic architecture: many cell types may be involved, and for each cell type, major genes regulating cellular functions, states and interactions (key processes) probably have large effects. Other genes with smaller effects: 
	\begin{itemize}
		\item The major genes/pathways are probably influenced by many other genes. 
		
		\item Specific genes in the key processes can affect disease risk. Ex. apoptosis: depends on regulatory genes, but also specific enzymes that act on apoptosis, e.g. enzymes involved in creating ROS.  
	\end{itemize}  
	
	\item Do risk genes of a trait converge on some genes? Possible, but not necessarily true. Ex. SCN2A is important for synaptic building, and may be regulated by multiple genes and pathways, so we can imagine that some risk genes converge to SCN2A level. On the other hand, it is possible that a trait may depend on the balance of two cell types (e.g. white vs brown fat cells), and a number of genes are involved in cell fate determination without a single key node.
	
	\item Implications on methodology: we will need to understand: (1) How cell functions, states and interactions are disrupted in diseases? (2) What are major genes and pathways regulate these processes? For (1), we can use pathway analysis of GWAS, comparison of transcriptome (ideally single-cell) between patients and controls. For (2), genetic networks (trans-eQTL), GRN via TFs or RBPs.     
\end{itemize}

Role of dynamics/stimulation-response CREs in human traits [personal notes]:
\begin{itemize}
	\item Ref: The impact of proinflammatory cytokines on the β-cell regulatory landscape provides insights into the genetics of type 1 diabetes [NG, 2019]. 
	
	\item Understand T1D genetics: Induced regulatory elements (IREs)  are those are those that change upon relevant stimulation (inflammatory cytokines), in contrast to stable regulatory elements (SREs). IREs are found to be enriched in T1D variants while SREs in T2D. Possible model: SREs are involved in maintaining beta cell identify, housekeeping functions, etc., while IREs are involved in response to inflammatory cytokines. Inappropriate IRE responses (e.g. by genetic variations) may lead to over-reaction to cytokines, e.g. apoptosis, which leads to T1D. 
	
	\item General model: the CREs of a cell have different functions, involved in different aspects of cellular functions: e.g. housekeeping, cellular differentiation and maturation, responses to cell function related stimuli. Variations of activities of different groups of CREs can lead to different phenotypes.  
	
	\item Neurons and psychiatric diseases: some CREs may be involved in neuron differentiation, some in neuronal signal response (neuron maturation). So variations in the former may lead to defects in neuro-development, e.g. not enough differentiated cells; while variations in the latter may lead to inappropriate response to electric stimuli. These may correspond to different phenotypes: e.g. autism vs. epilepsy.
\end{itemize}

\subsection{Genetic Stuides of Common Diseases} 

Breast cancer: [Human Molecular Genetics, Chapter 15]
\begin{itemize}
\item BRCA1 and BRCA2: identification in linkage studies on near-Mendelian families. 
\item BRCA1 account for 80-90\% of families with both breast and ovarian cancer, but a much smaller proprotion of families with breast cancer alone. Male breast cancer was seen mainly in BRCA2 families. 
\item Risk: in affected families, BRCA1 mutation had an 85-90\% chance of developing breast cancer; however the risk is much lower (36\%) in broad families. 
\end{itemize}

Alzheimer disease (AD): [Human Molecular Genetics, Chapter 15]
\begin{itemize}
\item Early onset: genes identified in near-Mendelian families - APP, presenilin-1/2. Mutations of these genes accout for 10\% of early onset AD. 
\item Late onset: different genes, ApoE (E4 variant) is a strong candidate, accounting for 50\% of suspectability of late-onset AD. ApoE is a class of apolipoprotein (lipid-binding protein), is essential for the normal catabolism of triglyceride-rich lipoprotein constituents. Apolipoprotein E enhances proteolytic break-down of this peptide, and E4 variant is inefficient at catalyzing these reactions. 
\end{itemize}

Type 1 diabetes (T1D): 
\begin{itemize}
	\item Disease: from autoimmune destruction of pancreatic $\beta$-cells, usually affecting young people. 
	\item Loci: HLA-DQB and INS (insulin gene, the mutations affect expression level of insulin). Together explain 50\% suspectability of T1D. 
\end{itemize}

Type 2 diabetes (T2D): 
\begin{itemize}
	\item Disease: combination of impaired insulin secretion and decrease end-organ responsiveness. Known risk factors: age, obesity. 
	\item Loci: calpain-10 (CAPN10), however, the evidence is limited. 
\end{itemize}

WTCCC studies [WTCCC, Nature, 2007]:
\begin{itemize}
\item Study design: 
\begin{itemize}
	\item 2,000 cases each for seven common diseases and 3,000 common controls (chosen from blood donors and from persons born in 1958) in British population. The cases are defined by clinical phenotypes, however, misclassification may happen in controls, as phenotypes of the controls are not collected, and some may develop diseases in the future. The misclassification rate is believed to be low. 153 individuals with non-Caucasian ancestry were excluded. 
	\item Affymetrix chip with about 500k SNPs, out of which 400k SNPs have MAFs $>1\%$ (MAF: minor allele frequency). 
\end{itemize}

\item Comparison of groups for population structure: the general issue is to compare two groups using SNP frequencies. For each SNP, test the difference of its allele frequencies in two group using some form of $\chi^2$-test (thus one test statistic for each SNP). Then the spectrum of the statistic can be compared with the null distribution (assume there is no difference between the two groups) using a Quantile-quantile plot. 

\item Association test: 
\begin{itemize}
	\item Power assessment: for case and control groups, simulate SNP data using population genetic models. Note that the MAFs of the causal SNPs in the case group is estimated from the effect sizes (using Bayes theorem). In this study, estimated to be (only for common variants, MAF $>5\%$, much lower for rare variants): 43\% for alleles with relative risk of 1.3, and 80\% for a relative risk of 1.5, for a $P$-value threshold $5 \cdot 10^{-7}$. 
	\item Trend test and genotyp test for any SNP: association of genotypes (three values per SNP) and phenotypes (case or control). A $P$ value is computed for each SNP. 
	\item Bayesian analysis: let $M_0$ be the null model (no association), and $M_1$ be the model of true association (additive effect for two alleles) and $M_2$ be a model where three genotypes could have different effects. A SNP is assessed by the posterior odds ratio: 
\begin{equation}
\frac{P(M_1|D)}{P(M_0|D)} = \frac{P(M_1)}{P(M_0)} \frac{P(D|M_1)}{P(D|M_0)}	
\end{equation}
The prior odds ratio can incorporate prior knowledge such as the distribution of SNPs (e.g. higher for nonsynonymous SNPs). A simple estimate would be, e.g. 10 causal SNPs in a region of 1Mb, and the prior odds ratio would be $10^{-5}$. The Bayes factor is computed from the logistic regression: the log-odds is equal to $\mu$ for $M_0$, and $\mu + \gamma Z_i$ for $M_1$, where $Z_i$ is the genotype of the $i$-th individual (0,1 or 2). The coefficient $\gamma$ is the increase in log-odds of disease for every copy of the allele encoded as 1. The Bayes factor integrates over parameters $\theta$ under each model. For both models, a prior of $N(0, 0.2)$ is used for $\gamma$, and a prior of $N(0,1)$ is used for $\mu$. 
	\item Significance threshold: use posterior odds ratio to select a $P$-value threshold, instead of multiple hypothesis correction (which is used for a single ``global'' hypothesis). The rationale is: whether a SNP is causal to the phenotype is one hypothesis to be tested. In the posterior odds ratio equation, the average $P(D|M_1)$ is the power of the study (given the hypothesis is true, $M_1$, what is the probability that we can detect it), assuming to be 0.5; and the average $P(D|M_0)$ is the $P$ value threshold. Setting posterior odds ratio at $10:1$ leads to the $P$-value threshold of $5 \cdot 10^{-7}$. 
\end{itemize}

\item Imputation: use HapMap data of SNPs (about six times of the SNPs in this study) to impute the untyped SNPs with a HMM. Verification of the imputation shows 98.4\% accuracy. Each imputed SNP will also be tested for association (with a somewhat stronger criteria). 

\item Results summary:
\begin{itemize}
\item Population structure: 153 individuals were excluded by comparing their SNP data with those from HapMap. The SNP data from different geographic regions were compared, and 13 regions were shown strong geographical variations. Some of them are probably due to natural selection, e.g. lactase, and for infectious diseases. 

\item Summary of findings: at $P = 5 \cdot 10^{-7}$, 1 in bipolar disorder (BD), 1 in coronrary artery disease (CAD), 9 in Crohn's disease (CD), 3 in rheumatoid arthritis (RA), 7 in type 1 diabetes (T1D) and 3 in type 2 diabetes (T2D). Also 58 loci with slightly less significant $P$ values. 12 of the 25 strong signals represent known findings, and most of the rest are confirmed by replication studies. 
\end{itemize}

\item Individual diseases: 
\begin{itemize}
	\item BD: PALB2 - stability of nuclear structures including chromatin; NDUFAB1 - mitochondrial respiratory chain; DCTN5 - intraceullar transport known to interact with DISC1 (known to be involved in BD). 
	\item CAD: CDKN2A/B - CDK inhibitors; CARD15 - caspase recruitment domain family member 15; IL23R; ATG16L1 - autophagy-related; MST1 - macrophage stimulating 1.  
	\item T2D: PPARG - peroxisomal proliferative activated receptor gamma; KCNJ11 - beta-cell KATP channel; TCF7L2 - transcription factor 7-like 2. 
\end{itemize}

\item Discussion: 
\begin{itemize}
	\item The effects of the population structure in this study is small. 
	\item A common set of controls: not necessary to have controls that map the socio-demogrpahic variables of every disease case group. 
	\item The effects of most SNPs are small. 
\end{itemize}
\end{itemize}

Large-scale whole-genome sequencing of the Icelandic population, [Gudbjartsson \& Stefansson, NG, 2015]
\begin{itemize}
\item Motivation: infer from WGS data the \textit{pattern of selection} in human genome (what might influence the strength of selection) and \textit{association analysis}. 

\item Data: 2,636 Icelander WGS with median depth of 20x. Found about 20M SNPs and 1.5M indels (about 7\%). Among indels, about twice are deletions, likely due to the challenge of calling insertions. 

\item Measuring selection: fraction of rare variants (FRV) at DAF $< 0.5\%$, and variant density. Higher FRV and lower density suggests negative selection. Note that these measures are affected by coverage, so limit the analysis only to regions with sufficient coverage (at least 15x). 

\item Pattern of genetic variants and selection: 
\begin{itemize}
	\item A definite of indels of length 3.
	
	\item LoF variants: 149 per individual, only 1.4 were seen in 1 or 2 out of all samples. Only 1 in 12 individuals have homozygous LoF variant with MAF $< 2\%$. 
	
	\item Selection at different functional classes of variants: higher selection at LoF, then missense mutations, then synonyous and UTR, then intronic and intergeic.  
	
	\item Selection at OMIM genes: strongest negative selection, in particular on variants that act through a dominant mode of inheritance. 
	
	\item Correlation of intra-species selection with mammalian conservation (GERP): when GERP score is positive (purifying selection), positive correlation with FRV, and negative correlation with density, as expected.  
	
	\item Selection at GO categories: high-density GOs related to communication of cells with environments, e.g. sensotry, defense. Low-density GOs: basic cellular processes. 
	
	\item Selection in different non-coding elements: higher selection in active promoters, strong enhancers, then weak enhancers, insulators, then heterochromatin.  
	
	\item \textit{Ultra-sensitive regions: high FRV and high density}, unexpectedly.  
\end{itemize}

\item Imputation: accuracy is improved by long-range phasing of 104K Icelanders. 

\item Association: Test additive model (16M) and recessive model (10M) with regression model, Correction by Bonferroni (27M tests). One result: MYL4 (myosin light chain) and early-onset atrial fibrillation. 

\item Remark/lessons: 
\begin{itemize}
	\item A key problem is to infer and understand pattern of selection in the genome. Simple measures: FRV and density. The methodological challenge is the local variation of mutation rates (thus exepected metrics of selection). \textit{What if we use SFS and corresponding tests? Are they robust to mutation rate difference}? 
	
	\item Imputation at isolated population: possible, and can greatly increase the power of studies. 
\end{itemize}
\end{itemize}     

\subsection{De Novo Mutations}

De novo mutations in human genetic disease [Veltman \& Brunner, NRG, 2012]: 
\begin{itemize}
\item Basics of de novo mutations: 
\begin{itemize}
\item Rate of de novo mutations per genome: CNV - 0.03, indel - 3, SNV - 74. Per exome rate is about 1 per person. 
\item Origin of de novo muations: an open question is whether these mutations occur mainly in the germline, during embryogenesis or somatically.
\item De novo mutations are more deleterious, on average, than inherited variation because they have been subjected to less stringent evolutionary selection. 
\end{itemize}

\item De novo mutations and their contribution to genetic diseases: 
\begin{itemize} 
\item Contribution of de novo mutations: higher for neurodevelopmental diseases, e.g. ID and autism, because (1) large mutational target; (2) higher selection of affected individuals, thus inherited variants play a smaller role.  Therefore, de novo mutations, although individually rare, may capture a significant part of the heritability for complex genetic diseases. 
\item Recessive inherited alleles are unlikely to explain most cases of these diseases, as the empirical sibling recurrence is much less than 25\% for intellectual disability, autism and schizophrenia. 
\end{itemize}

\item De novo mutations in rare sporadic genetic disease: 
\begin{itemize}
\item De novo CNVs: A well-known example is Down syndrome, which is caused by a de novo trisomy of chromosome 21. Recurrent de novo microdeletions and microduplications are now recognized as a common cause of clinically defined malformation syndromes. 
\item De novo SNVs: application in Mendelian diseases. The first example: de novo SNV in SETBP1 in Schinzel-Giedion syndrome. 
\end{itemize}

\item De novo CNVs in common genetic diseases: De novo CNVs larger than 100 kb are infrequent in the normal population, occurring in approximately one in 50 individuals.  By contrast, these large de novo CNVs occur in approximately 10\% of all patients with sporadic ID, ASD and schizophrenia. 

\item De novo SNVs in cmmone genetic diseases: 
\begin{itemize}
\item Sporadic ID: nine non-synonymous de novo SNVs were validated in seven out of the ten individuals. In two patients, de novo nonsense mutations were found in known ID-associated genes: RAB39B and SYNGAP1. Four of the remaining mutations are likely to be detrimental to protein function, and they affect plausible candidate genes. 

\item ASD: Note that rate between 0.63 and 1.00 per unaffected sibling: Different exome enrichment assays, sequencing methods and data-filtering steps. 
\end{itemize}

\item Challenges of interpreting de novo mutations: the detection of de novo mutations is no longer the limiting factor, the next pressing question is how to interpret any given de novo change in the context of a patient's phenotype. 
\begin{itemize}
\item Recurrently mutated genes: only 18 genes were found to be mutated de novo multiple times, a number that is not significantly different from simulated control data [Neale, 2012]. 
\item Adding functional information of genes to facilitate the interpretation of de novo mutations: gene group test, network methods, etc. 
\item Mutant types: however, Neale et al did not observe a difference in the PolyPhen-2 classification of 101 non-synonymous de novo mutations identified in patients with an ASD as compared to random simulations. 
\end{itemize}

\item Future directions: 
\begin{itemize}
\item Development of algorithms that are targeted to the analysis of de novo mutations in the context of exome studies, possibly incorporating several of the elements outlined in Fig. 2 (gene function, impact of mutations on protein function, correlation with phenotypes). 
\item Prenatal screening: the interpretation of these rare de novo events will be extremely challenging in a prenatal setting, especially because many of these mutations have variable penetrance and no phenotype information is available to guide interpretation.
\end{itemize}
\end{itemize}

Incorporating Functional Information in Tests of Excess De Novo Mutational Load (fitDNM) [Jiang \& Allen, AJHG, 2015]
\begin{itemize}
\item Model: the number of de novo mutations at each locus follow multinomial distribution (of sample size $n_l$ for locus $l$), with rate $\lambda_{lk}$ for mutational type $k$ (three mutations). To determine the rate, $P(X_l = k | A = 1)$, where $X_l$ is the mutation at $l$ and $A$ is the disease status, use Bayes theorem to relate it to $P(A=1|X_l=k)$. Suppose each mutation is associated with an indicator $D$, whether it disrupts the protein function, and the probability of disruption is $\rho_{lk}$. We have:
\begin{equation}
\lambda_{lk} \approx [1 + (\gamma - 1) \rho_{lk}] \pi_{lk}
\end{equation}
where $\pi_{lk}$ is the mutation rate and $\gamma$ the RR given that $D = 1$. The model assumes that $\rho_{lk}$ are known, given by PolyPhen for missense mutations and 1 for LoF.  

\item Test: likelihood for all the sites, then the score test. 

\item Simulation studies: randomly sample $\pi_{lk}$ and $\rho_{lk}$ for all positions in three chosen genes, then sample mutations, and sample the disease status, which depends on the total number of disruping mutations in a gene (summing over $D$'s for all positions) through a logistic regression model. The parameters of logistic regression are chosen to mimic full penetrance (0.8 to 0.95 if a gene has one LoF). 

\item Simulation results: fitDNM has twice of power of TADA-denovo. In an alternative scenario (similar to TADA): only LoF and highly probably missense mutations are causal, the power is similar to TADA. 

\item Results: similar to TADA, effectively one new gene when combining all four diseases (TRIO - 5 de novo mis3). 

\item Remark: the assumption that $\rho_{lk}$ is known and given by PPH2 is highly problematic. In fact, most PPH2 mis3 mutations have probabilities close to 1 ($>0.95$), but mis3 mutations are nowhere close to LoF mutations in terms of damaging effects. So the method places a much larger emphasis on missense mutations, which is not supported by data. In fact, even though the simulation finds that it doubles the power, in practice, it gave essentially identical results to TADA. 

\end{itemize}

\subsection{Somatic Mutations}

Somatic mutation in single human neurons tracks developmental and transcriptional history [Lodato and Walsh, Science, 2015]
\begin{itemize}
	\item Background: Ultra-deep sequencing for identifying somatic mutations. For most sites, germline genotypes are homozygous. Somatic mutations are present in a certain fraction of cells, detectable in sequencing reads. Problems of ultra-deep sequencing: (1) power is limited, if somatic mutations present in a small fraction of cells; (2) cannot know which mutations occur together. 

	\item Fluroescence activated nuclear sorting (FANS), followed by WGS (40x) in 3 inividuals. Total of 36 neurons. 
	
	\item Mutation rate and pattern: about 1,500 SNVs per neuron. Not correlate with replication timing, but show signatures of TCR (strand bias and expression). For neurons: replication-dependent mutations are less important. 
	
	\item Polyclonal derivation of brain: 5 major clades. A cell is more related to cardiomycytes than 75\% of its neighbors. 
\end{itemize}

Somatic mutations in diseases and in development [Chris Walsh, HG seminar, 2017]
\begin{itemize}
	\item Part I. Single gene mutations: change brain structure. Ex. microcephany (ASPM, COH1).
	
	\item Hemimegalencephaly (HME): enlargement of half of the cerbral cortex, caused by somatic mutations. Patient: removing half brain to remove seisure. Still largely normal. Mechanism: AKT3 mosaicism is responsible, a small percent of heterozygotes. ATK3: ser/thr kinase in mTOR pathway.
	
	\item A variety of mosaic overgrowth syndromes: mTOR gain-of-function mutations. Timing of mTOR mutation determines malformation type.
	
	\item Additionally, 5-15\% of DNMs are somatic mosaic mutations.
	
	\item Part II. Single cell sequencing of neurons: nuclear sorting.
	
	\item Retrotransposons copy and paste via an RNA intermediate.
	
	\item Average $<$1 somatic LINE jump per neuronal genome. Most have no LINE.
	
	\item Somatic SNVs are extremely common in single neurons. About 1400 somatic SNV per neuronal genome, or 15-30 per exome. Most SNVs are found in just one neuron.
	
	\item Shared mutations can be used to order cells into lineage. BLocks/nested mutations of blocks. Found 4 major clades.
	
	\item Clonal mixing: common in human brain (not in mouse). Some neurons are more related to heart than neighbors.
	
	\item Single neuron SNVs: show signatures of transcriptional damage (ss DNA damage during transcription).
	
	\item DNA damage: increase somatic mutations, role in ageing.
	
	\item ASPM: microcephaly. ASPM dN/dS $>$ 1 in human.
	
	\item Part III. Excess of recessive mutations in HARs (about 3000) in affected vs. unaffected.
	
	\item Mutations in a distant Cux1 enhancer, in a HAR. Cux1 is dosage-sensitive gene: modify neuron/synapse density.
	
	\item 2-3 SNVs per cell division. Higher rate of microsatelites. Mutation pattern is different in cancer: not enrichment of late-replicating genome; and opposite to TCR.
	
	\item Q: Sign of positive selection in somatic mutations?
	
	\item Q: Clonal mixing: same pattern in different individuals? Why? Functional advantage? Patterns consistent with known neuronal migrations? Can we use this to infer neuronal migrations?
	
	\item Q: estimation of somatic mutation rates? Number of cell divisions from fertilized eggs to neurons? Consistent across tissues? Ex. fast replicating cells have higher mutation rates?
\end{itemize}

\subsection{Mendelian Diseases} 

FSS [Ng \& Shendure, Nature, 2009]: 
\begin{itemize}
\item Data: exome sequences of 8 HapMap individuals and 4 unrelated FSS patients (FSS: dominant Mendelian disease). 78\% of genes have $>95$\% coding bases covered. Sequencing is done by hybridization capture (with short-gun libraries).

\item Pattern of genetic variations: 
\begin{itemize}
	\item SNPs: on average, 17,272 cSNPs were called per individual, with 92\% in dbSNP. Comparing with the reference human genome, each individual has about 10K (Yoruba) and 8,489 (non-Africans) cSNPs. 
	\item Indels: on average 166 coding indels per individual. 
\end{itemize}
	
\item Causal gene of FSS: two criteria for the test gene (1) at least one nonsynonymous cSNP, splice-site disruption or coding indel is observed in the gene; (2) the mutations are not in dbSNP, nor in the eight HapMap exomes. Only MYH3 meets the two criteria (the known causal gene). 

\item Remark: 
\begin{itemize}
	\item SNP pattern: most SNPs were accumulated during human evolution, which took about 3 million years. In one generation, the number of cSNPs per genome is about 0.1. This gives the number of SNPs per individual: $(3 \cdot 10^6 / 30) \cdot 0.1 = 10^4$, where 30 is the human life span. Also note that: the SNP distribution should be continuous (i.e. a continuous spectrum from rare to common) because the SNPs were accumulated over time, and should be biased towards less frequent SNPs because human popuation grows faster in recent times. 
	\item Causal gene identification: genetic heterogeneity is taken into accout here (allowing different mutations in 4 unrelated individuals). 
\end{itemize}
\end{itemize}
%%%%%%%%%%%%%%%%%%%%%%%%%%%%%%%%%%%%%%%%%%%%%%%%%%%%%%%%%%%%
\subsection{Mitochondria}

Genetics: Mitochondrial DNA in evolution and disease [NG, 2016]
\begin{itemize}
	\item Two mouse strain differ in mtDNA: transfer mtDNA from one strain to the nuclear DNA background of another strain. The two strains have profound difference in disease, longevity, T2D, etc. 
	
	\item Consequence of mtDNA variation: (1) Bioenergentic role of mt: permit accommodation to new diets or adjustment to thermal stress and activity demands. (2) Affecting Nuclear gene expression: modulating the levels of high-energy molecules (ATP, alpha-KG, etc.) generated through mitochondrial metabolism, which drive the modification of cytoplasmic signalling proteins and also add molecular modifications to nuclear proteins.
	
	\item Normal mtDNAs can be present in the cell in different proportions, a state known as heteroplasmy. Ex. 3243G mutation present in low proportion, T2D or autism; at 50–90\%, neurological, heart and muscle problems; 100\% childhood disease or death. 
\end{itemize}
%%%%%%%%%%%%%%%%%%%%%%%%%%%%%%%%%%%%%%%%%%%%%%%%%%%%%%%%%%%%
\section{Genetic Architecture of Complex Traits}

Why are the alleles that increase the risk of diseaes not eliminated by natural selection? [Nesse \& Williams, Why we get sick]
\begin{itemize}
\item Mutation-selection: high mutation rate (e.g. in large proteins), or recent mutations. 
\item Balacing selection: beneficial in some special circumstances, e.g. G6PD deficiency in areas with maleria (deficiency in glucose metabolism, protect against malarai parasite: kill the cell when the parasites use oxygen in red blood cells). A special form of balacing selection is heterozygote advantage: e.g. sickel-cell disease gene - heterozygotes protect against malaria. 
\item Late-onset diseases: the diseases occur only later in life, thus very weak selection, e.g. Alzheimer disease. 
\item Changing environment: genes that are normal or beneficial in ancestral environment may cause disease in modern environment, e.g. myopia. This involves gene-environment interaction. 
\item Selfish genes: e.g. meiotic drive (genes compete to enter a sperm or egg, even at the cost of the carriers).  
\end{itemize}

Synthetic associations [Dickson \& Goldstein, PLoS Biol, 2010]:
\begin{itemize}
\item Hypothesis: rare variants may create significant associations in common variant SNPs in GWAs (called synthetic associations). 

\item Background: the vast majorty of GWAS associations have never been tracked to causal sites, even though many surrounding regions have been resequenced. 

\item Intuition: 
\begin{itemize}
	\item In the genealogy, any varirant ``higher up in the genealoty'' that partitions the parts of genealogy containing more causal disease variants will be identified as disease-associated (Figure 1). 
	\item Alternatively, if at time of occurrence of the disease allele at the causal site, the disease allele is associated with some varirant of one common SNP, and no/few recombinations occur afterwords, then the common SNP will appear disease-associated. 
\end{itemize}

\item Methods: simulation of genealogy of about 2000 cases and controls, the baseline risk is $(0.01,01)$ and the genotypic relative risk (GRR) at 2 to 6. 

\item Significant associations caused by rare variants: in 30\% of cases, a significant association with a common SNP was detected ($P < 10^{-8}$). The association with the causual variants is almost always stronger than with the synthetic associations. Also, sigifnicant synthetic associations depend on the associtions that occur within a single gene genealoty (no recombinations). 

\item Effect of recombinations: increasing recombination rates may acutally increase synthetic associations. 

\item Distance between causal variants and significant SNPs: in simulation, the median distance between the causal and synthetic varirants is 5 Mb. Also in GWAS of sickle cell anemia, synthetic associations in about 2.5 Mb region. Thus, the true associations can travel across multiple LD blocks to create synthetic associations. 
\end{itemize}

The mystery of missing heritability: Genetic interactions create phantom heritability [Zuk \& Lander, PNAS, 2012; NIH Lecture, From the `Genetic Code' to the `Genetic Code']: 
\begin{itemize}
\item Explaining heritability: Let $h^2_{known}$ be the heritability explained by the known loci, and $h^2_{all}$ be the (true but unknown) heritability of the trait, then we have: 
\begin{equation}
\Pi_{explained} = \frac{h^2_{known}}{h^2_{all}}
\end{equation}
is the fraction explained by the known loci. The heritability estimated from data (e.g. from twin-studies) is often different from the true heritability, and is denoted as $h^2_{pop}$. When
\begin{equation}
h^2_{pop} > h^2_{all}	
\end{equation}
we overestimate $h^2_{all}$, thus $\Pi_{explained}$ is underestimated. 

\item Known heritability: suppose we have a linear model involving $p$ loci: 
\begin{equation}
Y = \sum_j X_j \beta_j + \epsilon
\end{equation}
Taking the variance, we have the variance explained by the $p$ loci is: 
\begin{equation}
h^2_{known} = \sum_j 2 p_j ( 1 - p_j) \beta_j^2 	
\end{equation}
where $p_j$ is the allele frequency of the $j$-th SNP. 

\item Overestimation of $h^2_{all}$ under ACE model: we have: 
\begin{equation}
h^2_{pop}(ACE) = 2 (r_{MZ} - r_{DZ})	
\end{equation}
When there are epistatic interactions, the covariance between relatives is bigger (see the section ``Covariance between relatives'' in Notes Biology-BG), and the effect is the largest with monozygotic twins, so $h^2_{pop}(ACE) > h^2_{all}$. 

\item Limiting pathway (LP) model: suppose the phenotype is a function of $k$ pathways: s.t. it is the maximum (or minimum) of $k$ pathway variables. Denote the model as $LP(k, h^2_{pathway}, c_R)$, where $h^2_{pathway}$ is the heritability of each of the $k$ pathway variables, and $c_R$ is the extend of shared environment between relatives. Our goal is to determine $h^2_{all}$ and $h^2_{pop}(ACE)$ under this model. 
\begin{itemize}
\item True heritability: let $Z_i$ be the $i$-th pathway variable (Gaussian) and $Z$ be the phenotype variable, then $Z = \max{Z_1, \cdots, Z_k}$. The heritability (assuming $V_P = 1$) is equal to the sum of variance explained by each pathway variable. The variance explained by the additive model of one pathway is equal to $h^2_{pathway}$, however, only part of this explains the variance of $Z$, and the fraction is $\Cov(Z_i, Z))$, which can be determined from maximum of $k$ Gaussian RVs. So we have: 
\begin{equation}
h^2_{all} = k \Cov(Z_i,Z) h^2_{pathway}	
\end{equation}

\item Apparent heritability: this involves computing $r_{MZ}$ and $r_{DZ}$ under the LP model. 
\end{itemize}

\item Case study: a good amount of heritability is explained for some traits such as T1D. Even higher when smaller-effect SNPs are included. 

\item Difficulty of epistasis mapping: need very large sample size to detect epistasis. 

\item Conclusion: 
\begin{itemize}
	\item The heritability estimation is based on additive models, which are obviously not true. Thus the true heritability is essentially unknown.
	\item What matters is the biology, not the heritability explained. 
\end{itemize}

\end{itemize}

Genetic interactions improve models of quantitative traits [Tyler \& Carter, NG, 2017]
\begin{itemize}
	\item Motivation: many model organism studies support the prevalence of epistasis, but they are not found to be important in human population studies, why? 
	
	\item Yeast growth QTL study: many strains, growth phenotype. Show that the prediction of growth rate is greatly improved by epistasis. In particular, find a number of \textbf{genetic capacitors}, defined as key genetic loci that each masked the effects of many other loci. Depending on its genotype, the capacitor either locks the phenotype near the population mean or permits its interaction partners to influence the phenotype. 
\end{itemize}

An Expanded View of Complex Traits: From Polygenic to Omnigenic, [Boyle \& Pritchard, Cell, 2017]
\begin{itemize}
	\item Distribution of GWAS signals across the genome: using height as an example, 3.8\% of causal SNPs (if allowing LD, 60\% of SNPs) based on ASH. This mean about 100K causal SNPs across the genome. 
	
	\item Weak enrichment of GWAS signals in functional gene groups: e.g. 5-10 enrichment of immune genes in Crohn's disease and RA (still not explain the majority of signal). In contrast, in rare variant studies, genes tend to be functionally connected, e.g. ASD and SCZ. 
	
	\item Omnigenic model: a small set of core genes. The cellular networks are highly interconnected (small-world hypothesis) s.t. regulatory variants in peripheral can affect the activity of core genes. Implications: 
	\begin{itemize}
		\item Even though peripheral genes have smaller effects, they outnumber core genes, so together explain most of heritability. 
		\item The cis-QTL are trans-acting QTL on core genes. 
	\end{itemize}
	
	\item Pleiotropy: if the omnigenic model is correct, we expect widespread pleiotropy, as a regulatory QTL can create small effects in different core genes at different traits (that share the same cell type). 
	
	\item Evolutionary changes of complex traits: most adaptive changes within a species are polgenic adaptation. Propose that it is also true in cross-species comparison. Problem of how pleiotropy may share selection and adaptation. 
	
	\item Remarks: 
	\begin{itemize}
		\item Small world model: does not support the omnigeic model, buffering, compensatory changes. 
		\item Alternative explanation of the data: many genes could affect cellular states: how fast they divide, how efficient they deal with stress, how quickly it responds to some stimuli, etc. These genes do not directly act on core genes (disease-specific processes). 
		\item Importance of trans-acting QTL: supported by Fehrmann, NG, 2012 and Sherlock. 
		\item Evolutionary implications: Different pictures in model organisms and in cross-species comparison. The impact of population size: how efficient deleterious variants are removed.   
	\end{itemize} 
\end{itemize}
%%%%%%%%%%%%%%%%%%%%%%%%%%%%%%%%%%%%%%%%%%%%%%%%%%%%%%%%%%%%
\section{Metabolism and Metabolic Diseases}

Review: genetic loci of plasma lipoproteins: [Hegele, NRG, 2009; Teslovich, Nature, 2010]
\begin{itemize}
\item Chylomicron and VLDL function: APOB; APOE - removing chylomicron and VLDL remnants through interaction with its receptor; LIPC - hepatic lipase, receptor-mediated lipoprotein uptake

\item LDL function: LDL receptor (LDLR); LDL receptor accessory protein 1 (LDLRAP1) - a chaperone through the early phase of endocytosis; PCSK9 - convertase involved in intracellular receptor degradation; LRP1 and LRP4, members of the LDL receptor-related protein family

\item HDL function: APOA1 - main protein component in HDL; APOA5 - component in HDL, important in in regulating the plasma triglyceride levels; ABCA1 - cholesteral efflux pump in peripheral cells, transporting cholesteral to HDL particles; LCAT - esterification of cholesterol; cholesteryl ester transfer protein (CETP) - collecting triglycerides from VLDL and LDL to exchange for cholesterol in HDL; SCARB1, a HDL receptor that mediates selective uptake of cholesteryl ester; PLTP - phospholipid transfer protein, transferring phospholipids from triglyceride-rich lipoproteins to HDL. 

\item Cholesterol metabolism: HMGCR - enzyme in the rate-limiting step of cholesterol synthesis; ABCG5, ABCG8 and NPC1L1 - sterol absorption in intestine; CYP7A1, cholesterol 7-alpha-hydroxylase; STARD3, a cholesterol transport gene.

\item TG metabolism: lipoprotein lipase (LPL); APOC2 - cofactor of LPL, endothelia lipase (LIPG); LPA, lipoprotein(a);  ANGPTL3 and ANGPTL4 - inhibit endothelial lipase; LMF1 - maturation and transport of lipoprotein lipase through the secretory pathway; PNPLA2 and ABHD5 (cofactor of PNPLA2) - hydrolysis of triglycerides in adipose tissue; LPIN1 - direct lipid to adipose storage sites; PLIN1 - protecting lipid droplets until they can be broken down by hormone-sensitive lipase in adipose tissue. 

\item Carbohydrate metabolism: (potentially related through the interactions of carbohydrate and lipid metabolism, e.g. precursors of lipids are from glycolysis and pyruvate) GALNT2 - function in the first step of O-linked oligosaccharide biosynthesis; PPP1R3B - may be involved in regulating glycogen synthesis in liver and skeleton muscle

\item Misc: TTC39B - encoding tetratricopeptide repeat domain 39B
\end{itemize}

Plasma lipoproteins: genetic influences and clinical implications [Hegele, NRG, 2009]
\begin{itemize}
\item Earlier studies with small case-control or cohort-based association studies and linkage studies: quantitative lipoprotein traits (normal variation) with candidate genes or genome-wide marker sets. Convincing data are lacking for significant metabolic roles for USF1, WWOX or numerous other genes found this way, and most are not replicated in later GWAS. 

\item Monogenic disease: use phenotypes of extreme levels of TG or cholesterol (hyperlipoproteinaemia, HLP). Some examples: 
\begin{itemize}
\item HLP type 2A: very high level (95 percentile) of LDL-C. 10\% of these subjects have a discrete monogenic syndrome with mutations in LDLR, LDLRAP1, APOB or PCSK9.  
\item HDL cholesterol level below the fifth percentile: have extremely rare monogenic disorders, some due to mutation in ABCA1, ApoA1 or LCAT
\item HLP type 1: plasma TG levels above the ninety-fifth percentile have rare monogenic disorders with mutations in lipoprotein lipase (LPL), APOC2 and APOA5 genes. 
\item Lesson: For instance, plasma LDL cholesterol levels depend crucially on LDL receptor function, which in turn requires proper binding of apolipoprotein B, the presence of LDL receptor accessory protein 1 (LDLRAP1) as a chaperone through the early phase of endocytosis, and regulated intracellular receptor degradation by the convertase PCSK9.
\end{itemize}

\item GWAS: 
\begin{itemize}
\item LDL cholesterol: approximately half of the associated genes had been identified previously, for example APOE, LDL receptor (LDLR), APOB, PCSK9 and HMGCR. Novel loci inclulde those containing sortilin 1 (SORT1), cartilage intermediate layer protein 2 (CILP2), basal cell adhesion molecule (BCAM) or the translocase gene TOMM40. 

\item HDL cholesterol: approximately three-quarters of significant SNPs were in loci harbouring known genes such as CETP, LIPC, LPL, ABCA1, endothelial lipase (LIPG) and LCAT. Only GALNT2 and the MVK-MMAB locus had no previous connection to HDL metabolism

\item TG: approximately one-third of genes in significantly associated loci were known, such as APOA5, LPL, LIPC, APOB and ANGPTL3, whereas loci harbouring CILP2, TRIB1, GCKR, CHREBP (also called MLXIPL) and GALNT2 had minimal prior connection to TG metabolism.

\item Some genes in newly identified loci link TG with carbohydrate metabolism: (this is relevant as the most common lipid disturbance observed in diabetes is elevated plasma TG). GCKR21 encodes glucokinase regulatory protein; CHREBP63 encodes a glucose-responsive transcription factor that is active in hepatic glycolysis, lipogenesis and VLDL secretion; and GALNT2 encodes UDP-N-acetyl-alpha-D-galactosamine:polypeptide N-acetylgalactosaminyltransferase. 
\end{itemize}

\item Animal studies: 
\begin{itemize}
\item Lipase maturation factor 1 (LMF1) is essential for the processing and secretion of both LPL and hepatic lipase. 

\item Two genes encoding ATP-binding cassette proteins, ABCG5 and ABCG8: role in sterol excretion from intestinal cells. 

\item NPC1L1: the role in intestinal sterol transport and as the target for a new class of drugs that inhibit sterol absorption. 

\item Several adipocyte-based genes: Atgl (also known as Pnpla2), encoding adipose triglyceride lipase, which is involved in the intracellular lipolysis of TG77; Abhd5, the cofactor for ATGL; Lpin1, directs lipids to adipose storage sites79; and Plin, provides a scaffold that coordinates access of enzymes to lipid droplets in adipocytes. 
\end{itemize}

\item Future work: Mendelian randomization has been proposed as an approach to assess whether genetically determined intermediate traits, such as lipoprotein levels, are causally related to end points, such as CVD. Association of CVD risk with plasma lipoproteins can be diluted by non-genetic factors that alter plasma lipoproteins, whereas the association of CVD with the genetic determinants of lipoprotein levels is more direct and less susceptible to confouding effects. 
\end{itemize}

Biological, clinical and population relevance of 95 loci for blood lipids [Teslovich, Nature, 2010]: 
\begin{itemize}
\item Meta-analysis: 46 lipid GWASs, more than 100,000 individuals of European descent. A total of 2.6 million genotyped or imputed SNPs were tested for association with each of the four lipid traits: TG, HDL, LDL and total cholesterol (TC), in each study and the results were combined with a fixed-effects meta-analysis. 

\item GWAS loci: identified 95 loci that showed genome-wide significant association ($P< 5 \cdot 10^{-8}$) with at least one of the four traits. The total set of mapped variants (both lead and significant secondary SNPs) explains about 10\% of the total variance in each lipid trait in the Framingham Heart Study, corresponding to about 25\% of the genetic variance for each trait. 

\item eQTL: data of $>39,000$ transcripts in liver (960 samples), omental fat (741 samples) and subcutaneous fat (609 samples). Identified cis-eQTL (500 kb) at $P< 5 \cdot 10^{-8}$: among 95 loci, 38 SNP-to-gene eQTLs in liver, 28 in omental fat, and 19 in subcutaneous fat. 

\item Clinical significance: association testing of lead SNPs for CAD: A limited number of loci met $P < 0.001$, with most of them being associated with LDL-C. Four novel CAD-associated loci related specifically to HDL-C or TG, but not LDL-C: IRS1 (HDL-C, TG), C6orf106 (HDL-C), KLF14 (HDL-C) and NAT2 (TG). Not clear if they affect CAD risk through HDL or pleiotropic effect (e.g. IRS1 for insulin)

\item Functional evidence of some novel genes: GALNT2, PPP1R3B and TTC39B, overexpression or knockdown of these genes in mouse liver significantly change the plasmid lipid level. 
\end{itemize}

Non-coding variant associated with cholesterol through SORT1 [Musunuru \& Rader, Nature, 2010]:
\begin{itemize}
\item Background: 1p13 locus is strongly associated with both LDL and Myocardial infraction (MI) diease. 

\item 1p13 SNPs are associated with SORT1 and PSRC1 expression in liver eQTL data, but none of the SNPs show association in adipose tissues and in lymphocytes. 

\item A causal 1p13 variant: from the strongest association with LDL level, identified 6 SNPs in a region of 6.1 kb between CELSR2 and PSRC1. Resequencing of this region identified 16 SNPs. Then test the functional difference of two haplotypes of this noncoding sequence: in terms of luciferase expression, or SORT1 expression, in human Hep3B cells. One SNP rs12740374 has a causal influence (through enumerating all SNPs). Also, the minor alleles creates a new binding site for CEBPA (verified by in vitro binding experiment). 

\item SORT1 expression and LDL: in mouse, overexpression or small interfering RNA-knockdown of SORT1 in liver changes significantly the level of LDL. Note that: the overexpression (using special viral vectors/promoters) and knockdown were limited to liver. 

\item Remark: 
\begin{itemize}
	\item Key experiments to establish the mechanism of the 1p13 locus is: (1) the association with LDL and SORT1 expression; (2) fine-mapping of causal variants: need sequencing and functional assay of variants (TF binding, gene expression); (3) SORT1 and LDL in mouse. 
	\item Lesson about the noncoding variant: located quite far away from the target gene (two genes between the SNP and SORT1). 
\end{itemize}
\end{itemize}

Twelve type 2 diabetes susceptibility loci identified through large-scale association analysis (DIAGRAM+ study) [Voight, NG, 2010]
\begin{itemize}
\item Stage 1: 
\begin{itemize}
\item Data: genome-wide association data from 8,130 individuals with type 2 diabetes (T2D) and 38,987 controls of European descent, combining data from WTCCC, DGI, FUSION, deCODE genetics, the Diabetes Gene Discovery Group, the Cooperative Health Research in the Region of Augsburg group (KORAgen), and the Rotterdam study and the European Special Population Research Network (EUROSPAN).

\item Meta-analysis: 2,426,886 imputed and genotyped autosomal SNPs into a fixed-effects, additive-model meta-analysis using the inverse-variance method. 

\item Modest genomic control inflation: $\lambda = 1.07$. After removing SNPs within established T2D loci (Supplementary Table 3), the resulting quantile-quantile plot was consistent with a modest excess of disease associations of relatively small effect. 
\end{itemize}

\item Stage 2: in Stage 1, 23 new autosomal regions showing the most compelling evidence for association, all $P < 10^{-5}$. Replication test. 
\begin{itemize}
	\item 21 showed directional consistency of effect between stage 1 and 2, binomial test $P = 3.3E-5$. For 15, the stage 2 $P$ value was $<0.05$
	\item Total: 31 loci including 20 previous reported and 11 new loci.
\end{itemize}

\item eQTL analysis: cis-eQTL of blood and adipose tissues [Emilsson, Nature08]. several are cis-eQTLs, the stronest is one of KLF14. Also use conditional analysis to test cauaslity.

\item Pathway analysis:  
\begin{itemize}
	\item Gene list: from a list of loci, generate the genes using the nearest recombination hotspots. Results: 31 confirmed loci $\rightarrow$ 82 genes; 110 expanded loci (no HLA, all $P < 1E-4$ in Stage 1) $\rightarrow$ 320 genes. 
	\item GRAIL analysis: literature evidence of connection
	\item Pathway enrichment: PANTHER (2 general categories significant after Bonferroni correction); REACTOME (some significant, at $FDR < 0.2$): e.g. cell cycle, Notch signaling, FOXA transcription from the confirmed list.
	\item PPI networks: some known interactions
	\item MAGENTA: test enrichment of known T2D pathways. Method: assign each gene a P-value - the strongest associated SNP, corrected for confounders; then test significance using GSEA. Overall, we observed that gene sets related to cell cycle, inflammatory response, and fatty acid oxidation were nominally enriched for genes association.
\end{itemize}
\end{itemize}

A genome-wide perspective of genetic variation in human metabolism [Illig \& Suhre, NG, 2010]: 
\begin{itemize}
\item Data: concentration of 163 metabolites in blood of 1,809 individuals. And replication study in another 422 individuals. 

\item Analysis: use metaboites or metabolite ratios ($163 \cdot 162$) as traits, and do linear regression on every SNP assuming additive model (i.e. each copy of the minor allele makes a contribution to the trait). The threshold: $10^{-7}$ for metabolites and $10^{-9}$ for metabolite ratios.  

\item 15 loci were identified in 1,809 individuals, and 9 were replicated ($P < 0.05$ after Bonferroni correction for 15 tests). The associated genes often have matching functions: 3 genes in $\beta$-oxidation, one in generating energy from $\beta$-oxidation, three genes in fatty acid biosynthesis, 2 in AA metabolism, and two are transporters. 

\item Some loci (genes) are associated to clinical parameters in previous GWAS: e.g. FADS1 is associated with LDL and HDL level in previous GWAS; a SNP in APOA1-APOC3-APOA4-APOA5 cluster was associated with blood triglyceride levels in previous GWAS (with phosphatidylcholines in this study). 

\end{itemize}

Expression-based genome-wide association study links the receptor CD44 in adipose tissue with type 2 diabetes [Kodama \& Butte, PNAS, 2012]
\begin{itemize}
\item Idea: genes differentially expressed in repeated experiments (cases vs. controls) are likely to be causal genes. Denote this as eGWAS. 
\item Data: 130 independent microarray experiments, totaling 1,175 samples collected from public repositories. 
\item Test: ranked all 24,898 genes by the likelihood that repeated differential expression for that gene was due to chance. For each experiment, compute a $d$-score measuring diff. expression, then (1) count the number of experiments where the gene is DEX, and compute the significance, or (2) weighted-Z or other method of combining $p$-values. 
\item Results: 127 genes after Bonf. correction. Top gene: CD44, markedly differentially expressed in experiments studying diabetes in adipose tissue compared with other tissues.
\item Evidence of CD44:
\begin{itemize}
\item One of the known ligands for CD44, SPP1, is also a top gene. SPP1 may serve as a link between adipose tissue inflammation and insulin resistence. 
\item In mouse model, CD44 expression increases in obese adipose tissue
\item CD44 deficient mouse: less adipose tissue inflammation and insulin resistence. 
\item CD44 Blockade Decreases Blood Glucose Levels and Adipose Macrophage Infiltration
\end{itemize}

\item Remark: genes repeatedly show diff. expression may be causal genes. The idea is: there may be multiple symptoms of T2D, thus many disease reactive genes, but in different patients, the symptoms may be different (e.g. kidney, eyes, etc. for T2D), and the reactive genes tend to be different. The repeatedly DEX genes tend to lie in the common denominator of all T2D patients (core T2D development pathway). 
\end{itemize}

Integration of disease-specific single nucleotide polymorphisms, expression quantitative trait loci and coexpression networks reveal novel candidate genes for type 2 diabetes [Kang \& Butte, Diabetologia, 2012]
\begin{itemize}
\item T2D expression traits: (1) 32 replicated SNPs from T2D GWAS, and then (2) 21 of these 32 SNPs were associated with 62 different expression traits (eQTL data from liver and subcutaneous and omental adipose tissue) at the uncorrected threshold of $p=0.05$.
\item Weighted co-expression networks: from mouse expression data. A total of 2,326 tissue-specific coexpression modules. 
\item Scoring T2D expression traits: (1) Out of the 62 type 2 diabetes expression traits, 33 were present in one or more coexpression network modules, resulting in the implication of 526 type 2 diabetes coexpression network modules. (2) for each gene, its score is the number of modules it appears (weighted by the inverse of the module size). 
\item Evaluation of predictions: map the T2D expression traits to eSNPs using the eQTL data, then assess the p-value distribution of these eSNPs (enrichment of low p-values). 
\item Remark: 
\begin{itemize}
\item The downstream genes of T2D SNPs from eQTL data are good candidate genes. 
\item How to use gene expression data to score T2D candidates? (1) If some genes are known T2D gene, then the modules or other genes in the module are likely T2D-related. (2) Correlation of modules with T2D phenotype (if expression and phenotype data are available in the same subjects). 
\end{itemize}

\end{itemize}

Whole-genome sequence-based analysis of thyroid function [Taylor \& UK10K, Nature Comm, 2015]
\begin{itemize}
	\item Traits: circulating concentrations of free ​thyroxine (FT4) and the pituitary hormone thyrotropin (TSH). 
	\item Single point analysis: WGS of 2,287 samples, 8M SNPs. 
	\item eQTL analysis: using Genevar database and UK-Twins study. 
	\item Rare variant analysis: candidate genes from GWAS, then SKAT on the rare variants in 50K regions near a gene. 
	\item GCTA and polygenic score analysis: GCTA to estimate heritability. A genetic score based on 67 SNPs previously associated with thyroid function in GWAS shows strong evidence of association with TSH and FT4. Also evidence of shared genetic pathways with TSH associated with the FT4 gene score.
	\item Lessons: some standard analysis on WGS association data include single-point analysis, RV analysis of regions, eQTL/rQTL, heritability, shared heritability. 
\end{itemize}

The Genetic Architecture of type 2 diabetes [Nature, 2016]
\begin{itemize}
	\item WGS analysis (GoT2D): about 2,600 samples. Single variant analysis: found 4 loci, three in previous GWAS. Imputation: more findings: 14 loci. 
	
	\item WES analysis (T2DGenes): 
	\begin{itemize}
		\item Single variant: imputation on 28K cases and 50K controls, found 18 coding variants (Table 1). Most are common, in previous GWAS. Help resolve GWAS candidate genes. 
		\item Gene based test: define variant groups, LoF, LoF + NS (strict) and LoF + NS (broad). SKAT-O test, no overall signal, limit to candidate genes (694 mapped to known GWAS regions), 1 gene reaches significance.   
	\end{itemize}
	
	\item Rare variant burden in gene sets: Mendelian diabetes genes. Significant burden in LoF (OR = 1.8) and LoF + NS.strict (OR = 1.5). 
	
	\item Fine-mapping and annotations: enrichment in CDS, TFBS, enhancers in adipose and pancreatic islets. 
	
	\item Genetic architecture of T2D: 
	\begin{itemize}
		\item Estimation of LVE: for each SNP, estimate its LVE, and add up LVEs. Ext Figure 7: 6.3\% LVE from CVs, 2.9\% LVE from AF 0.1-5\%. 
		
		\item Simulation: under different scenarios, depending on how much $h^2$ is explained by CVs or RVs. The pattern (number of discovered loci at a particular $p$-value threshold) is consistent with polygenic model, RV explains 25\%. 
	\end{itemize} 
	
	\item Remark: The method for estimating LVE: accounting for LD? 
\end{itemize}

A functional genomics pipeline identifies extensive allelic heterogeneity and cross-tissue effects within obesity-associated GWAS loci [Joslin and Nobrega, submitted to NG, 2020]
\begin{itemize}
	\item Background: BMI genetics, 97 loci, enriched in brain, then adipose.
	
	\item Differentiation of pre-adipocytes to mature adipocytes, and iPSC into hypothalamic neurons: pcHiC, ATAC, RNA.
	
	\item PcHiC data: 600K-900K interactions per time point (30/gene), fragment size 422 bp median, and interaction: 100-200kb.
	
	\item Dynamics of RNA, OCR and interactions: (1) Gene: three main clusters, down (2000 genes), up (4000), and V shape (6000 genes). (2) HSV plots of temporal patterns. ATAC: mostly up. PcHiC: more show down-regulation; also seems to have a delay in activation.
	
	\item Mapping genes to OCRs: use pcHiC, 3-4 peaks per gene.
	
	\item MPRA: 97 loci, lead variants and LD (.8), a total of 2400 variants. 800 regions have enhancer activities in brain and 500 in adipose, with 400 shared. Validation of enhancer using Luciferase: 65\% validation rate (Figure S3c).
	
	\item Among all variants: 94 have enhancer-modulating activities (EMVars). 61 brain EMVars and 70 adipose EMVars and at least one was identified in 40/97 (41\%) of tested GWAS loci. 2/3 of loci contain more than 1 EMVars. 37/94 (39\%) of these variants affected enhancer activity in both cell types.
	
	\item Prioritization of candidate genes by assigning EMVars to targets using pcHiC: about half do not have interactions, but for those with interactions, often more than 1, with median about 3. Also use adipose and brain eQTL to assign targets. Define classes of 150-200 candidate genes in either adipose or neuron: some hi-C and eQTL in the right cell type (class I), some only hi-C and eQTL in a different type (class II), some either evidence (class IV).
	
	\item rs4776984: brain EMVar, it is adipose and brain eQTL, hi-C interaction with MAP2K5 in both adipose and neuron. Another SNP about 50kb away: EMVar, and hi-C with MAP2K5 and eQTL in both adipose and neuron.
	
	\item 16p11.2 locus: 600kb, spanning two independent GWAS loci, 10 EMVars. Also pcHiC interaction between the two loci. (1) Region 1: has 3 EMVras, but two are not eQTLs nor PC-HiC interaction. The third SNP at 3’ UTR of SBK1, EMVar in both adipose and brain, eQTL of 5 genes in adipose and 7 in brain, and pcHiC of 18 genes. (2) Region 2: 7 EMVars, 5 are in perfect LD. All are eQTLs of 9 genes in adipose and 5 in brain, and each contacts several promoters.
	
	\item CRISPR deletion of enhancers containing rs2650492 (from SBK1 region) and rs9972768 (from the second region): in iPSC, then differentiation into hypothalamic neurons. (1) SBK1 expression is reduced in one stage for both deletions. SBK1 homolog in zebrafish and mouse: neurodevelopment defects. (2) Another nearby gene with pcHiC support: NUPR1 showing DE in HEK293 cells.
	
	\item Lesson: functional variants often have pleiotropic effects in Hi-C and eQTLs, and can have effects across tissues. Additionally, in any single locus, there could be multiple functional variants.
	
	\item Remark: (1) Limitation of MPRA: in natural context, may have fewer EMVars and be more tissue-specific. (2) Does HT neurons capture brain enrichment? Could redo brain enhancer enrichment, but removing/conditioning HT neurons. (3) While there are multiple EMVars, it only proves that regulatory variants are relatively common, but it does not prove that there are multiple disease-causing variants. 
	
\end{itemize}

%%%%%%%%%%%%%%%%%%%%%%%%%%%%%%%%%%%%%%%%%%%%%%%%%%%%%%%%%%%%
\section{Immune-Related Traits}

Genetics of immune related disorders [personal notes]: 
\begin{itemize}
	\item Role of infection in AIDs: possible models support the association of infection with AIDs: (1) Antigen mimicry: some self-antigens are similar to antigens from pathogens. (2) Pre-existing responses against self-antigens: exacerbated by infections, which act as adjuvants. (3) Inappropriate/excessive immune response against infections lead to AIDs. 
	
	\item Sero-positive vs. sero-negative AIDs: for some diseases, it is important to have self-antigens (first two models), and in other diseases (model 3), not necessary. 
	
	\item Difference of AIDs vs. allergic diseases: because infection is an important part of AIDs, so these AIDs involve genes in innate immunity against infections (e.g. autophage). Allergy: reaction against other foreign antigens that are not from infections. 
	
	\item Question: Why in developed countries, the decrease of bacterial infection is not associated with a decrease of AID? 
	
	\item Question: Organ transplantation: rejection of foreign MHC, how CD8 T cells are activated? The TCRs of CD8 T cells do not match the foreign MHC I. 
\end{itemize} 

Overview of autoimmune diseases: [Marrack, Nature Medicine, 2001], [Rious, Nature, 2005], [Zhernakova \& Wijmenga, NRG, 2009]
\begin{itemize}
\item Autoimmune diseases occur in up to 3-5\% of the general population. 

\item Target tissues: There is an autoimmune disease specific for nearly every organ in the body, e.g. T1D (beta pancreatic cells); MS (brain/spinal cord). In other autoimmune diseases, such as systemic lupus erythematosus (SLE), RA, no particular cell type seems to be targeted. 

\item Antigen specificity: AIDs are antigen-specific. For AIDs that target many cell types, the antigens are probably expressed throughout the body. However, recognition of widely expressed antigens sometimes results unexpectedly in organ selective manifestations, e.g. in RA, antibodies against a widely expressed antigen (IgG, fibrin) can target destruction of the joints selectively. 

\item Evidence of environmental impact: Bacterial LPS and mycobacteria can induce various autoimmune diseases, sometimes in the absence of any additional antigen besides that provided by the host itself. Numerous anecdotal reports describe association between the onset of various autoimmune diseases and infections. The environment can also affect the immunoreactivity of the individual by shifting the balance of T cells within the individual between Th1 and Th2 cells. 

\item Antigen (molecular) mimicry or cross-reactivity: If the agent codes for a peptide that is closely related to a peptide of the host, a vigorous responses might powerfully induce T cells. Ex. (1) In RA, host Hsp60 in joint is similar to Mycobacteria/Hsp65; (2) T1D: host Pancreatic beta cells/GAD is similar to Coxsackie B/P2-C; (3) MS: host Brain/myelin basic protein is similar to Papillomavirus/L2. 

\item Epidemiological difficulties: in developed countries, bacterial infections have dropped in frequency, and asthma disease thought to be driven by Th2 cells has increased without a concomitant decrease in the incidence of autoimmune diseases. 
\end{itemize}

Model of AID genetics: AIDs develop when self-reactive lymphocytes escape from tolerance and are activated. This is believed to result from a combination of genetic variants, acquired environmental triggers such as infections, and stochastic events [ibid]: 
\begin{itemize}
\item Basic model: some genetic polymorphism changes part of the immune system, e.g. antigen over-expression or negative feedback of TCR signaling, and these changes make the immune system more likely to be activated by self-antigens. With certain environmental trigger (providing adjuvants), e.g. infection or tissue injury, these self-antigens lead to auto-immune responses. 

\item Heritability and clustering of diseases in families: this can be explained by the basic model (1) heritability: both genetics and shared environment; (2) clustering: different environment triggers may lead to different diseases for the same genetic variations. 

\item Specificity of auto-antigens: why the change of the general component of the immune system, e.g. CTLA4, may lead to response to specific auto-antigens? Possible explanations: 
\begin{itemize}
	\item Most of these component are probably associated with specific cell types, e.g. different JAK, STAT family members may be used by different Th cell types. 
	\item Most auto-antigens are not harmful, and only those vulnerable to environmental triggers such as virus infection can lead to disease. Certain types of pathogens may be common. 
\end{itemize}

\item Pre-existing anto-antibodies: this is the evidence of the model. Auto-antibodies are produced early in some immune diseases before the clinical symptoms appear; however, auto-antibodies are also found in healthy individuals. Epidemiological studies could help to elucidate whether the presence of auto-antibodies predisposes to autoimmune disease, or whether auto-antibody production is a consequence of disease.  
\end{itemize}

Common autoimmune/inflammatory diseases: the symptoms, auto-antigens and disease pathogenesis [ibid]: 
\begin{itemize}
\item Asthma: Chronic condition in the respiratory system in which the airways occasionally constrict, become inflamed, and are lined with excessive amounts of mucus. Inflammation in response to exposure to an environmental stimulant, such as an allergen, smoke or perfume, which is mediated by a TH2-type immune response and includes mast cells, eosinophil infiltrates and IgE antibodies

\item Crohn's disease: Chronic, episodic, inflammatory bowel disease, which primarily causes ulceration of the small and large intestines but can affect any region of the digestive system. Unknown; involves an inappropriate immune response to commensal bacteria.

\item Multiple sclerosis: Autoimmune attack of the central nervous system, which leads to demyelination of neurons, causing potentially debilitating physical and mental symptoms. After infection in the brain, trapped T cells initiate an autoimmune response to foreign myelin, thereby triggering inflammatory processes, stimulating other immune cells, cytokines and antibodies. 

\item Rheumatoid arthritis: Chronic inflammation of synovial joints. Autoimmune reaction against connective tissue components. Presence of rheumatoid factor and ACPA. 

\item Systemic lupus erythematosus: Chronic inflammation, can affect any part of the body, but often the heart, joints, skin, lungs, blood vessels, liver, kidneys and nervous system. Autoimmune reaction against nuclear proteins, which leads to the formation of immune complexes. 

\item T1D: Destruction of pancreatic $\beta$-cells, which leads to insufficient release of insulin from the pancreas. T cell-mediated autoimmune response, and production of auto-antibodies against islet cells, insulin, glutamic acid decarboxylase and protein tyrosine phosphatase.
\end{itemize}

Genes implicated in AIDs from earlier studies (single gene diseases) [ibid]: 
\begin{itemize}
\item MHC: has been associated with almost all autoimmune diseases. The MHC locus spans approximately 4 MB and contains about 250 genes, of which about 60\% have immune-related functions. The MHC region is characterized by extended LD blocks (up to 3 MB), and by a strong and complicated LD pattern between the blocks. 

\item AIRE is mutated in APS-1, autoimmune attack against multiple endocrine organs, the skin and other tissues. Mechanism: Decreased expression of self antigens in the thymus, resulting in defective negative selection of self-reactive T cells. 

\item CTLA4 works by competitively blocking the engagement of the activating receptor CD28 (by CD80 or CD86). Several AIDs, including Graves's disease, type 1 diabetes and other endocrinopathies, show a striking association with a CTLA4 polymorphism that results in reduced production of a truncated splice variant.

\item FOXP3 is a regulator of regulatory T cells (CD4 CD25 T cells). Induced knockout or spontaneous mutation of the mouse Foxp3 gene led to a systemic autoimmune disease associated with the absence of CD4+CD25+ regulatory T cells. 

\item The Fas death receptor contributes to the deletion of mature T and B cells that recognize self antigens. 
\end{itemize}

Background: T cell differentiation 
\begin{itemize}
	\item TH1 cells: induced by IL12, IL18, IL27 and IFN$\alpha$, $\beta$, $\gamma$. Signaling: STAT1,3, or 4 $\rightarrow$ TBX21. Proliferation: IL18, IFN$\gamma$. IFN$\gamma$ inhibits development of TH2 and TH17 cells. 
	\item TH17 cells: A subset of CD4+ T-helper cells that produce interleukin 17 (IL-17), now thought to be more important than TH1 cells as mediators in immune-related diseases. Induced by IL6 and TGF$\beta$. Signaling: STAT3, ROR$\gamma$t. Proliferation: IL23, IL21, IL17. 
	\item Treg cells: There is increasing evidence that Treg cells are less active in chronic immune-related diseases. IL-2 and its receptor (encoded by IL2RA, IL2RB and IL2RG) are crucial in the activation and function of Treg cells. Also induced by TGF$\beta$. IL2 deficient mice develop autoimmunity. 
\end{itemize}

Prevention and treatment of AIDs [ibid]: 
\begin{itemize}
	\item Vaccination: associations with bacterial and viral infections have been suggested for most of the diseases discussed, e.g. a positive correlation between MS or SLE and auto-antibodies against EB virus. Defining the profile of the genetic susceptibility pathway, together with knowledge of environmental triggers, might help prevent immune-related diseases through vaccination.
\end{itemize}
	
Genes associated with AIDs from GWAS: identified 23 genes that are shared by two or more diseases (among 11 diseases, from 22 GWAS) [Zhernakova09]
\begin{itemize}
\item Immune cell signaling: 
\begin{itemize}
	\item Shared TCR signaling genes: CTLA4, the protein tyrosine phosphatases PTPN2 and PTPN22 and the adaptor protein SH2B3. PTPN2, plays an important part in the negative regulation of the inflammatory response in T cells. 
	\item TH1 cell signaling: association of IL18RAP (interleukin-18 receptor accessory protein), IL12, IL10, STAT3 and STAT4 with almost all the diseases analyzed. 
	\item TH17 cell signaling: Chronically inflamed tissues are infiltrated with highly differentiated TH17 cells. Genes associated with TH17 cells (for example, IL23R and IL21) are associated with nearly all immune-related diseases.
	\item Treg cell signaling: these cells are less active in chronic immune-related diseases. Two genes in the Treg activation cascade have now been associated with multiple autoimmune diseases: IL2RA (also known as CD25) and the locus that includes IL2 and IL21. 
	\item PDCD1 (programmed cell death 1) gene: associated with SLE. It has been shown to regulate peripheral tolerance in T and B cells.
\end{itemize}

\item Antigen representation and expression: 
\begin{itemize}
	\item MHC class II: different alleles have different abilities to present peptides from target cells to autoreactive CD4+ T cells. E.g. the major genetic contribution to RA involves particular HLA-DR alleles. Certain HLA alleles might be particularly good at presenting glutamic acid decarboxyase (GAD)-65 or insulin peptides to T cells, thus contributing to recognition and ultimate destruction of pancreatic beta cells.  
	\item Genes affect the expression, or distribution of auto-antigens either in lymphoid tissues or in the target organ. Ex. polymorphism of upsream sequence of insulin gene influences transcription of the insulin gene within the thymus and might thus affect T-cell tolerance to this antigen. 
	\item CLEC16A is associated with T1D and multiple sclerosis and encodes a C-type lectin receptor, which might play a part in antigen sampling by dendritic cells. 
\end{itemize}

\item Innate immunity and TNF signaling: 
\begin{itemize}
	\item Barrier function: the association of the NOD2 gene and the MUC19 (mucin 19)-containing locus to Crohn's disease. Deficiency in another component of the mucosal mucus layer, MUC2, leads to inflammatory bowel disease. Another barrier risk factor is copy number variation (CNV) in defensin genes. 
	\item Autophage is known to be involved in Crohn��s disease, underscoring the role of intracellular processing of bacteria in disease pathogenesis. Three Crohn's disease genes, ATG16L1 (autophagy 16 related-like 1), IRGM (immunity-related GTPase M) and LRRK2 (leucine-rich repeat kinase 2). SLE is associated to another key autophagy molecule, ATG5. 
 	\item Type 1 interferons mediate the early innate immune response to viral infections. IRF5 and IFIH1 (interferon induced with helicase C domain; also known as MDA5), involved in the IFN$\alpha$ pathway, are associated with several autoimmune and inflammatory diseases. 
	\item TNF signaling: TNFSF15, which is associated with Crohn's disease, is activated via stimulation by LPS. TNFAIP3, which is associated with SLE, rheumatoid arthritis and coeliac disease, encodes the A20 protein, which is required for termination of the NF$\kappa$B signal that is mediated by innate immune receptors. 
\end{itemize}

\item Tissue response: the gene variants may influence relative isolation of tissues from the immune system and inhibition of function of invading lymphocytes. 
\begin{itemize}
	\item The eye, one of the beststudied examples of a protected site, has barriers to T-cell infiltration and produces immunosuppressive cytokines, such as transforming growth factor (TGF)-beta.
	\item Fc$\gamma$RIIA and Fc$\gamma$RIII alleles coding for proteins with lower than normal activity are associated with susceptibility to SLE and lupus nephritis, probably because these alleles clear circulating immune complexes inefficiently and allow increased immune complex deposition in the kidney. 
\end{itemize}

\item Other genes: 
\begin{itemize}
	\item Chemokines: their major role is to regulate the immune response and recruit effector immune cells to sites of inflammation. CCL21 (chemokine (C-C motif) ligand 21) is associated with rheumatoid arthritis, CCR6 (chemokine receptor 6) is associated with Crohn's disease. Chemokine genes usually form a cluster with strong LD between genes. 
	\item TSHR (thyroidstimulating hormone receptor): associated with autoimmune thyroid disease.
	\item The ORMDL3 gene is associated with asthma and Crohn��s disease and was prioritized for study owing to the strong cis-correlation between ORMDL3 expression and its associated genotype.
\end{itemize}
\end{itemize}

Genetic insights into common pathways and complex relationships among immune-mediated diseases [NRG, 2013]
\begin{itemize}
	\item Sero-positive and sero-negative: whether a patient has auto-antibodies. Positive: T1D, RA. Negative: AS, Coelic disease, IBD, psorasis.
	
	\item Allelic heterogenity and rare variants: CARD9 and NOD2 loci, rare variants have distinct effects from common variants.
	
	\item Shard genetics: correlated and concordant (increase risk for both disease), correlated and discordant (opposite effects), non-correlated (different risk haplotypes in the same locus).
	
	\item Patterns of shared loci: Box 2, for some traits (e.g. all seronegative), most correlated and concordant, but for some pairs, e.g. IBD and T1D, most are discordant and non-correlated. Over 400 pairs, about 40\% concordant, 14\% discordant and the rest non-correlated.
	
	\item Specific loci: (1) Th1 and IL-23 pathway: IL12R and IL23 pathway activation in Th1 and Th17 cells. (2) Other loci: MHC, NF-kappa B. IRF.
	
	\item Disease specific loci: NOD2 (autophagy) in CD: important for gut bacterial. HNF4A in UC: epithelia barrier. INS (insulin) in T1D: auto-antibody generation.
\end{itemize}

Genetics of allergy and allergic sensitization: common variants, rare mutations [Curr Opinion Immuno, 2015]
\begin{itemize}
	\item Background: allergy is mediated by Th2 cells (B cell activation) and AIDs by Th1 cells (macrophages, neutrophils)
	
	\item Allergy vs. asthma: likely high sharing, in one study, 9/10 allergy loci are also associated with asthma.
	
	\item Biological pathways from GWAS (Figure 1): Epithelia barrier function: FLG. Innate immunity sensing: some TLR genes. T cell activation: MHC, IL-2, IL2RB. T cell response: especially Th2 pathway, IL13, IL33. T-reg and immune tolerance: LRRC32, TGF-beta, SMAD3, FOXP3. 
	
	\item Allergy vs. IBD: many shared loci, 12/18 allergy loci also IBD. However, the effects may be in the same direction or not.
\end{itemize}

Crohn's disease: WTCCC [WTCCC, Nature, 2007]
\begin{itemize}
\item Background: The pathogenic mechanisms are poorly understood, but probably involve a dysregulated immune response to commensal intestinal bacteria and possibly defects in mucosal barrier function or bacterial clearance. 

\item Replicate the genes/regions previously reported: IL23R, NOD2, ATG16L1 (ATG16 autophagy related 16-like 1) gene, a noncoding intergenic SNP mapping 14-kb telomeric to gene ZNF365 and 55-kb centromeric to the pseudogene antiquitin-like 4, within a 1.2Mb gene desert on chromosome 5p13.1

\item Four new strong associations: all successfully replicated
\begin{itemize}
\item IRGM: a GTP-binding protein which induces autophagy and is involved in elimination of intracellular bacteria
\item MST1 (macrophage stimulating 1), which encodes a protein influencing motile activity and phagocytosis by resident peritoneal macrophage. (The strongest SNP is a synonymous coding SNP within the BSN gene, involved in neurotransmitter release.)
\item NKX2-3: Targeted disruption of the murine homologue of NKX2-3 results in defective development of the intestine and secondary lymphoid organs. 
\item PTPN2: encodes the T cell protein tyrosine phosphatase TCPTP, a key negative regulator of inflammatory responses. The same locus also shows strong association with T1D susceptibility and a weak association with RA (P = 1.9E-2)
\end{itemize}

\item Other putative genes with weaker evidence, based on biological candidacy. 
\begin{itemize}
\item HLA: $P = 8.7E-7$
\item TNFAIP3 (TNFa induced protein 3): same pathway as NOD2
\item TNFSF15 (tumour necrosis factor super family, member 15): $P = 9E-5$, previously reported associated with CD
\item STAT3: $P = 3.1E-5$
\item CD40LG: CD40 ligand, $P = 1.3E-7$
\end{itemize}
\end{itemize}

T1D: WTCCC [WTCCC, Nature, 2007]:
\begin{itemize}
\item Six genes/regions for which there is strong pre-existing statistical support for a role in T1D-susceptibility: MHC, insulin gene, CTLA4, PTPN22, (IL2RA/CD25), IFIH1/MDA5. Five of these previously identified associations were detected in this scan ($P \leq0.001$), the exception being the INS gene. 

\item Three novel regions from single point analysis, all replicated (12q13, 12q24 and 16p13). Four regions from multipoint analysis or from analysis of all AIDs, one replicated (18p11). 
\begin{itemize}
\item 12q13 region: extensive LD of $>10$ genes. Candidates: ERBB3 (receptor tyrosine-protein kinase erbB-3 precursor)
\item 12q24 region: extensive LD of $>10$ genes. Candidates: SH2B3/LNK (SH2B adaptor protein 3), TRAFD1 (TRAF-type zinc finger domain containing 1) and PTPN11 (protein tyrosine phosphatase, non-receptor type 11). Of those listed, PTPN11 is a particularly attractive candidate given a major role in insulin and immune signalling, and also the same family as PTPN22. 
\item 16p13 region: two genes of unknown function
\item 18p11 region: seems to confer susceptibility to all three autoimmune conditions. PTPN2.
\end{itemize}
\end{itemize}

Crohn's disease: meta-analysis [Barrett \& Daly, NG, 2008]: 
\begin{itemize}
\item Data: 3 studies including WTCCC. The combined sample has 74\% power at an OR of 1.2. Total: about 4,000 cases and controls respectively, and about 2,000 cases and controls, respectively, in the replication data.  
\item Method: the meta-analysis used a test that combined the results from each study (rather than mixing the raw data and compromising the case-control matching of each study). We summarized the standard 1 d.f. allele-based test of association as a Z score within each scan and combined scores across studies to produce a single meta-statistic for each SNP across all three datasets. 
\item A marked excess of significant associations, well beyond what would be attributable to the modest overall distributional inflation ($\lambda_{GC} < 1.16$). 
\item 526 SNPs from 74 distinct genomic loci that were associated with $P<5E-5$, which is more than seven times the number of SNPs expected by chance even after correction for the modest overall inflation detected.
\item Eleven associations previously replicated were among the 74 regions represented. 
\item The signficance of 63 new regions: strong departure of null distribution (even removing 21 loci, below), suggesting an enrichement of true associations. 
\item 19 new associations were replicated, and 2 other regions, 19p13, and MHC. The total of 32 expanded associations. Some newly implicated genes: 
\begin{itemize}
\item CCR6: homing receptor (GPCR), expressed by immature dendritic cells and memory T cells and is important for B-cell differentiation and tissue-specific migration of dendritic and T cells
\item IL12B: this gene encodes the p40 subunit, which is a constituent of both heterodimeric interleukins IL-12 and IL-23
\item STAT3 and JAK2: the role of both genes in IL23R signaling and the central role of STAT3 in Th17 differentiation. 
\item ICOSLG (inducible T-cell co-stimulator ligand): expressed on intestinal (and other) epithelial cells and may have a role in their antigen presentation to and regulation of mucosal T lymphocytes
\item ITLN1 (intelectin-1): expressed in human small bowel and colon, and encodes a 120-kDa homotrimeric lectin recognizing galactofuranosyl residues found in cell walls of various microorganisms
\end{itemize}

\item Coding sequence variation: just 9 of the 32 genome-wide significant associations were correlated with a known nsSNP ($r^2 > 0.5$)
\item The possible role of cis-regulatory variation: five correlations between expression of a nearby gene and a CD-associated variant in [Dixon07] data. Expected number = 0.001. 
\begin{itemize}
\item PTGER4
\item IBD5 region: the CD-associated SNPs were associated with decreased SLC22A5 mRNA expression
\item The most significant CD-associated eQTL reported here affects ORMDL3, SNPs in precisely the same region were recently shown to be strongly associated with childhood asthma
\end{itemize}
\item Using a liability-threshold model, we estimate that the 32 loci identified to date explain about 10\% of the overall variance in disease risk, which may be as much as a fifth of the genetic risk, given previous estimates of CD heritability of approximately 50\%
\end{itemize}

T1D meta-analysis [Barrett \& T1DGC, NG, 2009]: 
\begin{itemize}
\item Data: The total sample set included 7,514 cases and 9,045 reference samples. WTCCC, GoKinD/NIMH	and T1DGC. The two earlier studies (WTCCC, GoKinD/NIMH) and the current one (T1DGC) used different platforms. As only 9\% of SNPs are shared between these platforms, we used imputation to combine results across studies.

\item Meta-analysis: Mantel's extension to the 1 degree-of-freedom (1-d.f.) Cochran-Armitage trend test that combined comparisons over the three studies.

\item Forty-one distinct genomic locations provided evidence for association with T1D in the meta-analysis ($P < 10^{-6}$)

\item Replication: After excluding previously reported associations, we further tested 27 regions in an independent set of 4,267 cases, 4,463 controls and 2,319 affected sib-pair (ASP) families. Of these, 18 regions were replicated ($P < 0.01$). 

\item Several of the 18 regions identified here contain genes of possible functional relevance to T1D. 
\begin{itemize}
\item The region 1q32.1 contains the immunoregulatory cytokine genes IL10, IL19 and IL20.
\item The region of strong LD at 9p24.2 contains only a single gene, GLIS3. 
\item The region on 12p13.31 harbors a number of immunoregulatory genes including CD69, and  calcium-dependent (C-type) lectin (CLEC) domain family with immune functions
\end{itemize}

\end{itemize}

GWAS of host control of HIV-1: Euro-CHAVI study [Fellay \& Goldstein, Science, 2007]: 
\begin{itemize}
\item Data: 486 patients, genotyping of 555,352 SNPs. The phenotype is (1) the viral set point, i.e. the level of circulating virus in the plasma during the nonsymptomatic phase preceding the progression to AIDS; (2) progression. 

\item Signficiant associations with viral load: with Bonferoni correction, two loci were found significant, both in HLA region. 
\begin{itemize}
	\item Polymorphism in HCP5 gene: explains 9.6\% of the total variation in set point. Also in strong LD with HLA-B*5701 ($r^2 = 1$). Both are possible candidates. For HCP5: it encodes a human endogenous retrovirus with sequence homology to HIV-1 pol, it may act as antisense RNA interfering with HIV-1 replication.
	\item Polymorphism in HLA-C gene: explains 6.5\% of the variation in set point. In weak LD with HCP5, however, the effect cannot be explained the HCP5 SNP. cis-eSNP data shows that the SNP is associated with expression of HLA-C. 
	\item No other significant association: no overall inflation of P values (indicating little contribution from population stratification), but an excess of low P values, beginning with the 355th most associated SNP. 
\end{itemize}

\item Significant associations with progression: SNP near ZND1 (RNA Pol I) gene (1 Mb from the previous candidates), explain 5.8\% variation. Also cis-eSNP of ZND1, and plausible functional evidence. 

\item Replication: 140 Caucasian patients. All three associations were confirmed at $P < 0.05$. 
\end{itemize}

GWAS of AIDS progression: GRIV study [Limous \& Zagury, Genomewide Association Study of an AIDS Nonprogression Cohort Emphasizes the Role Played by HLA Genes (ANRS Genomewide Association Study 02), J Infect Dis, 2009]
\begin{itemize}
\item GRIV cohort: 275 human immunodeficiency virus (HIV) type 1-seropositive nonprogressor patients in relation to a control group of 1352 seronegative individuals. Genotyping: HumanHap300 BeadChips

\item Statistical analysis: 
\begin{itemize}
	\item For each SNP, we performed a standard case-control analysis using Fisher��s exact test (with PLINK software) to compare allelic distributions between the nonprogression group and the control group. Bonferroni correction
	\item Population stratification: genomic inflation factor: $\lambda = 1.064$. 
	\item Meta-analysis: 286,529 SNPs are common between GRIV and Euro-CHAVI [Fellay07]. Meta-analysis using Fisher's method. 
\end{itemize}

\item Significant associations: HCP5 has the only significant SNP after Bonferroni corrections, the same SNP identified in the Euro-CHAVI study.  
\begin{itemize}
	\item HCP5-independent signals: Most of the signals from chromosome 6 disappeared because of the genetic linkage with the HCP5 rs2395029-G.
	\item The strongest signals were still found in the HLA region, with 2 SNPs of the ZNRD1/RNF39 region. Unlike the HCP5 rs2395029 SNP, none of the ZNRD1/RNF39 SNPs alleles seemed to correlate with viral load (figure 4), suggesting that this locus influences disease progression.
\end{itemize}

\item Results of meta-analysis: found other associations. 
\begin{itemize}
	\item C6orf48 rs9368699 SNP, in LD with the HCP5 SNP. 
	\item The PSORS1C1 gene exhibited 2 significant SNPs, rs3823418 and rs3815087. PSORS1C1 is a psoriasis-susceptibility candidate gene. 
	\item HLA-C-related SNP rs10484554 (identified in Euro-CHAVI study)
\end{itemize}

\item The list of best SNPs: 
\begin{itemize}
	\item Of the 50 best signals found in the  meta-analysis, 46 originated from the HLA locus, emphasizingthe massive role played by HLA in the nonprogression phenotype.
	\item In this GWAS alone, 31 of the 50 best signals were not from chromosome 6 (table 1) and were not found in the meta-analysis (table 6), suggesting that positive signals outside the HLA locus may be associated with the nonprogression phenotype without influencing viral load.
\end{itemize}

\item Remark: 
\begin{itemize}
	\item Viral load and progression may be associated with different loci. 
	\item There may be significant number of loci outside HLA, especially when considering disease progression. 
	\item An important problem for HLA loci is to identify the causal gene(s). This is difficult because of the extensive LD in the HLA region. 
\end{itemize}
\end{itemize}

GWAS of HIV control: CHAVI study [Fellay \& Goldstein, Common Genetic Variation and the Control of HIV-1 in Humans, PLG, 2009]
\begin{itemize}
\item Data: 2,554 infected Caucasian subjects. The study was powered to detect the effects of common variants down to 1.3\% of explained variability. From Euro-CHAVI Consortium ($N=1,397$) and MACS cohort ($N = 1,157$). A total of 2362 individuals were included in the set point association analyses  and 1071 seroconverters were eligible for the analysis of disease progression. 

\item Statistical analysis: each SNP passing the QC step were tested for association with HIV-1 viremia (quantitative trait) at set point in separate linear regression models that included gender, age, and the 12 significant PC axes as covariates.

\item Associations with viral load at set point: $\lambda = 1.006$. 
\begin{itemize}
	\item The 2 SNPs previously reported as genome-wide significant (HCP5 and HLA-C) were confirmed to be the strongest determinants of variation in HIV-1 viral load. 
	\item Further independent associations in the MHC region: using stepwise regression model, found 4 additional SNPs. Altogether, a model including 6 SNPs, 4 alleles and homozygosity status shows that MHC variation explains 12\% of the set point variability in this cohort.
\end{itemize}

\item Association with disease progression: 
\begin{itemize}
	\item Top associations: SNPs at HCP5/B*5701 and HLA-C. If viral load at set point is added to the models, the association signals are much weaker, demonstrating that the HCP5/B*5701 and HLA-C effects on disease progression are mainly driven by their impact on early viral control.
	\item Another set of variants reached genome-wide significance ($P < 1E-8$): in high-LD and located around the ZNDR1 and RNF39 genes, close to the HLA-A locus. This association is largely independent of viremia, suggesting that a different mechanism of action is here modulating HIV disease progression.
\end{itemize}

\item Candidate genes: tested a total of 34 SNPs in 21 genes, representing 27 previously reported associations with HIV-1 control. Most of them are not significant ($P > 0.05$). Signifciant ones are variants in CCR5 (HIV-1 co-receptor) and CCR2 (minor receptor). 
\end{itemize}

GWAS of HIV-1 control: International HIV Controller Study [Perayra \& Zhao, The major genetic determinants of HIV-1 control affect HLA class I peptide presentation, Science, 2010]
\begin{itemize}
\item Data: 974 controllers (cases) and 2648 progressors (controls) from multiple populations, genotyped on 1M SNPs. HIV controllers: able to control viral replication without therapy, typically maintain stable CD4+ cell counts and do not develop clinical disease. 

\item Statistical analysis: logistic regression including the major principal components as covariates to correct for population substructure. Genome-wide significance defined as $P < 5E-8$ (Bonferroni correction). 

\item Associations: 
\begin{itemize}
	\item In the largest group, comprising 1712 individuals of European ancestry: 313 SNPs with genome-wide significance, all in MHC region. 
	\item Candidate genes: Only variants in the CCR5-CCR2 locus replicate with nominal statistical significance in our study. 
	\item Independent markers with associations to host control: using stepwise regression, identified 4 SNPs in the MHC region, near or in genes: HLA-C (cis-eSNP), HLA-B*57:01, MICA and PSORS1C3. These four SNPs explain 19\% of the observed variance of host control in the European sample
\end{itemize}

\item Causal polymorphisms in MHC genes: Specific amino acids in the HLA-B peptide binding groove, as well as an independent HLA-C effect, explain the SNP associations. 
\end{itemize}

A CD8+ T cell transcription signature predicts prognosis in autoimmune disease [McKinney \& Smith, NM, 2010]
\begin{itemize}
\item Data: purified CD8+ T cells in 32 individuals with AAV (a disease). 

\item Clustering of transcriptome: identify two subgroups of patients, each has a characteristic expression pattern (one subgroup of genes). In fact, only three genes are needed to define the two groups of patients.

\item Unsupervised gene clusters also predict prognosis of patients, one group has little replase. The signatures: IL-7 signaling and TCR signaling in one group.   
\end{itemize}

Human SNP links differential outcomes in inflammatory and infectious disease to a FOXO3-regulated pathway. [Lee \& Smith, Cell, 2013]
\begin{itemize}
\item Motivation: what determines the prognosis of CD patients? Some are aggressive CD, some are indolent. 

\item GWAS of prognosis: aggressive CD (668) and indolent CD (389). Limit to 81 genes involved in IL-2 or IL-7 signaling (from [McKinney, NM, 2010]). Found one SNP in FOXO3, OR = 0.62, and $p$ value highly significant in combined (with replication) dataset. 

\item Regulatory effect of the SNP: not in LD with any coding region of FOXO3, and use ASE to show that minor allele is associated with higher expression (in heterozygous individuals) in monocytes. 

\item The effect of SNP on cellcular phenotype: association of the SNP with cytokine production (TNF$\alpha$ and IL-10) in stimulated condition. 

\item Mechanism of the SNP on FOXO3A: FOXO3 nuclear/cytoplasmic ratio is higher in G (minor allele), and show that it is due to de novo synthesis of FOXO3A in nucleus. 

\item Mechanism of FOXO3: affect TGF$\beta$ dependent production of cytokines. Show that FOXO3 binds to TGF$\beta$ promoter using ChIP-qPCR. 

\item Phenotypic study of FOXO3: deletion of the gene in mice leads to higher disease severity. 

\item \textbf{Lessons}: 
\begin{itemize}
	\item Limit to candidate genes, e.g. from DE studies, to increase the power. 
	\item Cellular phenotypes: cell proliferation, cytokine production, etc.
	\item Studying mechanism of SNP: regulatory effect (eQTL, ASE), downstream process, e.g. TF expression could affect downstream genes. 
\end{itemize}
\end{itemize}

The Allelic Landscape of Human Blood Cell Trait Variation and Links to Common Complex Disease [Soranzo, Cell, 2018]
\begin{itemize}
	\item Study: 150K subjects, 36 blood trait indices. GWAS analysis using multiple regression. Conditional regression plus LD pruning. 2700 sentinel variants: each representing high LD group ($r^2 > 0.8$). About 20\% have AF $< 5\%$. 
	
	\item Heritiability analysis: PVE from common variants (LDSC) somewhat higher than the total explained by discovered variants (both common and rare).
	
	\item Distribution of disease variants: introns explain twice more than intergenic, which is more than coding. Near genes similar to coding. By epigenomic states: transcription about the same as enhancer, and more than promoter.  Enrichment of cell-type specific active enhancers: 5-10 fold enriched in matched cell types.
	
	\item Colocalization with molecular QTL: use BLUEPRINT eQTL, splicing QTL (sQTL) and caQTL. Half of overlap are with hQTL (no effect on expression). In all cases of overlap: 25\% can be attributed to colocalization (SMR). Overall, 10\% of GWAS loci can be assigned to molecular traits.
	
	\item MR analysis with related phenotypes: use multiple IVs. Some very large effects between myeloid cells and AIDs (e.g. asthma). With SCZ, find lymphocyte counts, however, this is driven by MHC locus.
\end{itemize}

Atopic dermatitis (AD) is associated with an increased risk for rheumatoid arthritis and inflammatory bowel disease, and a decreased risk for type 1 diabetes. [Schmidt, J Allergy Clin Immunol. 2016]
\begin{itemize}
	\item Background: AD genetics, 13 European loci and 10 Asian loci. FLG has skin barrier function, and others immune dysregulation. Most loci are also shared with other immune diseases that are Th1/Th17 mediated, including IBD, RA and T1D. However, the sharing patterns are complex, with loci often having opposing effects. Ex. IL6R allele increases the risk of AD and Asthma and protective of RA.
	
	\item Co-morbidities of AD and immune diseases: AD is associated with higher risk of IBD and RA (RR = 1.3 - 1.7), and lower risk of T1D (RR = 0.7).
	
	\item Association of immune risk alleles with AD risk: 10 passes significance, 7 have agnostic effects on AD. For those with suggestive associations with AD: $<50\%$ show effect direction consistent with epidemiological results.
	
	\item Discussion: AD patients receive systemic steroids, which may mediate the effects on other immune diseases.
	
	\item Discussion: AD, IBD and RA: T-cell mediated inflammation. For RA and IBD: TH1 and TH17 responses promoting autoimmunity contribute. Autoreactivity in up to one third of the patients with AD. Hypothesis: the development of RA and IBD in subgroups of patients with AD is precipitated by a sustained skin inflammation with increased TH1/TH17 signaling and secretion of proinflammatory cytokines such as TNF-alpha.
\end{itemize}

Shared genetic origin of asthma, hay fever and eczema elucidates allergic disease biology [Ferreira and Paternoster, NG, 2017]
\begin{itemize}
	\item GWAS of three phenotypes (treated as a single trait): 130 loci. 18 loci have more than 2 signals.
	
	\item Candidate genes: use coding and eQTL, 132 candidate genes.
	
	\item Cell type analysis using S-LDSC: enrichment in multiple cell types, Th1, Th2, Th17, T-reg, CD4, CD8, NK, B cells.
	
	\item Pathway analysis using candidate genes: T cell activation, B cell activation, B cell proliferation, isotype switching, IL-2 and IL-4 production (IL-2: T-cell activation, IL-4: Th2 cells).
	
	\item Comparison with other traits (Table S23): (1) Strong genetic correlation with asthma, r = 0.7. (2) Significant with obesity: 0.2. (3) Correlation with IBD/CD, r = 0.13, p = 0.004. With UC, not significant, r = 0.07.
\end{itemize}

Interrogation of human hematopoiesis at single-cell and single-variant resolution [Ulirsch and Sankaran, NG, 2019]
\begin{itemize}
	\item Data: UKBB, blood cell related traits. Fine-mapping: on 3Mb regions, FINEMAP. ATAC-seq data in 18 cell populations. 
	
	\item Summary of fine-mapping results: 38K variants with PP $> 1\%$, and 1000 regions with one variant with PP $> 0.5$ (Figure 1CD). Enrichment in enhancers, coding, but very modest in UTRs.
	
	\item Finding cell types: clustering of cell type specific chromatin accessibility of fine-mapped SNPs. Show several clusters.
	
	\item Molecular mechanisms of fine-mapped variants: (1) Coding variants: suggest genes of Mendelian diseases. (2) Motif disrupting variants in ChIP-seq targets (2000 ChIP-seq data): 145 instances. A small number of instances per TF, about 1-2-fold enrichment than chance. Top motifs: SPI1, SP1, EGR1, KLF1, MAZ, MYC, JUND.
	
	\item Target genes of fine-mapped variants: (1) Hi-C from monocytes. (2) ATAC-RNA correlation in 16 cell types.
\end{itemize}

A genome-wide cross-trait analysis from UK Biobank highlights the shared genetic architecture of asthma and allergic diseases [Zhu and Liming Liang, NG, 2019]
\begin{itemize}
	\item Data: allergy, asthma, eczema, hay fever of UKBB.
	
	\item Genetic correlation among immune traits: high correlation among these traits, but not with IBD.
	
	\item Shared loci by cross-trait meta-analysis: found 38 shared loci.
	
	\item Tissue enrichment: for shared genes, skin shows highest enrichment, then whole blood, vagina, esophagus and lung.
	
	\item Model of shared and distinct genetics of immune diseases: allergic diseases, IgE mediated hypersensitivity. RA: immune-complex mediated hypersensitivity. IBD: delayed cell mediated hypersensitivity.
	
	\item Why skin and other tissues are enriched of shared genes? (1) Why skin: e.g. FLG gene: function of barrier. Mutations of FLG: sensitive to external allergens and dry skin. This can activate allergic immune response to many organs via blood. (2) Why other epithelial tissues? Share similarity in functions, e.g. epithelia lining in lung and skin.
	
	\item \textbf{Lesson}: The effect of genetic variants may be very indirect: e.g. a variant acts on skin, which changes permeability and in turn affects immune reactions.
	
	\item \textbf{Lesson}: different types of autoimmune and allergic diseases, mediated by different molecules/cell types.
\end{itemize}

Landscape of stimulation-responsive chromatin across diverse human immune cells [Calderon and Pritchard, NG, 2019]
\begin{itemize}
	\item Data: 25 primary immune cell types from 4 donors, in both resting and stimulated states, also thymus cells. T cells simulated by cross linking TCRs and co-stimulatory receptors, monocytes by LPS and NK cells with cytokines (IL2, etc).
	
	\item Stimulation response in terms of OCR changes: large changes in B and T cells, but not innate immune cels. Also stimulation mostly increases OCR and expression in T/B cells.
	
	\item Calling ASC variants: WASP for mapping bias, then do binomial test, and FDR < 0.1. Found 607 ASC on average per sample.
	
	\item Contribution of TFs to ASC: motif enrichment on OCRs find some cell-type specific TFs, e.g. B-ATF. Motif break type of analysis of B-ATF targets: in resting states, motif breaks do not correlate with allele imbalance; in stimulated cells, show correlation (Figure 4b: comparison of allele imbalance across heterozygous SNPs).
	
	\item Cell type specificity of ASC variants: For ASC in one cell type, three cases in the second cell type: shared, inaccessible chromatin and accessible chromatin but different effect (Figure 4c). Estimate the proportion of three scenarios (Figure 4d).
	
	\item Enrichment of AID h2g in stimulated OCRs: control tissues, calf muscle, breast epithelium. S-LDSC enrichment: progenitor cells not much enriched comparing with controls, resting PBMC higher and stimulated even higher (Figure 5a). The enrichment is widespread across multiple cell types. Clustering peaks: clusters of shared and cell type-specific peaks show enrichment (Figure 5c).
	
	\item Stimulated ASCs vs. resting ASCs: similar enrichment in GWAS (suggesting both are important); however, in eQTLs, resting ASCs more enriched, because the current eQTL data have less stimulated cells (Figure 6).
	
	\item Case study (Figure 7): candidate risk variant of RA and UC, OCR and ASC in stimulated T cells, but not other cells. SNP changes motif of NF-kappaB1, and confirm the effect on TF binding (ChIP-seq).
	
	\item Lesson: it is sometime difficult to define cell-type specific features (expression, OCRs, etc). Instead of performing DE or DA analysis, use clustering analysis to group features, and identify groups with shared, or cell-type specific patterns.
	
	\item Lesson: (about presentation) emphasize the novelty of findings, stimulated states are important for genetics. Shown this in multiple ways: h2g analysis, lack of representation in current eQTL data, case study.
\end{itemize}

The impact of proinflammatory cytokines on the β-cell regulatory landscape provides insights into the genetics of type 1 diabetes [NG, 2019]
\begin{itemize}
	\item Background: T1D genetics, about 60 loci mapped. Enrichment: T- and B-cell enhancers, but not much in beta cells.
	
	\item Experiment: human islet cells (HI) and beta cell lines (EC), treated with IL-1beta and IFN-gamma. ATAC, H3K27ac, RNA, DNAm, UMI-4C.
	
	\item Identification of cytokine induced regulatory elements (IREs): 3,800 OCRs that gain H3K27ac upon cytokine - called IREs. Most IREs map to distal regions. IREs associate with expression and protein changes.
	
	\item Groups of IREs: opening IREs: 2.4K, both OCR and H2K27ac induced by cytokine (most are not detectable by ATAC). Primed IREs: 1.3K, already open (primed) before cytokine.
	
	\item TF occupancy of two groups of IREs: both enriched with INF-gamma response elements and targets of STAT and NF-kappaB (all involved in response). Primed IREs: also enriched with islet lineage specific TFs.
	
	\item DNAm of IREs: low CpGs, and not changed much by cytokines. Neo-IREs (newly activated): demethylation by cytokine treatment.
	
	\item Model of IRE changes (Figure 2f): Primed IREs: already open, occupied by islet-lineage TFs, and low CpG, and upon activation, binding by cytokine response TFs and activate gene expression. Neo-IREs: low accessibility and high DNAm before stimulation, and activated by cytokine response TFs.
	
	\item Changes of chromatin interactions by UMI-4C: 13 regions (IREs), show evidence of more chromatin interaction upon stimulation.
	
	\item Enrichment of GWAS loci: (1) T2D: using index SNPs in LD proxy, enriched in stable regulatory elements (SREs), but not IREs. (2) T1D: enriched in IREs, but not SREs.
	
	\item Specific GWAS variants: one SNP, likely to be causal based on GWAS, overlaps with IRE. Validated by allele-specific reporter stimulated by cytokine. Use 4C to map target gene, TNFSF18, whose expression is induced by cytokine.
	
	\item Lesson: stimulation induced changes of epigenome and transcriptome: (1) Different groups of CREs show different dynamics/epigenome changes: stable, primed and neo (newly activated). (2) Different TFs likely contribute to behavior of different groups.
	
	\item Remark: in the absence of hi-C, we can potentially use gene expression changes to identify targets of IREs.
\end{itemize}

%%%%%%%%%%%%%%%%%%%%%%%%%%%%%%%%%%%%%%%%%%%%%%%%%%%%%%%%%%%%
\section{Neuro-Psychiatric Traits}

News Feature: Better models for brain disease [PNAS, 2016]
\begin{itemize}
	\item Mouse model: (1) Criticism of mouse models that are based entirely on behavior signs. (2) Mouse model carrying disease mutations. SHANK3 study (Guoping Feng): autism-version of Shank3, weakened neuronal signaling in the striatum during early development (area involved in repetitive behavior); SCZ-version of Shank3, reduced signaling in the medial prefrontal cortex in later development. 
	  
	\item Monkey model: MECP2 study. 
	
	\item iPSC neurons: (1) bipolar study: iPSC neurons from bipolar patients, hyperexcitability. (2) Autism study: from patients with large brain, show overproduction of inhibitory neurons. Limitations: only neurons of very early developmental lineages. 
\end{itemize}

X-linked diseases [ELS: Chromosome X: General Features]:
\begin{itemize}
\item Rett Syndrome: one of the most common causes of mental retardation in females. Rett syndrome is a dominant X-linked disease, so affected females are heterozygous. Affected males are rare because it is usually lethal in early male development. Rett syndrome is caused by mutations in the gene MECP2, which is involved in X-chromosome inactivation.

\item Fragile X syndrome is a common X-linked mental retardation condition (one in 4000 males) with an unusual inheritance pattern. The disease is characterized by moderate to severe mental retardation, prominent jaw, large ears and high-pitched jocular speech. It is caused by mutation of one region of the X chromosome - the amplification of a short triplet repeat of CGG. The amplified triplet repeat becomes highly methylated and disrupts expression of the nearby gene fragile X mental retardation 1 (FMR1), leading to the disease. The inheritance of fragile X syndrome is unusual and complex because the repeats are unstable and can be amplified in germ cells or tissue. 
\end{itemize}

The inheritance of Tourette Disorder: A review [Pauls \& Scharf, J Obcessive-Compulsive and Related Disorders, 2014]
\begin{itemize}
	\item TD is a genetic disease: 
	\begin{itemize}
		\item Aggregation studies: the risk of TD in first degree relatives in 10-100 times higher than general population. 
		\item Twin studies: MZ concordance rate of 50-77\%, while DZ has 10-23\% rate. 
	\end{itemize}
	Comorbility of TD: with OCD and ADHD. 
	
	\item Linkage analysis: only one site is confirmed, in the gene HDC, a rate-limiting enzyme in histamine (HA) biosynthesis. 
	
	\item Genome rearrangement and rare CNVs: some regions have been identified from cytogeneti abnormalities, sometimes overlapping with autism regions/genes. Three CNV studies with a few hundred to 1,000-2,000 samples (cases and controls): overlapping regions with ASD, and HA signaling. In the largest CNV study, found 3.3 fold burden of large deletions in recurrent pathogenetic CNVs in subjects with other neurodevelopmental disorders. 
	
	\item GWAS: first GWAS of 1285 cases and 4964 controls, found no genome-wide significant loci, but the top signals enriched for brain eQTL. Heritability explained by common SNPs: 0.58 out of total of 0.6-0.8. TS and OCD have estimated genetic correlation of 0.41 (using two GWAS data). 
	
	\item Summary of findings: HA signaling in TD, supported by (1) LoF mutation in HDC in a dense pedigree; (2) CNV studies found overrepresentation of HA signaling pathway; (3) overtransmission of SNPs in HDC region in 520 nuclear families.  
\end{itemize}


Brain Expression Genome-Wide Association Study (eGWAS) Identifies Human Disease-Associated Variants [Zou and Ertekin-Taner, PLG, 2012]
\begin{itemize}
\item Data: 197 subjects with Alzheimer's disease (AD) neuropathology and 177 with other pathologies (non-AD), in cerebellar tissues and in temporal cortex.
\item eGWAS analysis: analyzed the ADs and non�CADs separately on cisSNPs. The direction and magnitude of associations in both groups demonstrate remarkable similarities (Pearson's correlation coefficient = 0.98, $p<0.0001$). Found 10,281 total eQTL (1,875 unique genes) in the combined datasets.
\item Effect size: We found that the ``best'' cisSNP explained a median of $3\%$ of the expression variation. For the top 746 probes, the ``best'' cisSNPs accounted for a median of 18\% of the expression variance. 
\item eQTL overlap across brain regions: the top 2,980 cerebellar eGWAS associations were followed up in the temporal cortex validation study. We found that 2,685 top cerebellar cisSNP/transcript associations could be tested in the temporal cortex. The effect sizes are also highly correlated, Pearson correlation 0.94. 
\item Among 2,596 top cerebellar eGWAS cisSNP:  identified 47 cisSNPs that were also associated with 36 diseases/traits in GWAS catalog, 2.4-fold enrichment than expected by chance. The results include both brain disease (Parkison, ADHD) and non-CNS diseases such as SLE. 
\item Intersection with GWAS of AD (ADGC data): SNPs with suggestive AD risk association ($p_{meta} < 1e-3$). 
\begin{itemize}
\item Imputation of SNPs in the current data: 77,126 cerebellar (63,652 unique SNPs, 2,338 unique genes) and 68,172 temporal cortex (57,922 unique SNPs and 2,201 unique genes)
\item 380 cisSNPs that were significant for the cerebellar transcript associations and also had suggestive AD risk associations (2.9-fold enrichment), 432 such temporal cortex cisSNPs (3.3-fold enrichment) and 356 cisSNPs significant in both the cerebellum and temporal cortex (2.7-fold enrichment
\item Did not identify strong transcript associations for some of the top genes recently implicated in AD risk in large LOAD GWAS studies
\end{itemize}

\item Remark: 
\begin{itemize}
\item Significant overlap of eQTL across different phenotypes and different brain regions
\item The top cis-eSNP usually different from the top genes found in GWAS.
\end{itemize}

\end{itemize}

Genome sequencing identifies major causes of severe intellectual disability [Gilissen and Veltman, Nature, 2014]
\begin{itemize}
	\item Comparison of diagnostic yield of ID: array, 12\%; WES, 27\%; WGS, 42\%. 
	\item Data: WGS of 50 patients with severe ID and their unaffected parents, an average genome-wide coverage of 80 fold. 
	\item De novo mutation rates: 82 high-confidence potential de novo SNVs per genome. A protein-coding de novo substitution rate of 1.58, higher than all previous estimate, and a total of 84 in coding regions.  
	\begin{itemize}
		\item Enrichment of LoF mutations among 84 de novo coding mutations: $p = 1.6 \times 10^{-5}$. 
		\item Enrichment of de novo coding mutations in ID gene sets: 528 known ID genes (from HGMD and other sources) and 628 candidate ID genes. 9 genes, $p = 0.04$. 
	\end{itemize}
	\item De novo CNVs: 8 were detected, 4 of which in known ID or candidate ID genes (significant). 
	\item De novo non-coding mutations: 43 in promoter regions (1), introns (38), splice site (1) or untranslated regions (3) of all known ID genes. Found no potential pathogenic mutations using ENCODE annotations (only one promoter has ENCODE annotation). 
	\begin{itemize}
		\item Effect on splicing was determined using Alamut software that integrates a number of prediction methods for splice signal detection as well as exonic splicing enhancer (ESE) binding site detection.
		\item ENCODE annotation was based on Chromatin state segments of nine human cell types (Broad ChromHMM) and transcription factor binding sites (Txn Factor ChIP). 
	\end{itemize}
	\item Inherited variants: a single proband with compound heterozyous deletions affecting the VPS13B gene. 
	\item Lesson: 
	\begin{itemize}
		\item Measure the value of WGS by \textbf{diagnostic yield} (the percent of patients whose genetic causes can be determined). 
		\item Mutations in introns could affect splicing. 
	\end{itemize}
\end{itemize}

De novo mutations in schizophrenia implicate synaptic networks [Fromer \& O’Donovan, Nature, 2014]
\begin{itemize}
	
	\item Data: 617 Scz trios. 
	
	\item De novo mutation rates: 
	\begin{itemize}
		\item Overall, the de novo rates in LoF, missense are not higher than controls (731 controls from published data sets)
		\item Loss-of-function de novo mutations were more common in patients with relatively poor school performance ($p = 0.018$). Note: psychiatrists were explicitly instructed to exclude people with known intellectual disability. 
	\end{itemize}
	
	\item Recurrent genes: 18 recurrent genes (both missense and LoF): more than expected, $p = 0.03$. A single double-hit (LoF) gene: TAF13. 
	
	\item Inherited variants: 
	\begin{itemize}
		\item Transmission: excess of transmission of nonsyn. singletons in de novo genes, $p = 0.01$. 
		\item Case-control: increased case-control ratio of rare (MAF $<0.1\%$) LoF in de novo genes, $p = 0.003$. 
	\end{itemize}
	
	\item Gene set enrichment: 
	\begin{itemize}
		\item Enrichment of de novo mutations in candidate gene sets: ARC, NMDAR, FMRP targets. 
		\item GO set enrichment: a single set, assembly of actin filament bundles, is significant. 
	\end{itemize}
\end{itemize}

A polygenic burden of rare disruptive mutations in schizophrenia [Purcell \& Sklar, Nature, 2014]
\begin{itemize}
	\item Data: WES of 2,536 schizophrenia cases and 2,543 controls. Focusing on 2,500 genes implicated by unbiased, large-scale genome-wide screens, including GWAS, CNV and de novo SNV studies. 
	
	\item QC procedures: 
	\begin{itemize}
		\item Filter 11 subjects with low-quality data along with likely spurious sites and genotypes. Per individual, 93\% of targeted bases were covered at $\geq 10$-fold (81\% at $\geq$ 30-fold).
		\item Cases and controls had similar technical sequencing metrics, including total coverage, proportion of deeply covered targets, and overall proportion of non-reference alleles. 
		\item 635,944 coding and splice-site passing variants of which 56\% were singletons. High specificity and sensivity were verified. 
	\end{itemize}
	 
	\item Individual variant analysis: a known common SNP. Gene-best test: SKAT results no significant gene; genic burden test with LoF variants, one gene with 10 in cases and 0 in control, however, not significant ($p = 1.7 \times 10^{-3}$). 
	
	\item Define variant groups: (1) by function: LoF, strict damaging missense (predicted by five algorithms), broad damaging missense (by only one algorithm). (2) By AF: private, rare ($<.1\%$) and up to $.5\%$. Total of 9 groups. 
	
	\item Rare variant burden analysis: in 2,500 genes (Table 1). 
	\begin{itemize}
		\item LoF variants: (1) rare: 1,547 in cases vs 1,383 in controls, OR = 1.12, $P = 10^{-4}$. (2) Singletons: enrichment, $P = 8E-4$. (3) Up to $.5\%$: $P = 2E-4$. 
		\item Missense variants: enrichment of strictly defined missense variants $P = 1.5 \times 10^{-3}$, but not broadly defined ones.    
	\end{itemize}
	
	\item Gene set burden analysis: consider 12 sets from de novo, GWAS and CNV studies. Focus on LoF variants at 3 different thresholds. Eight out of 12 sets were nominally significant. Three smaller sets (synatpic genes) have OR $> 5$. 
	
	\item Comparison with autism/ID genes: e.g. de novo LoF genes in ASD. Found no enrichment in these gene sets.
	
	\item Refined burden analysis on 1,796 genes comprising all members of the most prominently enriched sets (Figure 1). 
	\begin{itemize}
		\item LoF: signal largely driven by novel/singleton variants. Signal is stronger in high expression genes. 
		\item NS variants: strict definition clearly better than PPH. Among all annotations, PPH and SIFT seem to perform better. 
	\end{itemize} 
\end{itemize}

Noncoding Variation in Schizophrenia [Roussos \& Sklar, Cell Reports, 2014]
\begin{itemize}
	\item Data:
	\begin{itemize}
		\item Brain eQTL: combine 8 published datasets (Table S1). 
		\item Brain CREs: K3K27ac, H3K4me1, DHS from ENCODE and Roadmap, including adult brain, fetal brain, primary cell culture and iPS. Use all datasets to define five types of CREs: active promoter; active enhancer; poised promoter; repressed enhancer; and open chromatin regions.
	\end{itemize}
	\item GWAS SNPs and functional annotations:
	\begin{itemize}
		\item Choose all GWAS SNPs at $p < 10^{-3}$ (42K SNPs), 37\% are eSNPs. Among this, 4.9\% were in active promoters, 9.6\% in active enhancers (H3K4me1 and [H3K9ac or H3K27ac]), 3.5\% in DHSs, 1.0\% in poised promoters, and 1.5\% in repressed enhancers. 
		\item Enrichment: highest in eSNPs (3.68 fold), active enhancers (2.30) and active promoters (2.13). However, active enhancers and promoters together cover about 10\% of 42K GWAS SNPs at $p < 10^{-3}$. 
		\item cre-SNPs: eSNPs in cis-regulatory elements (CREs). At $p < 10^{-3}$, we have about 2,000 creSNPs;  At $p < 10^{-5}$, about 400 creSNPs. Enrichment is strongest in creSNPs: about 5 fold at $p < 10^{-3}$, and 29 fold at $p < 10^{-5}$.  
	\end{itemize}
	\item Finding causal loci in GWAS loci: 
	\begin{itemize}
		\item 22 significant SNPs. 10 out of 22 overlap with eSNP, and have high RTC scores (test if eSNP and GWAS SNP tag the same causal variant). 
		\item Example: SNP in CACNA1C region, also its eSNP. Find 4 SNPs in perfect LD with the GWAS SNP, and lie within two predicted enhancers. Confirm the interaction of enhancer and promoter through 3C in human dorsolateral prefrontal cortex and iPS-derived neuron. 
	\end{itemize}
	\item Lessons: High fraction of GWAS SNPs in eQTL, much higher proportion than in CREs. Also active enhancers (H3K27ac) appear to cover more functional sequences than DHS. 
\end{itemize}

Schizophrenia genetics complements its mechanistic understanding on Sekar et al, Nature, 2016 [NN, 2016]
\begin{itemize}
	\item GWAS evidence of C4: (1) strongest signal in MHC near C4; (2) another hit in CSMD1, which regulates C4. 
	
	\item C4 expression model: CNV in C4, associated with C4 expression. Also influenced by genotype. Build a model to predict C4 expression from genotype and CNVs.  
	
	\item Correlation of predicted C4 expression and SCZ risk. 
	
	\item Functional evidence of C4: (1) Expression elevated in SCZ brain. (2) Complement system: critical for synatpic pruning. 
\end{itemize}

Rare loss-of-function variants in SETD1A are associated with schizophrenia and developmental disorders [Singh \& Barrett, NN, 2016]
\begin{itemize}
	\item Study design: (1) Case-control: 1,000 cases in UK10K and 2500 from published Swedish study. (2) Trios: 1000 from 7 published studies. 
	
	\item Enrichment of DNMs: no burden in missense mutations, only in LoF. General burden trend: SCZ $<$ ASD $<$ DD. 
	
	\item Case-control: (1) Use LoF, LoF + damaging missense at different AF, run burden test and SKAT. No signal. (2) Found enrichment in rare LoF. 
	
	\item Combining DN and case-control, using only LoF mutations. TADA or Fisher' test. Single gene: SETD1A. 
	
	\item Biology of SETD1A and other evidence: LoF clustered in H3K4methylation domain. Top 3\% constrained gene in ExAC. LoF in DDD. 
\end{itemize}

Sparse whole-genome sequencing identifies two loci for major depressive disorder. CONVERGE, Nature, 2016
\begin{itemize}
	
	\item Background: no loci of MDD found previously (9k cases). Likely due to heterogeneity of the genetics.
	
	\item Data: 5000, low coverage WGS (1.7x), Chinese women and 5k controls. 6M SNPs.
	
	\item Reducing genetic heterogeneity: Chinese women, only recurrent cases (more severe). Known risk factors recorded. In China, the MDD cases tend to be more severe (reluctance of reporting).
	
	\item Two loci reaching genome-wide significance: SIRT1 and LHPP. Replicated in 3k samples.
	
	\item Comparison with PGC results: the top SNPs not replicated, but direction yes. PGC polygenic risk score is significantly associated, $p < 0.01$, but explains only 0.1\% of MDD risk. The most strongly associated loci have low AF in European: $>40$ in CONVERGE vs. 3 and 8\% in European.
\end{itemize}

Comprehensive integrative analyses identify ALMS1, GLT8D1 and CSNK2B as schizophrenia risk genes [Yang \& Luo, review for Biological Psychiatry, 2017]
\begin{itemize}
	\item Sherlock analysis of brain eQTL (Meyers, NG, 2007) and PGC GWAS: 10 risk genes at Bonferroni threshold, 6 cis and 4 trans. 
	
	\item Expression patterns: most of these genes are expressed in brain (not surprising, they are found by brain eQTL); and most (6) genes have higher expression in early developmental stages. 
	
	\item Network analysis: DAPPLE analysis on PPI network (GeneMania) found modest evidence of ALMS1 and CSNK2B. Co-expression network: some genes are co-expressed with known SCZ risk genes. 
	
	\item Additional evidence on three risk genes: dysregulation in SCZ, independent brain eQTL, association with hipocampal structure (ENIGMA data) and cognitive traits. 
	
	\item Experimental validation of GLT8D1 using NSC: promotes the self-renew and proliferation abilities of neural stem cells (NSCs) and inhibits NSCs differentiation. 
	
	\item \textbf{Lesson}: It is OK to find genes using computational tool, then focus on several candidate genes and support them with multiple lines of evidence - we do not have to show that every new predicted gene is real. 
\end{itemize}

Mutations in Human Accelerated Regions Disrupt Cognition and Social Behavior [Doan and Walsh, Cell, 2016]
\begin{itemize}
	\item Evidence that HARs are functional: (1) enrichment in CTCF binding. (2) Epigenomic marks suggest that 29\% act as enhancers in brain, heart and limb development. (3) HAR regions show enrichment of loci associated with SCZ.
	
	\item HARs are depleted of variants in human population.
	
	\item HARs are enriched of neural CREs: using Roadmap data.
	
	\item Many HARs act as regulatory elements of dosage-sensitive neural genes.
	
	\item Connection between HARs and ASD: de novo CNVs, point mutations in the HARs are enriched in ASD probands vs. controls.
	
	\item Lesson: to demonstrate functional connection with cognition and social behavior (1) regulation of neuronal genes; (2) association or enrichment with neuropsychiatric phenotypes.
\end{itemize}

Model of complex social behavior (vocal learning) using songbird or zebra finch [Somayeh Ahmadiantehrani from Sarah London lab, 2017]
\begin{itemize}
	\item Background: 50\% of monogenic ASD cases result in the disruption of mTOR cascade, e.g. PTEN and TSC2.
	
	\item Background: mTOR protein complex: ser/thr kinase. Control translation, lipid biosynthesis, autophage, etc. mTOR disruption in brain: blocks LTP and learning/memory deficits.
	
	\item Background: Some known brain region are associated with song learning.
	
	\item Experiment: isolated bird, then song from the speaker. Measure phos-S6 (downstream effector) in CMM, NCM.
	
	\item Treatments with drugs disrupting mTOR pathway reduces the similarity of junavile song vs. tutor. This is only effective when treating before tutoring experience.
	
	\item In vivo eletroporation (instead of virus) to change gene expression in bird brains. Proposal: PTEN knockdown.
	
	\item Q: Why treatment of drugs during development has no effect?
	
	\item Q: Song learning and language learning: evolutionary conserved?
\end{itemize}

De novo mutations in regulatory elements in neurodevelopmental disorders [Short and Hurles, Nature, 2018]
\begin{itemize}
	\item Background: Targeted sequencing: (1) Capture arrays; (2) PCR on a set of primers.
	
	\item Targeted sequencing of 4.2Mb noncoding sequences: conserved sequences (4000), heart enhancers, VISTA enhancers (600). Sample size: 8000, majority do not carry a diagnostic variant in a protein coding gene (exome-negative). 
	
	\item Quantifying constraint in human population: mutability (tri-nucleotide model) adjusted proportion of singletons (MAPS, from ExAC paper [Lek, Nature, 2017]). Estimated from 8000 unaffected DDD parents. 
	
	\item Pattern of negative selection: (1) Heart enhancers: very little conservation (across species) and not constrained in human population. (2) VISTA (experimental) enhancers: large range of cross-species conservation, constrained in human (similar to exonic sequences). (3) Comparison of DHS constraint (in human): most constrained are fetal brain, HSC, ESC-derived neurons. But difference across tissues not statistically significant. 
	
	\item Burden analysis: using mutation models. (1) Small burden in conserved sequences, $p =0.04$, no burden in heart or VISTA enhancers. (2) Conserved elements + fetal brain chromHMM, $p < 1E-3$, 238 vs. 194 (expected). Conserved + fetal brain DHS, $p = 0.002$. Even higher burden in patients with neurodevelopmental phenotypes. (3) Control: no burden in same elements in patients without ND phenotypes.  
	
	\item No enrichment in enhancers of known DD genes or pLI genes. To assign enhancers to genes: fetal Hi-C, DHS expression correlation. Remark: likely due to small DNM counts. 
	
	\item Motif analysis: 45 TFs whose motifs are enriched in DNMs (DNMs increase the affinity).
	
	\item Recurrently mutated elements: limit to fetal brain active CNEs and evolutionarily conserved enhancers. The number (30) is twice higher than expected. Create clusters of fetal brain CNEs, but found no enrichment of individual clusters. 
	
	\item The proportion of highly penetrant mutations is small: simulation of highly penetrant mutations (RR = 120), and assess the number of recurrent elements that are genome-wide significant. Since we observe 0, the proportion of highly penetrant mutations must be low.
\end{itemize}

Genome-wide association studies of brain imaging phenotypes in UK Biobank [Elliott and Smith, Nature, 2018]
\begin{itemize}
	\item Background: (1) structure MRI: structural volumess, e.g. hippocampus, and other biomarkers such as microbleeds, white matter microstucture. (2) Diffusion MRI: brain structural connectivity. (3) Functional MRI: resting state activities and functional connectivities.
	
	\item Data: 3000 Image-derived phenotypes (IDPs) in 8K individuals from UKBB.
	
	\item Heritability: about half show significant heritability, mostly in 0.1-0.5. Resting state fMRI features show lowest heritability.
	
	\item Significant associations: at stringent thresholds, about 300 associations. Some associations across multiple phenotypes. (1) Association of genes involved in iron deposition, associated with some dMRI features. (2) SNP in iron channel associated with grey matter volume effects across the brain. The SNP is also associated with IQ, blood pressure and SCZ. (3) EM and EGF signaling associated with dMRI IDPs.
	
	\item Genetic correlations: suggestive correlations with ALS, SCZ and stroke with dMRI features in white matter tracts.
\end{itemize}

Risk loci for ADHD [NRG, 2018]
\begin{itemize}
	\item 12 loci for ADHD, including FOXP2 for development, DUSP6 regulating dopamine levels. 
	
	\item Genetic correlation: positive with MDD, neuroticism, and negative with intelligence and education.
\end{itemize}


The genetic and epigenetic basis of pediatric epilepsy [Gemma Carvill seminar, 2019]
\begin{itemize}
	\item Epilepsy: recurrent and unprovoked seizures. About 1/26, wide spectrum.
	
	\item I. CHD2 mutation patterns in patiens vs. gnomAD: controls no LoF, and depletion in ATPase and helicase domains.
	
	\item Mouse phenotypes: het. CHD2, learning deficits, leions on multiples organs (no brain). No seizure.
	
	\item CHD2 binding: enrichment with promoters of Epilepis genes. half are pormoters.
	
	\item CHD2 phenotype in NPC: 1000 up-regulated genes, NPC diff. much faster. Targets: axongenessis (FE = 7), other migration. However, no DEs in neurons.
	
	\item Conclusion: premature neuronal differentiation. Neurons fire more ferquently. Organoid model: comparison of w.t. and hets.
	
	\item Q: if chr. remoderler, deletion of the gene should lead to down-regulation, but this is opposite to what is observed. Why?
	
	\item II. Dravet syndrome (form of DEE). 90% have DNM in SCN1A.
	
	\item SCN1A: poison exon in an intron. Expression in astrocyte (to degrade SCN1A).
	
	\item Mechanism: intronic variants disrupts SRSF1 binding sites. Increased inclusion  of poison exon. Motif: GGAGGA.
	
	\item Poison exons are presents at other epi. genes.
	
	\item Use ASO (anti-sense oligo) to target poison exon: stop poison exons to increase expression of genes.
	
	\item Lesson: chromatin genes may change the development/cellular phenotype of cells.
\end{itemize}

Genome-wide association analysis of Parkinson’s disease and schizophrenia reveals shared genetic architecture and identifies novel risk loci [review for Biological Psychiatry, 2019]
\begin{itemize}
	\item Background: it is possible that two traits share some loci, but because some of them have same direction, some opposite, overall, the genetic correlation is 0.
	
	\item CondFDR analysis of SCZ and PD: increase the power if discovery. 9 distinct loci jointly associated with both SCZ and PD. Five of them have consistent effect directions.
	
	\item Validation of shared loci: nearest genes. Some show highest expression in astrocytes, endothelia cells and neurons. PPIs are more connected in astrocytes than in neurons.
\end{itemize}

Genetics of Verbal communication [Mellissa DeMille]
\begin{itemize}
	\item World-wide language: by phonemes or specifically number of consonants. Ex. two African: 18 vs. 46.
	
	\item Remark: fitness of phonemes is possible, e.g. separation of t vs. d.
	
	\item Three aspects of languages: Speak, hearing and processing/phonological processing (convert sounds to meaning). Genetics: FOXP2, hearing/deafness gene, DCDC2 (associated with reading disability). DCDC2: localized in relevant brain regions.
	
	\item Consonants: precision of AP firing. Vowels: number of AP firing. DCDC2 K/O mice: poor precision of AP.
	
	\item READ1: CRE of DCDC2. Repeat alleles, complex.
	
	\item READ1 variation (RU1.1) correlates with number of consonants across many populations. Use PCs to adjust for relatedness.
	
	\item Future directions: polygenic selection? When evolved?
	
	\item Remark: is LMM sufficient to address for confounding - genetics is now confounded with population. Populations also differ in culture.
	
	\item Q: genetic sharing with education attainment? Reading/language genes.
	
	\item \textbf{Lesson}: genetic basis of language is hard, however, we can break it down in different parts, from vocalization to hearing. For each component, it is easier to relate molecular functions of genes to phenotypes, e.g. detection of sound patterns.
\end{itemize}
%%%%%%%%%%%%%%%%%%%%%%%%%%%%%%%%%%%%%%%%%%%%%%%%%%%%%%%%%%%%
\subsection{Autism}

Biology of autism: 
\begin{itemize}
	\item Biological basis of brain region specialization: two hypothesis: 
	\begin{itemize}
		\item Mainly reflect the difference of neuron behavior or history, i.e. the neurons in different regions are largely interchangable, but they have different history. Evidence: most genes do not show significant difference in expression across regions in neurocortex.  
		\item Different expression programs are active in different regions. Evidence: neurotransmitters (and corresponding signaling pathways) that are specific to functions, e.g. oxytocin and vasopressin for social behavior. 
	\end{itemize}
	
	\item Focal vs. diffusive disruption: does autism involve disruption of specific brain circuits? Yes to some extent (frontal lobe, temporal lobe, etc.). But then how to explain that the mutaiton of some broadly expressed genes can lead to disruption in specific areas? 
\end{itemize}

Review of Autism-related genes [Holt \& Monaco, Links between genetics and pathophysiology in the autism spectrum disorders, EMBO Mol Med, 2011]: 
\begin{itemize}
	\item Candidate gene studies: approximately one hundred are reported as showing at least nominal association with ASD. 
	\begin{itemize}
		\item The serotonin transporter gene SLC6A4: serotonin levels have been shown to be altered in autism cohorts. Multiple studies have also identified association to variants in this gene. 
		
		\item An important example of linkage data being utilized is CNTNAP2, with further evidence from a subsequent GWAS. CNTNAP2 is a neurexin, and there is strong evidence that this gene family is influential in ASD. It is expressed in brain regions related to ASD, and there is evidence from Copy number variations (CNVs) identified in schizophrenia. Brain changes in patients with CNTANP2 mutation, suggesting underconnectivity. 
		
		\item SNPs in promoters and intron of MET: affect TF binding (SP1 and other TF).  MET is also associated with schizophrenia, and again, the associated SNPs appear to affect gene expression
		
		\item Four LRR (leucine rich repeat) genes are enriched in the brain and two show significant associations with autism, LRRN3 and LRRTM3. There are 313 members in this class of genes, some of which have been implicated by GWAS, but do not reach genome-wide significance thresholds. 
	\end{itemize}
	
	\item GWAS:
	\begin{itemize}
		\item Wang et al found strong association to a locus between Cadherin 9 (CDH9) and Cadherin 10 (CDH10). Both genes encode neuronal cell adhesion molecules, and CDH10 is expressed in the frontal cortex, a region of the brain associated with ASDs. The SNP is likely influence gene regulation. 
		\item Weiss et al found significant association to a SNP between SEMA5A and TAS2R1. SEMA5A has been implicated in axonal guidance and is expressed at lower levels in cell lines and brains from individuals with ASD. The associated SNPs were 80kb upstream of SEMA5A, consistent with a role in gene regulation 
	\end{itemize}
	
	\item CNVs: 
	\begin{itemize}
		\item Sebat et al demonstrated a significant difference in frequence of de novo CNVs between sporadic cases (10\%), familial cases (3\%) and controls (1\%)
		\item Pinto et al: PTCHD1, likely to be involved in development of the cerebellum, and SHANK2 (Noor et al, 2010; Pinto et al, 2010). SHANK2 is related to SHANK3 (Durand et al, 2007; Moessner et al, 2007), which encodes a scaffolding protein located at synapses in the brain, since implicated in other studies
	\end{itemize}
	
	\item Sequencing and point mutations: in SHANK3, Durand et al identified mutations in individuals with ASD including a single base insertion resulting in a frameshift, rare non-coding mutations not present in controls, as well as CNVs of potential significance. 
	
	\item Additional pathways implicated in ASD: e.g. genes involved in synthesis and degradation of ketone bodies. The latter could affect $\gamma$-aminobutyric acid (GABA) levels. 
\end{itemize}

Focal vs diffuse circuit disruption in ASD [Autism: Many Genes, Common Pathways? Geschwind, Cell, 2008]:
\begin{itemize}
	\item Challenge: ASD susceptibility genes must converge on the disruption of function in brain regions supporting language, social cognition, and behavioral flexibility. What kind of expression pattern underlies the relative specificity in brain regions? 
	
	\item Scenarior 1: focal gene expression of the specific gene product during development; when the risk allele is expressed, there is disruption of the cortical and subcortical brain networks supporting social responsiveness or language. Example: CNTNAP2, enriched in anterior regions of the developing cerebral cortex that overlap with circuitry involved in the development of joint attention (related to language)
	
	\item Scenarior 2: most known ASD susceptibility genes do not demonstrate regionally restricted expression. The core areas affected in autism involve integration of information from multiple, higher-level areas. Such functions could be easily perturbed by minor, but widespread disruptions in neural transmission. 
	
	\item Evidence of Scenarior 2: one would expect to find subtle, widespread differences in many brain systems, many of which may not be the direct cause of the core features of autism. Such abnormalities, may explain the differences in sensory processing, motor function, and sensory-motor integration, etc. that have been variably associated with ASD. 
	
	\item Concepts of focal versus diffuse circuit disruption are not mutually exclusive and both may cause different forms of ASD.
\end{itemize}

Brain regions affected by ASD [Connecting genes to brain in the autism spectrum disorders. Arch Neurol, 2010]: 
\begin{itemize}
	\item Background: ASDs can be conceptualized in terms of multiple genetic etiologies that disrupt the development and function of brain circuits mediating social cognition and language. 
	
	\item Lessons from autism-related Mendelian/genetic diseases: 
	\begin{itemize}
		\item Fragile X syndrome: accelerated early head growth. Structural imaging found abnormalities in the caudate (increased), lateral ventricles (increased), and posterior vermis of the cerebellum (reduced). Functional MRI: involvement of frontostriatal circuitry (connecting frontal lobe regions with the basal ganglia). 
		\item Rett syndrome: cerebellar atrophy. Similarly, frontal and temporal cortices, the caudate are subject to the greatest regional reductions in gray matter volume.
		\item Mutation in CNTNAP2: cortical dysplasia and abnormalities in neuronal migration provide support for an early developmental insult, particularly in the frontal and temporal neocortex.
		
		\item Tuberous sclerosis: a comparison of IQ-matched patients with tuberous sclerosis with and without a diagnosis of autism identified altered energy metabolism in temporal neocortex, caudate, and cerebellum among ASD cases. 
		\item Together, these results suggest that these different monogenic risk factors for autism share a common involvement of frontal and temporal neocortex, caudate, and  cerebellum. 
	\end{itemize}
	
	\item Lessons from idiopathic ASD: accelerated postnatal growth in defined brain regions and preferential involvement of specific regions, including frontal and temporal cortex, cerebellum, and amygdala. 
	
	\item Neuropathological findings in ASD: 
	\begin{itemize}
		\item The most consistent neuropathological finding among the ASDs is the observation of errors in neuronal migration, particularly in frontal and temporal lobes. 
		\item Minicolumns: distance is reduced in the frontal and temporal cortex of individuals with a spectrum condition. Minicolumn findings are predominant in the frontal and temporal cortex, consistent with expression patterns for known ASD genes CNTNAP2 and MET. 
		\item The amygdala: involved in the modulation of social behavior, has long been implicated in the ASDs. Neurons in the amygdala were abnormally small and showed elevated packing density. 
	\end{itemize}
	
	\item Conclusion: connectivity between frontal, temporal, and additional interconnected regions mediating language and social behavior is critical to understanding the ASDs
\end{itemize}

Underconnectivity hypothesis [Links between genetics and pathophysiology in the autism spectrum disorders, Holt \& Monaco, EMBO Mol Med, 2011]: 
\begin{itemize}
	\item Brain changes of ASD patients in a gross scale: localized failure of cerebellar development, cerebral cortical abnormalities, altered amygdala development and decreased corpus callosum size. Excessive growth of white matter in the first 2 years of life, followed by undergrowth, leading to macrocephaly in 20\% of children with ASD. 
	
	\item Brain changes of ASD patients in a fine scale: within minicolumns, neurons smaller in size, but of increased density, and their organization within minicolumns being altered. Minicolumns are of decreased width and increased number, resulting in the increase in density of neurons reported. Changes in minicolumns and neuron size may promote shorter connecting fibres, increasing local connectivity at the expense of connections between different cortical regions. 
	
	\item Underconnectivity theory: fewer long range connections, leading to decreased synchronization between regions of the brain and decreased global information processing. 
	
	\item Underconnectivity is particularly likely to affect connections between frontal and temporal lobe regions, and what connectivity there is, is likely to be unorganized. Neuroimaging studies have demonstrated a lack of synchronization and decreased communication and connectivity between them. 
	
	\item Additional evidence of underconnectivity: individuals with ASD show reduced capacity for joint attention. Joint attention utilizes the prefrontal and anterior regions of the brain.
	
	\item Genetic evidence: genes involved in neuronal development, migration, growth and maturation, axonal growth, synapse and synaptic complexity have been implicated. 
	
\end{itemize}

Genes, circuits, and precision therapies for autism and related neurodevelopmental disorders [Sahin \& Sur, Science, 2015]
\begin{itemize}
	\item Genetics of autism:
	\begin{itemize}
		\item Approaches for mapping autism genes: recessive inherited variants in consauguinity families; families with high risk among women. 
		
		\item Environmental factors that may affect autism: maternal infection (immune activation), perinatal injury: premature infants with cerbellar hemorrhage have 30-fold higher risk, GI symptoms (microbiome). 
	\end{itemize}
	
	\item Molecular pathways: 
	\begin{itemize}
		\item Synapse: calcium channel (SCN2A), GluR and scaffolding proteins (SHANK, SYNGAP1, PSD95). 
		
		\item Transcriptional and translational regulation: MeCP2, FMRP. 
		
		\item Signaling pathways: (1) PI3K/mTOR pathway: TSC1, TSC2, PTEN, IGF1 (for treatment). (2) Ras-MAPK pathway. 
	\end{itemize}
	
	\item Neural circuitry: challenge is to identify cell types, regions, circuits critical for autism. 
	\begin{itemize}
		\item Approaches: conditional knockout, imaging studies (often difficult with autism patients). 
		
		\item Cortical projection neurons (extending axons to distant targets) from human co-expression networks. 
		
		\item Neuron types from mouse model: e.g. loss of certain neuronal types from deletion of ASD genes. GABAergic neurons, interneurons, inhibitory neurons. 
		
		\item Regions: basal ganglia, cerebellum, non-neuronal cells such as astrocytes and microglia. 
		
		\item A gene may be expressed in different regions, and control multiple behaviors (Figure 2), e.g. cortex: soical interaction; cerebellum: repetitive behavior. 
		
		\item Two subtypes of epilepsy: each involving a different neuronal types, and responding to different treatments. 
	\end{itemize}
	
	\item Treatment of autism: 
	\begin{itemize}
		\item Mechanism-based: e.g. mGluR antagonist (FXS). 
		\item Biomarkers for subtypes: e.g. biochemical measure, functional feature from imaging. 
		\item Preclinical models: e.g. iPS-neurons. 
	\end{itemize}
\end{itemize}

Getting to the core of autism [Iakoucheva and Sebat, Cell, 2019]
\begin{itemize}
	\item Autism genes that are DBPs and RBPs: bind to other ASD genes, e.g. TBR1, CHD8, FMRP.
	
	\item Trans- effects of ASD genes: CHD8 and FOXP1 in iPSCs, strongest targets are trans, but not direct targets. Mouse models of developing brain: SETD5, TBX1 and FOXP1.
	
	\item DE analysis from postmortum brains of patients: upregulation of immune-microglia and mitochondrial modules, and downregulation of neuronal and synaptic modules in ASD and schizophrenia.
	
	\item Core ASD genes? May be hard to define, because traits do not emerge from a small number of genes. Master TFs and key synaptic genes may be equally important/core.
	
	\item Cell proliferation phenotypes of ASD patients: macrocephaly for some genetic subtypes of ASD - e.g. CHD8, PTEN, whereas the opposite phenotype (microcephaly) is associated with others - e.g. DYRK1A. iPSC model: more proliferation, but fewer excitatory synapses and matured into defective neuronal networks with less bursting.
	
	\item Dendritic Arborization and Synapse Number: MECP2 mutant reductions in neurite outgrowth, dendritic arborization, and excitatory synapses.
	
	\item Change of Electrophysiological behavior of neurons: e.g. SHANK2, KCNQ2.
	
	\item Reverse-genetic approach: starting with the genotype and determining how genes influence clinical phenotypes.
	
	\item Lesson: ASD genes form regulatory networks. Measurable cellular phenotypes of ASD gene mutations: proliferation, synapse numbers and morphology, electrophysiology.
\end{itemize}

A genome-wide scan for common alleles affecting risk for autism [Anney, Hum Mol Genetics, 2010]:
\begin{itemize}
	\item Background: only rare de novo and inherited variants are soundly established genetic risk factors for ASD, and thus far these only account for a small proportion of the total genetic risk. In contrast, common variants rarely have such an impact on risk for any disorder, especially one like ASD that is known to diminish reproductive success.
	
	\item GWAS method: 1558 ASD families (4712 subjects). A priori we planned and conducted four nonindependent GWA analyses corresponding to data partitions along axes of diagnosis and ancestry: spectrum versus strict and European versus all ancestries. 
	
	\item Largest associations: in a 300 kb intronic region of MACROD2, $P = 2.1E-8$, (below the threshold of Bonferroni correction). Recent genome-wide studies have highlighted copy number variation at MACROD2 in an individual with schizophrenia (27), brain infarct (28) and brain volume in multiple sclerosis. Also $\tilde 500$ kb from the association signal is FLRT3, a cell adhesion molecule with functions in neuronal development.
	
	\item Replication: under independent ASD families from the Autism Genetics Resource Exchange (AGRE) database. Most genes are not replicated, e.g. for MACROD2, its $P = 0.13$ in AGRE, and $4.7E-8$ in AGP and AGRE. 
	
	\item Exploratory analysis: using traits such as verbal and IQ. We do observe signals that are close to the threshold ($P < 1E-7$) in the discovery sample in PLD5, POU6F2 and an intergenic region on 8p21.3. 
	
	\item Discussion: unbiased estimates of odds ratios detected by GWA studies are typically in the range of $1.1-1.3$ to have good power to detect such effect sizes requires many thousands of samples. 
\end{itemize}

Functional impact of global rare copy number variations in autism spectrum disorders [Pinto, Nature, 2010]: 
\begin{itemize}
	\item Background: 
	\begin{itemize}
		\item Although ASDs are known to be highly heritable ($\tilde 90\%$), the underlying genetic determinants are still largely unknown. 
		\item CNV examples include de novo events observed in 5-10\% of ASD cases. 
	\end{itemize}
	
	\item Data: 1,275 ASD cases and their parents using the Illumina Infinium 1M single SNP microarray. 1,981 controls. The array contains a total of 1,072,820 markers (50-mer probes) for SNP and CNV analyses. 
	
	\item Defining the CNVs: (1) CNV present at $<1\%$ frequency in the total sample (cases and controls); (2) CNV $\geq 30$kb in size (because $>95\%$ of these could be confirmed). This stringent data set of 5,478 rare CNVs in 996 cases and 1,287 controls of European ancestry. Among these CNVs at least 5.6\% (49/876) of trio families carried at least sone de novo CNV (average of 1.1 verified de novo CNVs/sample). 
	
	\item Association test using CNVs: 
	\begin{itemize}
		\item Measures for CNV burden analysis: (1) CNV rate: by the number of CNVs per sample and the proportion of samples with one or more CNVs; (2) CNV size was assessed as both the total genomic segment covered by CNVs, as well as the average CNV size; (3) Gene-count: the average number of genes intersected by CNVs.  
		\item CNV burden analysis: compare the three measures of CNV burden in cases vs. controls (i.e., hypothesizing that cases will show greater burden of rare CNVs than controls). Permutation procedure for statistical significance of one-sided tests.
	\end{itemize}
	
	\item Results of CNV-based gene association test: examples of novel ASD loci include SHANK2, SYNGAP1, and DLGAP2 based on the observation that de novo CNV affects these genes in cases but not controls. Also, a combination of rare de novo and inherited CNVs affecting NRXN1, IL1RAPL1, DMD, and the DiGeorge 22q11.2 region in ASD. 
	
	\item Results of CNVR burden analysis: CNVR defined by merging overlapping CNVs. CNVR at DDX53/PTCHD1 emerged as a significant ASD risk factor. Specifically, we observed 7 ASD male cases with overlapping deletions at DDX53/PTCHD1 (Xp22.1) and no CNVs were observed at this locus for the initial 1,287 controls
	
	\item Analysis of ASD candidate genes: 
	\begin{itemize}
		\item Curated list of ASD candidate genes: (1) ASD-implicated: 36 genes and 10 loci strongly implicated in ASD; (2) Intellectual disability (ID): 110 genes and 17 loci know to be implicated in ID but not yet in ASD; (3) ASD candidates: 103 genes drawn from previous studies of common and rare variants for ASD. They include case reports of cytogenetic abnormalities, allelic association and CNV studies. 
		
		\item A higher proportion of cases with rare CNVs overlapping ASD implicated disease genes compared to controls, corresponding to a significant enrichment for genes in this set. 
	\end{itemize}
	
	\item Gene set analysis:
	\begin{itemize}
		\item Fisher's exact test to assess which gene sets were more frequently affected by rare CNV events in ASD cases compared to controls.
		\item Results: 76 gene sets affected by deletions (2.18\% of sets tested) were found to be enriched and used to construct a functional map. Gene sets involved in cell and neuronal development and function (including projection, motility and proliferation) previously reported in ASD were identified. Novel gene sets include GTPase/Ras signalling, with component Rho GTPases known to be involved in regulating dendrite and spine plasticity and associated with intellectual disability. 
	\end{itemize}
	
	\item Questions/remarks: 
	\begin{itemize}
		\item SNP markers of CNVs: cannot detect multi-allelic CNVs. 
		\item Testing association of rare CNVs: burden test is essentially pooling/collapsing of all (rare) CNVs within a gene. 
	\end{itemize}
\end{itemize}

Etiological heterogeneity in autism spectrum disorders: more than 100 genetic and genomic disorders and still counting [Betancur, Brain Res, 2011]:
\begin{itemize}
	\item Background: 
	\begin{itemize}
		\item Over 70\% of individuals with autism have intellectual disability (ID), while epilepsy occurs in about 25\%. 
		\item ASDs are identified in about 1\% of children (Baird et al., 2006) and are four times more common in males than in females.
		\item About 10\% of individuals with an ASD have an identified genetic etiology.
		\item The genetic architecture of autism resembles that of ID, with many genetic and genomic disorders involved, each accounting for a small fraction of cases.
	\end{itemize}
	
	\item Methods: An extensive literature search about genetic disorders with autism, ASD, pervasive developmental disorder, Asperger syndrome, PDD-NOS, or autistic/autistic-like traits. Results from common variant studies were not included because of the absence of replications. 
	
	\item Identified more than 100 loci for which there is evidence for a causal role in ASDs. The majority have not been explored in ASD. 
	
	\item Conclusion: autism represents the final common pathway for numerous genetic brain disorders. It is also of interest to see that the genes implicated in ASD go beyond those involved in synaptic function and affect a wide range of cellular processes.
	
	\item Remark/questions: 
	\begin{itemize}
		\item For some of the genes in Table 1, only a single case with ASD/autistic features ($n=21$) or a single family with 2-3 males ($n=6$) were identified. How to confirm the gene with such a small sample?
	\end{itemize}
\end{itemize}

Autism [Talk by Kathryn Roeder at Lane Center meeting, Nov, 2011]: 
\begin{itemize}
	\item GWAS of autism: 
	\begin{itemize}
		\item With more than 30,000 individuals, find no SNPs with OR $> 1.05$. 
		\item Relation to schizophrenia: a much larger study identifies some SNPs, and if taking these SNPs and test in autism, some are significant. 
	\end{itemize}
	
	\item Copy number variants in autism: [Multiple Recurrent De Novo CNVs, Including Duplications of the 7q11.23 Williams Syndrome Region, Are Strongly Associated with Autism, Neuron, 2011]
	\begin{itemize}
		\item Simpsons simplex family: parents are non-autism, and one child is autistic (only one, or the other child non-autistic). 
		\item Association study of CNVs: map about 500 Simpson families. Focus on CNVs in children, but not in parents. Found some CNVs that greatly increase the autism risk, e.g. 4 copies of 7q.11 and 14 copies of 16p.11 [Multiple recurrent de novo CNVs, including duplications of the 7q11.23 Williams syndrome region, are strongly associated with autism. Neuron, 2011]
	\end{itemize}
	
	\item Case-constrol studies using exom sequencing data: no genes pass the threshold of $10^{-4}$, using a variety of gene-based tests. 
\end{itemize}

Multiple recurrent de novo CNVs, including duplications of the 7q11.23 Williams syndrome region, are strongly associated with autism. [Sanders \& State, Neuron, 2011]
\begin{itemize}
	
	\item Background: 
	\begin{itemize}
		\item Overrepresentation of large (mean size of 2.3 Mb) rare de novo events were more frequent in ASD probands in simplex families, compared to controls, or versus probands from multiplex families. 
		\item Across all studies, the burden of rare de novo CNVs in simplex probands (i.e., the percentage of individuals carrying =1 rare de novo event) has ranged from 5.0\% to 11\%. 
		\item Incomplete penetrance and diversity of phenotypic outcomes: 16p11.2 deletions or duplications have been found in individuals with ASD and intellectual disability (ID), seizure disorder, obesity, macrocephaly, and schizophrenia. 
	\end{itemize}
	
	\item Data: 4457 individuals from 1174 families, 872 quartets and 252 families trios. At a threshold of greater than 20 Illumina probes mapping within a genomic interval a combined total of 58 rare de novo CNVs were identified across the two studies (and Nimblegen). The sensitivity for small de novo events was low for both arrays. 
	
	\item Pattern of de novo CNVs in 872 probands vs. 872 siblings (Table 1): 
	\begin{itemize}
		\item Rate in siblings: 16 in 872 siblings, or 1.7\% (of persons containing at least one de novo CNV). 
		\item Rate in probands: 54 in 872 probands, or 5.8\%. 
		\item Deletions vs. duplications: about equally split in siblings, slightly more in probands. 
		\item Autosomal vs. X-chromosome CNVs: only 2 in chrX. 
		\item Size distribution: small ($<100$kb) - about 1 gene, medium ($100-1000$kb) - about 4 to 10 genes, and large ($>1$Mb) - about 10-25 genes. In siblings: 3:9:4, in probands: 5:26:23. 
		\item Single Occurrence De Novo CNVs: 14 in siblings, 37 in probands. Double occurrence: 2 in siblings, 8 in probands. More than two occurrence: 0 in siblings, 9 in probands. 
	\end{itemize}
	
	\item Rare recurrent de novo CNVs: 
	\begin{itemize}
		\item 23 probands carried recurrent de novo CNVs in six distinct regions of the genome. Each of these intervals contained from 2 to 11 de novo CNVs in unrelated probands. 
		\item Statistical sigificance: first estimate the number of possible CNV regions, using the ``unseen species problem'', $C = 242$. Next, estimate the probabililty of multiple hits in the same region (close to binom.test). Ex. $n = 4$, $p = 7e-6$. 
		\item Rare transmitted CNVs in probands: 8 overlap with one of the 51 de novo CNVs. In siblings, no overlapped region was found. 
	\end{itemize}
	
	\item Lessons: 
	\begin{itemize}
		\item Rare de novo CNV rate: about half of LoF.
		\item OR about 3 for de novo medium to large CNVs. 
		\item Substantial overlap of de novo and transmitted CNVs in probands: 8 in probands vs. 0 found in siblings
	\end{itemize}
\end{itemize}

Autistic-like social behaviour in Shank2-mutant mice improved by restoring NMDA receptor function [Won, Nature, 2012]
\begin{itemize}
	\item Shank2 microdeletion leads to ASD in human: loss of exons 6, 7 and frameshift, leading to loss of PDZ domain.
	
	\item Characterizing Shank2 deletion mouse:
	\begin{itemize}
		\item Normal reproduction and brain structure
		\item Reduced social interaction: social interaction with a stranger mouse.
		\item impaired spatial learning and memory in the Morris water maze
		\item impairments in social communication by ultrasonic vocalizations (USVs)
	\end{itemize}
	
	\item Effect on synaptic transmission: use hippocampal neurons. Basal excitation normal, but long-term potentiation (LTP) induced by high-frequency stimulation or theta-burst stimulation was severely impaired in Shank2 -/- mice.
	
	\item Marked decrease in NMDA glutamate receptor (NMDAR) function: reduced  NMDA/AMPA ratio. NMDAR-mediated transmission is selectively decreased (those mediated by AMPAR is normal)
	
	\item Shank2 deletion impairs NMDAR-associated signaling: phosphorylation but not total levels of CaMKII-α/β (T286), ERK1/2 (p42/44) and p38 were significantly reduced.
	
	\item Agonist of NMDAR improved social interaction in Shank2 -/- mice.
	
	\item Lesson: to characterize the KO of risk genes: (1) Organism level: brain structure, learning and memory, social interactions. (2) Cellular level: e.g. synaptic transmission, basal state and stimulation. (3) Molecular level: processes disrupted, here focusing on particular type of receptor important for synaptic transmission.
\end{itemize}

Genome-wide transcriptome profiling reveals the functional impact of rare de novo and recurrent CNVs in autism spectrum disorders. [Luo \& Geschwind, AJHG, 2012]: 
\begin{itemize}
	\item Motivation: genetic associations for most individual rare CNVs are not clear. Gene-expression data, might confirm the presence of functional alterations (change of gene expression) related to a particular CNV and would thus be of significant utility.
	
	\item Background: Lymphoblasts provide useful data for a significantly overlapping set of genes expressed in the CNS.
	
	\item Data: 221 probands, 188 siblings from SSC. Microarray expression of 11,150 genes. 330 samples characterized by both genotyping data and expression data.
	
	\item Outlier genes: defined for each individual (proband or sibling), more than 3 SDs (99.7\% confidence interval) from the mean expression of that gene across all samples. 
	\begin{itemize}
		\item Probands and siblings had a similar number of outlier genes per individual (about 10 down-regulated and 16 up-regulated). For brain-expressed genes: 77\% and 73\% of outlier genes were expressed in the human fetal brain in probands and siblings, respectively (no enrichment for genes expressed in the adult brain). 
		\item GO enrichment analysis: proband outlier genes, but not sibling, show an enrichment in neuron-related pathways. Note: CNV plays a small role here, because $>90\%$ of the dysregulated genes in GeneGo neural pathways are outside CNVs. 
	\end{itemize}
	
	\item CNVs and the impact on gene expression: 
	\begin{itemize}
		\item The proportion of dysregulated genes within a given CNV: defined by dividing the number of dysregulated genes by the number of expressed genes within CNVs. A significantly higher proportion of dysregulated genes in rare de novo CNVs than in rare transmitted CNVs and common CNVs in probands. 
		\item Note: this analysis is independent of ASD phenotype, thus only demonstrate that de novo and rare CNVs have larger effect on gene expression on average than common CNVs. Similar to SNVs: RVs on average have bigger impact on gene function than CVs. 
	\end{itemize}
	
	\item Twenty-seven out of 40 rare de novo CNVs identified in probands had significantly more dysregulated genes than did the genome background. Percent of dysregulated genes range from 50-100\% (for the 27 CNVs). Small non-recurrent CNVs: the outlier gene in the region are good candidates. Ex. TMLHE in Xq28 deletion (Figure 4D). 
	
	\item 16p11.2 deletion and duplication: 
	\begin{itemize}
		\item Most genes in the region show expression correlation with dosage (12 out of 19), Figure 5. The genes showing the best correlation with dosage: include potassium channel tetramerisation domain containing 13 (KCTD13), aldolase A, fructose-bisphosphate (ALDOA), and MYC-associated zinc finger protein (MAZ). All are plausible candidate genes.
		\item Consequence of the events on gene expression (trans-regulation): 70 DEX genes in 16p11.2-deletion cases and 135 DEX genes in 16p11.2-duplication cases. Not much overlap between the two lists, providing a functional basis for the different phenotypes observed in these two conditions.
		\item Correlation of gene expression in 16p11.2 region with head size. The changes in these genes' expression accounted for more than 50% of the variance in head circumference.
	\end{itemize}
	
	\item Question: for a rare CNV, many genes are outliers. Do these genes appear as outliers only in carriers of CNVs, or outliers even in other individuals? 
	
	\item Remark/lessons: 
	\begin{itemize}
		\item The genetic, expression and phenotype data are collected in the same set of individuals, thus analysis can be performed at the invididual level. 
		\item Outlier genes can be defined based on their differential expression between probands and siblings, however, due to heterogeneity of ASD, we do not expect the same gene is often differentially expressed in different individuals (about 10 DEX genes per individual on average). So any gene that is differentially expressed in one individual is considered. 
		\item For a given CNV, not all the genes in the regions will show sigificant expression changes. This is similar to missense mutations. So the consequence on gene expression (both in cis and in trans) can be important. 
	\end{itemize}
	
\end{itemize}

Patterns and rates of exonic de novo mutations in autism spectrum disorders [Neale \& Daly, Nature, 2012]: 
\begin{itemize}
	\item Data: 175 ASD probands and their parents across five centres. 
	
	\item Mutation rate estimation: 
	\begin{itemize}
		\item Genome-wide average, $1.2E-8$ (from earlier estimate) and exome is higher because of higher (50\% vs. 40\% in whole genome) GC content. Per bp mutation rate in exome: 1.5E-8. 
		\item Mutation rate matrix: 64 by 3, as the rate at each nucleotide depends on its two neighbors. 
		\item The proportion of mutation rates: from 1000 Genome project or human-primate comparisons. The equlibrium frequencies (or conditional frequencies) depend on the rates. 
	\end{itemize}
	
	\item Overall pattern of de novo mutations: 
	\begin{itemize}
		\item 161 coding region point mutations (101 missense, 50 silent and 10 nonsense), with an additional two conserved splice site (CSS) SNVs and six frameshift insertions/deletions (indels).
		\item The power of individual gene is low, so assess the enrichment of de novo mutations among all genes. Similar to QQ plot in GWAS (enrichment of high-significance SNPs). 
		\item Expect 0.87 per exome per family vs. 0.92 observed de novo mutations per exome per family. 
		\item Missense, nonsense and Synonymous mutations: the proportion of the three categories in the observed events are similar to what is expected. Nonsense mutation is about 2-fold higher (6.2\% observed vs. 3.3\% expected). 
	\end{itemize}
	
	\item Secondary phenotype analysis (covariates): arental age strongly predicts the number of de novo events per offspring.
	
	\item Genes with multiple hits: 
	\begin{itemize}
		\item Three genes with two de novo mutations: BRCA2 (two missense), FAT1 (two missense) and KCNMA1 (one missense, one silent). 
		\item Simulations show two hits not enough to define a gene as a conclusive risk factor (Table S7): expect about 1.5 genes by chance. 
	\end{itemize}
	
	\item Genetic architecture: the number of causal genes and the effect size distribution:
	\begin{itemize}
		\item Strategy: simulation the data under the assumption of fraction of causal genes and the average effect size (RR), to match the distribution of number of de novo mutations in samples (number of families with 0, 1, 2, etc., events in the genome). 
		\item Model: suppose $X$ is the number de novo events, $X$ follows Poisson distribution in Autism trios, and we want to find the rate. The model is similar to our de novo model (at gene level), but models the case of $H = 0$ or $H > 0$ ($H$: number of bad hits in the genome). Roughly: $P(A | X = x) = P(H = 0 | X = x)P(A| H = 0) + P(H > 0 | X = x)P(A| H > 0)$. 
		\item Results: 1000 causal genes with average $\gamma = 200$, highly inconsistent with the observed count distribution. 
		\item De novo SNV events will probably explain $<5\%$ of the overall variance in autism risk (Table S4). This second quantity is calculated assuming a liability threshold model and additive contributions from the many genes contributing to autism risk.
	\end{itemize}
	
	\item Protein network analyis:
	\begin{itemize}
		\item Motivation: since we have little have confidence on individual genes, we want to see if there is any pattern in the connection/relation of genes. 
		\item Higher connectivity among de novo genes: In the set of 113 genes with missense or LOF mutations, significant enrichment of PPIs using DAPPLE. 
		\item Higher connectivity with the known ASD/ID genes: Distance between previously known ASD/intellectual disability genes and the current list in the PPI network: significantly smaller than control (genes with de novo variants in unaffected siblings), but the difference is small (3.66 vs. 3.78). 
	\end{itemize}
	
	\item Combined analysis with three papers: 
	\begin{itemize}
		\item 18 genes with two functional de novo mutations are observed in the complete data.
		\item Expect 11.92 genes by chance. Simulation: draw a random set of mutations (by mutation rates), and count the number of times a gene is hit multiple times (Table S7) 
	\end{itemize}
	
\end{itemize}

De novo mutations revealed by whole-exome sequencing are strongly associated with autism [Sanders \& State, Nature, 2012]
\begin{itemize}
	\item Data: 238 families from the Simons Simplex Collection (SSC), two unaffected parents, an affected proband, and, in 200 families, an unaffected sibling. 
	\begin{itemize}
		\item Only those bases showing greater than 20 independent reads in all family members were considered for de novo mutation detection.
		\item Only consider SNV, given the uncertainty of indel detection
	\end{itemize}
	
	\item De novo mutations: overall pattern:
	\begin{itemize}
		\item Analysis strategy: compare probands vs. siblings, using case/control (2 by 2 table) analysis, for enrichment test and OR estimation. 
		\item Among 200 quartets, 125 non-synonymous de novo SNVs were present in probands and 87 in siblings: 15 of these were nonsense (10 in probands; 5 in siblings) and 5 altered a canonical splice site (5 in probands; 0 in siblings). 
		\item The total number of non-synonymous de novo SNVs was significantly greater in probands compared to their unaffected siblings. Similarly for the odds ratio (OR) of non-synonymous to silent mutations in probands versus siblings (2 by 2 table test), $OR = 1.93$. 
		\item Comparison of de novo LOF mutations in brain-expressed genes: $N = 13$ in probands and 3 in siblings (significant), and OR about 5.65. 
	\end{itemize}
	
	\item Covariate analysis: the rate of de novo SNVs indeed increases with paternal age, and that paternal and maternal ages are highly correlated. Re-analysis accounting for age did not substantively alter any of the significant results reported here. 
	
	\item Simulation for estimating genetic architecture: 
	\begin{itemize}
		\item Simulation: assume there are a certain number of ASD and non-ASD genes, for each gene, sample de novo events according to the mutation rates; then sample phenotypes by whether a subject has de novo mutations in ASD genes: (1) if not, sample by baseline penetrance, $K$; (2) if yes, sample by the penetrance ($\gamma K$). After this procedure, select a certain number of probands and siblings, and count the number of events in each group. Tune $\gamma$ and number of ASD genes s.t. the simulated counts match the counts in the real data. 
		\item Results (Table S3, S4): use $K = 0.21\%$, average relative risk under different numbers of genes, in particular, assuming there are 1000 ASD genes, the RR is $1.82\% / 0.21\% = 8.7$ for nonsyn. variants (Table S3, 1000 genes) and $11.5\% / 0.21\% = 54.7$ (Table S4, 1000 genes) for nonsense.
	\end{itemize}
	
	\item Multiple-hit genes: 
	\begin{itemize}
		\item Procedure for estimating FDR at certain number of de novo mutations via simulation: Note that only 1 gene in real data pass the threshold (say 2 de novo LOF mutations), so need to simulate ``real'' data as well (not just non-ASD genes). First simulate the data (according to the model tuned before), then count the genes with certain number ($k$) of de novo events. FDR is defined as the number of non-ASD genes have k events, divided by the total number of genes having k events. 
		\item Under all models, two or more nonsense and/or splice site de novo mutations were highly unlikely to occur by chance ($Q = 0.005$). Only a single gene in our cohort, SCN2A, met these thresholds. 
		\item No evidence that PolyPhen, SIFT, GERP, PhyloP or Grantham Score, either alone or in combination, differentiated de novo non-synonymous SNVs in probands compared to siblings. 
		\item In the SSC cohort at least three, and most often four or more, brain-expressed non-synonymous de novo SNVs in the same gene would be necessary to show a significant association. Unlike the case of nonsense and splice site mutations, these simulations were highly sensitive to both sample size and heterogeneity models 
	\end{itemize}
	
	\item Remark: 
	\begin{itemize}
		\item Estimating FDR via simulation: we want to know the FDR at 2 de novo events in a gene. In real data, only 1 gene is predicted, so insufficient to estimate FDR. Instead, we simulate data from both $H_0$ and $H_1$, count the de novo events in genes, and calculate FDR. In general, when we need to evaluate a statistical method in the context of multiple testing, we may need to estimate its FDR, and this can be done via simulation. 
	\end{itemize}
\end{itemize}

Sporadic autism exomes reveal a highly interconnected protein network of de novo mutations [O'Roak \& Eichler, Nature, 2012]
\begin{itemize}
\item Network analysis: on 126 events (turncating or severe missense mutations)
\begin{itemize}
	\item Building PPI network: GeneMANIA. 
	\item Intra-connectivity among 126 genes: 49 genes mapped to a highly connected network, significant more edges than expected $P < 0.0001$.  
	\item Connectivity with curated 103 ASD genes: degree-aware disease gene prioritization (DADA). Genes with severe mutations ranked significantly higher than all other genes (Mann-Whitney test). Signal overwhelmingly driven by 49 highly connected genes. Sibling events as control. 
\end{itemize}
\end{itemize}

De Novo Gene Disruptions in Children on the Autistic Spectrum [Iossifov \& Wigler, Neuron, 2012]:
\begin{itemize}
	\item Data: 343 families, a subset of the Simons Simplex Collection. In each family, only one is affected and at least one unaffected sibling. 
	
	\item Overall pattern in probands vs. siblings (Table 2):
	\begin{itemize}
		\item Overall SNVs are very similar in two groups: all SNVs (380 versus 364), synonymous (79 versus 69), or missense (207 versus 207).
		\item Nonsense mutations (19 versus 9) and point mutations that alter splice sites (6 versus 3)
		\item Filter for genes expressed in brain, count missense mutations that cause nonconservative amino acid changes, or count missense mutations at positions conserved among vertebrates, no statistical evidence for contribution from this type of mutation. 
		\item Indels: 53 indels in probands and 32 in siblings. Of these, 32 in probands and 15 in siblings caused frame shifts. 
	\end{itemize}
	
	\item Origin of mutations:
	\begin{itemize}
		\item From the original sequencing and validation of our data, we were able to ascertain the parental haplotype for some de novo mutations. We found that the father is more frequently the parent of origin than the mother: 50/17 for SNVs and 6/1 for indels
		\item De novo SNVs in children with the youngest fathers has lower mutation rate than in those with the oldest (p value of 0.013).
		\item The assumption is the mutations arise in germline, but somatic mutations may also be possible. Because of the filters used, the de novo events reported are largely and perhaps almost entirely germline in origin. 
	\end{itemize}
	
	\item Multiple-hit genes:
	\begin{itemize}
		\item No recurrences among our 59 LGD (likely gene disruption) targets. There are two overlaps with the 72 most likely candidate genes from our previous CNV study: NRXN1 and PHF2. 
		\item Missense: a few overlaps of the LGD targets and targets of missense mutations, two in siblings and one in probands, but this is well within random expectation.
		\item 14 of our 59 LGD targets and 13 of 72 CNV target genes, with one in common, overlap with the 842 FMRP-associated genes. No significant overlap in siblings.
	\end{itemize}
	
	\item Effects due to inheritance: This study lacks the power to discover small effects due to inheritance. Consider only rare variants, and then examined transmission to children, by affected status. No statistically significant transmission bias of either missense or LGDs.
	
	\item Purifying selection on FMRP genes: onsense and splice site rare variants: the proportion in FMRP genes is one-fourth of that in all genes. Missense variants show a much less extreme depletion in the FMRP-associated genes.
	
	\item Genetic architecture:
	\begin{itemize} 
		\item The total contribution from LGD mutations can be estimated as 31 events in 343 families (59 events in probands minus 28 events in siblings), or roughly 10\% of affected children.   
		\item A simple power calculation indicates that we cannot rule out confidently even a 20\% contribution to autism from de novo missense mutation. Despite these caveats, it is worth considering that de novo mutation causing merely amino acid substitution may only rarely create a dominant allele of strong effect.
		\item We project that nearly half of autism target genes will be among the list of FMRP-associated genes.
	\end{itemize}
	
	\item Combined analysis of all four datasets:
	\begin{itemize} 
		\item LOF mutations (including nonsense, splice site and frameshift indels): with the 59 from this study, a total of 127 hits in probands have been found. Judging from our two-fold differential rate in probands and siblings, we expect that at least half of the 127 hits, about 65, are causal.
		\item Five genes were hit twice. DYRK1A and POGZ are the new recurrences found by combining our data with theirs.
		\item From our estimate of 65 causal gene disruptions and 5 recurrent gene targets, we project that the total number of dosage-sensitive targets for autism is about 370 genes. Recurrence analysis: For a target size $T$, and $K$ picks with replacement,we can calculate analytically how many targets $R$ are picked twice or more. Given $K = 65, R = 5$, the most likely $T$ is 370. 
	\end{itemize}
	
	\item Remark/questions:
	\begin{itemize}
		\item Family data analysis: two basic strategies (1) de novo mutations as in this paper; (2) transmission disequilibrium: i.e. disease alleles are preferrably transmitted to affected children vs. unaffected ones. Not in multiplex cases (multiple affected cases in a family): transmission genetics plays a greater role. 
		\item Issues with sequencing data processing: data quality is one major isssue because of uneven coverage across the genome. To ensure quality, one may apply coverage filter: e.g. only variants above a certain coverage will be analyzed. 
		\item Indels: some are frameshift ones (most likely LOF), even in-frame indels are more disruptive to a peptide than a mere substitution. 
	\end{itemize}
\end{itemize}

Rare Complete Knockouts in Humans: Population Distribution and Significant Role in Autism Spectrum Disorders [Lim \& Daly, Neuron, 2013]
\begin{itemize}
	\item Background: there are relatively few homozygous or compound heterozygous LoF variants (i.e., complete gene knockouts) in healthy individuals. Most of these complete knockouts found are common (minor allele frequency [MAF] $> 5\%$) and are distributed across a very small number (100-200) of genes, such as the olfactory receptors.
	\item In the ASD dataset, an average individual has 5 complete knockouts, however, most of them are common LoFs. If limiting to rare LoF (less than 5\%), fewer than 5\% of individuals harbor even a single rare complete knockout.
	\item There are a total of 91 such rare complete knockouts in the case-control data sets, with 62 of these found in the cases compared to 29 in the controls, representing a roughly 2-fold enrichment (933 cases and 869 controls).
	\item Enrichment of Rare Complete Knockouts Observed on the X Chromosome: we examined LoF variants with population frequency (assessed in female control samples) of $\leq 0.25\%$ (to match 5\% LoF frequency in autosomes - the same homozygosity).
	\begin{itemize}
		\item A significant enrichment of rare hemizygous LoFs in male cases, with 88 such events observed - 60 of them were found in male cases (784) and 28 of them were found in male controls (417) (OR = 1.5).
		\item We found 2 of 170 female cases bearing a rare complete knockout on the X chromosome and 0 of 452 female controls.
	\end{itemize}
	\item We found that there was a higher rate of rare complete knockouts in females (5.4\%) compared to males (4\%). Although 16\% of the cases sequenced were female, 25\% of the cases harboring rare complete knockouts were female. This is consistent with the model that females need a higher dose of genetic risk to manifest a diagnosis of ASD.
\end{itemize}

Synaptic, transcriptional, and chromatin genes disrupted in autism [De Rubeis \& Buxbaum, Nature, 2014]
\begin{itemize}
	\item Main results of TADA: 22 genes at FDR $<0.05$ and 107 genes at FDR $<0.3$.
	\item Differential gender analysis: females have a higher liability threshold (ASD is less common in females), the consequence is that: if a variant has the same effect on autism liability in males as it does in females, the RR of the variant in the females will be higher than that in the males. As a result, the variant will be at higher frequency in female ASD cases compared to males.
	\begin{itemize}
		\item Intuition: males already have high risk, so the effect of a very strong mutation may increase the penetrance from 0.1 to 0.15; for females, the baseline penetrance is low, say 0.01, and a strong mutation can move it to, say 0.04. So in general, the RR (the ratio of penetrance) is higher in females than in males.
		\item Liability threshold model: the thresholds are $t_m = 1.98$ and $t_f = 2.56$. Suppose the effect of a mutation is $\Delta Z$. Plug in the relationship between RR and threshold (Equation 1.3 in Genetics Notes), and we can explore the difference of RR between males and females.
		\item Relation to AF difference between males and females: suppose the AF is the same between male and female controls, then the difference of OR means that the AF in male cases will be different from AF in female cases.
	\end{itemize}
	\item Enrichment analysis on FDR $<.3$ genes: evolutionarly constraints, FMRP and RBFOX (splicing factor) targets, nominal enrichment of synaptic and PSD genes.
	\item DAWN results: using constraint scores as prior. Found 160 genes, including 97 not in the FDR $<0.3$ gene list. Found four clusters (Figure 2). C2 - some genes for neurodegenerative diseases; C3 - known ASD genes; C4 - chromatin regulation.
	\item Molecular analysis: SCN2A, CACNA1D. The locations of variants/mutations - relative to known pathogenetic mutations of these genes in other diseases.
	\item Analysis of HMG genes: enrichment and interconnection of all HMG genes found by TADA. First define transcriptional network using ChIP-seq data (ChEA database), then map the connections of TADA genes and other HMGs. 
	\item Common genes of ASD and other mental diseases, Schiz., cogential heart disease and metabolic disorders.
	\item \textbf{Lessons}: analysis following gene discoveries may include: molecular-level analysis (where variants are located), enrichment of gene groups, interconnection among the discovered genes (regulatory relations, PPIs, and so on) and relationship with other diseases (shared genes). 
\end{itemize}

The contribution of de novo coding mutations to autism spectrum disorder [Iosiffov and Wiggler, Nature, 2014]
\begin{itemize}
	\item Rate of de novo mutations: LGD: 0.12 in siblings and 0.21 in probands. Missense: 0.82 for unaffected siblings and 0.94 for affected proband. So about 43\% LGD and 13\% missense DNMs are causal. 
	
	\item Estimation of contribution of DNMs: ascertainment differential (increased rate of DNM in probands per individual) is 0.21, adding missense and LOF mutations. Adding CNVs, make the proportion 0.27. `` Including copy number variants, coding de novo mutations contribute to about 30\% of all simplex and 45\% of female diagnoses.''.  
\end{itemize}

Brain-expressed exons under purifying selection are enriched for de novo mutations in autism spectrum disorder, [Uddin \& Scherer, NG, 2014]
\begin{itemize}
	
	\item Hypothesis: (1) brain-expressed exons are under stronger selection; (2) brain-expressed, highly constrained exons are candidates of ASD. 
	
	\item Contingency index: (1) Burden of rare missense mutations: in ESP data, count the number of such mutations (MAF below 0.05) divided by exon length; (2) Brain expression level of exons. An exon is considered ``critical exons'' if it has low burden and high brain expression (both defined as 75\% percentile across the entire dataset, using the multiple tissue microarray expression indices). Also brain critical exons are defined by removing all critical exons highly expressed in at least one non-brain tissue. 
	
	\item Association of exon expression and constraint
	\begin{itemize}
		\item On several known ASD genes, an inverse correlation between the burden of rare missense mutations and the brain expression levels for exons. 
		\item Test 11 tissues, brain cerebellum expression showed a strong association with the missense mutation burden. 
	\end{itemize}
	
	\item Comparing entire ASD de novo genes and sibling de novo genes (SSC data): no significant difference in conservation scores, the distribution of the burden of rare missense mutations and PPH2 scores. 
	
	\item Critical exons are enriched in putative ASD genes: 
	\begin{itemize}
		\item Among exons affected by ASD de novo mutations, 45\% are critical exons; the ratio is 28\% in the exons affected by de novo mutations in unaffected siblings. 
		\item 3,955 'brain-critical exons' (from 1,744 genes) with high expression specific to the brain and a low burden of rare mutations: enriched for FMRP targets, genes associated with ASD risk (autosomal dominant or X linked). 
	\end{itemize}
	
	\item Remark: 
	\begin{itemize}
		\item Table S7 has all the 1,744 genes containing all brain critical exons. No exon-level data is available, however.  
		\item Brain critical exons: defined using all exons, ``For each dataset exons were sorted using a threshold of the 75th percentile for expression''. In the brain critical exon genes: KATNAL2, CHD8, TBR1 not found.
	\end{itemize}
\end{itemize}

ExAC talk [ASC, March, 2015]
\begin{itemize}	
	\item $n = 3,982$ families
	
	\item ExAC browser: information of variants, including AF and number of homozygoties in each population (African, Asian, etc.)
	
	\item Using ExAC to interpret DN mutations: focus on constrained genes
	\begin{itemize}
		\item Among all DN LoFs found in ASD cases, a large fraction are not found in ExAC (OR = 2.8). 
		\item Mis3 mutations not found in ExAC: ASD de novo rate = .049, Control de novo rate = .024 (OR = 2.1)
	\end{itemize}
	
	\item Using ExAC to interpret inherited mutations: focus on constrained genes
	\begin{itemize}
		\item LoFs not found in ExAC: 1666 trios, 169 families, 105:64 - transmission bias.
	\end{itemize}
	
	\item Constraint analysis: 19 genes have 4 dn LoF (ASD + ID), all have constraint Z scores $> 4.5$. 
\end{itemize}

Excess of rare, inherited truncating mutations in autism. [Krumm \& Eichler, NG, 2015]
\begin{itemize}
	\item Data: 2,377 SSC families. Recall the DNMs, found a total of 1,544 DNMs. 21 new recurrently mutated genes (including missense). 
	
	\item SNV transmission disequilibrium: private LGD (similar to LoF) variants in genes with the lower 50\% RVIS values (constraint scores), OR = 1.14. And the signal is stronger for more constrained genes (strongest 1\%, OR = 1.4). The signal is strongest for private variants, but also persists for rare LGDs. A clear bias in transmission from mothers to sons. 
	
	\item CNV transmission disequilibrium: all transmitted CNVs, OR = 1.10. Deletion only 1.11 and duplication 1.12. 
	
	\item Integrating SNVs and CNVs: obtain $p$-values from de novo SNVs and the rest, then combine with Fisher's method. None of the convergent genes is significant, lowest $p = 0.01$. 
\end{itemize}

Whole-genome sequencing of quartet families with autism spectrum disorder [Yuen \& Scherer, Nature Med, 2015]
\begin{itemize}
	\item Motivations of WGS: (1) nongenic, non-coding RNA, CNVs; (2) WGS provides more uniform coverage of the exomes than WES. 
	\item Data: WGS (Complete Genomics) of 85 multiplex families with 2 affected children. Also collect clinical info such as ID, language, adaptive functioning, family history, and so on. Average coverage of genome: 96.8\% with average depth 56; average coverage of exome: 99.6\% with at least 5x coverage and 74.8\% with at least 40x read depth.  
	\item De novo mutation rate: 59.3 de novo SNVs / genome, and 13.2 de novo small indels ($<100$ bp). From the validation rate, 90\% for SNVs and 60\% for indels, estimate that 62 de novo events per genome.
	
	 
	\item Burden analysis (combine de novo and inherited): rare LoF and missense mutations (MAF $<.01$), significant difference between children and parents in two genes groups: all genes related to abnormal mental function (687 genes), and all neuronal and brain function–related genes from GO (309 genes). 
	\item Comparison between two siblings: test heterogeneity of ASD causes. 
	\begin{itemize}
		\item Only 29.5\% of the variants (rare variants in LoF and missense) within the two genesets are shared between two siblings. 
		\item Classification of ASD-relevant mutations: known ASD, candidate ASD, novel ASD, and neurodevelopmental disorder. Consider only de novo LoF and damaging missense, inherited LoF, CNVs (both de novo and inherited). 
		\item Identified ASD-relevant mutations in 36 of 85 (42.4\%) families.  In only 14 of these 36 families did both affected siblings carry ASD-relevant mutations. 
		\item Similar discordant rate when limit to highly-confident ASD gens and LoF mutations, and pathogenic CNVs.  
	\end{itemize}
	\item Affected siblings could carry the same DN mutations: an example of a family where both affected siblings have the same de novo CNV in SCN2A. Apparent de novo events shared between siblings are not uncommon in ASD-affected families, and mechanistically they can be attributed to gonadal or germline mosaicism or parental somatic mosaicism. Also identified 21 de novo SNV events shared between two siblings in 16 families, none in exonic regions. 
	\item Different clinical features of siblings with different ASD-relevant mutations: In siblings with shared mutations (11 out of 36 families), autism symptoms related to social and communication domains are not significantly different. But different in other sibling pairs. 
	\item The involvement of deafness-associated genes: THRA gene (thyroid hormone) in one family. Also higher burden of LoF mutations in deafness-associated genes in children vs. parents. 
	\item Remark: 
	\begin{itemize}
		\item Hypothesis that explain the finding: if autism generally requires multiple mutations, and our ASD gene list is quite incomplete, then it is possible that the shared genes are not detected. 
		\item The benefits of WGS: higher coverage in exomes and CNVs. Not using non-coding sequences. 
	\end{itemize}
	\item Lesson: 
	\begin{itemize}
		\item Burden analysis for de novo mutations: in general, need to control for gender and father's age. 
		\item Multiplex families, so likely that de novo mutations play a less important role. Still, it is possible that both siblings have the same de novo mutations because of genetic mosaicism (not uncommon). 
	\end{itemize}
\end{itemize}

Loss of $\delta$-catenin function in severe autism [Turner \& Chakravarti, Nature, 2015]
\begin{itemize}
	\item Idea of identifying of autism genes: females are more protected, thus need more deleterious variants to reach the threshold. So deleterious variants are more enriched in female-enriched multiplex families. 
	
	\item Genetic finding of CTNND2: 13 urelated female cases from multiplex families with severe autism. Found CTNND2 with multiple severe variants (highly conserved missense or LOF). Burden is significant: 362 additional females cases vs. 1GP or EPS as controls. 
	
	\item CNV evidence of CTNND2: among all CNVs overlapping CTNND2, significant enrichment of exon-disrupting CNVs vs. non-disrupting CNVs in cases than in controls. 
	
	\item Functional studies of CTNND2: (1) Phenotypic relevance to ASD: use in vitro model, dentric spine density; mouse ermrbyo morphology. (2) Function affected by CTNND2: expression of Wnt signaling genes in CTNND2 deletion. 
	
	\item Expression pattern of CTNND2: (1) Expression trajectory (higher in fetal brain) and tissue specificity. (2) Co-expression with 500 SFARI autism genes. 
	
	\item Pathway analysis with GeneMania: the GO functions of genes related to CTNND2, enrichment of dentrite morphogenesis and chromatin modification.  
\end{itemize}

Genome Sequencing of Autism-Affected Families Reveals Disruption of Putative Noncoding Regulatory DNA [Turner \& Eichler, AJHG, 2016]
\begin{itemize}
\item Data: WGS on 40 Simons quads initially, then WGS of 13 trios (affected) and 3 control trios. The selection of 40 families: 39 have no previously found LOF or CNV mutations. Coverage: 31.5, sequenced at NYGC on Illumina Hiseq X Ten. 

\item Variant calling: GATK HaplpotypeCaller, FreeBayes and Platypus. Private SNVs and indels found (not in dbSNP) will then be called by DNMFilter or TrioDenovo for DNMs. Comparison of callers: validation rate = 89\% for variants called by both GATK and FreeBayes, and 29\% for GATK only and 10\% FreeBayes only. DNMFilter outperforms TrioDenovo.  

\item Validation of 691 de novo SNVs and indels. Overall 75.2\% validation of all DNMs.  

\item Defining putative regulatory variants: fetal CNS DHS sites, with PhyloP scores $> 4$. Defining distance: $d$ kb away from upstream of TSS or downstream of TES (not including introns). 

\item Variant statistics: 70 DNMs (both SNVs and indels) per individual, and that increases by 28 or so from using TrioDenovo (lower validation rate, about 50\%). Total: 7,936 SNVs and 42 indels in 40 WGS families. The most common cause of false positives was under-calling in the parent. 
\begin{itemize}
	\item SNV concordance between GATK and FreeBayes: very high, close to 90\%. 
	\item Indels: low overlap between GATK and FreeBayes, only 30\% or so of all indels called. 
\end{itemize}

\item Burden in non-coding regulator sites (Table 1): total DNM counts in cases and controls: 3787 vs. 2997. Total DNMs in non-coding: 204 vs. 171 (not significant). Burden in potential ASD genes (57 genes): combine de novo SNVs and de novo \& Priviate CNVs that overlap with DHS: (1) 10kb: 5 vs. 0; (2) 25kb - 50 kb: 6 vs. 0. (3) 100kb: 8 vs 1. (4) 500kb: 21 vs 9.   
\end{itemize}

De novo synonymous mutations in regulatory elements contribute to the genetic etiology of autism and schizophrenia [Takata \& Karayiorgou, Neuron, 2016]
\begin{itemize}
	\item Enrichment of near-splice site DN syn. mutations: about 2-fold enrichment in both ASD and SCZ. No enrichment in distal splice sites. Also in the distance distribution of DN syn. mutations: ASD and SCZ mutations are closer. 
	
	\item Enrichment of mutations that affect exonic splicing regulator (ESR): similar pattern, about 2-fold enrichment in cases. Not in mutations that do not change ESR. 
	
	\item Enrichment of DN syn. mutations within DHS: in cerebrum and frontal cortex, but not cerebellum. Not in DHS compiled from 125 cell types.  
	
	\item Lack of enrichment with other annotations: miRNA binding sites (in coding), codon optimality or RNA secondary structure. 
	
	\item Adjusting for multiple testing: 87 hypothesis tested in total, adjust using BH. Significant results: near splice site mutations for ASD and mutations in frontal cortex-derived DHS for SCZ. 
	
	\item Gene set enrichment: genes with functional DNSMs are more likely to be intolerant (RVIS), and relevant gene sets. 
\end{itemize}

Genome-wide characteristics of de novo mutations in autism [Yuen \& Scherer, NPJ Genomic Med, 2016]
\begin{itemize}
	\item Data: 200 trios. On average, 50 SNVs, 3.9 indels and 0.05 CNVs per individual. High validation rates for all DNMs. 
	
	\item Pattern of DNMs: 239 clustered DNMs ($\geq 2$ DNMs within 20kb of the same individual). About half are within 200 bp. Most clustered DNMs are maternal in origin, and are close to dn CNVs.  
	
	\item Method of testing DNM enrichment: comparing with controls, 258 Dutch trios. The coverages are very different: 32x vs. 13x. Difference in GC. To adjust for the GC difference, use logistic regression, where $y$ is proband or control, and $x$ the features, e.g. GC of the DNMs. 
	
	\item Enrichment in splicing and UTR: (1) Splicing: use SPIDEX score, modest enrichment, $p$ close to 0.05. (2) UTR3: use PhyloP, find enrichment, $p$ about 0.05 to 0.01. 
	
	\item Enrichment of DNMs in promoter/enhancer regions: DeepBind loss at DHS: $p$ around 0.05. DeepBind loss and PCons (conservation): much larger OR, and $p < 10^{-4}$. 
	
	\item Remark: the analysis on various gene sets in the paper do not distinguish coding and noncoding DNMs. 
\end{itemize}

Genome-wide prediction and functional characterization of the genetic basis of autism spectrum disorder [Krishnan \& Troyanskaya, NN, 2016]
\begin{itemize}
	\item Prediction of ASD risk genes: (1) Training genes: about 500 from SFARI, assigned to four evidence levels. (2) Brain-specific gene network. (3) For each gene, its features as connectivity to other genes (20k features per gene), then train SVM: weigh the genes by assigning partial labels on genes with lower evidence (weighting improve the results). 
	
	\item Validation of predictions: focus on top 2,500 genes. Enriched with various sets of genes, e.g. 60\% of all LGD genes are in the 2500 gene list. 
	
	\item Spatial and temporal pattern of ASD: in 2,500 genes, study their expression patterns. Found that the expression of these genes are highly enriched in prenatal (early, mid and late fetal) stages. In contrast, spatially, it is not very specific, enrichment across most regions. 
\end{itemize}

Post-zygotic single-nucleotide mosaicisms contribute to the etiology of autism spectrum disorder and autistic traits and the origin of mutations [Dou \& Wei, review for Human Mutations, 2017]
\begin{itemize}
	\item Post-zygotic mutations: mutations occur after zygote formation. Imagine a starting embryo: at some point in development, a mutation occurs (PZM) in some progeny cells, this would lead to mosaicism. Mosaicism in somatic cells: somatic mutations, and cannot pass to offsprings; when it occurs in germline cells: there is a chance that the PZM can pass to offsprings (DNM). 
	
	\item Mosaicism in severe/lethal genes could lead to mild phenotype, but not diagnosis. Particularly, parental mosaicism could lead to recurrent risks in offsprings. 
	
	\item Detecting pSNM in children: excessive G$>$T mutations in Yale dataset (single strand). Most are FPs, probably due to oxidative damage during experiment. 
	
	\item Detecting parental pSNM transmitted to children: heterozygous in child and mosaic in parents. 
	
	\item pSNM detection: 1000 child pSNM and 300 transmitted pSNMs. Validation: 50\%. Validated pSNMs per subject was 0.152 for child pSNMs and 0.030 for transmitted parental pSNMs. 
	
	\item Burden analysis of child pSNM: compare distribution of MAFs of pSNMs between probands and siblings. In missense/LoF pSNMs: seems that the distribution is skewed towards higher MAF in probands than in siblings, and the trend is not observed in neutral mutations. At MAF $>0.2$, OR about 2, and $p = 0.02$. 
	
	\item Burden analysis of parental transmitted pSNM: Missense/LoF pSNM with low MAF ($<0.2$) are enriched in probands vs. siblings. OR = 5.36 for LoF and 1.63 for missense. 
	
	\item Combine pSNM and DNMs for gene discovery: genes carrying pSNMs tend to have some statistical evidence in TADA-Denovo analysis of WES. 
		
	\item Plausible explanations of results: 
	\begin{itemize}
		\item For parental transmitted pSNM, there is generally selection against deleterious mutations, so the rare mutations tend to be more deleterious (simliar to RVs vs. CVs).
		
		\item For child pSNM, the high MAF mutations are likely more deleterious than low MAF ones: negative selection here may not apply because the mutations happen randomly, and MAF may reflect only the timing of mutations.
	\end{itemize} 

	\item Why RR of child missense pSNMs much larger than de novo missense mutations? Germline cell competition in parents: the more deleterious mutations cannot reach high frequency in parental germlines, which limits the effect size of DNMs. In contrast, child pSNM with high MAF emerger early during develepment, and there is not a lot of competition to remove these mutations.
\end{itemize}

Genomic Patterns of De Novo Mutation in Simplex Autism [Turner and Eichler, Cell, 2017]
\begin{itemize}
	\item Coding DNM analysis: LGD depletion (ascertainment of SSC data), but DN missense mutations at CADD $>30$ show OR = 2.
	
	\item UTR: combine 5’ and 3’ UTRs, OR = 1.1, p = 0.03.
	
	\item Promoters: in fetal brain, defined by ChromHMM, within a TFBS, OR = 1.8, p = 0.03 (34 vs. 19).
	
	\item Putative non-coding regulatory regions (pNCR): conserved TFBSs mapped to fetal brain DHS. Conservation: GERP scores $> 2$. OR = 1.3, p = 0.02. A strong burden in ESC enhancers, 8 vs. 0 (p = 0.02).
	
	\item GO analysis of genes closest to DNMs in TFBSs: top 5 GO BPs are related to neuro-development.
	
	\item Remark: the non-coding analysis are only nominally significant and do not survive multiple testing.
	
	\item Enrichment analysis in autism genes: Use Turner57 genes. (1) all putative functional mutations: LoF, missense, UTR, pNCN and exonic deletions, show OR = 2.2, p = 1.7E-3. (2) Individuals with two or more DNMs near SFARI (800) genes: more enriched in probands.
	
	\item Q: definition of TFBS?
\end{itemize}

An analytical framework for whole-genome sequence association studies and its implications for autism spectrum disorder (CWAS) [Werling and Sanders, NG, 2018]
\begin{itemize}
	\item Linear regression test for mutation burden: let $y_i$ be the number of DNMs in sample $i$, $x_i$ is the sample label (proband or sibling), and covariates are paternal age and sample-level sequencing metrics (e.g. average coverage). 
	
	\item The effect of sequencing metrics and paternal age: do step-wise linear regression, the selected features are paternal age, percent of genome with coverage 30x or higher and percent of total excluded reads. The effects are generally small, e.g. RR = 1.024 before removing paternal age effect and RR = 1.005 after.
	
	\item Use permutations to obtain p-values of variant set tests: 10,000 permutations on case-control status within each family. 
	
	\item Estimate correlation structure of annotation categories: Perform random simulations of DNMs across the genome many times, then burden test of categories using binomial test. For each simulation, obtain p-values of all categories, and transform p-values to Z-scores. Then we obtain correlations of Z-scores of different categories.  
	
	\item MOM to estimate the effective number of tests: given a matrix of simulated p-values for each category and each simulation, let $M$ be the number of simulations, and $p_i$ be the minimum p-value in i-th simulation, then we calculate $M / \sum_i p_i - 1$. When all categories are perfectly correlated, it is easy to check that this gives 1. 
	
	\item To define effective number of tests using spectral clustering: $R$ correlation of categories, and $A = D^{1/2} R D{1/2}$, where $D$ is the degree. Use EVD of $A$, and the number of eigenvectors that explain most of the variations gives the number of tests. Remark: similar to finding the number of connected components in a weighted graph. Note: The effective number is higher as one increases the sample size. 
	
	\item Results for noncoding annotations (51000 tests): adjusting for 4,000 effective tests, no signal in CWAS plot. Perform K-means clustering of categories. 
	
	\item \textbf{Remark}: reduce the problem of number of independent tests to number of clusters. A related problem is the rank of a matrix. An alternative approach is to cast as a prediction problem (mutations in cases or controls), with variable selection.  
\end{itemize}

Genome-wide de novo risk score implicates promoter variation in autism spectrum disorder [An and Sanders, Science, 2018]
\begin{itemize}
	\item CWAS: 50K annotation categories. 500 coding annotations significant, but none from noncoding.
	
	\item Enriched noncoding categories: build de novo risk scores, Lasso classification of DNM status (cases vs controls). Find non-coding annotations contributing to the score, then test differences in cases vs. controls. Almost all signals are found in promoters, 2kb upstream of TSS (Figure 2).
	
	\item Conserved promoters: most promoter signals are from conserved sequences (PhastCons and/or PhyloP scores). Using conserved promoters: RR = 1.28 (Figure 3D).
\end{itemize}

Inherited and De Novo Genetic Risk for Autism Impacts Shared Networks [Ruzzo and Wall, Cell, 2019]
\begin{itemize}
	\item Data: WGS of 439 multiplex families, total of 960 affected children, and some unaffected ones.
	
	\item Rare inherited coding mutations: no burden of number of transmitted variants (affected vs. unaffected). Also no burden in private promoter variants, even limited to known ASD genes.
	
	\item Define high-risk inherited variants: transmitted to all affected but no unaffected children, then focus on those disrupting (coding or promoters) of constrained genes pLI $> 0.9$. Results: 96 genes, with 40 SVs disrupting promoters and 62 PTVs. Significant burden in PTVs: five times depletions.
	
	\item SVs disrupting promoters of two genes: case study.
	
	\item Remark: multiplex families, the variants transmitted to all affected children, but not unaffected are more likely to be pathogenic.
\end{itemize}

Autism-like phenotype and risk gene mRNA deadenylation by CPEB4 mis-splicing [Nature, 2018]
\begin{itemize}
	\item Background: CPEB4 is a risk gene of syndormic ASD.
	
	\item Model: (1) CPEB4 splicing dsyregulation in ASD: decreased inclusion of a neuron-specific microexon. (2) CPEB4 binds to many ASD risk genes, and due to CEPB4 dysregulation, the ASD risk genes show reduced polyA tails and expression.
	
	\item Lesson: mRNA stability regulation, possibly via poly-A, is important for diseases.
\end{itemize}

Large-Scale Exome Sequencing Study Implicates Both Developmental and Functional Changes in the Neurobiology of Autism [Satterstrom and Buxbaum, Cell, 2020]
\begin{itemize}
	\item Data: 6430 trios with 2000 siblings. Additionally, 14,365 case- control samples (5,556 ASD cases, 8,809 controls).
	
	\item Variant categories: three ties of PTVs by pLI scores, and 3 tiers by MPC scores. One tier of syn. variants.
	
	\item Enrichment of PTVs: 3.5 in highest PTV category in DNMs, and 1.2 in transmitted and 1.8 in case-control. Modest in next PTV category: 1.3 in DNM and case-control.
	
	\item Enrichment of missense: RR = 2.1 in strongest missense category in DNMs, no burden in transmitted (require LOFTEE tag to be high confidence HC) and 1.2 in case-control.
	
	\item TADA: de novo PTV, missenes and case-control PTV. Found 102 genes at FDR $<$ 0.1. (1) PTV: gamma is a continuous function of pLI. See Figure S2K, at the highest level, gamma about 20-30. (2) Missense: two categories, misA and misB with effects 4.18, 22.15. 
	
	\item Simulations to assess FDR: (1) Use real data, but use syn. variants to randomly set as PTV and missense, in random genes. (2) Pure simulation using multinomial. FDR is calibrated (FDR vs. q-value - Figure S2).
\end{itemize}

%%%%%%%%%%%%%%%%%%%%%%%%%%%%%%%%%%%%%%%%%%%%%%%%%%%%%%%%%%%%
\section{Neurological Diseases}

Exome sequencing in amyotrophic lateral sclerosis identifies risk genes and pathways, [Cirulli \& Goldstein, Science, 2015]
\begin{itemize}
\item Background: ALS is a neurodegenerative disease, characterized by loss of motor neurons. Approximately 10\% are familiar, abut 20 known genes explain only a minority 10\% of all sporadic ALS cases.  

\item Data: WES of 2,869 cases and 6,405 controls. 

\item Statistical analysis: for each gene, run one of 6 models and find the one with lowest $p$-value. The models are burden test, where the burden is the number of qualifying variants in a sample. The qualifying variants are defined as: 
\begin{itemize}
	\item Either dominant or recessive coding: for dominant, only consider RVs with MAF less than 0.05\% and 0.005\% in ExAC (1 in 20K, ExAC sample size is 60K). For recessive, requies two variants, MAF 1\%. 
	\item Coding: NS and splice variants. Not-benign: NS and splice, but remove all benign ones by PPH2. LoF. 
\end{itemize}
 
\item Results of gene mapping: 
\begin{itemize}
	\item A strong ALS gene (known) SOD1 (dominant coding model). Burden: 0.8\% in cases, 0.08\% in controls. Spatial pattern: concentrated in 3' portion, however, many are predicted to be benign by PPH2. 
	
	\item A new finding, TBK1, combined $P < 10^{-10}$ (dominant not-benign model). Burden: 1.0\% in cases and 0.19\% in controls, with LoF variants having even higher burden, 0.38\% in cases vs. 0.034\% in controls. Spatial pattern: diffused. 
\end{itemize}

\item Autophagy in ALS: TBK1, OPTN and SQSTMP1 (previous findings) are involved in autophagy and inflammation. Function in clearance of protein and protein-RNA aggregates. 

\item Lesson: for burden analysis, important to use very strigent MAF threshold. Spatial distribution of variants are variable: sometimes they are concentrated, but often not. PPH2 predictions may not be very informative (SOD1, most case variants are ``benign''). 
\end{itemize}
	
%%%%%%%%%%%%%%%%%%%%%%%%%%%%%%%%%%%%%%%%%%%%%%%%%%%%%%%%%%%%
\section{Cardiovascular Diseases}

De novo mutations in histone-modifying genes in congenital heart disease (CHD) [Zaidi \& Lifton, Nature, 2013]
\begin{itemize}
\item Data: trios of 362 affected probands and 264 unaffected probands (siblings of autism cases). WES: 96.0\% of bases read eight or more times.
\item Specificity and sensitivity of de novo calls: de novos are called using Bayesian quality scores (QS). At QS $> 50$, 90 calls are confirmed with 100\% accuracy. Consequently, de novo mutation calls with QS $\geq 50$ were included in the study. Sensitivity is further diminished by about 5\% owing to bases with very low read coverage.
\item Defining genes expressing in developing heart: about 4,000 genes are expressed at high levels (top quantile) in developing mouse, defined as HHE (high heart expression) genes. The rest LHE genes.
\item Rates of de novo mutations:
\begin{itemize}
\item Overall: 0.88 de novo mutations per subject in CHD cases and 0.85 in controls. Cases and controls have similar maternal and paternal ages.
\item All missense in HHE genes: 81 in cases and 32 in controls, or 0.22/case vs. 0.11/control.
\item LoF in HHE genes: 15 in cases and 2 in controls, or 0.04/case vs. 0.01/control.
\item All the signals are contributed by HHE genes: there was no significant difference in mutation frequency in CHD cases versus controls among LHE genes in all comparisons.
\end{itemize}
\item Pathway analysis: 8 de novo mutations in a pathway related to H3K4 methylation. Three genes in this pathway (MLL2, KDM6A, CHD7) have previously been implicated in rare syndromic CHD. Also the H3K4me pathway is the only enriched pathway in GO analysis.
\item Biology of the pathway: The combination of both activating (H3K4 methylation) and inactivating (H3K27 methylation) chromatin marks characterizes poised promoters and enhancers, which regulate expression of key developmental genes
\item Promising genes (Table 2):
\begin{itemize}
\item A total of six double-hit genes (Table S11). SMAD2: double hit, splice site, conserved missense. NAA15: double hit, both LoF.
\item Among the 17 above genes (?), ten have no damaging variants and seven have one to five among $>9,500$ exomes in National Heart, Lung, and Blood Exome Sequencing Project, 1000 Genomes and Yale exome databases.
\end{itemize}
\item Other structural, neurodevelopmental and growth abnormalities were common in subjects carrying de novo mutations in interesting genes. Ex. CUL3, neurodev. abnormality.
\item From the increased fraction of patients with protein-altering mutations in HHE genes in CHD patients (0.22) versus controls (0.12), we estimate that such mutations have a role in about 10\% of these patients (about 40 extra de novo in cases, thus covering about 40 people, or 10\% of cases).
\item Remark: The odds ratio calculation is weird, using the number of silent mutations (happen to be the same, 21, for cases and controls), instead of sample sizes. Also the odds ratio is defined on de novo as a whole (viewed de novo as an ``exposure'' or risk factor), but not on individual causal mutations.
\item Lesson: For some disease genes: a broader phenotypic spectrum resulting from mutations, e.g. CUL3, CHD8 (H3K4 pathway)
\end{itemize}

Systems biology with high-throughput sequencing reveals genetic mechanisms underlying the Metabolic Syndrome in the Lyon Hypertensive Rat [John Ma, Cardivascular Genetics, 2015]
\begin{itemize}
\item Data: F2 intercross of of LH and LN rats. The two strains are genetically similar, but one is selected for metS phenotype, while the other (LN) is normal. Genotype of 1536 SNPs, 23 physiological traits (blood pressure, lipid level, blood glucose, body weight, plasma leptin, etc.) and liver RNA-seq.  

\item Phenotypic QTL (pQTL) mapping: 
\begin{itemize}
	\item 169 offsprings from F2 intercross. 453 SNPs tagged all haplotypes differing between LH and LN. 	
	\item Data analysis: R/qtl, 5\% FDR using permutation testing. 	
	\item Results: 17 pQTL were identified. Two overlapping between traits. 
\end{itemize}

\item eQTL mapping: in liver and kidney. 

\item TFBS analysis: compare the two strains LH and LN, find genes whose promoters/enhancer (also in selected regions) have TFBS disruptions. 

\item Candidate genes from integrated analysis: intersect multiple sets, including expression correlated with phenotype, within pQTL, having cis-eQTL, and containing disrupted TFBSs between two strains. 

\item \textbf{Remark/Lesson}: 
\begin{itemize}
	\item The TFBS analysis is not based on eQTL or pQTL (that falls into some enhancers), instead it is based on direct sequence comparison of strains with different phenotypes. 
	\item Intersecting multiple gene lists to priorize candidate genes. 
\end{itemize}
\end{itemize}

%%%%%%%%%%%%%%%%%%%%%%%%%%%%%%%%%%%%%%%%%%%%%%%%%%%%%%%%%%%%
\section{Pharmacogenetics and Pharmacogenomics}

Common variants near ATM are associated with glycemic response to metformin in type 2 diabetes [GoDARTS, NG, 2011]:
\begin{itemize}
\item Backgroud: metaformin is one major front-line drug of T2D. Its target may be AMPK: by activating AMPK through inhibition of the mitochondrial respiratory chain, it signals lack of energy in the cells, and may push cells to absorb glucose more efficiently from blood. 

\item GWAS of metaformin response: 
\begin{itemize}
\item Phenotype: the ability to reduce HbA1c (the most widely used measure of medium-term glycemic control) in the first 18 months of therapy to below 7\%. 
\item About 700K SNPs in 1,024 Scottish individuals with type 2 diabetes. Replication in two cohorts including 1,783 Scottish individuals and 1,113 individuals from the UK Prospective Diabetes Study.
\item Association test: logistic regression with covariates of metaformin response, such as the baseline HbA1c level. 
\end{itemize}

\item Association findings: $\lambda = 1.003$. 

\begin{itemize}
	\item Only one locus at 11q22 has $P < 1.0 \cdot 10^{-6}$, the strongest SNP rs11212617. Replicated in two independent cohorts, and the combined OR is about 1.35. Explain explains only 2.5\% of the variance in metformin response.  
	\item rs11212617 is not associated with T2D, lipids, and other glucose-related traits. 
\end{itemize}

\item Candidate gene: 
\begin{itemize}
	\item Within the LD region of rs11212617, only ATM is a possible candidate for (1) ATM mutation causes ataxia telangiectasia, some of the patients show insulin resistence; (2) activation or inhibition of ATM alters AMPK activation. 
	\item ATM inhibition in rat cell lines: affects whether AMPK activity responds to metaformin treatment. 
	\item ATM function: involved in cell-cycle arrest upon DNA damage and DNA repair. This study shows a link between cancer pathways (both ATM and metaformin are associated with cancer risk), type 2 diabetes and metformin activation of AMPK.
\end{itemize}
\end{itemize}
%%%%%%%%%%%%%%%%%%%%%%%%%%%%%%%%%%%%%%%%%%%%%%%%%%%%%%%%%%%%
\section{Misc. Phenotypes}

GWAS of exceptional longevity (EL) in humans [Sebastiani \& Perls, Science, 2010]:
\begin{itemize}
\item Data source: 801 Caucasians subjects in NECS study (age of 95 to 119), 243 NECS controls, and 683 genetically matched Illumina controls (see below). Also a smaller data set as replication data. 

\item Genetic matching: significant population stratification was observed in randomly chosen controls (GC $= 1.3$) probably because most of the case subjects immigrated to US in a certain period and may be biased towards certain ethnicity. To address this, the idea is to choose among a large number of controls whose genotypes match those in the cases: 
\begin{itemize}
\item Analysis of ancestry: PCA on the subjects (cases and potential controls), and use the first four PCs to cluster subjects into 20 clusters. 
\item Selection of controls: for each cluster that is represented in the cases, select controls in that cluster s.t. the ratio of case/control is the same in all clusters. 
\end{itemize}

\item Prediction/classification: 
\begin{itemize}
\item Naive Bayes classifier: for a given set of SNPs $\Sigma_k$, a basic classifier can be built: 
\begin{equation}
P(EL | \Sigma_k) = \frac{P(EL) \prod_{i=1}^k P(SNP_i|EL)}{P(EL) \prod_{i=1}^k P(SNP_i|EL) + P(AL) \prod_{i=1}^k P(SNP_i|AL)}	
\end{equation}
where $AL$ is the average longevity. The conditional probabilities of $P(SNP_i|EL)$ and $P(SNP_i|AL)$ are from the gentoype frequencies in cases and controls. 

\item Ensemble classifier: rank all SNPs by their posterior probability of association, and create the SNP sets: $\Sigma_1, \Sigma_2, \cdots, \Sigma_K$. The ensemble classifier: 
\begin{equation}
P(EL|\Sigma_1, \cdots, \Sigma_K) = \sum_{i=1}^K P(EL|\Sigma_i) / K	
\end{equation}
In the paper, $K = 150$ is chosen by the performance (spec. and sens. under cross validation). 
\end{itemize}

\item Genetic risk profiles: the cases may have distinct genetic signatures that relate to the subtypes of EL, so the cases are clustered according to their risk profiles. 
\begin{itemize}
	\item Risk profiles: instead of directly cluster of genotypes (150 features), first convert them into risk profiles, defined as $P(EL|\Sigma_1), P(EL|\Sigma_2), \cdots, P(EL|\Sigma_{150})$. Thus different genotypes will be manifested as different profiles/curves. 
	\item Clustering: the profiles are clustered according to a Bayesian clustering algorithm. 
\end{itemize}

\item Significant SNPs (longevity-associated variants, or LAVs): 70 are found and 33 were replicated. 
\begin{itemize}
	\item Alzheimer's disease (AD): APOE, CTNNA3, STX8 (cell cycle regulator, elevated in AD patients), PLCB3 (enzyme phosphlipase C $\beta 3$ involved in extracellular signals)
	\item Insulin signaling: GIP, RAPGEF4, both involved in regulating insuline secretion. However, FOXO1, FOXO3A and IGF-IR showed no significant associations. 
	\item Chromosomal stability: HJURP
	\item Immune response: IL7. 
\end{itemize}
  
\item Comparison with common disease SNPs: only 5 out of 1389 common disease SNPs (from all published studies) show significant associations with EL. Also compare the allele frequencies of the known disease-associated SNPs: no significant difference between cases and controls for the majority of these SNPs. 

\item EL classification: 150 SNPs are chosen. They are uncorrelated, with the average distance of 8Mb. The ensemble classifier achieves 77\% spec. and sens. in the replication set. 

\item EL subtypes: 
\begin{itemize}
	\item 19 clusters of 8 or more centenarians in the discovery set; and 11 in the replication set, with 10 of these clusters in both sets. 
	\item The comparison of clusters: different age distribution, and prevalence/onset of common diseases. Ex. C1 cluster had a significant delay in the onset of cardiovascular disease, dementia and hypertension. 
	\item Cluster 19 of 17 centenarians: lack most of the LAVs, suggesting there may be many more modifiers of EL to be discovered. 
\end{itemize}

Remark: the paper may have serious methodological flaws: 
\begin{itemize}
\item Genotyping platforms are different in cases and in controls: this would commonly produce false associations. Need to replicate with the same platform on both cases and controls (not done). The platform on the cases is known to have problems. 

\item The results seem too good: some SNPs have large effects, and 77\% accuracy is extremely high, comparing with other complex traits. Also, the recent meta-analysis of GWAS of longevity does not find any significant SNPs. 

\item Indication of genotyping problems: For the 70 genome-wide-associated SNPs, the median missing data rate in EL samples was 9\%, compared to 3\% in controls. The two SNPs with strongest evidence for association, rs1036819 and rs9576827, are far out of HWE. 

\item Diagnosis of Manhattan plots: the significant SNPs stand out alone, without evidence in the nearby SNPs (which is the usual pattern of the Manhattan plots). 
\end{itemize}

\end{itemize} 


%%%%%%%%%%%%%%%%%%%%%%%%%%%%%%%%%%%%%%%%%%%%%%%%%%%%%%%%%%%%
\section{Personalized Medicine \& Clinical Genetics}

What is personalized medicine? 
\begin{itemize}
\item Definition: Personalized medicine is ``a form of medicine that uses information about a person's genes, proteins, and environment to prevent, diagnose, and treat disease'' (National Cancer Institute 2011). 

\item Personal data: 
\begin{itemize}
	\item Genomic data: genotype, DNA sequences, transcriptomics, epigenomics, proteomics, metabolomics, meta-genomics/microbiome. 
	\item Clinical data: medical records, family health history (FHH), other diagnostic variables. 
\end{itemize}

\item Clinical applications: 
\begin{itemize}
	\item Risk prediction/preventation: from genotype/sequencing data, from FHH, from risk factors (body weight, etc.)
	\item Diagnosis: disease markers, especially of subtypes; identification of putative therapeutic targets
	\item Prognosis and Treatment: types of treatments, dosage. 
\end{itemize}

\end{itemize}

Reference: [Green \& Guyer, Charting a course for genomic medicine from base pairs to bedside, Nature, 2011], [Chan \& Ginsburg, Personalized Medicine: Progress and Promise, ARGHG, 2011], [Personalized medicine: new genomics, old lessons, Offit, Human Genetics, 2011]

Background: understand the biology of genomes: 
\begin{itemize}
\item Catalog of genetic variations, especially those that are associated with phenotypes. These provide both markers and candidate causal variants of diseases. 

\item Omics data: DNA modifications (epigenomics), gene products such as RNAs (transcriptomics) and proteins (proteomics), and indirect products of the genome such as metabolites (metabolomics) and carbohydrates (glycomics). 

\item Functional elements: their functions, and especially the role of non-coding sequences in health and disease. 

\item Gene regulation: its spatially and temporally dynamic nature presents formidable challenge, as some critical regulatory processes only occur during brief developmental periods or in difficult-to-access tissues. 

\item Gene networks: network analysis will benefit from understanding the dynamics of gene expression, protein localization and modification, as well as protein-protein and protein-DNA associations. The ultimate challenge will be to decipher the ways that networked genes produce phenotypes.

\item Evolutionary relationships: the use of model organisms in functional studies, and diverse data sets from unicellular organisms to mammals.
\end{itemize}

From genomics to the biology of diseases: 
\begin{itemize}
\item The power of genomic approaches to elucidate the biology of disease: e.g Crohn's disease. Following GWAS of Crohn's disease, cellular models and animal models have been developed for the effects of causal variants and the knowledge of the relevant biological pathways. Chemical screens have been designed to identify new candidate therapeutic agents. 

\item GWAS and NGS mapping for complex diseases: 
\begin{itemize}
\item Successful examples of GWAS: the complement pathway in age-related macular degeneration, autophagy pathways in Crohn's disease, a number of pathways not evident from the somatic genetics of cancer. 

\item Lesson: Human disease susceptibility is the result of rare genetic variants of high penetrance as well as common genomic variants of low penetrance [Offit]. 
\end{itemize}

\item Non-Medelian inheritance: imprinting, de novo germline mutations, and epigenetic mechanisms of inheritance. 

\item Catalogues of somatic mutations that contribute to all aspects of tumour biology for each major cancer type are under development. 

\item Non-coding sequences: GWAS have implicated hundreds of non-coding genomic regions in the pathogenesis of complex diseases. 

\item Phenotyping: widely accessible databases containing extensive phenotypic information linked to genome sequence data (genotype) are needed (e.g. dbGaP). Such efforts will benefit greatly from the linkage of genomic information to electronic medical/health records.

\item The integration of genomic information and environmental exposure data: help to understand the links between biological factors and extrinsic triggers. 

\item Metagenomics: offers unprecedented opportunities for understanding the role of endogenous microbes and microbial communities in human health and disease. 
\end{itemize}

Personal and genomic data: 
\begin{itemize}
\item Family health history (FHH): a simple yet invaluable tool for the delivery of personal health risk information. [Guttmacher et al 2004. The family history: more important than ever. N. Engl. J. Med.] 

\item Personal risk factors: Framingham coronary heart disease model, the Gail model breast-cancer risk assessment [Gail \& Greene 2000. Gail model and breast cancer. Lancet]

\item Genomewide variation of sequences: 
\begin{itemize}
	\item Hepatitis C treatment: e.g. a polymorphism on chromosome 19, 3 kb upstream of IL-28B, encoding interferon-lambda-3 was found to be associated with a twofold change in treatment response. [Ge et al. 2009. Genetic variation in IL28B predicts hepatitis C treatment-induced viral clearance. Nature]
	\item Statin response: a polymorphism located on SLCO1B1, a gene regulating hepatic uptake of statins, was associated with an increased risk of myopathy following statin treatment (odds ratio 4.5) [Link et al. 2008. SLCO1B1 variants and statin-induced myopathy: a genomewide study. N. Engl. J. Med.]
	\item Next-generation sequencing: being applied to understand cancer, rare genetic disease, and microbial infection, etc., with the goal of elucidating functional gene variants. 
\end{itemize}

\item Transcriptomics: 
\begin{itemize}
	\item Cancer classification using gene expression data: microarray data have been used for diagnosis, prognosis, and response to therapy [Parker et al. 2009. Supervised risk predictor of breast cancer based on intrinsic subtypes. J. Clin. Oncol.] [Van't et al. 2002. Gene expression profiling predicts clinical outcome of breast cancer. Nature]
	\item Classification in other complex diseases: cardiovascular disease, rheumatic diseases, neurologic diseases such as multiple sclerosis, psychiatric disorders such as schizophrenia, bipolar disorder, and major depression [Goes et al, 2008. The genetics of psychotic bipolar disorder. Curr. Psychiatry Rep.] [Bray, 2008. Gene expression in the etiology of schizophrenia. Schizophr. Bull.]
	\item Patterns of differentially expressed miRNAs can conceivably be used clinically in the diagnosis and prognosis of disease. 
\end{itemize}

\item Metabolomics: 
\begin{itemize}
	\item Background: It is estimated that the human metabolome contains approximately 5,000 discrete small-molecule metabolites, and the identification of metabolic changes associated with disease immediately suggests enzymatic drug targets.
	\item Application: chronic disease states, such as diabetes, obesity, cardiovascular disease, cancer, and mental disorders [Bain et al. 2009. Metabolomics applied to diabetes research: moving from information to knowledge. Diabetes] [Griffin et al. 2004. Metabolic profiles of cancer cells. Nat. Rev. Cancer] [Newgard et al. 2009. A branched-chain amino acid-related metabolic signature that differentiates obese and lean humans and contributes to insulin resistance. Cell Metab]
	\item Metabolomics profiling has also directly been used as a tool in assessing drug toxicity
\end{itemize}

\item Epigenomics: 
\begin{itemize}
	\item Cancer: hypomethylation in oncogenes and regional hypermethylation in genes that are tumor suppressors 
	\item Other diseases: Methylation changes have also been shown in other diseases such as type 2 diabetes, cardiovascular pathologies, and autoimmune diseases.  
\end{itemize}

\end{itemize}

Genomics in personalized medicine: 
\begin{itemize}
\item Risk Prediction: 
\begin{itemize}
\item Mendelian disorders: the variant genes responsible for most Mendelian disorders will be identified and an immediate benefit will be an accurate diagnosis.
\item Cancer: e.g. BRCA1 and BRCA2 and susceptibility to breast cancer [Trainer et al 2010. The role of BRCA mutation testing in determining breast cancer therapy. Nat. Rev. Clin. Oncol.]. Microsatellite instability in mismatch repair genes MLH1 and MSH2 and the early detection of colon cancer. 
\end{itemize}

\item Disease diagnosis and molecular characterization: 
\begin{itemize}
\item Disease subtypes: genomic and molecular analyses have revealed distinct subtypes of disease, which have been traditionally defined by broad clinical or descriptive phenotypes.
\item Cancer genomics: [Lee et al., 2010. The mutation spectrum revealed by paired genome sequences from a lung cancer patient. Nature] [Pleasance et al. 2010. A small-cell lung cancer genome with complex signatures of tobacco exposure. Nature]
\item Frequent novel mutations of the PI3 kinase regulatory subunit gene were found along with an association between MGMT methylation status and mismatch repair mutations in posttreatment GBM [Comprehensive genomic characterization defines human glioblastoma genes and core pathways. Nature, 2008] 
\end{itemize}

\item Disease prognosis: 
\begin{itemize}
\item Tumor-gene-expression signature models: combined with clinically relevant data such as survival outcomes [Van't Veer et al. 2002. Gene expression profiling predicts clinical outcome of breast cancer. Nature] 

\item Example: Oncotype DX is a 21-gene signature used to predict distant recurrence over 10 years [Sparano et al, 2008. Development of the 21-gene assay and its application in clinical practice and clinical trials. J. Clin. Oncol]
\end{itemize}

\item Treatment: 
\begin{itemize}
\item New drug targets: development of drugs based on genomic knowledge is becoming increasingly commonplace, particularly for cancer drug development, e.g. HER2, Bcr-Abl inhibitor. 

\item Patient stratification: using genomic information can aid clinical trials (allow the use of smaller numbers of participants and increase statistical power). Correlation of genomic signatures with therapeutic response will enable the targeting of appropriate patients at appropriate stages of their illness.
\end{itemize}

\item Pharmacogenomics: 
\begin{itemize}
\item Resistance to cancer treatment: Two studies using differential gene expression found that genes involved in chemoresistance were also associated with a worse prognosis [Holleman et al. 2004. Gene-expression patterns in drug-resistant acute lymphoblastic leukemia cells and response to treatment. N. Engl. J. Med.]

\item Response to cancer treatment: human epithelial growth factor receptor (HER2), was shown to be amplified in 25\%-30\% of breast cancers, and its overexpression correlated with a worse prognosis. Trastuzumab, a monoclonal antibody that targets HER2, is effective in reducing tumor burden. 

\item HIV: genetically guided prescription of the antiretroviral drug abacavir is now the standard of care for HIV-infected patients. 

\item An important example of pharmacogenetics is in the management of warfarin therapy [Klein et al. 2009. Estimation of the warfarin dose with clinical and pharmacogenetic data. N. Engl. J. Med.]
\end{itemize}

\item Monitoring Disease Response to Therapy: 
\begin{itemize}
\item Peripheral blood mononuclear cell (PBMC) gene expression profiling (a set of 11 genes) is now used routinely in some centers to monitor the status of grafts following solid organ transplantation [Deng et al. 2006. Noninvasive discrimination of rejection in cardiac allograft recipients using gene expression profiling. Am. J. Transplant.] 
\end{itemize}

\end{itemize}

Gene-environment interactions and Microbiomics: 
\begin{itemize}
\item Gene-environment interactions: 
\begin{itemize}
\item A host-based gene expression signature was recently identified that may someday be used for early detection of viral infection.  The same data were also used to show that gene expression patterns distinguish between bacterial and viral infections

\item Gene expression patterns in host cells also have been shown to change given other environmental stressors such as smoking as well as during disease states like asthma and chronic obstructive pulmonary disease [Seibold \& Schwartz 2011. The lung: the natural boundary between nature and nurture. Annu. Rev. Physiol.]
\end{itemize}

\item Pathogen genomes: 
\begin{itemize}
\item Virulence: Comparative proteomic analysis between virulent and avirulent phenotypes [Bechah et al. 2010. Genomic, proteomic, and transcriptomic analysis of virulent and avirulent Rickettsia prowazekii reveals its adaptive mutation capabilities. Genome Res.]

\item Resistance: how pathogens develop resistance, track the genomic changes over time in a patient after exposure to antibiotic treatment. 

\item Several diseases have been associated with large-scale imbalances in the gut microbiome, including inflammatory bowel disease, antibiotic-resistant diarrhea, and obesity [Ley et al 2006. Microbial ecology: human gut microbes associated with obesity. Nature]

\item The Human Microbiome Project: creating not just a database that contains disease-causing pathogens, but also a control set of normal flora [Peterson et al. 2009. The NIH human microbiome project. Genome Res]
\end{itemize}

\end{itemize}

Areas/Challenges of personalized medicine: 
\begin{itemize}
\item Association mapping with sequencing data: rare variants. 

\item Role of gene regulation and non-coding sequences in complex diseases, including cancer. 

\item Integrative mapping of complex diseases: with multiple types of dataset, e.g. sequences, transcriptomics and epigenoimcs. This strategy has been used in cancer genomics (TCGA). 

\item Gene networks and diseases: how the changes of genes affect the gene networks, and how the network changes lead to phenotypic changes. This is related to the study of epistatis (gene interactions). Ex. the effect of regulatory genes may be mediated through (more direct) effector genes. 

\item Gene-environment interaction: environmental variables may be manifested as other genomic data (e.g. expression, or metagenomic). 


\item Risk prediction and diagnosis with multiple types of personal data: genotype/sequencing data, electronic record and family history. 

\item Functional impact prediction of sequence changes: important for risk prediction. 

\item Patient stratification and disease subtyping: use genomic features to identify subtypes and use them for treatment. 

\item Drug target development: prediction of what genes may serve as good therapeutic targets. 
\end{itemize}

Superheroes of disease resistance [NBT, 2016], on [Chen, NBT, 2016]
\begin{itemize}
	\item A large number of samples, 1/2 Million: find all candidates with mutations in severe Mendelian diseases. 
	
	\item Filtering: by AF, health of individual carriers, Sanger sequencing, penetrance of the mutations. Found 13 individuals resilient to one of eight Mendelian diseases. 
	
	\item Difficulty of identifying modifier variant: for each variant (highly penetrant), perhaps only a few individuals in 1/2M samples contain this variant, thus little power of finding modifier loci. 
\end{itemize}

Phenome-Wide Association Studies as a Tool to Advance Precision Medicine [Denny and Roden, ARGHG, 2016]
\begin{itemize}
	\item Resources: eMERGE, China Kadoorie Biobank, GERA cohort, MVP.
	
	\item PheWAS: association of a SNP with many phenotypes.
	
	\item Phenotype definition: ICD codes from bills, most commonly used. Exist algorithms to map ICD to phenotypes: phewascatalog.org. Other data: Laboratory data, medication records such as endophenotypes and drug response.
	
	\item Application of PheWAS: often replicate existing association results, also report associations with new phenotypes.
	
	\item Other uses of EHR: define disease comorbidities, e.g. periodental disease with T2D and hypertension. Define disease subtypes: e.g. T2D subtypes associated with distinct traits, also different genetic variants.
	
	\item Challenges: multiple testing burden. Separate true pleiotropic effects with shared clinical comorbidity.
\end{itemize}
%%%%%%%%%%%%%%%%%%%%%%%%%%%%%%%%%%%%%%%%%%%%%%%%%%%%%%%%%%%%
\section{Genetics of Model Organisms} 

Using model organisms to establish phenotypic role of a gene: \begin{itemize}
	\item Need both deletion phenotype and genetic rescue (introducing the original gene to see if it restores the phenotype). Need rescue because the deletion experiment may introduce other unwanted changes. 
	
	\item Specificity of phenotype: e.g. a mutation may affect growth, and thus obesity phenotype. 
	
	\item Genetic interactions to explore pathways: if $A \rightarrow B$, then B would dominate the effect of A. 
\end{itemize}

The future of model organisms in human disease research [NRG, 2011]
\begin{itemize}
	\item Using model organism to gain a better understanding of fundamental biological processes that are related to human health. Ex. epistasis map from yeast; study the pathway of blood vessle generation in human using yeast (orthologous system). 
	
	\item Model organisms may have advantages of gene mapping including population structure and experiment design: balanced alleles to avoid the rare allele issue, etc. 
	
	\item Model organisms also have advantages of functional studies: mainly access to tissues, e.g. eQTL on multiple tissues, integration with phenotypic data. 
	
	\item Model organsisms as model system to study the function of genes/variants, especially in the context of complex behavior/phenotypes. 
\end{itemize}

Variation of sporulation efficiency in natural yeast strains [Gerke \& Cohen, Science, 2009]: 
\begin{itemize}
\item Problem: sporulation efficiency in oak tree strains and vineyard strains are very different (nearly 100\% vs 3.5\%). What is the genetic basis? 

\item QTL analysis (by crossing the two strains) identified 5 QTLs, out of which 3 could explain most of the effect ($R^2$ = 0.87), allowing two- and three way interactions between loci. 

\item Mapping nucleotide changes: candidate genes are relatively easy to locate within about 50-100 kb confidence intervals of the QTLs. (1) Rme1 (sporulation regulator): non-coding region substiution; (2) Ime1 (sporulation regulator): one subsitution in coding sequences (involved in PPI with other sporulation regulators), and the other in non-coding region; (3) Rsf1 (activator of mitochondrial genes): coding sequence substitution. May promoter Ime1 expression (which is sensitive to respiratory signal). 

\item Discussion: 
\begin{itemize}
	\item Selection of sporulation efficiency in woodland environment, but not in vineyard. Consistent with the protein polymorphism of Ime1 (relaxed selection in vineyard strain). 
	\item Loci of a complex trait: coding sequence of regulators; noncoding/regulatory DNA sequence of regulators; genes that may influence the expression of regulators. 
\end{itemize}
\end{itemize}

Rsu1 regulates ethanol consumption in Drosophila and humans [Ojelade, PNAS, 2015]
\begin{itemize}
	\item Gal4 system in fruit fly: use $P$ element containing Gal4 to delete a gene. Then introduce UAS-transgene (UAS: upstream activation sequence): activation by Gal4. TG is only expressed in individuals with P-element insertion. 
	
	\item Background: fMRI. Comparison between two conditions. Visualization of fMRI images: multiple views. ROI (region of interest) analysis: extract $\beta$ for each pixels (voxel), then average. Need to control for covariates such as gender. 
	
	\item Joint CV-RV analysis using kernel method. 
\end{itemize}


%%%%%%%%%%%%%%%%%%%%%%%%%%%%%%%%%%%%%%%%%%%%%%%%%%%%%%%%%%%%
%%%%%%%%%%%%%%%%%%%%%%%%%%%%%%%%%%%%%%%%%%%%%%%%%%%%%%%%%%%%

\end{document}