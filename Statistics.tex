\documentclass{report}

\usepackage{amsmath,braket,fullpage}
\usepackage{setspace}
\usepackage{url}
\usepackage{amsfonts}
\usepackage{natbib}

\def\Var{{\rm Var}\,}
\def\E{{\rm E}\,}
\def\Cov{{\rm Cov}\,}
\def\tr{{\rm tr}\,}

\providecommand{\abs}[1]{\lvert#1\rvert}
\providecommand{\norm}[1]{\lVert#1\rVert}
\providecommand{\e}[1]{\ensuremath{\times 10G^{#1}}}

\newcommand\independent{\protect\mathpalette{\protect\independenT}{\perp}}
    \def\independenT#1#2{\mathrel{\setbox0\hbox{$#1#2$}%
    \copy0\kern-\wd0\mkern4mu\box0}}

\title{Notes on Statistics and Machine Learning}
\author{Xin He}
\begin{document}
%\maketitle
\tableofcontents
%\newpage
%%%%%%%%%%%%%%%%%%%%%%%%%%%%%%%%%%%%%%%%%%%%%%%%%%%%%%%%%%%%
%%%%%%%%%%%%%%%%%%%%%%%%%%%%%%%%%%%%%%%%%%%%%%%%%%%%%%%%%%%%
\chapter{Statistical Theory}
\section{Overview of Statistical Data Analysis}

\subsection{Principles of Statistical Modeling} 

Overview of data analysis: the goal is to infer the underlying structure in the data and make predictions. 
\begin{itemize}
\item Prediction problem: we consider the problem of predicting $y$ from $x$. The challenges and ideas can be similarly applied in any data analysis problems. Examples: 
\begin{itemize}
	\item Image analysis: image $\rightarrow$ semantics/category. 
	\item Text analysis: sentence/text $\rightarrow$ semantics/category. 
	\item Genetics: genotype $\rightarrow$ phenotype. 
\end{itemize}

\item Model multiplicity/overfitting: when we do not know clearly the underlying process generating the data, we fit a ``statistical model'' to the data, and this leads to the problem of overfitting or model multiplicity: there are many ways of fitting the data and it is not clear whether the model fits the true signals or simply noises (also called Roshoman Effect, [Breiman, Stat Sci, 2001]). Specific sources of Roshoman Effect may include: 
\begin{itemize}
	\item Number of parameters is much larger than the number of observations: curse of dimenionality. The solution of a system of linear equation is underdetermined. 
	\item The predictors are correlated: singular design matrix. 
	\item Nonlinearity/Interactions among predictors: this could increase the number of variables exponentially. 
\end{itemize}

\item Heterogeneity: related to model multiplicity is the heterogeneity: data are generated from different processes. In the context of prediction, the same $y$ may be caused by many different combinations of $x$, with complex rules. Examples: 
\begin{itemize}
	\item Text analysis: the same meaning can be encoded by various syntactic structures, and the same entity may be represented by many different noun phrases. 
	\item Genetics: the same disease can be caused by changes in different genes, in different combinations. 
\end{itemize}
\end{itemize}

Signal and noise: the difficulty of extracting signals from noises:
\begin{itemize}
\item Mistaking noises for signals: for example, consider a problem of predicting $y$ from many explanatory variables $x_j$'s. Even if none of the $x_j$ is relavant to $y$, there may be correlations between some $x_j$ and $y$ in the data, due to chance. This is basically the overfitting problem with relatively complex models, or type I error (in classical statistics). 

\item Mistaking signals for noises: for example, suppose we are trying to identify the relation between $x$ and $y$ (a true predictor). Our model may include irrelevant covariates $z$ that are correlated with $x$: then we may explain away the correlation between $x$ and $y$ using the relationship between $z$ and $y$. Thus incorporating irrelevant covariates may lose power. 
\begin{itemize}
\item Example: a LRT with high df. loses power, because a lot of signals are simply explained by the model as noises (a high df. model implies more noises). 
\end{itemize}
\end{itemize}

The challenges of overfitting: we understand this through typical examples: 
\begin{itemize}
\item Regression problem: suppose we predict $Y$ from features $X_j, 1 \leq j \leq D$, where $D$ is large. Then many $X_j$'s may correlate with $Y$ by chance (noise), and it is not clear which one is the true signal. 

\item Generative model for classification: suppose we have a generative model $Y \rightarrow X_1, \cdots, X_D$, where each $X_j|Y$ follows Bernoulli distribution, with parameter depends on $y$. For most $j$'s, the parameters should be the same in different classes, but because of sampling noises, they may appear different. 

\item Clustering: suppose data are generated from a small number of true clusters, but by chance, some points may appear close to each other, forming extra clusters. 

\item Density estimation: suppose our true density is $f(x) = U(x)$, the uniform distribution. But the histogram of data generated may not appear very smooth. So if we use kernel density estimation with a small $\lambda$, we obtain irregular density function. 
\end{itemize}

Paradigms of statistical data analysis: 
\begin{itemize}
\item Probabilistic modeling paradigm: the data is generated from a probabilistic process, thus infer the underlying structure/process through statistical inference. 
\begin{itemize}
	\item The model should respect the underlying process. Often a multi-step process, and may involve latent variables in some stage (which help to expose the structure/pattern of the data/problem). 
	\item The characteristics / desired properties of solutions, should have high probability to occur in the model. 
	\item The statistical model should capture the structure, dependency, relation in the data. For example, in a spatial model, the correlation of variables in adjacent regions. The dependency can be very informative: a variable would carry information of another correlation variable. 
\end{itemize}
The heterogeneity challenge can be addressed best through Bayesian statistics: model averaging, latent variables (structure), priors and random-effects (allow variations across related processes), etc. 

\item Predictive modeling paradigm: find the underlying associations between $x$ and $y$, for future predictions. The heterogeneity and nonlinearity challenges can be addressed through: model shrinkage, model averaging, partition-based methods, non-parametric/prototype methods, etc. 

\item Informative patterns: recognize the patterns informative of the underlying process and/or desired properties of the solutions being searched form. These can be expressed as statistic (in classical statistics) or features (in machine learning), and used for testing hypothesis or make predictions. Ex. in ANOVA, the informative pattern is the intra- vs. inter-group variations. 

\item Model analysis: for a prediction problem, need to address the multiplicity and heterogeneity challenges. The main idea for the former is: a simple model is prefered - regularization, and additional structure in the data to limit the space of possible models; for the latter, important to analyze the source of heterogeneity and model it properly. The general solution to a statistical learning problem from Nature is: hierarchical representation of data/object that at each step, extract key/stable features that normalize the data/input. Examples: 
\begin{itemize}
	\item Image processing: image $\rightarrow$ geometrical features $\rightarrow$ object $\rightarrow$ image semantics. 
	\item Natural language processing: sentence $\rightarrow$ syntatic strutucre and entities $\rightarrow$ semantics.
	\item Genetics: genotype (mutations at nucleotide level) $\rightarrow$ changes of genes $\rightarrow$ effects on cells $\rightarrow$ phenotype. 
\end{itemize}
Normalization need not be unidirectional: e.g. for a sentence, may try different parsing to find out which one leads to meaningful semantics. 
\end{itemize}

Principles of statistical data modeling: 
\begin{itemize}
\item Goal: translate the basic understanding/intuition of data into formal models. Need to capture the characteristics of data (or data generating process), or speaking in different terms, explain the data (variations in the data). It is important to control model complexity. 
\begin{itemize}
	\item The analysis of problem: the underlying structure/process, the source of heterogeneity in classification problem, the prior knowledge/extra data that can be used, an appropriate measure of similarity/distance, whether the developed model and inferred results could reflect the properties, etc. 
	\item Model flexibility: a model should be able to explain various possible scenarios. Ex. in genetics, a model be able to account for the situation where a small number of loci each having strong effect or the situation where many loci each having small effect. In practice, consider several special cases and see if the model can explain all cases. 
	\item Model identifiability/complexity: a model cannot be too complex (over-parameterized) or it may not be identifiable, or very diffusive posterior. This is particularly important when the model contains missing variables. 
\end{itemize}

\item Modeling associations: some property of an object may suggest another property; or if two objects are similar in some aspect, then they may be similar in another. This is the basic idea of statistical learning: infer some property $Y$ from features $\mathbf{X}$. Alternatively, the principle can be applied to group similar objects, as in cluster analysis. 
\begin{itemize}
	\item Examples: cancer prediction from gene expression patterns; prediction of document quality from hyperlink structures; etc. 
	\item Strategies of modeling associations/dependency: conditional distribution (regression as a special case), samples from the same distribution (to capture similarity), MRF-like model (undirectional dependency), kernel method (local smoothing). 
\end{itemize}
	
\item Features/patterns/properties: from the raw data, define features, as functions of the data, that are better predictive of the property of interest. Patterns (particular arrangements of basic elements) or functional properties can be powerful features. 
\begin{itemize}
\item Ex. to predict function of a sequence, use its physical properties (stability, etc.) as features. 
\item Ex. SVM method, by using more features (e.g. functions of the original features), the positive and negative classes are more likely to be separatable in the hyperplane.
\item Ex. a complex geometric object may be represented as a function/curve with a small number of parameters, then these parameters are treated as features. 
\end{itemize}

\item Dealing with overparameterizatio/overfitting: Bayesian random effects or sparse models. Under Bayesian framework: the parameters come from a common (or similar) distribution, thus the effective number of parameters is much smaller. Under the sparsity assumption: e.g. most parameters are zero, and the free parameters is also small. 

\item Dealing with uncertainty by integration/averaging: in general, if there are uncertain parameters/variables, integrate over the unknowns, or average over multiple observations/models, etc. This is a fundamental idea of Bayesian approach, where the parameters are integrated over. A special case is: combining models - aggregating over a large set of competing models can reduce the nonuniqueness while improving accuracy.  
\end{itemize}

Probabilistic modeling: 
\begin{itemize}
\item Generative models: for some data (such as experimental measurements), some physical/natural process can be used; for many other data types, this is not the case (e.g. image data). The idea is to model the data as generated from a sampling process. Ex. in genetics, even though genotype determines phenotype, we could assume we sample genotypes from persons with a given phenotype. 

\item Explaining the data: the goal of a statistical modeling can be stated as finding a model that ``explains'' the data. Most importantly, this can be said to explain the variations in the data. Two general sources of variations: (1) other related variables; (2) errors/probabilistic processes.  

\item Informative statistic(s): inference relies on the statistic (summary of data) that contains information of the parameters of interest, where ``information'' means the different values of the parameters would lead to different distributions of the statistic. In particular, we could obtain $E(T)$ of the test statistic $T$, as a function of the parameter $\theta$, then the value of $T$ would suggest the value of $\theta$ (similar to how physical parameters are estimated, treating $T$ and $\theta$ just as two physical quantities). Examples: testing parameters in a linear model: 
\begin{itemize}
	\item Estimator of the parameter: this is essentially the correlation between random variables, and is informative of the regression coefficient (larger coefficient would imply a larger correlation). 
	\item ANOVA: the informative statistics are between-group variance, which contains information of whether the level means are equal, and the within-group variance, which contains information of the intrinsic errors. 
\end{itemize}

\item Likelihood function: this is a special informative statistic that is applicable whenever a full probabilistic model is available. The information that this statistic captures is the Fisher information. 
\end{itemize}

Sparse modeling/regularization: this can be expressed as $p >> n$ problem.   
\begin{itemize}
\item Most parameters are 0/Lasso: Take an example of generative model for classification, let $p_1(x) = f(x|y=1)$, and $p_0(x) = f(x|y=0)$. $p_1$ shold be mostly close to $p_2$ except in a few features, e.g. suppose $p_c(x_j) = \theta_j^{(c)}$ where $c = 0,1$, then we should have $\theta_j^{(1)} = \theta_j^{(0)}$ for most $j$'s. 

\item Group and spatial structure: the parameters within a group should be similar. In spatial model, the adjacent parameters should be similar.  
	
\item Leveraging other hidden structure: e.g. tree structure, which implies that some features are more correlated than with other features.  
\end{itemize}

Classical vs. Bayesian statistics: 
\begin{itemize}
\item Classical statistics: if the data model is valid, a strong framework for inference. It aims to solve: 
\begin{itemize}
	\item Parameter estimation: the estimators may be developed from MLE, MOM, error minimization, etc., and the assessment is usually carried out by the Mean Squared Error, which is the sum of the square of bias and variance. 
	\item Test of hypothesis concerning parameters: determine the distribution of the estimators (under $H_0$), and assess the power of test. 
\end{itemize}
However, classical statistics has limited tools to establish validity of the data model, most imporrantly, goodness-of-fit test, residual analysis. Very limited power with more than four to five independent variables [Breiman, Stat Sci, 2001].

\item Model selection in classical statistics: one creates a general model, and make the problem of model selection a problem of inference on parameters. Examples: 
\begin{itemize}
\item Variable selection in regression: inference of whether $\beta_j = 0$. 
\item Mixture model: selecting the number of components. Suppose the number is bounded by $K_{\max}$, then a model of $K_{\max}$ components, inference of whether $\theta_k = 0$, where $\theta_k$ is the mixture weight of the $k$-th component. 
\item Bayesian networks: to select among many possible structure $G$, define a multivariate normal distribution (suppose all variables are normally distributed), then infer a certain structure exists in the covariance matrix $\Sigma$. 
\end{itemize}

\item Difficulty with model selection: classical statistics is not designed to select models from many alternatives, with possibly different complexities. 
\begin{itemize}
\item Maximum likelihood parameter estimation: cannot compare models with different complexities, thus a poor strategy for selecting models. 
\item Hypothesis testing: classical statistics is mainly concerned with $H_0$ vs. $H_A$, and this could be very inadequte. Example: for variable selection problem, testing individual $\beta_j = 0$; however, because variables tend to be correlated, it is likely that few of them will pass the threshold (especially consider multiple testing), even for true variables. 
\item Non-nested model: even when we limited to two model comparison, when the two are not nested, they may pose a problem for classical statistics. 
\end{itemize}
 
\item Bayesian statistics: a model $M$ is assessed by the posterior probability $P(M|D)$, where $D$ is the data. Bayesian statistics provides a number of strategies to deal with the model multiplicity issue: 
\begin{itemize}
	\item Averaging: over possible values of parameters or types of models. 
	\item Random effects: the data need not be generated by a single process with a single set of parameters. Instead, the parameters may be variables (related in some way), and this may allow more explanation of data, and a better normalization of data (by incorporating structure of parameters, and by averaging over parameters). 
	\item Latent variables: this could achieve the effect of normalization, or capturing variations (a lot of variations may be explained by a few unobserved variables). 
	\item Prior: this would generally favor simpler models, and also could guide the model search using prior knowledge. 
\end{itemize}

\item Remark: 
\begin{itemize}
\item Despite the limitations, sometimes classial statistics can be very useful for some model selection problem. Example, when comparing a small set of alternative models for a Bayesian network. 

\item Classical statistics may address the problem of model complexity by introducing contraints in the parameters. Ex. for variable selection in regression, constraints on the $L_1$ norm of the parameters (Lasso). 
\end{itemize}
\end{itemize}

Evaluating model fit: 
\begin{itemize}
	\item Principles: (1) agreement of model predictions and observation. (2) how much variation in the data is explained by the model. 
	
	\item Example: linear model. (1) Residual plot. (2) $R^2$ measure of goodness-of-fit. 
	
	\item Example: fitting parameteric distribution. (1) Histogram vs. PDF of the fitted model. (2) Similar $R^2$ measure? 
\end{itemize}

Predictive modeling: the essence is to impose constraints on the model explaining the data. 
\begin{itemize}
\item General strategy: search for a model that minimizes the generalization error, defined as $E[L(Y,\phi(X)]$, where $L()$ is the loss function, and $\phi$ is the model. The generalization error is often estimated by cross-validation in practice. Since generalization error is often hard to obtain, often search for a model that minimize some loss/error function or maximize the explanation of variations in the training data, with appropriate regularization. Statistical perspective is generally important: data is often modeled as generated from some stochastic processes. 

\item Geometric perspective: a model can be viewed as a geometric surface approximating the data points $(x_i,y_i)$ (in regression), or as providing a decision boundary (in classification). 

\item Regularization: (e.g. in the case of classification) geometrically, a naive/complex model has a complex/rugged decision boundary. A good model should be simpler, have a smooth decision boundary (hence, the term, ``regularization''). Shrinkage methods directly impose constraints/penalty on parameters such as simpler models are favrored. Ex. Lasso. 

\item Margin methods: the intution is for any classifier of the training data, the one with the low margin of error is likely to be wrong in the future (small perturbations may cross the decision boundary), and this class of models is more complex (many possible modesl to have low margin of error). Thus favor models with high margins.

\item Partition-based methods: partition the data into regions, where in each region, there may be simpler (linear) model. Ex. decision tree. The general difficulty is that the correct partitioning is unknown. Some form of soft partitioning, and combining partition with prediction may help (form a partition such that within each region, there is a simple model). 

\item Prototype/non-parametric methods: the idea is that locally the model may be simple/linear. Use the training data to define the local regions. This is similar to the partition-based methods in that essentially each sample point defines a local region. The difficulty is that how to define locality is not clear, and in high-dimensional space, there may not be an instance in training data that is close to an instance to be predicted, thus the model may be not very generalizable. 

\item Semi-supervised learning and the use of external data: first, it could help uncover additional structure on features s.t. a better partition can be formed where local models can be learned; second, it may allow better estimation of parameters relevant to the data. 

\item Model averaging/ensemble learning: if not a single good is available. Similar to the idea of partitioning, though does not do that explicitly.  

\item Connection between parametric and non-parametric methods: Ex. linear regression (model-based): the solution can be written as a linear function of $\mathbf{y}$ (response), thus may be interpreted as weighted contribution of the training examples; KNN (instance-based): may be understood as a model using certain centroids, but the position of centroids must be learned from the data.  
\end{itemize}
  
Comparison and relation between generative and predictive modeling paradigms:  
\begin{itemize}
\item Interpretation: both could have appropriate interpretations. E.g. $X = $ gene expression pattern, and $Y = $ phenotype (e.g. growth rate), (1) phenotype is a function of expression pattern; (2) phenotype indicates the internal status or environmental condition of the cell (e.g. growth rate reflects the nutrient availability), then expression pattern is determined by this status/condition. 

\item Comparison of two paradigms: 
\begin{itemize}
\item Feature expansion: it is much easier to incorporate additional features in a regression framework than probabilistic models.
\item Density modeling: this may be difficult, especially as the process of sampling $X$ may be biased. Thus the generative approach may be more sensitive to outliers, e.g. in LDA, the estimation of class centroids (mean of normal distribution is sensitive to outliers). 
\item Latent variables: easier in the generative models. 
\item Model averaging: easier in the generative models. 
\item Nonlinearity: if no structure is known, then regression models may be more flexible by using prototype/non-parametric methods, or by adding more features (kernel trick). 
\end{itemize}

\item Connection between regression and generative modeling: the optimal method may need to model both $P(x)$ and $P(y|x)$. In the regression approach, $P(x)$ is ignored, however, it may be informative: e.g. there is cluster structure in the space of $X$, and the same cluster tends to have the same class label. In the classification problem, this is semi-supervised learning that takes advantage of the unlabeled data. 

\item Combine generative and regression models: suppose we want to classify using BFs - a generative model approach. But if we can partition the information into multiple parts, each part captured by a BF. Then we can use each BF as a feature and train a classifier. The idea is not limited to BFs: we can use LRT, even p-values. 
\begin{itemize}
	\item Example: classify cancer genes. Information from mutation rates and spatial clustering. BF from two features, and classify using the linear combination of two features. 
\end{itemize}
\end{itemize}

Strategies of parametric probabilistic models:  
\begin{itemize}
\item Simple linear model: suppose we condiser the problem of predicting $Y$ from $X_1, \cdots, X_p$. The simplest model would be a linear model of $Y$ from $X$. This model suffers from a number of problems, including: noisy features, nonlinearilty, dependence between features (e.g. one feature affects only with certain values of another features). 

\item Group and locality structure: often exists despite heterogeneity, this can be at the sample level or the variable level. Possible structures: (1) The same group of samples has the same model; or related/close samples have similar models; (2) Variables in the same group tend to have non-zero effects at the same time; or within each group, only one variable should be chosen. 
\begin{itemize}
	\item Benefit: heterogeneity in the data is captured, while at the same time regularization and variance stabilization (without overparameterization). 
	\item Models: a number of ways of modeling these structures, including structure models (e.g. introducing group variables), hierarchical models (modeling group parameters), kernel smoothing (e.g. varying coefficient model). 
	\item Example: in genomics, $X$ is gene features (such as regulator binding), $Y$ is expression, then genes in the same group should share a model (using the same features); or related genes should have similar regulators. 
	\item Example: decision tree is effectively partitioning the samples and learn models in each sample. Ex. partition the samples by the discrete variables using decision tree, and then at each leaf node, learn a linear model with the remaining contiuous variables. 
	\item Remark: sample and variable structures are related: e.g. one could introduce group membership variables for samples, then the structure is stated in terms of variables.  
\end{itemize}

\item Feature interaction and expansion: introduce additional variables as functions of features. They may be non-linear functions of individual features or interaction terms of multiple features. 
\begin{itemize}
	\item Benefit: non-linearity and feature dependence (the effect of one feature depends on another features). Feature dependence is one way of modeling heterogeneity: e.g. the model parameters may depend on some other variable (such as time).   
	\item Example: define features $\sigma(X_j) \sigma(X_k)$ where $\sigma(\cdot)$ is the sigmoid function, to approximate logic AND, and OR. 
\end{itemize}

\item Structure models: the conditional distribution or the MRF-like models of the variables, in particular, graphical models. 
\begin{itemize}
	\item Benefit: greatly limit the possible models (version space), i.e. reduce model complexity.  
	\item Example: model joint distribution of $X_j$'s and $Y$'s using a MRF model, e.g. predicting the spin state of a grid in the Ising model. 
\end{itemize}

\item Hierarchical model: a probabilistic model (prior distribution) of parameters. 
\begin{itemize}
	\item Benefit: regularization of parameters or variance stablization. 
	\item Example: model parameters as functions of additional variables or features. Could be used in variable selection, e.g. $\beta_j \sim \text{Mixture}(0, N(\tau, \sigma^2)$. 
\end{itemize}

\item Latent variable models: the actual explanatory variables for $Y$ may be latent, or missing. Latent space model: all true explanatory variables are latent. 
\begin{itemize}
	\item Benefit: a smaller number of explanatory variables (assuming a good model relating observed variables and latent variables), thus reducing variance. 
	\item Example: document class is a function of latent topics; phenotype if a function of latent gene activities. 
\end{itemize}

\item Variable selection/sparsity: in general, when $p$ is large, there may be only a subset of variables that influence $Y$. The correlation of one variable $X_j$ to $Y$ may actually be due to a correlated variable $X_k$ to $Y$. 
\begin{itemize}
	\item Benefit: a simpler model reduces the variance of the estimators/prediction, or reduces model complexity in general. 
	\item Example: learn the structure of $X_j$'s from unlabeled data (e.g. a Bayesian network model), this may limit the possible explanatory variables or variable interactions for $Y$.  
\end{itemize}

\item Nonlinear functions: directly capture the non-linear aspect in the data, e.g. GLM, splines, decision tree, cyclic structure, semi-parametric models (e.g. varying coefficient model), etc.
\begin{itemize}
	\item Benefit: capture the specific aspect/characteristic of data. 
	\item Example: a continuous version of decision tree, $Y$ is a sum of product terms of features. 
\end{itemize}

\item Independence assumption: one often makes independence assumption in the model. Sometimes this may be violated. Ex. in Naive Bayes model, if features are correlated, the OR (or BF) will be inflated. Suppose we have 10 highly correlated features, each feature contributes to OR by 2; the total contribution would be $2^10=1,024$, while the true contribution is much smaller. 

\item Prior knowledge and external data: when available, use them to constrain the model. In Bayesian, this can be added as priors; in sparse model learning, can be added as regularization terms of the objective function. 
\begin{itemize}
	\item Benefit: limit the model search space, thus reducing complexity. 
	\item Example: in regression problem, one may know the importance of variables before hand, e.g. $\beta_1 > \beta_2$, or $\beta_1 > 0$ if and only $\beta_2 > 0$, this can be incorporated in the model. 
\end{itemize}

\item Example: application in document/sentence classification: 
\begin{itemize}
	\item Group and locality structure: using document meta-data, e.g. authorship (one author has certain topic preference), date, link structure, etc. 
	\item Feature interaction and expansion: N-gram features. 
	\item Structure models: HMM or CRF models at sentence level. 
	\item Hierarchical models \& variable grouping: words in the same category (or synonymous words) have similar weights. 
	\item Latent variable models: supervised LDA. 
	\item Variable selection: IDF weighting or removing of stop words. 
\end{itemize}

\item Example: application to genotype to phenotype mapping: 
\begin{itemize}
	\item Group and locality structure: family relations among samples.  
	\item Feature interaction and expansion: SNP interactions.  
	\item Structure models: Markov model of sequential SNPs.  
	\item Hierarchical models \& variable grouping: genes in the same pathway tend to have similar regression coefficients.  
	\item Latent variable models: gene activity as the latent variables. 
	\item Variable selection: weighting SNPs by prior evidence; causal variants in LD blocks.  
	\item Nonlinear function: non-additive genetic models (dominant and recessive models). 
\end{itemize}

\item Example: the same ideas can be applied to model joint probability distributions (no explicit response variables), e.g. model DNA sequence evolution. The structure in the data, i.e. the variation of evolutionary rates, can be modeled in a number of different ways: 
\begin{itemize}
	\item Structure models \& Latent variable model: a hidden variable of functional class of a position (fast or slow), and allow switching of classes across different positions with HMM. 
	\item Hierarchical model: the rate itself (across all positions) follow a random distribution. 
	\item Kernel smoothing: the adjcant positions should have similar rates, model the rate as a function of position, and apply the varying coefficient model. 
\end{itemize}

\item Remark: many ideas can be combined and further improve inference. Examples: 
\begin{itemize}
	\item Variable grouping and variable selection: choose one variable per group. Ex. genetic association, where features are SNPs in LD. 
	\item Structure models with latent variables: e.x. HMM. 
	\item Feature expansion and latent variables: the expanded features (interactions) are related to latent variables. Ex. the interaction between two genes correspond to the state of a pathway (latent). 
	\item Sample grouping and variable grouping: one sample group may use one variable group. 
\end{itemize}
\end{itemize}

Understanding probabilistic models:
\begin{itemize}
\item Importance: while in an inference problem, one can often follow the generic procedure (e.g. MLE, MCMC), it is often important to understand the properties of the probability models/distributions, such as expectations, covariance, etc. This would help one to understand the consequences/implications of a prob. model: what patterns are informative of model parameters (and thus help parameter estimation and hypothesis testing). 

\item General steps: 
\begin{itemize}
	\item Model identification: the first step of understanding a model, for a complex model, it may not be identifiable. 
	\item Patterns/consequences of the model: how informative patterns are related to the model parameters. 
	\item Estimation procedure: an intuitive understanding of the estimator/test statistic, the estimation algorithm (e.g. many algorithms are itereative procedures). 
\end{itemize}

\item Examples: 
\begin{itemize}
\item Multivariate normal distribution: the covariance matrix, marginal and conditional distributions. 
\item Bayesian networks: the conditional independence structure. 
\item Ising model (MRF): the covariance structure of the spin states of the lattice sites (not necessarily adjacent) - this would help understand the equilibrium of the Ising model.
\end{itemize}

\item Another way of ``understanding'' a model is: analyzing the statistic that one uses to solve the inference problem, e.g. Bayea factor for model selection problem - how does this statistic depend on the properties of data? What is its behavior (for different types of data/input)? Does it make intuitive sense? 

\item Remark: related to Method of Moments parameter estimation, though the idea of understanding the consequences (patterns) of a prob. model is very general.  
\end{itemize}

Optimization perspective: 
\begin{itemize}
\item Data analysis with optimization: one can formulate certain desired properties of the solutions of a problem, and usually these can be expressed as an objective function to be optimized. 

\item Regression: the parameters of such a problem should maximize the fitting of data, or minimize the prediction error. Ex. least square fitting of linear models. 

\item Clustering: the positions of clusters should minimize the total intra-cluster distance. 

\item Missing data: the values should minimize some kind of errors (with respective to known data). Ex. in alignment problem: maximize the similarity of two sequences. 
 
\item Statistical perspective: while the optimization perspective sometimes is enough to solve a problem, a statistical perspective often brings benefits, including: uncertainty of estimation; parameter estimation when training data is available; etc. 
 
\end{itemize}

Common statistical considerations: 
\begin{itemize}
	\item Model identifiability: whether data is sufficient to estimate parameters/models (if not, choose simpler models, etc.)
	
	\item Information: whether all the information is used. Ex. when a procedure involves discretization of contiuous values, information may be lost (two different objects may be treated equal). 

	\item Alternative models/data explanations: whether there are alternative models not considered in the model. Ex. the alternative hypothesis may not include all possibilities, thus rejection of null may not guarantee the acceptance of the current alternative hypothesis. (Bias is one special case.)
 
	\item Distribution assumption/outliers and independence assumption: whether data follows the assumed distribution: e.g. normality assumption; whether data points can be considered as i.i.d. If these assumptions not held, how sensitive the method is to outliers, or to dependence of data points. 
	
	\item Mathematical functions: whether it is appropriate to add terms (often used in models such as regression, SVM, may need transform variables so that they may be added), etc. 
	
	\item Simplification of problems by using data summary: we may not need to model the complete/original data. If we can use statistics to capture all the information in the data, we can work on the summary statistics, which is often much easier. 
	\begin{itemize}
		\item Ex. meta-analysis of linear regression. 
		\item Ex. hierarchical linear model, where $\beta$ follows some distribution. We can make inference of $\beta$ on each group, and then model the estimated and standard error of $\beta$ using normal distribution. 
	\end{itemize}
\end{itemize}

\subsection{Statistical Theory and Methods} 

Reference: [Breiman, Stat Sci, 2001]

Hypothesis testing: a key consideration is to increase power in testing hypothesis. 
\begin{itemize}
\item Degree of freedom: $H_A$ needs to fit significantly better than $H_0$ for one to accept $H_A$, thus if one has a complex $H_0$, it will be difficult for $H_A$ to do significantly better. Therefore, a complex $H_0$ could result in power loss. 

\item Multiple hypothesis testing correction: reduce the number of hypothesis tested will increase the power of testing. 
\end{itemize}

Regression modeling: 
\begin{itemize}
\item Variable selection: when independent variables are correlated, the additivity assumpion in the linear model may not hold (only independent effects are additive), thus variable selection is crucial. From the hypothesis testing perspective, having more independent variables than necessary (a complex model) generally reduces power. 
\begin{itemize}
	\item Shrinkage method: Lasso, etc. that penalizes more parameters.
	\item Bayesian variable selection: select a subset that provides a balance of data fitting and model complexity. 
\end{itemize}
\end{itemize}

Classification through hypothesis testing: 
\begin{itemize}
\item Class density modeling and classical hypothesis testing: the class density approach, where $P(x|y)$ is modeled, can be treated as testing two hypothesis: $y = 1$ and $y = 0$. The classical hypothesis testing can then be applied, where some test statistic $T$ is used to reject or accept $H_0$. If use LRT, then the test score is similar to the Bayesian posterior probability, $P(y|x)$, usually used for classification. 

\item Determine the threshold under the supervised setting: the threshold $t$ can be determined by minimizing some appropriately defined error function. 

\item Determine the threshold under the unspuervised setting: when the training data is not available, the threshold is chosen to meet certain level of type I error, or FDR (usually need multiple hypothesis testing correction). Alternatively, choose a certain sensitivity level (e.g. threshold as the test statistic of the top $K$ prediction), and evaluate the FDR. The latter is particularly useful when comparing methods. 

\end{itemize}

Unsupervised learning: two basic perspectives for analysis of unsupervised data:
\begin{itemize}
\item Hidden patterns: search for objects with certain properties or certain types of relations in the data. Examples:
\begin{itemize}
	\item Objects: (1) human behavior patterns: infer terrorist suspect; (2) genes with certain expression profiles. 
	\item Relations among objects: (1) human data, infer social network; (2) functional interactions among genes. 
	\item Relations among features: (1) items that are bought togeter from transaction data; (2) grouping tissues from expression data. 
	\item Relations among both objects and features: e.g. identify a set of people that work for the same company (share features such as income, city, favorite restaurant, etc.). 
\end{itemize}
Note that the types of objects and relations considered depend on the problem, and can be very complex. Ex. for association rule mining, the rules can be any logic/algebric functions the variables satisfy. 

\item Latent structure: identify latent variables or structure/grouping (how objects are organized) in the data. Example: 
\begin{itemize}
	\item Object grouping: e.g. social network (how people are related to each other). 
	\item Latent variables: e.g. gene expression pattern of cells, the cellular condition is not directly measured, but important latent variable. 
\end{itemize}
\end{itemize}

Dealing with heterogeneity/overparameterization: 
\begin{itemize}
\item Problem: in a problem of many objects, each of them may have some unique properties (heterogeneity). Using the same parameter for all would be too simple and may lead to false conclusions, while using different parameters for each object lead to a model with too many parameters (overparameterization). 

\item Strategies: 
\begin{itemize}
	\item The general idea is to introduce structure into the model. Important cases include: certain objects share certain properties (random effect), hierarchical model, a smaller number of hidden variables (principle component analysis). 
	\item Random effects: each parameter is a random sample from a common effect. Then one could test/estimate the shared distribution. Its advantage is that the evidence of multiple objects can thus be combined to make inference. 
	\item Mixture model/grouping: an important special case of random effects is the mixture model approach. The idea is that: each object belongs to one of multiple classes (each class: the same parameter), and the class assignment is a hidden variable. Effectively, this is to group similar objects together. 
\end{itemize}

\item Examples: 
\begin{itemize}
	\item Molecular evolution of proteins: different sites may evolve at different rates. Random effects model: the rate of each site is from a probability distribution, and test the parameter of this distribution. 
	\item Functional properties of pathways: different genes may be related, but still have individual differences. Random effects model: each gene has unique contribution, but the parameter is from a common distribution, and test this distribution. [Gene group association with clinical outcome, Goeman \& van Houwelingen, Bioinfo, 2004]
\end{itemize}

\end{itemize}

Latent variable models: 
\begin{itemize}
\item Strategy: in many problems, it is natural/advantageous to introduce additional latent variables: one can build a better model/explanation of data in terms of these latent variables. 

\item Applications: 
\begin{itemize}
\item Dimensionality reduction: e.g. PCA, a small set of latent variables explain most of the variations of observations. 
\item Causal model with latent variables: e.g. SEM. 
\item Prediction with latent variables (instead of the observables): e.g. factor regression (or factor analysis). 
\end{itemize}

\item Benefits of latent variable models: 
\begin{itemize}
\item Imposing additional structure in the model, reducing model complexity. This is similar to hierarchical models, where instead of having one model per group, the models of all groups are related in some way. 
\item Reducing dimensions and better interpretability: these latent variables may represent concepts/themes (text analysis), objects/patterns (vision), pathway activity (genomics), etc. 
\item Better predictive model: a response may better be predicted with (fewer) latent variables, in particular, all the relevant observations are used to learn the effect of latent variables, and this achieves the effect of pooling, and improve inference. 
\end{itemize}

\item Relation to multi-level modeling: in some applications of multi-level models, the group-level parameters (that are subject to modeling) can be viewed as latent variables (e.g. group means), and so they are special cases of latent variable models. 

\item Latent variable in regression models: suppose we are predicting $Y$ (response) from $X$ (observations), and $Z$ are latent variables that are better predictors of $Y$ (e.g. more direct relations with $Y$). There are a number of ways of modeling the relation among these variables: 
\begin{itemize}
	\item Generative model: $Y \rightarrow Z \rightarrow X$. Ex. in text analysis, we have Document class $\rightarrow$ Topic $\rightarrow$ Words. 
	\item Regression model: $X \rightarrow Z \rightarrow Y$. Ex. in genetics, we have SNP $\rightarrow$ Gene $\rightarrow$ Phenotype.
	\item Joint model of predictors and responses: $Y \leftarrow Z \rightarrow X$. Ex. in genomics, we have Class $\leftarrow$ Module/Pathway activity $\rightarrow$ Gene expression. 
\end{itemize}
In a particular problem, we may use any of the three possible models, e.g. in text analysis, joint modeling may be the used, where topics (latent) determine both document class and words (supervised LDA). 
\end{itemize}

Feature development: 
\begin{itemize}
\item Goals of feature development:
\begin{itemize}
	\item Feature representation: for complex objects, need a relatively simple representation that allows e.g. comparison of similar objects. Ex. image analysis: to recognize similarity between images, represent images by vectors, where features correspond to spatial regions; then image similarity can be simply defined as the correlation. 
	\item An important part of the learning procedure is to develop/recongize features that may distinguish different types of objects, e.g. ``fingerprints''. 
\end{itemize}

\item Developing features: 
\begin{itemize}
	\item Elements: of the objects. Ex. text classification problem, the words are natural elements. 
	\item Patterns: elements are often insufficiently discriminative (e.g. single words for text, or single AA for sequences) or not recognizable (e.g. image data), then the recurrent patterns (some particular arrangements of elements) may serve as good features. 
	\item Properties: functions defined on the elements/patterns/objects that reflect certain properties of the objects. Ex. DNA sequence classification: the physicochemical properties inferred from DNA sequences; CpG islands as markers of genes; evolutionary footprints. 
\end{itemize}

\item Features as compact representations of complex objects: it may be possible that a few simple features explain variations of complex objects. Ex. variation of beak shapes in Darwins's finches can be explained by three geometric parameters (scaling, etc.). 

\item Examples of features for different types of objects: 
\begin{itemize}
	\item Sequences: presence of motifs/k-mers; co-occurrence of motif pairs; property of sequences (e.g. conservation, DNA bending, stability, TF-binding, etc.)
	\item Sentences/text: words and phrases; the synatic structure of sentences (e.g. Entity-Verb structure, where Verb is one of a list of key verbs). 
	\item Gene expression profiles: the expression of pathways (e.g. an entire is up-regulated), the co-expressed genes (modules) - the module strcuture may characterize one type of profiles vs the other. 
	\item Images/structures: spatial patterns of geometric objects or atoms (values at spatial units within a reference framework). 
\end{itemize}
\end{itemize}

Feature learning: to learn features important for a class of objects is often part of the learning problem/goal. 
\begin{itemize}
	\item Classical statistics: by testing statistical significance of the parameters, e.g. testing the hypothesis that $\theta > 0$ vs. $\theta = 0$ for some parameter $\theta$. 
	\item Nonparametric tests: e.g. testing overrepresentation: the features important for discrimination may be overrepresented in one class only relative to the other. Ex. motif finding in sequences.
	\item Regression: by testing the effect on the performance when the feature is removed or permuted [Breiman, Stat Sci, 2001]. 
	\item Difficulty of learning important features: features are often correlated, then removing one feature (as in both classical statistics and ML) may have small effect, as the correlated feature may make up for the lost feature. 
\end{itemize}

Data normalization: 
\begin{itemize}
	\item Why do we need normalization? Often we need to compare some variables, but the measurement (data) is influenced not only by the quanity of interest, but also other sources. So we will need to remove the influence of other sources/noises. 
	
	\item Examples: 
	\begin{itemize}
		\item Compare expression of a gene in multiple conditions. The expression in a condition is influenced by batch effects, biological variations, sample quantity, etc. 
		\item Identify CNVs from arrayCGH or sequencing data. Variation of measurement is large across genome even in background (no CNV) regions due to differences in capture efficiency (for WES), PCR, sequencing, etc.  
		\item Calling peaks from ChIP-seq data. Read depth is influenced by GC content, mappability, etc. 
	\end{itemize}
	
	\item Strategies of normalization: in general, analyze what factors could cause problems (data points not directly comparable), and develop strategies accordingly. 
	\begin{itemize}
		\item Paired treatment-control design: e.g. for arrayCGH or ChIP-seq, use paired controls to obtain background signal. 
		\item Controlling known confounders: e.g. for aCGH data, control for GC content, read depth. For expression data, control batch effect (covariate), etc. 
		\item Controlling unknown confounders: find out if some hidden variables explain the variation of variables, then these variables can be controlled. Typically through PCA. 
		\item Using ``local background'' as controls: the idea is to mimic paired treatment-control, using similar data points. Ex. in ChIP-seq, use local genomic regions as control. Quantile normalization. 
		\item Data transformation: we could transform the data in such a way that it becomes comparable across data points. Ex. expression of different genes: obtain values such as RPKM that removes dependency on gene length and library size.  
	\end{itemize}
\end{itemize}

Confounding variables and bias: 
\begin{itemize}
	\item Association not equal to causation \& confounding factors: these are essential considerations for drawing a valid conclusion, as there may be factors not considered or encoded in the model (which contribute to the observed patterns). To deal with this issue: defining an appropriate controls; incorporating the confounding factors/variables in the models; etc. 
	
	\item Bias: what are the possible biases/whether data are comparable, e.g. to compare statistic of objects of different sizes. This often involves the analysis of what other factors (other than the main effect we are studying) may contribute to observed data. 
	
	\item Subtle distinction between confounding and bias: in confounding, we are concerned with relationship between two RVs. In bias, we are talking about a general issue of some estimate quantity or some test. Ex. we compare genes in GWAS, and define gene p-value as min-p. There is no confounding here (as we are not explicitly studying two variables), but there is a gene-size bias. 
	
	\item Testing associations with genomic features: often we are interested in whether some genomic features are associated with some properties, for example: 
	\begin{itemize}
		\item GWAS: test if GWAS hits are more likely to be localized within enhancers. 
		\item De novo mutations: test if de novo mutations are more likely to be in enhancer regions.  
		\item TFBS distribution: test if TFBSs tend to be in evolutionarily conserved regions. 
	\end{itemize}
	In all these cases, there are possible confounding variables that may create association, such as: distance to TSS, GC content of genomic regions, mutation/recombination rates, mappability. 
	
	\item Inferring genomic properties: we may want to infer some underlying properties of genomic regions, e.g. TFBS, their interaction, chromatin states, etc. Examples: 
	\begin{itemize}
		\item Chromatin interaction from Hi-C: infer the strength of interaction. Confounding variables: random DNA looping and genomic context (e.g. GC content).
		\item ChIP-seq: infer the location of peaks. Confounding variables: mappability. 
	\end{itemize}
	
	\item Strategies of dealing with confounding variables/bias: 
	\begin{itemize}
		\item Linear model and matching confounding variables: test the hypothesis at matching values of confouding variables. Under the linear model, assume the coefficent is the same with different values of $Z$.  
		
		\item Permutation that controls for confounding variables: permutate data in such a way that preserves confounding of $x$ and $z$ (confouding variable) and $y$ and $z$. 
	\end{itemize}
\end{itemize}

Strategies of quality control: 
\begin{itemize}
	\item Data filtering: often the first step is remove dubious data points, outliers. Example: 
	\begin{itemize}
		\item GWAS: filter SNPs by HWE, by AF. 
	\end{itemize}
	
	\item Proper normalization of data: controlling for confounders, data transformation, etc.  
	
	\item Negative control: generally needed to obtain the ``background'' - how would data look like if there is no signal. 
	
	\item Positive control: can we find the signals that we are supposed to find? 
	
	\item Metrics for QC: what do we expect if the data is good and data processing procedure is working? What are the expected properties of things that we are trying to find. Examples:
	\begin{itemize}
		\item GWAS: few causal SNPs, thus the QQ plot should be roughly linear. If a study finds SNPs, are they enriched in functional regions?  
		\item ChIP-seq: the peaks should be close to TCC, evolutionarily conserved, etc.  
	\end{itemize}
\end{itemize}

Physical/biological process-motivated models: 
\begin{itemize}
	\item Statistical mechanics: Markov random field. 
	
	\item Random walk/diffusion: Markov chains. 
	
	\item Network models: network flow, etc. 
\end{itemize}

\subsection{Statistical Problems}

Statistical theory: 
\begin{itemize}
\item LRT for non-nested hypothesis: what is the asymptotic distribution? A simple case is: $H_0: \theta = \theta_0$ vs. $H_1: \theta = \theta_1$. The proof of $\chi^2$ distribution depends on Taylor expansion around MLE, however, for non-nested hypothesis, the parameters (under two hypothesis) are not necessarily close. 

\item Model identifiability: in a complex model, e.g. hierarchical model with missing data, the identifiability is not obvious. How to formally analyze the model identifiability? Bayesian approach using the posterior distribution/sample of $P(\theta|D)$ may be the solution. 

\item Information theoretical approach to inference: it can be shown that minimizing KL divergence is equivalent to MLE. Similarly, we could use information theory for hypothesis testing, e.g. to test $H_0: \theta = \theta_0$ vs. $H_1: \theta = \theta_1$, we may want to compare $KL(\hat{f}||f_{\theta_0})$ and $KL(\hat{f}||f_{\theta_1})$, where $\hat{f}$ is the empirical distribution. 

\item Evaluating estimators: Ex. for parameter estimation problems, the unbiased estimator with minimum variance can be thought of as the optimal one. In general, what is the theoretical framework of assessing estimators, say statistical decision theory? Can we prove that our common estimator (e.g. MLE) are optimal under reasonable loss functions?  

\item Performance analysis of statistical learning: a general problem in statistical learning (similar to estimator evaluation) how to evaluate the performance of a method and comparison of different methods, and can we prove optimality in some cases. 
\begin{itemize}
	\item Ex. the error of the KDE method; the MSE of linear classifiers such as SVM; etc.
	\item Ex. Lasso regression, why Lasso penalty is good (how the performance may depend on the correlation structure of independent variables)? Can we prove it is optimal under some assumptions (e.g. a small number of true variables and they are generally uncorrelated with each other)? This may be analyzed using bias-variance decomposition, Equation~\ref{eq:EPE_bias_variance}.  
\end{itemize}

\item Statistical analysis of algorithms: in machine learning, we often have an algorithm of doing things, say regression or classification (e.g. Lasso). How do we analyze the statistical properties of these algorithms: what is the rate of false positive findings (e.g. for Lasso, what is the FDR of the selected variables)? 
   
\item Learning important features: when features are correlated, testing individual coefficients or even groups of coefficients by $F$ test is insufficient. Ex. $X_1$ and $X_2$ are correlated, then removing $X_1$ would have a small impact on SSE. What is the best way of learning feature importance, taking into account the feature correlations? 

\item Bayesian statistics: introduce prior distributions would reduce the variance of the learned model (informative prior vs. uniform/noninformative prior)? 
\end{itemize}

Bayesian statistics: 
\begin{itemize}
\item Types of priors: Jeferreys, objective, etc. And how they may influence the inference if the priors are not ``fully anchored in past experience'' [Efron, A Two-Hundred-and-Fifty-Year Argument]? 

\item Frequentist behavior of Bayesian inference/posterior: a genearl question is that as we increase sample size, how would the posterior converge/behave? 
\begin{itemize}
	\item Example: suppose we have a linear regression model, and we have two correlated variables $X_1$ and $X_2$. Intuition: if the two are highly correlated, we need a big sample to learn the separate effects - posterior converge to one peak for each of the coefficients. How do we formalize such intuition? 
\end{itemize} 

\item Parametric bootstrap as approximation of Bayesian [Efron, A Two-Hundred-and-Fifty-Year Argument]? 

\item Derivation of Bayesian information criterion? 

\item Related to Variational inference: if a distribution can be factorized, what does it say geometrically in terms of the contour plot of the PDF? 
\begin{itemize}
	\item Remark: for MVN, independence means orthogonality (of eigenvectors). In general, do we have something similar (orthogonality)? 
\end{itemize}

\item Convergence of the variational inference algorithm? The relationship to convexity [Bishop, Chapter 10]
\end{itemize}

Statistical models: 
\begin{itemize}
\item Two sample comparison: e.g. differential expression between two samples. How do we control the hidden confounders? More generally, use a linear model to study the effect of $X$ on $Y$, what's the impact of missing confounders? (False associations: the effect can be explained by the hidden confounders, but we falsely attribute the effect to $X$). 

\item Linear regression: suppose our goal is to test if an independent variable is associated with a response. Does including additional variables (covariates) increase the power of test or reduce the false positives? Intuitively, stratification on the covariates, and testing the variable. 

\item Linear regression: do we gain power (of testing $\beta$) if we model $P(x,y)$, instead of $P(y|x)$? 

\item Linear regression with multiple testing: suppose we have a complex testing/model selection problem, e.g. whether a subset of parameters are 0, vs. only one parameter is 0. And among many such tests, we want to estimate the fraction of each scenario. An example: the effect of something (treatment, SNP) on gene expression could be: tissue-specific, or all-tissue, or 0 for all-tissues. 

\item ANOVA: can the idea be generalized to, e.g. non-linear models, for selecting variables or dimensionality reduction, etc.?  

\item Lasso: does group Lasso helps aggregate statistics across multiple members of a group to increase power? If formulating in hypothesis testing terms, what is the criteria by Lasso to select features? Guess: group Lasso, to choose a group (suppose there are no other groups), is effectively comparing two hypothesis: $H_0: \beta_j = 0, \forall j$, and $H_A$: for some $j$, $\beta_j \neq 0$. Thus this is similar to Hotelling's $T^2$ test. 

\item Logistic regression: how standard errors of coefficients are computed? 

\item PCA: can we have a model where we have both latent variables and known covariates? In the typical gene expression data analysis, PCA first, then correct for PCs and other covariates. Can we correct covariates and do PCA at the same time? 

\item Hierarchical models: as a tool to deal with heterogeneity in the data. How would this compare with mixture model? Ex. among $K$ groups, instead of modeling some parameter $\beta_k$ for each group as a sample from some common distribution, we could also have a mixture model: some groups have parameter $\beta_1$ and the others with $\beta_2$. 

\item Hierarchical models: how to model the network (as opposed to group) structure of samples? How to model the overlapping groups? 

\item Markov random fields: only certain conditional independence conditions are satisfied, a probability distribution can be called a MRF. Why?

\item Structural preference in regression and Bayesian networks: in a regression setting, it may be helpful if put constraints on $X_j$'s, e.g. certain pairs of features may have similar coefficients. There is a similar problem in learning Bayesian networks, where one may prefer how the nodes are linked. In general, the constraints may take the form: certain variables should be grouped (similar explanatory or outcome variables), similarity between pairs of variables, etc.

\item Soft partitioning: in partition-based methods, how to perform soft partitioning? Regularization on partitions: so that the models in related groups are also closer? Combining partitioning with prediction: otherwise, partitioning may not be relevant to the prediction task. 

\item Sample imbalance in nearest neighbor methods: when there is an imbalance in the samples, these methods such as KNN easily give biased results. What is the source of this problem and how to address it?

\item Justification of local likelihood methods? 

\item Physical analogy of Markov random field: what is a continuous version of Markov random field? Can we view function $f(x)$ as the state at any point over the space, e.g. the particle density, or potential at the point $x$. Can this perspective be applied to other problems, e.g. curve fitting? 

\item Causal inference: how to translate correlations to causality? In two group comparison, if two groups are randomized in every other aspect except the test factor, then then difference between the two groups must be due to the test factor. In general, if $X \rightarrow Y$, then when $X$ varies, $Y$ should also varies; for a potential confounder, its variation is generally random wrt. $X$. 

\end{itemize}
%%%%%%%%%%%%%%%%%%%%%%%%%%%%%%%%%%%%%%%%%%%%%%%%%%%%%%%%%%%%
\section{Probability Theory}

Expectation and variance: 
\begin{itemize}
\item Total expectation and variance: Suppose we are interested in $\E(X)$ and $\Var(X)$ for some random variable $X$. The conditional expectation and variance of $X$ under given $Y$ are easier to find, so we can express $\E(X)$ and $\Var(X)$ in terms of the conditional expectation and variance. 
\begin{equation}
\E(X) = \E_Y (\E(X|Y)))
\end{equation}
\begin{equation}
\Var(X) = \E_Y(\Var(X|Y)) + \Var_Y(\E(X|Y))	
\end{equation}
\end{itemize}

Approximating a distribution: 
\begin{itemize}
\item Motivation: how do we approximate a probabilty distribution (or summariz a set of numbers) using a single number? 

\item Theorem: let $x_1, \cdots, x_n$ be any numbers, for any number $c$, we have: 
\begin{equation}
\label{eq:min_mean}
\sum_{i=1}^n (x_i - c)^2 = \sum_{i=1}^n (x_i - \bar{x})^2 + n (\bar{x} - c)^2
\end{equation}
The number $c$ that minimizes the sum of squared deviation (SSD) is $\bar{x}$. \\

\item Theorem (continuous RV with $L_2$ loss): let $X$ be a RV, for any constant $c$, we have: 
\begin{equation}
\E(X-c)^2 = [\E(X) - c]^2 + \Var(X)
\end{equation}
A corollary of this theorem is: the constant $c$ that minimize $\E(X-c)^2$ is $\E(X)$. 

\item Theorem: let $x_1, \cdots, x_n$ be any numbers, for any number $c$, we define the sum of absolute deviation (SAD) of $x_i$ to $c$ as:
\begin{equation}
\label{eq:min_median}
\text{SAD}(c) = \sum_i \lvert x_i - c \rvert	
\end{equation}
The number $c$ that minimizes SAD is the median (if $n$ is even, then any point between the two middle elements is fine). \\
Proof: consider all possible interval where $c$ may fall into (thus removing the absolute sign). 

\item Theorem (continuous RV with $L_1$ loss): let $X$ be a RV, then the median of $X$ minimizes $\E(\lvert X - c \rvert)$. 

\item Theorem (discrete RV): Let $X$ be a discrete RV with $K$ possible values $g_1, \cdots, g_K$, define a $0/1$ loss function for any categorical value $c$: $L(g_k, c) = 0$ if $c = g_k$ and 1 otherwise, and the deviation from $c$ is defined as: 
\begin{equation}
D(c) = \sum_{k=1}^K L(g_k, c) p_k	
\end{equation}
It is easy to show that $c$ that minizes $D(c)$ is: $ \hat{c} = \text{argmax}_{k} p_k$. 


\item \textbf{Remark}:
\begin{itemize}
	\item Decomposition of deviation: it has two components: (1) departure of the mean to the number and (2) the variance. This idea is generally applicable for analyzing the errors/variance. 
	\item Approximation: the general idea of using something simpler distribution to approxiate a more complex distribution, e.g. normal distribution to approximate any RV that is bell-shaped. To define an approximation problem, need: a loss function and averaging. 
	\item Approximation perspective in different contexts: (1) point estimate of a probability distribution: since the functional form is known, only parameter value matters, thus use MSE as loss function; (2) prediction problem: loss function defined on the joint distribution of $(X,Y)$; (3) approximation probability distribution: KL divergence. 
\end{itemize}

\end{itemize}

Markov's Inequality: [Wiki]
\begin{itemize}
\item Theorem: if $X$ is an non-negative RV and $a > 0$, then 
\begin{equation}
P(X \geq a) \leq \frac{E(X)}{a}	
\end{equation}

\item Proof: let $f(x)$ be the PDF of $X$, 
\begin{equation}
a \cdot \int_a^{+\infty} a f(x) dx \leq \int_a^{+\infty} x f(x) dx \leq \int_0^{+\infty} x f(x) dx = E(X)
\end{equation}

\item Remark: the intuition is that for any non-negative RV, it cannot be too large, and this upper bound depends on its expectation (of course, the higher the expectation is, the more likely $X$ is large). 
\end{itemize}

Chebyshev's inequality: 
\begin{itemize}
\item Intuition: suppose $X$ has a finite variance, the departure of $X$ from its mean should depend on the variance: if the variance is small, small departure. We could define a bound of the departure using the variance. 

\item Theorem: $X$ is a random variable with mean $\mu$ and variance $\sigma^2$, then we have: 
\begin{equation}
P(\lvert X - \mu \rvert \geq k \sigma ) \leq \frac{1}{k^2}	
\end{equation}

\item Proof 1: we consider only the case of $\mu = 0$. The idea is with large departure, i.e. large $x$, $x^2 f(x)$ integration can be large, but this integral is bounded by $\sigma^2$. We have: 
\begin{equation}
\sigma^2 = \int_{-\infty}^{+\infty} x^2 f(x) dx \geq \int_{-\infty}^{-k\sigma} k^2 \sigma^2 f(x) dx + \int_{-k\sigma}^{+k\sigma} x^2 f(x) dx + \int_{+k\sigma}^{+\infty} k^2 \sigma^2 f(x) dx
\end{equation}
The RHS is larger than or equal to $k^2 \sigma^2 P(\lvert x\rvert \geq k\sigma)$. 

\item Proof 2: we can also apply Markov's Inequality to the random variable, $Y = \lvert X - \mu \rvert / \sigma$. 
\end{itemize}

Functions of random variables: 
\begin{itemize}
\item Application of Change of Variable Theorem: suppose we have $n$-dim. random variable $X$, and $Y = \phi(X)$ be a function of $X$ also in $n$-dim. We consider the probability mass $f_Y(y) dy$ near $y$, where $f_Y$ is the pdf. of $Y$. Under the mapping $\phi^{-1}$, the volume of the region $dy$ becomes $\det D \phi^{-1}(y) dx$ where $D$ is the derivative (Jacobian) of $\phi^{-1}$. The probability mass should be equal, thus we have (eliminating $dx$): 
\begin{equation}
f_Y(y) = f_X(\phi^{-1}(y)) \abs{\det D \phi^{-1}(y)}
\end{equation}

\item Unequal dimensions: when the dimensions of $X$ and $Y$ are not equal, e.g. we know the joint distribution of $(X,Y)$, and want to find the distribution of $g(X,Y)$, we could add additional auxilary variables s.t. the dimensions are equal. In this example, we could define: 
\begin{equation}
U = g(X,Y) \qquad V = Y	
\end{equation}
And apply the Theorem on the mapping $(X,Y) \rightarrow (U,V)$. 
\end{itemize}

Moment generating functions (MGF): 
\begin{itemize}
\item MGF: a function may be characterized (or even defined) via all of its moments, thus we could define a generating function of the moments, and the original function can be studied using this MGF. Defintion: MGF of a random variable $X$:
\begin{equation}
M_X(t) = \sum_{k=0}^{\infty} \frac{\E(X^k)}{k!} t^k = \E[e^{tX}]	
\end{equation}
Note that the last step comes from the Taylor expansion and we add the coefficients $1/k!$ so that the MGF is an expectation. For continuous RVs, we have:
\begin{equation}
M_X(t) = \int_{-\infty}^{+\infty} e^{tx} f_X(x) dx
\end{equation}
If $X$ is discrete: 
\begin{equation}
M_X(t) = 	\sum_k P(X = k) e^{kt}
\end{equation}
From the Taylor expansion above, it is easy to see that the $k$-th moment is the $k$-th derivative of the MGF at $0$. 
\begin{equation}
\E(X^k) = M_X^{(k)}(0)	
\end{equation}
Note that the MGF of a RV may not exist if the series does not converge. 

\item Uniqueness: it is possible to have two differnet distributions with exactly the same sets of moments. The following conditions uniquely define a distribution: 
\begin{itemize}
	\item If $X$ and $Y$ have bounded support, then $F_X(u) = F_Y(u)$ for all $u$ if and only if $\E(X^r) = \E(Y^r)$ for all integers $r = 0, 1, \cdots$. 
	\item If the MGF exist and $M_X(t) = M_Y(t)$ for all $t$ in a neighborhood of 0, then $F_X(u) = F_Y(u)$ for all $u$. 
\end{itemize}

\item Convergence: under some conditions, the convergence of MGF implies the convergence of CDF. Suppose 
\begin{equation}
\lim_{n \to \infty} M_{X_n}(t) = M_X(t) \qquad \text{for all $t$ iin a neighborhood of 0}	
\end{equation}
then there exists a unique CDF $F_X$ whose moments are determined by $M_X(t)$ and for all $x$, we have: 
\begin{equation}
\lim_{n \to \infty} F_{X_n}(t) = F_X(t)
\end{equation}

\item Remark: the proof of uniqueness and convergence Theorems rely on the theory of Laplace transform. 
\end{itemize}

Properties of MGF: 
\begin{itemize}
\item Linear function of RVs: the MGF of the random variable $aX + b$ is given by: 
\begin{equation}
M_{aX +b}(t) = e^{bt} M_X(at)	
\end{equation}

\item Sum of independent RVs: $X$ and $Y$ are independent RVs, then
\begin{equation}
M_{X+Y}(t) = M_X(t) M_Y(t)	
\end{equation}
\end{itemize}

\subsection{Convergence of Random Variables}

Reference: [Casella, Statistical Inference, Chapter 5]

Conceptual foundation of frequentist statistics: 
\begin{itemize}
\item Estimator behavior: the basic idea is that estimator is a random variable indexed by $n$ (the sample size), and under frequentist statistics, we are interested in whether the estimator, $W_n$ (1) converges to the true parameter $\theta$, and (2) how fast/efficient the convergence is, e.g. the variance of $\sqrt{n} (W_n - \theta)$. 

\item Simple estimators: we start from the simplest case, that $\bar{X}_n$ is an estimator of $\mu$, and the behavior of this estimator. These are provided by WLLN and CLT. The most sophisticated estimators then can be built from sample mean, sample variance, etc. 

\item Idea of approximation: a distribution is characterized by its mean, variance, and other moments, and intuitively, the higher-order moments are less important. Ex. if $Z_n \to 0$, then the distribution of $Z_n$ is mostly determined by its variance, so we can study the convergence behavior of $Z_n$ through the behavior of its variance. 
\begin{itemize}
	\item Technically, this can be addressed using MGF or characteristic function. 
\end{itemize}

\item Remark: in real analysis, we define or approximate a function through an infinite sequence or series of functions. Similarly, we could study/approximate a random variable by a sequence of random variables, or conversely, study the convergence behavior of a sequence of random variables (e.g. large-sample behavior of estimators). 
\end{itemize}

Convergence in probability and Weak Law of Large Numbers (WLLN): 
\begin{itemize}
\item Convergence in probability: a sequence of RVs $\{X_n\}$ converges in probability to a RV $X$ if $\forall \epsilon > 0$, $P(\abs{X_n - X} < \epsilon) \to 1$ as $n \to \infty$. 
\begin{itemize}
	\item Consistency: a squence of the ``same'' sample quantity approaches a constant as $n \to \infty$. 
	\item Remark: convergence in probability means as $n \to \infty$, most of the probability mass is concentrated around $X$. It is a strong form of convergence, often used for convergence to a constant. For instance, if $X_n = X$, then clearly the sequence converges in distribution, but not in probability in general. 
\end{itemize}

\item Theorem (10.1.3): if $\{X_n\}$ satisfies: (1) $\E(X_n) \to \mu$, and (2) $\Var(X_n) \to 0$, then $X_n \to \mu$ in probability as $n \to \infty$. \\
Proof: by Chebshev's inequality, 
\begin{equation}
P(\abs{X_n - \mu} \geq \epsilon) \leq \frac{\Var(X_n)}{\epsilon^2}	
\end{equation}

\item Theorem: if $\{X_n\}$ convergence in probability to $X$, and $h$ is a continuous function, then the sequence $\{h(X_n)\}$ converges in probability to $h(X)$.  

\item Consistency of sample mean (WLLN): let $X_1, X_2, \cdots$ be i.i.d random variables with $\E(X_i) = \mu$ and $\Var(X_i) = \sigma^2 < \infty$. Then $\bar{X}_n$ converges in probability to $\mu$. \\
Proof: check the condition of the previous theorem, in particular, the variance of $\bar{X}_n$ is equal to $\sigma^2/n$, which converges to 0. 

\item Consistency of sample variance: let $X_1, X_2, \cdots$ be i.i.d random variables with $\E(X_i) = \mu$ and $\Var(X_i) = \sigma^2 < \infty$. Define: 
\begin{equation}
S^2_n = \frac{1}{n - 1} \sum_{i=1}^n (X_i - \bar{X}_n)^2	
\end{equation}
Then $S^2_n$ converges in probability to $\sigma^2$. 

\end{itemize}

Convergence in distribution and Central Limit Theorem (CLT): 
\begin{itemize}
\item Convergence in distribution: a sequence of RVs $\{X_n\}$ converges in distribution to $X$ if
\begin{equation}
\lim_{n \to \infty} F_{X_n}(x) = F_X(x)	
\end{equation}
at all points $x$ where $F_X(x)$ is contiuous, i.e. the CDF of $X_n$ converges pointwise to the CDF of $X$. 

\item Relationship to convergence in probability: if a sequence $\{X_n\}$ converges to probability in $X$, then it must converge to $X$ in distribution. The two are equivalent if $X$ is a constant. 

\item CLT: suppose $\{X_n\}$ is a sequence of i.i.d. random variables with $\E(X_i) = \mu$ and $\Var(X_i) = \sigma^2 < \infty$. Then 
\begin{equation}
\sqrt{n} (\bar{X}_n - \mu) / \sigma \to N(0,1) \text{ in distribution}	
\end{equation}
Proof: we consider the case where the MGF of $X_n$ exists in a neighborhood of 0. Let $Y_i = (X_i - \mu) / \sigma$, and $M_Y(t)$ be the common MGF of $Y_i$, then the MGF of our target RV can be expressed as: 
\begin{equation}
M_{\sqrt{n} (\bar{X}_n - \mu)/\sigma}(t) = \left[ M_Y\left(\frac{t}{\sqrt{n}}\right)\right]^n	
\end{equation}
Note that as $n \to \infty$, $t/\sqrt{n} \to 0$, thus near 0, the above MGF can be approximated by Taylor expansion, noting that $M_Y(0) = 1, M_Y'(0) = 1, M_Y''(0) = 1$: 
\begin{equation}
\left[ M_Y\left(\frac{t}{\sqrt{n}}\right)\right]^n \approx \left( 1 + \frac{t^2}{2n}\right)^n	\to e^{t^2/2}
\end{equation}
\end{itemize}
%%%%%%%%%%%%%%%%%%%%%%%%%%%%%%%%%%%%%%%%%%%%%%%%%%%%%%%%%%%%
\subsection{Random Vectors}

Expectation and variance of random vectors: 
\begin{itemize}
	\item Ref: \url{http://www.statpower.net/Content/313/Lecture%20Notes/MatrixExpectedValue.pdf}
		
	\item Expectation: let $x$ and $y$ be $n$-dim. random vectors. Suppose $A$ is a given matrix, then: 
	\begin{equation}
	\E(Ax) = A \E(x) \qquad \E(x + y) = \E(x) + \E(y) \qquad \E(x^T) = (\E(x))^T
	\end{equation}
	The implication is that we can do expected value algebra for matrices, e.g. 
	\begin{equation}
	\E(A B x y C) = A B \cdot \E(x y) \cdot C
	\end{equation}
	
	\item Covariance matrix: suppose we have a $n$-dim. random vector (column) $x$, with $\E(x) = \mu$. The covariance matrix of $x$ is given by: 
	\begin{equation}
	\Cov(x) = \left[ \Cov(x_i,x_j) \right]_{n \times n}	= \E[(x_i - \mu_i) (x_j - \mu_j)] = \E[(x-\mu) (x -\mu)^T] = \E(x x^T) - \mu \mu^T
	\end{equation}
	
	\item Covariance of two random vectors: let $x$ and $y$ be two random vectors of $m \times 1$ and $n \times 1$ respectively. The covariance of $x$ and $y$ is $m \times n$ matrix defined by:
	\begin{equation}
	\Cov(x, y) = \E(x y^T) - \E(x) \E(y)^T
	\end{equation}
	Suppose $A$ and $B$ are two matrices that can multiply with $x$ and $y$, then we have:
	\begin{equation}
	\Cov(Ax, By) = A \Cov(x, y) B^T
	\end{equation}
	
	\item Linear transformation of random vector: suppose $x$ is a $n \times 1$ random vector with mean $\mu$ and variance $\Sigma$, and $A$ a matrix $m \times n$, then we have: 
	\begin{equation}
	\E(Ax) = A \mu \qquad \Var(Ax) = A \Sigma A^T
	\end{equation}
	Note: this result is true even if $x$ is not normally distributed.\\
	Proof: we use the fact that $\Sigma = \E(x x^T) - \mu \mu^T$: 
	\begin{equation}
	\Var{Ax} = \E((Ax)(Ax)^T) - (A\mu)(A\mu)^T = A \E(x x^T) A^T - A \mu \mu^T A^T = A \Sigma A^T. 
	\end{equation}
	Similarly, we have: 
	\begin{equation}
	\E(x^T A) = \E(x^T) A \qquad \Var(x^T A) = A^T \cdot \Var(x^T) \cdot A
	\end{equation}
	
	\item Covariance of random variables that are linear functions of a random vector: let $G$ be a random vector, and $u, v$ are vectors (constants). Let $X = G^T u$ and $Y = G^T v$ (both are scalar), then the sample covariance between the two is:
	\begin{equation}
	\Cov(X, Y) = \Cov(\sum_i G_i u_i, \sum_i G_j v_j) = \sum_{i,j} u_i v_j \Cov(G_i, G_j) = u^T \Cov(G) v
	\end{equation}
	where $\Cov(G)$ is the covariance matrix of the random vector $G$. We also have: the variance $\Var(X) = u^T \Cov(G) u$.\\
	Remark: this is useful in the case of MR and TWAS, where $X$ and $Y$ are exposure and outcome, respectively. 
\end{itemize}

Quadratic form of random vectors: 
\begin{itemize}
	\item Ref: \texttt{quadratic-form-random-vector.pdf}. 
	
	\item Theorem: $x$ is $n$-dim. random vector with mean $\mu$ and variance $\Sigma$, and $A$ is a symmetric matrix, we have: 
	\begin{equation}
	\E(x^T A x) = \tr(A \E(x x^T)) = \tr(A \Sigma) + \mu^T A \mu
	\end{equation}
	
	\item Proof: we first use the fact that scalar of trace is just the scalar, and the fact that scalar and expectation could commute: 
	\begin{equation}
	\E(x^T A x) = \tr(\E(x^T A x)) = \E(\tr(x^T A x))
	\end{equation}
	Now we use $\tr(AB) = \tr(BA)$ for any two matrices: 
	\begin{equation}
	\E(x^T A x) = \E(\tr(A x x^T)) = \tr(\E(A x x^T)) = \tr(A \E(x x^T))
	\end{equation}
	This completes the first part of the theorem. For the second part, we use the covariance matrix of $x$: $\Sigma = \E(x x^T) - \mu \mu^T$: 
	\begin{equation}
	\E(x^T A x) = \tr(A \cdot (\Sigma + \mu \mu^T)) = \tr(A\Sigma) + \tr(A \mu \mu^T) = \tr(A\Sigma) +  \mu^T A \mu
	\end{equation}
	where we use: $\tr(A \mu \mu^T) = \tr((A\mu) \cdot \mu^T) = \tr(\mu^T A \mu) = \mu^T A \mu$. 
\end{itemize}

%%%%%%%%%%%%%%%%%%%%%%%%%%%%%%%%%%%%%%%%%%%%%%%%%%%%%%%%%%%%
\section{Probability Distributions}

Reference: [Casella, Statistical Inference, 5.2]

Negative Binomial distribution: 
\begin{itemize}
	\item Ref: Wiki and  \url{https://probabilityandstats.wordpress.com/tag/poisson-gamma-mixture/}
	
	\item Waiting time in Bernoulli process (Pascal distribution): when $r$ is an integer, NB is the number of successes before the $r$-th failure in a Bernoulli process, with probability $p$ of successes on each trial. When $r=1$, this is geometric distribution. The PMF of NB: 
	\begin{equation}
	P(X=k|r,p) = \binom{k+r-1}{k} p^k (1-p)^r
	\end{equation}
	The expectation and variance of $X$: 
	\begin{equation}
	\E(X) = \frac{p r}{1-p} \qquad \Var(X) = \frac{p r}{(1-p)^2} \qquad \frac{\Var(X)}{\E(X)} = \frac{1}{1-p}
	\end{equation}
	The last term is the ``index of overdispersion'', and it depends only on $p$. When $p$ is small, the index is close to 1, and the distribution is close to Poisson. 
	
	\item Poisson-Gamma mixture: if $X | \theta \sim \text{Pois}(\theta)$ and $\theta \sim \text{Gamma}(\alpha, \beta)$, where $\alpha, \beta$ are shape and rate parameter, respectively, then $X \sim NB(r = \beta, p = \frac{\alpha}{\alpha+1})$. 
	
	\item Parameterization in NB regression: in regression problem, we parameterize with mean, which depends on covariates. Typically, we use $X \sim NB(\mu, \theta)$, where $\mu = \E(X)$, and $\theta$ is the overdispersion parameter defined by $\Var(X) = \mu + \theta \mu^2$. Some authors parameterize using overdispersion parameter as $1/\theta$. 
\end{itemize}

Gaussian integral:
\begin{itemize}
	\item Gaussian integral: 
\begin{equation}
\int_{-\infty}^{\infty} e^{-x^2} dx = \sqrt{\pi}	
\end{equation}
	\item Proof: compute the following integral by double integration: 
\begin{equation}
\int_{-\infty}^{\infty} e^{-(x^2+y^2)} dx dy = 	\left(\int_{-\infty}^{\infty} e^{-x^2} dx\right)^2
\end{equation}
The LHS can also be computed using the polar coordinate transformation. Equating the two gives the equation. 
\end{itemize}

Sample mean and sample variance: parameter estimation for $N(\mu, \sigma^2)$. Suppose $X_1, \cdots, X_n$ iid $N(\mu, \sigma^2)$: 
\begin{itemize}
\item Sample mean: the estimator of $\mu$: $\bar{X} \sim N(\mu, \sigma^2 /n)$. 
\item Sample variance: the estimator of $\sigma^2$ is the mean squared error: 
\begin{equation}
S^2 = \frac{1}{n-1} \sum_{i=1}^n (X_i - \bar{X})^2
\end{equation}
Intuition: $S^2$ is the estimator of the variability in the sample, if $\mu$ is known, then $S^2$ should be the mean variance (divided by $n$); since $\bar{X}$ is used, the $n$ terms are subject to one constraint (sum to 0), thus the total variability is slightly less (thus divided by $n - 1$). 
%$S^2$ is independent of $\bar{X}$ and $(n-1) S^2 / \sigma^2 \sim \chi^2_{n-1}$. \\

\item Theorem: given $x_1, \cdots, x_n$, we have
\begin{equation}
(n-1) s^2 = \sum_{i=1}^n (x_i - \bar{x})^2 = \sum_{i=1}^n	x_i^2 - n \bar{x}^2
\end{equation}
Proof: take $c = 0$ in the Equation~\ref{eq:min_mean}. 

\item Theorem: $E(\bar{X}) = \mu$, $\text{Var}(\bar{X}) = \sigma^2 /{n}$, $E(S^2) = \sigma^2$. \\
Proof: the first two can be easily proven with the iid. property of the random sample. For the last, using the Theorem above: 
\begin{equation}
\sum_{i=1}^n (X_i - \bar{X})^2 = 	\sum_{i=1}^n (X_i - \mu + \mu - \bar{X})^2 = \sum_{i=1}^n (X_i - \mu)^2 - n(\bar{X} - \mu)^2
\end{equation}

\item Theorem: $\bar{X}$ and $S^2$ are independent RVs, $\bar{X} \sim N(\mu, \sigma^2/n)$ and $(n-1)S^2 / \sigma^2 \sim \chi^2_{n-1}$. 
\begin{itemize}
	\item Independence of $\bar{X}$ and $S^2$: define $Y_1 = \bar{X}$, $Y_2 = X_2 - \bar{X}, \cdots, Y_n = X_n - \bar{X}$, and $S^2$ can be expressed as a function of $Y_2, \cdots, Y_n$. It can be shown that $Y_1 = \bar{X}$ is indepenent of $Y_2, \cdots, Y_n$: (1) by apply the linear transformation to the joint pdf. of $X_1, \cdots, X_n$); (2) using the fact that both $\bar{X}$ and $X_j - \bar{X}$ are linear functions of $X_1, \cdots, X_n$ (calculate the covarance of the two RVs). 
	\item Distribution of sample variance: proof by induction. Wlos., assume that $\mu = 0$ and $\sigma = 1$. First, at $n = 2$, $S_2^2 = (X_2 - X_1)^2 / 2$, and it is easy to show that $S_2^2$ follows $\chi^2_1$. Then at $n = k + 1$, $k S_{k+1}^2$ can be decomposed as a sum of two $\chi^2$ distributions: $(k-1)S_k^2$ and $(X_{k+1} - \bar{X}_k)^2$ (up to a constant). 
	\item Remark: when $\mu$ is known, $(n-1)S^2/\sigma^2 = \sum_i (\frac{X_i - \mu}{\sigma})^2$ is a sum of $n$ i.i.d standard normal RVs, thus follows $\chi^2_{n}$ distribution. When $\mu$ is unknown, needs some correction, but still $\chi^2$ distribution. 
\end{itemize}

\item $Z$-score: when $\sigma^2$ is unknown, the distribution of $\bar{X}$ is unknown. Thus using $S$ instead of $\sigma$: 
\begin{equation}
\sqrt{n} (\bar{X} - \mu) / S \sim t_{n-1}	
\end{equation}
This would allow one to determine the confidence interval of $\mu$. 

\item Evaluating the estimators: from the previous theorems, $\bar{X}$ and $S^2$ are unbiased estimators of $\mu$ and $\sigma^2$. The variance of $\bar{X}$ is $\sigma^2/n$, and the variance of $S^2$ can be obtained from the $\chi^2$ distribution: 
\begin{equation}
\Var((n-1)S^2 / \sigma^2) = \frac{(n-1)^2}{\sigma^4} \Var(S^2) = 2 (n-1) \Rightarrow \Var(S^2) = \frac{2 \sigma^4}{n-1}
\end{equation}
Thus, both $\bar{X}$ and $S^2$ converge to the true values of the parameters at the rate of $1/n$. 

\end{itemize}

Properties of normal distribution: 
\begin{itemize}
\item Sum of normal random variables: suppose $X_i \sim N(\mu_i, \sigma_i^2)$, then
\begin{equation}
\sum_i w_i X_i \sim N(\sum_i w_i \mu_i, \sum_i w_i^2 \sigma_i^2) 
\end{equation}

\item Product of normal density functions: suppose we have $N(x|\mu_1, \sigma_1^2)$ and $N(x|\mu_2, \sigma_2^2)$, the product of the density functions: 
\begin{equation}
N(x|\mu_1, \sigma_1^2) N(x|\mu_2, \sigma_2^2)	= N(x|\mu, \sigma^2) N(\mu_1|\mu_2, \sigma_1^2 + \sigma_2^2)
\end{equation}
where: 
\begin{equation}
\mu = \frac{1/\sigma_1^2}{1/\sigma_1^2 + 1/\sigma_2^2} \mu_1 + \frac{1/\sigma_2^2}{1/\sigma_1^2 + 1/\sigma_2^2} \mu_2
\end{equation}
and 
\begin{equation}
\frac{1}{\sigma^2} = \frac{1}{\sigma_1^2} + \frac{1}{\sigma_2^2}  	
\end{equation}
The proof follows from the quadratic form of $x$. Take the integral of $x$: 
\begin{equation}
\int N(x|\mu_1, \sigma_1^2) N(x|\mu_2, \sigma_2^2) dx = N(\mu_1|\mu_2, \sigma_1^2 + \sigma_2^2)
\label{eq:normal_mixture}
\end{equation}

\item Mixture of normal random variables: suppose we have $X|\theta \sim N(\theta, \sigma^2)$ and $\theta \sim N(\mu, \tau^2)$, then the marginal distribution: 
\begin{equation}
X \sim N(\mu, \sigma^2 + \tau^2)	
\end{equation}
To see this, we write the marginal as the integral of $N(\theta|x,\sigma^2) N(\theta|\mu, \tau^2)$ over $\theta$, and apply the Equation~\ref{eq:normal_mixture} above. 
\end{itemize}

Pooled variance: 
\begin{itemize}
\item Problem: suppose there are $K$ groups, each group is from a normal distribution, with different mean but the same variance (e.g. additional dependent variable that changes mean, but not variance), and our goal is to estimate the variance by pooling all groups. 

\item Pooled variance [Wiki]: suppose the sample variance of the $i$-th group is $S_i^2$, and the sample size of the $i$-th group is $n_i$, then the estimated variance $S_p^2$ is given by: 
\begin{equation}
S_p^2 = \frac{\sum_i (n_i - 1) S_i^2 }{\sum_i (n_i - 1)}	
\end{equation}
\end{itemize}

Multinomial distribution: 
\begin{itemize}
\item Suppose $X_1, \cdots, X_k \sim \text{Mul}(n; p_1, \cdots, p_k)$, where $\sum_{i=1}^k p_i = 1$. The pmf is given by: 
\begin{equation}
P(x_1, \cdots, x_k) = \frac{n!}{x_1! \cdots x_k!}	p_1^{x_1} \cdots p_k^{x_k}
\end{equation}

\item Properties: $E(X_i) = n p_i$. The covariance matrix is given by: 
\begin{equation}
\text{Var}(X_i) = n p_i (1 - p_i) \qquad \text{Cov}(X_i, X_j) = -n p_i p_j
\end{equation}
Proof: the variance is easy to prove using binomial distribution. For the covariance, use proof by induction: simple at $k = 2$; at larger $k$, reduce by letting $X_k + X_{k+1}$ as a single RV, and apply the induction hypothesis. 
\end{itemize}

Gamma and inverse gamma distributions: 
\begin{itemize}
\item Gamma distribution: defined on $x \geq 0$
\begin{equation}
f(x;\alpha,\beta) = \frac{\beta^{\alpha}}{\Gamma(\alpha)} x^{\alpha - 1} e^{-\beta x}	
\end{equation}
where $\alpha$ and $\beta$ are called the shape and rate(scale) parameter, respectively. 

\item Inverse gamma distribution: if $X$ is gamma RV, then $1/X$ follows inverse Gamma distribution. Its density:  
\begin{equation}
f(x;\alpha,\beta) = \frac{\beta^{\alpha}}{\Gamma(\alpha)} x^{-\alpha - 1} \exp\left(-\frac{\beta}{x}\right)
\end{equation}

\item Remark: a number of other distributions have this form of density function and special cases of gamma and inveser-gamma distributions. 
\end{itemize}

$\chi^2$ distribution: 
\begin{itemize}
\item Definition: the density of $\chi^2$ distribution with dof. equal to $\nu$: 
\begin{equation}
f(x;\nu) \propto x^{\nu / 2 - 1} \exp(-x/2)
\end{equation}

\item Theorem: if $Z \sim N(0,1)$, then $Z^2 \sim \chi^2_1$. 

\item Theorem: if $X_1, \cdots, X_n$ are independent, and $X_i \sim \chi^2_{p_i}$, then $X_1 + \cdots + X_n \sim \chi^2_{p_1 + \cdots + p_n}$. 

\item Mean and variance: if $X \sim \chi^2_k$, then $\E(X) = k$ and $\Var(X) = 2k$. 
\end{itemize}

Inverse-$\chi^2$ distribution: 
\begin{itemize}
\item Inverse $\chi^2$ distribution: if $X$ has the $\chi^2$ distribution with $\nu$ degrees of freedom, then $1 / X$ has the inverse-$\chi^2$ distribution with $\nu$ degrees of freedom. 
\begin{equation}
f(x;\nu) \propto x^{-(\nu / 2 + 1)} \exp\left(-\frac{1}{2x}\right)	
\end{equation}
Intuitively, with large degree, chi-square RV would have large mean, thus its inverse (inverse-$\chi^2$ RV) would have a large peak near 0. 

\item Scaled-inverse $\chi^2$ distribution: if $X \sim \text{Scaled-Inv-}\chi^2(\nu, \sigma^2)$, then $\frac{X}{\sigma^2 \nu} \sim \text{Inv-}\chi^2(\nu)$. Thus it is basically an inverse $\chi^2$ distribution with $\sigma^2$ as the unit. Its shape is determined by $\nu$, and its scale determined by $\sigma^2$ (large $\sigma$ means that the distribution is broader). 
\begin{equation}
f(x;\nu,\sigma^2) \propto x^{-(\nu / 2 + 1)} \exp\left( -\frac{\nu \sigma^2}{2x}\right)	
\end{equation}
\begin{itemize}
\item Mean: when $\nu > 2$, the mean is $\frac{\nu}{\nu - 2} \sigma^2$. 
\item Mode: $\frac{\nu}{\nu + 2} \sigma^2$. 
\end{itemize}

\item Remark: these are all special cases of inverse gamma distribution. 
\end{itemize}

Student's $t$ and $F$ distribution: 
\begin{itemize}
\item Student's $t$ distribution: if $U$ is a standard nomral distribution, $V$ is $\chi^2_p$, then $U/\sqrt{V/P}$ follows $t$ distribution with dof $p - 1$. 

\item Theorem: given a random normal sample, 
\begin{equation}
\frac{\bar{X} - \mu}{S/\sqrt{n}} \sim t_{n-1}	
\end{equation}
Proof: divide by $\sigma/\sqrt{n}$ in both the numerator and denomiator. 

\item $F$ distribution: if $U \sim \chi^2_p$ and $V \sim \chi^2_q$ and $U$, $V$ are independent, then $(U/p)/(V/q)$ follows $F$ distribution with dof $(p-1,q-1)$. 

\item Theorem: let $X_1, \cdots, X_n$ iid. $N(\mu_X, \sigma_X^2)$, and $Y_1, \cdots, Y_m$ iid. $N(\mu_Y, \sigma_Y^2)$ (independent of $X$), then:
\begin{equation}
\frac{S_X^2/\sigma_X^2}{S_Y^2/\sigma_Y^2}	\sim F_{n-1, m-1}
\end{equation}
Proof: use the $\chi^2$ distributions of $S_X^2$ and $S_Y^2$. 
\end{itemize}

Laplace distribution [Murphy, Section 2.4]
\begin{itemize}
	\item PDF: $\text{Lap}(x|\mu, b) = \frac{1}{2b} \exp\left(\abs{(x-\mu)}/b\right)$, where $\mu$ is the mean and $b$ scale parameter. The probability mass is more concentrated near 0, and has a large tail (similar to spike-and-slab). 
	
	\item Application in linear regression with outliers: normal error function is sensitive to outliers, which have large effect on the regression estimates. Using Laplace error function, the estimates are less sensitive. 
\end{itemize}

Log-normal distribution: 
\begin{itemize}
	\item Defintion: a random variable $X$ is log-normally distributed, if $Y = \log X$ is normally distributed. We write: 
	\begin{equation}
	X \sim LN(\mu, \sigma^2) \Leftrightarrow Y = \log X \sim N(\mu, \sigma^2) 
	\end{equation}
	
	\item Moments of log-normal distribution: if $X \sim LN(\mu, \sigma^2)$, we have: 
	\begin{equation}
	\E(X) = e^{\mu + \frac{1}{2}\sigma^2}
	\end{equation}
	The $s$-th moment, where $s$ is a real or complex number is given by: 
	\begin{equation}
	\E(X^s) = e^{s\mu + \frac{1}{2}s^2\sigma^2}
	\end{equation}
	
	\item Multivariate log-normal distribution: if $X$ is MNV $N(\mu, \Sigma)$, then $Y = \exp(X)$ follows multivariate log-normal, and the mean of $Y$ has a simple closed form: 
	\begin{equation}
	\E(Y_i) = e^{\mu_i + \frac{1}{1}\Sigma_{ii}}
	\end{equation}
	
	\item For computing the higher-order central moments of multivariate log-normal distribution: see Reference: A Recursive Formula for Computing Central Moments of a Multivariate Lognormal Distribution. 
\end{itemize}
%%%%%%%%%%%%%%%%%%%%%%%%%%%%%%%%%%%%%%%%%%%%%%%%%%%%%%%%%%%%
\section{Parameter Estimation}
\begin{enumerate}

\item{Methods of point estimation} 

Reference: [Casella, Statistical Inference, 7.2, 7.3]

Overview of parameter estimation: 
\begin{itemize}
\item Intuition: suppose $W$ is an estimator of $\theta$, then $W$ should contain information of $\theta$. Intuitively, different values of $\theta$ should lead to different (mean) values of $W$. Example, in normal distribution, $\bar{X}$ is informative of $\mu$ but not $\sigma^2$, as its expectation is equal to $\mu$ but independent of $\sigma$. 

\item The perspective of matching histogram: from the frequentist perspective, as the sample size approaches infinity, the empirical distribution should match the true distribution. Thus the parameter $\theta$ that leads to a match between $f_{\theta}$ and the empirical distribution $\hat{f}$ is a good estimator. Based on this perpsective, we have: 
\begin{itemize}
\item MOM estimator: match the moments of $f_{\theta}$ and the sample moments. 
\item MLE estimator: minimizing the KL divergence, $KL(\hat{f}||f_{\theta})$, gives the ML estimator (see the section on information theoretical perspective on statistical inference). 
\end{itemize}

\item Evaluating estimators: we need a way to quantify how much information $W$ contains on $\theta$. The important measures of an estimator are: whether it is unbiased, the variance of $W$, and how fast it converges to the true value $\theta$. In particular, $\Var(W)$ can be used to derive the confidence interval of $\theta$, thus important. 

\item Method of moment (MOM) estimation: the general idea is suppose we define an informative pattern $W$, i.e. $\E(W)$ is some function of $\theta$, $h(\theta)$, then we could use $h^{-1}(W)$ as an estimator of $\theta$. Often the patterns are mean, variance, covariance, but any other patterns follow the same idea. 

\item MLE: follows from the likelihood principle, that all information of $\theta$ is contained in the likelihood function $L(\theta)$. 

\item Error minimization for prediction problems: the idea is to minimize the difference between the model predictions and observations. We could define the error as the objective function to be minimized. For the least square method, let $f(\cdot)$ be our function, the parameters are estimated by: 
\begin{equation}
\min F(\theta) = \sum_i [y_i - f(x_i;\theta)]^2	
\end{equation}

\item Partial likelihood and conditional distributions: sometimes we do not have to model the complete likelihood of the data, especially if we are interested in only part of the parameters. We could then use part of the likelihood that contain the interested parameter; or use conditional distributions that is independent of nuisance parameters. 
\begin{itemize}
	\item Two-sample Poisson test: suppose we only know the ratio of $t_1/t_2$, then use the conditional distribution $P(x_1|x_1+x_2)$ has the advantage that it does not depend on the absolute values of $t_1$ and $t_2$. 
	
	\item TDT in genetics: using $P(G|G_p, Y)$ where $G, G_p$ are genotypes of child and parent, and $Y$ the phenotype of child, we can avoid parameters of genotype frequency in the population. 
\end{itemize}
\end{itemize}

Properties of point estimator: these are desirable properties of $\hat{\theta}_n$: 
\begin{itemize}
\item Unbiasedness: $E(\hat{\theta}_n) = \theta$. 
\item Sufficiency: $f(x_1, \cdots, x_n|\hat{\theta}_n)$ does not depend on $\theta$. 
\item Minimum MSE: $E[(\hat{\theta}_n - \theta)^2]$ is minimum. 
\item Minmum variance unbiased: $\text{Var}(\hat{\theta}_n) \leq \text{Var}(\hat{\theta}^*_n)$ for any other estimator $\hat{\theta}_n^*$. 
\end{itemize}

Procedure of point estimation:
\begin{itemize}
\item Estimator: for a given parameter estimation problem, find the estimator (MOM, MLE, etc.) of $\theta$, called $W$. 
\item Evaluating the estimator: using the Theorem of Point Approximation, the mean squared error (MSE) of an estimator $W$ is given by: 
\begin{equation}
\text{MSE}_{\theta}(W) = E_{\theta}(W-\theta)^2 = \left[E_{\theta}(W) - \theta\right]^2 + \text{Var}_{\theta}(W)	
\end{equation}
The subscript means MSE, bias and variance all depend on the true value of $\theta$. Since this is usually unknown, the evaluation of $W$ in a practical problem is performed on the estimated value of $\theta$, e.g. MLE. For unbiased estimator, the MSE is given by the variance of the estimator. 

\item Obtaining variance of estimator: and similarly for condifence interval. For difficult problem, one has some options: 
\begin{itemize}
\item Asymptotic results: e.g. asymptotic normality of MLE. 
\item Parametric bootstrap: suppose $\hat{\theta}$ is the estimator (often MLE) of $\theta$ from the data. We could simulate data many times assuming $\hat{\theta}$ is the true value of $\theta$, and compute the estimator $W$ for each data set (note: $W$ is the estimator of the paramter of interest). This allows one to obtain the distribution of $W$. 
\item Sampling with replacement (nonparameteric bootstrapping): the benefit is that it is independent of the assumption on the parametric form of the distribution. 
\end{itemize}

\item What determines the variance of an estimator? Sample size is a major determinant. In many other cases, the variance may be thought of some variation or effective sample size. 
\begin{itemize}
	\item Normal distribution: $X_i \sim N(\mu, \sigma^2)$, the variance of $\hat{\mu}$ is $\sigma^2 / n$.
	
	\item Poisson distribution: $X \sim \text{Pois}(\lambda t)$, then $\hat{\lambda} = X/t$. It follows Poisson distribution, and $\Var{\hat{\lambda}} = X / t$, where $t$ is similar to the sample size. 
	
	\item Simple linear regression: $\Var(\hat{\beta}_1) = \sigma^2 / \sum_i (X_i - \bar{X})^2$, where the variance of $X$ can be viewed as some effective sample size. Effectively, when $X_i = \bar{X}$, then sample $i$ does not contribute (no information of $\beta$). 
\end{itemize}
 
\item Remark: 
\begin{itemize}
	\item The decompoision of the MSE of the estimator is the result of Theorem of Point Approximation, the perspective here is opposite: we are not approximating a RV with a constant, but rather, estimate the error when we are infering an (unknown) constant using some distribution. 
	\item The general idea of MSE decomposition is: partition the error into two parts, one part depends on the truth (bias), and the other only depends on the property of the estimator itself (variance). 
\end{itemize}
\end{itemize}

Method of moment (MOM) estimation: 
\begin{itemize}
\item General strategy: suppose $T$ is some statistic (pattern) from data, if we have $\E(T) = h(\theta)$, then we could define $W = h^{-1}(T)$ as an estimator of $\theta$. If $h$ is linear, we have: 
\begin{equation}
\E(W) = \E[h^{-1}(T)] = h^{-1}(\E(T)) = \theta	
\end{equation}
Thus $W$ is an unbiased estimator of $\theta$. In general, if $h$ is not linear (e.g. convex), then the above inequality does not hold, but $W$ may still be an estimator. 

\item MOM: a special type of patterns in the data can be expressed as the tendency (mean) of variables and the relationship (covariance) between variables. To formulate this idea, we consider the joint distribution of random variables involved in the model, and analyze the population mean, variance and covariance of the random variables: (generally)
\begin{equation}
\E(X_i) = f_i(\theta) \qquad \Cov(X_i,X_j) = g_{ij}(\theta)	
\end{equation}
where $\theta$ is model parameters. If we can derive the sample mean, variance, covariance (that estimate the population quantities above), then we could equate these statistics with the functions $f_i$ and $g_{ij}$ above. 

\item Other examples of MOM strategy: 
\begin{itemize}
\item Regression: the conditional covariance of $Y$ on $X_j$ when other variables are fixed, carries information of $\beta_j$. 
\item SEM: a special case of MVN, the sample covariance matrix is equal to covariance matrix $\Sigma(\theta)$, which is a function of $\theta$. 
\item Markov chain: suppose we have a $k$-state Markov chain, the statistics such as: frequencey of each state, the mean length of sequential runs of each state, the frequency of transitions, are informative of the parameters of the Markov chain. 
\end{itemize}

\item Remark: the difficulties of MOM: 
\begin{itemize}
	\item Informative estimators: may not be obvious. Ex, for a HMM where the states are not observed, the informative statistics are not obvious, unlike the Markov chain case. 
	\item Non-linearity: suppose $\E(T) = h(\theta)$, if $h$ is nonlinear, then $W = h^{-1}(T)$ is not an unbiased estimator of $\theta$. 
\end{itemize}
\end{itemize}

Model identifiability: 
\begin{itemize}
\item Intuition/motivation: even if a statistical model is well-defined, the parameters may not be identifiable (see examples below). Intuitively, it is identifiable only when different model parameters lead to different distributions/data characteristics. The definition: a statistical model with parameters $\theta$ is identifiable if: 
\begin{equation}
P_{\theta_1} = P_{\theta_2} \Rightarrow \theta_1 = \theta_2 \qquad \forall \theta_1, \theta_2 \in \Theta	
\end{equation}
where $\Theta$ is the parameter space. 

\item Examples: 
\begin{itemize}
	\item Linear regression: if there is linear dependence among features, then the matrix $(X^T X)$ is singular, and the coefficients not identifiable. Intuitively, if $X_1$ and $X_2$ are correlated, then having large $\beta_1$, small $\beta_2$ would be indistinguishable from small $\beta_1$, large $\beta_2$. 
	\item Mixture model: two classes of sites with the mixing  $\theta$ unknown, some evolve faster $\alpha$, some slower $\alpha_0$. The model may not be identifiable, as having large $\theta$ and small $\alpha$ may be similar to having a smaller $\theta$, but larger $\alpha$. 
\end{itemize}

\item Model identification via the number of parameters: intuitively, if there are more parameters than the number of constraints/degree of freedom in the data, then the model is not identified. 
\begin{itemize}
	\item Example: linear regression, $Y = \beta_0 + \beta_1 X_1 + \beta_2 X_2$, $X_1$ is perfected correlated to $X_2$. Then the joint distribution of all variables has only one true parameter related to the covariance between $X_1$ (or $X_2$) and $Y$, but we have two model parameters $\beta_1$ and $\beta_2$, thus not identifiable. 
	\item Example: Error-in-variable (EIV) model. 
\end{itemize}

\item Model identification via symmetry: if the parameters are exchangable, then they may not be identifed. Ex. mixture model, the indices of components are clearly exchangable. 

\item Fisher information matrix: the parameter $\theta$ is identified at $\hat{\theta}$ if and only if the inverse of the information matrix, $I(\theta)$, exists at $\hat{\theta}$. 
\begin{itemize}
\item Intuition: at $\hat{\theta}$, the log-likelihood function has derivative 0. If the second derivative is also 0 (or singular information matrix), the the function is locally flat (hyperplane), thus the parameters are not identified.  
\item Alternatively, a small value of $I(\theta)$ implies a large value of $\text{Var}(\hat{\theta})$, at the extreme case of $I(\theta)$ is singular, this suggests that the variance is infinitely larger, i.e. the model is not identified. 
\end{itemize}

\end{itemize}

\item{Fisher information and Cramer-Rao lower bound}

Log-likelihood function and entropy: 
\begin{itemize}
\item Motivation: while we are generally inferring an unknown parameter $\theta$ from the data, we need to consider the fact that, different $\theta$ might generate data with simliar characteristics. So to infer the unknown parameter, we need to consider a family of distributions parameterized by $\theta$, and study how the data characteristics depend on $\theta$ - technically this is the likelihood function. 

\item Log-likelihood function and parameter identification: suppose we have data $x$, and we form the log-likelihood function $l(\theta) = \log f(x|\theta)$. In the region where $\log f(x|\theta)$ is flat, different values of $\theta$ could lead to the same data, so it is difficult to infer the true value of $\theta$. To characterize the intrisic difficulty of inference (instead of basing on a particular dataset), we should consider the ``flatness''/curvature averaged over $X$, or the curvature of the log-likelihood function when $n \to \infty$. Note: this is the average over the distribution parameterized by the same $\theta$. 

\item Uncertainty (information): if $f(X|\theta)$ is always 1, then there is no uncertainty, and in general, we can use the expectation of $f(X|\theta)$ as a measure of uncertainty: 
\begin{equation}
H(\theta) = \E(-\log f(X|\theta)) = -\int f(x|\theta) \log f(x|\theta) dx
\end{equation}

\end{itemize}

Fisher information: 
\begin{itemize}
\item Efficient score (or just score): describes how fast the likelihood function changes with the parameter values. Let $f(\theta;X)$ be the dentify function, it is defined as: 
\begin{equation}
V(\theta) = \frac{\partial \log f(\theta;X)}{\partial \theta}	
\end{equation}
$V$ is random variable defined on $X$. However, we cannot use expectation of $V$ as a measure of how flat the likelihood surface is, as the mean of $V$ is 0. The proof follows from (rewriting the likelihood in terms of the PDF): 
\begin{equation}
V(\theta) = \frac{1}{f(X;\theta)} \frac{\partial f(X;\theta)}{\partial \theta}	
\end{equation}
The integral over the RV $X|\theta$: 
\begin{equation}
\E(V|\theta) = \int \frac{1}{f(X;\theta)} \frac{\partial f(X;\theta)}{\partial \theta} f(X;\theta) dX = \int \frac{\partial f(X;\theta)}{\partial \theta} dX = \frac{\partial}{\partial \theta} \int f(X;\theta) dX = 0
\end{equation}
A better measure is to use the expectation of $V^2$ (see below). 

\item \textbf{Remark}: It is important to understand what the property says. When evaluating $\E(V|\theta)$, we are averaging over all $X$, assuming $X$ are generated from the same $\theta$. In other words, suppose $\theta$ is the true value, and we generate $x_i$ (data) from $\theta$, then we compute the score ($E_i$) at $\theta$ from $x_i$. Repeat the experment $n$ times, the average of score $E_i$ would be close to 0. Had we evaluate score at a different value of $\theta$, the expectation is not necessarily 0. 
\begin{itemize}
\item This property says that the expectation of score should be 0. Intuitively, this should be easy to understand, on average, the log-likelihood function is maximized at the true value $\theta$, so its derivative at $\theta$, on average, should be 0. 
\end{itemize}

\item Fisher information: measures how much information, the data points $X$ carries on the unknown parameter $\theta$, averaging over possible $X$. It is defined as the square of $V$, averaging over all possible data: 
\begin{equation}
I(\theta) = E\left[ \left(\frac{\partial \log f(\theta;X)}{\partial \theta}\right)^2 \right]
\end{equation}
Since the mean of $V$ is 0, it is also the variance of the $V$. It can be shown that: 
\begin{equation}
I(\theta) = -E\left( \frac{\partial^2}{\partial^2 \theta} \log f(\theta;X) \right)	
\end{equation}
The proof follows from applying the product rule to the second partial derivative: 
\begin{equation}
\frac{\partial^2}{\partial^2 \theta} \log f(x|\theta) = -\frac{1}{[f(x|\theta)]^2} \left[\frac{\partial}{\partial \theta} f(x|\theta)\right]^2 + \frac{1}{f(x|\theta)} \frac{\partial^2}{\partial^2 \theta} f(x|\theta)
\end{equation}
The expectation of the second term above is 0. 
Therefore, Fisher inforamtion at any $\theta$ may be seen as a measure of the ``curvature'' of the log-likelihood curve, averaging over data points. It measures the intrinsic difficulty of making inference at any $\theta$, thus independen of data ($x$) and independent of the estimator function used to infer $\theta$. 

\item Fisher informtion in the iid. case: if we expand the likelihood function using iid data points, we have: 
\begin{equation}
I_n(\theta) = n I_1(\theta)	
\end{equation}
where $I_n(\theta)$ is Fisher information with $n$ data points, and $I_1(\theta)$ is information at $n = 1$. 

\item Example: a Bernoulli process with $x$ 1's and $n-x$ 0's. Suppose the probability of success per trial is $\theta$, the score is given by: 
\begin{equation}
V = \frac{\partial l(\theta)}{\partial \theta} = \frac{x}{\theta} - \frac{n-x}{1-\theta}
\end{equation}
It is easy to check that $\E(V) = 0$: 
\begin{equation}
\E(V|\theta) = \frac{\E(x)}{\theta} - \frac{\E(n-x)}{1-\theta} = \frac{n \theta}{\theta} - \frac{n (1 - \theta)}{1 - \theta} = 0
\end{equation}
And the variance is given by: 
\begin{equation}
I_n(\theta) = \Var(V|\theta) = \Var\left(\frac{x - n\theta}{\theta(1-\theta)}\right) = \frac{\Var(x)}{\theta^2(1-\theta)^2} = \frac{n}{\theta (1 - \theta)}
\end{equation}
We see that for Bernoulli distribution, it is easier to infer $\theta$ when $\theta$ is close to 0, and more difficult when $\theta$ is close to 1/2 (intuitively, large number of 1's and 0's - large noise). 

\item Fisher information matrix [Wiki]: in the multivariate case:
\begin{equation}
I(\theta)_{ij} = -\E\left[ \frac{\partial^2}{\partial \theta_i \partial \theta_j} \log f(\theta;X) \right]	
\end{equation}

\end{itemize}

Cramer-Rao lower bound: 
\begin{itemize}
\item A single parameter: let $\hat{\theta}$ be an unbiased estimator of $\theta$, and it satisfies some regularity condition, then we have: 
\begin{equation}
\text{Var}(\hat{\theta}) \geq \frac{1}{I(\theta)}
\end{equation}
Since $\hat{\theta}$ is unbiased, its variance is also its MSE. The intuition of this inequality: the LHS is the error of the estimator. It depends on how difficult it is to estimate $\theta$ from data (or how discriminate $\theta$ is), or the information observation carries on $\theta$. 

\item Function of parameter: suppose $T$ is an unbiased estimator of $\tau(\theta)$, we have: 
\begin{equation}
\text{Var}(T) \geq \frac{[\tau'(\theta)]^2}{I(\theta)}	
\end{equation}
The dependence on $\tau'(\theta)$ can be explained by: when it is large, smaller error of $\theta$ (from estimation) means larger error of $\tau(\theta)$. 
\end{itemize}

\item{Common parameter estimation problems}

Comparing the means of two samples: 
\begin{itemize}
\item Problem: given two independent samples of sizes $n_1$ and $n_2$ respectively, where the first sample is from $N(\mu_1, \sigma_1^2)$, and the second from $N(\mu_2, \sigma_2^2)$. We want to estimate the effect size $\Delta = \mu_1 - \mu_2$. 

\item The estimator is given by: 
\begin{equation}
D = \bar{X_1} - \bar{X_2}
\end{equation}
The variance of $D$ is given by: 
\begin{equation}
\text{Var}(D) = \text{Var}({\bar{X_1}})	+ \text{Var}({\bar{X_2}}) = \frac{\sigma_1^2}{n_1} + \frac{\sigma_2^2}{n_2} \approx \frac{S_1^2}{n_1} + \frac{S_2^2}{n_2}
\end{equation}
where $S_1^2$ and $S_2^2$ are the sample variances. The approximation is applied as the true variances of two samples are unknown. 
\end{itemize}

Estimating odds-ratio in a 2 by 2 table: [Odds ratio, Wiki]
\begin{itemize}
\item Problem: given a 2 by 2 table, let $p_{ij}$ be the probability of the $(i,j)$ cell, and $n_{ij}$ be the observed counts of the $(i,j)$ cell. The odds ratio is defined by:
\begin{equation}
\text{OR} = \frac{\left(p_{11} / (p_{11} + p_{10})\right)/\left(p_{10} / (p_{11} + p_{10})\right)}{\left(p_{01} / (p_{01} + p_{00})\right)/\left(p_{00} / (p_{01} + p_{00})\right)} = \frac{p_{11}p_{00}}{p_{10}p_{01}}
\end{equation}
Our problem is to estimate $\text{OR}$ from the observed counts. 

\item The problem of estimating odds: to simplify, we first consider the case where we have a binomial sample, and estimate the odds (the prob. of success over the prob. of failure). Suppose our sample has $x$ successes in $n$ trials, with $X \sim \text{Bin}(n,p)$, the log-odds is thus: $\log \frac{p}{1-p}$. The estimator is simply: 
\begin{equation}
\hat{\text{Odds}} = \log \frac{\hat{p}}{1 - \hat{p}}
\end{equation}
where $\hat{p} = x/n$. The variance of the estimator (MLE) can be calculated using the asymptotic results of MLE. The likelihood function is: $L(p) = p^x (1-p)^{n-x}$, and thus the Fisher information can be computed as the negative second derivative of $\log L(p)$: 
\begin{equation}
I(p) = \frac{x}{p^2} + \frac{n-x}{(1-p)^2}	
\end{equation}
The derivative of the $\log (p/(1-p)) = 1/p + 1/(1-p)$, and plug the asymptotic variance of MLE: 
\begin{equation}
\text{Var}\left( \log \frac{\hat{p}}{1-\hat{p}}\right) = \frac{\left( \frac{1}{p} + \frac{1}{1-p}\right)^2}{\frac{x}{p^2} + \frac{n-x}{(1-p)^2}} \bigg|_{p = \hat{p}}	= \frac{1}{x} + \frac{1}{n-x}
\end{equation}

\item Estimator of log odds-ratio: it can be shown easily that the estimator of log-odds ratio is: 
\begin{equation}
L = \log \text{OR} = \log \frac{n_{11} n_{00}}{n_{10} n_{01}}	
\end{equation}
The variance of $L$ is simply the sum of the variance of log-odds in the two groups (given above): 
\begin{equation}
\text{Var}(L) = \frac{1}{n_{11}} + \frac{1}{n_{10}} + \frac{1}{n_{01}} + \frac{1}{n_{00}} 	
\end{equation}

\end{itemize}

\item{Missing data problem} 

EM algorithm: 
\begin{itemize}
\item Intuition of EM algorithm: suppose we have observed data $x$, parameters $\theta$, and missing data $y$. Our goal is to maximize the likelihood: 
\begin{equation}
l(\theta) = \log P(x|\theta)	= \log \int P(x,y|\theta) dy
\end{equation}
Our intuition is this: if we know $\theta$, then we could estimate the missing data $y$; using the estimated $y$ in the computation of log-likelihood  would lead to a better estimate of $\theta$. However, since the missing $y$ cannot be completely determined, we need to consider the log-likelihood averaging over all possible values of $y$. 

\item EM algorithm: at the E-step, we compute this function: 
\begin{equation}
Q(\theta|\theta^t) = \E_{y|x,\theta^t}\left[\log P(x,y|\theta)\right]	
\end{equation}
assuming that the current estimate is $\theta^t$. At the M-step, we maximize $Q(\theta|\theta^t)$ as a function of $\theta$. For the generalized EM (GEM) algorithm, we only need to find $\theta$ that increases $Q(\theta|\theta^t)$.   

\item Proof of convergence: we show that the EM algorithm always increases the $Q$ function. We note that the log-likelihood of $x$ is related to the complete log-likelihood: 
\begin{equation}
\log P(x|\theta) = \log P(x,y|\theta) - \log P(y|x,\theta)	
\label{eq:ll_missing}
\end{equation}
Since this is true for any value of $y$, we could average over $y|x,\theta^t$: 
\begin{equation}
\log P(x|\theta) = \E_{y|x,\theta^t}\left[\log P(x,y|\theta)\right] - \E_{y|x,\theta^t}\left[\log P(y|x,\theta)\right]	
\end{equation}
The first term is exactly the $Q$ function. We have: 
\begin{equation}
\log P(x|\theta) - \log P(x|\theta^t) = Q(\theta|\theta^t) - 	Q(\theta^t|\theta^t) + \E_{y|x,\theta^t}\left[\frac{\log P(y|x,\theta^t)}{P(y|x,\theta)}\right]	
\end{equation}
The last term is always nonnegative according to the KL divergence. 

\item EM algorithm for independent samples: the $Q$ function can be written as: 
\begin{equation}
Q(\theta|\theta^t) = \sum_i \E_{y_i|x_i,\theta^t}\left[\log P(x_i,y_i|\theta)\right]	
\end{equation}	
\end{itemize}

Extensions of EM: 
\begin{itemize}
\item Missing information principle: according to Equation~\ref{eq:ll_missing}, we take second derivative wrt. $\theta$, and state in terms of Fisher information: 
\begin{equation}
I_o(\hat{\theta}|x) = I_{oc} - I_{om}	
\end{equation}
where $I_o$ is the Fisher information of the observed data: 
\begin{equation}
I_o(\theta|x) = - \frac{\partial^2 \log p(x|\theta)}{\partial \theta^2}	
\end{equation}
and $I_{oc}$ is the Fisher information of the complete data (averaging over $y|x,\theta$): 
\begin{equation}
I_{oc} = \E_{y|x,\theta} [I_o(\theta|x,y)]|_{\theta = \hat{\theta}}	
\end{equation}
and $I_{om}$ is the information from the missing data: 
\begin{equation}
I_{om} = \E_{y|x,\theta} \left[-\frac{\partial^2 \log p(y|x,\theta)}{\partial \theta^2}\right]|_{\theta = \hat{\theta}}	
\end{equation}

\item Finding the covariance matrix of $\theta$ (SEM algorithm): [Using EM to obtain asymptotic variance-covariance matrices: The SEM algorithm] we often need to determine the covariance matrix at $\hat{\theta}$, e.g. for the confidence interval of $\theta$. The covariance matrix at $\hat{\theta}$ is the inverse of the $I_o$ matrix. See the paper for the details of the SEM algorithm. 
\end{itemize}

\end{enumerate}
%%%%%%%%%%%%%%%%%%%%%%%%%%%%%%%%%%%%%%%%%%%%%%%%%%%%%%%%%%%%
\section{Hypothesis Testing} 
	
Neyman-Pearson paradigm: the goal is to choose a decision rule for accepting or rejecting a hypothesis. 
\begin{itemize}
\item Intuition of hypothesis testing: to distinguish two hypothesis $H_0$ and $H_A$, suppose we choose some informative statistic/pattern $T$ to be test statistic if it is expected to be different under $H_0$ or $H_A$. The two distributions $T|H_0$ and $T|H_A$ should be different, and the more different they are, the easier to distinguish the two hypothesis (Figure: hypothesis-testing.gif). 

\item Assessing a decision rule: Suppose the rule takes the form: $H_0$ is rejected iff $T \in C$, where $T$ is the test statistic and $C$ is the critical region. Then the rule can be assessed by the two types of errors: 
\begin{itemize}
	\item Type I error: $\alpha = P(T \in C | H_0)$.  
	\item Type II error: $\beta = P(T \notin C| H_A)$. 
\end{itemize}
Alternatively, a test can be assess by these two measures: 
\begin{itemize}
	\item Significance level: probability of incorrectly rejecting the null hypothesis ($\alpha$). 
	\item Power: probability of correctly rejecting the null hypothesis ($1 - \beta$). 
\end{itemize}

\item Remark: for a given pair of hypothesis, to choose one of them is similar to assign the distribution of which a data point comes from under a mixture model. While this task is generally performed by Bayesian or likelihood ratio test, the hypothesis testing approach is different. 

\item Procedure: suppose we choose $T$ as the test statistic, and we need to determine the critical region $C$. Typically, we choose $C$ s.t. Type I error is bounded by a pre-specified level $\alpha$: 
\begin{equation}
P(T \in C|H_0) \leq \alpha	
\end{equation}
To design a test: choose among all tests that meet certain significance level (e.g. $\alpha < 0.05$), the one that maximizes the power. 

\item Power of a test: generally depends on the significance level, the sample size, and the alternative hypothesis. When the parameter of the alternative hypothesis is not specified ($\theta$), then we define power function as $\text{Pw}(\theta) = 1 - \beta(\theta)$.
\begin{itemize}
	\item Most powerful test: of a specified $H_A$, if no other test of the same sample size has greater power. 
	\item Uniformly most powerful test: of a class of alternative hypothesis, if it is the most powerful test for each specified alternative in this class. 
\end{itemize}

\item Comparison of tests: generally compare power at a specified significance level. Note that the relative power may depend on alternative hypothesis. Also, robustness to violations of parametric assumptions is another important consideration in practice (see Nonparameteric tests). 

\item Relation to classification: the metrics of classification performance measure:  
\begin{itemize}
	\item Type I error or false positive rate (FPR): specificity or true negative rate ($1 - \alpha$). 
	\item Type II error or false negative rate (FNR):  sensitivity, recall or true positive rate, TPR ($1 - \beta$). 
	\item The fraction of correct predictions among all that are predicted in the positive class: precision or false dicovery rate: $= P(H_A | H_A\text{ is accepted})$
\end{itemize}

\item Trade-off between specificity-sensitivity or type I/II errors: a more accurate method (high specificity) may not generalize well (low sensitivity). A common situation is the preference of simpler models, which tend to be less specific. 

\end{itemize}

Limitations of Neyman-Pearson paradigm: 
\begin{itemize}
\item Limitations of the number of hypothesis: choose among a large set of models, e.g. structural learning in Bayesian networks. 

\item Decision theory: a more general framework for hypothesis testing and model selection is probably decision theory. Ex. for choosing among multiple models, we could define the cost function that may take into account the similarity of different models (thus even if we choose a wrong model, but if it is close to the true model, our penalty would be smaller). 
\end{itemize}

$p$ value: 
\begin{itemize}
\item Motivation: it is often not sufficient to just have a decision rule. One may need to quantify the confidence of a decision, e.g. if $T$ is far from the critical region, then one may know that it is safe to accept $H_0$ (as opposed to the case where $T$ lies in the boundary). 

\item Idea: since in general, the value of $T$ is not comparable across different tests, we want to map $T$ to some function $\phi(T)$ s.t. $\phi(T)$ is comparable, and then effectively we could have a simple decision rule for all tests, e.g. $\phi(T) < 0.05$. 

\item Percentile transformation: this is similar to the situation where we want to say whether a number randomly sampled from a distribution is large or not. Clearly, it has to be normalized against the underlying distribution. This can be achieved by transform any value $x$ to its percentile or CDF $CDF(x)$. It is easy to show that the new R.V., $CDF(X)$ follows uniform distribution (if not, some ranges must have more probability mass than expected, violating our definition of percentile at the first place).

\item $p$ value: we apply the percentile transformation to test statistic $T$, and this gives the $p$ value of an observed statistic (thus $p$ value can be simply understood as the normalized test statistic). Assume we reject $H_0$ when $T$ is large, then $p$ value of the observed statstic $t$is given by: 
\begin{equation}
p = P(T > t|H_0)	
\end{equation}
Alternatively, we can understand $p$ value as: if we choose the observed statistic as the decision boundary, what is the significance level. 

\item Uniform distribution of $P$-value: the intuition is that percentile score should be unbiased, e.g. we would only see 1\% of samples with $p$ value 0.01. Proof: we want to show that $P(p<u) = u$. Suppose $x_u$ is the value of $X$ s.t. $P(X > x_u) = u$, then whenever $x > x_u$, we have $p < u$ and vice versa, thus $P(p < u) = P(x > x_u) = u$. 

\item Remark: $p$-value is simpler than the Neymen-Pearson framework of testing, because it does not require one to specify the alternative distribution, and it has the advantage providing confidence of the results, instead of a yes/no answer. Neyman-Pearson however has the advantage of computing the power of the test. 
\end{itemize}

Power analysis:  
\begin{itemize}
\item Assessing a test: suppose we have test statistic $T$ for a hypothesis, to evaluate its performance, we compute the power of the test at a certain significance level $\alpha$. The analysis consists of two steps: 
\begin{itemize}
	\item Determine the threshold of $T$ according to $\alpha$: from the distribution of $T$ under $H_0$, choose the threshold $t$ (suppose $T < t$ will reject $H_0$) s.t. $P(T < t | H_0) \leq \alpha$. 
	\item Power calculation at the threshold: suppose $t$ is the chosen threshold, the power of the test is: $P(T < t | H_A)$. The crucial step is thus to calculate the distribution of $T$ under $H_A$. 
\end{itemize}

\item Example: binomial distribution. Suppose we want to test the parameter $p$ of a distribution, $\text{Bernoulli}(p)$: $H_0: p = p_0$ and $H_A: p = p_1$ (suppose $p_1 < p_0$). The goal here is here to choose sample size $n$ s.t. the test reaches power $1 - \beta$, at significant level $\alpha$. We choose the test statistic $\hat{p}$ as the fraction of successes in the sample. 
\begin{itemize}
\item Threshold determination: the expectation and variance of the statistic under $H_0$: 
\begin{equation}
E(\hat{p}) = p_0 \qquad \text{Var}(\hat{p}) = \frac{p_0(1-p_0)}{n}
\end{equation}
Thus the distribution of $\hat{p}$ can be approximated by a normal distribution $N(p_0, p_0 (1-p_0)/n)$, and at the significance level $\alpha$, we have the test: 
\begin{equation}
\hat{p} < p_0 - z_{\alpha} \sqrt{\frac{p_0 (1 - p_0)}{n}}	
\end{equation}
 	
\item Power calculation: under $H_A$, the distribution of $\hat{p}$ can be similarly derived: $\hat{p} \sim N(p_1, p_1 ( 1 - p_1) /n)$. To have the power above a threshold $ 1- \beta$ (or type II error below $\beta$), we should have: 
\begin{equation}
P( \hat{p} > p_0 - z_{\alpha} \sqrt{\frac{p_0 (1 - p_0)}{n}} | H_A) \leq \beta
\end{equation}
This is equivalent to: 
\begin{equation}
p_0 - z_{\alpha} \sqrt{\frac{p_0 (1 - p_0)}{n}}	- p_1 \geq z_{\beta} \sqrt{\frac{p_1 (1 - p_1)}{n}}
\end{equation}
And the sample size should satisfy: 
\begin{equation}
n \geq \left[ \frac{z_{\alpha} \sqrt{p_0(1-p_0)} + z_{\beta} \sqrt{p_1(1-p_1)}}{p_0 - p_1} \right]^2	
\end{equation}

\end{itemize}

\end{itemize}

Why $p$ value is not a good measure? 
\begin{itemize}
\item Multiple testing issue: under this situation, $p$ value does not tell the false positive rate. Thus need correction or Bayesian methods. 
	
\item Power is not considered: suppose we are doing hypothesis testing for two tasks, if the power of two tests are different, then the $p$ values under two tests are not really comparable (thus fails to reflect our confidence of accepting the alternative hypothesis). Ex. testing mean of normal distribution: $H_0: \mu = 0$, vs two alternative hypothesis, A) $H_1: \mu = 0.1$; B) $H_1: \mu = 3.0$. Clearly, the test of A) is much harder than that of B). Both A) and B) would use the same Z-test, thus for the same $\bar{\mu} = 3$, even though the $p$ value is the same under A) and B), our confidenence of accepting A) is much smaller than accepting B). 

\item Why Bayesian approach is a promising solution: 
\begin{itemize}
	\item First, the multiple testing problem is addressed by introducing prior, the confidence of an hypothesis is evalauted by posterior ratio, which is a product of prior ratio and Bayes factor. The prior ratio could encode the information: the fraction of true signals. 
	\item Second, the power problem is addressed by Bayes factor: $\text{BF} = \frac{P(D|H_1)}{P(D|H_0)}$. In the above example: A) $P(D|H_1)$ is also small, thus BF is small; B) $P(D|H_1)$ is large, thus BF is large. Therefore, for the same $p$ value, we have different BF, reflecting the difference confidence resulting from different plausibility under alternative hypothesis. 
\end{itemize}

\end{itemize}

Confidence interval [Rice, Mathematical Statistics and Data Analysis, Section 8.5.3]
\begin{itemize}
	\item Definition: a confidence interval for a population parameter $\theta$ is a random interval, calculated from the sample, that contains $\theta$ with some specified probability. 
	
	\item Example: normal distribution $N(\mu,\sigma^2)$, the sample mean and sample variance are: 
	\begin{equation}
	\bar{X} = \frac{1}{n} \sum_{i=1}^n X_i
	\end{equation}
	\begin{equation}
	S^2 = \frac{1}{n-1} \sum_{i=1}^n (X_i - \bar{X})^2
	\end{equation}
	The MLE of $\mu$ is $\bar{X}$. The confidence interval of $\mu$ is based on the fact that $\frac{\sqrt{n} (\bar{X} - \mu)}{S} \sim t_{n-1}$. Let $t_{n-1}(\alpha/2)$ denote the point beyond which the $t$ distribution with $n-1$ degrees of freedom has probability $\alpha/2$ ($t$ distribution is symmetric). Then the confidence interval is given by $\bar{X} \pm S t_{n-1}(\alpha/2) / \sqrt{n}$.
	
	\item Duality of confidence intervals and hypothesis tests [Rice, 9.4]: A $100 (1-\alpha)\%$ confidence interval for $\theta$ consists of all those values of $\theta_0$ for which the hypothesis $\theta$ equals $\theta_0$ will not be rejected at level $\alpha$. 
	\item The hypothesis $\theta$ equals $\theta_0$ is accepted if $\theta_0$ lies in the confidence interval.  
\end{itemize}

Techniques for finding confidence interval: 
\begin{itemize}
	\item Exact method: if the distribution of the estimator (or related functions) can be found, then it can used for finding the confidence inteval. The idea is to transform the distribution (which usually contains unknown parameters) s.t. the random interval no longer contains unknown parmaters. Ex. confidence interval for $\mu$ of the normal distribution (see above). 
	\item Asympotic method for confidence interval of MLE: normal distribution.  
	\item Bootstrapping method: sample from the estimated values of the population parameter and obtain the distribution of the estimator. 
\end{itemize}

\subsection{Strategies of Developing Tests}

How to construct a test?
\begin{itemize}
\item General idea: look for patterns in the data that would be different under $H_A$ and $H_0$. This is expressed as a statistic $T$, that is informative of the relevant hypothesis. Technically, it would have different distributions under $H_A$ and $H_0$. Typically, one may have: $\E(T|H_0) = 0$ and $\E(T|H_A) > 0$ (one could use expectation as a surrogate of whether the distributions under two hypothesis are equal). See below for some common ideas. 
\begin{itemize}
	\item Ex. for linear model, when $\beta = 0$, $X$ and $Y$ are independent - so we test this independence. 
\end{itemize} 

\item Estimator-based test: to test hypothesis about a parameter $\theta$, suppose $W_n$ is an estimator of $\theta$, then we obtain the sampling distribution of $W_n$ (dependent on the parameter $\theta$). In general, we could then construct a test statistic using this distribution through CLT, Wald test, etc. 
\begin{itemize}
\item Often, we will need to determine the variance or standard deviation of the estimator, called standard error. The distinction is that the standard deviation often depends on some unknown parameter, while the standard error needs to be estimated from the data (so we replace the unknown parameter with its estimate). 

\item The idea can be extended to functions of parameters: if we have a statitsic that is an estimator of a function of parameter (e.g. $E(T_n)$ is a function $\tau(\theta)$), then $T_n$ has information of $\theta$. Suppose we can obtain the sampling distribution of $T_n|H_0$, then we could form a test of $\theta$. Since the form of function $\tau(\cdot)$ is very general, this is a very broad strategy to test hypothesis.
\end{itemize}

\item Likelihood-basesd test: LRT or score test, this allows one to construct a simpler test statistic, or one could use asymptotic $\chi^2$ distribution. Under some conditions (e.g. testing simple hypothesis), Neyman-Pearson Lemma guaranttes that LRT is optimal. 

\end{itemize}

Patterns in the data that carry information of the parameters being tested: 
\begin{itemize}
\item Variance and variance partitioning: the hypothesis regarding parameters may be manifested as how big the variance is, and how variance is partitioned. The best example is ANOVA: if the treatment has no effect, then the variance between the group (treatment and no-treatment) would be the same as the variance within the group (up to some constant). 

\item Covariance based tests: covariances between variables are important patterns encoded in the data. Example: $Y = \beta X$, the larger $\beta$, the bigger the covariance $\Cov(X,Y)$. 

\item Errors or model fit: for prediction problems, the parameter values are related to the predictor error or other measures of model fitting (e.g. log-likelihood). Ex. $Y = \beta X$ with error term $\epsilon \sim N(0,\sigma^2)$, to test $\sigma^2$, the smaller $\sigma^2$, the smaller the predictor error (i.e. most of $Y$ can be explained by $X$). 

\item Matching histogram: to test if $\theta$ is a certain value, we could compare the distribution $f_{\theta}$ and the empirical distribution $\hat{f}$, if the two are very close, then we know that $\theta$ is a good value. Example, $\chi^2$ test or goodness-of-fit test. 

\item Combining multiple statistics: a test statistic may be constructed from multiple components, e.g. $\chi^2$ test statistic from the sum of variables for all cells. The random variables to be combined (added, subtracted, etc.) need to be comparable.

\item Nonparametric test in terms of ranks, e.g. Mann-Whitney test. 

\item Remark: the ideas of parameter estimation: MOM, MLE, LS, etc. can all be applied to develop hypothesis testing strategies. 
\end{itemize}

Lessons for constructing tests: 
\begin{itemize}
 	\item Consideration of sampling variance: if we use $T$ as the test statistics, in general, lower variance of $T$ (under a hypothesis) would be preferred. Intuitively, if $T$ does not vary much under $H_0$, then it is easy to tell if any departure of $T$ (from expectation) is significant or not. 
	\begin{itemize}
		\item Example: suppose we have ChIP-seq type of experiment, and we want to test if a peak is statistically significant. If the count (or normalized counts, or some statistic) does not vary much across the genome, then it's easier to determine the significant peaks. 
		\item Example: linear model, where $y$ is response variable and $x$ the independent variable of interest. Suppoze we have another variable $z$, even if we are not interested in $z$ per se, we should include $z$ if it correlates with $y$. The intuition is that conditioned on $z$, we can remove some of the variance of $y$, thus our test statistic of $x$ will have lower variance (which depends on the variance of $y$). 
	\end{itemize}
	The general idea is that if we can explain away some variance/noise in the data, then we should always do that. 
	
	\item Borrowing information if possible: the idea is that if we additional information of the same parameter of interest, we could borrow information. This is the basic idea of hierarchical Bayes. 
	\begin{itemize}
		\item Example: we are interested if a SNP is associated with a trait $y_1$. Now it is reasonable to speculate that if it is associated, then it is likely associated with another trait $y_2$ as well. So we could test the association simultaneously. 
	\end{itemize}
\end{itemize}

Preference of simple models:  
\begin{itemize}
\item Occam's Razor: It is harder to accept complex hypothesis: under hypothesis testing, if $H_A$ is complex, then a higher LRT statistic would be needed to reject $H_0$ (it would be easier to get a good LRT by chance, thus LRT should be discounted). This is reflected by higher d.f. of the test under complex $H_A$. When the actual $H_A$ is simpler than the specified $H_A$: the model fitting statistics would be hard to reach the LRT threshold (because the actual model generating data is not that complex), resulting a lower power. 

\item Example: Fisher's method vs. combined likelihood ratio test. Suppose we have two independent datasets $D_1$ and $D_2$ testing the same null hypothesis. The true model of $D_1$ and $D_2$ are related by a single parameter $\theta$. Suppose with single dataset, we use $\chi^2$ test and get $p$-value $p_1 = 0.01$ and $p_2 = 0.01$. We consider two tests: 
\begin{itemize}
\item Fisher's method: combining $p_1$ and $p_2$, and use the $\chi^2$ distribution with dof. 4, we have $p = 10^{-3}$. 
\item Combined LRT: the $\chi^2$ statistic for each dataset is 6.6. Since the same $\theta$ should maximize (approximately) the likelihood in both datasets, the combined LRT is $6.6 + 6.6 = 13.2$, giving $p$-value $2 \cdot 10^{-4}$ under $\chi^2$ with d.o.f. 1.  
\end{itemize}
In this example, Fisher's method ignores the common $\theta$ in both datasets; effectively, the alternative model under Fisher's method can fit $\theta$ independently in $D_1$ and $D_2$, increasing the model complexity. 

\end{itemize}

Dealing with nuisance parameters:
\begin{itemize} 
\item Standarization and pivotal quantity: it is  desirable to have $T$ whose distribution does not depend on nuisance paramaters, e.g. $T|H_0$ follows some standard distribution, such as $N(0,1)$ or $\chi^2$ with certain dof. Intuitively, we want $T$ whose distribution (our decision rule) is not influenced by things other than what we are testing. 
\begin{itemize}
	\item Identifying factors that may affect the proposed statistic ($T$): if there are additional factors (beyond the one being tested) that may influence the distribution of $T$, then it is important to control for them. These factors may be: the degree of freedom, sample size. 
	\item Ex. for univariate linear regression, the parameter $\beta_1$ is related to the fraction of total variance explained by the regression, i.e. $R^2 = SSR / SST$. However, $R^2$ does not control for sample size, so cannot be directly used for testing $\beta_1$. 
	\item Ex. for ANOVA, need to standardize the bewteen-group variation using within-group variation. 
\end{itemize}

\item Replacing nuisance parameters with their estimators: develop the test statistic $T$ that is independent of the nusiance parameter(s) $\tau$. E.g. if $T$ is a test statistic of $\theta$, whose distribution involves $\tau$, we could develop a new test statistic $T'$ by replacing $\tau$ with its estimator $\hat{\tau}$ s.t. the distribution of $T'$ is no longer dependent on $\tau$. 

\item Using partial likelihood or derived statistic from data: suppose the likeihood is $f(D|\theta, \eta)$ where $\eta$ is the nuisance parameter. Our idea is to derive some new RV $T$ from data, that does not depend on $\eta$; in other words, the likelihood in terms of $T$, $g(T|\theta)$ depends only on $\theta$. 
\begin{itemize}
\item Two-sample Poisson test: event 1 follows Poisson distribution of rate $\lambda$ and event 2 $\lambda R$, and we are interested in if $R > 1$. Our statistic is: given that event 1 or 2 occurs, how often event 2 occurs, or 
	\begin{equation}
	P(x_2 | x_1 + x_2 = n) = \text{Binom}\left(n, \frac{R}{R+1}\right)
	\end{equation}
	
\item Example: Parent Assymetry Test (PAT) in testing imprinting and matenral effect, the event $M > P$ (mother has more minor alleles than father) conditioned on the mating type follows binomial distribution whose parameter does not depend on the genotype frequency of parents. 

\item This represents a genearl case where the nuisance parameter is a scale parameter (the base time of Poisson distribution). In this case, one can use conditional distribution of some test statistic - this means one consider fractions, effectively canceling out the scale parameter (similar to $Z$-score, which normalize a test statistic, e.g. difference of mean).

\item More generally, when some test statistic $Z$ is a sufficient statistic of the nuisance parameter $\eta$, then the conditional distribution of data given $Z$ is independent of $\eta$. 
\end{itemize} 

\end{itemize}

Composite hypothesis and nuisance parameters: 
\begin{itemize}
\item Problem: the key step of constructing a test is the distribution of $T$ under $H_0$. However, $H_0$ may not completely identify a parameter (for composite hypothesis), or $T$ distribution depends on additional (unknown) parameters. 

\item Obtaining $T$ distribution under given parameters: under Neyman-Pearson paradigm, we need to control type I error, i.e. find $t$ s.t. (assuming $T \geq t$ rejects $H_0$):
\begin{equation}
\sup_{\theta \in \Theta_0} P(T \geq t|\theta) \leq \alpha	
\label{eq:null_distr}
\end{equation}
Thus as long as we know $T|\theta$ distribution, we could construct a test. 

\item Problems with maximization: the technical problem with a test that depends on nuisance parameters is that the maximization in Equation~\ref{eq:null_distr} may not well-behave. Ex. to test mean in the distribution $N(\mu,\sigma^2)$ where $\sigma^2$ is unknown, if our test statistic is simply $T = \bar{X}$, without normalization, maximization in Equation~\ref{eq:null_distr} is unbounded or equal to 1. 

\end{itemize}

Obtaining null distribution by sampling: for complex test statistic, its null distribution may be obtained by simulation. We sample data under the null hypothesis $T|H_0$. 
\begin{itemize}
\item Parametric bootstrap: normally, the model has other parameters than the one(s) specified by $H_0$, so we need to estimate those parameters (e.g. MLE), and then sample data from $H_0$ and the fitted nuisance parameters. 
	
\item Permutation test: for hypothesis involving relationship between variables, permutating the data s.t. the sample can be considered to be generated from $H_0$ (no relation).  
\end{itemize}

Design considerations: to avoid the cases where departure from $H_0$ is caused by some reasons other than $H_A$. 
\begin{itemize}
	\item Comparablility: if some variables used in $T$ are not comparable, the test may be biased (size bias is one common problem). Ex. find pathways that are differentially expressed in samples by the number of DE genes: the size of pathways are different. In statistical terms, it means the relevant quantities should have the same distributions. 
	\item Avoid information loss: this often occurs, for instance, when some form of cut-off is considered. Ex. test enrichment of gene groups: if only the most significant genes are tested, the information in the marginally signicant genes is lost. 
	\item Correlation/dependency in the data: this may make tests invalid. Ex. to find genes DE across many samples: if samples are correlated (e.g. some samples are from the same patients), then the test statistic may be inflated. 
	\item Other considerations: outliers, heterogeneity, etc. Any other implicit assumptions that may be violated. 
\end{itemize}

Strategies of dealing with possible biases: 
\begin{itemize}
	\item Modifying test statistic: e.g. to make some variables comparable, use $P$ value, or normalization/calibration (the general idea is to use relative values as test statistics); to maximize use of information, use some weighting scheme s.t. weak evidence can still be utilized. 
	\item Probabilistic modeling: a full model with ML or Bayesian methods can avoid many possible biases. 
	\item Null distribution: the null distribution should take the possible biases/dependence/etc. into account. 
	\item Examples: (1) gene set enrichment analysis, pathway association with diseases - the pathway sizes are different [Tian \& Park, PNAS, 2005]. (2) Gene association with diseases in GWAS studies: gene size, the SNP density and LD pattern in the gene region, etc. are different, thus need normalization. 
\end{itemize}

A general form of hypothesis testing is comparison: $H_0$: foreground (FG) = background (BG). Important considerations: 
\begin{itemize}
	\item The positive and negative (or FG and BG) sets should be identical in all aspects except the one that is being investigated. The test statistic should reflect the difference of FG and BG and the null distribution in general should be sampled from the BG distribution
	\item Dealing with confounding variables: the general strategy is stratification of the confounding variables, which could be implemented in a regression framework. Or seeking better controls that match the test objects in the confounding variables (which can be used to obtain the null distribution of test statistic).  
\end{itemize}
Some exampels: 
\begin{itemize}
	\item Comparison of sequence groups: the two groups may have some systematic difference, e.g. GC content, or level of conservation. For instance, to find motifs enriched in the positive group, if the positive group is AT rich, then any AT-rich motif may be found to be enriched (not specific).
	\item Finding differentially expressed genes in two conditions: the expression profiles of the two conditions must be comparable, otherwise, many genes will be differentially expressed. Solution: use a reference set for each condition. 
	\item Association mapping: if the case and control groups have different population structure, then may give false SNPs. 
\end{itemize}

\subsection{Common Statistical Tests}

Bernoulli distribution:
\begin{itemize}
\item Suppose $X_i, 1 \leq i \leq n$ iid from $\text{Bernoulli}(p)$, we want to test the hypothesis: $H_0: p = p_0$, vs. some alternative, $H_1: p \neq p_0$, or $H_1: p = p_1$. 

\item Estimator-derived test: we have the estimator of $p$ as $\hat{p} = X/n$. The sampling distribution of $\hat{p}$ is given by CLT, as $n \to \infty$:  
\begin{equation}
\sqrt{n} (\hat{p} - p) \to N(0, p(1-p))	
\end{equation}
Under $H_0: p = p_0$, thus we have the test statistic: 
\begin{equation}
Z = \frac{X - np_0}{\sqrt{np_0(1-p_0)}}	
\label{eq:binomial_test}
\end{equation}
which follows a standard normal distribution. 

\item LRT: suppose we are testing $p_0$ vs. $p_1$. The LRT statistic would have the form: 
\begin{equation}
-2 (\log L(p_0|X) - \log L(p_1|X)) = -2 X \log \frac{p_0}{p_1} -2 (n-X) \log \frac{1-p_0}{1-p_1}	
\end{equation}
where $X$ follows $\text{Bin}(n,p)$ distribution. The distribution of the LRT statistic is easily derived as a linear function of binomial random variable. 

\item Pattern-based test: let $X = \sum_i X_i$, the number of 1's in the data, then the extreme value of $X$ would reject $H_0$. We have the sampling distribution $X|H_0 \sim \text{Binom}(n,p_0)$, and this allows one to compute $p$-value for any observed $X$. 

\end{itemize}

Poisson distribution: 
\begin{itemize}
\item Two sample test: comparing the rates of two samples [Krishnamoorthy \& Thomson, A more powerful test for comparing two Poisson means]. Suppose we have $x_1$ events in interval $t_1$, and $x_2$ events in $t_2$, we are testing if the two rates are equal: $\lambda_1 = \lambda_2$. The idea is that given a total of $x_1+x_2$ events, the expected number of events in $t_1$ follows Binomial distribution: 
\begin{equation}
X_1 \sim \text{Binom}(x_1 + x_2, p)
\end{equation}
where $p = (\lambda_1 t_1) / (\lambda_1 t_1 + \lambda_2 t_2)$. Under $H_0$, we have $p = t_1 / (t_1 + t_2)$. This test can be implemented in R using \texttt{binom.test()} or \texttt{poisson.test()} (two sample version) and the results are equivalent. 
 
\end{itemize}

Normal distribution: 
\begin{itemize}
\item Test $\mu$ with known $\sigma^2$: 
\begin{itemize}
\item Estimator-based test: the estimator of $\mu$ is $\bar{X}$, and its distribution:
\begin{equation}
\frac{\sqrt{n} (\bar{X} - \mu)}{\sigma} \sim N(0,1)	
\end{equation}
When $\sigma$ is known, we could have test statistic $T = \sqrt{n} (\bar{X} - \mu_0) / \sigma$ for $H_0: \mu = \mu_0$. 

\item LRT: suppose we are testing $H_0: \mu = \mu_0$ vs. $H_1: \mu \neq \mu_0$. The LRT statistic is reduced to: 
\begin{equation}
\sum_i \left[ (x_i - \mu)^2 - (x_i - \bar{x})^2\right]	= n (\bar{x} - \mu)^2
\end{equation}
The test statistic is thus similar to the estimator based test. 
\end{itemize}

\item Test $\sigma^2$ with known $\mu$: the estimator of $\sigma^2$ is the sample variance, $S_n^2$. We know that the sampling distribution of $S_n^2/\sigma^2$ follows $\chi^2$ distribution of dof. $n$. 

\item Test $\mu$ with unknown $\sigma^2$: 
\begin{itemize}
\item Estimator-based test: similar to the case above where $\sigma^2$ is known, the difference being that we replace $\sigma$ with its estimator $S_n$. 

\item LRT: see [Casella, Example 8.2.2]. 
\end{itemize}
\end{itemize}

Simple linear regression: 
\begin{itemize}
\item Problem: a simple univariate regression $Y = \beta_1 X + \beta_0 + \epsilon$, and we want to test $H_0: \beta_1 = 0$. 

\item Estimator-based test: the estimator of $\beta_1$ is $b_1 = \hat{\Cov}(X,Y) / \hat{\Var}(X)$. The sampling distribution of $b_1$ is normal, with mean $\beta_1$ and variance $\sigma^2/\sum_i (x_i - \bar{x})^2$. We can thus form a test of $\beta_1$ by $b_1$ divided by its standard deviation ($t$-test, replacing $\sigma^2$ by its estimator, MSE). 

\item LRT: when $\beta_1 = 0$, we fit the data of $Y$ using the mean of $Y$; when $\beta_1$ is free, we fit the data of $Y$ using $X$ and thus have larger likelihood or smaller squared error. We could form the LRT (equivalent to $F$-test using RSS). 

\item Pattern-based test/variance partitioning: the variance of $Y$ is partitioned by: those explained by $X$, and those not (within group variance). If $b_1 = 0$, the variance explained by $X$ should be 0. We thus look at $MSR$ of the data:  
\begin{equation}
MSR = \hat{\beta}_1^2 \sum_i (x_i - \bar{x})^2	
\end{equation}
$MSR$ is the estimator of the function of parameter:
\begin{equation}
\E(MSR) = \sigma^2 + \beta_1^2 \sum_i (x_i - \bar{x})^2	
\end{equation}
thus clearly carrying information of $\beta_1$. We already know $\hat{\beta_1}$ follows normal distribution, thus it can be shown that: 
\begin{equation}
\frac{MSR}{\sigma^2} = \frac{\sum_i (x_i - \bar{x})^2 \hat{\beta_1}^2}{\sigma^2} \sim \chi^2_1
\end{equation}
With $\sigma^2$ unknown, we replace it by its estimator $MSE$, thus we have the $F$-test $MSR/MSE$. 

\end{itemize}

Two sample $t$-test:
\begin{itemize}
\item Suppose we have two samples: $X_i, 1 \leq i m$, and $Y_j, 1 \leq j \leq n$, $X_i \sim N(\mu_1, \sigma_1^2)$, $Y_j \sim N(\mu_2, \sigma_2^2)$. We want to test the hypothesis: $H_0: \mu_1 = \mu_2$ vs. $H_A: \mu_1 \neq \mu_2$. 
\item Test: the idea is the difference of the means is represented by $\hat{\delta} = \bar{X}_m - \bar{Y}_n$. This needs to be normalized by the standard error of $\hat{\delta}$. If $\sigma_1$ and $\sigma_2$ are known, we have: 
\begin{equation}
\text{se}(\hat{\delta}) = \sqrt{\frac{\sigma_1^2}{m} + \frac{\sigma_2^2}{n}}	
\end{equation}
Since they are not known, we use the sample variance $s_1^2$ and $s_2^2$ instead. Thus we have the test statistic: 
\begin{equation}
T = \frac{\bar{X}_m - \bar{Y}_n}{\sqrt{\frac{s_1^2}{m} + \frac{s_2^2}{n}}}	
\end{equation}
 
\end{itemize}

Pearson's $\chi^2$ test: 
\begin{itemize}
\item The general idea is to compare an observed distribution/counts with the expected distribution counts. The strategy is to discretize, obtain counts in each category, then compare the expected vs. observed counts. 
	
\item Given data from a multinomial distribution $p_1, \cdots, p_K$, we want to test $H_0: p_i = p_{i0}, 1 \leq i \leq K$ vs. $H_A: \exists i, p_i \neq p_{i0}$.
\item The idea is: for each cell $i$, the difference between observed and expected, $O_i - E_i$, is a indicator of the departure of $H_0$. Since different cells have different $E_i$'s, this statistic needs to be normalized s.t. they are comparable across cells. To estimate the variance, we assume that the cell count follows a binomial distribution, $\text{Bin}(N, p_i)$, and the variance is: $N p_i (1 - p_i) = E_i (1 - p_i)$. But since the cell counts are not independent, we obtain the variance as $E_i$ (to be proved). The test statistic is: 
\begin{equation}
X^2 = \sum_{i=1}^K \frac{(O_i - E_i)^2}{E_i}	
\end{equation}
\end{itemize}

Mann-Whitney test: 
\begin{itemize}
\item Problem: compare the mean of two groups, where no parameteric distribution can be assumed. 
\item The idea is if the means are equal, then if we rank all numbers from two groups, the ranks should appear random. Specifically, consider the total rank of one group, and the total rank of the other, and the difference between the two total ranks is a good indicator of how different the means are. 
\end{itemize}

Fisher's method of combining tests: 
\begin{itemize}
\item Method: suppose we are testing the same hypothesis using different independent datasets. Let $p_i$ be the $p$-value of the test using the $i$-th dataset, then we have: 
\begin{equation}
T = -2 \sum_{i=1}^k \log p_i	
\end{equation}
It's easy to show that $T$ follows the exact $\chi^2$ distribution with dof. $2k$: negative log of uniform distribution is exponential, and its sum $\chi^2$. 

\item Remark: limitations
\begin{itemize}
\item Fisher's method does not assume any relationship between the tests, if actually the datasets share parameters, Fisher's method loses power. 
\item The sample sizes of the multiple datasets are not taken into account. If the sample sizes are very unbalanced, the $p$-value from low-powered tests may actually hurt the performance. 	
\end{itemize}
\end{itemize}

Testing non-nested models by J-test [Google search: ``A Specification Test for Non-Nested Regression Models'']: 
\begin{itemize}
\item Problem: suppose we want to test two models: 
\begin{equation}
y = f(\beta_0; X) + u_0	
\end{equation}
\begin{equation}
y = g(\beta_1; Z) + u_1	
\end{equation}

\item J-test: the idea is to estimate the comprehensive model: 
\begin{equation}
y = (1-\lambda) f(\beta_0; X) + \lambda g(\beta_1; Z) + u
\end{equation}
When no a priori information is available, the mixing parameter is not identifiable in the comprehensive model. The J-test works around this by replacing $g(\cdot)$ with the fitted values from a regression of $y$ on $Z$ and testing the mixing parameter, $\lambda$ for statistical significance. 

\item Remark: testing non-nested models is an active area of research, especially in econometrics. Probably no established, widely-used method available. 
\end{itemize}

%%%%%%%%%%%%%%%%%%%%%%%%%%%%%%%%%%%%%%%%%%%%%%%%%%%%%%%%%%%%
\section{Large Sample Theory}

Reference: [Casella, Statistical Inference, Chapter 10]

Convergence in distribution of combination of random variables: 
\begin{itemize}
\item Motivation: suppose we have multiple random variables with known convergence properties (in probability or in distribution), how do we know the sum/product/etc. of these random variables? 

\item Slutsky's Theorem: if $X_n \to X$ in distribution and $Y_n \to a$ in probability, then $Y_n X_n \to aX$ in distribution; and $X_n + Y_n \to X + a$ in distribution. 

\item Example: suppose 
\begin{equation}
\frac{\sqrt{n} (\bar{X}_n - \mu)}{\sigma} \to N(0,1)	
\end{equation}
but $\sigma$ is unknown. Suppose we have $S_n^2 \to \sigma^2$ in probability, then $\sigma/S_n \to 1$ in probability, we have: 
\begin{equation}
\frac{\sqrt{n} (\bar{X}_n - \mu)}{S_n} = \frac{\sigma}{S_n} \frac{\sqrt{n} (\bar{X}_n - \mu)}{\sigma} \to N(0,1)
\end{equation}
This result is very useful for normal approximation when the variance is unknown, but an estimator of variance is available. 
\end{itemize}

Delta Method: convergence in distribution of functions of random variables: 
\begin{itemize}
\item Motivation: similar to before, suppose we know the convergence properties of some random variable (e.g. sample mean from CLT), how do we know the functions of these RVs? 

\item Theorem: suppose $Y_n$ satisfies $\sqrt{n} (Y_n - \theta) \to N(0,\sigma^2)$ in distribution, for a given function $g$ and a specific value of $\theta$, suppose $g'(\theta)$ exists and is not equal to 0. Then
\begin{equation}
\sqrt{n} [g(Y_n) - g(\theta)] \to N(0,\sigma^2 [g'(\theta)]^2) \text{ in distribution}
\end{equation}
Proof: Taylor expansion of $g(Y_n)$ around $Y_n = \theta$: 
\begin{equation}
g(Y_n) = g(\theta) + g'(\theta) (Y_n - \theta) + \text{Remainder}
\end{equation}
The remainder term $\to 0$ in probabiliyty. Applying Slutsky's Theorem to 
\begin{equation}
\sqrt{n} [g(Y_n) - g(\theta)] = g'(\theta) \sqrt{n} (Y_n - \theta)	
\end{equation}
the result follows. 
\end{itemize}

\subsection{Asymptotic theory of point estimation} 

Reference: [Casella, Statistical Inference, Chapter 10]

Tools for studying asymptotic behavior of estimators or statistics: 
\begin{itemize}
\item Sample moments: the WLLN and CLT gives the asymptotic behavior of sample mean, and similar results exist for other sample moments. If an estimator or statistic (including likelihood function) can be expressed as a function of sample moments, then its behavior may be obtained. 

\item Convergence of empirical distribution: at large sample size, the histogram of data points converges to the true distribution. Consider the case of log-likelihood function, $l(\theta,x)$, where $\theta_0$ is the true parameter. As $n \to \infty$: the data points $\{x_i\} \to f(x|\theta_0)$, aka. the true distribution. This property can be combined with the fact that $\hat{\theta}_n$ maximizes likelihood to understand the asymptotic behavior. The same can be said for e.g. derivatives of $l(\theta,x)$. 

\item Approximating log-likelihood function and its derivatives: around the true value $\theta_0$ (since we are interested in the convergence to some constants dependent on $\theta_0$). This allows us to express log-likelihood function and its derivatives in terms of $\theta_0$, entropy at $\theta_0$ and Fisher information at $\theta_0$, etc.  

\end{itemize}

Consistency and efficiency of estimators: 
\begin{itemize}
\item Consistency: if the estimator $W_n$ converges in probability to the true value $\theta_0$ as $n \rightarrow \infty$, then $W$ is a consistent estimator. 

\item Theorem: if $W_n$ is a sequence of estimators of $\theta$ satisfying $\E_{\theta}(W_n) - \theta \to 0$ and $\Var_{\theta}(W_n) \to 0$ as $n \to \infty$ for all values of $\theta$, then $W_n$ is a consistent estimator of $\theta$. \\
Proof: follows from the property of convergence in probability (using Chebshev's inequality). 

\item Motivation for efficiency: consistency does not say anything about the variance of the estimator. Ex. different estimators may all be consistent estimators of $\theta$, but their rate of convergence (or variance) is different. 
\begin{itemize}
\item Remark: the concept of efficiency is similar to the ``rate of convergence'' defined in the context of numerical analysis. Ex. for two sequences: 
\begin{equation}
\lim_{n \to \infty} \frac{1}{n} = 0 \qquad \lim_{n \to \infty} \frac{1}{n^2} = 0
\end{equation}
Apparently, both converge to 0, but the rate is different. So, we use the value of $k$ s.t. $n^k a_n$ converges to a positive number to measure the rate of convergence. 
\end{itemize}
Our idea is thus to study the rate of convergence of the variance of the estimator $\Var(W_n)$ (it converges 0 if $W_n \to \theta$ in probability). For a good estimator, its variance should approach the Cramer-Rao lower bound. 

\item Limiting variance and asymptotic variance: for technical reasons (see Example 10.1.8), we do not directly study the rate of convergence of the sequence $\Var(W_n)$. Instead, we define: for estimator $T_n$, if $k_n (T_n - \theta) \to N(0,\sigma^2)$ in disribution, where $k_n$ is a sequence of constants, then $\sigma^2$ is called the asymptotic variance of $T_n$. 

\item Efficient estimator: estimator $W_n$ is asymptotically efficient for a parameter $\tau(\theta)$, if 
\begin{equation}
\sqrt{n} \left[ W_n - \tau(\theta) \right] \to N(0, v(\theta)) \text{ in distribution}
\end{equation}
where $v(\theta)$ is the Cramer-Rao Lower Bound: 
\begin{equation}
v(\theta) = \frac{[\tau'(\theta)]^2}{I(\theta)}	
\end{equation}

\end{itemize}

Convergence of log-likelihood function and its derivatives: 
\begin{itemize}
\item Motivation: we often need to study the convergence of log-likelihood function and its derivatives (these are all sample statistics). These are related to the entropy, score and Fisher information of the distribution. 

\item Log-likelihood function: we evaluate the log-likelihood function at $\theta$: 
\begin{equation}
\frac{1}{n} \log L(\theta|X) = 	\frac{1}{n} \sum_i \log f(x_i|\theta) \to \E[\log f(X|\theta)]
\end{equation}
where the expectation is taken over the true distribution. We approximate this at $\hat{\theta}$: 
\begin{equation}
\frac{1}{n} \log L(\hat{\theta}|X) \approx H(\theta)
\end{equation}
Thus the average log-likelihood function at MLE approximates the entropy of the distribution. 

\item Derivatives of log-likelihood function: similarly, we have
\begin{equation}
\frac{1}{n} \frac{\partial}{\partial \theta}\log L(\theta|X) = 	\frac{1}{n} \sum_i \frac{\partial}{\partial \theta} \log f(x_i|\theta) \to \E\left[\frac{\partial}{\partial \theta} \log f(X|\theta)\right]
\end{equation}
where the expectation is taken over the true distribution. Approximation at $\hat{\theta}$: 
\begin{equation}
\frac{1}{n} \frac{\partial}{\partial \theta}\log L(\hat{\theta}|X) \approx 0
\end{equation}
Similary for the second derivatitive: 
\begin{equation}
\frac{1}{n} \frac{\partial^2}{\partial^2 \theta}\log L(\theta|X) = 	\frac{1}{n} \sum_i \frac{\partial^2}{\partial^2 \theta} \log f(x_i|\theta) \to \E\left[\frac{\partial^2}{\partial^2 \theta} \log f(X|\theta)\right]
\end{equation}
Approximation at $\hat{\theta}$, we have the observed information as an estimator of Fisher information: 
\begin{equation}
\hat{I}_n(\hat{\theta}) = - \frac{\partial^2}{\partial \theta^2} \log L(\hat{\theta}|X) \approx I_n(\theta)
\end{equation}

\item Remark: the log-likelihood and its derivatives are sample means of entropy, average score (0) and Fisher information, so they follow asymptotic normality according to CLT. 
\end{itemize}

Consistency and efficiency of MLE: 
\begin{itemize}
\item Intuition: the true value $\theta_0$ should generally lead to large likelihood. Speaking in other words, as the sample size gets very large, the MLE and the true value $\theta_0$ should be very close.  

\item MLE consistency: let $\hat{\theta}_n$ be the MLE of $\theta$, and $\tau(\theta)$ be a continuous function of $\theta$, we have $\tau(\hat{\theta}_n) \rightarrow \tau(\theta_0)$ in probability, as $n \rightarrow \infty$.

\item Proof of MLE consistency: only consider the case $\tau(\theta) = \theta$. We first see that the log-likelihood function: 
\begin{equation}
l_n(\theta)	= \sum_{i=1}^n \log f(x_i|\theta)
\end{equation}
where $f(\cdot)$ is the pdf. As $n \rightarrow \infty$, the distribution of $x_i$ approaches the true PDF, $f(x|\theta_0)$, thus we have: 
\begin{equation}
\frac{1}{n} l_n(\theta) \rightarrow \int f(x|\theta_0) \log f(x|\theta) dx = \E[\log f(X|\theta)]
\end{equation}
From the KL divergence, we know that the above integral is maximized at $\theta = \theta_0$ (i.e. the log. of the density equal to the true density). Therefore, as $n \rightarrow \infty$, $\theta_0$ maximizes $l_n(\theta)$, i.e. $\hat{\theta}_n \rightarrow \theta_0$. 

\item Efficiency of MLE - single parameter: let $n$ be the sample size, $\hat{\theta}_n$ be MLE, and let $\tau({\theta})$ be a continuous function of $\theta$, then under certain regularity conditions, as $n \rightarrow \infty$, we have: 
\begin{equation}
\sqrt{n} (\tau(\hat{\theta}_n) - \tau(\theta_0)) \rightarrow N\left(0, \frac{[\tau'(\theta_0)]^2}{I_1(\theta_0)}\right)
\end{equation}
where $\theta_0$ is the true value of $\theta$. The variance term is the Cramer-Rao lower bound, thus, $\tau(\hat{\theta}_n)$ is asymptotically efficient estimator of $\tau(\theta)$. Note: in the RHS, to have a constant, instead of variance that depends on $n$, we have $I_1(\theta)$, thus $\sqrt{n}$ term in the LHS. 

\item Proof idea: we consider only the case where $\tau(\theta) = \theta$. To show $\hat{\theta}_n$ is close to $\theta_0$, the idea is, if for some function, the values at $\hat{\theta}_n$ and $\theta_0$ and very close, then the two must be close by the Taylor expansion. Another idea is that: the convergence of log-likelihood function (and its derivatives) can be approximated by CLT since it can be written as an average. We choose this function to be the derivative of log-likelihood function, taking its expansion near $\theta_0$: 
\begin{equation}
l'(\theta|x) = l'(\theta_0|x) + (\theta - \theta_0) l''(\theta|x)
\end{equation}
At $\theta = \hat{\theta}_n$, the LHS is 0, so we have: 
\begin{equation}
\sqrt{n} (\hat{\theta}_n - \theta_0) = \frac{-\frac{1}{\sqrt{n}} l'(\theta_0|x)}{\frac{1}{n}l''(\theta_0|x)}
\end{equation}
For the numerator, we apply the CLT:
\begin{equation}
-\frac{1}{\sqrt{n}} l'(\theta_0|x) \rightarrow N(0, I_1(\theta_0))	
\end{equation}
where the Fisher information is the variance of the score (first derivative of the log-likelihood) For the denomator, we apply the WLLN: 
\begin{equation}
\frac{1}{n}l''(\theta_0|x) \rightarrow I_1(\theta_0)	
\end{equation}
where the Fisher information is the second derivative of the log-likelihood. 

\item Efficiency of MLE - multiple parameters: [Wiki] for the case of simple function $\tau(\theta) = \theta$: 
\begin{equation}
\sqrt{n} (\hat{\theta} - \theta_0) \rightarrow N(0,I^{-1}(\theta_0))	
\end{equation}
where $I$ is the Fisher information matrix (single point, or PDF) evaluated at $\theta_0$ (in practice, replace with $\hat{\theta}$). This theorem can be used to derive the confidence interval of MLE.  

\end{itemize}

Studying asymptotic efficiency/variance of estimators: 
\begin{itemize}
\item General strategy: for a parameter estimation problem, 
\begin{itemize}
\item Consistent estimator: the first step is generally to find a consistent estimator $W_n \to \theta$ in probability. For functions of parameters, we could use the Theorem: $h(W_n) \to h(\theta)$ in probability if $h$ is a continuous function. 

\item Standard error: the next question is often the variance (or standard error) of the estimator. The first step:  
\begin{equation}
\Var_{\theta}(W_n) = \sigma^2_n(\theta)	
\end{equation}
Note that the variance is a function of the true value $\theta$. Since $\theta$ is unknown, we have the second step, which replaces $\theta$ with $\hat{\theta}$:
\begin{equation}
\hat{\Var}_{\theta}(W_n) = 	\Var_{\theta}(W_n)|_{\theta = \hat{\theta}}
\end{equation}

In some other cases (e.g. Wald test for parameters), we need a consistent estimator of the variance, i.e. $S_n / \sigma_n \to 1$ in probability. 

\item Asymptotic normality: for many cases, e.g. for testing parameters, we want to estabilish normality of the estimator. Typically, we have: 
\begin{equation}
\frac{W_n - \theta}{\sigma_n} \to N(0,1)	
\end{equation}
Or we may replace $\sigma_n$ with $S_n$ (may not be necessary if we are working on the distribution under $H_0$). 

\item Basic tools: for studying asymptotic distribution include CLT and the Delta Method (if the estimator is a function of sample mean), and the asymptotic normality of MLE. 
\end{itemize}

\item Approximation of variance of MLE: we apply the two-step procedure for the variance of MLE. First,  
\begin{equation}
\text{Var}(\tau(\hat{\theta})) \approx \frac{[\tau'(\theta)]^2}{I_n(\theta)} 
\end{equation}
Next, from our discussion of covergence of derivative of log-likelihood function, we know that observed information is a consistent estimator of $I_n(\theta)$. 
\begin{equation}
\text{Var}(\tau(\hat{\theta})) \approx \frac{[\tau'(\theta)]^2|_{\theta = \hat{\theta}}} {\hat{I}_n(\hat{\theta})} 	
\end{equation}

\item Example: suppose $X_1, X_2, \cdots X_n$ iid. Bernoulli$(p)$ distribution. The estimator of $p$ is $\hat{p} = X/n$. We have two ways of obtaining its variance. First, direct calculation: 
\begin{equation}
\Var_p(\hat{p}) = \frac{p(1-p)}{n}	
\end{equation}
We approximate this at $p = \hat{p}$: 
\begin{equation}
\hat{\Var}_p(\hat{p}) = \frac{\hat{p}(1-\hat{p})}{n}	
\end{equation}
And the distribution is normal. The second way is to apply the MLE approximation above: 
\begin{equation}
\hat{I}_n(\hat{p}) = -\frac{\partial^2}{\partial^2 p} \log L(\hat{p}|X)	= \frac{n}{\hat{p}(1 - \hat{p})}
\end{equation}
From both methods, we have: 
\begin{equation}
\sqrt{n} \frac{\hat{p} - p}{\sqrt{\hat{p}(1 - \hat{p})}} \to N(0,1)
\end{equation}
\end{itemize}

\subsection{Large Sample Tests}

Reference: [Casella, Statistical Inference, Chapter 10]

Likelihood ratio test: 
\begin{itemize}
\item LRT: suppose we are testing $H_0: \theta \in \Theta_0$ vs. $H_1: \theta \in \Theta_1$, the test statistic
\begin{equation}
\lambda(x) = \frac{\sup_{\Theta_0}L(\theta|x)}{\sup_{\Theta_1}L(\theta|x)}	
\end{equation}

\item Theorem (asymptotic distribution of LRT - simple $H_0$): for test $H_0: \theta = \theta_0$ vs. $H_1: \theta \neq \theta_0$, under $H_0$, as $n \to \infty$: 
\begin{equation}
-2 \log \lambda(X) \to \chi^2_1 \text{ in distribution}	
\end{equation}
Proof: suppose $\hat{\theta}$ is the MLE under $H_1$, Taylor expansion of $\log L(\theta|X)$ near $\hat{\theta}$: 
\begin{equation}
l(\theta|x) = l(\hat{\theta}|x) + l'(\hat{\theta}|x) (\theta - \hat{\theta}) + l''(\hat{\theta}|x) \frac{(\theta - \hat{\theta})^2}{2} + \cdots
\end{equation}
Thus: 
\begin{equation}
-2 \log \lambda(x) = -2 l(\theta_0|x) + 2l(\hat{\theta}|x) \approx -l''(\hat{\theta}|x) (\theta - \hat{\theta})^2
\end{equation}
Using the Theorem of MLE efficiency, we have that this converges in distribution to $\chi^2_1$. 

\item Theorem (asymptotic distribution of LRT - composite $H_0$): for composite $H_0$, we have that $-2 \log \lambda(x)$ converges in distribution to $\chi^2_k$, where $k$ is the difference between the number of free parameters in $\Theta_0$ and $\Theta_1$. 
\end{itemize}

Wald test:  
\begin{itemize}
\item Test: suppose we have an estimator $W_n \in \theta$, if $W_n$ has asymptotic normal distribution then we can construct a $Z$ score. Suppose $S_n$ is an estimator of $\sigma_n$, the standard error of $W_n$, with $\sigma_n / S_n \to 1$. Then to test $H_0: \theta = \theta_0$, we define: 
\begin{equation}
Z_n = \frac{W_n - \theta_0}{S_n}	
\end{equation}
which converges in distribution to $N(0,1)$ under $H_0$. When $W_n$ is the MLE, we have: 
\begin{equation}
S_n = \frac{1}{\sqrt{\hat{I}_n(W_n)}}	
\end{equation}

\item Example: suppose $X_1, X_2, \cdots X_n$ iid. Bernoulli$(p)$ distribution. The estimator of $p$ is $\hat{p} = X/n$. From CLT, 
\begin{equation}
\frac{\hat{p} - p}{\sigma_n} \to N(0,1) \text{ in distribution}	
\end{equation}
where $\sigma_n = \sqrt{p(1-p)/n}$. An estimate of $\sigma_n$ is $S_n = \sqrt{\hat{p}(1-{p})/n}$, so we have: 
\begin{equation}
\sqrt{n} \frac{\hat{p} - p_0}{\sqrt{\hat{p}(1 - \hat{p})}} \to N(0,1)
\end{equation}

\item Remark: note that under $H_0$, the value of $\theta$ may be specified, so $\sigma_n$ may be a given function of $\theta$, and in this case, we could use $\sigma_n$ instead of $S_n$ in the test. For the Bernoulli example, to test if $p = p_0$, this is: 
\begin{equation}
\sqrt{n} \frac{\hat{p} - p_0}{\sqrt{p_0(1-p_0)}} \to N(0,1)
\end{equation}
If $H_0$ is not simple, e.g. to test if $p \leq p_0$, then $\sigma_n$ is not known, and we would need $S_n$ instead of $\sigma_n$ in the test. 
\end{itemize}

Score test: 
\begin{itemize}
\item Intuition: our goal is to test $H_0: \theta = \theta_0$. If $\theta_0$ is the true value that generates the data, then the score $S(\theta_0) \approx 0$ because $\E(S(\theta_0)) = 0$. So we use $S(\theta_0)$ as our test statistic, whose null distribution has mean 0. Larger values would reject $H_0$. When $H_0$ is not true, the data is not generated from $\theta_0$, but we evaluate the score at $\theta_0$, so the score is not necessarily close to 0. 

\item Relation between score and MLE: suppose our data is generated by $\theta$, then the score at $\theta$, $S(\theta)$ is close to 0 as sample size gets large, because the expectation of $S(\theta)$ is 0. On the other hand, at MLE $\hat{\theta}$, the derivative $\partial l(\theta) / \partial \theta = 0$, thus the MLE must be close to the true value $\theta$. 
\item Test: the derivative of the log-likelihood function is score:
\begin{equation}
S(\theta) = \frac{\partial}{\partial \theta} \log L(\theta|x)	
\end{equation}
This is a function of RV $x$. Its mean over $x$ is 0 (see the sectoin on ``Fisher information''), and variance is simply the Fisher information $I(\theta)$. To test $H_0: \theta = \theta_0$, we define: 
\begin{equation}
Z_S = \frac{S(\theta_0)}{\sqrt{I(\theta_0)}}	
\end{equation}
Under $H_0$, $Z_S$ has mean 0 and variance 1, and $Z_S|H_0$ follows $N(0,1)$ (see Convergence on the log-likelihood function and its derivatives). To test composite $H_0$, we replace $\theta_0$ with the maximum of $\theta$ under $H_0$. 

\item Alternative (multiple variable) forms of the score test: let $U$ be the score and $I$ be its variance (Fisher information matrix), then $U^2/I \sim \chi^2_1$. In multi-dim. case, we have 
\begin{equation}
S = U^T I^{-1} U|_{\theta_0} \sim \chi^2_k	
\end{equation}
where $k$ is the rank of $I$. 

\item Binomial score test. To test $H_0: p = p_0$, we have score and Fisher information:
\begin{equation}
S(p) = \frac{x}{p} -\frac{n - x}{1 - p} = \frac{\bar{p} - p}{p(1-p) / n}	\qquad I(p) = \frac{n}{p(1-p)}
\end{equation}
The score test statistic is thus: 
\begin{equation}
Z_S = \frac{S(p_0)}{\sqrt{I(p_0)}} = \frac{\bar{p} - p_0}{\sqrt{p_0 (1 - p_0) / n}}	
\end{equation}
which is the same as Equation~\ref{eq:binomial_test}. 

\item Pearson's chi-square test as a score test. We are testing if a table follows multinomial distribution with paramater $\mu_i$ for the $i$-th cell (or if rare, we use Poisson distributions). Our data is $\{O_1, \cdots, O_n\}$. The likelihood is then: 
\begin{equation}
L(\mu) = \prod_i P(O_i|\mu_i) = \prod_i \frac{\mu_i^{O_i} e^{-\mu_i}}{O_i!}	
\end{equation}
The log-likelihood is: 
\begin{equation}
l(\mu) = \sum_i (O_i \log \mu_i - \mu_i - \log O_i!)	
\end{equation}
This allows to compute the derivative and second derivative and we have: 
\begin{equation}
\frac{\partial l}{\partial \mu_i}	= \frac{O_i}{\mu_i} - 1 \qquad \frac{\partial l^2}{\partial \mu_i \partial \mu_j} = - \delta_{ij} \frac{O_i}{\mu_i^2}
\end{equation}
The Fisher information matrix is: 
\begin{equation}
I = - \E\left[ \frac{\partial l^2}{\partial \mu_i \partial \mu_j} \right]	= \left[ \delta_{ij} \frac{\E(O_i)}{\mu_i^2} \right] = \text{Diag}\left( \frac{1}{\mu_1}, \cdots, \frac{1}{\mu_n} \right)
\end{equation}
And its inverse is $\text{Diag}(\mu_1, \cdots, \mu_n)$. The score test is: 
\begin{equation}
S = U^T I^{-1} U = \sum_i \mu_i \left(\frac{O_i}{\mu_i} - 1\right)^2 = \sum_i \frac{(O_i - \mu_i)^2}{\mu_i}
\end{equation}
where $U$ is the score vector. $S$ follows chi-square distribution, as discussed earlier. 

\item Linear model score test: suppose we have a simple regression model, $y = \beta_0 + x\beta + \epsilon$, and we test $H_0: \beta = 0$. For simplicity, assume $\beta_0$ is known. The log-likelihood function: 
\begin{equation}
l(\beta) = -\frac{1}{2 \sigma^2} \sum_i (y_i - \beta_0 - x_i \beta)^2 + \text{const}
\end{equation}
The score is then: 
\begin{equation}
S(\beta) = \frac{\partial l(\beta)}{\partial \beta} = \sigma \sum_i (y_i - \beta_0 - x_i \beta) x_i
\end{equation}
At $\beta=0$, we have 
\begin{equation}
S(\beta) = \sigma \sum_i (y_i - \beta_0) x_i = \sigma (Y - \beta_0) \cdot X 
\end{equation}
So it is proportional to the inner product of $Y$ (if it is standardized) and $X$. The Fisher information:
\begin{equation}
I(\beta) = - \frac{\partial S(\beta)}{\partial \beta} = \sigma \sum_i x_i^2
\end{equation}
The test statistic is thus: 
\begin{equation}
Z = \left.\frac{S(\beta)}{\sqrt{I(\beta)}}\right|_{\beta=0} = \frac{(y-\beta_0) \cdot x}{\norm{x}}
\end{equation}

\item Remark: it is also called Langrange Multiplier test (in econometrics literature), because the test involves maximization of likelihood of a restricted model ($H_0$), which may often be obtained by Langrange Multiplier. 
\end{itemize}

Comparison of LRT, Score test and Wald test: suppose we have $H_0: \theta=\theta_0$, and $\hat{\theta}$ denotes the MLE of $\theta$ under $H_1$ (more general model).   
\begin{itemize}
\item In the log-likelihood function surface: LRT is based on the difference in the $y$-axis, $l(\hat{\theta}) - l({\theta}_0)$; Wald test is based on the difference in the x-axis, $\hat{\theta} - \theta_0$; and Score test is based on the derivative $\frac{\partial l(\theta)}{\partial \theta}|_{\theta_0}$. See Figure in: \url{http://www.ats.ucla.edu/stat/mult_pkg/faq/general/nested_tests.htm}

\item LRT needs to evaluate MLE under both $H_0$ and $H_1$; Wald test to evaluate MLE under $H_1$; and Score test to evaluate MLE under $H_0$. Since $H_0$ is often simple, Score test may avoid the difficult optimization problem of calculating MLE under $H_1$. 
\end{itemize}
%%%%%%%%%%%%%%%%%%%%%%%%%%%%%%%%%%%%%%%%%%%%%%%%%%%%%%%%%%%%
\section{Information Theory}

Concepts:
\begin{itemize} 
	\item Entropy: for a RV with pdf. $f(x)$, its entropy is defined as the negative of the expected information content: 
	\begin{equation}
	H(X) = - \int f(x) \ln f(x) dx	
	\end{equation}
	
	\item KL divergence. Given two probability distributions $P$ and $Q$, the ``distance'' of $P$ from $Q$ is defined by the KL divergence: 
	\begin{equation}
	H(P||Q) = \int \log \frac{P(x)}{Q(x)} P(x) dx	
	\end{equation}
	So it is the expected log ratio of pdf of $P$ and $Q$, averaged over $P$. The KL divergence has the important property: for any $P$ and $Q$
	\begin{equation}
	H(P||Q) \geq 0	
	\end{equation}
	\begin{itemize}
		\item Proof: let $Y = \frac{Q(X)}{P(X)}$ be a random variable (function of $X$). $Y$ represents the difference of density between the two distributions and $\E_P(Y) = 1$. Thus we have $H(P||Q) = - E_P(\log Y)$. Using Jensen's inequlity: 
		\begin{equation}
		\E_P(\log Y) \leq \log \E_P(Y)= 0
		\end{equation}
	\end{itemize}
	
	\item Remark: in KL divergence, we can think of $Q$ as true distribution, and $P$ as the empirical distribution (data). Then we should integrate over $P$, as we consider the log-likelihood over all data. 
\end{itemize}

Maximum entropy method: 
\begin{itemize}
	\item Background: calculus of variation. Example, for a physical system with certian state function $f(x)$ (e.g. pressue), where $x$ represents the spatial coordinate. Suppose the entropy (density) at the point $x$ is related to the state at $x$ by: $\phi(f(x))$, where $\phi$ is given, then the total entropy of the system is: 
	\begin{equation}
	S[f] = \int \phi(f(x)) dx 
	\end{equation}
	We see that $S$ is a ``functional'' of the state function $f$, and to apply the Second Law of Thermodynamics, we need to maximize $S$ wrt. $f$, typically subject to certain constraint (e.g. mass conservation): 
	\begin{equation}
	\int f(x) dx = C
	\end{equation}
	
	\item Maximum entropy: given certain constraints of a probability distribution, typically given in the form of moments, the unknown distribution should be the one that maximizes the entropy. Ex. the distribution that maximizes the entropy subject to:
	\begin{equation}
	\E(X) = \mu \qquad \Var(X) = \sigma^2
	\end{equation}
	is the normal distribution $N(\mu, \sigma^2)$. 
	
	\item Remark: a generalization of the Method of Moments of parameter estimation. Instead of assuming a parameteric form, we find the distribution (with maximum entropy) that matches the moments of unknown distribution and the empirical distribution (data). 
\end{itemize}

Parameter estimation from information theory perspective:
\begin{itemize}
	\item Minimizing KL divergence: the idea is that $\theta$ should minimize $f_{\theta}$ and the empirical distribution $\hat{f}$. We have: 
	\begin{equation}
	KL(\hat{f}||f_{\theta}) = \int \hat{f}(x) \log \hat{f}(x) dx - \int \hat{f}(x) \log f_{\theta}(x)	dx
	\end{equation}
	Clearly, minimizing KL divergence is equivalent to maximizing the second term, which is the log-likelihood function (divided by $n$). Thus minimizing KL divergence leads to ML estimator (for parameteric distributions). 
	
	\item Implications to nonparameteric methods: the above equation applies to all cases, even if we do not have a parameteric form of $f$. In this case, we could choose $f$ (subject to certain constraints) that has the lowest $KL(\hat{f}||f)$. 
	
	\item Sample entropy: the term
	\begin{equation}
	S = \int \hat{f}(x) \log \hat{f}(x) dx	
	\end{equation}
	is the entropy in the sample. It represents all ``uncertain'' in the data, and in the perfect case, equal to the maximum log-likelihood funciton. 
\end{itemize}

Hypothesis testing from information theory perspective: 
\begin{itemize}
	\item KL divergence: to test if $\theta = \theta_0$, the idea is that the KL divergence $KL(\hat{f}||f_{\theta_0})$ should be really close to 0. This could be used for a goodness-of-fit test, e.g. the normality of a distribution [A test of normality based on sample entropy]. 
\end{itemize}
%%%%%%%%%%%%%%%%%%%%%%%%%%%%%%%%%%%%%%%%%%%%%%%%%%%%%%%%%%%%

\section{Model Selection}

Overview of model comparison and selection:
\begin{itemize}
	\item Purposes: compare multiple models: choose the best one or testing if one is significantly better than another. Often, this is needed to compare models with different complexities or select models with the appropriate complexity.
	
	\item Scenarios: we have basically two scenarios: 
	\begin{itemize}
		\item Probability distributions/processes: which distribution/process better explains the data, $\{x_1, \ldots x_n\}$.  
		\item Predictive models: which model better predicts responses $y_i$ from predictors, $x_i$, where $1 \leq i \leq n$.  
	\end{itemize}
	The two different scenarios are equivalent, in that an approach for one case could be adapted to the other. Specifically, assuming some error distribution of $y_i$ from $f(x_i)$, where $f(\cdot)$ is the true function, then the latter problem can be reduced to the former. Similarly, the former problem, is the latter problem when the predictors are empty. 
	
	\item Goodness-of-fit: another scenarior is to test if a model is good enough to explain the data, without involving comparison of multiple models. 
\end{itemize}

Methods for model selection: 
\begin{itemize}
\item Hypothesis testing: if models are nested, then reduced to the problem of testing parameter values: Neyman-Pearson paradigm, Likelihood-ratio test (or asympotic results in general), etc. If non-nested, could still use LRT, but the distribution will have to be obtained from e.g. bootstrapping [Goldman, JME, 1993]; or use other tests, e.g. J-test. 
\item Bayesian model selection: computing the Bayes factor, which involves integrating over all parameter values.
\item Analytic methods: BIC as an approximation of Bayes factor, MDL, etc.  
\item Cross validation: divide the data into training and validation sets, and define a measure of error, typically expected prediction error for predictive models (with certain loss function). But it could be other reasonable measure of errors. 
\end{itemize}

Model assessment problem [Hastie, Section 7.2]: 
\begin{itemize}
	\item Prediction error: our problem is to learn a function $f: X \rightarrow Y$. Suppose the function learned from training data is $\hat{f}$, the loss function is denoted as $L(Y,\hat{f}(X))$, then the model is assessed by the test error or generalization error, defined as: 
	\begin{equation}
	\textrm{Err} = E[L(Y,\hat{f}(X))]
	\end{equation}
	
	\item Training error: the training error is the average loss over the training samples:
	\begin{equation}
	\bar{\textrm{err}} = \frac{1}{N} \sum_{i=1}^N L(y_i, \hat{f}(x_i))	
	\end{equation}
	However, the training error is not a good prediction of the test error.
\end{itemize}

Typical loss functions: 
\begin{itemize}
\item Squared error loss: $L(Y,f(X)) = (Y - f(X))^2$. 
\item Zero-one loss: $L(Y,f(X)) = I(Y \neq f(X))$. 
\item Log likelihood loss: if instead of a function $f$, we want to estimate the parameter $\theta$ of some density function, then $L(Y,\theta(X)) = -2 \log P(Y|\theta(X))$. 
\end{itemize}

Structural risk minimization [Murphy, Section 6.5]
\begin{itemize}
	\item Let $\lambda$ be a parameter for penalizing model complexity, we should choose a model, denoted as $\delta$, to minimize: 
	\begin{equation}
	\delta_{\lambda} = \text{argmin}_{\delta} \left[R_{emp}(D,\delta) + \lambda C(\delta)\right]
	\end{equation}
	where $R_{emp}(D,\delta)$ is the empirical risk (assess the fitting of data by $\delta$), and $C(\delta)$ controls complexity. The issue is: how to choose $\lambda$? 
	
	\item Choosing $\lambda$ by cross-validation: suppose we partition the data into $K$ fold, and let $D_{k}$ be the test data of fold $k$, and $D_{-k}$ be the training data. Under a given $\lambda$, the CV estimate of the risk is: fit $D_{-k}$ and let the best function be $f_{\lambda}^k(\cdot)$. We then apply it to every data point $i \in D_k$, and let the error/loss be $L(y_i, f_{\lambda}^k(x_i))$. Summing over all data points in $D_k$ gives the risk in the $k$-th fold. We then sum over all folds, and obtain the average risk for each data point. See Equation (6.60) in Murphy: 
	\begin{equation}
	R(\lambda, D, K) = \frac{1}{N} \sum_{k=1}^K \sum_{i \in D_k} L(y_i, f_{\lambda}^k(x_i))
	\end{equation}
	So $\lambda$ should be chosen to minimize this function. 
	
	\item Example: using CV to pick $\lambda$ for ridge regression. The loss function is Negative log-likelihood (NLL), or equivalently, squared error. The parameter estimation for a given $\lambda$ is the MAP estimate with normal prior. 
	
	\item 1 standard error (1SE) rule: the difference of loss may not be significant under different $\lambda$, so we choose the largest $\lambda$ s.t. it is still within 1 SE of the value of $\lambda$ that minimizes the risk. 
	
	\item Choosing 	$\lambda$ by Empirical Bayes: suppose $\lambda$ is a hyperparameter of $\beta$ prior, we choose $\lambda$ by: 
	\begin{equation}
	\hat{\lambda} = \text{argmax}_{\lambda} \int P(y|\beta) P(\beta|\lambda) d\beta
	\end{equation}
	Note that this involves integration over $\beta$, which can be computationally expensive. 
\end{itemize}
%%%%%%%%%%%%%%%%%%%%%%%%%%%%%%%%%%%%%%%%%%%%%%%%%%%%%%%%%%%%
%%%%%%%%%%%%%%%%%%%%%%%%%%%%%%%%%%%%%%%%%%%%%%%%%%%%%%%%%%%%
\chapter{Bayesian Inference}

\section{Bayesian Statistics Background}

Bayesian paradigm: 
\begin{itemize}
\item Setting up the model: this involves two parts, the likelihood function and the prior. The prior distribution should reflect prior belief, or noninformative. An important issue is to ensure the posterior distribution is proper (finite integral) when the prior is improper. 

\item Posterior inference: the basic equation: 
\begin{equation}
P(\theta|y) \propto P(\theta) P(y|\theta)	
\end{equation}
In some simple cases, the posterior distribution can be analytically determined. In most cases, however, we will need to sample $\theta$ from $p(\theta|y)$. From the posterior samples, it is easy to obtain posterior summary (mean, median, quantile and posterior interval, etc.) and any quantity/function of parameters of interest.  

\item Posterior predictive distribution: defined as 
\begin{equation}
P(\tilde{y}|y) = \int P(\tilde{y}|\theta) P(\theta|y) d\theta
\end{equation}
When we already have the posterior sample, $\theta^l, l = 1, \cdots, L$, we could sample $\tilde{y}$ from $p(\tilde{y}|\theta)$ using the sampled values of $\theta$. 

\item Model checking: one strategy is to simulate replicate data using the posterior predictive distribution, and compare with the observed data to see if there is discrepancy. 
\begin{itemize}
	\item Example: in the ETS example (Section 5.5), simulate the data of 8 schools, and check the maximum of the simulated data. The observed maximum can be compared with simulation to estimate the probability that the maximum could reach the observed maximum. 
\end{itemize}

\item Remarks: 
\begin{itemize}
	\item Problem of posterior mode: this may not reflect the uncertainty in the inference, and insufficient to capture the inference results. Ex. in the ETS example [Section 5.5] of hierarchical normal model, the posterior mode of $\tau$ (the variance of group means) is 0, and if we accept it, all groups have the same mean (complete pooling), and there is no benefit of Bayesian. 
\end{itemize}
\end{itemize}

Advantages of Bayesian over frequentist statistics (especially on hypothesis testing): [personal notes]
\begin{itemize}
\item Reference: [Servin \& Stephens, Imputation-Based Analysis of Association Studies: Candidate Regions and Quantitative Traits, PLG, 2007]. 

\item Advantage of Bayes factor over $p$-values: $p$-values do not reflect the power of the test, thus a small $p$-value may not mean much if the test is also very unlikely under the $H_1$ for a test with very low power. Thus when the problem effectively involves combining multiple tests, $p$-values from low-powered ones (less informative ones) create noises that hide signals from the most informative ones. Examples:
\begin{itemize}
	\item GWAS: test association of multiple SNPs in a region with the trait. If a number of SNPs in the region are not informative (irrelevant or in low LD with the causal variant), $p$-value of the region loses power. 
	\item Meta-analysis of multple studies: when the power (sample size, MAF for SNPs, etc.) of the studies are different. 
	\item Combining multiple evidence of a gene: de novo data, case/control data. The power may be very different. 
\end{itemize}
\end{itemize}

Lessons for Bayesian modeling in practice [personal notes]: 
\begin{itemize}
\item Analyzing influence of prior: the use of prior may create bias in the inference problem (overriding data), so the results sometimes may not be desired. This has to be analyzed carefully. Ex. TADA case-control model, if the prior of $q$ is strong, then it is possible to have weird cases, e.g $B(1,2) > 1$. So one should analyze the nature of this bias and see if it is acceptable. 

\item Check distribution of Bayes factors: when Bayesian statistics is used for decision: model selection/hypothesis testing, it is important to check if the model leads to false decision. If the inference is based on Bayes factor (BF), then it's important to check the distribution of BF under the null model (when we should not make a certain decision). 

\item Sensitivity analysis: how robust the results are to the parameters, specifically prior parameters. In the hypothesis testing problem, this means analyzing how power and false positive rate depend on parameters. 
\end{itemize}

Issues of Bayesian inference: 
\begin{itemize}
\item Exchangability: the data $\{y_i, 1 \leq i \leq n\}$ is exchangable, if the joint density $p(y_1, \cdots, y_n)$ is invariant to permutations of the indices. Exchangability reflects our ignore of the difference between data points (other than those reflected in the explanatory variables). 
\begin{itemize}
	\item Exchangability vs iid: often we model exchangable data as iid. conditioned on unknown parameters. However, the two concepts are not the same, e.g. $(X_1, X_2)$ follows bivarirate normal distribution, thus they are exchangable, but certainly not iid. 
\end{itemize}

\item Inference of models with nusiance parameters: suppose we have a model with parameters $\theta_1$ and $\theta_2$, and we are interested in only $\theta_1$. The joint posterior distribution of $\theta_1, \theta_2$ is: 
\begin{equation}
p(\theta_1, \theta_2|y) \propto p(\theta_1, \theta_2) p(y|\theta_1, \theta_2)	
\end{equation}
There are two ways of obtaining $p(\theta_1|y)$. First, we average the joint posterior over $\theta_2$: 
\begin{equation}
p(\theta_1|y) = \int p(\theta_1, \theta_2|y) d\theta_2	
\end{equation}
Second, suppose the conditional posterior distribution of $\theta_1$ when $\theta_2$ is given is easy to obtain (e.g. for normal distribution with known variance), then we have: 
\begin{equation}
p(\theta_1|y) = \int p(\theta_1| \theta_2,y) p(\theta_2|y) d\theta_2		
\end{equation}
Note that this would need to determine $p(\theta_2|y)$. Yet another way of obtaining $p(\theta_1|y)$ through conditional distribution is: 
\begin{equation}
p(\theta_1|y)	= \frac{p(\theta_1, \theta_2|y)}{p(\theta_2|\theta_1,y)} = \frac{p(\theta_2|y) p(\theta_1|\theta_2,y)}{p(\theta_2|\theta_1,y)}
\end{equation}
Note that the equation is valid for any value of $\theta_2$, thus one may plug-in a special value of $\theta_2$. However when applying this equation, the constant term in the denominator depends on $\theta_1$, thus effectively, we need to determine the normalizing constant of $p(\theta_2|\theta_1,y)$, requiring integration over $\theta_2$. 
\end{itemize}

Prior distributions: [GCSR, Section 2.9]
\begin{itemize}
\item Conjugate prior: choose the form of prior s.t. the posterior distribution would have the same form of distributions. 

\item Proper and improper prior: prior may often be improper, e.g. 
\begin{equation}
p(\theta) \propto 1
\end{equation}
However, given improper prior, the posterior distribution may be proper. Ex. for normal distribution $N(\mu,\sigma^2)$ where $\sigma^2$ is know, under the uniform prior of $\mu$, the posterior is proper. 

\item Checking posterior by simulation: to check if posterior is proper (when prior is improper), sample the posterior distribution, and test if it is proper (e.g. the probability should be close to 0 when the parameters approach infinity). 

\item Common noninformative prior distributions: in general, find the right scale (transformation) s.t. the prior is uniform. Examples: 
\begin{itemize}
\item $\theta > 0$: uniform at the log. scale, i.e. $\log(\theta) \propto 1$ or $\theta \propto 1/\theta$. 
\item $\theta \in [0,1]$: uniform at the logit scale, i.e. $\text{logit}(\theta) = \log\frac{\theta}{1 - \theta} \propto 1$. 
\item Location and scale parameters: normally, noninformative prior for location parameter is uniform, and the noninformative prior for the scale parameter is uniform in the log. scale. 
\end{itemize}

\item Dealing with improper posterior: generally, if some improper prior makes the posterior improper, change the prior distribution. The idea is to reduce the probability mass at $\theta \rightarrow \infty$ s.t. the posterior will be less dispersed. 
\begin{itemize}
	\item Ex. rat tumor example [Section 5.3]: if $p(\theta) \propto \theta^{-1}$ does not lead to proper posterior, change it to $p(\theta) \propto \theta^{-3/2}$. 
\end{itemize}
\end{itemize}

Normal approximation of posterior distribution: [GCSR, Chapter 4]
\begin{itemize}
\item Normal approximation: the Taylor expansion of $\log p(\theta|y)$ near the posterior mode: 
\begin{equation}
\log p(\theta|y) = \log p(\hat{\theta}|y) + \frac{1}{2} (\theta - \hat{\theta})^T \left[ \frac{d^2}{d \theta^2} \log p(\theta|y)\right]_{\theta = \hat{\theta}} (\theta - \hat{\theta}) + \cdots	
\end{equation}
Ignoring the higher order term, this is a quadratic function of $\theta$, thus it follows normal distribution: 
\begin{equation}
\theta | y \sim N(\hat{\theta}, [I(\hat{\theta})]^{-1})	
\end{equation}
where $I(\theta)$ is Fisher information: 
\begin{equation}
I(\theta) = - \frac{d^2}{d \theta^2} \log p(\theta|y)	
\end{equation}

\item Convergence to normal distribution at large sample size: [Appendix B]
\begin{itemize}
\item Theorem: given a parameter $\theta$, we define the KL divergence between the distribution $p(\cdot|\theta)$ and $f(y)$, defined as: 
\begin{equation}
H(\theta) = \int \log \left( \frac{f(y_i)}{p(y_i|\theta)}\right) f(y_i) d y_i
\end{equation}
Thus $H(\theta)$ measures how good $\theta$ is or how close $\theta$ is to the true value. Suppose the true parameter is $\theta_0$, i.e. $f(y) = p(y|\theta_0)$, then $\theta_0$ minimze the KL divergence. 

\item Theorem: under some regularity conditions, as $n \rightarrow \infty$, the posterior distribution of $\theta$ approaches normal distribution with mean $\theta_0$ and variance $(n J(\theta_0))^{-1}$, where $J$ is the Fisher information, and $\theta_0$ minimizes the KL divergence. 
\end{itemize}

\item Using normal approximation to estimate the posterior interval: suppose we want to find the posterior interval covering 95\% of posterior probability mass (in terms of log. density relative to the density at the mode), we note that if $X$ is a RV of d-dimensional MVN, then $\log p(X)$ is a quadratic function of $X$, and thus 
\begin{equation}
Y = -2 \log p(X)	
\end{equation}
is a RV with $\chi^2_d$ distribution. The interval of $X$ thus can be found via the interval of $Y$. Example, $d = 2$, the 95\% interval of $\chi^2_2$ is 5.99, this corresponds to the log. density above $\exp(-5.99/2) = 0.05$ relative to the density at the mode. This is useful in posterior sampling to find the region around the mode that covers most probability mass. 
\end{itemize}

Relation of Bayesian posterior distribution and estimator distrubtion under classical statistics [personal notes]: 
\begin{itemize}
	\item Motivation: suppose we are interested in the posterior of $\theta|D$. It often depends on the estimator $\hat{\beta}$. How would this posterior relates to the estimator distribution under classical statistics? Intuitively, if the estimator has large standard error under classical statistics, the posterior of $\theta$ would also have large posterior interval. 
	
	\item Let $\hat{\theta}$ be the estimator of $\theta$. Suppose it is a sufficient statistic, then we have $P(\theta|D) = P(\theta|\hat{\theta})$ since $\hat{\theta}$ captures all the information in the data. By Bayes Theorem: 
	\begin{equation}
	p(\theta|\hat{\theta}) \propto p(\hat{\theta}|\theta) p(\theta)
	\end{equation}
	The form of the distribution $p(\hat{\theta}|\theta)$ would be given by the results of classical statistics. In particular, when $p(\theta)$ is uniform, the two distributions would be identical if we view $p(\hat{\theta}|\theta)$ from a Bayesian perspective. The key of applying these results are: (1) show that $\hat{\theta}$ is a sufficient statistic; (2) from the distribution of estimator $p(\hat{\theta}|\theta)$, determine the distribution of $\theta$ given $\hat{\theta}$ (view $\hat{\theta}$ as given).  
	
	\item Example: normal distribution with known variance. Let $x_i$ iid $N(\mu, \sigma^2)$, and $|hat{\mu} = \frac{1}{n} \sum_i x_i$. From classical statistics, we know that $\hat{\mu} \sim N(\mu, \sigma^2/n)$. Our posterior: 
	\begin{equation}
	p(\mu|x) = p(\mu|\hat{\mu}) \propto p(\hat{\mu}|\mu) p(\mu) \propto N(\hat{\mu}|\mu, \sigma^2/n)
	\end{equation}
	assuming uniform prior of $\mu$. Suppose $\hat{\mu}$ is given, it is easy to see that $\mu$ follows normal $N(\hat{\mu}, \sigma^2/n)$. 
\end{itemize}

\subsection{Bayesian Model Selection} 

Reference: [Mackay, Chapters 27, 28], [Bishop, Section 3.4, 4.4]

Issues/thoughts about Bayesian model selection: 
\begin{itemize}
	\item Advice from Gelman: avoid model selection, rather, what matters in practice is the posterior predictive distribution. BFs are often very sensitive to prior (model specification).  
	
	\item Expected log-pointwise predictive density (ELPD): defined as 
	\begin{equation}
	\text{Elpd} = \E[\log \int p(x^*|\theta) p(\theta|x) d\theta]
	\end{equation} 
	where $x$ is the given data and $x^*$ is what is to be predicted. The expectation is over $x^*$. Remarks: 
	\begin{itemize}
		\item Averaging over many data points, so it should be normally distributed. 
		\item It is similar to prediction error (generalized) in Machine Learning. 
	\end{itemize}
	
	\item Using posterior predictive density for model comparison: this could be problematic. Suppose we compare two models $M_1$ and $M_2$ with very different priors for $\theta$. Now the parameter $\theta$ in our data comes from $M_2$, say, and we have very large data. Then the posterior under $M_1$ and $M_2$ are very similar, and thus lose the ability to distinguish the two models. 
	\begin{itemize}
		\item Remark: perhaps we can use how fast the posterior converges, as a measure of how good a prior model is? Similar to PAC learning. 
	\end{itemize} 
\end{itemize}

Model selection: 
\begin{itemize}
	\item Model evidence: a model is assessed by its posterior probability: 
	\begin{equation}
	P(M|D) = P(D|M) P(M) / P(D)
	\end{equation}
	where $P(D|M)$ is the model evidence. It is given by: 
	\begin{equation}
	P(D|M) = \int P(D|w,M) P(w|M) dw
	\end{equation}
	where $w$ is the model parameters. 
	
	\item Bayes factor: the comparison of two models is determined by the ratio: 
	\begin{equation}
	\frac{P(M_1|D)}{P(M_2|D)} = \frac{P(D|M_1)}{P(D|M_2)}	\cdot \frac{P(M_1)}{P(M_2)}
	\end{equation}
	In practice, we often use log-BF as the evidence, and the posterior odds ratio is given by: 
	\begin{equation}
	\text{log-posterior odds} = \text{log-BF} + \text{log-prior odds}	
	\end{equation}
\end{itemize}

Interpretation of model evidence: 
\begin{itemize}
	\item Parameter constraint: for a model $M$, assumming the posterior distribution $P(D|w)P(w)$ is approximated by its peak at $w_{MAP}$, with width $\Delta w_{\text{posterior}}$ and the prior $P(w|M)$ is flat with width $\Delta w_{\text{prior}}$, so that $P(w) = 1/\Delta w_{\text{prior}}$, we have: 
	\begin{equation}
	\ln P(D|M) \approx \ln P(D|w_{MAP}) + \ln \frac{\Delta w_{\text{posterior}}}{\Delta w_{\text{prior}}}	
	\end{equation}
	Thus for complex models: the prior distribution is more diffused, thus $\Delta w_{\text{prior}}$ is larger; or $\Delta w_{\text{posterior}}$ is smaller. Therefore the second term penalizes the complex models. If there are multiple parameters, we would also have more penalization terms. 
	
	\item Data generation: the model evidence $P(D|M)$ is the probability of generating a specific dataset $D$ from $M$. While simpler models can only generate datasets that are fairly similar to each other, complex models can generate a great variety of different datasets. Thus the simpler model cannot fit the data well, whereas the more complex model spreads its predictive probability over too broad arange of data sets and so assigns 
	relatively small probability to any one of them. 
	
\end{itemize}

The Laplace approximation [MacKay, Chapter 27]: 
\begin{itemize}
	\item Problem: suppose we want to compute the integral: 
	\begin{equation}
	Z = \int f(z) dz	
	\end{equation}
	where $z$ is $K$-dimensional variable and $f(z)$ is the (unnormalized) density function. 
	
	\item The log-density function can be approximated with normal distribution, or equivalently, we apply the Taylor expansion of $\ln f(z)$ at its maximum $z_0$: 
	\begin{equation}
	\ln f(z) \approx \ln f(z_0) - \frac{1}{2}	(z - z_0)^T A (z - z_0)
	\end{equation}
	where 
	\begin{equation}
	A_{ij} = -\frac{\partial^2}{\partial z_i \partial z_j} \ln f(z) \bigg \vert_{z=z_0}
	\end{equation}
	Plug in the approximation of $\ln f(z)$ to the integral, and using Gaussian integral, we have: 
	\begin{equation}
	Z \approx f(z_0) \sqrt{\frac{(2 \pi)^K}{\det A}}	
	\end{equation}
\end{itemize}

Approximating model evidence by Laplace approximation: 
\begin{itemize}
	\item Apply Laplace approximation to $P(D|M)$ (dropping $M$):  
	\begin{equation}
	\ln P(D) = \ln P(D|\theta_{\text{MAP}}) + \ln P(\theta_{\text{MAP}}) + \frac{K}{2} \ln(2 \pi) - \frac{1}{2} \ln \det A
	\label{eq:model_evidence}
	\end{equation}
	where $A$ is the Hessian matrix of second derivatives of the negative log posterior: 
	\begin{equation}
	A = - \nabla \nabla \ln P(D|\theta_{\text{MAP}}) P(\theta_{\text{MAP}})
	\end{equation}
	
	\item Bayesian information criterion (BIC): If we assume that the Gaussian prior distribution over parameters is broad, and that the Hessian has full rank, then we can approximate very roughly using Equation~\ref{eq:model_evidence}: 
	\begin{equation}
	\ln P(D) = \ln P(D|\theta_{\text{MAP}}) - \frac{1}{2} K \ln N
	\end{equation}
	where $K$ is the number of parameters and $N$ is the number of data points (the additive constants are omitted). 
\end{itemize}

Bayesian model comparison [BDA ed3, Chapter 7]
\begin{itemize}
	\item Example: linear regression with noninformative prior. $P(\sigma^2 | y)$ follows inverse-chi square and $P(a, b|\sigma^2, y)$ follows normal.
	
	\item Measure of predictive accuracy: \textbf{log predictive density}, $\log p(y|\theta)$. Connection with KL divergence. However, we cannot use it directly to assess a model in future observations. The out-of-sample predictive accuracy for a single observation $\hat{y}_i$ can be measured by \textbf{out-of-sample log predictive density}:
	\begin{equation}
	\log p_{\text{post}}(\hat{y}_i) = \log \int p(\hat{y}_i|\theta) p(\theta|y) d\theta
	\end{equation}
	But since the future data is not given, we should average over them, and this leads to \textbf{expected out-of-sample log predictive density}, or ELPD: 
	\begin{equation}
	\text{ELPD} = \E_f(\log p_{\text{post}}(\hat{y}_i)) = \int \log p_{\text{post}}(\hat{y}_i) f(\hat{y}_i) d\hat{y}_i
	\end{equation}
	
	\item Evaluating a model fit in practice: we generally cannot compute expectation of LPD, since we do not know $f$ in general. So we use the \textbf{log pointwise predictive density} (LPPD):
	\begin{equation}
	\text{llpd} = \sum_{i=1}^n \log p_{\text{post}}(y_i) = \sum_{i=1}^n \log \int p(y_i|\theta) p(\theta|y) d\theta 
	\end{equation}
	The posterior integration can be achieved by sampling: let $\theta^s$ be the $s$-th draw from $\theta|y$, we have: 
	\begin{equation}
	\text{llpd} \approx \sum_{i=1}^n \log \left( \frac{1}{S} \sum_{s=1}^S p(y_i|\theta^s) \right)  
	\end{equation}
	To estimate predictive accuracy in future data, we have to adjust by: (1) Adjust for the model complexity using within-sample predictive accuracy. (2) Cross-validation.
	
	\item Background: if $x \sim N(\mu, \Sigma)$ is $n$-dim random vector, then $(x - \mu)^T \Sigma^{-1} (x - \mu)$ follows $\chi^2_n$.
	
	\item Background: Deviance measures the model fit. It is -2 times the likelihood ratio of two nested models. Based on LRT, it follows $\chi^2$ distribution with dof. $k$, where $k$ is the difference of number of parameters. Note that the deviance has the scale of $k$, as the expectation of $\chi^2_k$ is $k$.
	
	\item AIC: log predictive density evaluated at the MLE. To account for model complexity, subtract number of model parameters $k$ to penalize complex models. 
	\begin{equation}
	\hat{\text{elpd}}_{\text{AIC}} = \log p(y|\hat{\theta}_{\text{MLE}}) - k
	\end{equation}
	The intuition is that the complex model would have higher log-likelihood, with the difference of LL follows $\chi^2_k$. 
	\begin{itemize}
		\item The asymptotic distribution of log predictive density $\log p(y|\theta)$: note that $y$ here refers to in-sample data (technically likelihood not predictive density). Here $\theta$ is a random variable follows posterior distribution (normal), and the predictive density is a function of $\theta$. Using Taylor expansion near $\theta_0$ (the posterior mean of $\theta$), one can show that the log predictive density follows $\chi^2_k$ distribution with $k$ being the dim of $\theta$ (in Taylor expansion, we have the second derivative of log predictive density, which is the covariance matrix of the posterior of $\theta$).
	\end{itemize}

	\item Deviance information criteria (DIC): when $\theta$ has informative prior distribution, the effect nubmer of parameters is not $k$, so AIC is not appropriate. Similar to AIC, but evaluate $\log p(y|\theta)$ at the mean posterior of $\theta$, and penalize with the effective number of parameters:
	\begin{equation}
	\hat{\text{elpd}}_{\text{DIC}} = \log p(y|\hat{\theta}_{\text{MP}}) - p_{\text{DIC}}
	\end{equation}
	where $p_{\text{DIC}}$ is the effective number of parameters, defined as: 
	\begin{equation}
	p_{\text{DIC}} = 2 \left[ \log p(y|\hat{\theta}_{\text{MP}}) - \E_{\theta|y} \log p(y|\theta) \right]
	\end{equation}
	The first term is the LPD at Bayesian point estimate, and the second is the LPD averaged over posterior of $\theta$. Clearly, the difference gets larger when we have a complex model, so the difference of the two gives the effective number of parameters.
	
	\item WAIC: similar to DIC, but slight variation in computing the effective number of parameters, averaging instead of using a single mean posterior. 
	
	\item LOO-CV: We first consider the general case: suppose we have training data $y$ and testing data $\hat{y}$, we can evaluate the model by its LPD at testing data: 
	\begin{equation}
	\log p_{\text{post}}(\hat{y}) = \sum_i \log \int p(\hat{y}_i|\theta) p(\theta|y) d\theta
	\end{equation}
	In the leave-one-out cross-validation setting, we treat each data point as a testing data, and all the other data as training data, this leads to the lppd. with LOO-CV: 
	\begin{equation}
	\text{lppd}_{\text{loo-cv}} = \sum_i \log p_{\text{post}(-i)}(y_i) = \sum_i \log \int p(y_i|\theta) p(\theta|y_{-i}) d\theta
	\end{equation}	
	To calculate this, we make inference on each $y_{-i}$, summarized as posterior draws $\theta^{is}$ for the $s$-th draw of $i$-th partition. LOO-CV can be compared with other ICs: the difference is the penalty (effective number of parameters).
	
	\item Estimation of LOO-CV ELPD: [Vehtari, Practical Bayesian model evaluation using leave-one-out cross-validation and WAIC, 2017]. To compute/sample $p(\theta|y_{-i})$ for each $i$ is expensive. One can importance sampling, however not stable. Using Pareto smoothed importance sampling. 
	
	\item Scenarios of model comparison: (1) Nest model: e.g. model expansion, see adding more parameters lead to better fit; (2) Non-nested models: e.g. compare two sets of predictors in regression.
	
	\item Evaluating predictive error comparison: if two models differ (in AIC, DIC, WAIC, or LOO-IC) by 5, say, does the difference matter? (1) Statistical significance: asymptotic theory say if order of 1, then due to chance. (2) Practical significance: e.g. AUC for prediction.
	
	\item Problems of all approaches: AIC, DIC not Bayesian. WAIC and LOO-IC: may not apply with hierarchical model. LOO-IC can be hard to compute.
	
	\item Remark: WAIC or LOO-IC, not incorporate prior, or not adjust for multiple testing. Could convert IC’s to p-values, then standard multiple testing correction.
\end{itemize}
	
\subsection{Bayesian Decision Theory}

Motivation:
\begin{itemize}
	\item Decision theory: an unified framework for inference and prediction problems. The goal of decision theory is to decide what is the best action to take under the uncertain circumstance. If we view the estimation or label prediction as an action to take, then all these problems can be formulated in a decision theory framework. 
	
	\item Bayesian decision theory: in Bayesian statistics, we are generally interested in the posterior distribution. But if we need to take an action, e.g. an estimator or selecting a model, decision theory may provide the theoretical background. 
	
	\item Inference problem: the action to take is the value of an unknown parameter (estimator), or the label of a new instance.  
	
	\item Learning problem: the action is defined on a set of putative instances, thus a procedure/function. 
\end{itemize}

Inference problem: parameter estimation and prediction on response variables/labels: 
\begin{itemize}
	\item Loss function: suppose we are predicting some variable, whose true value is $y$, and our action is $a$, the loss function is written as, $L(y,a)$. For example, we may have 0-1 loss if $y$ is binary; or $L_2$ loss for continuous $y$: 
	\begin{equation}
	L(y,a) = (y - a)^2
	\end{equation}
	
	\item Posterior expected loss: given an instance $x$, and we want to predict $y$, the optimal action should minimize the expected loss:
	\begin{equation}
	\delta(x) = \text{argmi}n_a \E[L(y,a)]	
	\end{equation}
	In the Bayesian approach, the expected loss is averaged over the posterior distribution $p(y|x)$. Thus the expected loss when the action is $a$ is defined by: 
	\begin{equation}
	\rho(a|x) = \E[L(y,a)] = \int L(y,a) p(y|x)	
	\end{equation}
	The Bayes estimator is thus given by: 
	\begin{equation}
	\delta(x) = \text{argmi}n_a \rho(a|x)	
	\end{equation}
	
	\item Binary classification and MAP estimator: when $y$ is binary, and we take 0-1 loss (symmetric), we have the rule: 
	\begin{equation}
	y^*(x) = \text{argmax} p(y|x)
	\end{equation}
	Thus the optimal $y$ is the one that maximizes the posterior. 
	
	\item Predicting continuous variables and posterior mean: the posterior expected loss is: 
	\begin{equation}
	\rho(a|x) = \E[(y-a)^2|x] = \E(y^2|x) - 2 a \E(y|x) + a^2	
	\end{equation}
	Minimize this as a function of $a$, we have: 
	\begin{equation}
	y^*(x) = \E(y|x) = \int y p(y|x) dy	
	\end{equation}
	Thus the optimal estimator/predictor is the posterior mean. 
	
\end{itemize}

Supervised learning problem: 
\begin{itemize}
	\item Goal: when we are solving a learning problem, we are not just minizming the loss over any single instance $x$, instead we will need to minimize the loss over a distribution of $x$. Furthermore, we will need to explicitly represent the unknown parameter $\theta$. 
	
	\item Generalization error: suppose our action is $\delta$ (a prediction procedure, as we would not to predict for any value of $x$), the loss of $\delta$ when the true parameters are $\theta$ is called the generalization error:  
	\begin{equation}
	L(\theta,\delta) = \E_{(x,y) \sim p(x,y|\theta)}[L(y,\delta(x))] = \int \int L(y,\delta(x)) p(x,y|\theta) dx dy
	\end{equation}
	
	\item Posterior expected loss: since $\theta$ is unknown, we need to take the expectation over the posterior distribution of $\theta$: 
	\begin{equation}
	\rho(\delta|D) = \int p(\theta|D) L(\theta,\delta) d \theta	
	\end{equation}
\end{itemize}
%%%%%%%%%%%%%%%%%%%%%%%%%%%%%%%%%%%%%%%%%%%%%%%%%%%%%%%%%%%%
\section{Bayesian Modeling in Practice}

Using Bayesian framework to borrow information [personal notes]
\begin{itemize}
	\item Motivation: a key advantage of Bayesian inference is the incorporation of prior, which allows one to use information elsewhere (prior). This is important, for example, when the data is sparse relative to the parameters to be estimated. This could be some knowledge one has before analyzing data, but also the information from other parts (e.g. other data samples) of the same dataset. Some general strategies how this could be done.  
	
	\item Hierachical model: this is the most common strategy of borrowing information from other data points. The idea is that some parameters (objects of the same group) share a common prior distribution.  
	
	\item Similar parameters in similar objects: discrete version. Hierarhical model requires some discrete grouping and the assumption of a common prior, this may be too stringent. A general idea is that the parameters of similar objects should be similar, but not necessarily of common distribution, and the similarity can be defined across a contiuous spectrum (thus less similar objects would have less similar parameter). Ideas capturing this intuition: 
	\begin{itemize}
		\item MVN: model the covariance of parameters, which is coupled to the similarity of objects. Ex. G-prior for regression. 
		\item Ising model: encourge similar $\beta$ for connected objects. 
	\end{itemize}
	
	\item Spatial model: suppose we have consecutive $\beta_1, \cdots, \beta_n$, we could imagine a stochastic process (e.g. HMM, random walk) that relates these parameters, e.g. $\beta_t \sim N(\beta_{t-1}, \sigma^2)$. 
\end{itemize}

Choosing a good prior: 
\begin{itemize}
	\item Importance of prior: often very important for the final results. Ex. for testing associations (using BF) in genetics: the prior of effect size has a large impact on the final BF. 
	
	\item Examine the prior distribution and see if it is consistent with domain knowledge. This often means examine the mean, variance, the probability of very rare events (tail distribution), and so. For example, the prior of the relative risk of a genetic variant, one could consider several criteria, including: the mean effect, the fraction of risk vs. protective variants, the percent of variants with very large effects.  
	
	\item Model selection vs. parameter estimation: in Bayesian inference, there is no clear-cut between the two. Model selection may be understood as a particular kind of prior: the mixture prior, i.e. the prior has multiple components. For example, in genetic association analysis, the effect size is generally close to 0 (for non-causal genes/variants), but could be large for causal genes/variants (e.g. LoF). In this case, it may be difficult to fit a single parametric distribution for prior, and a mixture prior is more appropriate (e.g. 0 and a normal prior). 
\end{itemize}

Modeling prior information: two general strategies of utilizing prior information [Personal notes]
\begin{itemize}
	\item Using prior information to set the prior distributions: effectively, our inference is conditioned on the prior data, which appears in the distribution $P(\theta|\phi, R)$ where $\phi$ is the hyperparameters and $R$ represents the prior data. The parameters of the prior, $\phi$, can be determined by empirical Bayes or posterior inference. 
	
	\item Modeling prior information as additional data: there may be advantages of modeling the prior information $R$ as additional data. The prior data $R$ may contain information of some hyperparameters $\phi$, and one can use independent data to bette estimate $\phi$. In other words, we model:
	\begin{equation}
	P(\theta, R|\phi) = P(R|\phi) P(\theta|\phi, R)
	\end{equation}
	where the distribution $P(R|\phi)$ encodes information of $\phi$, and may be estimated from independent data. Example, in rare variant association analysis, $V_j$, the variant information can be treated as data, and $P(V_j|Z_j=1)$ can be estimated from independent data. 
\end{itemize}

Sensitivity analysis: 
\begin{itemize}
	\item Goal: how the results depend on prior parameters. For model selection/hypothesis testing problems, we calculate how the BF of a test (or BF distribution of multiple tests in the genome-wide seting) depends on the prior parameters. 
	
	\item Example: [Bayesian statistical methods for genetic association studies, NRG, 2009] Sensitivity tends to be greatest in situations with less information, such as small studies, or low MAF. For a SNP with $p = 4.1e-9$, The $\log_{10}(BF)$ depends on $\sigma$ (prior standard deviation of effect size): e.g. when $\sigma = 0.2$ (low), it is only 2.2, and when $\sigma = 0.8$, it is 5.2.
	
	\item Example: [A Bayesian Measure of the Probability of False Discovery in Genetic Epidemiology Studies, Wakefield, AJHG, 2007]. In a lung cancer study of 131 SNPs,  the number of predictions change dramatically with $\pi_0$ (about 20-30 positive discoveries vs. 2-3 discoveries). However, the predictions (number, and ranking of SNPs) vary little with the other parameters (Figure 8). 
\end{itemize}

Bayesian diagnistic/goodness of fit: 
\begin{itemize}
	\item Idea: suppose our data is $x$, we could compute the marginal distribution using the estimated model, and compare that with the observed distribution (histogram) of the data. 
	
	\item Example: [Detecting differential gene expression with asemiparametric hierarchical mixture method, Newton et al, Biostatistics, 2004] Figure 3: QQ plot of the marginal distribution of the model and the empirical distribution. Figure 5: histogram of the fitted and observed expression measurements. 
	
\end{itemize}

Performance evaluation: 
\begin{itemize}
	\item Individual test: type I error and power. 
	
	\item Multiple tests: under a target FDR, run the Bayesian methods on simulated data, and estimate the realized FDR and power (number of true discoveries). The results could be represented in a ROC curve. 
\end{itemize}

Examples of Bayesian inference in genetics: 
\begin{itemize}
	\item A hidden Markov random field model for genome-wide association studies (PMID:19822692). 
	\begin{itemize}
		\item Data: NB is a common and lethal pediatric malignancy. GWAS of 1000 cases and 2000 controls, and anlaysis on 31K SNPs in Chr. 6. Single SNP analysis identified three SNPs. 
		\item Using PPA from HMRF model identified two additional SNPs in LD with these SNPs (PPA close to 1). In addition, one SNP in chr. 6 has PPA 0.74. Including all these 6 SNPs give FDR 0.046. 
		\item Analysis on 2 predicted regions: 100 permutations, only in 1 simulation, find a SNP with PPA greater than 0.5. 
	\end{itemize}
	
	\item Detecting differential gene expression with asemiparametric hierarchical mixture method, Newton et al, Biostatistics, 2004: 
	\begin{itemize}
		\item Data: $n$ genes, with expression in the first set of conditions $x_{g,i}$, and expression in the second set of conditions $y_{g,j}$. The goal is to compare if the means of the two conditions are the same.  
		\item Model: (1) Individual gene: expression depends on the mean, modeled as Gamma distribution (constant CV, so not normal distribution). (2) Hierarchical model: mixture of three cases: equal mean in two conditions; one of the condition has higher mean. 
		\item Simulation: three secenarios (corresponding to different $\pi$'s), each method (Bayesian and $t$-test plus FDR control) targets FDR at 0.05, and compare the performance: the sensitivity and realized FDR. 
	\end{itemize}
	
	\item Reporting and interpretation in genome-wide association studies [Wakefield, Int. J Epiderm, 2008]. 
	\begin{itemize}
		\item Approximate BF (ABF): in terms of the estimate of the effect size, its confidence interval, and the standard deviation of the prior effect size.
		\item Dependence of BF on MAF: at small MAF, the same small $p$-value has lower BF because to achieve such $p$-value with low MAF, the effect size must be big, which is quite unlikely under the prior. 
	\end{itemize}
\end{itemize}
%%%%%%%%%%%%%%%%%%%%%%%%%%%%%%%%%%%%%%%%%%%%%%%%%%%%%%%%%%%%
\section{Bayesian Inference of Common Probability Distributions}

Reference: \cite{Gelman03}

Overview: for Bayesian inference of common probability distributions, we are interested in: 
\begin{itemize}
\item Posterior distribution of parameters: this is often the goal of Bayesian inference. 
\item Posterior predictive distribution: important for the purpose of making predictions (e.g. in classification or regression). 
\item Marginal likelihood: the dependence between data and the hyperparameter(s). This is important in hierarchical models, where we are often interested in the higher-level parameters. Also important in model selection problems. This can be obtained in two ways: (1) by integrating out the parameter: 
\begin{equation}
p(y) = \int p(y|\theta) p(\theta) d\theta
\end{equation}
Or (2) by using the posterior distribution: 
\begin{equation}
p(y) = \frac{p(y|\theta) p(\theta)}{p(\theta|y)}	
\end{equation}
The latter may be easier when the posterior has already been solved. 
\end{itemize}

Bernoulli and binomial distribution: 
\begin{itemize}
\item Model: our data follows the distribution $y \sim \text{Bin}(n;\theta)$, and the prior distribution $\theta \sim \text{Beta}(\alpha,\beta)$. 

\item Posterior: this is given by: 
\begin{equation}
\theta |y \sim \text{Beta}(\alpha + y,\beta + n - y)	
\end{equation}

\item Posterior predictive distribution: suppose we want to predict a new data point $\hat{y}$ (single observation), which is 0 or 1, thus Bernoulli distribution. The mean of the distribution is: 
\begin{equation}
\E(\hat{y}|y) = \int \E(\hat{y}|\theta) p(\theta|y) d\theta = \int \theta p(\theta|y) d\theta = \E(\theta|y) = \frac{\alpha + y}{\alpha + \beta +n}
\end{equation}
Thus $\hat{y} | y \sim \text{Ber}(\hat{\theta})$, where $\hat{\theta}$ is the posterior mean of $\theta$. 

\item Marginal likelihood: this is obtained by integrating $\theta$ in the likelihood function: 
\begin{equation}
p(y|\alpha,\beta) = \int p(y|\theta) p(\theta|\alpha,\beta) d\theta = \frac{\binom{n}{y}}{B(\alpha, \beta)} \int \theta^y (1-\theta)^{n-y} \theta^{\alpha-1} (1-\theta)^{\beta-1} d\theta
\end{equation}
Apply the definition of Beta function: 
\begin{equation}
p(y|\alpha,\beta) = \binom{n}{y} \frac{B(\alpha+y, \beta+n-y)}{B(\alpha, \beta)}	
\end{equation}
\end{itemize}

Poisson distribution: 
\begin{itemize}
\item Basic model: suppose we have $y_i$ i.i.d. with $y_i \sim \text{Pois}(\theta), 1 \leq i \leq n$, and the prior distribution $\theta \sim \text{Gamma}(\alpha,\beta)$. The posterior distribution is: 
\begin{equation}
\theta | y \sim \text{Gamma}\left(\alpha + \sum_i y_i, \beta + n\right)	
\end{equation}
Thus the prior can be viewed as $\alpha$ events in $\beta$ observations. The marginal likelihood is given by: 
\begin{equation}
p(y|\alpha,\beta) = \frac{\prod_i \text{Pois}(y_i|\theta) \cdot \text{Gamma}(\theta|\alpha, \beta)}{\text{Gamma}(\theta|\alpha+\sum_i y_i, \beta+n)}	
\end{equation}
Plug-in the relevant terms, we have: 
\begin{equation}
p(y|\alpha,\beta) = \frac{\Gamma(\alpha+\sum_i y_i)}{\Gamma(\alpha) \prod_i y_i!}	\frac{\beta^{\alpha}}{(\beta+n)^{\alpha + \sum_i y_i}}
\end{equation}

\item Model with rate and exposure: suppose for the $i$-th observation, the count depends on the exposure $x_i$, we have: 
\begin{equation}
y_i \sim \text{Pois}(x_i \theta)	
\end{equation}
And we have the same prior $\theta \sim \text{Gamma}(\alpha, \beta)$. The posterior distribution is given by: 
\begin{equation}
\theta | y \sim \text{Gamma}\left(\alpha + \sum_i y_i, \beta + \sum_i x_i\right)
\end{equation}
The marginal likelihood is: 
\begin{equation}
p(y|\alpha,\beta) = \prod_i \frac{x_i^{y_i}}{y_i!} \cdot \frac{\Gamma(\alpha+\sum_i y_i)}{\Gamma(\alpha)}	\frac{\beta^{\alpha}}{(\beta+\sum_i x_i)^{\alpha + \sum_i y_i}}	
\end{equation}

\item Relation with negative binomial distribution: when there is a single observation $y$ with exposure $x$, the marginal likelihood is:
\begin{equation}
p(y|\alpha,\beta) = \frac{\Gamma(\alpha+y)}{\Gamma(\alpha) y!} \cdot \left( \frac{\beta}{\beta+x}\right)^{\alpha} \left( \frac{x}{\beta+x}\right)^{y} = \text{NegBin}\left(y|\alpha, \frac{x}{\beta+x}\right)
\end{equation}
The expectation of $y$ is: 
\begin{equation}
\E(y|\alpha,\beta) = \frac{\alpha \cdot \frac{x}{\beta+x}}{1 - \frac{x}{\beta+x}}	= x \frac{\alpha}{\beta}
\end{equation}
which is the product of exposure and the average prior rate. Thus the negative binomial can be understood as a model of discrete distribution, similar to Poisson, but with variance possibly different from the rate parameter. 
\end{itemize}

Univariate normal distribution with known variance:  
\begin{itemize}
\item Likelihood function: given observations $y_1, \cdots, y_n$ iid. $N(\mu, \sigma^2)$. The likelihood is: 
\begin{equation}
p(y|\mu, \sigma^2) = \left( \frac{1}{\sqrt{2\pi} \sigma}\right)^n	\exp \left[ -\frac{1}{2 \sigma^2} \sum_{i=1}^n (y_i - \mu)^2\right]
\end{equation}

\item Conjugate prior: the likelihood function is the exponential of a quadratic function of $\mu$, thus choose the prior of the same form (normal distribution): 
\begin{equation}
\mu \sim N(\mu_0, \tau^2)	
\end{equation}

\item Posterior distribution: express the posterior distribution in the form of exponential of a quadratic, and we find that the posterior is also normal: 
\begin{equation}
\mu |y \sim N(\mu_n, \tau_n^2)	
\end{equation}
where: 
\begin{equation}
\mu_n = \left(\frac{1}{\tau_0^2} + \frac{n}{\sigma^2}	\right)^{-1} \left(\frac{1}{\tau_0^2} \mu_0 + \frac{n}{\sigma^2} \bar{y} \right) \qquad \frac{1}{\tau_n^2} = \frac{1}{\tau_0^2} + \frac{n}{\sigma^2} 
\end{equation}
Thus the posterior is a combination of prior and data. Its expectation is the weighted average of $\mu_0$ and $\bar{y}$, with weights equal to the precision (inverse of variance) of prior and data. Its precision is the sum of precision of prior and data. 

\item Posterior predictive distribution: is the average over $\theta | y$. It is normal distribution with:
\begin{equation}
\E(\tilde{y}|y) = \E_{\theta|y}[\E(\tilde{y}|\theta,y)] = \E(\theta|y) = \mu_n
\end{equation}
\begin{equation}
\Var(\tilde{y}|y) = \E_{\theta|y}[\Var(\tilde{y}|\theta,y)]	+ \Var_{\theta|y}[\E(\tilde{y}|\theta,y)] = \E(\sigma^2|y) + \Var(\theta|y) = \sigma^2 + \tau_n^2
\end{equation}
Thus the variance of $\tilde{y}|y$ has two components: one from the variance of $\theta|y$, and the other from the inherent error ($\sigma^2$). 
\end{itemize}

Univariate normal distribution with known mean but unknown variance:
\begin{itemize}
\item Conjugate prior: from the likelihood function, the conjugate prior should have the form: 
\begin{equation}
p(\sigma^2) \propto (\sigma^2)^{- (\alpha + 1)} e^{-\beta/\sigma^2}
\end{equation}
We thus choose the scaled inverse $\chi^2$ distribution: $\sigma^2 \sim \text{Inv-}\chi^2(\nu_0, \sigma_0^2)$ where $\nu_0$ is dof. and $\sigma_0^2$ is (roughly) the mean. The density function: 
\begin{equation}
p(\sigma^2) \propto (\sigma^2)^{- (\nu_0/2 + 1)} \exp\left(-\frac{\nu_0 \sigma_0^2}{2\sigma^2}\right)
\end{equation}

\item Posterior: the mean is $\theta$, the sample variance is: 
\begin{equation}
v = \frac{1}{n} \sum_{i=1}^n (y_i - \theta)^2	
\end{equation}
The posterior of $\sigma^2$ is given by: 
\begin{equation}
\sigma^2 | y \sim \text{Inv-}\chi^2\left(\nu_0 + n, \frac{\nu_0 \sigma_0^2 + nv}{\nu_0 + n}\right)	
\end{equation}
Thus the posterior of $\sigma^2$ has dof. equal to $\nu_0 + n$ (larger $n$, sharper distribution), and the scale parameter is the weighted average of prior mean (roughly) and sample variance, where weights are given by the dof. 

\item Alternative proof using distribution of sample variance: consider a problem $y_i$ iid $N(0, \sigma^2)$ and we want to infer $\sigma$. Let $S^2 = \frac{1}{n} \sum_i y_i^2$ be the sample variance. It is easy to show that $S^2$ is a sufficient statistic. We have this result from classical statistics: 
\begin{equation}
\frac{(n)S^2}{\sigma^2} \sim \chi^2_{n}
\end{equation} 
Note that we have $n$ instead of $n-1$ here because mean is given. From this we have (using inverse of $\chi^2$ distribution): 
\begin{equation}
\frac{\sigma^2}{nS^2} \sim \text{Inv}-\chi^2_{n} \Rightarrow \sigma^2|S^2 \sim \text{Scaled-Inv}-\chi^2(n, S^2)
\end{equation}
Thus the posterior of $\sigma^2$ is inverse chi-square with dof $n$ (large number of samples, sharper peak), and the scale determined by $S^2$. 

\end{itemize}
  
Univariate normal distribution with unknown mean and variance: conjugate prior  
\begin{itemize}
\item Conjugate prior: 
\begin{equation}
\begin{array}{lll}
\sigma^2 & \sim & \text{Inv-}\chi^2(\nu_0, \sigma_0^2)	\\
\mu | \sigma^2 & \sim & N(\mu_0, \sigma^2 / \kappa_0)
\end{array}
\end{equation}
Thus $\nu_0$ is the dof. of the $\sigma^2$ (higher dof., more accurate), $\sigma_0$ is the scale of $\sigma^2$; $\mu_0$ is the location of $\mu$, and $\kappa_0$ is the number of measurements. 
\begin{itemize}
	\item Remark: To see why we want prior of $\mu$ to depend on $\sigma$, we note that if $\mu$ has a different variance, then in the posterior, the exponential terms for the prior and the likelihood cannot be combined.
\end{itemize}

\item Posterior distribution: 
\begin{equation}
p(\mu,\sigma^2|y) \propto (\sigma^2)^{-(\frac{\nu_0 + n}{2} + \frac{3}{2})} \exp\left( -\frac{1}{2\sigma^2} [\nu_0 \sigma_0^2 + \kappa_0(\mu - \mu_0)^2]\right) \exp\left(-\frac{1}{2\sigma^2} [(n-1) s^2 + n(\bar{y} - \mu)^2] \right)
\end{equation}
where $s^2$ is the sample variance: 
\begin{equation}
s^2 = \frac{1}{n-1}	\sum_i (y_i - \bar{y})^2
\end{equation}

\item Conditional posterior distribution $p(\mu|\sigma^2,y)$: this is similar to the case of known variance, we have $\mu|\sigma^2, y \sim N(\mu_n, \sigma^2/\kappa_n)$, where 
\begin{equation}
\mu_n = \frac{\kappa_0}{\kappa_0 + n} \mu_0 + \frac{n}{\kappa_0 + n} \bar{y}	\qquad \kappa_n = \kappa_0 + n
\end{equation}

\item The marginal posterior distribution $p(\sigma^2 | y)$: we integrate out $\mu$ in the joint posterior density:
\begin{equation}
p(\sigma^2|y) \propto (\sigma^2)^{-(\frac{\nu_0 + n}{2} + \frac{3}{2})} \exp\left( -\frac{1}{2\sigma^2} [\nu_0 \sigma_0^2 + (n-1) s^2]\right)	\cdot I
\end{equation}
where 
\begin{equation}
I = \int_{-\infty}^{\infty} \exp\left(-\frac{1}{2\sigma^2} [\kappa_0(\mu - \mu_0)^2 + n(\bar{y} - \mu)^2] \right) d\mu
\end{equation}
We apply the complete-the-square trick: 
\begin{equation}
\kappa_0(\mu - \mu_0)^2 + n(\bar{y} - \mu)^2 = (\kappa_0 + n)\left( \mu - \frac{\kappa_0 \mu_0 + n\bar{y}}{\kappa_0 +n}\right)^2 + \frac{\kappa_0 n}{\kappa_0 + n} (\bar{y} - \mu_0)^2
\end{equation}
Plug in this to the integral $I$: 
\begin{equation}
I = \sigma \sqrt{\frac{2 \pi}{\kappa_0 + n}}  \exp\left( -\frac{1}{2\sigma^2} \frac{\kappa_0 n}{\kappa_0 + n} (\bar{y} - \mu_0)^2\right)	
\end{equation}
The posterior distribution $\sigma^2|y$ thus follows scaled inverse-$\chi^2$ distribution: $\text{Inv-}\chi^2(\nu_n, \sigma_n^2)$, where 
\begin{equation}
\nu_n = \nu_0 + n \qquad \nu_n \sigma_n^2 = \nu_0 \sigma_0^2 + (n-1)s^2 + \frac{\kappa_0 n }{\kappa_0 + n}(\bar{y} - \mu_0)^2	
\end{equation}

\item The marginal posterior distribution $p(\mu | y)$: by integrating out $\sigma^2$ in the joint posterior density, we have:
\begin{equation}
\mu | y \sim t_{\nu_n}(\mu_n, \sigma_n^2 / \kappa_n)	
\end{equation}

\item Sampling: first sample $\sigma^2$ from $p(\sigma^2|y)$, then sample $\mu$ from $p(\mu|\sigma^2,y)$. For posterior predictive distribution: after sampling $\mu$ and $\sigma^2$, sample $\tilde{y}$ from $p(\tilde{y}|\mu, \sigma^2)$. 

\item Noninformative prior: as a special case of the conjugate prior, we have: 
\begin{equation}
p(\mu, \sigma^2) \propto (\sigma^2)^{-1}
\end{equation}
Note that the prior is improper, i.e. the integrate is infinite; however, the posterior is proper. 

\item Alternative form of conjugate prior [Bishop]: it is sometimes easier to work with precision (inverse of covariance), $\tau = 1/\sigma^2$. The conjugate prior of $\tau$ is the Gamma distribution: 
\begin{equation}
p(\tau) = \text{Gamma}(\tau|a_0,b_0) = \frac{b_0^{a_0}}{\Gamma(a_0)} \tau^{a_0 - 1} e^{-b_0 \tau}
\end{equation}
The conjugate prior of the mean: 
\begin{equation}
p(\mu|\tau) = N(\mu|\mu_0, (\lambda_0 \tau)^{-1})	
\end{equation}
The posterior distribution of $\tau$ and $\mu$ follow the Gamma and normal distributions, respectively (see [Bishop, 2.3.6]). 

\item Alternative derivation: we use the properties of MVN, given the distribution of $\mu$ and $y|\mu$ (multivariate), we derived the distribution $\mu|y$ and $y$ ($\sigma^2$ is treated as constant): 
\begin{equation}
\mu | \sigma^2 \sim N(\mu_0, \sigma^2 / \kappa_0)	
\end{equation}
\begin{equation}
y | \mu, \sigma^2 \sim N(\mu \mathbf 1, \sigma^2 I)	
\end{equation}
where $\mathbf 1$ is the vector consisting of 1's. Then we obtain $\mu|y, \sigma^2$ as before, and:  
\begin{equation}
y | \sigma^2 \sim N(\mu_0 \mathbf 1, S_n)	
\end{equation}
where 
\begin{equation}
S_n = \sigma^2 \left(I + \frac{1}{\kappa_0} \mathbf 1 \cdot \mathbf 1^T\right)
\end{equation}

\end{itemize}

Univariate normal distribution with unknown mean and variance: semi-conjugate prior  
\begin{itemize}
\item Semi-conjugate prior: in some cases, we don't want the prior of $\mu$ to be dependent on the variance parameter. So we have the prior: 
\begin{equation}
\mu \sim N(\mu_0, \tau_0^2) \qquad \sigma^2 \sim \text{Inv-}\chi^2(\nu_0,\sigma_0^2)
\end{equation}

\item Conditional posterior distribution $p(\mu|\sigma^2,y)$: 
\begin{equation}
\mu|\sigma^2,y \sim N(\mu_n, \tau_n^2)	
\end{equation}
where
\begin{equation}
\mu_n = \frac{\frac{1}{\tau_0^2} \mu_0 + \frac{n}{\sigma^2} \bar{y}}{\frac{1}{\tau_0^2} + \frac{n}{\sigma^2}}	\qquad \frac{1}{\tau_n^2} = \frac{1}{\tau_0^2} + \frac{n}{\sigma^2}
\end{equation}

\item Posterior distribution $p(\sigma^2|y)$: since we already know $p(\mu|\sigma^2,y)$, we could solve it by: 
\begin{equation}
p(\sigma^2|y) = \frac{p(\mu,\sigma^2|y)}{p(\mu|\sigma^2,y)} \propto \frac{N(\mu|\mu_0, \tau_0^2) \text{Inv-}\chi^2(\sigma^2|\nu_0,\sigma_0^2) \prod_i N(y_i|\mu,\sigma^2)}{N(\mu|\mu_n, \tau_n^2)}	
\end{equation}
This is true for any value of $\mu$, so we choose $\mu = \mu_n$ s.t. the denominator is simple: $(\sqrt{2\pi} \tau_n)^{-1}$. So we have: 
\begin{equation}
p(\sigma^2|y) \propto \tau_n N(\mu_n|\mu_0, \tau_0^2) \text{Inv-}\chi^2(\sigma^2|\nu_0,\sigma_0^2) \prod_i N(y_i|\mu_n,\sigma^2)	
\end{equation}
Even though this does not have a simple conjugate form, this can be easily computed for any value of $\sigma^2$. 

\item Posterior sampling: first sample from $\sigma|y$ using the numerical method; and then sample $\mu|\sigma^2,y \sim N(\mu_n, \tau_n^2)$. 
\end{itemize}

Multivariate normal distribution with known variance:  
\begin{itemize}
\item Likelihood function: given $y_1, \cdots, y_n$ i.i.d. $N(\mu, \Sigma)$, 
\begin{equation}
p(y_1, \cdots, y_n|\mu, \Sigma) \propto |\Sigma|^{-n/2} \exp\left[ -\frac{1}{2} \tr(\Sigma^{-1} S)\right]	
\end{equation}
where $S$ is the matrix of sum of squares: 
\begin{equation}
S = \sum_{i=1}^n (y_i - \mu) (y_i - \mu)^T	
\end{equation}

\item Conjugate prior: $\mu \sim N(\mu_0, \Lambda_0)$. 

\item Posterior distribution: this is similar to the univariate case. $\mu | y \sim N(\mu_n, \Lambda_n)$, where 
\begin{equation}
\mu_n = (\Lambda_0^{-1} + n \Sigma^{-1})^{-1}	(\Lambda_0^{-1} \mu_0 + n \Sigma^{-1} \bar{y}) \qquad \Lambda_n^{-1} = \Lambda_0^{-1} + n \Sigma^{-1}
\end{equation}

\item Posterior predictive distribution: we have
\begin{equation}
\E(\tilde{y}|y) = \mu_n \qquad \Var(\tilde{y}|y) = \Sigma + \Lambda_n	
\end{equation}
\end{itemize}

Multivariate normal distribution with unknown variance:  
\begin{itemize}
\item Conjugate prior: similar to the univariate case, but replace inverse-$\chi^2$ with inverse Wishart distribution: 
\begin{equation}
\begin{array}{lll}
\Sigma & \sim & \text{Inv-Wishart}_{\nu_0}(\Lambda_0^{-1})	\\
\mu | \Sigma & \sim & N(\mu_0, \Sigma / \kappa_0)
\end{array}	
\end{equation}

\item Posterior distribution: similar to the univariate case, the conditional posterior distribution
\begin{equation}
\mu | \Sigma, y \sim N(\mu_n, \Sigma / \kappa_n)	
\end{equation}
where 
\begin{equation}
\mu_n = \frac{\kappa_0}{\kappa_0 + n} \mu_0 + \frac{n}{\kappa_0 + n} \bar{y}	\qquad \kappa_n = \kappa_0 + n
\end{equation}
And the marginal posterior of $\Sigma|y \sim \text{Inv-Wishart}_{\nu_n}(\Lambda_n^{-1})$, where
\begin{equation}
\nu_n = \nu_0 + n \qquad \nu_n \Lambda_n = \Lambda_0 + S + \frac{\kappa_0 n }{\kappa_0 + n}(\bar{y} - \mu_0) (\bar{y} - \mu_0)^T		
\end{equation}
The marginal posterior $\mu | y$ is multivariate $t$-distribution (see the book).  
\end{itemize}

%%%%%%%%%%%%%%%%%%%%%%%%%%%%%%%%%%%%%%%%%%%%%%%%%%%%%%%%%%%%
\section{Bayesian Computation: MCMC}

Lessons in Bayesian computation [personal notes]: 
\begin{itemize}
	\item Adaptive MCMC [Stephens]: in general, the adaptive MCMC is not theoretically correct: not Markov Chain (the distribution changes). So it is hard to prove convergence. In practice, often stop at a fixed step size after some point, then a standard MCMC. 
	
	\item Computing Bayes factors [Stephens, Gelman]: generally difficult with more than a few parameters. 
	
	\item How to study the convergence of MCMC: study the geometric rate of convergence [Stephens]. 
\end{itemize}

MCMC review [personal notes]
\begin{itemize}
	\item Initialization: warm start, but need to have multiple chains started at different places in the parameter space.
	
	\item Strategy: mixing MH and Gibbs.
	
	\item Visualization of posterior draws: could help monitor chain behavior.
	
	\item Testing convergence: multiple chains, and test both mixing and stationarity.
\end{itemize}

Sampling and expectation [MacKay, Chapter 29]: 
\begin{itemize}
	\item Two computational problems: sample from a distribution $P(x)$ and computing the expectation of some function under $P(x)$: 
	\begin{equation}
	\Phi = \langle \phi(x) \rangle = \int \phi(x) P(x) dx
	\end{equation}
	
	\item If we can solve the first problem, we can solve the second one, by sampling $x^{(r)}, 1 \leq r \leq R$, and compute the estimator
	\begin{equation}
	\hat{\Phi} = \frac{1}{R} \sum_r \phi(x^{(r)})
	\end{equation}
	It is easy to check that the estimator is unbiased, $\E(\hat{\Phi}) = \Phi$ and its variance: 
	\begin{equation}
	\Var(\hat{\Phi}) = \sigma^2 / R
	\end{equation}
	where $\sigma^2 = \Var(\phi(x))$. 
	
	\item \textbf{Remark}: we can use the frequentist approach to evaluate the estimator of the integral to be evaluated: its mean and variance (and how it changes with $R$). In particular, most estimators are unbiased, so we need to analyze the variance. 
	
	\item Why sampling is difficult? Often we can evaluate $P^*(x) = P(x)/Z$, but cannot evaluate $P(x)$ as $Z$ is unknown. Ex. we want to sample $p(\theta|y) = p(\theta) p(y|\theta) / p(y)$: we can evaluate the numerator, but do not know the denomiator. Even if we know the true $P(x)$, we may not be able to sample it: we do not know whether a $P(x)$ is large unless we evaluate $P(x)$ for all $x$. 
\end{itemize}

Importance sampling [MacKay, Chapter 29]
\begin{itemize}
	\item Suppose we want to evaluate $\Phi = \E(\phi(x))$, we assume that we can sample from a distribution $Q(x)$. The idea is that, if $P(x) < Q(x)$, we over-sampled $x$, and if $P(x) > Q(x)$, we under-sampled $x$, so we need to adjust for this.   
	Suppose we can evaluate $Q^*(x)$, where $Q(x) = Q^*(x) / Z_Q$. We then define: 
	\begin{equation}
	w_r = \frac{P^*(x^{(r)})}{Q^*(x^{(r)})}
	\end{equation} 
	Our estimator is: 
	\begin{equation}
	\hat{\Phi} = \frac{\sum_r w_r \phi(x^{(r)})}{\sum_r w_r}
	\end{equation}
	
	\item Proof: intuitively, the numerator is proportional to $\Phi$, and one can show that the constant is given by the denominator. 
	
	\item Behavior of importance sampling: in general, if $Q$ is substantially smaller than $P$ at the places where $P$ has prob. mass (typical set), then we will have very large $w_r$ in those points. This would lead to large variance of the estimator. So in general, the distribution $Q$ should be long-tailed. 
\end{itemize}

Practical issues in sampling posterior distribution (general issues): [GCSR, Section 3.7, 5.3, Chapter 10]
\begin{itemize}
\item Estimating the range of parameters: the first step is often to obtain a crude estimate of the parameters. 
\begin{itemize}
	\item Ex. for hierarchical models, one can obtain the rough estimate through complete pooling (population mean) and no-pooling (group mean). 
	\item Posterior mode approximation: normal distribution centered on the posterior mode can serve as an approximation of the posterior. 
\end{itemize}

\item Sampling strategy: if appropriate analytic forms exist, to sample $(\theta, \phi)$, first sample marginal posterior distribution, e.g. $p(\phi|y)$, then sample conditional posterior distribution $p(\theta|\phi,y)$. If no analytic forms exist, MCMC is the general strategy. 
\begin{itemize}
	\item Discretization: if involved (sampling in a lattice with grids), the range of grids can be roughly estimated via normal approximation [Section 4.1, page 103]. Also, since we can only sample at the resolution of grids, could add random jitter within each gride for sampled points. 
\end{itemize} 

\item Visualization and inspection: 1-D we could use histogram; for 2-D, use scatter plot or contour plot for density. 

\item Debugging via fake data: to test if the sampling procedure works properly, simulate the data using the some known parameter value, and apply the sampling procedure, and test if the correct value can be recovered (within posterior interval). 

\item Example: bioassay experiment [Section 3.7]. Two parameters $\alpha, \beta$, describe the dose-response relation of a drug. The posterior distribution $p(\alpha, \beta|x)$ can be calculated for given value of $\alpha, \beta$ (but no analytic form). Perform simulation with 2D contour plot. 
\end{itemize}

MCMC: [GCSR, chapter 11]
\begin{itemize}
\item Idea of MCMC: to sample from a distribution $P(x)$, we design a Markov chain whose equilibrium distribution is $P(x)$. This is done through implementing the detailed balance. 

\item Metropolis algorithm: suppose we want to sample from $P(x)$, the Metropolis algorithm proposes a move from $x^t$ to $x'$ using the jumping distribution $Q(x'|x^t)$. Note that $Q$ is symmetric in Metropolis. $x'$ is accepted with probability:
\begin{equation}
r = \frac{P(x')}{P(x^t)}	
\end{equation}
If accepted, we have $x^{t+1} = x'$. Intuitively, this is similar to stochastic hill climbing for optimization: if $P(x')$ increases, then we should accept it; otherwise accept with only a probability. 

\item M-H algorithm: relax the assumption that $Q(\cdot)$ must be symmetric: if not symmetric, we accept with probability:
\begin{equation}
r = \frac{P(x') Q(x^t|x')}{P(x^t) Q(x'|x^t)}	 	
\end{equation}
This is important: e.g. to have Gibbs sampler as a special case. The most common proposal function $Q$ is the normal jumping kernel, $X'|X \sim N(X, \Sigma)$. 

\item Gibbs sampler as a special case of M-H algorithm: in each cycle, Gibbs sampler performs $d$ steps ($d$ variables). We only need to show that the true distribution $P$ is an equilibrium distribution of the Markov chain. To see that: 
\begin{itemize}
\item First, the chain is irreducible, i.e. one can move from any state to any other state of the chain (accessible). So there is a unique equilibrium distribution. 
\item Suppose the chain is at the distribution $P$, it is easy to see that in each step, the distribution at $t+1$, $P^{t+1} = P^{t} = P$, since detailed balance is satisfied at each of the $d$ steps. Therefore, $P$ is an equlibrium distribution. 
\end{itemize}
Remark: for a MC, as long as each step the detailed balance is satisfied, the distribution will converge to the target distribution. 

\item Combining Gibbs sampler and M-H algorithm: in a practical problem, some conditional distributions are conjugate (easy to sample) while the others maybe not, so mixing MCMC and Gibbs may be necessary. The convergence to the target distribution follows from the fact that each step satisfies the detailed balance (thus not changing the equilibrium distribution). 
\begin{itemize}
\item M-H embedded in a Gibbs sample structure: to sample $(X,Y)$, we repeatedly sample $X|Y$ and $Y|X$, each with M-H algorithm. 
\item Mixing M-H and Gibbs sampling: to sample $(X,Y)$, we repeatedly (1) sample $X|Y$ using Gibbs, then (2) sample from $Y|X$ with M-H updating. 
\item Block-level MCMC: mix M-H and Gibbs at the level of blocks of variables/parameters. 
\end{itemize}

\end{itemize}

Practical issues of MCMC: [GCSR, Chapter 11]
\begin{itemize}
\item Unnormalized probability: needs to checked. In most cases, working with unormalized density would not affect the sampling algorithm. 

\item Inference from iterative simulation: challenges include (1) the simulations may be unrepresentative of the target distribution, if the chains have not converged. (2) Within sequence correlation reduces the number of effective draws. 

\item Discard early runs: called warm-up. Generally discard half. Ex. we run 200 iterations, if not converge, then run another 200, and discard the original 200. 

\item Thinning: we could improve the efficiency of simulation runs by use every $k$-th simulation draw. 

\item Assessing convergence: general strategies
\begin{itemize}
	\item Run with different sequences/chains with starting points dispersed throughout parameter space. If converged, the different sequences should have the same distributions. See example of drawing from 2D-MVN (BDA V3: Figure 11.1). 
	
	\item Monitor convergence by some scalar estimands: parameters or some functions on parameters. Often good to check log-posterior density. It is better to transform the estimands s.t. they are normally distributed.
	
	\item Importance of both mixing and stationarity (BDA V3: Figure 11.3): (1) two chains each stationary, but not mixing; (2) two chains mix well, but neither reach stationary distribution. 
\end{itemize}

\item Multi-chain strategy for testing convergence: after burn-in period, split each chain in half and check that all the resulting half sequences have mixed. This checks both mixing and stationarity (comparing first and second half). A test statistic for convergence (Rhat): let $W$ be within chain variance and $B$ between chain variance of the scalar estimand. We compute the variance of posterior samples of the estimand $\psi$ as:
\begin{equation}
\hat{\Var}(\psi | y) = \frac{n-1}{n} W + \frac{1}{n} B
\end{equation}
as the weighted average of $W$ and $B$. We note that $W$ always under-estimates with finite samples, but converge to truth as $n \to \infty$. So we compute: 
\begin{equation}
\hat{R} = \sqrt{\frac{\hat{\Var}(\psi|y)}{W}}
\end{equation}
This ratio should decline to 1 as $n \to \infty$. 

\end{itemize}

MCMC convergence analysis [Statistical Rethinking, Chapter 8]
\begin{itemize}
	\item Defining Stan model: define the distribution of each parameter, and the distribution of data (likelihood). Define/initialize data.
	
	\item Summarizing posterior samples: use extract.samples() function, then plot pairwise correlation and histogram with pair() function.
	
	\item Diagnosis of chain: use plot() function to show traces of parameters. Check for: stationarity and mixing (no strong auto-correlation, zigzag behavior).
	
	\item Number of samples needed: no simple answer. If the goal is to find mean, generally fewer. If interested in the distribution at extreme values, need more samples. Generally discard the earlier samples (warm-up). Note: different from burn-in, the warm-up are used to adapt sampling, not from target distribution.
	
	\item Number of chains needed: use multiple chains (often 4) to check. And if confirm the correct behavior, use a single long chain is more efficient - no multiple warm-up.
	
	\item Dealing with un-identifiable models: using flat prior can lead to wildly large parameters, e.g. in regression with strong colinearity. Use weakly informative prior can help a lot.
\end{itemize}

Efficient MCMC samplers [BDA, Chapter 12]
\begin{itemize}

\item Parameterization: Gibbs sampler is most efficient when the parameters are independent, so if possible, reparameterize s.t. the posterior distribution is independent (if normal, then perform linear transformation). 

\item Jumping rule: for M-H algorithm, suppose we could approximate the target distribution by a normal distribution with variance matrix $\Sigma$. Then the normal jumping kernel is: 
\begin{equation}
Q(X'|X^t) \sim N(X^t, c^2 \Sigma)	
\end{equation}
The most efficient has scale $c \approx 2.4 \sqrt{d}$, where $d$ is the number of variables. 

\item Adaptive algorithm: the jumping rule (the scale) can be tuned. Intuitively, when close to the optimum, we reduce the step size s.t. the optimum is not missed. However, we need to fix the step size in the end to draw samples (otherwise, the chain may not converge). 

\item Acceptance rate: one could monitor the MCMC runs by checking the acceptance rates. For Metropolis jumps, tune the step size s.t. the acceptance rates are near 20\% (when altering a vector of parameters) or 40\% (when altering a single parameter a time). 

\end{itemize}

Obtaining an approximate sample through posterior mode: 
\begin{itemize}
\item Crude estimate: the first step of sampling is often to find a crude estimate of the parameter values. Ex. for hierarchical normal model, this could be done through estimation of the mean of each group, then estimate the population mean and variance from group means. 

\item Finding posterior modes: any optimization algorithm can be used if the (normalized) density can be evaluated for any parameter values. 
\begin{itemize}
	\item For Bayesian problems, the conditional maximization (CM) algorithm and EM (for marginal posterior, see below) are often useful.
	\item Finding multiple local modes: run the algorithm with different starting points to obtain all the local modes. 
\end{itemize}

\item Normal or normal-mixture approximations: suppose we find the posterior mode at $\hat{\theta}$, and the covariance matrix $V_{\theta}$ can be found (e.g. through numerical derivative), then we could approximate the posterior by $N(\hat{\theta}, V_{\theta})$. If we find multiple modes, then the posterior is approximated by normal mixture: 
\begin{equation}
p_{\text{approx}}(\theta)	\propto \sum_k w_k N(\theta|\hat{\theta}_k, V_{\theta k})
\end{equation}
where $w_k$ is the weight of the $k$-th mode. The weights can be determined by equating the actual density at each mode to the approximate density from the equation above. For instance, if the modes are well separated, we can solve $w_k$: 
\begin{equation}
w_k = q(\hat{\theta}_k|y) |V_{\theta k}|^{1/2}	
\end{equation}
where $q(\cdot)$ is the unnormalized density. 
 
\item Marginal posterior: this is imporant for two reasons: 
\begin{itemize}
	\item Nuisance parameters and missing data: we may have a number of parameters/variables in the model that are of no interest. 
	\item Conditional sampling: when the number of parameters in the model is large, sampling/approximation may be difficult (especially with techniques based on posterior mode). Using marginal posterior may greatly reduce the number of parameters/variables, and samping in a low dimensional space is much safer. 
\end{itemize}

\item Finding marginal posterior via EM algorithm: suppose we have a model of $\theta = (\gamma, \phi)$ and want to find the posterior mode of $p(\phi|y)$. The EM algorithm for missing data can be applied. At the E-step, we find the expectation of the log posterior density: 
\begin{equation}
Q(\phi|\phi^t) = \E_{\gamma|\phi^t,y} [\log p(\gamma,\phi|y)]	
\end{equation}
And at the M-step, the $Q$ function is maximized. 
\begin{itemize}
	\item ECM algorithm: maximiation of the $Q$ function can be achieved through conditional maximazation (CM). 
	\item Multiple modes: the EM algorithm can be run with different starting points to obtain multiple local modes. 
	\item Covariance matrix: normal approximation requires the covarance matrix of $\phi|y$ at $\hat{\phi}$. This could be done via the SEM algorithm. 
\end{itemize}

\end{itemize}


Example: hierarchical normal model: 
\begin{itemize}
\item Model: within the $j$-th group ($1 \leq j \leq J$), we have: 
\begin{equation}
y_{ij} \sim N(\theta_j, \sigma^2)	
\end{equation}
And the mean of each group: 
\begin{equation}
\theta_j \sim N(\mu, \tau^2)	
\end{equation}
The unknown parameters are $\theta$, $\mu$, $\tau$ and $\sigma^2$. The prior is: $p(\mu, \log\sigma, \tau) \propto 1$. The posterior: 
\begin{equation}
p(\theta, \mu, \tau, \log \sigma) \propto \prod_{j=1}^J N(\theta_j|\mu, \tau^2) \prod_{j=1}^J \prod_{i=1}^{n_j} N(y_{ij}|\theta_j, \sigma^2)	
\end{equation}

\item Gibbs sampling: the conditional posteriors can be easily determined: 
\begin{itemize}
	\item Conditional posterior of $\theta$: $\theta_j|\mu,\tau,\sigma,y$ follows normal distribution with conjugate prior $N(\mu, \tau^2)$. 
	\item Conditional posterior of $\mu$: only depends on $\theta_j$ and $\tau$, $\mu|\theta,\tau,\sigma,y \sim N(\hat{\mu}, \tau^2/J)$ where $\hat{\mu}$ is the mean of $\theta_j$. 
	\item Conditional posterior of $\sigma^2$: $\sigma^2|\theta,\mu,\tau,y$ only depends on $\theta$ and $y$, inverse-$\chi^2$ distribution. 
	\item Conditional posterior of $\tau^2$: $\tau^2|\theta,\mu,\sigma,y$ only depends on $\theta$ and $\mu$. Also inverse $\chi^2$ distribution. 
\end{itemize}

\item M-H and Gibbs sampling: notice that at the population level, only three parameters $\mu,\tau,\sigma$. So we could perform Metroplis jumping of the three parameters in low-dimensional space, and once the three parameters are sampled, we sample $\theta_j$ conditioned on these values. 

\item Normal approximation: the marginal posterior of $\mu,\tau,\sigma|y$ can be approximated by normal distribution (low-dimensional space), so we need to determine the mode of the posterior. The marginal posterior is obtained by integrating over $\theta$ in the joint posterior (similar to the posterior predictive of normal distribution). It can be maxizied by EM, the log. of joint posterior density: 
\begin{equation}
\log p(\theta,\mu,\tau,\log\sigma|y) = -n \log \sigma - (J-1) \log \tau -\frac{1}{2 \tau^2}\sum_j (\theta_j - \mu)^2 - \frac{1}{2\sigma^2} \sum_j \sum_{i=1}^{n_j} (y_{ij} - \theta_j)^2 + \text{Const}
\end{equation}
At the E-step, only the last two terms depend on $\theta$ (this missing parameters), and the expectation can be easily determined (both are quadratic of $\theta_j$). 

\end{itemize}

Computing marginal likelihood and Bayes factors: 
\begin{itemize}
	\item Laplacian approximation. 
	
	\item Sampling from prior: suppose we want to compute $P(D) = \int p(D|\theta) p(\theta) d\theta$, we sample $\theta_i$ from $p(\theta)$, then 
	\begin{equation}
		P(D) \approx \frac{1}{n} \sum_i P(D|\theta_i)
	\end{equation}
	The problem of this is that it has a large variance. The posterior of $\theta$ is often a narrow peak, so $P(D|\theta)$ is close to 0 most of times (when it is outside that peak), and occasionally get big values.  
	
	\item Harmonic mean estimator: the problem of using prior is that we often sample from the region with little support. So instead, we sample $\theta$ from the posterior, $\theta^{(i)}, 1 \leq i \leq m$. Now we can compute the integral with importance sampling where the weights are: 
	\begin{equation}
		w_i = \frac{P(\theta^{(i)})}{P(\theta^{(i)}|D)} = \frac{P(D)}{P(D|\theta^{(i)})}
	\end{equation}
	Plug in this to the equation of importance sampling, we have the estimator: 
	\begin{equation}
		\hat{P}(D) = \left[\frac{1}{m} \sum_i P(D|\theta^{(i)})^{-1}\right]^{-1}
	\end{equation}
	The problem of the Harmonic mean estimator is that: it depends only the sample from the posterior, which is somewhat insensitive to the prior. When we have large $D$, then two different models would have the same posterior, and thus same $\hat{P(D)}$. The true marginal likelihood, on the other hand, depend strongly on the prior. See \url{https://radfordneal.wordpress.com/2008/08/17/the-harmonic-mean-of-the-likelihood-worst-monte-carlo-method-ever/}. So the Harmonic mean estimator works well only when the prior has a large influence on the posterior. 
	
	\item Using long-tailed distribution: the problem with Harmonic mean is that the posterior distribution is often 0, outside the peak; so it has a vary large variance, and not a good proposal distribution. To fix it, the general intuition of choosing the proposal distribution is to have a long tail distribution, e.g. mixture of prior and posterior distribution [Weighted Average Importance Sampling and Defensive Mixture Distributions, Hesterberg]. We can also sample from posterior, but combining it with KDE to obtain a smooth distribution. 
\end{itemize}

Improving marginal likelihood estimation for Bayesian phylogenetic model selection [Xie and Chen, Syst Biol, 2011]
\begin{itemize}
	\item Problem: let $y$ be the data, $\theta$ the model parameters. We define $f(\theta)$ be the prior of $\theta$, $f(y|\theta)$ the likelihood. Our goal is to evaluate $f(y) = \int f(y|\theta) f(\theta) d\theta$. 
	
	\item Harmonical mean estimator is biased: intuitively, we would mostly sample from high posterior regions, and as a result, we will have overrepresentation of high likelihood. 
	
	\item Idea of Stepping stone sampling: we use importance sampling, but we use a progression of importance distributions that vary from prior to posterior. During progression, the distributions change incrementally so that at every step, the importance distribution is a good approximation for computing the marginal likelihood. 
	
	\item Power posterior density: we define density function
	\begin{equation}
	q_{\beta}(\theta) = f(y|\theta)^{\beta} f(\theta)
	\end{equation}
	This function is prior when $\beta = 0$ and posterior (unnormalized) when $\beta = 1$. The normalization constant is: 
	\begin{equation}
	C_{\beta} = \int q_{\beta}(\theta) d\theta
	\end{equation}
	And the normalized PDF as $p_{\beta}(\theta) = q_{\beta}(\theta) / C_{\beta}$. It is easy to see that the marginal likelihood is $c_1$, and we can write it as a product of $c_{\beta_k} / c_{\beta_{k-1}}$ as we vary $\beta$ from 0 to 1. 
	
	\item Computing $c_{\beta_k} / c_{\beta_{k-1}}$ by importance sampling: when evaluating both numerator and denomator, we use $p_{\beta_{k-1}(\theta)}$ as the importance distribution. The weights is given by: 
	\begin{equation}
	w(\theta) = \frac{f(\theta)}{p_{\beta_{k-1}(\theta)}} = \frac{c_{\beta_{k-1}}}{f(y|\theta)^{\beta_{k-1}}}
	\end{equation}
	Apply the importance sampling equation: 
	\begin{equation}
	c_{\beta_k} = \frac{1}{n} \sum_i f(y|\theta_i)^{\beta_k} w(\theta_i) / \frac{1}{n} \sum_i w(\theta_i) = c_{\beta_{k-1}} \sum_i \frac{f(y|\theta_i)^{\beta_k}}{f(y|\theta_i)^{\beta_{k-1}}} / \sum_i w(\theta_i)
	\end{equation}
	\begin{equation}
	c_{\beta_{k-1}} = \frac{1}{n} \sum_i f(y|\theta_i)^{\beta_{k-1}} w(\theta_i) / \frac{1}{n} \sum_i w(\theta_i) = c_{\beta_{k-1}} \sum_i \frac{f(y|\theta_i)^{\beta_{k-1}}}{f(y|\theta_i)^{\beta_{k-1}}} / \sum_i w(\theta_i) = n c_{\beta_{k-1}} / \sum_i w(\theta_i)
	\end{equation}
	where $\theta_i$ are samples of $\theta$ from $p_{\beta_{k-1}(\theta)}$. Now divide the two, we have: 
	\begin{equation}
	\frac{c_{\beta_k}}{c_{\beta_{k-1}}} = \frac{1}{n} \sum_i \frac{f(y|\theta_i)^{\beta_k}}{f(y|\theta_i)^{\beta_{k-1}}}
	\end{equation}
	
\end{itemize}
%%%%%%%%%%%%%%%%%%%%%%%%%%%%%%%%%%%%%%%%%%%%%%%%%%%%%%%%%%%%
\subsection{Advanced MCMC methods}

Background: Hamiltonian mechanics
\begin{itemize}
\item Reference: Chapter 5 of MCMC handbook. \url{http://www.mcmchandbook.net/HandbookChapter5.pdf}. See \text{Reference/Stat-ML/Bayesian/}. 

\item Hamiltonian mechanics: a dynamic system (e.g. a particle) that can be characterized by vectors $(q,p)$ where $q$ is position vector and $p$ momentum vector. Let $H(q,p)$ be the Hamiltonian of the system (energy): $H(q,p) = U(q) + K(p)$ where $U(q)$ is the potential energey and $K(p)$ the kinetic energy. One can thus obtain the description of the system in terms of $H$: 
\begin{equation}
	\frac{d q_i}{dt} = \frac{\partial H}{\partial p_i} \quad \frac{d p_i}{dt} = \frac{-\partial H}{\partial q_i}
\end{equation} 
Consider a single particle. The first equation says: the velocity equals the derivative of its kinetic energy with respect to its momentum. The second equation is effectively Newton's second law, where the force equals the negative gradient of potential energy. Geometrically, we can think of a system in the generalized coordinates $(q_i)$ and generalized velocities $(p_i)$. 

\item Common assumption about $K$ function: usually kinetic energy, $K = p^T M^{-1} p/2$, where $M$ is the diagonal matrix of mass. Or $K = \sum_i p_i^2 / m_i$. 

\item Example: simple harmonic oscillator, $U(q) = q^2/2, K(p) = p^2/2$. The system can be described by a \textit{phase portrait}: how $(q,p)$ changes in the phase diagram. The solution/behavior of the system: circle in the phase diagram, where $p$ and $q$ both are periodic functions. 

\item Properties of Hamiltonian dynamics: Reversibility, Conservation of the Hamiltonian, Volume Preservation in the phase space (Liouville’s theorem). 

\item Numerical algorithms to solve Hamiltonian system: Euler's method, we update $p$ and $q$ at each step of size $\epsilon$: 
\begin{equation}
	p_i(t+\epsilon) = p_i(t) - \epsilon \frac{\partial U}{\partial q_i}(q(t)) \qquad q_i(t+\epsilon) = q_i(t) + \epsilon \frac{p_i(t)}{m_i}
\end{equation}
where the first equation describes the change of moment because force (from potential energy change), and the second is the update of position from momentum. However, Euler's method does not preserve volume and may diverge. Modified Euler method does better. Leapfrog is even better: do update of $q$ for a half-step, then update $p$ for a full step, and then update $q$ for another half-step.  

\end{itemize}

Hamiltonian MC (HMC) [MCMC handbook, Chapter 5]
\begin{itemize}
\item Basic idea of HMC: suppose we are minimizing a function, we model it as a physical system to minimize the potential energy. The idea is that given a physical/dynamic system, let it evolve over time, it will reach the point of lowest potential energy. So we create a dynamic system whose potential energy is the objective function and take advantage of the Hamiltonian dynamics. This is generally faster than random walk. In HMC case (as opposed to minimization), we are not minimizing a function, but simulation under Hamiltonian dynamics would give the distribution one wishes to sample from.  

\item Intuitions of HMC: consider the landscape of the potential energy, which may have many local minimum. Imagine we have a ball that travels this landscape, to reach the mimimum, we will ``release'' it and follow its motion. Naturally it will reach some local minimum. To escape from the trap, we give the ball some random momentum, then it will have a chance to escape from the current minimum, and by natural dynamics reach a second minimum. If we repeat this many steps, we will reach the global minimum.    

\item HMC procedure: to reach a posterior defined on parameters $q$, we define a system with
\begin{equation}
U(q) = - \log [\pi(q) P(D|q)], \quad K(p) = \sum_i \frac{p_i^2}{2m_i}
\end{equation} 
where $m_i$ are parameters of the procedure (similar to step size in MCMC). To create a MC, we define 
\begin{equation}
P(q,p) = \frac{1}{Z} \exp(-H(p,q)/T)
\end{equation}
The procedure has two steps: (1) it updates $p$ by sampling $p_i \sim N(0, m_i)$. (2) Run the leapfrog method to update $q$: following the Hamiltonian dynamics. The leapfrog method is effectively numerical method of solving the PDE of Hamiltonian equations.

\item Why HMC converges to the target distribution? Ex. normal distribution. Hamiltonian dynamics: given by simple Harmonic oscillator with $U(q) = q^2/2$. Given an initial $p$: the distribution of $q$ is oscillation around $q = 0$. Initial value of $p$ itself follows normal distribution. The overall results samples from the entire normal distribution: intuitively, $p$ is usually close to 0, and $q$ revolves around the circle defined by $p$; overall the entire distribution of $p$, $q$ would follow normal distribution. 

\item Why is HMC more efficient than standard MCMC? Hamiltonian dynamics is a more efficient way of exploring parameters using gradient information. Consider one parameters case, the posterior can be thought of as surface of a bowl. At the MAP, when the posterior changes quick or large gradient (steep surface), Hamiltonian dynamics will quickly take the particle to the MAP; when the gradient is small (flat surface), the particle will move more freely and could be far from the MAP. 

\item More general case: given an initial $p$, HMC samples the landscape determined by $p$: usually, HMC explores local neighborhood. With small probability $p$ is large, this allows HMC to explore distal regions. 
 
\end{itemize}

%%%%%%%%%%%%%%%%%%%%%%%%%%%%%%%%%%%%%%%%%%%%%%%%%%%%%%%%%%%%
\section{Variational Inference}

Reference: [Bishop, Chapter 10], [Murphy, Chapter 21]

Background: Calculus of Variations [Wiki]
\begin{itemize}
	\item Problem: given a functional, a mapping from a function to $\mathbf{R}$, our goal is to optimize it, i.e. finding the function that optimizes the real value. Ex. given two points, find the curve with the minimum length. 
	
	\item Euler-Lagrange Equation: consider the functional (e.g. length of curve) defined on function $y$ and its derivative $y'$: 
	\begin{equation}
	J[y] = \int_{x_1}^{x_2} L(x, y(x), y'(x)) dx
	\end{equation}
	Our goal is to minimize $J[y]$. We can derive conditions similar to the result in calculus (vanishing gradient). Consider a function around $f$, denoted as $f + \epsilon \eta$ for some arbitrary function $\eta$ and small number $\epsilon$, where $\epsilon \eta$ is called \textit{variation}. The key idea is that at the minimum, we should have $\frac{dL}{d\epsilon}= 0$. Using the basic rule of total derivative, we have: 
	\begin{equation}
	\int_{x_1}^{x_2} \eta(x) \left( \frac{\partial L}{\partial f} - \frac{d}{dx} \frac{\partial L}{\partial f'} \right) dx  = 0
	\end{equation} 
	Since this must be held for any function $\eta(x)$, we must have: 
	\begin{equation}
	\frac{\partial L}{\partial f} - \frac{d}{dx} \frac{\partial L}{\partial f'} = 0
	\end{equation}
	This is called the Euler-Lagrange equation, and the left hand is called the \textit{functional derivative} of $J[f]$. In general this gives a second-order ordinary differential equation which can be solved to obtain the extremal function $f(x)$.  
	
	\item Example: given two points $(x_1, y_1)$ and $(x_2, y_2)$, show that the straight line minimizes the curve length connecting the two points. 
\end{itemize}

Variational inference: 
\begin{itemize}
\item Motivation: approximate a complex joint distribution. Suppose we have a distribution $p(z)$ to sample from, or to calculate the integral. Example: Bayesian posterior distribution, Ising model, etc. We want to approximate $p(z)$ by some simpler distribution: certain parameteric form, or factorizable into multiple distributions. Relation to calculus of variations: find a function $q$ that minimizes the functional $L(q)$, the distance of $q$ to the target distribution $p$. 

\item Minimizing KL divergence: we want to find the distribution $q(z)$ s.t. the KL divergence $D(q||p)$ is minimized: 
\begin{equation}
D(q||p) = \int q(z) \ln \frac{q(z)}{p(z)} dz
\end{equation}
Note: we minimize this instead of $D(p||q)$ because it would involve computing expectation over true distribution $p$, which is unknow. We define $L(q)$ as: 
\begin{equation}
L(q) = 	\int q(z) \ln \frac{p(z)}{q(z)} dz
\end{equation}
It is easy to see that $D(q||p) + L(q) = 0$, thus minimizing $D(q||p)$ is equivalent to maximizing $L(q)$. We write it as: 
\begin{equation}
L(q) = 	\int q(z) \ln p(z) dz	- \int q(z) \ln q(z) dz = \E_q[\ln p(Z)] + H(q)
\end{equation}
The first term is called ``energy'' (the target distribution $p$ fits $q$, imagining data is generated from $q$; in other words, $q$ should generate a good sample according to $p$), and the second term is the entropy of $q$. We thus call $L(q)$ the negative free energy. Thus $q$ should balance (1) approximate the target distribution and (2) maximizing the entropy.  

\item Intuition of KL divergence: $KL(q||p)$ avoids regions where $q(x)$ is large/moderate, but $p(x)$ small. $L(q) = E_q[\ln p(z)] + H(q)$, so intuitively, we want $q$ to cover ``critical regions'' of $p(z)$, at the cost of sacrificing regions where $p(z)$ is small. Generally this means the approximation will be too compact comparing with the true distribution. See Bishop Figure 10.2. 

\item Multi-mode distributions: another example of Variational approximation. $p$ is a mixture of two normal. Numerical algorithm usually leads to approximation of a single mode. Bishop Figure 10.3. 
\end{itemize}

Mean-field approximation: 
\begin{itemize}
\item Factorization: a common assumption is that $q$ can be factorized: 
\begin{equation}
q(z) = \prod_{i=1}^M q_i(z_i)	
\end{equation}

\item Maximizing $L(q)$: we infer how $q_j$ is related to other components $q_{i\neq j}$. 
\begin{equation}
L(q) = \int q_j(z_j) \left(\ln p(z) \prod_{i \neq j} q_i(z_i) d z_{i \neq j}\right) dz_j + H(q_j) + \sum_{i \neq j} H(q_i)
\end{equation}
We used the fact that the entropy of the product of distributions is the sum of entropy of each component. We define: 
\begin{equation}
\ln \tilde{p}(z_j) = \E_{q_i, i \neq j}[\ln p(z)]	= \int \ln p(z) \prod_{i \neq j} q_i(z_i) d z_{i \neq j}
\end{equation}
Then: 
\begin{equation}
L(q) = \int q_j(z_j) \ln\tilde{p}(z_j) dz_j - \int q_j \ln q_j(z_j) dz_j + \sum_{i \neq j} H(q_i)	= -\int q_j \ln \frac{q_j(z_j)}{\tilde{p}(z_j)} dz_j + \sum_{i \neq j} H(q_i)
\end{equation}
When $q_i(z_i)$ is fixed for all $i \neq j$, and only $q_j$ is free to change, we note that the above is the negative of KL divergence between $q_j$ and $\tilde{p}(z_j)$ and is maximized at $q_j^*(z_j) = \tilde{p}(z_j)$. Thus we have the general expression for the optimal solution: 
\begin{equation}
\ln q_j^*(z_j) = \E_{q_i, i \neq j}[\ln p(z)]
\label{eq:variational}
\end{equation}
%Intuitively, this means that when we infer/sample $Z_j$, assuming the distributions of other variables are known (from previous iterations), then we should take $\ln p(z)$ but averaging over other variables (the marginal distribution of $Z_j$ under $p$ is also averaging over other variables, we are just slightly more complicated). 

\item Intuition of the mean-field equation: wlos, we assume we only have two dimensions $z_i$ and $z_j$ (imagine 2D normal distribution, with $j$ horizontal dim.). Apparently, the best approximation of $z_j$ should be $p_j(z_j)$, marginalizing over $z_i$. Why don't we do that? This is often difficult, e.g. imaging Bayesian Variable Selection for regression, the marginal of $p(\beta_j)$ is the PIP, which is difficult to find. So the mean-field VB approximation equation is computing the ``pseudo-marginal'' over other variables, by assuming they follow the approximate distribution. Specifically, to obtain $q_j(z_j)$ at a strip near $z_j$: we need to marginalize the density $p(z_i, z_j)$ over all values of $z_i$. The density near $z_i$ can be approximated by $q_i(z_i) dz_i$. In VB, we actually consider log-pdf, so we have:
\begin{equation}
p_j(z_j) = \int p(z_i, z_j) dz_i \Rightarrow \ln q_j(z_j) = \int \ln p(z_i, z_j) \cdot q_i(z_i) dz_i
\end{equation}
Eventually, after VB, $q_j(z_j)$ should converge to the true marginal of $z_j$.

\item Procedure of variational inference: We cycle through the components $q_j(z_j)$: at each time, assume the distribution of all other components are known, and solve the optimal $q_j^*$ according to Equation~\ref{eq:variational}. Remark: the idea is a generalization of EM or Gibbs sampling: instead of sampling or maximization given the other variables, variational inference find the optimal distribution of one component given the distribution of the rest. See [Bishop, Figure 10.4] for an example of $(\mu, \tau)$ in Gaussian distribution - very intuitive. 

\item Why the procedure is more tractable than sampling the joint distribution? Each time, we are dealing with distribution of 1D, and the expectation of $\ln p(z)$ over other variables can be simpler.     

\item Properties of factorized approximation: the main assumption of variational method is the independence assumption. Example: we could approximate a bivariate normal distribution, but the independence assumption is clearly invalid when the two components are correlated. See [Bishop, Figure 10.2]. 
\begin{itemize}
\item Minimizing $\text{KL}(q||p)$: from the integral, the integrand is large at small $p(z)$, so the solution (which tries to minimize KL) will try to avoid the regions where $p(x)$ is small, but $q(x)$ large. Thereforem, the solution will cover the modes of the true distribution, but do not expand beyond those. 
\item Minimizing $\text{KL}(p||q)$: the opposite situation. The solution will try to avoid regions where $p(x)$ is substantial. 
\end{itemize}

\end{itemize}

Variational Bayesian: approximate the posterior distribution
\begin{itemize}
\item Maximizing $L(q)$: suppose we want to solve the posterior distribution $p(z|x)$, where $Z$ represents all the unknowns (parameters and missing variables), and $X$ are data. We first relate to the joint distribution $p(x,z)$ by: 
\begin{equation}
\ln p(x) = \ln p(x,z) - \ln p(z|x)	
\end{equation}
The above is true for any $Z$. Let $q(z)$ be the approximation of $p(z|x)$. We take the expectation over $q(z)$ in the equation above, and this leads to:  
\begin{equation}
\ln p(x) = L(q) + \text{KL}(q||p)	
\end{equation}
where we have defined: 
\begin{equation}
L(q) = \int q(z) \ln \frac{p(x,z)}{q(z)} dz	
\end{equation}
\begin{equation}
\text{KL}(q||p)	= \int q(z) \ln \frac{q(z)}{p(z|x)} dz	
\end{equation}
Since $p(x)$ is independent of $q$, thus minimizing KL divergence is equivalent to maximizing $L(q)$. We note that $L(q)$ provides a lower bound of the marginal log likelihood $\ln p(x)$:
\begin{equation}
L(q) \leq \ln p(x)
\end{equation}
It is called ELBO. It can be used in several ways: (1) In VB, we should see monotonic increase of $L(q)$. (2) It provides estimate of marginal likelihood, which is useful for Bayesian selection. 

\item Procedure of Variational Bayesian: similar to the general case, we have the iterative equation: 
\begin{equation}
\ln q_j^*(z_j) = \E_{q_i, i \neq j}[\ln p(x,z)] + \text{const}
\end{equation}
where the expectation is defined as: 
\begin{equation}
\ln \tilde{p}(x,z_j) = \E_{q_i, i \neq j}[\ln p(x,z)]	= \int \ln p(x,z) \prod_{i \neq j} q_i(z_i) d z_{i \neq j}
\end{equation}

\item Direction optimization of ELBO $L(q)$ [SuSiE paper]: we can write $L(q)$ as functions of $q$: 
\begin{equation}
L(q) = \E_q[\ln p(z)] + \E_q[\ln p(x|z)] - \E_q [\ln q(z)]
\end{equation}
where the three terms correspond to expectation of prior, of likelihood, and the entropy of $q$. It is possible to directly optimize this function by coordinate descent: find the best $q_j$, supposing other $q_i, i \neq j$ are given. 
\end{itemize}

Thoughts about VB:
\begin{itemize}
	\item How does VB equation relate to conditional distribution? When VB converges, can we say that the expectation of an unknown is correct?
\end{itemize}

Example: Variational Inference of Univariate Gaussian distribution: [Bishop, 10.1.3]
\begin{itemize}
\item Problem: suppose $X \sim N(\mu,\tau^{-1})$, where the parameters follow conjugate prior: 
\begin{equation}
p(\tau) = \text{Gamma}(\tau|a_0,b_0)	
\end{equation}
\begin{equation}
p(\mu|\tau) = N(\mu|\mu_0, (\lambda_0 \tau)^{-1})	
\end{equation}
Given the data $X_1, \cdots, X_n$,  the posterior of $\tau,\mu|X$:
\begin{equation}
\ln p(\tau,\mu|X) = \ln p(\tau) + \ln p(\mu|\tau) + n/2 \cdot \ln \left(\frac{\tau}{2 \pi}\right) - \left[ \frac{\tau}{2} \sum_i (x_i - \mu)^2 \right]	+ \text{const}
\end{equation}

\item Variational inference: we start with inference of $\mu$ given $\tau$. In this case, any term in $\ln p(\tau, \mu|X)$ that does not depend on $\mu$ would not matter. We have: 
\begin{equation}
\ln q^*_{\mu}(\mu) = - \frac{\E_{\tau}[\tau]}{2} \left[ \lambda_0 (\mu - \mu_0)^2 + \sum_i (x_i - \mu)^2 \right] + \text{const}
\end{equation}
We see that $q_{\mu}(\mu)$ has normal density with mean and variance dependent on $\E_{\tau}$. And 
\begin{equation}
\ln q^*_{\tau}(\tau) = (a_0 - 1) \ln \tau - b_0 \tau + \frac{n + 1}{2} \ln \tau - \frac{\tau}{2} \E_{\mu} \left[  \lambda_0 (\mu - \mu_0)^2 + \sum_i (x_i - \mu)^2 \right] + \text{const}
\end{equation}
We see that $q_{\tau}(\tau)$ is Gamma with parameter dependent on $\E[\mu]$ and $\E[\mu^2]$. This leads to VB iteration: we determine the parameters of $q_{\mu}(\mu)$ (normal) and $q_{\tau}(\tau)$ (Gamma) using $\E[\tau], \E[\mu]$ and $\E[\mu^2]$ from the previous iteration. Note: Equations (10.28) and (10.29) should have $(N + 1)/2$ instead of $N/2$. 

\item Remark: the challenge of variational inference is that when computing $q_j(z_j)$, we do not know the distribution (form) of other components. To address this, we take expectation over other parameters, then the unknown distributions of $q_i(z_i)$ ($i \neq j$) enter into the equation in the form of some expectation, e.g. $\E_{\mu}$ over a quadratic function of $\mu$ in the example above. As a result, we may obtain the form of $q_j(z_j)$ with some unknown constants (expectations over other parameters) - these constants can be obtained from previous iterations. It makes problem easier if we can recognize the form of $q_j(z_j)$ from its log-likelihood function, and how its parameters depend on expectation over other dimensions. 
\end{itemize}

Variational Bayesian inference for linear and logistic regression [Drugowitsch, arxiv, 2013]
\begin{itemize}
	\item Model: linear regression
	\begin{equation}
	P(y|x, w, \tau) = N(y|w^T x, \tau^{-1}) = \left(\frac{\tau}{2 \pi}\right)^{1/2} \exp \left(-\frac{\tau}{2} (y-w^Tx)^T (y-w^T x)\right)
	\end{equation}
	The prior of $w$ and $\tau$ is conjugate normal inverse-gamma:
	\begin{equation}
	p(w, \tau | \alpha) = N(w|0, (\tau \alpha)^{-1} I) \cdot \text{Gamma}(\tau|a_0, b_0)
	\end{equation}
	
	\item VB update: assuming $\alpha$ is given, we update $w$ and $\tau$. We first write $\log p(w, \tau, \alpha, D)$ as (ignoring the term constant wrt. $w$): 
	\begin{equation}
	\log p(w, \tau, \alpha, D) = \log N(w|0, (\tau \alpha)^{-1} I) -\frac{\tau}{2} (y-w^Tx)^T (y-w^T x) + \text{const}
	\end{equation}
	This is a quadratic function of $w$: the first term has $\tau \E(\alpha) w^T w$. We can thus solve mean and variance of $w$ in terms of $\E(\alpha)$ and $\tau$. Similarly, we can show that $\tau$ would follow Gamma distribution, with parameters depend on $\E(\alpha)$ and mean and variance of $w$. 
\end{itemize}

Variational Bayes EM (VBEM) [Murphy, 21.6]
\begin{itemize}
	\item Idea of VBEM: (1) M-step: account for uncertainty of parameters, instead of MAP. (2) E-step: posterior of $z_i$, averaging the posterior of parameters, instead of at MAP. In practice, posterior at posterior mean of parameters.
	
	\item VBEM for GMM: let $z_i$ be cluster label, with $z_{ik} = 1$ if sample $i$ belongs to cluster $k$ and 0 o/w. The probability of $z_{ik} = 1$ is $\pi_k$.  
	\begin{equation}
	x_i | z_{ik} = 1 \sim N(\mu_k, \Lambda_k^{-1})
	\end{equation}
	The prior probabilities of parameters $\theta = (\pi, \mu, \Lambda)$ are given by:
	\begin{equation}
	p(\theta) = \text{Dir}(\pi|\alpha_0) \prod_k N(\mu_k | m_0, (\beta_0 \Lambda_k)^{-1}) \text{Wi}(\Lambda_k | L_0, \nu_0)
	\end{equation}
	We define $q(\theta, z) = q(\theta) \prod_i q(z_i)$ as approximate posterior of $\theta$ and $z_i$. Inference has two steps: 
		
	\item Variational M-step: determine forms of $q(\theta)$. We first write the log-likelihood as:
	\begin{equation}
	\log p(x, z, \theta) = \log p(\pi) + \sum_k \log p(\mu_k, \Lambda_k) + \sum_i \log p(z_i | \pi) + \sum_i \sum_k z_{ik} \log N(x_i | \mu_k, \Lambda_k^{-1})
	\end{equation}
	To determine $q(\pi)$ and $q(\mu_k, \Lambda_k)$, we take expectation over $z_i$ with density $q(z_i)$. Suppose we have $r_{ik}$ as the expectation of $z_{ik}$ (probability of $i$ assigned to cluster $k$) in the previous iteration. We can show that expectation over $z_i$ can be expressed in terms of $r_{ik}$. So we have:
	\begin{equation}
	q(\pi) = \text{Dir}(\pi | \alpha) \qquad \alpha_k = \alpha_0 + N_k \qquad N_k = \sum_i r_{ik}
	\end{equation}
	So this is similar to standard EM (M-step). Similarly, we can show that $q(\mu_k, \Lambda_k)$ is also normal with parameters similar to what we have under EM. Ex. the mean of cluster $k$ is weighted average of all data points assigned to $k$, where weights are given by $r_{ik}$. 
	
	\item Variational E-step: determine $q(z_i)$. Problem becomes computing expectation of functions defined on $\theta$. From the equation of $\log p(x, z, \theta)$, it is easy to see that only two terms depend on $\pi$ are: $\log p(\pi) + \sum_i \log p(z_i|\pi)$. Expectation over $q(\theta)$ are given by Equation (21.128). The problem reduces to: (1) finding $\E(\log \pi_k)$, when we know $\pi_k$ follows a given Dirichlet distribution; (2) find expectation of quadratic form $(x_i - \mu_k)^T \Lambda_k (x_i - \mu_k)$, over $q(\mu_k, \Lambda_k)$. 
	
	\item Model selection: using ELBO as approximation of marginal likelihood.
	
	\item Large number of clusters with small $\alpha_0$: shrink mixing weight to 0. However, the difference between mixture model and variable selection is: the total weight is fixed at 1, so small clusters get shrinked, but big clusters get bigger.
\end{itemize}

%%%%%%%%%%%%%%%%%%%%%%%%%%%%%%%%%%%%%%%%%%%%%%%%%%%%%%%%%%%%
\section{Hierarchical Model and Empirical Bayes}

Modeling strategy: 
\begin{itemize}
\item Intuition: suppose we have a baseline model, where a number of parameters are used. If there are additional structure in the data, e.g. certain objects are similar, then we could incorporate the structure by modeling the distributions of parameters. So we have an additional layer of model of parameters.

\item Modeling group structure: suppose the samples can be grouped, and within each group, we have a parametric model. The parameters of the different groups may be related, e.g. the group parameters may depend on certain group-level variables. Ex. hierarchical normal model; hierarchical regression model. 

\item Modeling structure of parameters: parameters may have specific interpretations, and can be modeled. Ex. a parameter may represent the influence of one variable on response variable, and the influence of multiple varaibles may be similar (thus parameters should be similar). 

\item Fixed vs random effects: under fixed effect model, certain variables/parameters are constant across groups(complete pooling) ; under the random effect model, the group-level variable/parameter are still random, even though they share certain things in common (partial pooling). Ex. hierarchical normal model: (1) fixed-effect model, the mean of each group is constant; (2) random-effect model, the group mean is a sample from a normal distribution. 

\item Analysis of multi-level modeling: analyze how the individual variation can be partitioned, and if the model appropriately captures all variations. Ex. under hierarchical regression, the individual variation consists of: individual variation within the group and variation of group, and the latter consists of the group mean plus some random variation from the mean. 
\end{itemize}

Example of multi-level modeling: the disease risk of individuals. 
\begin{itemize}
\item Group-level modeling: suppose the risk depends on genetics (individual), and environmental factors such as diet (individual), pollution (region) and water quality (region). Under the multi-level modeling, one would write the risk as a function of genetics, diet and a variable for regions; then the region variable can be modeled as a function of pollution, water quality, etc. 

\item Parameter-level modeling: the genetics can be partitioned into many genes, however, the effect of a gene can be modeled. Suppose $\beta_j$ is the effect of the $i$-th gene, and the gene belongs to $K$ groups, then $\beta_j$ is a function of the sum of effect of all groups the gene belongs to, plus some random variation. 
\end{itemize}

Applications of multi-level modeling: [Ji \& Liu, NBT, 2010]
\begin{itemize}
\item Motivation: the data contains additional structures in the form of groups (which could be nested), or similarity between objects; or additional determinants that may influence the group-level properties. These additional structure or factors can be treated by modeling the relevant parameters/effects. Specifically, grouping or similarity can be expressed as: the relevant parameters viewed as samples from a common distribution; the group-level effect may be determined from other group-level factors; etc. 

\item Benefits: data can be aggregated for inference of certain parameters: all the groups contain information of top-level parameters, which in turn change of our inference on group-specific parameters (acting like prior distributions). This benefit is stronger when the heterogeneity is small. When hierarchical modeling is applied to estimate variance, it is called ``variance stabilization''. 

\item Variance partition: similar to ANOVA (or ANOVA can be viewed as a special form of multi-levle models), the variation can be parititioned into the top-level variation (group variation), and the intrinsic variations within groups. 

\item Examples: (1) Differentially expressed (DE) genes: assume the variance of each gene is a sample from a top-level distribution. Important when the sample size is small, thus variance of individual genes is not robust. (2) SNPs: of the same genes/pathways may follow the same top-level distribution; or the effect of a SNP can be regressed on the properties of the SNP such as its position, cross-species conservation, etc. 
\end{itemize}

\subsection{Bayesian Hierarchical Models} 

Reference: [Gelman04, chapter 5]

General procedure for Bayesian hierarchical models: 
\begin{itemize}
\item Overview: in general, we are interested in two types of problems in a hierarchical model setting:
\begin{itemize}
	\item Population-level parameters: average over all groups, while taking the heterogeneity across groups into account. It is thus not a simple average over all data points, e.g. two groups, but one has a much larger group variance than the other (thus should be discounted). 
	\item Group-level parameters: borrow information from other groups (i.e. population level parameter) to better infer group-level parameters. Ex. a group with a very small number of instances. 
\end{itemize}
The key problem is to infer the posterior distribution of the hyperparameter, $p(\tau|y)$, as: (1) it may be the objective of study; (2) it helps the inference of group parameters ($\theta$), as once $\tau$ is known (sampled), the inference of $\theta$ is often a standard Bayesian problem. 

\item Model: suppose there are $J$ groups, for the $j$-th group, the data $y_j, 1 \leq j \leq J$ (vector) is generated from the parameters $\theta_j$. The parameters come from a common population distribution, parameterized by the hyperparameter(s) $\phi$. We write the model as: 
\begin{equation}
\phi \rightarrow \left[\theta_j \rightarrow y_j \right]	
\end{equation}
The prior distribution: 
\begin{equation}
p(\phi, \theta) = p(\phi) p(\theta|\phi)	
\end{equation}
and the posterior distribution: 
\begin{equation}
p(\phi,\theta|y) \propto p(\phi, \theta) p(y|\theta, \phi) = p(\phi, \theta) p(y|\theta)
\end{equation}
Note: in hierarchical model, it is important to note that the constant terms (in non-hierarchical model) can depend on the parameters (see the rat tumor example below). 

\item Posterior sampling: some similarlity with the case of nuisance parameters. The difference is that we are interested in both $\phi$ and $\theta$; while we may want to integrate out the nuisance parameters. Typically, sampling may consists of the three steps (if marginal posterior is possible to sample, Step 3): 
\begin{itemize}
\item The posterior distribution: $p(\phi, \theta|y)$. 
\item The conditional posterior distribution: $p(\theta|\phi,y)$, this may be reduced to a familiar case. 
\item The marginal posterior distribution: $p(\phi|y)$. This is often the key step. Several strategies: 
\begin{equation}
p(\phi|y) = \int p(\phi, \theta|y) d\theta	
\end{equation}
Or apply the following equation to any value of $\theta$: 
\begin{equation}
p(\phi|y) = \frac{p(\phi, \theta|y)}{p(\theta|\phi,y)}
\label{eq:marginal_posterior}
\end{equation}
Often if the marginal likelihood is available, the inference of $p(\phi|y)$ is made easier: 
\begin{equation}
p(\phi|y) \propto p(\phi) p(y|\phi) = p(\phi) \int p(y|\theta) p(\theta|\phi) d\theta
\end{equation}
\end{itemize}
The sampling step thus involves: first sampling from $p(\phi|y)$, then $p(\theta|\phi,y)$. One could use alternate Gibbs sampling: 
\begin{equation}
\theta \sim p(\theta|\phi,y)	
\end{equation}
\begin{equation}
\phi \sim p(\phi|\theta,y)	
\end{equation}
We use hierarchical linear model as an example. The first equation: within group regression, sample each group-specific effect parameter, suppose $\phi$ is given. This is a Bayesian regression problem with prior parameterized by $\phi$. The second equation: once each group-specific effect is estimated/sampled, the between group variation and population parameters can be estimated using group-level regression. With both steps, additional Gibbs sampling may need to be performed. 

\item Remark: in general, the inference of hierarchical model consists of (1) inference of population level parameters $\phi|y$; (2) inference of group-level parameters: $\theta_j | \phi, y$. 
\end{itemize}

Hierarchical binomial model: rat tumor experiment [Section 5.3] 
\begin{itemize}
\item Model: the data consists of $J$ groups, with each group of size $n_j$, and $y_j$ is the number of rats survived. 
\begin{equation}
y_j \sim \text{Bin}(n_j, \theta_j)	
\end{equation}
where $\theta_j$ is the mean of the $j$-th group, and is assumed to follow a prior of Beta distribution: 
\begin{equation}
\theta \sim \text{Beta}(\alpha, \beta)
\end{equation}
We are interested in learning the general trend of the population (over all groups).

\item Inference: we first infer $\alpha$ and $\beta$. Using the marginal likelihood of binomial distribution: 
\begin{equation}
p(y|\alpha,\beta) = \prod_j p(y_j|\alpha,\beta) \propto \prod_j \frac{B(\alpha+y_j, \beta+n_j - y_j)}{B(\alpha,\beta)}
\end{equation}
And the posterior $p(\alpha,\beta|y) \propto p(\alpha,\beta) p(y|\alpha,\beta)$. We would use numerical method to compute this function for any value of $\alpha, \beta$. Once we have $\alpha, \beta$, the parameter of the $j$-th group: 
\begin{equation}
\theta_j |y_j, \alpha, \beta \sim \text{Beta}(\alpha + y_j, \beta + n_j - y_j)	
\end{equation}

\item Prior of $\alpha,\beta$: uniform prior in the scale of $(\alpha/(\alpha+\beta), (\alpha + \beta)^{-1/2})$. 
\end{itemize}

Hierarchical normal model:
\begin{itemize}
\item Model: consider $J$ groups, the goal is to estimate the mean of each group. Within each group, the data points are from iid sample: 
\begin{equation}
y_{ij} | \theta_j \sim N(\theta_j, \sigma^2) \qquad i = 1, 2, \cdots, n_j	
\end{equation}
The parameters $\theta_j$ are from a common distribution: 
\begin{equation}
\theta_j|\mu, \tau \sim N(\mu, \tau^2)	
\end{equation}
where $\theta_j$ are independent. Note that the parameter $\tau$ reflects the between-group variation, if it is close to 0, then complete-pooling; if it is approaching infinity, then no pooling. By exploring the distribution of $\tau$, one can assess the heterogeneity of groups. 

\item Classical approach: either complete pooling - use all population to estimate a single mean; or no pooling - estimate one mean for each group. To choose between the two, ANOVA would test if the means of groups (from no pooling) are significantly different (heterogeneity test). If significant, no pooling; otherwise, complete pooling. 

\item Bayesian approach: 
\begin{itemize}
\item The prior distribution of hyperparameters: choose noninformative prior: 
\begin{equation}
p(\mu, \tau) \propto p(\tau) \propto 1	
\end{equation}
Note that the prior $p(\tau) \propto 1/\tau$ leads to improper posterior (so in fact, the prior is determined after some experimentation). 

\item Joint posterior distribution: let $\bar{y_j}$ be the mean of group $j$, and $\sigma_j^2 = \sigma^2 / n_j$ be the sample variance of group $j$, then: 
\begin{equation}
p(\theta,\mu,\tau|y) \propto p(\mu, \tau) \prod_j N(\theta_j | \mu, \tau^2) \prod_j N(\bar{y_j} | \theta_j, \sigma_j^2)	
\end{equation}

\item Conditional posterior distribution given the hyperparameters $p(\theta|\mu,\tau,y)$: this is from the standard results of posterior distribution of normal means - normal distribution, with mean equal to the weighted average of $\bar{y_j}$ and $\mu$, and precision equal to the sum of precision (of prior and of the data). 

\item Marginal posterior distribution $p(\mu,\tau|y)$: 
\begin{equation}
p(\mu,\tau|y) \propto p(\mu,\tau) p(y|\mu,\tau)
\end{equation}
This can be achieved by marginalization of $\theta$ (similar to finding the posterior predictive distribution in normal distribution). 

\item Conditonal posterior distribution $p(\mu|\tau,y)$: from the equation above, $p(\mu|\tau)$ is normal (or uniform), and log of $p(y|\mu,\tau)$ is quadratic of $\mu$, thus this is normal distribution. 
u

\item Marginal posterior distribution $p(\tau|y)$: 
\begin{equation}
p(\mu,\tau|y) = p(\mu|\tau,y) p(\tau|y)	
\end{equation}
The first term is a normal distribution, and the second from integrating out $\mu$ (the integrand is exponential to the quadratic of $\mu$), or from $p(\mu,\tau|y) / p(\mu|\tau,y)$.  

\item Sampling: first sample $p(\tau|y)$ (e.g. 1-D grid sampling), then sample from $p(\mu|\tau,y)$ (normal), and then $p(\theta_j|\mu,\tau,y)$ (also normal). 
\end{itemize}

\item Example: the ETS data, $J$ schools, for each school, the improvement of student scores from a policy (one student per score). The goal is to estimate the treatment effect, i.e. whether the score improvement is significantly different from 0. 
\begin{itemize}
\item $p(\tau|y)$: maximum at $0$ and is very small at $\tau > 15$ points, meaning the within-group variation is small. However, still signficiant probability mass from 0 to 15, thus signficant uncertainty. 

\item $p(\theta_j | \tau, y)$: draw the distribution at different values of $\tau$. At $\tau = 0$: all groups have the same mean score $\theta_j$. The difference increases at larger $\tau$. Obtain standard deviation or posterior quantile of $\theta_j$, this allows one to analyze each school, e.g. for school $A$, the probability that the score effect is great than 28 is very small, even at large $\tau = 10$. 

\end{itemize}

\item Example: meta-analysis of drug effects. $J$ trials, at each trial, the death rate of the control group, and the treatment group. Estimate the treatment effect. Let $p_{1j}$ and $p_{0j}$ be the death probability in group $j$ of treatment and control respectively. For a single group, the hypothesis is to test $p_{1j} = p_{0j}$. The Bayesian approach models $p_{1j}$ and $p_{0j}$ as from common distributions. 

\item Comparison with non-Bayesian model: e.g. the ETS example [Section 5.5]. The problems with the non-Bayesian approach: 
\begin{itemize}
\item No pooling: estimate $\theta_j$ for each group separately, and then obtain $\mu$ by averaging over $\theta_j$. First, $\theta_j$ estimation is not accurate (because of small sample size per group); second, the average over $\theta_j$ does not weigh groups properly. 

\item Complete pooling: ignores variation across groups. Ex. suppose there is a very large group in the data, then the average (after complete pooling) is dominated by the average of this group, but this average is simply one random sapmle from the population mean. 
\end{itemize}

\end{itemize}

\subsection{Empirical Bayes}

Reference: [Efron, Large-Scale Inference: Empirical Bayes methods for estimation, testing and prediction, 2010], [Casella, An Introduction to Empirical Bayes Data Analysis, Am Stat, 1985]

Bayesian point estimation: 
\begin{itemize}
\item Strategy overview: instead of full posterior inference, we could performs point estimation, i.e. find a best value that ``summarizes'' the posterior distribution. This may make inference much simpler. 

\item Mean posterior estimator: one can find a number that best represent the posterior distribution of the unknown parameter. This could often be mean posterior (MP) estimator, or maximum a posterior (MAP) estimator. Example: normal distribution, the posterior distribution of the mean is also normal, thus the MP estimator is simply the mean of the normal distribution. 

\item Marginal distribution: the second main strategy for estimation. Suppose we want to infer $\phi$, with extra parameter $\theta$ in the model. We could derive the marginal likelihood function of $\phi$ by integrating out $\theta$: 
\begin{equation}
P(y|\phi) = \int P(y|\theta, \phi) d\theta
\end{equation}
This is important for cases like: nuisance parameters (e.g. variance parameter in normal distribution), hierarchical model (infer the population parameters by marginaling group parameters). 

\item Assessing estimators: suppose we have $\hat{\theta}(X)$ as our estimator of $\theta$, we could assess it by the mean squared error (MSE): 
\begin{equation}
\text{MSE} = \E(\theta - \hat{\theta}(X))^2	
\end{equation}
where the expectation is taken over $X$ over given $\theta$. The interpretation is thus the same as MSE for classical point estimation (with fixed $\theta$). Note that we could also define the overall Bayes risk of an estimator by averaing over the prior distribution of $\theta$. 

\item Remark: the two strategies of point estimation: MP estimator and marginal distribution can be used in the same problem for different parameters. In particular, one may need to estimate one set of parameters first, and then estimate the second set of parameters conditioned on the first set of parameters. 

\end{itemize}

Empirical Bayes overview: 
\begin{itemize}
\item Learning from the experience of others: the general idea is that in large-scale inference problems (simultaneous inference of multiple unknowns), there may be ``mysterious'' evidence lurking among the objects, and hierarchical Bayes modeling strategy could take advantage of that. For Empirical Bayes, the prior ``may exist only as a motivational device''. 

\item Example: baseball play, Table 1.1. in [Efron10]. We are estimating the rate of hits of players. Even though the relationship between all players is far from clear, the James-Stein (JS) estimator gives better estimate of the rates than MLE. 

\item EB strategy: the prior distribution is estimated from the data in contrast to the standard Bayesian approach. It may be viewed as an approximation to a fully Bayesian treatment of a hierarchical model wherein the parameters at the highest level of the hierarchy are set to their most likely values, instead of being integrated out. Also known as Maximum Marginal Likelihood. 
\end{itemize}

Normal hierarchical model and JS estimator:
\begin{itemize}
\item Model: suppose we have $n$ observations $z_i$ with mean $\mu_i$, and $\mu_i$ follows a prior normal distribution: 
\begin{equation}
\mu_i \sim N(0, A) \qquad z_i | \mu_i \sim N(\mu_i, 1)	
\end{equation}
where $A$ is a constant. Our goal is to infer $\mu_i$, trying to leverage all the data. The posterior distribution can be found easily: 
\begin{equation}
\mu_i | z_i \sim N(B z_i, B)	
\end{equation}
where $B = A / (A + 1)$. The Bayes point estimator is thus: 
\begin{equation}
\hat{\mu}^{\text{Bayes}} = B z = \left( 1 - \frac{1}{A + 1}\right) z	
\end{equation}

\item Evaluating an estimator by overall risk: to evaluate the estimator, we could use the mean squared error (MSE) loss function over all parameters, as in frequentist statistics. First, the loss function: 
\begin{equation}
L(\hat{\mu}, \mu) = \sum_i (\hat{\mu}_i - \mu_i)^2 = \norm{\hat{\mu} - \mu}^2
\end{equation}
Next, write $\hat{\mu} = t(z)$ as a function of data, then the expected squared error, over all possible data ($z$) is: 
\begin{equation}
R(\mu) = \E[L(\hat{\mu}, \mu)] = \E[\norm{t(z) - \mu}^2]
\end{equation}
For the MLE, $\hat{\mu}^{\text{MLE}} = z$, it's easy to show that 
\begin{equation}
R^{\text{MLE}}(\mu)	= N
\end{equation}
Plug in the Bayes estimator, we can show that: 
\begin{equation}
R^{\text{Bayes}}(\mu)	= (1 - B)^2 \norm{\mu}^2 + NB^2
\end{equation}
The MSE of an estimator depends on the true values of parameters. We could average MSE over the prior distribution of parameters, and this gives the overall Bayes riks. It's easy to see that: 
\begin{equation}
R^{\text{MLE}} = N \qquad R^{\text{Bayes}} = N \frac{A}{A + 1}	
\end{equation}
Thus the Bayes estimator is a better estimator than MLE when all parameters are estimated simultaneously. 

\item JS estimator: since $A$ is generally unknown, we want to estimate $A$ and replace it with the estimator in the Bayes estimator of $\mu$. We note that we only need to estimate $1 / (A +1)$. This could be obtained by the marginal distribution of $z$: 
\begin{equation}
z_i \sim N(0, (A + 1))	
\end{equation}
The problem is thus estimating the variance (acutally its inverse) from normal distribution. Let $S$ be the sample variance $\norm{z}^2$, and we have:
\begin{equation}
\E\left( \frac{N - 1}{S} \right) = \frac{1}{A+1}
\end{equation}
JS estimator is defined as: 
\begin{equation}
\hat{\mu}^{\text{JS}} = \left( 1 - \frac{N - 2}{S}\right) z		
\end{equation}
Its overall Bayes risk is: 
\begin{equation}
R^{\text{JS}} = N \frac{A}{A+1} + \frac{2}{A+1}
\end{equation}
which is slightly higher than the risk of Bayes estimator (given that the true $A$ is known), but still below the MLE risk. 

\item Theorem (The superiorty of JS estimator): for $N \geq 3$, the JS estimator everywhere dominates the MLE in terms of expected total squared error for every choice of $\mu$: 
\begin{equation}
\E_{\mu} \left[\norm{\hat{\mu}^{\text{JS}} - \mu}^2 \right]	< \E_{\mu}\left[\norm{\hat{\mu}^{\text{MLE}} - \mu}^2 \right]
\end{equation}
Note that this result does not require any prior belief about $\mu$, and is completely frequestist (not overall Bayes risk). To understand this theorem: 
\begin{itemize}
\item Note that in comparing Bayes estimator and MLE, $A$ can be any arbitrarily large number, so the fact that we have a prior distribution of $\mu_i$ does not really matter, i.e. the superiorty should be true regardless of our prior belief. 
\item When three or more unrelated parameters are measured, their total MSE can be reduced by using a combined estimator such as the JS estimator. 
\end{itemize}

%\item Inference of $\mu$ and $\tau$: the marginal distribution of $X_i$, by marginalizing $\theta_i$, is:
%\begin{equation}
%f(x_i) \sim N(\mu, \sigma^2 + \tau^2)	
%\end{equation}
%We could then form the estimator of $\mu$ and $\tau$ by sample mean and sample variance. 
%
%\item Inference of $\theta_i$: under given $\mu$ and $\tau$, we could easily obtain the MP estimator of $\theta_i$: 
%\begin{equation}
%\hat{\theta}_i = \frac{\sigma^2}{\sigma^2 + \tau^2} \mu + \frac{\tau^2}{\sigma^2 + \tau^2} x_i
%\end{equation}
%Since $\mu$ and $\tau$ are unknown, we will replace them with their estimators. Note that since we know the estimator of $\tau^2$, we could use that to derive the estimator of $\sigma^2 / (\sigma^2 + \tau^2)$. The result is: 
%\begin{equation}
%\hat{\theta}_i = \frac{(p - 3) \sigma^2}{\sum_i (x_i - \bar{x})^2} \bar{x} + 	\left[1 - \frac{(p - 3) \sigma^2}{\sum_i (x_i - \bar{x})^2}\right] x_i
%\end{equation}
\end{itemize}

Problems and extensions of EB estimators: 
\begin{itemize}
\item Tyranny of the majority: with EB approach, the estimation of an outlier, may be influenced largely by the mean of multiple related objects. This could be a problem if the object is truely an outlier (e.g. a trully distinguished baseball player). One idea to deal with this problem is to use JS estimation subject to the restriction of not deviating too far from the MLE (limited translation estimator). 

\item Application to the regression setting: the prior distribution of $\mu_i$ may not be simple, but depends on other parameters. Example: 
\begin{equation}
\mu_i \sim N(M_0 + M_1 \cdot \text{age}_i, A) \qquad z_i | \mu_i \sim N(\mu_i, \sigma_0^2)	
\end{equation}
So our estimation of $\mu_i$ is based not only on $z_i$, but also other players with the age effect corrected for. 

\end{itemize}

False Discovery Rates, A New Deal (ASH) [Stephens, Biostatistics, 2016]
\begin{itemize}
	\item Motivation: two main limitations of existing approaches to FDR:
	\begin{itemize}
		\item Zero assumption: most would assume that near $p = 1$ or $Z = 0$, the observed test statistics are from $H_0$. This leads to overestimation of $\pi_0$. This is necessary so that the model is identifiable, e.g. in Efron's approach. 
		\item Different power of different tests: e.g. different MAF in GWAS or different coverage in RNA-seq. Thus p-values mean different things for different tests. Mixing them reduce the power (for high-signal tests). 
	\end{itemize}
	
	\item Idea: model the underlying effects as a mixture (unimodal), and including the standard error (thus taking power into account - higher powered test will have smaller standard error). 
	
	\item Model: we observed $\hat{\beta}_j$, assuming it follows distribution: $N(\beta_j, s_j^2)$ where $s_j$ is the standard error (known). So our likelihood is $L(\beta_j) \propto \exp((\hat{\beta}_j - \beta_j)^2/s_j^2)$. The prior of $\beta_j$ is: 
	\begin{equation}
	\beta_j \sim \pi_0 \delta_0(\cdot) + (1-\pi_0) g(\cdot)
	\end{equation}
	where $\delta_0$ is Dirac's delta function and $g(\cdot)$ needs to be infered. A convenient way to model $g$ is a mixture of normal:
	\begin{equation}
	g(\beta; \pi) = \sum_k \pi_k N(\beta; 0, \sigma_k^2)
	\end{equation}
	Another option is to use uniform prior: 
	\begin{equation}
	g(\beta; \pi) = \sum_k \pi_k U(\beta; a_k, b_k)
	\end{equation}
	where $U(a_k, b_k)$ is the uniform distribution on $[a_k, b_k]$. Or more generally, $g(\beta,\pi) = \sum_k \pi_k f_k(\beta)$. In both cases, we choose a large number of components in the mixture - very dense grid. We estimate $\pi_k$ and $\sigma_k$ through Empirical Bayes (EM algorithm). Once we estimate these parameters, we can infer the posterior of $\beta_j$: which will be shrinked towards 0. This method is thus called ``adaptive shrinkage'' (ASH). 
	
	\item Penalty of $\pi_0$: let $l(\pi)$ be the log-likelihood function. We add a penalty term: $h(\pi; \lambda) = \prod_{k=0}^K \pi_k^{\lambda_k-1}$. Default option: $\lambda_k = 1$ for all $k > 0$, and $\lambda_0 = 10$. This would introduce no penalty for $\pi_k$'s, but a penalty for $\pi_0$ to encourage a large $\pi_0$. This is motivated by Dirichlet density, though we do not explicitly use prior of $\pi$. 
	
	\item Inference: we maximize the penalized log-likelihood: 
	\begin{equation}
	l(\pi) = \log P(\hat{\beta}|s, \pi) + \log \pi_k^{\lambda_k-1}
	\end{equation}
	Next we describe the marginal likelihood:
	\begin{equation}
	p(\hat{\beta_j}|s, \pi) = \sum_{k=0}^K \pi_k \tilde{f}_k(\hat{\beta_j})
	\end{equation}
	where 
	\begin{equation}
	\tilde{f}_k(\hat{\beta_j}) = \int_{\beta_j}f_k(\beta_j) N(\hat{\beta_j}|\beta_j, s_j^2) d\beta_j
	\end{equation}
	When we use normal prior, we have: 
	\begin{equation}
	\tilde{f}_k(\hat{\beta_j}) = N(\hat{\beta_j}|0, s_j^2 + \sigma_k^2)
	\end{equation}
	When we use uniform prior, we can show that the marginal is CDF of normal (integration of normal density over an interval): 
	\begin{equation}
	\tilde{f}_k(\hat{\beta_j}) = \frac{1}{b_k - a_k} \left[ \Psi\left( \frac{\hat{\beta_j} - a_k}{s_j}\right) - \Psi\left( \frac{\hat{\beta_j} - b_k}{s_j}\right) \right]
	\end{equation}
	Maximization is simple: the function is convex, so we can use convex optimization method (e.g. Interior Point method) or EM. The EM alternates between two steps: let $\phi_{jk}$ be the posterior probability that data point $j$ belongs to the $k$-th component, and $L_{jk}$ be the probability data point $j$ is from component $k$, then:
	\begin{itemize}
		\item E-step: $\phi_{jk} = \pi_k L_{jk} / \sum_l \pi_l L_{jl}$.  
		\item M-step: $\pi_k = \sum_{j} \phi_{jk} / p$, where $p$ is the number of tests.  
	\end{itemize}
	
	\item Application to the estimation problem: when we only estimate $\beta_j$, we may not need the special component $\pi_0 \delta_0$. Or more generally, we can view this as a special case of normal or uniform prior.  
	
	\item Model behavior: intuition of the shrinkage effect. Suppose we have a large $\hat{\beta}_j$, and population mean at 0. The estimate of $\beta_j$ will be shrinked towards 0: how much it will be shrinked, i.e. the weights of data ($\hat{\beta}_j$) and prior, depends on the local density of $g(\cdot)$. If this is large near $\hat{\beta}_j$, then we will have small shrinkage. To apply this to the uniform prior case, if many intervals contain $\hat{\beta}_j$, then the prior density is high in the neighborhood, and we will small degree of shrinkage. 
	\begin{itemize}
		\item Remark: The problem of unstable estimate of $\pi$: when the number of components is large, it is probably hard to get a stable estimate of $\pi$, especially, $\pi_k$ of adjacent components would be unidentifiable. This is likely not a main problem, however, as our goal is to obtain the shape of $g(\cdot)$ and the posterior of $\beta_j$, which should not be sensitive to the exact values of $\pi$. 
	\end{itemize}
	
	\item Local false sign rate (lfsr): when most of the tests are not null, FDR will be small, even for those observations with $p$-value $\approx 1$. The reason is that FDR cannot distinguish $\beta = 0$ and $\beta$ ''very small''. The solution is to use local FSR: the probability that we mis-claim the sign of the effect size. This is based on the posterior of $\beta$. For example, suppose $P(\beta > 0|\hat{\beta}) = 0.9$ is large, we would claim that it is positive, then our probability of being wrong is $P(\beta \leq 0|\hat{\beta}) = 0.1$. More generally, our rule is to choose: 
	\begin{equation}
	\max\{P(\beta > 0|\hat{\beta}), P(\beta < 0|\hat{\beta})\}
	\end{equation}
	Our probability of error is thus the minimum of the two possible errors: 
	\begin{equation}
	lfsr_r := \min\{P(\beta \leq 0|\hat{\beta}), P(\beta \geq 0|\hat{\beta})\}
	\end{equation}
	The case where we decide to choose $P(\beta= 0 |\hat{\beta})$ is just LFDR. However, FDR can be underestimated (as explained previously, e.g. most alternative models have small, but non-zero effects). We have: $lfsr_j \geq lfdr_j$. FSR is more robust to model specificiations than FDR. We can also define aggregate measure of FSR at a certain thresold of lfsr, just as in the Bayesian FDR case. 
	
	\item ASH has different behavior than current methods \texttt{qvalue} or \texttt{locfdr} (Figure 1): a simulation with only true effects (mean 0), FDR methods produce a hole in the $H_1$ component because it assumes $Z$-scores near 0 are from $H_0$. ASH does not have this behavior, because it has explicit alternative distribution with mean 0. 
	
	\item Estimation of $\pi_0$ by ASH: simulation under different $g(\cdot)$, e.g. spiky, flat or bimodal (Figure 2A). $\pi_0$ estimates of ASH are generally conservative, except the case of bi-modal (Figure 2B). They are more accurate than FDR methods, which are too conservative. 
		
	\item LFSR is less conservative than LFDR (Figure 2C): FDR methods are sensitive to $\pi_0$, since $\pi_0$ tends to be significantly overestimated, LFDR can be over-estimated as well. LFSR is a better metric: it is much closer to true LFSR. 
	
	\item Estimation of $g(\cdot)$: ``post-selection'' interval estimate is extremely desirable, however, it is hard to achieve this with frequentist paradigm. 
	
	\item Calibration of posterior interval: generally close to 0.95. 
	
	\item Remark/Questions: 
	\begin{itemize}
		\item How the model is sensitive to the parameter $\lambda_0$? The value of 10 seems arbitrary. Comparing this way of estimating $\pi_0$ vs. William Wen's model, which uses the fact that BF under $H_0$ has expectation 1. 
		
		\item Generalize to other likelihood model: e.g. in RVAT, we are testing if $\sigma = 0$. Or the data has some dependence, e.g. HMM or LD. 
		
		\item Multi-parameter cases: e.g. suppose we analyze two GWAS traits simultaneously, we need a prior for $P(\beta_{1j}, \beta_{2j})$ for two traits. 
		
	\end{itemize}
\end{itemize}
%%%%%%%%%%%%%%%%%%%%%%%%%%%%%%%%%%%%%%%%%%%%%%%%%%%%%%%%%%%%
%%%%%%%%%%%%%%%%%%%%%%%%%%%%%%%%%%%%%%%%%%%%%%%%%%%%%%%%%%%%
\chapter{Basic Probabilistic Methods}
\section{Multivariate Normal Distribution}

\subsection{Properties of Multivariate Normal Distribution (MVN)} 

Reference: [Bishop, Pattern Recognition and Machine Learning, 2.3]
 
Definition of multivariate normal distribution: 
\begin{itemize}
\item Background: Gaussian integral at 1D is given by: 
\begin{equation}
\int_{- \infty}^{+ \infty} e^{-x^2} dx = \sqrt{\pi}	
\end{equation}
The proof follows variable substitution (using polar coordinates). Its general form in $n$-D is: 
\begin{equation}
\int \exp \left( -\frac{1}{2} x^T A x \right) dx = \sqrt{\frac{(2 \pi)^n}{\det A}}
\end{equation}

\item Definition: a random vector $X = (X_1, \cdots, X_D)$ follows $N(\mathbf{\mu}, \Sigma)$ if:  
\begin{equation}
p(\mathbf{x}) = \frac{1}{(2 \pi)^{D/2}} \frac{1}{\det \Sigma^{1/2}} \exp\left[ -\frac{1}{2} (\mathbf{x} - \mathbf{\mu})^T \Sigma^{-1} (\mathbf{x} - \mathbf{\mu}) \right]
\end{equation}
where $x$ is a column vector and $\Sigma$ is a (symmetric) positive definite matrix, i.e. all eigenvalues of $\Sigma$ are positive (otherwise, the PDF is not properly normalized). The exponent is the quadratic form. So it is equivalent to say any distribution in the above form, where the exponent can be written in quadratic form (with both quadratic and linear terms) is a Gaussian distribution. 

\item Interpretations: the Gaussian distribution follows from the average of multiple i.i.d. random variables according to the Central Limit Theorem. In addition, it is the distribution that maximizes the entropy among all continuous random variable with finite first and second moments (could be multivariate): 
\begin{equation}
H(X) = - \int{p(x) \ln p(x)}	
\end{equation}
\end{itemize}

Representation and interpretation of multivariate Gaussian distribution: 
\begin{itemize}
\item Diagonalization: the matrix $\Sigma$ is real and symmetric, thus according to the Spectral Theorem, $\Sigma$ has the Eigen Decomposition. Let $\mathbf{u_i}$ be the $i$-th eigenvector of $\Sigma$, and $\lambda_i$ be the eigenvalue, then: 
\begin{equation}
\Sigma = U^T \cdot \text{diag}(\lambda_1, \cdots, \lambda_D) \cdot U	
\end{equation}
where $U$ is a matrix whose rows are given by the vector $\mathbf{u_i}^T$. Plug in this equation into the quadratic form, and let $\mathbf{y} = U (\mathbf{x} - \mathbf{\mu})$, we have: 
\begin{equation}
\label{eq:MVN_eigen}
(\mathbf{x} - \mathbf{\mu})^T \Sigma^{-1} (\mathbf{x} - \mathbf{\mu}) = \sum_i y_i^2 / \lambda_i	
\end{equation}

\item Interpretation: this equation implies that the random variable $\mathbf{y}$ follows $D$ independent Gaussian distribution in each dimension, where the dimensions are defined by the orthogonal eigenvectors of $\Sigma$. The PDF of $\mathbf{y}$ is given by the theorem of variable transformation (Jacobian is equal to 1 since $U$ is an orthogonal matrix): 
\begin{equation}
p(\mathbf{y}) = p(\mathbf{x}) \det J = \prod_{j=1}^D \frac{1}{(2\pi \lambda_j)^{1/2}} \exp\left( -\frac{y_j^2}{2\lambda_j}\right)	
\end{equation}
Thus $\lambda_j$ is the covariance of $Y_j$. The contour plot of $\mathbf{x}$ is an ellipse centered on $\mathbf{\mu}$ in 2D, with two axis given by $y_1$ and $y_2$. When $\Sigma$ is diagonal, the axis are parallel to the axis of $\mathbf{x}$ (Figure 2.7). 

\item Alternative interpretation: suppose, suppose $x$ follows MVN with covariance $\Sigma$. We can think view $x$ as linear combination of independent RVs, i.e. the covariance between $x$ is due to the linear combination. Ex. if $z_1, z_2$ are independent with variance $\sigma_1^2$ and $\sigma_2^2$, then $x_1 = z_1 + z_2$ and $x_2 = z_1 - z_2$ are generally not independent: 
\begin{equation}
\Cov(x_1, x_2) = \Cov(z_1 + z_2, z_1 - z_2) = \sigma_1^2 - \sigma_2^2
\end{equation}
Formally, we can write $x = U^T y$, where $U$ is the matrix in the EVD of $\Sigma$ and $y$ independent RVs. A special case is the result used in sampling MVN: if $Z_i$ iid $N(0,1)$, and $A$ is the Cholesky decomposition of $\Sigma$, then $X = AZ$ follows $N(0, \Sigma)$. 

\item Regression perspective: MVN can also be viewed from regression. We can treat one variable, say $x_1$ as response variable. The fact that $x_1$ corrleates with other variables means that we can view $x_1$ as a linear model of other variables. 

\item Relation to factor analysis: the idea of writing MVN as linear combination of independent RVs is similar to PCA and factor analysis in general. 
\end{itemize}

Moments of multivarirate Gaussian distribution: 
\begin{itemize}
\item Normalization constant: the integral of the distribution in the $\mathbf{y}$ coordinate system is 1 by multiplying the integral along each dimension (using Gaussian integral in each dimension). 

\item Expectation: replace $\mathbf{z} = \mathbf{x} - \mathbf{\mu}$:
\begin{equation}
E[\mathbf{x}] = \frac{1}{(2\pi)^{D/2}} \frac{1}{\det \Sigma^{1/2}} \int \exp \left[ -\frac{1}{2} \mathbf{z}^T \Sigma^{-1} \mathbf{z} \right] (\mathbf{z} + \mathbf{\mu}) d\mathbf{z}
\end{equation}
The term $\mathbf{z}$ in $\mathbf{z} + \mathbf{\mu}$ vanishes because of symmetry, and so: $E[\mathbf{x}] = \mathbf{\mu}$. 

\item Covariance matrix: defined as the following matrix: 
\begin{equation}
\text{Cov}(X) = E[(\mathbf{x} - E(\mathbf{x})) (\mathbf{x} - E(\mathbf{x}))^T]	
\end{equation}
We need to compute $E(\mathbf{x} \mathbf{x}^T)$. Let $\mathbf{z} = \mathbf{x} - \mu$, and plug in to the equation of $E(\mathbf{x} \mathbf{x}^T)$, we have: 
\begin{equation}
E(\mathbf{x} \mathbf{x}^T) = E(\mathbf{z} \mathbf{z}^T)	+ \mu \mu^T
\end{equation}
The first term can be computed using diagonalization of $\Sigma$: $\mathbf{y} = U \mathbf{z}$: 
\begin{equation}
E(\mathbf{z} \mathbf{z}^T) = \frac{1}{(2\pi)^{D/2}} \frac{1}{\det \Sigma^{1/2}} \sum_{i,j} \mathbf{u_i} \mathbf{u_j}^T \int \exp \left( -\sum_k \frac{y_k^2}{2\lambda_k}\right) y_i y_j d\mathbf{y} = \sum_i \mathbf{u_i} \mathbf{u_i}^T \lambda_i = \Sigma
\end{equation}
where the $i \neq j$ terms vanish because of symmetry. So we have the covariance matrix as: $\text{Cov}(X) = \Sigma$. 
\end{itemize}

Linear normal distributions: based on [Bishop] Eq 2.113-2.117, but with different notations. 
\begin{itemize}
	\item Property 1: if $y|x \sim N(x, \Sigma)$ and $x \sim N(\mu, \Lambda)$, we have: $y \sim N(\mu, \Lambda + \Sigma)$. 
	
	\item Property 2: if $y|x \sim N(A x + b, \Sigma)$ for a matrix $A$, and $x \sim N(\mu, \Lambda)$, the marginal distribution is: $y \sim N(A x + b, A \Lambda A^T + \Sigma)$, and the conditional distribution is: 
	\begin{equation}
	x | y \sim N(V(A^T \Sigma^{-1}(y-b) + \Lambda^{-1} \mu), V), \qquad V = (\Lambda^{-1} + A^T \Sigma^{-1} A)^{-1}
	\end{equation}
	An important special case is: if $x \sim N(\mu, \Sigma)$, then for a matrix $A$, we have $Ax \sim N(A \mu, A \Sigma A^T)$. Note: this can be easily proved using the fact that: $\Var(Ax) = A \cdot \Var(x) \cdot A^T$. 
	
	\item Remark: the interpretation of the marginal distribution, the variance of $y$ has two components: (1) $\Sigma$, which is given by $y$ itself (when $x$ is given), and (2) $A \Lambda A^T$, which is introduced by $x$: the variance of $x$, $\Lambda$ is scaled by $A$. 
\end{itemize}

Marginal and conditional distributions: a joint distribution $N(\mathbf{\mu}, \Sigma)$ with precision matrix $\Lambda = \Sigma^{-1}$, and $\mathbf{x} = (\mathbf{x_a}, \mathbf{x_b})^T$ be some partition of all dimensions.  
\begin{itemize}
\item ``Completing the square'' technique: we could write the exponent of a general Gaussian distribution as: 
\begin{equation}
-\frac{1}{2} (\mathbf{x} - \mathbf{\mu})^T \Sigma^{-1} (\mathbf{x} - \mathbf{\mu}) = 	-\frac{1}{2} \mathbf{x}^T \Sigma^{-1} \mathbf{x} + \mathbf{x}^T \Sigma^{-1} \mu + \text{const}
\end{equation}
Thus the second order term in $\mathbf{x}$ has coefficient $\Sigma^{-1}$, and the linear term in $\mathbf{x}$ has coefficient $\Sigma^{-1} \mathbf{\mu}$. For a distribution of interest, write its first and second order terms, and this would allow one to determine mean and covariance. 

\item Relationship between covariance and precision matrix: 
\begin{equation}
\Sigma_{aa} = (\Lambda_{aa}	- \Lambda_{ab} \Lambda_{bb}^{-1} \Lambda_{ba})^{-1}
\end{equation}

\item Joint $\rightarrow$ marginal distribution: $\mathbf{x_a}$ follows Gaussian distribution $N(\mu_a, \Sigma_{aa})$ where $\Sigma_{aa}$ is the corresponding submatrix of $\Sigma$. 

\item Joint $\rightarrow$ conditional distributions: expand the exponent of the Gaussian distribution into $\mathbf{a}$ and $\mathbf{b}$ parts, and apply the ``completing the square'' technique. This leads to $\mathbf{x_a}|\mathbf{x_b}$ follows normal distribution with: 
\begin{equation}
\mu_{a|b} = \mu_a - \Lambda_{aa}^{-1} \Lambda_{ab} (\mathbf{x_b} - \mu_b)
\end{equation}
\begin{equation}
\Lambda_{a|b} = \Lambda_{aa}^{-1}	
\end{equation}
Alternative, state in terms of covariance matrix: 
\begin{equation}
\mu_{a|b} = \mu_a + \Sigma_{ab} \Sigma_{bb}^{-1} (\mathbf{x_b} - \mu_b)  
\end{equation}
\begin{equation}
\Sigma_{a|b} = \Sigma_{aa} - \Sigma_{ab} \Sigma_{bb}^{-1}	\Sigma_{ba}
\end{equation}
Note: $\mu_{a|b}$ is linear function of $\mathbf{x_b}$, and the covariance matrix is independent of the value of $\mathbf{x_b}$ (Figure 2.9). 

\item Conditional $\rightarrow$ joint distribution: suppose we have: 
\begin{equation}
p(x) = N(x| \mu, \Lambda^{-1})
\end{equation}
\begin{equation}
p(y|x) = N(y|Ax + b, L^{-1})	
\end{equation}
We have the joint distribution: 
\begin{equation}
\left( \begin{array}{l} x\\ y \end{array} \right)	\sim N\left( \left( \begin{array}{c} \mu\\ A \mu + b \end{array} \right), \Sigma \right)
\end{equation}
where 
\begin{equation}
\Sigma = \left( \begin{array}{cc} \Lambda^{-1} & \Lambda^{-1} A^T \\ A \Lambda^{-1} & L^{-1} + A \Lambda^{-1} A^T \end{array} \right)
\end{equation}
See [Gelman04] (Apppendix A) for $\Sigma$ in terms of covariance matrix. The margininal distribution of $\mathbf{y}$ (also known as compound distribution) is given by: 
\begin{equation}
p(y) = N(y|A \mu + \mathbf{b}, L^{-1} + A \Lambda^{-1} A^T)	
\label{eq:MVN_marginal}
\end{equation}
The conditional distribution $x|y$ is given by: 
\begin{equation}
p(x|y) = N(x|\Sigma_{x|y} \left[ A^T L (y-b) + \Lambda \mu\right], \Sigma_{x|y})
\end{equation}
where 
\begin{equation}
\Sigma_{x|y} = (\Lambda + A^T L A)^{-1}	
\end{equation}

\item A special case of the linear model: if $\mathbf{x}$ is $N(\mu, \Sigma)$, and $\mathbf{y} = A \mathbf{x} + \mathbf{b}$, then $\mathbf{y} \sim N(A \mu + \mathbf{b}, A \Sigma A^T)$. 
\end{itemize}

Bivariate normal distribution: [KNNL, chapter 2]
\begin{itemize}
\item Probability density function: consider the normal distribution with mean $(\mu_1, \mu_2)$ and covariance matrix: 
\begin{equation}
\Sigma = \left(
\begin{array}{ll}
\sigma_1^2 & \rho \sigma_1 \sigma_2\\
\rho \sigma_1 \sigma_2 & \sigma_2^2
\end{array}
\right)	
\end{equation}
Its pdf. is given by: 
\begin{equation}
f(x_1,x_2) = \frac{1}{2 \pi \sigma_1 \sigma_2 \sqrt{1 - \rho^2}} \exp\left\{-\frac{1}{2(1-\rho^2)} \left[ \left(\frac{x_1 - \mu_1}{\sigma_1}\right)^2 -2\rho \left(\frac{x_1 - \mu_1}{\sigma_1}\right)\left(\frac{x_2 - \mu_2}{\sigma_2}\right)  + \left(\frac{x_2 - \mu_2}{\sigma_2}\right)^2 \right] \right\}	
\end{equation}
The parameter $\rho$ is the correlation coefficient between $X_1$ and $X_2$: let $\sigma_{12} = \text{Cov}(X_1,X_2)$, then
\begin{equation}
\rho = \frac{\sigma_{12}}{\sigma_1 \sigma_2}	
\end{equation}

\item Conditional probability distribution $X_2|X_1$: this is normal distribution with mean $\alpha_{2|1} + \beta_{2|1} X_1$ and standard deviation $\sigma_{2|1}$ with: 
\begin{equation}
\begin{array}{ll}
\alpha_{2|1} & = \mu_2 - \mu_1 \rho \sigma_2/\sigma_1 \\
\beta_{2|1} & = \rho \sigma_2/\sigma_1\\
\sigma_{2|1}^2 & = \sigma_2^2 (1 - \rho^2)
\end{array}	
\end{equation}

\item Parameter estimation: the MLE of $\rho$ is: 
\begin{equation}
r = \frac{\sum (X_{1i} - \bar{X}_1)(X_{2i} - \bar{X}_2)}{\sqrt{\sum (X_{1i} - \bar{X}_1)^2 \sum (X_{2i} - \bar{X}_2)^2}}	
\end{equation}

\end{itemize}

\subsection{Inference of MVN}

Sample variance, sample covariance and geometrical interpreations: 
\begin{itemize}
\item Sample variance of one random variable: Suppose $X$ is a random variable with sample $x_1, \cdots, x_n$, then: 
\begin{equation}
\hat{\Var}(X) = \frac{1}{n-1} \sum_i (x_i - \bar{x})^2 = \frac{1}{n-1} (\mathbf{x} - \bar{x})^T  (\mathbf{x} - \bar{x})	
\end{equation}
It is the norm of the vector $\mathbf{x} - \bar{x}$ in the $n$-dim. space (up to the constant $1/(n-1)$). If $X$ is centered, then we have the simple relation: the sample variance of $X$ is simply the norm of the sample vector in the Eucledian space (up to the constant). 

\item Sample covariance between two RVs: Suppose $X$ and $Y$ are two random variables, with sample $x_1, \cdots, x_n$ and $y_1, \cdots, y_n$, respectively, then: 
\begin{equation}
\hat{\Cov}(X,Y) = \frac{1}{n-1} \sum_i (x_i - \bar{x}) (y_i - \bar{y}) = \frac{1}{n-1} (\mathbf{x} - \bar{x})^T  (\mathbf{y} - \bar{y})	
\end{equation}
It is the inner product of the $\mathbf{x} - \bar{x}$ and $\mathbf{y} - \bar{y}$ (up to constant). Similarly, if $X$ and $Y$ are centered, we have: the sample covariance is simply the dot product of the two sample vectors in Eucledian space. We could write the inner product as: 
\begin{equation}
\hat{\Cov}(X,Y) = \norm{\mathbf{x} - \bar{x}} \norm{\mathbf{y} - \bar{y}} \cos(\theta) = \sqrt{\hat{\Var}(X) \hat{\Var}(Y)} \cos(\theta)
\end{equation}
Thus the angle between the two sample vectors is simply the (sample) correlation coefficient of the two random variables. 

\item Sum of random variables: suppose $X = Y + Z$ where $Y$ and $Z$ are independent RVs, then $\Var(X) = \Var(Y) + \Var(Z)$, in terms of sample variance, we have: 
\begin{equation}
S_X = S_Y + S_Z	
\end{equation}
where $S_X$, $S_Y$ and $S_Z$ are sample variance of $X$, $Y$ and $Z$ respectively. Geometrically, since sample variance is the norm of the vector representing the RV, this is essentially the Pythagorean Theorem, since $Y$ and $Z$ are independent (thus the two vectors are orthogonal). 

\item Sample covariance matrix (sample covariance of random vector): given a $N \times p$ data matrix, let $\mathbf{x_i}$ be the $i$-th data point (row vector) and $X_j$ be the $j$-th random variable ($j$-th column). The covariance between $X_j$ and $X_k$ is given by: 
\begin{equation}
\hat{\Cov}(X_j,X_j) = \frac{1}{N-1} (X_j - \bar{X_j})^T (X_k - \bar{X_k})
\end{equation}
where $\bar{X_j}$, $\bar{X_k}$ are the sample means. The sample covariance matrix is thus given by: 
\begin{equation}
\hat{\Cov}(X) = \frac{1}{N-1} \left( \begin{array}{l}
X_1^T - \bar{X_1}\\
X_2^T - \bar{X_2}\\
\cdots\\
X_p^T - \bar{X_p} 
\end{array}	\right) \cdot (X_1 - \bar{X_1} \cdots X_p - \bar{X_p}) = \frac{1}{N-1} (X - \bar{x})^T (X - \bar{x})
\end{equation}
where $\bar{x} = (\bar{X_1}, \cdots, \bar{X_p})$ is the mean vector. The proof simply follows the results for two RVs. When the matrix is standardized, we have sample mean is 0 for every variable, thus: 
\begin{equation}
\hat{\Cov}(X) = \frac{1}{N-1} X^T X	
\end{equation}

\item Sample covariance matrix in terms of data points: using the equation above, but write the data matrix in terms of data points:  
\begin{equation}
\hat{\Cov}(X) = \frac{1}{N-1} \sum_{i=1}^{N} (\bf{x_i - \bar{x}})^T (\bf{x_i - \bar{x}})
\end{equation}
The interpretation is simple, the basic equation of $S$ applies to sample of any data points, suppose we have only 1 data point, we would have an estimate of $\Sigma$, now with $n$ data points, we simply take the average. 
\end{itemize}

MOM estimators: 
\begin{itemize}
\item Simiilar to the UNV case, we have sample mean and sample covariance as unbiased estimators:  
\begin{equation}
\E(\bar{X}) = \mu \qquad \E(S) = \Sigma	
\end{equation}

\item Proof: show that the expectation of the $jk$-th element of the sample covariance matrix is equal to $\Sigma_{jk}$, using the results from univariate (for diagonal terms) and bivariate normal distributions (non-diagonal terms).  

\end{itemize}

ML parameter estimation: [Matrix calculus and MLE for the multivariate Normal, Berkeley CS 281A notes]
\begin{itemize}
\item Log-likelihood function: Given a sample consisting of $n$ independent observations $\bf{x_1}, \cdots, \bf{x_n}$ of $d$-dimensional MVN $N(\mu, \Sigma)$, the log-likelihood function is given by: 
\begin{equation}
l(\mu,\Sigma) = \log P(\mathbf{x}|\mu, \Sigma) = - \frac{nd}{2} \log(2\pi) - \frac{n}{2} \log |\Sigma| - \frac{1}{2} \sum_{i=1}^n (x_i - \mu)^T \Sigma^{-1} (x_i - \mu)
\end{equation}
where $|.|$ is the determinant. The last term can be rewritten by using the invariance of cyclic permutations of matrix trace, we have: 
\begin{equation}
l(\mu,\Sigma) = \log P(\mathbf{x}|\mu, \Sigma) = - \frac{nd}{2} \log(2\pi) - \frac{n}{2} \log |\Sigma| - \frac{n}{2} \tr(\Sigma^{-1} S)
\end{equation}
where $S$ is the sample covariance matrix: 
\begin{equation}
S = \frac{1}{N} \sum_i (x_i - \mu) (x_i - \mu)^T	
\end{equation}

\item MLE of $\mu$: take the derivative wrt. $\mu$ and apply the result of the derivative of the quadratic form: 
\begin{equation}
\begin{array}{lll}
\frac{\partial l}{\partial \mu} & = & -\frac{1}{2} \sum_i \frac{\partial}{\partial \mu} \left[ (x_i - \mu)^T \Sigma^{-1} (x_i - \mu)\right]\\
 & = & -\frac{1}{2} \sum_i (\mu - x_i)^T 2 \Sigma^{-1} \\
\end{array}	
\end{equation}
Solving the equation: 
\begin{equation}
\sum_i (\mu - x_i)^T \Sigma^{-1} = 0	
\end{equation}
We have: 
\begin{equation}
\hat{\mu} = \frac{1}{n} \sum_i x_i
\end{equation}
The unbiased estimator would replace $n$ by $n-1$. 

\item MLE of $\Sigma$: take the derivative wrt. $\Sigma^{-1}$ of the second form of log-likelihood above (the trace form), and use the results of matrix derivatives (the derivative of $\tr(AB)$ and of $\log |A|$): 
\begin{equation}
\frac{\partial l}{\partial \Sigma^{-1}}	= \frac{n}{2} \Sigma - \frac{nS}{2} 
\end{equation}
Thus we have: 
\begin{equation}
\hat{\Sigma} = S = 	\frac{1}{n} \sum_i (x_i - \hat{\mu}) (x_i - \hat{\mu})^T
\end{equation}

\item Remark: in MLE, the scale constant is $n^{-1}$ vs. the MOM estimators, the constants is $(n-1)^{-1}$. 
\end{itemize}

Distribution of sample mean and sample covariance: [Manchester MT3732 class notes; Anderson, An Introduction to Multivariate Statistical Analysis, 3ed]
\begin{itemize}
\item Distribution of sample mean: $\bar{X} \sim N(\mu, \Sigma/n)$. \\
Proof: to obtain the covariance matrix of $\mu_j$, we show the general result: if $X$ and $Y$ are two MVN variables with iid. $N(\mu, \Sigma)$, then we have $\Cov(X+Y) = 2 \Sigma$. Apply this result to the variance of $\bar{X}$. 

\item Wishart distribution: given $X_i, 1 \leq i \leq n$ iid. $p$-dim. MVN $N(0, \Sigma)$, let $X$ be the $n \times p$ data matrix, and $M = X^T X$ be $p \times p$ matrix. Then $M$ has Wishart distribution with scale matrix $\Sigma$ and dof. $n$: 
\begin{equation}
M \sim W_p(\Sigma,n)	
\end{equation}
The Wishart distribution has the following properties: 
\begin{equation}
\E(M) = n \Sigma	
\end{equation}
\begin{equation}
\Var(M_{ij}) = n (\Sigma_{ij}^2 + \Sigma_{ii} \Sigma_{jj})	
\end{equation}
When $p = 1$ and $\Sigma = 1$, this is the $\chi^2$ distribution with dof. equal to 1. 

\item Distribution of sample covariance: for MVN $N(\mu,\Sigma)$, the scaled sample covariance matrix follows Wishart distribution: 
\begin{equation}
(n - 1) S \sim W_p(\Sigma, n - 1)	
\end{equation}
Furthermore, $\bar{X}$ and $S$ are independent. 

\end{itemize}

Testing parameters: 
\begin{itemize}
\item Testing the mean of individual variable: suppose we want to test $H_0: \mu_j = \mu_{j0}$. From the distribution of $\bar{X}$, we could obtain the marginal distribution of $\bar{X_j} \sim N(\mu_j, \sigma_j^2)$, where $\sigma_j^2$ is the variance of $X_j$. Replacing the variance with sample variance, we could construct the test statistic: 
\begin{equation}
T_j = \sqrt{n} (\bar{X_j} - \mu_{j0}) / \hat{\sigma_j} 
\end{equation}
It follows $t_{n-1}$ distribution under $H_0$. 

\item Testing the mean of MVN: suppose we want to test $H_0: \mu = \mu_0$. The intuition is that we form a test statistic at each of the $p$ dimensions, measuring the departure from $\mu_0$, and add them together (squared from, since each individual statistic may be signed). This is easy when $X_j$'s are orthogonal, so we use the EVD of the MVN distribution. Let $y = U (X - \mu)$, under $H_0: \mu = \mu_0 \Rightarrow E(y) = 0$, thus we are testing if the mean of $y$ is 0. We form this test statistic: 
\begin{equation}
T^2 = n \sum_j \frac{\bar{y}_j^2}{\hat{\sigma}_j^2}	
\end{equation}
To express $T$ in the orignal space, we use the Equation~\ref{eq:MVN_eigen}: 
\begin{equation}
T^2 = n (\bar{x} - \mu_0)^T W^{-1} (\bar{x} - \mu_0)
\end{equation}
where $W$ is the sample covariance matrix (in place of $\Sigma$, which is unknown), and given by: 
\begin{equation}
W = \frac{1}{n-1} \sum_i (x_i - \bar{x}) (x_i - \bar{x})^T	
\end{equation}
$T^2$ under $H_0$ follows Hotelling's $T^2$ distribution, thus this test is called Hotelling's $T^2$ test. 
\end{itemize}

\subsection{Applications of MVN}

Sampling from multivariate normal distribution (MVN): 
\begin{itemize}
\item Theorem: $d$ iid. random variables, $Z_i \sim N(0,1)$. Let $Z = (Z_1, \cdots, Z_d)^T$, and 
\begin{equation}
X = \mu + A Z	
\end{equation}
where $A$ is the Cholesky decomposition of the matrix $\Sigma$, $A A^T = \Sigma$. Then $X \sim N(\mu, \Sigma)$. 
\end{itemize}

Projection of data matrix on a direction: 
\begin{itemize}
\item Suppose $v$ is a unit vector in $p$-dim space, the projection of $X$ on $v$ can be written as: 
\begin{equation}
Xv = \left( \begin{array}{l} x_1 \\ \vdots\\ x_n \end{array} \right) v = \left( \begin{array}{l} x_1 v\\ \vdots\\ x_n v \end{array} \right) 	
\end{equation}
Let $\Sigma$ be the sample covariance matrix of $X$, then the sample variance of the vector $Xv$ is: 
\begin{equation}
\label{eq:projection_variance}
\Var(Xv) = \frac{1}{N} (Xv - \bar{x}v)^T (Xv - \bar{x} v) = \frac{1}{N} [(X - \bar{x})v]^T (X - \bar{x})v	= \frac{1}{N} v^T \hat{\Sigma} v
\end{equation}
\end{itemize}

%%%%%%%%%%%%%%%%%%%%%%%%%%%%%%%%%%%%%%%%%%%%%%%%%%%%%%%%%%%%
\section{Categorical and Count Data}

Poisson ASH [Mengyin Lu thesis, 2018]
\begin{itemize}
	\item Motivation: scRNA-seq data, estimate distribution of expression for each gene. Use ASH prior: mixture of uniform distributions can approximate any unimodal distribution.  
	
	\item Model: let $Y_{cg}$ be expression of gene $g$ of cell $c$. Let $\alpha_c$ be the scaling factor of cell $c$,  
	\begin{equation}
	Y_{cg} \sim \text{Pois}(\alpha_c \lambda_{cg}) \qquad \lambda_{cg} \sim G_g(\cdot) = \pi_g \delta_0 + (1-\pi_g) H_g 
	\end{equation}
	where $\pi_g$ captures zero-inflation, and $H_g$ is the distribution. Use ASH for $H_g$: a mixture of uniform distributions. Note: do not use log-link function for $\lambda_{cg}$, not stable. Possible explanation: many cells with very low expression, log. expression would be very small (negative), thus need a large number of grids.   
\end{itemize}

\subsection{Contingency Tables} 

Exact tests of categorical data: [Hartl, Principles of Population Genetics, Section 2.3]
\begin{itemize}
	\item Discrete test statistic: suppose the test statistic is a discrete RV, $T$, with probabilities $p_i$ for the value $a_i$ (un-ordered), what is the appropriate transformation that computes the $p$ value of $T$? The solution: rank all $i$'s by the value of $p_i$ (ascending order), and the $p$ value of $T$ is the sum of all $p_i$'s below $T$.  
	
	\item Exact test of sample configuration: suppose we are testing a count table, we call the counts at each cell (possible values of combinations of variables) as a sample configurations. Ex. in HWE test, the counts of $AA, Aa, aa$ are a sample configuration. The probability of obtaining any sample configuration under $H_0$ can be computed, and this allows one to calculate the $p$ value of any observed sample configuration. 
	
	\item Applications: in population genetics, test HWE of allele frequencies, or LD between two loci. 
\end{itemize}

$\chi^2$ test of multinomial distribution and tables: [Rice, Mathematical Statistics and Data Analysis]
\begin{itemize}
	\item Testing multinomial distribution: suppose we have $k$ categories (could be $k$ cells in a table), with the count $(X_1, \cdots, X_m) \sim \text{MN}(p_1,\cdots,p_m)$. We are testing the hypothesis $H_0: p_1(\theta), \cdots, p_m(\theta)$, where $\theta$ is $k$-dim. parameter, $k < m$, against $H_1: p_1, \cdots, p_m$, i.e. $m$ free parameters. 
	
	\item Likelihood ratio $\chi^2$ test: also called $G$-test. Let $p_1(\hat{\theta}), \cdots, p_m(\hat{\theta})$ be the MLE under $H_0$, and $\hat{p}_i = x_i / n$ be the MLE under $H_1$. We form the LRT: 
	\begin{equation}
	-2 \log \lambda = 2 \sum_{i=1}^m x_i \log \frac{\hat{p}_i}{p_i(\hat{\theta})}	=  2 \sum_{i=1}^m O_i \log \frac{O_i}{E_i}
	\end{equation}
	where $O_i = n \hat{p}_i$ is the observed count at the $i$-th cell, and $E_i = n p_i(\hat{\theta})$ is the expected count (under $H_0$). At large sample size ($x_i$ generally greater than 5), the test follows $\chi^2$ distribution with dof equal to $m - k$. 
	
	\item Pearson's $\chi^2$ test: defined as
	\begin{equation}
	\chi^2 = \sum_{i=1}^m \frac{(O_i - E_i)^2}{E_i}	
	\end{equation}
	This similarly follows $\chi^2$ distribution with dof. $m - k$. We can show the test is aympotically equivalent to LRT by expanding $-2 \log \lambda$ near $p_i(\hat{\theta})$ (easier to calculate, but LRT is recommended). 
	
	\item Multiple Binomial test: suppose we have $m$ independent binomial distributions, with $X_i \sim \text{Bin}(N_i, p_i)$. We are testing $H_0: p_i(\theta)$ against $H_1: p_i, 1 \leq i \leq m$. We could similarly form the LRT: 
	\begin{equation}
	-2 \log \lambda = 2 \sum_{i=1}^m \left[x_i \log \frac{\hat{p}_i}{p_i(\hat{\theta})} + (N_i - x_i) \frac{1 - \hat{p}_i}{ 1 - p_i(\hat{\theta})} \right]	
	\end{equation}
	Using similar notations of $O_i$ and $E_i$: 
	\begin{equation}
	-2 \log \lambda =	2 \sum_{i=1}^m \left[ O_i \log \frac{O_i}{E_i} + (N_i - O_i) \log \frac{N_i - O_i}{N_i - E_i}\right]
	\end{equation}
\end{itemize}

McNemar's test: [Sprent, Applied Nonparametric Statistical Methods]
\begin{itemize}
	\item Example: (Section 5.2) climbing records of two rocks: number of successes and failures in two rocks respectively, and want to know if one is harder than the other. Since if a climber succeeds or fails in both rocks, no information is provided for the relative difficulty, only the diagoal cells provide information, and need to be considered. 
	
	\item McNemar's test: give a 2-by-2 table, and we want to test if the diagonal cell counts are equal. Suppose the cell counts are $a,b,c,d$ where $b,c$ are diagonal counts. Our null hypothesis is $H_0: p_b = p_c$, where $p_b$ or $p_c$ is the probability of cells. The McNemar's test is given by: 
	\begin{equation}
	X^2 = \frac{(b-c)^2}{b+c}	
	\end{equation}
	$X^2$ follows $\chi^2$ distribution with dof 1 under the null hypothesis. 
	
	\item Normal approximation to binomial test: we are testing the cell count $b \sim \text{Bin}(b+c,1/2)$. Use the normal approximation, our test statistic should be: 
	\begin{equation}
	Z = \frac{b - (b+c)/2}{0.5 \sqrt{b+c}}	
	\end{equation}
	Apply the $\chi^2$ distribution (square of standard normal distribution) and we obtain the McNemar's test. 
	
	\item Pearson's $\chi^2$ test: we consider only the two diagonal cells and apply the Pearson's $\chi^2$ test, where the expected count is $(b+c)/2$ for both cells. 
	
	\item Difference of cell counts: we define the test statistic $T = b - c$, which should indicate the difference of two cells, i.e. $E(T|H_0) = 0$. We need to determine the variance of $T$ under $H_0$. Use $b|H_0 \sim \text{Bin}(b+c,1/2)$, we have $\text{Var}(b) = (b+c)/4$. Then 
	\begin{equation}
	\text{Var}(T)	= \text{Var}(2b - (b+c)) = 4 \text{Var}(b) = b + c
	\end{equation}
	Assume $T$ follows normal distribution, we have $T^2/\text{Var}(T)$ as our test statistic with $\chi^2_1$ distribution. 
\end{itemize}
%%%%%%%%%%%%%%%%%%%%%%%%%%%%%%%%%%%%%%%%%%%%%%%%%%%%%%%%%%%%
\section{Naive Bayes and Discriminant Analysis}
\begin{enumerate}

\item{Naive Bayes classifier}

Naive Bayes (NB) model and model fitting: 
\begin{itemize}
\item Naive Bayes classifier: suppose the data has $D$ features, the likelihood of one data point is: 
\begin{equation}
p(x|y=c,\theta) = \prod_{j=1}^D p(x_j|y=c,\theta_{jc})	
\end{equation}
where $\theta_{jc}$ is the model parameter for the $j$-th feature of class $c$. The commonly used models are: Gassussian distribution for continuous features, multivariate Bernoulli distribution for binary features. 

\item Model fitting by MLE: clearly to fit the model, we fit the distribution for each class separately, and the problem is easily reduced to known problems for fitting Gaussian or Bernoulli distributions. Let $\pi_c$ be the fraction of class $c$, and $\theta_{jc}$ be the Bernoulli parameter of the feature $j$ of class $c$, then we have: 
\begin{equation}
\hat{\pi}_c = \frac{N_c}{N} \qquad \hat{\theta}_{jc} = \frac{N_{jc}}{N_c}	
\end{equation}
where $N_c$ is the number of points in class $c$, and $N_{jc}$ is the number of examples whose $j$-th feature is 1 in class $c$. The MLE suffers from overfitting, notably, the zero-count problem, where $\hat{\theta}_{jc} = 0$ if $N_{jc} = 0$. 

\item Bayesian NB: assume a prior distribution $\pi \sim \text{Dir}(\alpha_1, \cdots, \alpha_C)$ and $\theta_{jc} \sim \text{Beta}(\beta_0, \beta_1)$, the posterior distribution can be easily determined: 
\begin{equation}
p(\pi|D) = \text{Dir}(N_1+\alpha_1, \cdots, N_C+\alpha_C)	
\end{equation}
\begin{equation}
p(\theta_{jc}|D) = \text{Beta}((N_c - N_{jc}) + \beta_0, N_{jc} + \beta_1)	
\end{equation}
\end{itemize}

Model prediction and analysis: 
\begin{itemize}
\item Posterior of class: suppose we train the model using data $D$, and we need to predict the class label of a new instance $x$, the posterior of class is: 
\begin{equation}
p(y=c|x,D) \propto p(y=c|D) \prod_{j=1}^D p(x_j|y=c,D)	
\end{equation}
Since the posterior distribution of the parameters are known, we can compute the posterior predictive of $y$ and $x_j$ respectively. The result is: 
\begin{equation}
p(y=c|x,D) \propto \bar{\pi}_c \prod_{j=1}^D (\bar{\theta}_{jc})^{I(x_j=1)}	(1-\bar{\theta}_{jc})^{I(x_j=0)}	
\end{equation}
where $\bar{\pi}_c$ and $\bar{\theta}_{jc}$ are posterior mean. If we use the MLE estimator, the prediction is similar, except that we replace posterior mean by MLE. 

\item Decision boundary and relation to linear model: we can take the log. of the posterior and write it as: 
\begin{equation}
\log p(y=c|x,\theta) = \log\pi_c + \sum_j \left[x_j \log \theta_{jc} + (1-x_j) \log (1-\theta_{jc}) \right] + \text{const}
\end{equation}
To write it as a linear model: 
\begin{equation}
\log p(y=c|x,\theta) = \log\pi_c + \sum_j \beta_{jc} x_j  + \beta_{0c} + \text{const}
\end{equation}
where
\begin{equation}
\beta_{jc} = \log \frac{\theta_{jc}}{1 - \theta_{jc}} \qquad \beta_{0c} = \sum_j \log (1 - \theta_{jc})
\end{equation}
In particular, when we have only two classes, the log. posterior ratio is given by: 
\begin{equation}
\log \frac{p(y=1|x,\theta)}{p(y=0|x,\theta)}	= \beta_0 + \sum_j x_j \beta_j = X \beta
\end{equation}
where
\begin{equation}
\beta_j = \log \frac{\theta_{j1}/(1 - \theta_{j1})}{\theta_{j0}/(1 - \theta_{j0})}
\end{equation}
Thus this is similar to logistic regression, the difference being how parameters are trained. 

\item Feature analysis/ranking: the weight of a feature for prediction is the log odds ratio between the two classes. For relative rare variables, this is roughly the log. of frequency ratio between the postive and negative classes. Consider document classification problem, in general, the scoring scheme favors rare words (desired); but for rare words, it is more likely that the log-OR may be large from random sampling, creating noises (undesired). 

\item Feature selection by mututal information: one way to reduce the noise to select only discriminative features. We could use the MI (which agrees with the analysis of the feature weights). The MI of the $j$-th feature in the case of multivariate Bernoulli model is: 
\begin{equation}
I_j = \sum_c \left[\theta_{jc} \pi_c \log \frac{\theta_{jc}}{\theta_j} + (1-\theta_{jc}) \pi_c \log \frac{1- \theta_{jc}}{1- \theta_j} \right]	
\end{equation}
where $\theta_j = \sum_c \pi_c \theta_{jc}$. 
\end{itemize}

Extending the basic NB model: 
\begin{itemize}
\item Document classification by bag of words: we could use multinomial distribution to model word counts, which may be more informative than word presence. The model is easy to train and use. However, it does not work well for classification, one reason being the burstiness problem: the rare words often do not occur at all in a document, but once they do, they occur in bursts. This is not easily modeled by multinomial distribution. One idea is to use Dirichlet compound multinomial model for the density. 

\item Modeling dependency of features: (Exercise 3.20) 
\begin{itemize}
\item Intuition: why modeling dependency may help? In document classification problem, phrases are often more informative than single words. For example, ``Windows operating system'', each word alone may not be very discriminative between classes, but the phrase is. Modeling dependency allows one to capture phrases. 

\item In general, feature dependency may help by capturing informative patterns (specific combination of individual features). Ex. we consider two classes in 2D space: all the points of the two classes belong to the same square, but the points of two classes are located in the two halfs of the square, separated by the diagonal line. Then neither $x$ nor $y$ dim. alone is very discriminative.  
\end{itemize}

\item $L_1$ regularization: [$L_1$-regularized naive Bayes classifiers] the idea is that $\theta_{jc}$ should be equal for any $c$ for most features. Thus we define the penalized log-likelihood, in the fashion of group lasso. Adding the penalty term to negative log-likelihood: 
\begin{equation}
\lambda J(\theta) = \sum_j \sqrt{\sum_c (\theta_{jc} - \theta_{j\cdot})^2}	
\end{equation}
where $\theta_{j\cdot}$ is the mean of $\theta_{jc}$ over all $c$'s. If the $j$-th feature can take multiple values (multinomial model instead of Bernoulli), we could extend the equation by summing over $k$ as well ($k$ is the index of the possible outcome of the $j$-th feature). 
\begin{itemize}
	\item Remark: the penalty term does not take word frequency into account, the term tends to be dominated by common words (regularization of common words but not rare ones). 
\end{itemize}
\end{itemize}

Lessons/questions: 
\begin{itemize}
\item Decision boundary analysis: for a classification problem, to understand a classifier, the first question is what is the decision boundary of the classifier: is it linear or not, etc. 

\item Feature analysis: what features contribute most to the prediction function? Often helpful to intuitively understand it, e.g. for document classification, whether common or rare words are more important. 

\item Feature dependency and informative patterns: what kind of dependency between features may provide extra information? For document classification, whether phrases or higher structures are informative, etc. 

\item Q: under Bayesian NB, we would like a prior distribution that favors equal $\theta_{jc}$ for each $c$, for most $j$'s. How would we define such prior? 

\item Q: how to deal with covariates in NB model: suppose we are mainly interested in finding relation between $x$ and $y$, but we have covariates $z$ that is associated with $x$ or $y$. It is easy to model this in the regression framework; in the NB model, we need to model $y \rightarrow x \leftarrow z$, and $y$ and $z$ are associated. 
\begin{itemize}
\item A simple model is: define the distribution of $x$ on each combination of $(y,z)$. For model complexity purpose, we may add additional priors, e.g. the effect of $y$ and $z$ on $x$ is independent (i.e. the effect size of $y$ on $x$ is the same across all values of $z$'s). 
\item More generally, both regression and generative models are special cases of a graphical model involving variables $x$, $y$ and $z$. 
\end{itemize}

\end{itemize}

Reference: Chapter 3 of \cite{Murphy12}. 

\end{enumerate}
%%%%%%%%%%%%%%%%%%%%%%%%%%%%%%%%%%%%%%%%%%%%%%%%%%%%%%%%%%%%
\section{Latent Variable Models}

Expectation-Maximization (EM) algorithm [Murphy, 11.4]
\begin{itemize}
	\item EM: let $x$ be the data, $z$ missing data and $\theta$ parameters. The complete data log likelihood is given by:
	\begin{equation}
	l_c(\theta) = \log p(x, z|\theta)
	\end{equation}
	The E-step computes the expected complete log-likelihood:
	\begin{equation}
	Q(\theta | \theta^t) = \E_{z|\theta^t, x}[\log p(x, z | \theta)]
	\end{equation}
	where the expectation is taken over the posterior of $z$ given data and current parameters. The M-step maximizes the function, treating $\theta$ as parameter, but $z|\theta^t, x$ as given.  
	
	\item Justification of EM: $Q(\cdot)$ is a lower bound of observed data log-likelihood $l(\theta) = \log p(x|\theta) = \log \sum_z p(x,z|\theta)$. One can show that in EM, $l(\theta)$ is always monotonically increasing, so this leads to local max. of $l(\theta)$. 
\end{itemize}

\subsection{Mixture Models and Missing Data Problem} 

Bayesian mixture model: [GCSR, Chapter 18]
\begin{itemize}
\item Latent variable model: suppose we have $M$ groups with the $m$-th group defined by the model $\theta_m$. Each sample belongs to one of the $M$ groups, with the membership variable unobserved. Let $\xi_i$ be the membership vector (unit vector) of the $i$-th sample, i.e. $\xi_{im} = 1$ if the $i$-th sample belongs to the $m$-th group and 0 otherwise. Our model for the latent variable is thus: 
\begin{equation}
z_i \sim \text{Multinomial}(\lambda_1, \cdots, \lambda_K)	
\end{equation}
And the complete likelihood: 
\begin{equation}
p(y, z| \theta, \lambda) = p(z|\lambda) p(y|z,\theta) = \prod_i \prod_{m=1}^M (\lambda_m f(y_i|\theta_m))^{z_{im}}	
\end{equation}
The prior of $\lambda$ follows the Dirichlet distribution: $\lambda \sim \text{Dir}(\alpha_1, \cdots, \alpha_M)$. 

\item Equivalent model: we could elimiate $\xi$ all together in the model. Instead, we have the likelihood as: 
\begin{equation}
p(y_i|\theta,\lambda) = \sum_{m=1}^M \lambda_m f(y_i|\theta_m)	
\end{equation}
This is thus a single-level model with no latent variables. However, there is an advantage of the latent variable formulation (see below). 

\item Comparison with hierarchical model: the group membership is observed in the hierarchical model, but not in the mixture model. Thus for hierarchical normal model, we have: $y_i \sim N(\theta_{j[i]}, \sigma^2)$ where $j[i]$ is the group that the $i$-th sample belongs to; and for the normal mixture model, we have: $y_i|z_i,\theta \sim N(\theta_{z_i}, \sigma^2)$. 

\item Inference with EM algorithm: let $\phi = (\theta,\lambda)$ be the parameters. Suppose we are maximizing the posterior mode of $\theta$, averaging latent variables $z$. In the E-step, we compute the objective function: 
\begin{equation}
Q(\phi|\phi^{(t)}) = \E_{z|\phi^{(t)},y}\left[ \log p(\phi,z|y)\right]	
\end{equation}
Plug in $\phi = (\theta, \lambda)$, we have: 
\begin{equation}
p(\theta,\lambda,z|y) \propto p(\theta) p(\lambda) p(z|\lambda) p(y|z,\theta)	
\end{equation}
Note that the decomposition makes the E-step easier since some of the terms do not depend on the latent variables. In the M-step, sometimes CM can be used: effectively, we iteratively update $\lambda$ assuming $\theta$ is given, and update $\theta$ assuming $\lambda$ given. 

\item Inference with Gibbs sampling: alternatively sample from the conditional distributions: $p(\theta|\lambda,z,y) = p(\theta|z,y)$, $p(\lambda|\theta,z,y) = p(\lambda|z)$ and $p(z|\theta,\lambda,y) \propto p(z|\lambda) p(y|z,\theta)$. 

\item \textbf{Data Augmentation}: the general principle is that in many cases, it may actually facilitate inference if one introduces additional latent variables. Conceptually, this makes it possible to reduce a complex distribution (to be maximized or sampled) into multiple simpler ones. 
\end{itemize}

Bayesian missing data problem: [GCSR, Chapter 21]
\begin{itemize}
\item Motivation: e.g. to infer a multivariate normal distribution, for some samples, some components are missing. Similarly, for regression problems, some explanatory variables of some samples may be missing. 

\item Model: note that we need to model the ``missing data mechanism'' (i.e. the random process by which some samples have missing data). Suppose $y = (y_{\text{obs}}, y_{\text{mis}})$ is the data, and $I$ is the indicator variable (whether a sample is observed or not). The missing data mechanism is modeled by the distribution, $p(I|y_{\text{obs}}, \phi)$. Then the likelihood is given by: 
\begin{equation}
p(y_{\text{obs}},I|\theta,\phi) = p(I|y_{\text{obs}}, \phi) \int p(y_{\text{obs}}, y_{\text{mis}}|\theta) d y_{\text{mis}}
\end{equation}

\item Inference: an example of Data Augmentation, Gibbs sampling or EM algorithm. It involves iteration of (1) imputation of missing data, this is based on: 
\begin{equation}
p(y_{\text{mis}}|y_{\text{obs}},\theta)	\propto p(y_{\text{obs}},y_{\text{mis}}|\theta)
\end{equation}
and (2) inference of parameters, assuming there is no missing data. 
\end{itemize}

Mixture model of multi-dimensional discrete variables: [Stephens' GTEx grant]
\begin{itemize}
\item Motivation: suppose we are trying to model variables $(Z_1, \cdots, Z_m)$ where $Z_i$ is a binary variable. For example, $Z$ may represent the expression of a gene in $m$ tissues (discretized expression). The $m$ variables are not independent, so we cannot use multi-Bernoulli to model it. There are $2^m$ different configurations, using a different prob. for each possible configuration is also unrealistics.  

\item Idea: we assume the samples form $K$ clusters, where each cluster represents a different tendency of being 1 in different dimensions. For example, one cluster represents genes likely to be active in the second tissue, but not the first and third one, so the distribution of samples from this cluster can be represented as $(0.1, 0.9, 0.1)$. Another cluster may represent no expression in all tissues $(.01, .01, .01)$. Under this model, there may be ambiguity of assigning a configuration $Z$, which may belong to multiple clusters.  

\item Model: for the $k$-th cluster, we represent its ``profile'' as $q_k = (q_{k1}, \cdots, q_{km})$, so the probability of a configuration $Z$ is:
\begin{equation}
p_k(Z|q_k) = \prod_{i=1}^m q_{ki}^{Z_i} (1 - q_{ki})^{1-Z_i}
\end{equation}
Furthermore, we define a prior/proportion of each cluster $\pi_k$. 

\item Remark: the model can be used in the context where $Z_i$ are latent variables too. 
\end{itemize}

Joint analysis of differential gene expression in multiple studies using correlation motifs (CorMotif) [Wei \& Ji, Biostatistics, 2015]
\begin{itemize}
	\item Biological motivation: suppose we want to test DE of a gene, and we have multiple conditions/samples. There is information shared between samples: e.g. a gene is activated in one tissue, then it's likely that it is also activated in a related tissue. The challenge is to model the relationship among conditions. 
	
	\item Model intuition: we model the state of a gene in a condition as an indicator. To model the dependency of conditions, instead of directly modeling multivariate binomial RVs, we use a mixture model: each gene belongs to one motif, where each motif specifies a pattern of the indicator variable, e.g. it is likely to be 1 in all conditions; or likely 1 in the first three conditions and 0 otherwise. We assume that there are a small number of motifs. 
	
	\item Model: let $\pi_k$ be the probability of motif $k$, and for a motif $k$, $q_{kd}$ is the probability of being 1 in the study $d$, i.e. $Q$ specifies activity patterns of motifs. We have latent variables, $B$, the motif membership of genes, and $A = (a_{gd})$ be the activity (1 or 0) of gene $g$ in study $d$. Given data $T$ (test statistic from limma), our model can be specified by:
	\begin{equation}
	P(T, A, B | \pi, Q) = P(B|\pi) P(A|B,Q) P(T|A)
	\end{equation}
	Each component: (1) $B|\pi$: multinomial distribution. (2) $A|B,Q$: Bernoulli distribution, $a_{gd}$ is given by the activity of the motif $g$ belongs to. (3) $T|A$: we have $t_{gd} | a_{gd} = 1 \sim f_{d1}$ and $t_{gd} | a_{gd} = 0 \sim f_{d0}$. The paper uses t-distribution. 
	
	\item Inference: to marginalize $B$ and $A$, use EM algorithm. To find the number of motifs, using BIC.  
	
	\item Interpretation of model: once we have the $Q$ matrix, the correlation between two studies are measured by $\sum_k \pi_k q_{k1} q_{k2}$, summing over all motifs. This is the probability that a gene is active in both studies. 
	
	\item Remark: alternative model using sparsity. Given $D$ conditions, we have $2^D$ possible configurations. However, in truth, there are probably only a small number of configurations, so we can allow all configurations, and let $\pi_k$ be the probability of configuration $k$ (binary). But we assume $\pi_k$ is sparse. 
	
	\item Remark: we can also model the continuous relationship using covariance matrix. If we pre-specify the possible covariance, this leads to Matrix ASH (MASH). If not, this is similar to sparse Gaussian mixture model. 
\end{itemize}
%%%%%%%%%%%%%%%%%%%%%%%%%%%%%%%%%%%%%%%%%%%%%%%%%%%%%%%%%%%%
\subsection{Principal Component Analysis (PCA)} 

Latent variable model and motivations of PCA:
\begin{itemize}
\item Motivation: in many problems/applications, there are certain unmeasured (latent) variables, which influence the observed variables. And variations of these latent variables explain the  variations of all observed RVs, and the covariance among related RVs (those sharing the same latent variable(s)).  

\item Generic latent variable model: suppose we have two latent variables $U$ and $V$, our observed variable $X_j$ can be expressed as (ignoring the mean): 
\begin{equation}
X_j = \beta_j U + \gamma_j V + \epsilon_j	
\end{equation}
For the $i$-th data point, we could write $x_i$ ($D$-dim. vector) in terms of $u_i$ and $v_i$ in vector form: 
\begin{equation}
x_i = u_i \beta + v_i \gamma + \epsilon_i	
\end{equation}
where $\beta = (\beta_1, \cdots, \beta_p)^T$ and $\gamma = (\gamma_1, \cdots, \gamma_p)^T$ are vectors representing the effect of $U$ and $V$ on $X_j, 1 \leq j \leq p$, respectively. Thus, a PCA model has two main components: 
\begin{itemize}
	\item Latent variables $\mathbf{u} \perp \mathbf{v}$: that explain the variance and correlation of the observed variables. If the observed variables have some correlation structure, then most likely some lower-dimensional representation with latent variables will be possible. 
	\item Weights/coefficients: this can be understood in terms of how the observed variables depend on the latent variables. They could be seen as ``contribution vectors'' (of the latent variables). 
\end{itemize}
More generally, we have $D$ observed variables with $N$ samples, and we find $L$ latent factors to explain the data. 

\item Examples:
\begin{itemize}
\item Gene expression: expression profiles of $D$ genes under $N$ conditions. The hidden variables are nutrient availability, stress (such as temperature), etc. The eigenvectors of the two latent variables represent the contribution/effect of the two on gene expression. 

%\item Genetics: genotypes of $p$ SNPs in $N$ individuals. Since the SNPs from the same ancestor, are highly correlated, we could assume that there are some latent variables representing the ancestors. For the $j$-th SNP, its genotype of an individual depends on the ancestral composition of this individual: e.g. 80\% Eurpoean ($U$) and 20\% African ($V$), and the genotypes of the two ancestors ($\beta_j$ and $\gamma_j$). 

\item Medicine: a person's risks of cancer, diabetes, heart disease, stroke depend on a few shared latent variables, e.g. the metabolic aspect/insuline resistence ($U$), and the inflammatory aspect ($V$). The eigenvectors represent the contribution of the two factors on the risks of various diseases. 

\item Economics: there may be many variables to measure the economic activities of a country, e.g. manufacturing, inventory, GDP, corporate spending, employment rate, etc. They may all depend on a few latent variables, e.g. one for the level of consumer spending, one for inflation. The eigenvectors would then represent how strongly each economic index depends on these two factors. 

\item Image processing: the pixel representation of an image really reflects the content of an image, e.g. in terms of what objects it has. 
\end{itemize}

\item Intuitive picture of PCA: identifiability issue, e.g. in 1D case ($D=1$), any finite mixture model is identifiable, but infinite mixture, $L =1$ is not. When $D > L$, the model is identifiable (with conditions, see below), and it tries to explain the correlation between variables in terms of some hidden variables. So for example, it first finds all strongly correlated variables, and define a latent variable to explain the correlation of these variables; it then repeats the process on the remaining variables or unexplained variances of the variables processed in the first step. 

\item The latent variable model in 2D: similar to least square regression, we could formulate the objective function as (2D case): 
\begin{equation}
\min_{\beta, \gamma, u_i, v_i} \norm{x_i - u_i \beta - v_i \gamma}^2
\end{equation}
where we assume $\mathbf{u} \perp \mathbf{v}$. Note that this is to assume that all variables have the same variance (thus variables need to be standarized first to apply PCA).

\end{itemize}

What is principal component analysis? [NBT, 2008]
\begin{itemize}
	\item Concept of PCA: reduces the dimensionality of the data while retaining most of the variation in the data set. It accomplishes this reduction by identifying directions, called principal components, along which the variation in the data is maximal.
	
	\item Example of two genes: Figure 1 A-C. 
	
	\item Dimension reduction and visualization: Figure 1 D-E. Data of 10000 genes can be projected into 2D space. 
\end{itemize}

Background: variance partition 
\begin{itemize}
\item Theorem: if $X = Y + Z$ is a sum of two independent RVs $Y$ and $Z$, then the sample variance of $X$ is the sum of sample variance of $Y$ and that of $Z$. The proof follows from the Pythagorean Theorem. 

\item Variance in multi-dimensional space: suppose we have $x_i \in \mathbb{R}^p$, let $\mu$ be the mean of all $x_i$'s, we could define the total variance of the data points as: 
\begin{equation}
V = \sum_i \norm{x_i - \mu}^2	
\end{equation}

\item Variance partition: consider the $j$-th component of $x_i$'s, let $\mu_j$ be the mean of the $j$-th component, and we could define the variance in the $j$-th component as: 
\begin{equation}
V_j = \sum_i (x_{ij} - \mu_j)^2	
\end{equation}
Then the total variance can be expressed as: $V = \sum_j V_j$. In general, if we have an orthogonal basis of $\mathbb{R}^p$, $v_1, \cdots, v_p$, and project $x_i$'s on $v_j$'s, and let $V_j$ be the variance in the projections on $v_j$, we have $V = \sum_j V_j$, simply from the Pythagorean Theorem. 

\item Error and variance: suppose we project $x_i$'s ($p$-dimensional) on a lower-dimensional hyperplane of dim. $q$. Let $V_p$ be the total variance, and $V_q$ be the total variance of projections. We define the error as: 
\begin{equation}
\text{Err} = \sum_i \norm{x_i - x_i'}^2	
\end{equation}
where $x_i'$ is the projection of $x_i$. Then according to the Pythagorean Theorem, we have 
\begin{equation}
\text{Err} = V_p - V_q	
\end{equation}
Thus the error is simply the unexplained variance in the data, which is the variance in the other dimensions (orthogonal to the hyperplane). 

\item Remark: this is similar to linear regression, where the total variance of the response variable can be partitioned: $SST = SSR + SSE$. Thus minimizing $SSE$ is equivalent to finding the linear model that maximizes $SSR$. In both cases, we have one RV as a sum of two independent RVs (explanatory and error), so we have variance paritioning.   
\end{itemize}

Geometric picture of PCA: 
\begin{itemize}
	\item A simple case of $D = 2, L = 1$: each $x_i$ is a vector in a 2D space, and we find the direction $w$ to project $x_i$ (the projected $x_i$ is $\hat{x_i}$). The objective function is the total distance from $x_i$ to $\hat{x_i}$. The coordinates of $x_i$ in $w$ is $z_i$ (scalar).  
	
%	\item Geometry of the model: when we have two latent variables, our model is: 
%	\begin{equation}
%	X_j = \beta_j u + \gamma_j v \qquad 1 \leq j \leq p
%	\end{equation}
%	Thus as $u$ and $v$ varies, we form a hyperplane in the $p$-dim. space. When we allow nonlinear model, we have a surface in the $p$-D space. To make this hyperplane explicit, we write in the vector form in $p$-dim. space: 
%	\begin{equation}
%	X = u \beta + v \gamma 
%	\end{equation}
%	Thus the plane is defined by the vectors $\beta$ and $\gamma$ (basis vectors of the hyperplane). With this interpretation, our model suggests that most $p$-dim. data points actually lie in a low-dimensional space (2D in our case here), and our primary problem is to infer this low-dim. surface/hyperplane. 
	
	\item Geometric mapping of the latent variable model: given $N$ data points $x_i$ in the $D$-dim. space, the latent variable model can be mapped geometrically: 
	\begin{itemize}
		\item The $N$ points are close to a low-dim. hyperplane, $P$, (2D if there are two latent variables): the objective function (MSE) of the latent variable model corresponds to the total distance of the $N$ points to the plane $P$. 
		
		\item In the low-dim. space, we have $L$ orthogonal vectors $w_1, \cdots, w_L$, $w_k \in \mathbf{R}^p, 1 \leq k \leq L$, each of them representing a ``principal component'' (or principal direction). 
		
		\item The vector $z_i \in \mathbb{R}^L$ are the coordinates of $x_i$ on principal components: the $k$-th coordinate is $z_{ik} = \langle w_k, x_i \rangle = w_k^T x_i$. In vector form: $z_i = W^T x_i$. 
		
		\item The projection of $x_i$ on the plane (coordinates in the original $D$-dim. space), $\hat{x_i}$, can be written as: $\hat{x_i} = \sum_k z_{ik} w_k = [w_1 \cdots w_q] z_i = W z_i$. 
	\end{itemize}
	
	\item Alternative geometric view: We view $X_j$ as a $N$-dim. vector, and the goal is to find an orthogonal set of $N$-dim. vector $Z_1, \cdots, Z_L$ s.t. linear combination of $Z_k$'s explain all $X_j$'s. Ex. suppose $X_j$'s are all parallel to each other, then we can easily choose $Z_1$ that is parallel to all $X_j$'s, which explains the data. 
\end{itemize}

Statistical inference of latent variable model: PCA
\begin{itemize}
	\item Reference: [Murphy, Chapter 12]
%\item Formal statement of the problem: minimizing reconstruction error. The hyperplane can be represented as: 
%\begin{equation}
%f(\lambda) = \mu + \sum_{j=1}^q \lambda_j v_j = \mu + V_q \lambda	
%\end{equation}
%where $v_1, \cdots, v_q$ are the orthogonal basis vectors of this hyperplane, and $\mu$ the location of this plane. Thus, the square distance/reconstruction error:
%\begin{equation}
%\min_{\mu,\{\lambda_i\},V_q} \sum_{i=1}^N \norm{x_i - \mu - V_q \lambda_i}^2	
%\end{equation}
%where $\lambda_i$ is the coordinate vector of $x_i$ on the basis $V_q$. Partially solving this equation when $V_q$ is given: 
%\begin{equation}
%\hat{\mu} = \bar{x}	
%\end{equation}
%\begin{equation}
%\hat{\lambda_i} = V_q^T (x_i - \bar{x})	
%\end{equation} 
%Geometrically, this means the latent variables (coordinates in the hyperplane) are the coordinates of the projection of $x_i$ on the hyperplane. This leaves us to find the orthogonal matrix $V_q$:  
%\begin{equation}
%\min_{V_q} \sum_{i=1}^N \norm{(x_i -\bar{x}) - V_q V_q^T (x_i - \bar{x})}^2	
%\end{equation}

\item Background: projection on orthogonal basis (see Linear Algebra notes, ``Orthogonality''). Let $U$ be an orthogonal basis ($n$-dim), consider the projection of a vector $v$ (in the subspace defined by $U$) onto $U$. Let $x$ be the coordinates of $v$ on $U$, then $v$ is a linear sum of basis vectors of $U$: $v = Ux$; and the coordinates $x = U^T v$.  

\item PCA model: we have $N$ samples, $D$ observed variables and $L$ latent factors. We have $x_i \in \mathbb{R}^D$ as data, and $z_i \in \mathbb{R}^L$ as the latent variable for sample $i$. The $k$-th variable of sample $i$ (assuming it is zero-centered) is given by: 
\begin{equation}
x_{ik} = \sum_{j=1}^L w_{kj} z_{ij} + \epsilon_{ik}, \qquad \epsilon_{ik} \sim N(0, \sigma^2) 
\end{equation}
Or in vector form: $x_i = W z_i + \epsilon_i$, where $W$ is $D \times L$ matrix (factor loading matrix). In matrix form, we can write it as: $X = Z W^T$ where $X$ is $N \times D$ data matrix, $Z$ is $N \times L$ matrix, representing the projection of $X$ on factors. The problem is: 
\begin{equation}
\min_{W, z} \sum_i \norm{x_i - \hat{x}_i}^2
\end{equation}
where $W$ is orthonormal matrix, with $w_j \in \mathbb{R}^D$ unit vector, and $\hat{x}_i = W z_i \in \mathbb{R}^D$. 

\item \textbf{Key notations of PCA}: let $x_i$ be $D$-dim. data point, $z_i$ its low-dim. representation (PC representation or projection), and $W = [w_1 \cdots w_L]$ be the $L$ PCs (i.e. $W$ is $D \times L$ matrix), we have: 
\begin{equation}
\left\{ \begin{array}{ll}
z_i = W^T x_i & \text{PC representation/projection of $x_i$} \\
\hat{x}_i = W z_i & \text{Reconstruction of $x_i$}
\end{array}
\right.
\end{equation}
This is similar to encoding and decoding, respectively. Write this in matrix form: 
\begin{equation}
Z = X W \qquad \hat{X} = Z W^T
\end{equation}
where $X$ is $N \times D$ matrix and $Z$ is $N \times L$ matrix. 

\item \textbf{Theorem}: Minimizing reconstruction error by PCA. We are minimizing this function: 
\begin{equation}
J(W, Z) = \frac{1}{N} \sum_{i=1}^N \norm{x_i - \hat{x_i}}^2
\end{equation}
where $\hat{x}_i = W z_i \in \mathbb{R}^D$, and $z_i \in \mathbb{R}^L$. The constraint is that the $D \times L$ matrix $W$ is orthonormal, i.e. $w_j \in \mathbb{R}^D$ has unit norm. The optimal solution is given by $\hat{W} = V$ where $V$ contains the $L$ eigenvectors of the empirical covariance matrix $\hat{\Sigma} = X^T X$, and the coordinates $z_i = W^T x_i$. 

\item Analysis view of PCA (maximum variance): consider the case of $L = 1$, our objective is $J(w_1, z_1)$. Taking derivative wrt. $z_{i1}$ and let it equal to 0, we can show that $z_{i1} = w_q^T x_i$, so $z_{i1}$ is the orthogonal projection of $x_i$ on $w_1$. And plug-in $z_{i1}$, we have: 
\begin{equation}
J(w_1) = \text{const} - \frac{1}{N} \sum_{i=1}^N z_{i1}^2
\end{equation}
This is effectively the variance of $z_{i1}$. So the problem is to find a direction where the projection has the largest variance. 

\item Solving the maximum variance problem: we plug-in $z_{i1}$, and the objective becomes: 
\begin{equation}
\sum_{i=1}^N z_{i1}^2 = \sum_i w_1^T x_i x_i^T w_1 = w_1^T \left(\sum_i x_i x_i^T\right) w_1 = w_1^T \hat{\Sigma} w_1
\end{equation}
where $\hat{\Sigma}= X^T X$ is the covariance matrix, assuming the data is centered. This is a quadratic form and the solution is given by the eigenvector of the maximum eigenvalue of $\hat{\Sigma}$, according to the Rayleigh quotient. 
The solution is the eigenvector, belonging to the largest eigenvalue, of the real symmetric matrix $X^T X$. Thus the maximum variance direction is the first PC, $w_1$, and the variance is $\lambda_1$, the first eigenvalue of $X^T X$. %and: 
%\begin{equation}
%\text{Var}(Xw_1) = \text{Var}(u_1 \sigma_1) = \frac{\sigma_1^2}{N}	
%\end{equation}
%where the relation follows from $Xw_1 = u_1 \sigma_1$ (from SVD).
Similarly, we have: $w_2$ maximizes the variance of projections among all vectors orthogonal to $w_1$ (also the variance unexplained by $w_1$), and $w_3$ maximizes variance among all orthogonal to $w_1$ and $w_2$ (the variance unexplained by $w_1$ and $w_2$, and so on. 

\item Interpretation of the factor loading matrix $W$: $W = [w_1 \cdots w_L]$ is the matrix consisting of $L$ PCs. $W$ is $D \times L$ matrix, and $W_{jk}, 1 \leq j \leq D, 1 \leq k \leq L$ is the effect of the $k$-th latent variable $Z_k$ on $X_j$. The row vector of $W$: $W_j$ is the (regression) coefficients of all factors on $X_j$. The column vector of $W$ (PC): $W_k$ is the effect sizes of the factor $k$ on all observed variables.     
%is the factor load matrix, representing the coefficients of factors on observed variables. Why $W$ can be understood as representation of PCs as linear combination of observed variables? We need to understand the relationship between relationship of RVs and the geometric picture. Ex. suppose we have $W = w_1 X_1 + w_2 X_2$ where $X_1, X_2, W$ are RVs. Then in the space defined by $X_1$, $X_2$, the coordinates of $W$ is $(w_1, w_2)$. So geometrically, the coordinates of a vector (PC in our case) describes the linear relationship of RVs.   

\item Model identifiability and orthogonality of PCs: according to the geometric representation, suppose $x_i$'s are close to a hyperplane, there are infinitely many ways of choosing the basis of the hyperplane, thus different coordinates of the projections (hence different values of the latent variables). To make the model identified, we could choose the PCs as the orthogonal basis of the hyperplane. Indeed, PCs are eigenvectors of $X^T X$, so they are orthogonal to each other. 

%\item Variance paritition: again, we consider the case of two latent variables, in the $p$-D space, we have: 
%\begin{equation}
%X = U \beta + V \gamma	+ \epsilon
%\end{equation}
%where $U$ and $V$ are latent variables, and $\beta$, $\gamma$ are PCs (orthogonal). This leads to the decomposition of the sample variance: 
%\begin{equation}
%\Var(X) = \Var(U) \norm{\beta}^2 + \Var(V) \norm{\gamma}^2 + \sigma^2
%\end{equation}
%The total variance thus is partitioned into the variance along each of the PC plus the error. To minimize the reconstruction error is thus to maximize the variance explained by the PCs. We can do this repeatedly by finding the PC with the maximum variance, one at a time. 

\item PCA and clustering: PCA only reflects the linearity in the data. If there is clustering structure in the data (e.g. all data points form clusters in low-dim. representation), this will not be captured by the matrix factorization. 
\end{itemize}

Connection of PCA with Singular value decomposition (SVD):
\begin{itemize}
	\item Motivation: suppose $X$ lies in lower-dim. space (lower-rank), how do we understand geometrically why this leads to small eigen-values of $X^T X$, and the idea of SVD as an approximation of $X$? 
	
	\item MVN perspective: suppose our data $X$ is generated from MVN with covariance matrix $\Sigma = X^T X$. The contour of the PDF forms ellipse, with the axis defined by the eigenvectors of $\Sigma$. We can now use the diagonalization trick: let $\Sigma = V D V^T$, then we can view the PDF as ellipse defined by the axis $v_i$, and scaling $d_i$. Now it is clear that: $X$ is low-dimensional means $d_i \approx 0$ for larger $i$ (assuming $d_i$ are sorted). 
	
	\item How do we interpret SVD geometrically when $X$ represents $n \times D$ data matrix? Let $v_1$ be the first singular vector, $D$-dim, of $X$. (1) In linear algebra, we view $X v_1$ as the linear map of $v_1$ in $\mathbf{R}^n$. In statistics, we view $X v_1$ as the projection of $X$ on $v_1$: it has $n$ coordinates along $v_1$. (2) In Linear algebra, we have the result: $X v_i$ and $X v_j$ are orthogonal. In statistics, it means that the projections along $v_i$ and $v_j$ are statistical independent. This means that the total variance of data can be partitioned along each $v_i$. 
	
	\item Geometric intuition of why SVD leads to dimensionality reduction: consider a simple case where $D = 2$. Let $v_1, v_2$ be singular vectors with singular values $d_1, d_2$. Suppose $d_2 \approx 0$. Consider the projection of $X$ on $v_1$ and $v_2$. We have $X v_2 = d_2 u_2 \approx 0$. This means that projections of $X$ along $v_2$ are mostly 0. This means that $X$ lies largely in 1-D space of $v_1$. 
		
	\item \textbf{Truncated SVD}: we consider only the top $L$ singular vectors of $X$, then $X = U \Sigma V^T$, where $U$ is $N \times L$ matrix, $\Sigma$ is $L \times L$ matrix and $V$ is $D \times L$ matrix. With this form, we have the PCs, $W$ in PCA is just $V$: $\hat{W} = V$. The projection of $X$ onto PCs $Z = X W$ is now:
	\begin{equation}
	Z = X W = U \Sigma V^T V = U \Sigma
	\end{equation}
	and the reconstruction $\hat{X} = Z W^T = U \Sigma V^T = X$, which is just the truncated SVD of $X$! 

%	\item $N \times D$ data matrix $X$, consider its SVD: 
%\begin{equation}
%X = U D V^T	
%\end{equation}
%where $U$ is an orthogonal basis of $\mathbb{R}^N$ consisting of eigenvectors of $X X^T$, $V$ is an orthogonal basis of $\mathbb{R}^D$ consisting of eigenvectors of $X^T X$ ($D \times D$ matrix), and $D$ is a $N \times D$ matrix with the diagonal terms $d_1 \geq d_2 \geq \cdots \geq d_D$ (singular values). We see that $v_1, v_2, \cdots, v_D$ are the principal components (PCs). The projection of data points to the PC directions are $Xv$, specifically, for the first direction, $X v_1$ is the projection of all data points into $v_1$, called the first principal component, etc. Furthermore, from SVD: 
%\begin{equation}
%X v_j = u_j d_j
%\end{equation}
%Thus the data points can be represented in $\mathbb{R}^D$ through their PCs (instead of original coordinates). 

\item Variance explained by PCs: from the SVD, it is clear to answer how much variance of data is explained by PCs. Because PCs are orthogonal, the total variation of data is the sum of variation along the direction of each PC/eigenvector of $X^T X$. Let $\lambda_j = d_j^2$ be the eigenvalue of the $j$-th eigenvector of $X^T X$. For the $j$-th PC, its variance is explained is: 
\begin{equation}
\Var(X v_j) = \Var(u_j d_j) = d_j^2 \norm{u_j} = d_j^2 = \lambda_j
\end{equation}
where we use the fact that $u_j$ is a unit vector. 

%	\item Low-rank approximation: PCA is effectively about approximation matrix $X$ by a lower rank factorization: $X = Z W^T$, where $Z$ is $N \times L$ and $W$ is $D \times L$ orthogonal matrix. This is exactly achieved by SVD (truncated): $X = U \Sigma V^T = (U \Sigma) V^T$ where $V$ is $D \times L$ matrix. $V$ satisfies: unit vectors, orthogonal and are eigenvectors of $X^T X$ (maximizes variance in projection). 
		
\item Another connection between PCA and SVD: the outer-product form of SVD. If we want to approximate the matrix $X$, then we should use the largest $L$ singular values. This is the PCA of $X$. 
\end{itemize}

Application and interpretation of PCA [personal notes]
\begin{itemize}
	\item Dimensionality reduction or ``denoising'': new representation of $x_i \in \mathbb{R}^D$ becomes $z_i \in \mathbb{R}^L$ in lower-dimensional space. Each PC $\in \mathbb{R}^D$ represents the effect of a latent variable on every dim. in the original space. 
	\begin{itemize}
		\item Ex. gene expression data: assuming expression of a gene is a result of multiple TFs. Then a PC could represent the effect of a TF on expression of every gene (mainly contributed by genes that are influenced by that TF); and all the samples can be represented now as the vector of all TF levels. 
		
		\item Ex. stock price: assuming stock prices are results of economics of multiple (lower-dim) sections. Then a PC may represent the activity of IT section (which is mainly contributed by stocks in the IT sector). And the stock price of one year can be represented as the vector of activities of all sectors. 
	\end{itemize} 
	
	\item Recovering latent variables: sometimes we can recover the information of these latent variables. (1) Suppose we have additional information of each sample, we can correlate these additional variables with latent variables. In the stock price example, if we have actual measures of each sector, then we can correlate $z_i$'s of every year with these measures. (2) We can also use information of variables: if a PC is mainly contributed by a subset of variables, then the common properties of this subset would be likely important for that PC. Stock price example: for each PC, see which stocks are weighted more.  		

	\item Adjusting for latent variables in comparison: suppose we have two transcriptome datasets, one treatment and one control. We would like to compare gene expression, but need to adjust for hidden covariates such as batch effects that could differ between treatment and controls. The idea is: we first do PCA to find out the latent factors, then for each sample, we have these $z_i$'s, we could treat them as if they are observed, and adjust for them in gene expression test for every gene. 	
\end{itemize}

Statistical analysis of the genomic distribution and correlation of regulatory elements in the ENCODE regions [Zhang \& Gerstein, GR, 2007] 
\begin{itemize}
\item Representation of PCA results by biplots. 

\item Choosing eigenvectors: after PCA, we choose a few eigenvectors that capture most of the variance in $X$, typically two, and represent the results in this 2D space. 

\item Data point representation in the low-dim space: each observation is represented in the 2D space, by its projection in the plane. This data point representation may reveal the additional structure in the data, e.g. many observations may come from the same group, thus sharing the similar values of the PCs. 

\item Variable representation the low-dim space: a variable $X_j$ is related to the latent variable $u$ and $v$ (2D case) by: 
\begin{equation}
X_j = \beta_j u  + \gamma_j v  + \epsilon_j
\end{equation}
This is a straight line in the $(u,v)$ space (ignoring the error term), with $\beta_j$ and $\gamma_j$ reflects the effect of $u$ and $v$ on this $X_j$. 

\end{itemize}

Extensions of PCA and relation with other methods: 
\begin{itemize}
\item Nonlinear dimensionality reduction: the model of how observed variables are related to the latent variables does not have to be linear, and for nonlinear cases, we have $x_i$'s close to a surface in the $p$-D space. 

\item PC-based clustering: one could perform clustering in the low-dim. space (PCs) to reveal the cluster structure. 

\item PC regression: regression of dependent variables on PCs (latent variables), instead of explanatory variables directly. 

\item Clustering vs PCA: clustering is based on the similar idea that a relatively small number of hidden variables would explain the data. The difference is: clustering - individual observation is a simple deviation from the hidden variables (cluster means) while PCA - individual variable is a linear combination of multiple hidden variables (principle components). 

\item Latent variable model perspective of PCA: e.g. the original model is not identifiable, but with more constraints, one may be able to learn the latent variables (e.g. latent variable must be some of the observed ones, with error - CFA model). 
\end{itemize}

Biclustering with heterogeneous variance [Chen, PNAS, 2013]
\begin{itemize}
	
	\item Problem: identify block structure in matrix. 
	
	\item Approximation of matrix with SVD: $X \approx U D V^T$,  where $U$ and $V$ are $r$-dimension. Write $X = \Psi + Z$, where $Z$ is $MVN(0, \sigma^2 I)$, or $X = UDV^T + Z$
	
	\item Approximate $\Psi$ using r-ranked matrix (lower than the dimension of $X$). To approximate the bi-cluster (checkboard structure): sparsity on the signal matrix, $U$ and $V$ are sparse. 
	
	\item SVD algorithm: start with $V$ (orthogonal matrix), then compute $U$ and $V$ in next iteration. To impose sparsity: only values in $U$ and $V$ larger than a threshold are chosen at each step. 
	
	\item Heterogenesou sparse SVD: heterogenous variance. $X_{ij} = \mu_{ij} + \rho \cdot \Sigma \cdot \Phi$. 
	
	\item Sigma: block structure for errors as well, e.g. case and control samples, genes in cases have similar errors. 
	
	\item Application to methylation data in normal vs. cancer samples. Cancer samples: methylation more variable across all cancers. 
	
	\item Lessons: express factor analysis as regression problem, where covariates are factors, then we can use machinery for regression, e.g. sparsity. Use SVD to approximate a matrix, and encode the structure.  
	
	\item Q: why sparsity of $U$ and $V$ in SVD captures checkboard structure? 
\end{itemize}

%%%%%%%%%%%%%%%%%%%%%%%%%%%%%%%%%%%%%%%%%%%%%%%%%%%%%%%%%%%%
\subsection{Factor Analysis}

Basics of factor analysis [Murphy, 12.1]
\begin{itemize}
	\item Model: our data is $x_i \in \mathbb{R}^p$. We assume they are generated from linear combination of a smaller number of latent variables $z_i \in \mathbb{R}^q$ with normal distribution. The model: 
	\begin{equation}
	z_i \sim N(\mu_0, \Sigma_0) \qquad x_i | z_i \sim N(W z_i + \mu, \Psi)
	\end{equation}
	where $W$ is the factor \textbf{loading matrix} of $p \times q$ dimension. We assume $\Psi$ is diagonal, and wlog, $\Sigma_0 = I$. Example: Figure 12.1.
	
	\item FA can be viewed as a way of representing MVN with fewer parameters: the marginal distribution of $x_i$ is given by: 
	\begin{equation}
	x_i \sim N(W \mu_0 + \mu, \Psi + W \Sigma_0 W^T)
	\end{equation}
	Note: covariance of $x$ has $O(p^2)$ parameters, but this representation has $pq + p$ parameters. 
	
	\item Posterior of latent variables $z_i$ if $W$ is given: it follows normal distribution with mean $m_i$ and variance $\Sigma$. For simplicity, we assume $z_i \sim N(0, I)$. 
	\begin{equation}
	\Sigma = (I + W^T \Psi^{-1} W)^{-1} \qquad m_i = \Sigma W^T \Psi^{-1} (x_i - \mu)
	\label{eq:FA_posterior}
	\end{equation}
	Note that $\Sigma$ is independent of data points (same for all $i$), and furthermore, it can be computed efficiently by Inversion Lemma. Also note that $m_i$ is a linear operator of $x_i$. 
	\begin{itemize}
		\item Remark: if we assume $\Psi = I$, and somehow ignore the covariance of $z_i$, we have $m_i \approx W^T (x_i - \mu)$. This is basically the projection of $x_i$ on PCs in PCA. 
	\end{itemize}
	
	\item Biplot: show $z_i$ and show the representation of features $X_j$, both in terms of the latent variables (Figure 12.2). 
	
	\item Identifiability of FA: we can rotate $z_i$, and have the same model (likelihood). Geometrically, suppose all data are close to a hyperplane, but the coordinates of the hyperplane can be chosen arbitrarily. Imagine a hyperplane, consider $z_i$ and 2 PCs, $w_1$ and $w_2$. If we first rotate $w_1$ and $w_2$, and then $z_i$ accordingly, we have the same likelihood. 
	
	\item Addressing identifiability problem: (1) PCA: forcing $W$ to be orthonomal, and rank the PCs by decreasing variance. Note: when $p = 2, q = 1$, maximizing variance leads to a unique solution. (2) Sparsity promoting priors on $W$. 
\end{itemize}

EM algorithm for factor analysis: [Murphy, 12.1.5]
\begin{itemize}
	\item Ref: The EM Algorithm for Mixtures of Factor Analyzers [Ghahramani and Hinton, 1996]. Also \texttt{Perera-FA-EM-slides.pdf}. 
	
	\item Model: we assume $z_i \sim N(0, I)$, and $x_i | z_i, W, \mu, \Psi \sim N(W z_i + \mu, \Psi)$. We denote $\theta = (W, \mu, \Psi)$. 
	
	\item E-step: we first obtain the posterior distribution of $z_i$ given $x_i$ and current parameters $\theta^{(t)}$, given by Equation~\ref{eq:FA_posterior}. To compute the $Q$ function, we first obtain the log-likelihood of the complete data: 
	\begin{equation}
	\log p(x, z|\theta) = \sum_i [\log p(z_i) + \log p(x_i|z_i, \theta)]
	\end{equation}
	We ignore the constant term and plug-in the normal likelihood: 
	\begin{equation}
	\log p(x, z|\theta) = -\frac{1}{2} \sum_i [x_i - Wz_i -\mu]^T \Psi^{-1} [x_i - Wz_i -\mu] - \frac{n}{2} \log \det \Psi + \text{const}
	\end{equation}
	We will now compute the expectation over $z_i$, given the current estimate of $\theta$. 
	\begin{equation}
	Q(\theta) = - \frac{n}{2} \log \det \Psi - \frac{1}{2} \sum_i \left[ x_i^T \Psi^{-1} x_i - 2 x_i^T \Psi^{-1} W \E(z_i|x_i) + \E(z_i^T W^T \Psi^{-1} W z_i | x_i)\right]
	\end{equation} 
	We now use the result about the expectation of the quadratic form of random vector (in this case, $z_i$):
	\begin{equation}
	\E(z_i^T W^T \Psi^{-1} W z_i | x_i) = \tr(W^T \Psi^{-1} W \E(z_i z_i^T | x_i))
	\end{equation}
	where $\E(z_i z_i^T | x_i)$ can be obtained from the posterior of $z_i$ given $x_i$. 
	
	\item M-step: See Appendix A of [Ghahramani and Hinton, 1996]. Use the fact that derivative and trace can commute to simplify the algebra. 
\end{itemize}

Probabilistic PCA [Murphy 12.2.4]
\begin{itemize}
	\item Marginal distribution and likelihood: wlog, we set $\mu_0 = 0$, $\Sigma_0 = I$, $\mu = 0$ (data is centered) and $\Psi = I$ (errors are independent), then the marginal distribution of $X$: 
	\begin{equation}
	x_i | W \sim N(0, \sigma^2 I + W W^T)
	\end{equation}
	The covariance matrix $C = \sigma^2 I + W W^T$, the LL function: 
	\begin{equation}
	\log P(X|W, \sigma^2) = -\frac{N}{2} \log \norm{C} + \frac{1}{2} \sum_i x_i^T C^{-1} x_i = -\frac{N}{2} \log \norm{C} + \text{tr}(C^{-1} S)
	\end{equation}
	where $S = (1/N)X^T X$ is the sample covariance matrix. 
	
	\item Interpretation of $W W^T$: we consider the row vectors of $W$, $W_j$ is $1 \times q$ vector (the effects of PCs on $X_j$): 
	\begin{equation}
	W W^T = \left[ \begin{array}{l}
	W_1 \\
	\cdots \\
	W_p
	\end{array}
	\right] \left[W_1^T \cdots W_q^T\right] = \left[ W_j W_l^T\right]_{p \times p}
	\end{equation}
	So the covariance of $X_j$’s (ignore $\sigma^2 I$) is determined by the loading matrix. Intuitively, if two variables have similar weights in the loading matrix, then they are highly correlated.  
	
	\item MLE: take derivative, $dl / dW$, and set it to 0. Analytic solution and $\hat{\sigma}^2$ is the average of the remaining eigenvalues. 
	
	\item Connection with classical PCA: consider the case when $\sigma^2 \to 0$: then $W$ should satisfy $\hat{W}^T \hat{W} = S$. Let the SVD of $X$ be $U D V^T$ then
	\begin{equation}
	X^T X = (UD V^T)^T (UDV^T) = (V^T D^T) (D V)
	\end{equation}
	So we have: $\hat{W} \to D V$, where $V$ is the $p \times q$ matrix where the columns are the first $q$ eigenvectors of $S$. So the direction of $\hat{W}$ is given by $V$. If we require PCs to be unit vector, then we can ignore the scaling $D$, and the PCs are just $V$. The posterior mean of latent vectors is the projection of $x_i$ on PCs. Difference: in classical PCA, we require the PCs to be unit vector; in PPCA, we require $z$ to have unit variance. 
	
	\item EM algorithm and interpretation: E-step: given the weight matrix, latent variables $\tilde{Z}$ can be determined by projection of $X$ on $W$; M-step: given the latent variables, the weight matrix $\tilde{W}$ can be determined by multiple regression. Physical analogy (Figure 12.11): minimize the potential energy of spring. 
\end{itemize}

Extensions of PCA [Murphy, 12.4, 12.5]
\begin{itemize}
	\item Background: softmax, generalization of logistic regression on multi-class response variables, map $K$-dim. real values to $K$-dim. probability values that sum to 1. Let $x$ be $K$-dim. values, and $y$ is a multinomial random variable, then 
	\begin{equation}
	P(y = j | x) = \frac{e^{x^T w_j}}{\sum_k e^{x^T w_k}}
	\end{equation} 
	where $w_k$ is $K$-dim. vector. Remark: relation to logistic normal distribution (similar to but more flexible than Dirichlet). 
	
	\item Categorical PCA: $y_{ij}$ are $j$-th response of sample $i$. It is a multi-class label ($C$ classes). Ex. multiple binary traits measured in the same individuals. Each response $y_{ij}$ depends on the latent variable $z_i$ ($q$-dim) via softmax function using the weight loading matrix $W_j$ (dim. $q \times C$), and there are $R$ such matrices, $1 \leq j \leq R$. Consider each of the response $j$: 
	\begin{equation}
	z_i \sim N(0, I) \qquad y_{ij} | z_i, \theta \sim S(W_j^T z_i + w_{0,j})
	\end{equation}
	where $S(\cdot)$ is the multi-class logistic regression above. Discrete data can be similarly visualized using categorical PCA (Figure 12.18)
	
	\item Supervised PCA: both $x_i$ and $y_i$ depends on common latent variables $z_i$ (Figure 12.19 (a)): 
	\begin{equation}
	z_i \sim N(0, I) \qquad y_i | z_i \sim N(w_y^T z_i + \mu_y, \sigma_y^2) \qquad x_i | z_i \sim N(W_x z_i + \mu_x, \sigma_x^2 I)
	\end{equation}
	To make inference, we can marginalize $z_i$ and infer the conditional distribution $p(y_i | x_i)$ in terms of $w_x$ (matrix) and $W_y$ (vector). This regression is called ``information bottleneck''. The idea is that: better to use information bottleneck to do regression on $Y$, instead of the original $x$. 
	
	\item Extensions of supervised PCA: not all data have labels, but we can learn $W_x$ via unlabeled data. 
	
	\item Partial least square (PLS): a lot of covariance structure of $x$ may have nothing to do with $y$, therefore, it’s better to do regression on $y$ but using only relevant parts of $x$. So we assume $x_i$ has two types of latent factors: $x$-specific factors, $z_i^x$, give covariance only specific to $x$, and shared factors, $z_i^s$, give common covariance between $x$ and $y$. The model can be written as: Equation 12.83-12.85 and see Figure 12.19 (b).  
	
	\item Canonical correlation analysis (CCA): similar to PLS, but make it more symmetric: $y_i$ can also have specific latent factors. See Figure 12.19 (c). Note: in CCA, we do not have to distinguish explanatory variables and response, so the model can be generally applied to joint analysis of multiple related datasets. 
\end{itemize}

Independent Component Analysis (ICA) [Murphy, 12.6]
\begin{itemize}
	\item Why ICA? PCA solves only half the problem as the likelihood is invariant to rotations. 
	
	\item ICA model ideas: use non-Gaussian distribution for latent variables. Assumptions: $z_j$’s are independent, and variance equal to 1. Example: Figure 12.21, 2D sources are uniformly distributed, PCA does not work well because of normality assumption. 
	
	\item Background: whitening of input data. Make covariance matrix of data equal to $I$. Definition: suppose $X$ is a random vector with mean 0 and covariance $\Sigma$, then $Y = W X$ follows $N(0, I)$, where $W$ is a whitening matrix $W^T W = \Sigma^{-1}$. This could be achieved by PCA: the PCs of the input matrix are orthogonal with variance 1. 
	
	\item ICA model and likelihood: we assume $X$ has be whitened. Let my $t$-th data point be $x_t$, then we have: 
	\begin{equation}
	x_t = W z_t \qquad p(z_t) = \prod_j p_j(z_{tj})
	\end{equation}
	and variance of $z_j$ is 1. Typically in ICA, we ignore the error. With this model we can show that $W$ is orthogonal:
	\begin{equation}
	\Cov(x) = \E(x x^T) = W \E(z z^T) W^T = W W^T
	\end{equation}
	since $\Cov(x) = I$. Assuming that we know the non-Gaussian distribution of each $z_j$ (Note: not a bad assumption, PPCA assumes all factors follow standard normal), we can now express the log-likelihood in terms of $V = W^{-1}$. Using the result about transformation of random variables. We can show that: 
	\begin{equation}
	\log p(D|V) = T \log \abs{\det(V)} + \sum_j \sum_{t=1}^T \log p_j(v_j^T x_j)
	\end{equation}
	where $v_j$ is the $j$-th row of $V$. Note that the first term is constant, since det. of orthogonal matrix is 1 or -1, so we need to compute only the second term. We maximize this function subject to the constraint that $V$ is orthogonal (actually orthonomal). 
	
	\item FastICA algorithm: let $G(z) = -\log p(z)$, and assume it is the same for latent dimensions. Given $V$, we can compute the objective function above. To maximize this function wrt $V$, we can do gradident descent or Newton's method (FastICA). Both derivative and second derivative of the objective can be computed. 
	
	\item Background: $k$-th central moment of a random variable $X$ is defined as $\mu_k = \E(X-\E(X))^k$. If $X$ is normal, we can show that $\mu_4 / \sigma^4 = 3$. This is based on the moment of chi-square distribution, which can be computed with MGF. 
	
	\item Modeling source densities with Non-Gaussian distributions: we have so far assumed source distributions are known. In general, we need to choose an appropriate foorm. $z$ is super-Gaussian, if kurtosis of $z$ is positive:
	\begin{equation}
	\text{kurt}(z) = \mu_4 / \sigma^4 - 3
	\end{equation}
	This means $z$ has long tail (spike near mean). Ex. Laplacian distribution. We say $z$ is sub-Gaussian, if $\text{kurt}(z) < 0$. Other possibilities are skewed distribution. In practice, super-Gaussian distribution is common. 
	
	\item EM algorithm: another strategy is to estimate the source densities, which are assumed to follow a mixture of normal. Key observation of the EM is: we can computer $\E(z_t|x_t, \theta)$. And then given $z_t$'s, we can estimate the parameters of mixture normal. These two steps will be alternated. 
	
	\item Remark: in the analysis, data is assumed to be whitened first, and then $W$ is orthogonal. So the dimension of $x$ is different from the original data. This would affect interpretation of $W$. 
\end{itemize}

Sparse coding [Murphy, 13.8]
\begin{itemize}
	\item Motivation: (1) Topic model: each document, a set of words, covers multiple topics. (2) Image analysis: each patch represents one or more “topics” (content). The number of topics can be very large. (3) Transcriptome of cells: each cell can express any type of programs (or combination). 
	
	\item Topic model by factor analysis: let $x_i$ be our document/image patch/etc, which is a vector of word counts. Let $z_i$ be the topics of the document. We can model $x_{ij}$, the word count of document $i$, as summation of the contribution of each topic on word $j$, $x_i = W z_i$. The matrix $W$ is known as Dictionary. 
	
	\item Sparsity inducing prior on $z_i$: when the number of topics is large, it may be reasonable to assume $z_i$ is sparse. Our problem is to infer both $W$ and $z_i$, and this can be approximated by minimizing over $W$ and $Z$: 
	\begin{equation}
	-\log p(D|W, Z) = \frac{1}{2} \sum_{i} \norm{x_i - W z_i}^2 + \lambda \sum_i \norm{z_i}_1 
	\end{equation}
	Optimization by iterative algorithm: When $W$ is given, minimizing $Z$ is just Lasso; when $Z$ is given, minimizing $W$ is least square. 
	
	\item Connection with basis functions/wavelet analysis: we can view $\{x_i\}$ as a set of data points (in spatial and temporal dimensions), then they can be thought of as linear combination of underlying signals, e.g. wavelets. 
	
	\item Compressed sensing (CS): e.g. in MRI, we do not observe $x_i$ directly, but rather linear combination $y_i = R x_i + \epsilon_i$. The goal is to infer $x_i$ from $y_i$ and given $R$. The idea of CS is that we leverage prior of $x_i = W z_i$, where $W$ can be a (large) dictionary, and $z_i$ sparse-inducing prior. 
	
	\item Application in image processing: we use a large database of images to train a Dictionary first: the ``topic'' of each possible patch. Then for any given image, we can learn the topics of each patch. Image inpainting: removing text in an image. We model image patches with the sparse coding model: some topics are images, and the rest are text. We select only the topics related to image to reconstruct. 
\end{itemize}

Enter the Matrix: Factorization Uncovers Knowledge from Omics [O’Brien and Fergit, TIG, 2018]
\begin{itemize}
	\item Behavior of PCA: tries to explain variations using a small number of factors, while biological pathways may be relatively uniform. So a PC may mix signals from multiple processes.
	
	\item Behavior of ICA and NMF: better than PCA in associating factors with processes. ICA: may have both over- and under-representation of genes, while the non-negative constraint of NMF may help avoiding under-representation.
	
	\item Hierarchical nature of factors: e.g. ICA on tumor and normal samples, using two factors only separate tumor vs. normal; use more factors may separate subtypes of tumor samples.
	
	\item Using factors to find biomarkers of sample patterns (e.g. subtypes): associate factors with subtypes, then associate genes with factors.
\end{itemize}

Analysis of population structure: a unifying framework and novel methods based on sparse factor analysis (SFA) [Engelhardt and Stephens, PLG, 2010]
\begin{itemize}
	\item Factor analysis: Let $G$ be $n \times p$ genotype matrix, where $G \in \{0,1,2\}$. Factor analysis can be generally written as: 
	\begin{equation}
	\E(G) = \Lambda F \Leftrightarrow \E(G_{ij}) = \sum_{k=1}^K \Lambda_{ik} F_{kj}
	\end{equation}
	where $\Lambda$ is $n \times K$ matrix, calling \textit{factor loading}, and $F$ is $K \times p$ matrix, called \textit{factors}. The term factor loading means: how each factor is loaded to a sample. See (Figure 1) for the dimensions of $\Lambda$ and $F$. 
	
	\item Factor analysis vs. PCA and mixed membership model: The general model reduces to the mixed membership model or PCA depending on the constraints on $\Lambda$ and $F$. (1) Mixed membership model: 
	\begin{equation}
	G_{ij} \sim \text{Bin}(2, r_{ij}) \qquad r_{ij} = \sum_k \Lambda_{ik} F_{kj}
	\end{equation}
	where $\Lambda_{ik}$ is the population composition of sample $i$, and $F_{kj}$ is the frequency of SNP $j$ in population $k$. (2) PCA: we usually assume standardized genotype matrix $G_{ij}$, then we have: $G_{ij} \sim N((\Lambda F)_{ij}, \psi^{-1})$. The constraints are: $K$ rows of $F$ are orthonomal and $K$ columns of $\Lambda$ are orthogonal. 
	
	\item Sparse factor analysis (SFA): the key idea is to induce sparsity of $\Lambda$: each sample is represented as a linear combination of a \textit{small} number of latent factors. Specifically, the model is: 
	\begin{equation}
	G_{ij} = \mu_j + \sum_{k=1}^K \Lambda_{ik} F_{kj} + \epsilon_{ij}
	\end{equation}
	where $\epsilon_{ij} \sim N(0, \psi_i^{-1})$. The model with $\mu$ is called SFAm, and with $\mu = 0$ is SFA. The ARD prior encourages sparsity of $\Lambda$ by: 
	\begin{equation}
	\Lambda_{ik} \sim N(0, \sigma_{ik}^2)
	\end{equation}
	Intuition: similar to ridge regression, mean 0 encourages small values of $\Lambda_{ik}$. In other words, suppose we know $\sigma_{ik}^2$, our model will shrink $\Lambda_{ik}$ towards 0, with the extent of shrinkage determined by $\sigma_{ik}^2$. With this model, we can integrate out $\Lambda$. We write the genotype of sample $i$ as: 
	\begin{equation}
	G_i = \mu + F^T \Lambda_i + \epsilon_i
	\end{equation}
	where $\Lambda_i \sim N(0, \Sigma_i)$ ($\Sigma_i$ is diagonal matrix with diagonal elements $\sigma_{ik}^2$), and $\epsilon_i \sim N(0, \psi_i^{-1} I_p)$. Using properties of MVN, we have: 
	\begin{equation}
	G_i \sim N(\mu, F^T \Sigma_i F + \Psi_i^{-1})
	\end{equation}
	where $\Psi_i^{-1} = \psi_{i}^{-1} I_p$. 
	
	\item Inference of SFA: our unknowns are $\mu, F, \Sigma, \Psi$, and $\Lambda$ is missing variable. Note that main variables of interest are $F$ and $\Sigma$ or $\Lambda$. The inference (ECME algorithm) has two parts: (1) Suppose $\Sigma$ is given, we can do standard EM, to update $\mu, F, \Psi$, treating $\Lambda$ as missing data. This involves maximizing the expected log-likelihood, where expectation is taken over $\Lambda$. The $Q$ function is given by Equation (11) and (12) in the paper. (2) Suppose the other parameters are given, to update $\Sigma$, we can maximizing the marginal likelihood, marginalizing $\Lambda$. 
	
	\item Behaviors of three models in discrete populations (admixture): simulations with three populations (Figure 3), the results: SFA and admixture show interpretable factors, with loading close to 0 if a sample does not belong to a population, but PCA does not have the pattern. Remark: PCA requires factors to be orthogonal, which is not the case here.   
	
	\item Behaviors of three models in continuous populations: (1) 1D isolation-by-distance. PCA: first factor is mean AF, and the second factor is the deviation from the mean - roughly location of a sample relative to the center. SFA and admixture: 2 factors are AFs at either end. (2) 2D isolation-by-distance. PCA: first factor is mean, and the second, third factors more closely capture spatial dimensions (diagonal), similar to SFAm. But SFA and admixture different. 
	
	\item Remark: can we understand the behavior of PCA in the case of discrete populations? What linear combinations (of the admixed populations) will PCA find in order to satisfy orthogonality? 
\end{itemize}

A Bayesian Framework to Account for Complex Non-Genetic Factors in Gene Expression Levels Greatly Increases Power in eQTL Studies (PEER) [Stegle and Winn, PLCB, 2010]
\begin{itemize}
	\item Background: (1) PCAsig: complexity control via significance testing of eigenvalues. (2) SVA: similar to PCASig for complexity control, also per-gene noise model and allow for sparse non-orthogonal components. 
	
	\item VBQTL model ideas: gene-specific noise model; joint inference of genetic effects, known and hidden factors; ARD prior (shrinkage) estimates of the effects of known and hidden factors.
	
	\item Model: let $y_{gj}$ be the expression of gene $g$ of sample $j$. The error is $1 / \tau_g$. The expression has three parts, from: genetic effect, known factors and hidden factors, denoted as $y_{gj}^{(1-3)}$. For genetic effects, consider $N$ SNPs, let $s_{nj}$ be the genotype of SNP $n$ in sample $j$, and $b_{ng}$ be the indicator of whether SNP $n$ is associated with gene $g$, and $u_{ng}$ its effect size. We have:
	\begin{equation}
	\E(y_{gj}^{(1)}) = \sum_{n=1}^N b_{ng} n_{ng} s_{nj} \qquad b_{ng} \sim \text{Bern}(p), u_{ng} \sim N(0, 1)
	\end{equation}	
	For known factors, denoted as $f_{cj}$ for factor $c$, we have: 
	\begin{equation}
	\E(y_{gj}^{(2)}) = \sum_c \nu_{gc} f_{cj} \qquad \nu_{gc} \sim N(0, 1/\alpha_c)
	\end{equation}
	And Gamma prior for $\alpha_c$.  Hidden factor model, let $x_{kj}$ be the $k$-th factor in sample $j$, we have: 
	\begin{equation}
	\E(y_{gj}^{(3)}) = \sum_k w_{gk} x_{kj} \qquad w_{gk} \sim N(0, 1/\beta_k) \qquad x_{kj} \sim N(0,1)
	\end{equation}
	Similarly Gamma prior for $\beta_k$. ARD prior: note that $\beta_k$ represents the prior importance of factor $k$ (variance explained), when it is large (low effects), it will drive $w_{gk}$ towards 0; and when it is small (large effects), it has less shrinkage towards 0 - hence named ARD.  
	
	\item Inference: VB (Figure 2). Given other parameters, regress out their effects, and update the effect size parameters for hidden or known factors. Initial values from MLE. Two versions: fVBQTL - single update of the full model. It is appropriate when known and hidden factors are unrelated to genetics. iVBQTL: iterative update. 
	
	\item Simulation: (1) Include 10 factors, 7 non-genetic and 3 genetic factors. (2) Comparison of recovered hidden factors (MSE). (3) Comparison of eQTL discovery: both cis- and trans. VB-QTL much better at trans-eQTL and also better at cis-eQTL. 
	
	\item Results in cis-eQTL mapping: improve over PCA even when the same amount of variance is explained (Figure 4c).
	
	\item Results in trans-eQTL mapping: in yeast data, adjusting for hidden factors reduces the power (Figure 4d). This means some PCs are heritable, thus mediates genetic effects. Regressing out these PCs reduces the genetic effects. 
	
	\item Remark: in classical Factor Analysis, $W$ is fixed, and estimated via EM. Here, $W$ is random, and shrinked towards 0 via ARD prior.
	
	\item Reference: details of Bayesian SFA at, ``Inference algorithms and learning theory for Bayesian sparse factor analysis'', Journal of Physics: Conference Series. 
\end{itemize}

Joint Genetic Analysis of Gene Expression Data with Inferred Cellular Phenotypes (FAQTL) [Parts and Durbin, PLG, 2011]
\begin{itemize}
	\item Motivation: PCA, PEER, SVA mostly capture broad variations in the data. The goal here is to learn latent factors capturing cellular phenotypes.  
	
	\item Model: let $y_{g,j}$ be expression of gene $g$ in sample $j$, we express it as:
	\begin{equation}
	y_{g,j} = \mu_g + \sum_{k=1}^K w_{g,k} x_{k,j} + \psi_{g,j}
	\end{equation}
	where $x_{k,j}$ is the $k$-th latent factor in sample $j$, and $w_{k,g}$ the effect of factor $k$ on $g$. The latent factors $x_{k,j} \sim N(0,1)$. Use spike-and-slab prior for the loading:
	\begin{equation}
	w_{g,k} \sim (1-\pi_{g,k}) N(0, \sigma_0^2) + \pi_{g,k} N(0,1)
	\end{equation}
	where $\sigma_0$ is small, use $10^{-4}$. Note that it is assumed that $y$ has been normalized, so $w$ is also scaled to have variance 1. 

	\item Note: in the full model, $y_{g,j}$ also depends on genetic effects and interactions. 
		
	\item Set the prior $\pi_{g,k}$: the factors represent TF activity or pathways. 167 TFs and KEGG pathways. So we know if $g$ is assigned to factor $k$. The prior is then chosen to reflect the link: if $g$ and $k$ are linked, use a large prior for $\pi_{g,k}$ (use 0.99); otherwise a small number - capture FP rate: 0.06 for ChIP-seq data (likely FP rate in ChIP-seq) and 0.001 for KEGG. 
	
	\item Orthogonality of factors to experimental covariates: factors are shown mostly to represent pathways or TF activation, and not correlated with batches etc. 

	\item Statistical identifiability: informative priors reduce ambiguity of factors. Shown that the factors are largely independent. Also do 20 random initializations. 
	
	\item Association and interaction testing of latent variables: test the association of SNPs and latent factors. For each set of inferred latent factors, test association and obtain local FDR $Q$ values using permutations to get empirical null (of all SNPs and all genes). Then average over 20 random starts to get the average $q$ values. This is valid if we interpret local FDR as $P(\beta = 0 | D)$ and random starts are interpreted as posterior sample. 
	
	\item Inferred factors are often genetically driven: (1) TF factors: association with genotypes and environment. Ex: PHO84 locus $>$ PHO4 targets. Another example, some locus is a cis-eQTL of the TF. (2) Pathway factors: LYS2 lous $>$ lyn. biosynthetic pathway. 
	
	\item Inferred TF activity correlate with expression: only 27/167 factors.
	
	\item Interactions of locus and inferred factors in determining gene expression: several TFs involved in metabolism, stress and IRF2 (stress response) show interactions in determining expression of some genes. Note: some interactions represent gene-environment interactions. 
		
	\item Remark: the hyperparameters of the prior of loading matrix are not learned. 
\end{itemize}

Empirical Bayes Matrix Factorization (FLASH) [Wang and Stephens, arxiv, 2018]
\begin{itemize}	
	\item Motivation: common distribution (to be learned from data) of $l$ (loading of factors on individuals) and $f$ (factors represented in terms of observed variables). The factors can be dense (affecting many variables) or sparse (affecting a small number). Ex: (1) Stock market: factors are not sparse, only factor $>$ stock is sparse. (2) Gene expression: both [TF] and TF $>$ gene are sparse.
	
	\item Note: the notations are different from other sources, such as Wiki and [Murphy]. 
	
	\item Single factor model: let $Y$ be $n \times p$ data matrix, let $l$ be the vector of loading ($Z$ in the common notation of factor analysis) and $f$ be the vector of factors ($W$ in the common notation), we have: 
	\begin{equation}
	Y = l f^T + E \qquad l_1, \cdots, l_n \sim g_l \qquad f_1, \cdots, f_p \sim g_f \qquad E_{ij} \sim N(0, 1/\tau_{ij})
	\end{equation}
	where $g_f$ and $g_l$ are common distributions (e.g. ASH or normal). $\tau = (\tau_{ij})$ is unknown but can have some structure, e.g. the same for each column. The role of $g_f$ and $g_l$ is to impose regularization on the factors and loading. 
	
	\item Variational inference: the goal is to infer the posterior $P(l, f, g_l, g_f, \tau | Y)$. Do this by Variational inference, we approximate the posterior distribution of $l$ and $f$ as $q(l,f)$, assuming independence of each component: 
	\begin{equation}
	q(l,f) = \prod_i q_i(l_i) \prod_j q_j(f_j)
	\end{equation}
	The problem is then to maximize ELBO over $q$: $F(q_l, q_f, g_l, g_f, \tau)$. This is done via an algorithm alternating between $l$ and $f$. When the distribution of $l$ is given (posterior of $l$ from the previous iteration), inference of $f$ is reduced to ASH type problem (EB normal mean problem): given the prior of $f$ in $g_f$, and the data given $f$, $Y_f$, follows normal distribution (parameterized by $l$), we can infer $g_f$ and $f$ posterior. Similarly, we can infer $l$ when the distribution of $f$ is given. 
	
	\item Reducing FLASH to EBNM problem: first, we state EBNM problem as: 
	\begin{equation}
	Y_i \sim N(\mu_i, s_i^2) \qquad \mu_i \sim g(\cdot)
	\end{equation}
	where $s_i^2$ are known. The goal is to infer $\mu_i$. Next, for FLASH, let's assume $f$ is given, since $K = 1$, we have:
	\begin{equation}
	Y_{ij} \sim N(l_i f_j, s_{ij}^2) \qquad l_i \sim g_l
	\end{equation}
	where $s_{ij}^2$ is also known. This is $N$ independent linear regression. We can derive $\hat{l}_i$ as MLE of $l_i$, which is a function of $Y_{ij}$ and $f$ - this is a sufficient statistic. Then $\hat{l}_l$ follows $N(l_i, s_i^2)$, this reduces to EBNM problem. 
	
	\item $K$-factor model: (1) greedy algorithm, start with 1 factor, then add factors 2, 3, etc. (2) backfitting algorithm, iteratively refines the estimates for each factor given the estimates for the other factors. Selection of $K$: the estimation of $g_l$ and $g_f$ could lead to 0 as MLE. The algorithm then stops. 
	
	\item Missing data: if ``missing at random'', in the VB iteration update, simply ignore missing data. This is implemented by setting $\tau_{ij} = 0$ for the missing entries (infinite variance, which leads to flat likelihood). 
	
	\item Orthogonal cross-validation (OCV): for selection of hyperparameters (other methods) and comparison between methods. E.x. 3-fold OCV: permutation of row and column indices, and create held-out data as balanced part of data matrix - Appendix B. Use the training data to fit the model (treat the rest as missing data), and then use the factors and loading for the held-out part to compute missing entries and compare with the observed values.   
	
	\item Simulation: start with 1 factor simulation, evaluate results by the low-rank structure $B = lf^T$, compare the estimated vs. the actual values by RRMSE, defined as:
	\begin{equation}
	RRMSE = \sqrt{ \frac{\sum_i (\hat{B}_i - B_i)^2}{ \sum_i B_i^2}}
	\end{equation}
	Simulation setting: $N = 200, p = 300$, $l_i$ ASH prior with $\pi_0 = 0.9, 0.3, 0$, and effect size variance $(0.25, 0.5, 1, 2, 4)$ with equal weights. And $f_i \sim N(0,1)$. Error variance $\tau = 1, 1/16, 1/25$ under the three values of $\pi_0$.  
	
	\item Assessing performance in real data: compare performance of imputing missing data. 
	
	\item Remark: in VB iteration, we should be given the distribution (posterior) of $f$, rather than the values of $f$, so this is similar to measure-error regression, rather than standard regression?
\end{itemize}

Covariate-dependent negative binomial factor analysis of RNA sequencing data (dNBFA) [Dadaneh and Qian, Bioinfo, 2018]
\begin{itemize}
	\item NBFA model: let $n_j$ be expression vector of $V$ genes in sample $j$. We model it as:
	\begin{equation}
	n_j \sim NB(\Phi \theta_j, p_j)
	\end{equation}
	where $\Phi$ is the factor loading matrix (factor-to-gene effects), and $\Theta$ is factor score matrix. $p_j$ is a parameter to account for overdispersion. Use a Dirichlet prior for $\Phi$: 
	\begin{equation}
	(\phi_{1k}, \cdots, \phi_{vk}) \sim \text{Dir}(\eta, \cdots, \eta)
	\end{equation}
	where $\eta$ controls smoothness, with small $\eta$ favors more specific/sparse factors (most concentrate on a small number of genes). 
	
	\item dNBFA model: allow factor scores to depend on covariates of sample $j$, $x_j$. Model:
	\begin{equation}
	\theta_{kj} \sim \text{Gamma}(r_k, e^{\beta_k^T x_j})
	\end{equation}
	where $r_k$ is the mean of factor $k$ across all samples, and $\beta_k$ the coefficients of covariates. Prior of $\beta_k$: mixture of normal distribution. Conjugate prior on hyperparameters. 
	
	\item Inference: Gibbs sampling. 3000 MCMC iterations, where after the first 1000 burn-in iterations. Number of factors $K = 250$ initially, and then only the top 100 factors with non-negligible baseline expressions were kept for further analyses. For evaluting module memebership, use only top 20 genes by $\phi_{vk}$. 
	
	\item TCGA data analysis: BRCA, lung cancer. To evaluate modules: extract eigen-gene of each module, and test association with disease factor by t-test. Comparison with WGCNA. (1) Higher associations of disease factors (cancer vs. normal) with modules - p-value distribution (Fig. 3). (2) Compare factor scores across two sample groups (Fig. 4). 
	
	\item ASD data analysis: 20-30 ASD and control expression in 3 brain regions. Covariates: age, sex and brain regions. Found stronger associations of modules in dNBFA vs. NBFA (Fig. 5).  
\end{itemize}

%%%%%%%%%%%%%%%%%%%%%%%%%%%%%%%%%%%%%%%%%%%%%%%%%%%%%%%%%%%%
\subsection{Canonical Correlation Analysis (CCA)}

CCA: 
\begin{itemize}
	\item Problem: let $x$ be $m$-dim. and $y$ be $n$-dim. vector. We believe there is a common structure between $x$ and $y$. Ex. $x$, arithmetic speed and arithmetic power are related to $y$, reading speed and reading power. The goal is to find projections $a^T x$ and $b^T y$, where $a, b$ are $m$ and $n$-dim. vectors, s.t. $a^T x$ and $y^T y$ have max. correlation. The idea can be extended: once we find the first projection, we would like to find the second projection that is independent of the first projection, and max. the remaining correlation. 
	
	\item Finding the first canonical correlation vector: see [Anderson, An Introduction to Multivariate Statistics, 3ed]. We first consider the first canonical correlation vectors, denoted as $a$ and $b$. Our problem is to maximize $\rho(a, b)$. It is given by: 
	\begin{equation}
	\rho(a, b) = \frac{a^T \Sigma_{xy} b}{(a^T \Sigma_{xx} a)^{1/2} (b^T \Sigma_{yy} b)^{1/2}}
	\end{equation}
	This is equivalent to solving this problem: 
	\begin{equation}
	\text{maximize } a^T \Sigma_{xy} b, \text{  subject to  } a^T \Sigma_{xx} a = 1, b^T \Sigma_{yy} b = 1
	\end{equation}
	We can solve this using Lagrange's Multiplier. Define the function:
	\begin{equation}
	f(a, b, \lambda, \mu) = a^T \Sigma_{xy} b - \frac{\lambda}{2} (a^T \Sigma_{xx} a - 1) - \frac{\mu}{2} (b^T \Sigma_{yy} b - 1)
	\end{equation}
	Taking derivative wrt. $a$ and $b$:
	\begin{equation}
	\left\{
	\begin{array}{l}
	\Sigma_{xy} b - \lambda \Sigma_{xx} a = 0 \\
	\Sigma_{xy}^T a - \mu \Sigma_{yy} b = 0 \\	
	\end{array}
	\right.
	\end{equation}
	With some algebra, we can show that $\lambda = \mu = a^T \Sigma_{xy} b$. Plugging in this to the equations can solve $\lambda$. To extend the results to next canonical correlation vector: we add the constraint that the vectors must be orthogonal to previous vectors. 
	
	\item CCA results: Section 16.1 [Hardle et al. Applied Multivariate Statistical Analysis, 4ed]. Let $\Sigma_{xx}, \Sigma_{yy}$ be covariance of $x$ and $y$, respectively, and $\Sigma_{xy}$ be the covariance between $x$ and $y$. We define:
	\begin{equation}
	K = \Sigma_{xx}^{-\frac{1}{2}} \Sigma_{xy} \Sigma_{yy}^{-\frac{1}{2}}
	\end{equation}
	And we denote $\gamma_i$, $\delta_i$ be the $i$-th eigenvectors of $K K^T$ and $K^T K$, respectively (assuming eigevalues are sorted from largest to smallest). Then our canonical correlation vectors are given by:
	\begin{equation}
	a_i = \Sigma_{xx}^{-\frac{1}{2}} \gamma_i \qquad b_i = \Sigma_{yy}^{-\frac{1}{2}} \delta_i
	\end{equation}
		
	\item Analogy with PCA: we consider the (normalized) covariance between $X$ and $Y$, denoted as $K$. The canonical correlation vectors are singular vectors of $K$ and $K^T$, respectively. 
	
	\item Probabilistic interpretations of CCA: see [Bach et al, A Probabilistic Interpretation of Canonical Correlation Analysis], \url{https://www.di.ens.fr/~fbach/probacca.pdf}. Let $x_1$ and $x_2$ be data of $m_1, m_2$ dimensions, respectively. CCA is the MLE of the following model: 
	\begin{equation}
	\begin{array}{ll}
	z \sim N(0, I_d) & d \leq \min(m_1, m_2) \\
	x_1 | z \sim N(W_1 z + \mu_1, \Psi_1) & W_1 \in \mathbf{R}^{m_1 \times d}\\
	x_2 | z \sim N(W_2 z + \mu_2, \Psi_2) & W_2 \in \mathbf{R}^{m_2 \times d}\\	
	\end{array}	
	\end{equation}
\end{itemize}
%%%%%%%%%%%%%%%%%%%%%%%%%%%%%%%%%%%%%%%%%%%%%%%%%%%%%%%%%%%%
\subsection{Mixed-Membership Model}

Topic models overview:
\begin{itemize}
	\item Motivation: in mixture model, each object belongs to one of multiple classes, each class with different distributions. However, we may want to model the case, where each object may belong to multiple classes, or each object is a result of ``actions/contributions'' of multiple classes. Ex. genotype of one individual: may come from multiple ancestral populations. 
	
	\item Mixed-membership model: our basic data is an object, e.g. a vector of words, a set of pixels, expression of multiple genes, and so on. Each object can be decomposed as resulting from multiple components; or each element of the object can be viewed as a mixture of some underlying distributions. Examples: 
	\begin{itemize}
		\item Text analysis: each document, instead of having a specific model, consists of a mixed set of topics, in varying proportions. 
		\item Genetics: each individual (a set of genetic polymorphisms) consists of a mixed set of ancestral genotypes - some part of genome may come from one ancestral population, and some other part may be from another ancestral population, etc. 
		\item Computer vision: an image consists of a mixed set of objects, in varying proportions. 
		\item Gene expression: expression of a set of cells results from expression of individual types of cells and proportion of cell populations. 
	\end{itemize}
	The model is characterized by: (1) component distributions; (2) proportion of components. A simple way of making inference is non-negative matrix factorization (NMF). 
	
	\item Comparison of mixed-membership and mixture models: in the mixture model, an object belongs to only one class (even though the class membership is not observed); in contrast, in the mixed-membership model, an object itself (usually a group, e.g. document, image) is a mix of multiple classes. Alternatively, in mixed-membership model, each object (document) is a mixture; but different objects (documents) share the same component distributions, with varying mixing proportions.  
	
	\item Mixed-membership model is a special case of the hierarchical model: the group model is a mixture of distribution, this is similar to hierarchical linear model, where we model the regression coefficient or slope of a group, as a linear function of some other variables of the group. 
	
	\item Latent variable model: the general idea of the topic model is to use a small set of latent variables to explain the observations. The idea can be applied even if there is no obvious group structure, e.g. all the documents are concatenated. 
\end{itemize}

Latent Dirichlet Allocation (LDA) model
\begin{itemize}
	\item Ref: [Introduction to Probabilistic Topic Models, Blei, 2011] \url{https://www.cs.cmu.edu/~mgormley/courses/10701-f16/slides/lecture20-topic-models.pdf}
	
	\item Dirichlet-Multinomial Mixture Model: the intuition is for any document, we first sample a topic, and then generate words according to the topic. Formally, suppose there are $K$ topics in total, where each topic model defines a multinomial distribution on the dictionary, $\phi_k$, with $\phi_k \sim \text{Dir}(\beta)$. Let $\theta$ be the proportion of these topics, $\theta \sim \text{Dir}(\alpha)$. Then for document $m$, we first sample its topic $Z_m$ according to $\theta$. Then we sample words at each position $n$, $x_{mn} \sim \text{Mul}(1, \phi_{Z_m})$. 
	
	\item Latent Dirichlet Allocation: Dirichlet-Multinomial Admixture model. For a document $m$, the topic proportions are $\theta_m \sim \text{Dir}(\alpha)$. For a word at position $n$, we first sample its topic, $Z_{mn} \sim \text{Mul}(1,\theta_m)$, and then sample the word $x_{mn} \sim \text{Mul}(1, \phi_{Z_{mn}})$. The model can be simply written as: 
	\begin{equation}
	\alpha \rightarrow \theta_m \rightarrow (Z_{mn} \rightarrow x_{mn}) \leftarrow \phi_k \leftarrow \beta
	\end{equation}
	where the parenthesis means repeat for every word of the document. And note that $Z_{mn}$ is usually a $K$-dim vector (with only one element 1 and the rest 0), instead of a category variable. 
	
	\item Inference: our unknowns are $(Z, \theta_m, \phi_k)$. The Collapsed Gibbs sampler sample from $p(Z|X, \alpha, \beta)$, where $\theta$ and $\phi$ are marginalized. The algorithm iteratively sample $Z_{i}$ (topic of a word, for simplicity, use $i$ instead of $mn$), assuming everything else $Z^{-i}$ is given. The conditional $p(z_i = k | Z^{-i}, X, \alpha, \beta)$ is basically the product of: (1) the probability of topic $k$ is in document based on other words; (2) the probability the topic $k$ contains the word at position $i$. 
	
	\item Extensions: in several directions: 
	\begin{itemize}
		\item Structure of documents: documents may be associated with other data, e.g. authorship, time, networks (link), etc.
		\item Topic structure: correlation and interaction between topics, tree structure of topics, etc. 
		\item Topic model representation: instead of bag-of-words, allow sequential order of words, entities, etc. 
		\item Sparsity: learn sparse set of topics. This may be associated with other tasks. 
	\end{itemize}
	
	\item Applications of LDA (mixed-membership model): population genetics, computer vision (see above). 
\end{itemize}

Extensions of LDA: 
\begin{itemize}
	\item Author-topic model [Rosen-Ziv \& Smyth, UAI, 2004]: each author is associated with a set of topics (with varying proportions), $\theta_a$; and for each document (whose authors are known), its topic proportions is a uniform mixture of the topics of all its authors. Or at the word level: first sample an author (uniformly), then sample the topic from this author, then sample the word. 
	
	\item Dynamic topic model [Blei \& Lafferty, ICML, 2006]: the topic model (words in a certain topic) and the corpus-level topic proportions change over time. The dsitributions $\alpha_t$ (prior parameter of $\theta$) and $\beta_t$ are modeled as random walk: 
	\begin{equation}
	\alpha_t | \alpha_{t-1} \sim N(\alpha_{t-1}, \sigma^2 I)	
	\end{equation}
	and similarly for $\beta_t$. 
	
	\item Relational topic model [Chang \& Blei, Hierarchical relational models for document networks, AAS, 2010]: the topics of neighboring documents should be close. This is modeled as: the link variable (whether two documents are linked) depends on the topic similarity between the two. Define variable $y_{d,d'}$ for each document pair, and $y_{d,d'}$ is modeled as logistic model:
	\begin{equation}
	P(Y=1) = \sigma(\eta^T  (\bar{z}_d \circ \bar{z}_{d'}))
	\end{equation}
	where $\bar{z}_d$ and $\bar{z}_{d'}$ are the observed topic proportions of $d$ and $d'$, and $\circ$ means inner product or similarity. 
\end{itemize}

Supervised LDA: 
\begin{itemize}
	\item Motivations: why a good way of document classification
	\begin{itemize}
		\item Unsupervised LDA followed by regression on topics: the problem is the topics discovered may not correspond to class labels. Ex. for classifying movie reviews (positive vs. negative), unsupervised LDA may find topics related to movie genres. 
		
		\item Regression using words: the document class depends on topics, and by regression on topics, one actually pool the (small) effects of many words of a topic, thus improve inference. 
	\end{itemize}
	
	\item Model: suppose $Y_d$ is the label of document $d$, we modeled it as dependent on the topic proportion of $d$, $\bar{z}_d$: $Y_d \sim N(\eta^T \bar{z}_d, \sigma^2)$, where $\eta$ represents the effect of each topic. 
	
	\item Alternative models: 
	\begin{itemize}
		\item Separate modeling of positive and negative classes: the topics of the two classes are not directly related, while in reality, many topics of the two classes are probably shared with only some topics different. 
		
		\item Modeling class-specific models: each topic is assigned to be general or class-specific. The drawback of this strategy is that the assumption may be too strong, i.e. the difference between the two classes lies more in the importance of topics, rather than presence/absence. 
		
		\item Modeling $Y_d$ as functions of $\theta_d$ (the average topic proportion): this would be equivalent to some regularization on topic proportion variables. However, conceptually, $Y_d$ depends on specific topics discussed in a document.  
	\end{itemize}
\end{itemize}

Topic flow model [TopicFlow model: Unsupervised learning of topic specific influences of hyperlinked documents. AI-STAT, 2011]
\begin{itemize}
	\item Idea: in a citation/hyperlink network, the topics of the linked documents are shared. This would allow better inference of topics of a document. Develop a flow model that capture the topic relations among network documents. 
	
	\item Model: parameterize with f to represent the topic influence between linked documents (f perhaps influences the topic distribution of linked documents), then optimize the likelihood as usual. 
	
	\item Question: is flow is a good metaphor? What does flow balance means? Ex. say $A \rightarrow B \rightarrow C$, require the flow of $A \rightarrow B$ to be equal to that of $B \rightarrow C$, however, the topical influences in the two cases are generally independent. 
\end{itemize}

Model based visualization of structure in biological data [Kushal Key, Thesis defense, 2018]
\begin{itemize}
	\item Part I. cell type compositions in bulk tissue. Topic model for gene expression: $p_{ij}$ is the expression of sample $i$, gene $j$. The read counts are related to $p_{ij}$ via multinomial model. The value of $p_{ij}$ is $\sum_k \omega_{ik} \theta_{kj}$, where $\omega_{ik}$ and $\theta_{kj}$ sum to 1 for $k$.
	
	\item Application to GTEx: fitting all samples together with K=20, clusters learned are generally tissue-specific with relevant biological functions. Also refined structure using only brain expression data.
	
	\item Q: Adjusting for batches and population ancestry? Answer: no, difficult with multinomial model. So some of the clusters may capture these covariants.
	
	\item Q: Compare the inferred clusters with actual transcriptome data, e.g. scRNA-seq? Is there any way to use the reference transcriptome to guide the search of the factors?
	
	\item Part II. Ancient DNA. DNA damage patterns (sequencing errors): ancient, spike of C$>$T near 5’ ends.
	
	\item Important features of C$>$T: two flanking bases, the base immediately before the break point (DNA fragment), distance to the 5’ end.
	
	\item Model: $x_{ij}$, for sample $i$, and position/mismatch $j$, five features. Model distribution of 5 features (mutational spectrum: distribution of features/contexts of all mutations) as mixed membership model.
	
	\item Learn several mutation profiles: one of them is modern DNA, another reflects ancient DNA damage pattern.
	
	\item UDG treatment: removing C$>$T mismatches.
	
	\item Q: Mismatches in modern DNA due to DNA damage during lib. prep. or sequencing errors?
	
	\item Part III. model bird distributions. Data: species presence x samples (locations). Fit STRUCTURE model: sample x clusters, cluster x species. Results: Ex. blue cluster: coastal birds.
	
	\item Lesson: mixed membership model is related to both mixture model and PCA. (1) If we have a mixture model, it may be natural to extend to mixed membership if the mixture weights (but not components) vary. (2) Relation to PCA: our data can be explained as composition (linear combination) of some unobserved structure. Ex. gene expression data across conditions: composition of multiple transcriptional programs (say each corresponds to a TF).
	
	\item \textbf{Remark}: in cell type deconvolution problem (and general problem of learning structure), it may be important to adjust for known covariates. However, need to consider if the known covariates can influence the factors/cell types. 
\end{itemize}

%%%%%%%%%%%%%%%%%%%%%%%%%%%%%%%%%%%%%%%%%%%%%%%%%%%%%%%%%%%%
%%%%%%%%%%%%%%%%%%%%%%%%%%%%%%%%%%%%%%%%%%%%%%%%%%%%%%%%%%%%
\section{Multiple Hypothesis Testing}
\subsection{Frequentist Approach}

Reference: [Applied Statistical Genetics, Chapter 4; Storey \& Tibshirani, PNAS, 2003; Efron, Chapter 2-4]
	
Multiple hypothesis testing problem: 
\begin{itemize}
\item Basic strategy: consider a set of $m$ tests with a fixed procedure (decision rule). Each hypothesis is true or not (but fixed values). We imagine that the data are randomly generated according to the true value of the hypothesis, and the number of TPs, FPs, etc. are random variables (fixed rule on random dataset, and fixed hypothesis). We will formulate the errors, etc. in terms of the distribution of these RVs. 

\item Problem: suppose we are testing $m$ hypothesis $H_{0i}$ vs. $H_{1i}, 1 \leq i \leq m$. We denote $h_i \in \{0,1\}$ the truth: it is 1 if $H_{1i}$ is true and 0 if $H_{0i}$ is true (these are parameters, and not considered random in the frequentist interpretation). The test statistic of the $i$-th hypothesis is $T_i$, and we decide whether $H_{0i}$ is accepted or rejected based on $T_i$ at level $\alpha$: we obtain the $p$-value of $T_i$, $p_i$, and define
\begin{equation}
\tilde{h}_i = [p_i \leq \alpha]	
\end{equation}
Thus $\tilde{h}_i$ is an estimator of $h_i$ under frequentist interpretation (thus random variable). 

\item TP, FP, TN, FP: they are defined based on $h_i$ and $\tilde{h}_i$: 
\begin{equation}
TN = \#[h_i = 0, \tilde{h}_i =0] = U = m_0 - V \qquad 	FP = \#[h_i = 0, \tilde{h}_i =1]	= V
\end{equation}
where we use $V$ to denote the number of false discoveries, and $m_0$ is the total number of tests where null model is true. Similarly, 
\begin{equation}
FN = \#[h_i = 1, \tilde{h}_i =0] = T = m_1 - S \qquad 	TP = \#[h_i = 1, \tilde{h}_i =1]	= S
\end{equation}
where $S$ is the number of true discoveries, and $m_1$ is the total number of tests where alternative model is true. 

\item False discovery proportion (FDP): we call $R = S + V$ the number of positive predictions, and the FDP is defined as: 
\begin{equation}
FDP = V / R	
\end{equation}
The expectation of FDP is the false discovery rate (FDR):
\begin{equation}
\text{FDR} = \E\left( \frac{V}{R}\right)	
\end{equation}

\item Remark: relation to parameter estimation problem. In this case, $h_i$ is a parameter, and $T_i$ or $p_i$ are our data, and $\hat{h}_i$ is our estimator. Instead of assessing individual estimator, $\hat{h}_i$, we evaluate the overall performance over all estimators. Comparing with usual problems, the distribution of $\hat{h}_i$ under $h_i = 0$ is known, but under $h_i = 1$ is unknown. 

\item Remark: relation to machine learning: evulation of a prediction procedure. In machine learning, we assume we are estimating $y_i$ and a procedure predicts $\hat{y}_i$ for each $y_i$. The procedure can be evaluated by summing over the errors, with certain loss function (training error). In our problem, we assess $\hat{h}_i$ using a 0-1 loss function. 

\end{itemize}

Family-wide error rate (FWER): 
\begin{itemize}
\item Definition: the probability that any true null hypothesis is rejected: 
\begin{equation}
\text{FWER} = P(V \geq 1)
\end{equation}

\item Bonferroni correction: to control for FWED at $\alpha$, we reject null hypothesis for which $p_i \leq \tilde{\alpha}$, 
\begin{equation}
P(V \geq 1) = P\left(\bigcup_{i=1}^{m_0} [\tilde{h}_i = 1]\right)	\leq \sum_{i=1}^{m_0} P(\tilde{h}_i = 1 | h_i = 0) = \sum_{i =1}^{m_0} P\left(p_i \leq \tilde{\alpha}\right) = m_0 \tilde{\alpha} 
\end{equation}
Clearly, if we choose $\tilde{\alpha} = \alpha / N$, we could control FWER at $\alpha$. Equivalently, we could say that to control for FWER using Bonferroni correction, we adjust the $p$-values by: $\tilde{p}_i = m p_i$, and then apply the cutoff at $\alpha$. 

\item Sidak correction: the Bonferroni bound can be improved if the $N$ hypothesis are independent ($p_i$ are independent): suppose our rule is $p_i \leq \tilde{\alpha}$, then
\begin{equation}
P(a \geq 1) = P(\forall i \in N_0, \tilde{h}_i = 0)	= 1 - (1 - \tilde{\alpha})^{N_0} \leq \alpha
\end{equation}
Solving this equation, we have:
\begin{equation}
\tilde{\alpha} = 1 - (1 - \alpha)^{1/N}	
\end{equation}
The adjusted $p$-value can be shown as: 
\begin{equation}
\tilde{p}_i = 1 - (1 - p_i)^N	
\end{equation}

\item Weakness of FWER: in general, control for FWER tends to be very conservative in genome-wide settings: given that a reasonable number of significant findings will be reported, requiring only one FP would be too stringent. 
\end{itemize}
	  
False discovery rate (FDR):
\begin{itemize}
\item Definition: the expected proportion of FPs among all features predicted significant, thus it is simply precision in Information Retrieval. Formally: $FDR = E(\frac{V}{R})$. At $R = 0$, define $V/R = 0$. Thus we can write FDR as: 
\begin{equation}
FDR = E(V/R|R>0)P(R>0) + E(V/R|R=0)P(R=0) = E(V/R|R>0)P(R>0)	
\end{equation}
We call $E(V/R|R>0)$ the positive FDR (pFDR). If $m$ is large, we usually have $P(R > 0) \approx 1$, thus $pFDR \approx FDR$. 

\item Benjamini-Hochberg (B-H) adjustment: (Algorithm 4.21 in the book) adjust $p$ value by: $p_i m / i$, and sort all adjusted $p$ values, and choose those that are below a specified value $\alpha$. This would control FDR less than $\alpha$. Intuitively, suppose BH adjustment finds $i$ hypothesis to reject, then at the threshold $p_i$, the number of FPs is $V \approx m_0 p_i$ and the number of positive predictions is $R = i$. The FDR is thus approximately $V / R \approx m_0 p_i / i \leq \alpha$ (by how BH selects $i$). 

\item Calculating FDR: suppose our test statistic is $T$, and we want to compute $FDR$ at threshold $t$, assuming $T > t$ would indicate signifcance (rejection of $H_0$). We have: 
\begin{equation}
FDR(t) = E\left[\frac{V(t)}{R(t)}\right] \approx \frac{E[V(t)]}{E[R(t)]}	= \frac{m_0 \cdot P(T > t |H_0)}{R(t)}
\end{equation}
where $R(t)$ is the observed number of significant features and $P(T > t |H_0)$ can be estimated when null distribution of $T$ is known (e.g. $P$ value or permutation test). 

\item Extrapolation approach to estimate $m_0$: we draw the empirical distribution and the null distribution of $T$, and find the point where the two distributions diverge, denoted as $d$. Then $m_0$ can be extrapolated from the number of features below $d$ (let it be $k$): 
\begin{equation}
m_0 = \frac{k}{P(T < d)}	
\end{equation}

\item Estimation of FDR in real data: Suppose in the real data, we have $R(t)$ cases where $T > t$ is true. And the distribution $T|H_0$ can be computed, from random sampling, permutation, or from analytic computation, and assuming $m_0 \approx m$, we have $E(V(t)) = m \cdot P(T > t|H_0)$. Then we have $FDR = V(t)/R(t)$. 
\begin{itemize}
\item Simulation to obtain $V(t)$: instead of obtaining $T|H_0$ distribution, we can also obtain $V(t)$ directly by simulation. Ex. sample/permutate data s.t. none of the hypothesis tested is true, and compute $T$ for all tests, and count the cases where $T > t$. 
\item In the case when the statistic is $P$-value: then $T|H_0$ follows uniform distribution, and FDR can be easily calculated [Zhong \& Schadt, AJHG, 2010]. 
\item Examples: (1) binding site prediction: random permutation of motifs and predict motif matches [BLS, GR07]; (2) pathway statistic in GWAS: permutation of phenotype labels to obtain the null distribution of pathway statistic [Wang \& Bucan, AJHG, 2007]; (3) de novo mutations, the FDR at $2$ de novo events in a single gene: simulate mutations randomly and count the number of genes with $2$ hits. 
\end{itemize}

\item $q$ value: this is motivated by assessing the significance of any individual feature (just as $p$ value). Define $q$ value of a particular feature as the expected proportion of FPs among all significant predictions (i.e. FDR) if we call that feature significant. 

\item Remark:
\begin{itemize}
	\item Intuition for B-H adjustment: first sort all $p$ values, at $i = 1$, Bonferroni correction. Then at $i = 2$, since $i = 1$ is ``safe'' (below the threshold), then we could relax by using $2 \cdot q/m$ as threshold, and so on. 
	\item Both B-H adjustment and $q$ value calculation assumes independence or at least weak-independence of multiple features. 
\end{itemize} 
\end{itemize}

Summary statistics under multiple hypothesis testing: [Laird \& Lange, Fundamentals of Modern Statistical Genetics]
\begin{itemize}
\item Summary statisics: similar to FDR, we can use some other statistics to summarize the findings under multiple hypothesis testing. Ex. suppose $t_i$ is the test statistic of the $i$-th hypothesis, we could calculate: (1) $\min t_i$; (2) the number of tests with $t_i < t_C$ for some threshold $t_C$; etc. Then the significance of these statistics would suggest the validity of the test.  
\item Null distribution of summary statistics: often difficult to calculate, resampling is a common way of obtaining the $P$-value of the summary statistics. 
\end{itemize}

Example: search for proteins similar to a query protein in a database. Suppose the score function $S$ between the query and any database protien is given: 
\begin{itemize}
\item Extreme value theory: suppose the best match in the database search has score $S_{\max}$, we want to test if it is significant. The null distribution of $S_{\max}$ is simply the maximum value of $N$ independent RVs (where $N$ is the database size), each following a distribution under the hypothesis that the subject is unrelated to the query. This distribution of maximum can be approximated by the extreme value theory. The drawback of this approach: only for the best match in the database search. 

\item Bayesian approach: suppose $S$ represents the log-BF of the comparison of two hypothesis, then $S$ can be transformed to the posterior odds or the posterior probability that the subject is related to the query. Suppose we set a target threshold of posterior odds, then the threshold of $S$ is determined by the prior odds. We could choose a smaller value of prior odds when $N$ gets larger, to penalize for large databases; alternatively, we could fix the prior odds (assuming there is a constant ratio of true positives).  
\end{itemize}

Genome-Wide Significance Levels and Weighted Hypothesis Testing [Roeder \& Wasserman, Stat Sci, 2009]
\begin{itemize}
\item Weighted testing to control FWER: suppose we are testing $m$ hypothesis, the test statistics $T_j \sim N(\xi_j, 1)$, where $\xi_j$ is the parameter of the alternative model (unknown). The hypothesis being testsed are: $\xi_j = 0$. Let $w_j$ be the weight of the $j$-th hypothesis, then we control FWER at $\alpha$ if we choose threshold: 
\begin{equation}
\frac{p_j}{w_j} \leq \frac{\alpha}{m}
\end{equation}
as long as $\sum_j w_j = m$. 

\item Theoretically optimal weights: when $\xi_j$s' are known, the optimal weights that maximize the number of discoveries can be derived. The power of a single test is:
\begin{equation}
\pi(\xi_j, w_j) = P\left(T_j > \Phi^{-1}(\frac{\alpha w_j}{m}) \right) = \Phi(z_{\alpha w_j/m} - \xi_j)
\end{equation}
where $\Phi$ is the normal CDF. We are interested in maximizing the power over all tests, defined as: 
\begin{equation}
R(w) = \sum_j \pi(\xi_j, w_j) = \sum_j \Phi(z_{\alpha w_j/m} - \xi_j)  
\end{equation}
subject to $\sum_j w_j = m$. The optimal weights are given by: $w = (\rho(\xi_1), \cdots, \rho(\xi_m)$, where 
\begin{equation}
\rho(\xi) = \frac{m}{\alpha} \Phi\left( \frac{\xi}{2} + \frac{c}{\xi} I(\xi > 0) \right)
\end{equation}
where $c$ is a normalization constant s.t. sum of $w_j$ is 1. So the optimal weight depends on $\xi_j$, but also on other $\xi$'s through the constant $c$. The paper shows figures of the function $\rho()$ at different values of $c$. Under some values of $c$, for the tests with large power (large $\xi_j$), it is better to reduce the threshold (higher weights). Under other values of $c$, it is better to give higher weights to intermediate values of $\xi_j$. 

\item Choosing external weights: a special case of assigning weights is binary scheme, where $k$ hypothesis have weights $w_1$ and the rest $w_0$. Let $\epsilon = k/m$ and $B = w_1/w_0$. In practice, we assume $\epsilon$ is given, and need to assign $B$. Show that this can be done with given $\xi$. 

\item Estimating weights from data: in practice, $\xi_j$ is unknown. The idea is to create multiple groups of hypothesis, and assume a mixture distribution of $T_j$ (null and alternative), and within a group, the same alternative distribution. Specifically in group $k$, the $i$-th test: 
\begin{equation}
T_{ik} \sim (1-\pi_k) N(0,1) + \pi_k N(\xi_k, 1)
\end{equation}
And we estimate $\pi_k$ and $\xi_k$. 

\item Remark: 
\begin{itemize}
	\item In multiple testing problem (or testing problem in general), the goal is to \textbf{maximize power while controling for false positives} (types I error or FDR).
	
	\item The power (and for estimation problem, SSE) under frequentiest statistics depends on the true values of parameters, which are unknown. One strategy is to use Empirical Bayes to effectively estimate the prior mean/distribution of parameters. 
\end{itemize}
 
\end{itemize}

Optimal Multiple Testing Under a Gaussian Prior on the Effect Sizes [Dobriban \& Owen, arxiv, 2015]
\begin{itemize}
\item Motivation: in the work by [Roeder \& Wasserman, 2009], the optimal weights are determined for known (or estimated) prior mean effect ($\xi_j$). However, this may be known only approximately, e.g. for testing one GWAS, we may use a different GWAS for weighting SNPs: the second GWAS is related, but effect sizes are probably different.
	
\item Model: let test statistics be $T_i \sim N(\mu_i, 1)$, however $\mu_i$ is unknown. Instead, we have its prior $\mu_i \sim N(\eta_i, \sigma_i^2)$. Our goal is to determine the optimal weights using values of $\eta_i$ and $\sigma_i$. To do that, we first integate over $\mu_i$: $T_i \sim N(\eta_i, \sigma_i^2 + 1)$. Let $\gamma_i^2 = \sigma_i^2 + 1$. Use the same objective function (power over all tests), we solve the optimization problem: 
\begin{equation}
\max_w \Phi\left[ \frac{\Phi^{-1}(q w_i) - \eta_i}{\gamma_i} \right] \quad \text{subject to } \sum_j w_j = m
\end{equation}
where $q = \alpha / m$ is the threshold. One can show that when $\gamma_i > 1$, the function is not concave. 
  
\item Solution to the optimization problem: by maximizing the Lagrangian.  
\end{itemize}

$p$ values for high-dimensional regression [Meinshausen \& Buhlmann, JASA, 2009]: 
\begin{itemize}
\item Motivation: Lasso regression in high-dim., what is the significance of the features selected by Lasso? And how do we control FWER or FDR (for selected features)? 

\item Method: the procedure consists of several steps: (1) random partition the data into two equal groups (many times); (2) train the Lasso in one group, and computer $p$-value of any single feature chosen by Lasso in the second group (e.g. $t$-test); (3) obtain the distribution of the $p$-value of any feature, and then obtain the value ranked at top $\gamma$ percentile in this distribution, as the adjusted $p$-value of this feature. 

\item Questions: 
\begin{itemize}
\item How to choose $\gamma$ to control FWER or FDR? 
\item Alternative approach: e.g. random permutation of data to obtain a null distribution of some Lasso statistic, and then get the $p$-value. What is the advantage of this approach? 
\end{itemize}
\end{itemize}

Computationally Efficient Estimation of False Discovery Rate Using Sequential Permutation $P$-Values [Bancroft and Nettleton, 2013]: 
\begin{itemize}
\item Goal: in a problem of testing multiple hypothesis where most are null (genome-wide testing), suppose we are using a permutation test. For most genes, we don't need to perform many simulations because after a small number, it will be clear that the observed stat has a large $p$-value. Thus one may stop earlier. This may produce a sequential permutation, with $p$-value discrete and not uniformly distributed. 

\item Idea: BH correction under the non-uniform distribution of $p$-values. 
\end{itemize}

The functional false discovery rate with applications to genomics (fFDR) [Chen and Storey, Biostatistics, 2019]
\begin{itemize}
	\item Motivation: quantitative informative variable to capture the prior probability of a hypothesis. Similar to stratified FDR control and IHW method, but is based on EB strategy. 
	
	\item Model: let $p_i$ be the p-values of $i$-th test. Let $Z$ be the value of the informative variable, $Z \sim U(0,1)$, e.g. $Z$ can be the quantile of a variable informative of hypothesis. Let $\pi_0(z)$ be the prior probability $H_0$ is true. Assuming $p_i$ follows $U(0,1)$ under $H_0$ and $f_1(p|z)$ under $H_1$. We define the joint density of $p$ and $z$ (indicator) as:
	\begin{equation}
	f(p,z) = \pi_0(z) + (1-\pi_0(z)) f_1(p|z)
	\end{equation}
	Inference: $\pi_0(z)$ follows GLM or GAM or non-parametric. Joint estimation of $\pi_0(z)$ and $f(p,z)$. 
	
	\item Application: eQTL testing (favoring close SNPs) and DEG (normalized per-gene read depth). 
\end{itemize}

\subsection{Efron's Empirical Bayes Approach}

Reference: [Efron10, Chapter 2-6]

Bayesian FDR: 
\begin{itemize}
\item $p$-values and $z$ values: $z$ values may convey more information than $p$-values in multiple testing problem. Ex. in a problem with more than 10,000 hypothesis, the departure from the null distribution is more evident with $z$ values, and more details are revealed in the histogram (Figure 3.1). 
\begin{itemize}
	\item Converting $p$-values to $z$-values: two-sided $p$ values tend to be favored. 
\end{itemize}

\item Two-group model: suppose we are testing $N$ hypothesis, and $z_i$ is our test statistic for the $i$-th test. The null model has prior density $\pi_0$, and the alternative model $\pi_1$. The distribution of the test statistic under $H_0$ and $H_1$ are $f_0(z)$ and $f_1(z)$ respectively. And we also assume that $\pi_0$ is close to 1. Our goal is to estimate, given a decision region of $z$, how often do we make a false discovery (reject a null model when it is actually true)? 

\item Bayesian FDR and local FDR: suppose $\mathbb{Z}$ is our critical region of $z$. We could define the probability over $\mathbb{Z}$: 
\begin{equation}
F_0(\mathbb{Z}) = \int_{\mathbb{Z}}	f_0(z) dz \qquad F_1(\mathbb{Z}) = \int_{\mathbb{Z}}	f_1(z) dz
\end{equation}
The mixture density and the mixture distribution are: 
\begin{equation}
f(z) = \pi_0 f_0(z) + \pi_1 f_1(z) \qquad F(\mathbb{Z}) = \pi_0 F_0(\mathbb{Z}) + \pi_1 F_1(\mathbb{Z}) 
\end{equation}
The FDR is defined as: 
\begin{equation}
\phi(\mathbb{Z}) = P(H_0|z \in \mathbb{Z}) = \frac{\pi_0 F_0(\mathbb{Z})}{F(\mathbb{Z})}	
\end{equation}
We could also define local FDR at a single point $z_0$: 
\begin{equation}
\phi(z_0) = P(H_0|z = z_0) = \frac{\pi_0 f_0(z_0)}{f(z_0)}	
\end{equation}
The two FDRs, also written as $\text{Fdr}(\mathbb{Z})$ and $\text{fdr}(z)$ respectively, are related by: 
\begin{equation}
\E[\phi(z)|z \in \mathbb{Z}] = \phi(\mathbb{Z})
\end{equation}
The condition expectation of local FDR (over the mixture density $f(z)$) is equal to the FDR. 

\item Why local FDR? With local FDR, the decision of accepting a hypothesis or not is based on the local FDR. For example, at level $\alpha$, we simply accept all hypothesis with local FDR below $\alpha$. 

\item Hierarchical model approach to multiple hypothesis testing: an alternative way of modeling multiple testing, this is based on directly modeling data instead of through $p$-values. For example, suppose we want to test $H_{0i}: \mu_i = 0$ vs. $H_{1i}: \mu_i \neq 0$, with the data $z_i | \mu_i \sim N(\mu_i, \sigma^2)$. We could have a mixture model of $\mu_i$, depending on whether $H_{0i}$ is true: 
\begin{equation}
\mu_i \sim g(\mu) = \pi_0 \Delta_0(\mu) + (1 - \pi_0) g_1 (\mu)	
\end{equation}
where $\Delta_0(\mu)$ is Dirac's delta function centered at 0, and $g_1(\mu)$ is the distribution of $\mu$ under $H_{1i}$. This allows one to estimate prior, once the prior is available, one can estimate the posterior probability that $H_{0i}$ is true. 
\end{itemize}

Empirical FDR estimates: 
\begin{itemize}
\item Empirical estimate: if the null distribution $f_0(z)$ is known, we could form the empirical estimate. First, we estimate the mixture probability (the fraction of positive predictions)
\begin{equation}
\bar{F}(\mathbb{Z})	= \#\left\{z_i \in \mathbb{Z} \right\} / N
\end{equation}
Then we estimate the FDR as: 
\begin{equation}
\bar{\text{Fdr}}(\mathbb{Z}) = \pi_0 F_0(\mathbb{Z}) / 	\bar{F}(\mathbb{Z})
\end{equation}
When $N$ is large, this is a good estimate of FDR. 

\item False discovery proportion (FDP): Bayesian FDR is defined as a probability; in a specific dataset, we are interested in knowning the FDP, defined as: 
\begin{equation}
\text{Fdp}(\mathbb{Z}) = N_0(\mathbb{Z}) / N_+(\mathbb{Z})	
\end{equation}
where $N_0(\mathbb{Z})$ is the total number of null $z_i$ falling into $\mathbb{Z}$, and $N_+(N_0(\mathbb{Z}))$ is the total number of $z_i$ falling into $N_0(\mathbb{Z})$. 

\item Assessing the FDR estimator: under the independence assumption ($z_i$ are independent), one can show that: 
\begin{equation}
\E(\tilde{\text{Fdr}}(\mathbb{Z}))	= \E(\text{Fdp}(\mathbb{Z})) = \phi(\mathbb{Z}) [1 - \exp(-e_+(\mathbb{Z}))]
\end{equation}
where $e_+(\mathbb{Z}) = N F(\mathbb{Z})$ is the expected total number of $z_i$ falling in $\mathbb{Z}$. Therefore, $\bar{\text{Fdr}}(\mathbb{Z})$ is an accurate estimator of $\phi(\mathbb{Z})$ when $e_+(\mathbb{Z})$ is large, say, $e_+(\mathbb{Z}) > 10$. 
\end{itemize}

Estimating FDR with theoretical null: 
\begin{itemize}
\item Estimator of FDR and local FDR: let $\pi_0$ be the proportion of $H_0$, $f_0$ and $f_1$ be the PDF of $z$-scores under $H_0$ and $H_1$ respectively, and similarly, $F_0$ and $F_1$ for the CDF. The estimator of FDR and local FDR are: 
\begin{equation}
\hat{\text{fdr}}(z) = \frac{\hat{\pi_0} \hat{f}_0(z)}{\hat{f}(z)} \qquad \hat{\text{Fdr}}(z) = \frac{\hat{\pi_0} \hat{F}_0(z)}{\hat{F}(z)}
\end{equation}
When the theoretical null is given, then we need to estimate $\pi_0$ and $f$ from the data, $z_1, \cdots, z_n$ (for local FDR). 

\item Estimation of $\pi_0$ [Section 4.5]: the idea is to use a region where $f(z) = 0$ to estimate $\pi_0$, as in this region, all the observed $z_i$'s are due to $H_0$ (called ``zero assumption'). Suppose the region is denoted as $\mathbb{A}_0$, let $N_+(\mathbb{A}_0)$ be the observed number of points in the region, then we have the simple estimate: 
\begin{equation}
\hat{\pi}_0 = \frac{N_+(\mathbb{A}_0)}{N \cdot F_0(\mathbb{A}_0)}	
\end{equation}

\item Estimation of $f(z)$ [Section 5.2]: Poisson regression estimate. The idea is that suppose we divide the range into $K$ bins, then the observed data points in each bin is a Poisson random variable with rate determined by the density function. Define $y_k$ as the count in the $k$-th bin, $y_k = \#\{z_i \in \mathbb{Z}_k\}$, and let $x_k$ be the center point of $\mathbb{Z}_k$. Then we have: 
\begin{equation}
y_k \sim \text{Pois}(N d f(x_k))	
\end{equation}
where $d$ is the bin size. To fit $f(x_k)$ for all $k$, suppose $f(z)$ has the form: 
\begin{equation}
f(z) = \exp\left(\sum_j \beta_j z^j \right)	
\end{equation}
Then we solve the parameters $\beta_j$ using the counts. This is a standard Poisson GLM. 
\end{itemize}

Why the theoretical null may fail? [Efron, Chapter 6]: 
\begin{itemize}
\item Example: a microarray experiment, $N$ genes, expressed in two different conditions (each condition, multiple samples, possibly correlated). Suppose we test the differential expression by a two-sample $t$-test, and plot the distribution of $p_i$ or $z_i$ of all genes. It is possible that the distribution is over-(more often) or under-dispersed. In the case of over-dispersion, we make more false predictions (i.e. we underestimate the FDR). 

\item Reason I: failed mathematical assumption. This would happen, for example, when we assume the data are normally distributed while in fact, there are significant outliers. 

\item Reason II: correlation among subjects/sampling units: in the gene expression example, the correlation between experiment conditions. The correlation between different subjects would also make the test statistic $z_i$ not following theoretical distribution (when there are correlations, $t$-test may not be applicable). 

\item Reason III: correlation among test cases, in the gene expression example, the correlation between test units, i.e. genes. In contrast to Reason I and II, which apply to single cases (genes), this correlation would create departure from theoretical null even if individual $z_i \sim N(0,1)$. To see this, suppose $z_i$ are correlated, and suppose our threshold is $z_i > 2.5$. In some cases, because of correlation, the number of points greater than 2.5 is higher than expected by chance, then assuming theoretical null would lead to a lower estimate of $FDR$. 

\item Reason IV: unobserved covariates. In the gene expression example, this would be other confounders, e.g. batch effect, that creates the difference of expression between conditions. This is the most common source of the failure of theoretical null. 

\item Permutation null distribution: for the gene expression experiment, this would be permuting the columns (experiments). This would address Reason I, but not Reason II (since the permutation would destroy the dependency between columns) and IV. This would preserve correlation among test cases (genes), since permutation would preserve this correlation, but of no direct assistance with Reason III. 
\end{itemize}

Estimating FDR when the theoretical null fails: [Efron, Chapter 6]
\begin{itemize}
\item Estimating empirical null distribution: once $f_0(z)$ is estimated, we could use the same procedure to estimate $\pi_0$ and $f(z)$ as before. So the only problem for estimating FDR is to construct the empirical null distribution. First we note that we need the zero assumption, i.e. in some region $\mathbb{A}_0$, $f_1(z) = 0$, otherwise, the model is not identifable (since $\pi_0$ also unknown). Suppose $f_0(z)$ is normally distributed: 
\begin{equation}
f_0(z) \sim N(\delta_0, \sigma_0^2)	
\end{equation}
Our goal is to estimate $(\pi_0, \delta_0, \sigma_0)$. 

\item Center matching: near $z = 0$, assume $f(z) \approx f_0(z)$, thus $\log f(z)$ is a quadratic function. The task is to approximate the quadratic function using the counts $z_i$ near $z = 0$. The simplest strategy is to perform the least square fit of $\log f(z)$. 

\item MLE: we consider all $z_i \in \mathbb{A}_0$, and assume $f_1(z) = 0$ if $z \in \mathbb{A}_0$. Let $I_0$ be the indices of those $z_i$, and $N_0$ be the size of $I_0$. Let $\mathbf{z}_0$ be the set $\{z_i: i \in I_0\}$, and $\phi_{\delta_0, \sigma_0}(z)$ be the density function of $N(\delta_0, \sigma_0^2)$. The likelihood function consists of two parts: (1) the probability of having $N_0$ points in $\mathbb{A}_0$: this is given by a binomial distribution; and (2) the probability of generating $z_i \in \mathbb{A}_0$: this is given by the normal density, conditioned on the fact that the point generated falls in $\mathbb{A}_0$. We have: 
\begin{equation}
f(\mathbf{z}_0|\delta_0, \sigma_0, \pi_0) = {\binom{N}{N_0}} \theta^{N_0} (1 - \theta)^{N - N_0} \prod_{i \in I_0} \frac{\phi_{\delta_0, \sigma_0}(z)}{H_0(\delta_0, \sigma_0)}
\end{equation}
where 
\begin{equation}
H_0(\delta_0, \sigma_0) = \int_{\mathbb{A}_0} \phi_{\delta_0, \sigma_0}(z)dz
\end{equation}
and $\theta = P(z_i \in \mathbb{A}_0) = \pi_0 H_0(\delta_0, \sigma_0)$. 

\item Problem of the method of estimating FDR using zero-assumption: in general, $f_1(z) = 0$ in the selected region $\mathbb{A}_0$ or the center (for the center matching method) is not realistic. This is particularly true for low-powered studies where one expect a significant overlap between $f_0$ and $f_1$. Even if the empirical null can be estimated, the zero assumption would overestimate $\pi_0$, leading to an overestimate of FDR. 
\end{itemize}

\subsection{Extensions of FDR}

Covariate-modulated local false discovery rate for genome-wide association studies (cmfdr) [Zablocki and Thompson, Bioinfo, 2014]
\begin{itemize}
	\item Background: QQ plots of SNPs in different categories (UTR, coding, etc.) show different distributions of SNPs. 

	\item Model: Let $z_i$ be test statistics, and $x_i$ be covariates of $i$-th hypothesis. The model assumes that both prior probability of $i$ and the test distribution depends on $x_i$. Let $f_0(z_i)$ be distribution of $z_i$ under $H_0$, $N(0, \sigma_0^2)$. Let $f_1(z_i|x_i)$ be the distribution of $z_i$ under $H_1$, with $\text{Gamma}(a(x_i), \beta)$. Model assumptions:
	\begin{equation}
	\text{logit}(\pi_1(x_i)) = x_i^T \gamma \qquad \log a(x_i) = x_i^T \alpha
	\end{equation}
	Prior: $\alpha, \gamma$ are normal with mean 0, and $\sigma_0^2$ and $\beta$ follow Gamma and Inv-Gamma. Inference: Gibbs sampling. 
\end{itemize}

LSMM: a statistical approach to integrating functional annotations with genome-wide association studies [Can Yang, Bioinfo, 2018]
\begin{itemize}
	\item Model: let $p_j$ be p-value of test $j$. Let $\gamma_j$ be indicator, $p_j | \gamma_j = 0 \sim U(0,1)$ and $p_j | \gamma_j = 1 \sim \text{Beta}(\alpha, 1)$. The prior of $\gamma_j$ depends on genic annotation $Z_j$ (e.g. UTR, CDS) and tissue annotations $A_j$: 
	\begin{equation}
	\text{logit}(P(\gamma_j = 1 | Z_j, A_j)) = Z_j b + A_j \beta
	\end{equation}
	where $b$ follows normal prior, and $\beta$ follows spike-and-slab prior. The effect of $Z_j$: fixed effect; the effect of $\beta$ treated as random (marginalize effect size). 
	
	\item Remark: the model treats all SNPs as independent. This would not lead to correct interpretation of $b, \beta$, which would NOT be log-OR of causal variant probability. 
\end{itemize}

Fast and covariate-adaptive method amplifies detection power in large-scale multiple hypothesis testing (AdaFDR) [James Zou, NC, 2019]
\begin{itemize}
	\item Problem formulation: Let $P_i$ be the p-value of the $i$-th test, and $x_i$ be covariates. The goal is to determine FDR threshold for a given $x$, or $t(x)$. Let $D(t)$ be the power at threshold function $t$. The goal is to maximize $D(t)$ s.t. FDP constraint. 
	
	\item Model of $t(x)$: a mixture of GLM and K-component Gaussian mixture (with diagonal covariance matrices). 
\end{itemize}

\subsection{Bayesian Approach}

Bayesian decsion theory approach:  
\begin{itemize}
\item Reference: A Bayesian Measure of the Probability of False Discovery in Genetic Epidemiology Studies [Wakefield, AJHG, 2007], Reporting and interpretation in genome-wide association studies [Wakefield, Int. J Epiderm, 2008]

\item False positive report probability (FPRP): The idea is at the probability of achieving $T > t_{\text{obs}}$ under $H_0$ is $p$ (the $p$-value), and the probability of achieving that under $H_1$ depends on the power. So we combine the two with a mixture model, and estimate the probability of $H_0$. 
\begin{equation}
\text{FPRP} = \frac{p \pi_0}{p \pi_0 + (1 - \pi_0) \times \text{Power}}	
\end{equation}
This is not a strictly speaking, Bayesian approach, but tries to approximate the it. The drawbacks are: (1) information is lost by considering only $T > t_{\text{obs}}$. (2) does not provide control of FDR because a variable threshold for $T$ is used. 

\item Bayesian false discovery probability (BFDP) is defined as posterior prob. of null, $P(H_0|D)$. Note that this depends on the prior probability of null $\pi_0$. 

\item The threshold of PPA (posterior probability of association) or BFDP: determined by the cost of false discoveries vs. false non-discoveries (posterior expected loss). It can be shown that the threshold should be chosen s.t.: 
\begin{equation}
P(H_0|D) < \frac{C_{\beta}/C_{\alpha}}{1+C_{\beta}/C_{\alpha}}	
\end{equation}
where $C_{\alpha}$ and $C_{\beta}$ are costs of  a false discovery a false nondiscovery, respectively. 

\item BFDP and $p$-value: for single SNP trend test, $p$-values are equivalent to BFs with a specific prior (exactly the same ranking). 
\end{itemize}

Direct posterior probability approach: [Murphy, Section 5.7.2.4], [Newton, Biostatistics, 2004, PMID:15054023]
\begin{itemize}
\item Why need multiple testing correction under Bayesian? We predict a hypothesis by its posterior, $P(H_{1i}|D)$, and could impose a cutoff (e.g. $\tau$) on this posterior. However, it does not tell the false discovery rate, as the rate (average over all predictions) is certainly smaller than $1 - \tau$. In other words, if we want FDR at $\alpha$, use $1 - \alpha$ as the posterior cutoff is too stringent.  

\item FDR of Bayesian multiple testing: let $p_i = P(H_{1i}|D)$, then our predictions are those with $p_i > \tau$. The posterior probability can be written in terms of Bayes factors as: 
\begin{equation}
p_i = \frac{(1-\pi_0) \text{BF}_i}{\pi_0 + (1-\pi_0) \text{BF}_i}	
\end{equation}
where $\pi_0$ is the prior proability of $H_0$. The total number of predictions satisfying the threshold $\tau$ is: 
\begin{equation}
N(\tau, D) = \sum_i I(p_i > \tau)	
\end{equation}
The number of false discoveries among these predictions is given by:
\begin{equation}
FD(\tau, D) = \sum_i (1-p_i) I(p_i > \tau)	
\end{equation}
where $1 - p_i$ gives the probability that $H_{0i}$ is true (thus a false discovery). The FDR is given by: 
\begin{equation}
\text{FDR}(\tau) = E(\text{FDR}|D) = \frac{FD(\tau, D)}{N(\tau, D)}
\end{equation} 

\item Procedure of FDR control: suppose we want to control FDR at the rate $\alpha$. Suppose there exists at least one test with $1 - p_i \leq \alpha$, then we can choose the largest $\tau$ s.t. the FDR is less than $\alpha$ [PMID:19822692]. 

\item Remark: Bayesian FDR is defined conditioned on data. The false positives have different semantics under Bayesian interpretation: given the data, what is the chance that the (unknown) hypothesis is false, as opposed to: given $H_0$ is true, how often we have a better test statistic. 
\end{itemize}

Issues of Bayesian multiple testing: for both decision theoretical and the direct posterior probability approach, we need:
\begin{itemize}
\item Determining $\pi_0$: this is specified a priori. For GWAS, it is typically $10^{-4}$ to $10^{-5}$ per SNP. Note that in the $q$-value approach, $\pi_0$ is estimated from data: this is relatively simple in the problem of determining differential expression, but much harder in GWAS where $\pi_0$ is usually much smaller. 

\item Assessing false discoveries by simulation: for both approaches, the results (the discoveries) depend on $\pi_0$, which is usually not known accurately, we may need to assess the performance empirically through simulation. Specifically, we simulate the data under $H_0$ (or permutation), and compute the BFs. Then count the number of tests with BFs above a chosen threshold (expected) vs. the observed number. The enrichment would suggest the FDR:
\begin{equation}
\text{FDR} = \frac{\#\text{Expected}}{\#\text{Observed}}	
\end{equation}
Table 3 of [Wakefield, Int. J Epiderm, 2008]. The same analysis can be done via PPA or BFDP, see [Li \& Maris, A hidden Markov random field model for genome-wide association studies, Biostatistics, 2009]. 

\item Inflation of BFs: for the BFs to properly behave, they should satisfy certain properties. In parciular, under $H_0$, the expectation of BFs should be less than or equal to 1. In general, for genomics problems, most of the hypothesis tested are false, so one could check the overall BF distribution to see if there is inflation. 
\begin{itemize}
	\item Remark: according to Xiaoquan Wen's paper, $\E(BF|H_0) = 1$ - need to check if this is true. Intuitively, as sample sizes goes to infinity, $BF|H_0 \rightarrow 0$. 
\end{itemize}
\end{itemize}

Bayes/non-Bayes compromise: 
\begin{itemize}
\item Reference: Imputation-Based Analysis of Association Studies: Candidate Regions and Quantitative Traits [Servin \& Stephens, PLG, 2007]

\item FDR estimation through permutation: permutation to obtain the null distribution of BFs, and obtain the cutoff through FDR estimation. This is essentially the same approach used to assess false discoveries by simulation with decision theoretical or direct posterior prob. approaches. 

\item $p$-value calculation: one can also obtain the null distribution of BFs, then convert each BF to a $p$-value, and apply the standard approach of FDR control on the $p$-values. 

\item Remark: the permutation approach is less sensitive to prior parameters, so more robust. 
\end{itemize}

Comparison of Posterior probability FDR control and permutation FDR control: 
\begin{itemize}
	\item Problem: suppose we have $m$ tests with BFs, $B_1, \cdots, B_m$, and the corresponding posterior probabilities are $v_1, \cdots, v_m$. Let $\pi_0$ be the prior of $H_0$. The direct posterior probability FDR at the threshold $v_i > t$ is given by: 
	\begin{equation}
	\text{FDR}_d(t)	= \frac{\sum_i I(v_i > t) (1-v_i)}{\sum_i I(v_i > t)}
	\end{equation}
	Suppose we have the null distribution of BFs from permutation, the FDR based on permutation at the threshold $b$ (choose $b$ and $t$ s.t. they are matched) is: 
	\begin{equation}
	\text{FDR}_d(b)	= \frac{m \pi_0 \cdot P(B > b|H_0)}{\sum_i I(B_i > b)}
	\end{equation}
	where $P(B > b|H_0)$ is the probability of BF greater than $b$ under $H_0$. Are these two generally equal? 
	
	\item Analysis: suppose we choose $b$ and $t$ s.t. the denominators are identical (or we look at the FDR at top $K$ predictions, where $K$ is fixed). The numerator of permutaion FDR is independent of the alternative distribution of BF. On the other hand, the numerator of direct posterior FDR depends on BF distribution under $H_1$, or the power of the test - when the power is large, the numerator might be very small (all top $K$ BFs are very large). 
\end{itemize}

Conservative estimation of $\pi_0$ [Xiaoquan Wen, Robust Bayesian FDR Control with Bayes Factors, arxiv, 2013]
\begin{itemize}
\item Motivation: in Bayesian FDR control using posterior probabilities, if $\pi_0$ is underestimated, then the posterior $v_i$ will be overestimated, leading to inflation of FDR control. The goal is to provide an upper bound of $\pi_0$, then if we replace $\pi_0$ in calculating $v_i$: 
\begin{equation}
\hat{v}_i = \frac{(1-\hat{\pi}_0) \text{BF}_i}{\hat{\pi}_0 + (1-\hat{\pi}_0) \text{BF}_i}
\end{equation}
the FDR control will be guaranteed. 

\item EBF procedure: the idea is that if we roughly the distribution of BFs under $H_0$, then we could estimate the proportion from $H_0$ from the overall distribution of BFs. It is easy to prove that: 
\begin{equation}
\E(\text{BF}|H_0)	= \int B f_0(B) dB = \int \frac{P(y|H_1)}{P(y|H_0)} f_0(B) dB = 1
\end{equation}
where $B$ is the BF and $f_0(B)$ is the PDF of $B$ under $H_0$. To see this, at any value of $B$, consider the value of $y$ s.t. the BF under $H_0$ near this $y$ is close to $B$, then at the neighborhood, we have: 
\begin{equation}
f_(B) dB = P(y|H_0) dy	
\end{equation}
Plug in this to the above equation we have the integral of $B$ under $H_0$ is equal to 1. The procedure is: we rank in increasing order of all $m$ BFs, and choose the maximum $d$ s.t. the mean of $\text{BF}_1$ to $\text{BF}_d$ is less than 1 (thus these BFs are likely from $H_0$). Then we choose $\hat{\pi}_0 = d/m$. It can be shown that when $m_0$ is large enough (the number of tests from $H_0$), then $\hat{\pi}_0$ provides a conservative estimate of $\pi_0$, i.e. 
\begin{equation}
P(\hat{\pi}_0 \geq \pi_0|\pi_0) \rightarrow 1	
\end{equation}

\item Proof of EBF: we first show that this is true when $\pi_0 = 1$, essentially, we need to show that the mean of all $m$ BFs would be less than 1. First we show that the top BF, $\text{BF}_m$ cannot be very large. By Markov's Inequality, for any $i$, we have: 
\begin{equation}
P(\text{BF}_i \geq m^2) \leq \frac{1}{m^2}	
\end{equation}
Thus using the extreme value distribution: 
\begin{equation}
P(\text{BF}_i < m^2) = \prod_i P(\text{BF}_i < m^2) \geq \prod_i \left(1-\frac{1}{m^2}\right) = \left(1-\frac{1}{m^2}\right)^m \rightarrow 1 
\end{equation}
Now because for each $i$, $\text{BF}_i$ is bounded, the expectation $\E(\text{BF}_i|H_0) < 1$. Apply WLLN, the mean of all $\text{BF}_i$ would be less than 1. To move to the general case of $\pi_0 < 1$, we apply this special case to all $\text{BF}_i$'s that are from $H_0$. 

\item QBF procedure: similar to EBF, but instead of considering the mean, we consider the $\gamma$-quantile of BF under $H_0$. Specifically, for the $i$-th test, suppose we have the null distribution of $\text{BF}_i$, and from this we obtain its quantile $q_{i,\gamma}$. Then our estimator of $\pi_0$ is: 
\begin{equation}
\hat{\pi}_0 = \frac{\sum_i I(\text{BF}_i \leq q_{i, \gamma})}{m \gamma}	
\end{equation}
Roughly speaking, the denomiator is the expected number of tests with BFs higher than $\gamma$-quantile; and the numerator is the expected number from $H_0$. 

\item Remark: 
\begin{itemize}
\item The idea is similar to the procedure of estimating $\pi_0$ by Storey et al: if we know the null distribution of the test statistic (either $p$-value or BF), we could use the fact to estimate the fraction from $H_0$ (assuming that in some range of statistics, most are from $H_0$). 

\item The proof of Proposition 1 (Appendix C) appears to be flawed. In particular, if the mean of a subset is less than 1, we cannot prove that the mean of the entire set is also less than 1 (imagine that the mean of all BFs from $H_0$ is close to 1, but we have a few from $H_1$ whose BFs are larger than 1, and this would push the mean of all BFs greater than 1). 

\item The analysis is not robust to $H_1$ model specification: when BFs are inflated, the FDR will still be underestimated. 
\end{itemize}
\end{itemize}

EB normal means with correlated noise [Lei Sun, NHS talk, 2018]
\begin{itemize}
	\item Problem of correlated noise: comparison of liver expression in GTEx across different groups (randomly partition samples into groups). In some cases, inflation (many more genes with low FDR) and others deflation. Observation: often change of shoulder comparing with standard normal, but not excess or depletion of tails. Existing framework is not adequate: 
	\begin{itemize}
		\item BH correction: correct on average, but in specific case, may fail (over or under-estimate false discovery proportion, FDP). 
		\item Efron's FDR control: normal mixture could not explain the pattern, e.g. $N(0, 2^2)$ would have a high shoulder, but also a long tail. This would lead to loss of power. 
	\end{itemize}
	
	\item Idea: consider $Z_i \sim N(0,1)$ but they may be correlated, we fit the histogram of $Z_i$'s to capture inflation or deflation. To do this, we model the CDF of $Z_i$ (across all tests), $F_i(Z)$. Correlation will influence this distribution, but will not be directly modeled.  
	
	\item Theory: let $F_i(Z)$ be the CDF of $Z_i$. Assume $Z_i, Z_j$ are bivariate normal with correlation coefficient $\rho_{ij}$. Our goal is to approximate $F_i(Z)$. To do that, we note $\E(F_i(Z)) = \phi(Z)$, the CDF of standard normal. The variance depends on $l$-th moment correlation: 
	\begin{equation}
	\bar{\rho}^{l} = 1/\binom{2}{n} \sum_{i,j} \rho_{ij}^l
	\end{equation} 
	Using these results, we can approximate the histogram (PDF) as: 
	\begin{equation}
	f(z) = \phi(z) + \sum_l w_l \phi^{(l-1)}(z)
	\end{equation}
	where $\phi(z)$ is the PDF of standard normal, $\phi^{(l)}(z)$ is the $l$-th derivative of standard normal, and $w_l$ is given by $\bar{\rho}^l$. 
	
	\item Remark: the correlation is defined as average over all pairs. If there is only local correlation, e.g. LD, it will not create inflation/deflation. 
	
	\item Application to large-scale multiple testing: our test statistic $x_j = \theta_j + z_j s_j$, where $\theta_j$ represents signal, $\theta_j \sim g(\cdot)$, e.g. ASH, and $s_j$ standard error. The term $z_j$ represents correlation of noise and we have $z_j \sim f(\cdot)$, where $f(\cdot)$ is defined as above. To fit the model, ASH parameters $\pi$, and $w$ for correlated noise, we solve constrained optimization problem, where we penalize large $w_l$ (expect exponential decay), and $f(\cdot)$ is constrained to be non-negative at specified values (across a large range). 
	
	\item \textbf{Lesson}: even with correlated noise, under null, hard to obtain tail (large Z score). Under signal, could obtain large tail. So the method is able to disentangle the two. 
	
	\item Lesson: we can study the distribution of test statistics using CDF of test statistic for an individual test. We can study the expectation and variance of the distribution. 
	
	\item Q: correlation of $Z_i$'s is not directly modeled. How much can we gain if we explicitly model them, e.g. use factor model/low rank approximation? 
\end{itemize}

\subsection{Post-hoc Analysis}

The dangers of post-hoc analysis: 
\begin{itemize}
\item General idea: when we test a hypothesis that is not specified a priori, instead, the hypothesis is formulated from the data (the parameters of the hypothesis, the explanatory variables included in the hypothesis, and so on), then it is possible that the statistical test of the hypothesis may not have a valid type I error.  

\item Example: test the difference of frequency in two groups. Suppose we have $x_1 \sim \text{Bin}(n, p_1)$ and $x_0 \sim \text{Bin}(n, p_0)$, and test if $p_1 = p_0$. If $p_0$ is known, the test statistic in cases is: 
\begin{equation}
T = \frac{x_1 - np_0}{\sqrt{n p_0(1-p_0)}}
\end{equation}
which follows $N(0,1)$ as $n \to \infty$. If $p_0$ is unknow, we could use for example, a chi-square test. However, if we use control only to define $\hat{p}_0 = x_0 / n$, and test $p_1 = x_0/n$, our test will be inflated. The new statistic: 
\begin{equation}
T' = \frac{x_1 - x_0}{\sqrt{x_0\left(1-\frac{x_0}{n}\right)}}	
\end{equation}
As $n \to \infty$, the denominator approaches $\sqrt{np_0 (1-p_0)}$ (the same as before), however, the numerator has a higher variance than previously: 
\begin{equation}
\Var{(x_1 - x_0)}	= \Var(x_1) + \Var(x_0) = 2 n p_0 (1-p_0)
\end{equation}
under $H_0$. So using this test, the variance and standard error is higher than 1, thus using the $N(0,1)$ as the null distribution of $T$ will lead to inflated type I error. 

\item Example: feature selection in linear regression. Suppose we do regression of $y$ over $D =100$ variables $x_j$'s, the significance of each $x_j$ is derived from a $F$-statistic. To control FWER at $\alpha = 0.05$, we would demand the significance $P < \alpha / D = 5\e{-4}$. Now suppose we use feature selection as a filter first, choose only features with $P < 0.05$ with a simple correlation test (single feature), then apply the same regression analysis, and demand $P < \alpha / m$, where $m$ is the number of features passing the filtering step. Clearly, this test is inflated, as we somehow reduces the multiple testing threshold (from 100 to $m$, whose mean is about 5) on exactly the same data. 

\item Example: test the significance of a group of $p$-values. Suppose we have a set of $m$ $p$-values, and our goal is to test if the group as a whole departs from the uniform null distribution (pathway association). One test, let $T$ be the minimum $p$-value, and we test $T$ against extreme-value distribution (the minimum); or we could use the Fisher's method of combining the smallest $k$ $p$-value as test statistic with fixed $k$:
\begin{equation}
T = -2 \sum_{i=1}^k \log p_i
\end{equation}
Now, suppose our $k$ is not fixed, but chosen s.t. $T$ is the most significant among all $k$'s then the null distribution is not valid. To see this: suppose we want to control type I error at $\alpha = 0.05$, then under $H_0$, we have probability 0.05 that the minimum $p$-value is 0.05 (actually somwhat different), however, since we always choose $k$ to make $T$ more significant, the $p$-value of our $T$ will be lower than 0.05, creating inflation. 

\item Remark: the general characteristics is that the form of hypothesis is derived from data, thus it appears more siginificant than it actually is. Some specific types may include: 
\begin{itemize}
\item Using parameters estimated from the data: another example is the genetic burden test, where the weights of variants are learned from data (and then fixed). The problem is: because the parameters are fixed (they are actually nuisance parameters), we underestimate the variance of the test statistic (part of it comes from the variance of the parameter). 

\item Filtering: incorrectly reduce the multiple hypothesis testing burden. We are testing $D$ hypothesis, but with filtering, we test only $m < D$ hypothesis, and fail to pay the cost of multiple testing. 
\end{itemize}

\item Reference: [Wiki, Post-hoc analysis], [Wiki, Testing hypotheses suggested by the data]. 
\end{itemize}

Solutions of post-hoc analysis: 
\begin{itemize}
\item Motivation: in some cases, post-hoc analysis would be beneficial, and we want to still use it, but control for type I error. Examples: 
\begin{itemize}
\item Feature selection: in the linear regression example, with a very large number of features (e.g. $p > n$), it may be computationally expensive, or numerically unstable. 
\item Increased power: in the pathway association example, choosing a fixed $k$ may not be the best strategy: for some pathway, it is more powerful to use only the few top gene; for other ones, it is more powerful to combine multiple weak signals. 
\end{itemize}

\item Permutation test: this is the general approach of controling type I error. 

\item Independence of filtering: if the filtering step is independent of the main test, then it is safe to do the filtering. Example, [Gene-Environment Interaction in Genome-Wide Association Studies, Murcray \& Gauderman, Am J Epiderm, 2009]

\end{itemize}
%%%%%%%%%%%%%%%%%%%%%%%%%%%%%%%%%%%%%%%%%%%%%%%%%%%%%%%%%%%%
\section{Resampling Methods}

Permutation tests [Moore, Intro. to Practice of Statistics, Chapter 14]
\begin{itemize}
\item Aim: test if some effect could reasonably occur ``just by chance''.  

\item Method: suppose we have a test statistic that measures the size of the effect of interest. We test its significance by: 
\begin{itemize}
	\item Compute the statistic for the original data.
	\item Choose permutation resamples from the data without replacement in a way that is consistent with the null hypothesis of the test and with
	the study design. Construct the permutation distribution of the statistic from its values in a large number of resamples.
	\item Find the P-value by locating the original statistic on the permutation
	distribution.
\end{itemize}
\end{itemize}

Applications of permutation tests: 
\begin{itemize}
	\item Two-sample problems: $H_0$ states the two populations are identical (e.g. same mean). If $H_0$ is true, any observation will not depend on which group it comes from. So form the permuation sample by randomly reassigning the groups of all data points. 
	
	\item Matched pairs designs: suppose each observation is associated with a label. If there is no effect (difference of the mean), the observation will not depend on its label, thus randomly switch the two labels for each pair. 
	
	\item Relationships between two quantitative variables: data $(x_i, y_i), 1 \leq i \leq n$ and use correlation coefficient as the measure of dependence. If the two variables are independent, then the values of $Y$ will not depend on $X$. So we can form the permutation sample by randomly permuate $y_i$ among all data points.  
	
	\item Correction for multiple hypothesis testing (when multiple hypothesis are dependent): suppose we test $M$ hypothesis and obtain test statistic $T_1, \cdots, T_M$ (e.g. test multiple markers for association in genetic studies), we are interested in testing whether the most significant one $T_m = \max\{ T_1, \cdots, T_M \}$ is truly significant. Since the test statistics are dependent, there is no simple way of correction. We can do permutation test, which will take into account the dependence among $T_i$'s.  
\end{itemize}

Examples of permutation tests: 
\begin{itemize}
	\item Test if the cis-regulatory elements (CREs) are randomly distributed in the genome [Zhang \& Gerstein, GR, 2007]: if it is randomly distributed, then where a CRE occurs will not depend on its genomic coordinates (regions), thus we form the sample by randomly permutating the positions of CREs. 
	
	\item Test if two or more words (or motifs) co-occur more often than by chance (as predicted from the densitiy of individual words) [Sharan \& Karp, Bioinfo, 2003]: permute the positions of these words, and count the co-occurrence of the word cluster under each permutation.
\end{itemize}

Permutation test in the presence of confounding variables [personal notes]:  
\begin{itemize}
\item Example: suppose we have $x$: gender and $y$: income, we want to test if gender has an effect on income (discrimination). But $z$: education level is a confounding variable: it is associated with both (suppose males tend to be more educated). Suppose our test statistic is: the ratio of male and female income, $T = y(x=1) / y(x=0)$, then even if gender is not related to income, we would have $T > 1$ from education effects. 

\item Idea of permutation test with confounding variables: we need to maintain the association of both $x$ and $z$ and $y$ and $z$. We can do this by: fix $x_i, z_i$, pair it with permuted $y_j, z_j$, but $j$ is chosen s.t. $z_j \approx z_i$. 

\item Application in the example: we permute $y_i$ to a new individual $j$ s.t. education levels are the same $z_i = z_j$. Then under null model, $T_0 > 1$ as $y(x=1) > y(x=0)$ because males tend to have higher education and thus higher income. But $T_0$ would be lower than the real $T$ if gender does have an effect independent of education, because permutation breaks the relation between gender and income. 
\end{itemize}

%%%%%%%%%%%%%%%%%%%%%%%%%%%%%%%%%%%%%%%%%%%%%%%%%%%%%%%%%%%%
\section{Meta-Analysis}

Reference: [BHHR - Borenstein et al, Introduction to Meta-analysis], [Meta-analysis in clinical trials, DerSimonian \& Laird, 1986]

Overview of meta-analysis: [BHHR, Chapter 1]
\begin{itemize}
	\item Effect size: the research problem is the estimation of certain effect size: generally the relationship between two variables. This could be treatment effect, correlation, etc, but could also be simply some unknown parameters. 
	
	\item Goal of meta-analysis: combining multiple studies to get a better estimation of effect sizes. This includes: (1) evaluation of heterogeneity of effect size; in particular, if this is heterogenious, what may be additional independent variable; (2) the summary effect. 
	
	\item General procedure: (1) effect size of each individual study, and the variance of the effect size; (2) the summary effect is usually the weighted average of the indivudal effect size, where the weight often depends on the variance. Intuition: low variance means we have an accurate estimation of the effect size, thus should have a higher weight. 
\end{itemize}

Why we need meta-analysis [BHHR, Chapter 2]
\begin{itemize}
	\item Meta-analysis vs. narrative review: meta-analsis addresses two issues: (1) whether the effect size is consistent; (2) if yes, estimate effect size; if not, quantify the variance. Narrative review, in contrast, are based on $p$-values. 
	
	\item Problems with narrative reviews: not taking power into account. Ex. power is 50\%, then even if true effect is consistent, in half of studies, $p$-values will not be significant. This leads to the wrong conclusion that there are ``conflicting'' results. 
\end{itemize}

Effect size indices: different ways of defining effect size, depending on the nature of problem and the experimental design (and whether indivudal studies are directly comparable). [BHHR, Chapter 4-5]
\begin{itemize}
	\item Mean difference between two groups ($D$): suppose the means of two groups are $\mu_1$ and $\mu_2$ respectively, then the effect size is $\Delta = \mu_1 - \mu_2$. The estimator of $\Delta$ is given by: 
	\begin{equation}
	D = \bar{X_1} - \bar{X_2}	
	\end{equation}
	If we assume the variances of the two groups are equal, the variance of $D$ is: 
	\begin{equation}
	V_D = \frac{n_1 + n_2}{n_1 n_2} S_p^2	
	\end{equation}
	where $S_p^2$ is the pooled variance (see the equation of Pooled variance in Math-Physics Notes). If we do not assume equal variance, the variance of $D$ is given by: 
	\begin{equation}
	V_D = \frac{S_1^2}{n_1}	+ \frac{S_2^2}{n_2}	
	\end{equation}
	where $S_1^2$ and $S_2^2$ are the sample variance of the two groups. 
	
	\item Standardized mean difference between two groups ($d$): when individual studies are not directly comparable (difference variances), we need to standarize $D$ across studies. For any given study, suppose the pooled sample variance is $S_p$, then the estimator of this effect size is: 
	\begin{equation}
	d = \frac{\bar{X_1} - \bar{X_2}}{S_p}	
	\end{equation}
	The variance of $d$ is approximated by: 
	\begin{equation}
	V_d = \frac{n_1 + n_2}{n_1 n_2} + \frac{d^2}{2(n_1+n_2)}	
	\end{equation}
	
	\item Odds ratio between two groups for binary data ($OR$): suppose we have a 2 by 2 table, with counts $A$, $B$, $C$ and $D$. The odds ratio is estimated by: 
	\begin{equation}
	OR = \frac{AD}{BC}	
	\end{equation}
	Typically we use log-odds ratio as the index of effect size ($LOR$), and its variance is given by: 
	\begin{equation}
	V_{\log OR} = \frac{1}{A} + \frac{1}{B} + \frac{1}{C} + \frac{1}{D}	
	\end{equation}
\end{itemize}

Fixed-effect model [BHHR, Chapter 11]: 
\begin{itemize}
	\item Model: assume that the true effect size in each study is identical, and the goal is to estimate this true effect size. Suppose $Y_i$ is the effect size of the $i$-th study, $1 \leq i \leq k$, and the true effect size is $\mu$. We have: $Y_i \sim N(\mu, V_i)$, where $V_i$ is the variance of the $i$-th study. 
	
	\item Summary effect: the likelihood function: 
	\begin{equation}
	P(Y|\mu) = \prod_i P(Y_i|\mu) \propto \exp \left[ - \sum_i \frac{(Y_i-\mu)^2}{2V_i}\right]	
	\end{equation}
	The MLE of $\mu$ is: 
	\begin{equation}
	\hat{\mu} = \frac{\sum_i w_i Y_i}{\sum_i w_i}	
	\end{equation}
	where $w_i$ is the weight of the $i$-th study, and given by $w_i = 1/V_i$. Thus the summary effect is the average of the effect sizes of indivudal studies, weighted by the inverse of variance. 
	
	\item Distribution of summary effect: the estimator is the weighted sum of normal variables ($Y_i$), and thus has normal distribution: 
	\begin{equation}
	\hat{\mu} \sim N\left(\mu, \frac{1}{\sum_i w_i}\right)	
	\end{equation}
\end{itemize}

Random-effect model [BHHR, Chapter 12]: 
\begin{itemize}
	\item Model: assume that the true effect size of each study may be different, and is a sample of the true/population effect size, which is to be estimated. Let $Y_i$ be the effect size of the $i$-th study, $Y_i \sim N(\mu_i, V_i)$, and $\mu_i \sim N(\mu, \tau^2)$. 
	
	\item Summary effect: we first note that the distribution of $Y_i$, marganalizing $\mu_i$ is: $Y_i \sim N(\mu, \tau^2 + V_i)$. Then similar to the fixed-effect model, we have the likelihood function: 
	\begin{equation}
	P(Y|\mu, \tau^2) \propto \prod_i \frac{1}{V_i + \tau^2} \cdot \exp \left[ -\frac{1}{2} \sum_i \frac{(Y_i - \mu)^2}{V_i + \tau^2}\right]	
	\end{equation}
	Define weight $w_i^* = 1/(V_i + \tau^2)$, the MLE of $\mu$ is: 
	\begin{equation}
	\hat{\mu} = \frac{\sum_i w_i^* Y_i}{\sum_i w_i^*}	
	\end{equation}
	Since $\tau^2$ is unknown, in practice, we replace this with the estimation $T^2$. From [DerSimonian \& Laird], we use MOM to estimate $\tau^2$. Define a test statistic that captures the heterogeneity of effects, Cochran's $Q$: 
	\begin{equation}
	Q = \sum_i w_i (y_i - \hat{\mu})^2 = \sum_i \left( \frac{y_i - \hat{\mu}}{\sqrt{V_i}}\right)^2
	\label{eq:CochranQ}
	\end{equation}
	So $Q$ is a measure of the total dispersion (standarized). The MOM estimator of $\tau^2$ is given by: 
	\begin{equation}
	\hat{\tau}^2 = \frac{Q - (k-1)}{\sum_i w_i - \left( \sum_i w_i^2 / \sum_i w_i\right)}
	\end{equation}
	when it is less than 0, it should be 0. \\ 
	Proof: The expectation of $Q$ is simple since $y_i$ follows normal distribution (we use $\mu$ instead of $\hat{\mu}$ for simplicity):
	\begin{equation}
	\E(Q) = \sum_i w_i (\tau^2 + V_i)
	\end{equation} 
	Solving the MOM equation: 
	\begin{equation}
	\sum_i w_i (\tau^2 + V_i) = Q
	\end{equation}
	and the result is:
	\begin{equation}
	\tau^2 = \frac{Q - k}{\sum_i w_i}
	\end{equation}
	The difference is due to the difference between $\hat{\mu}$ and $\mu$. 
	
	\item Distribution of the summary effect: similar to fixed-effect model, the weights are replaced with $w_i^*$. 
	
	\item Remark: in random-effect model, MLE of $\tau^2$ does not have a closed form. So use MOM approach, where the test statistic is intuitive: departure of the study effect from the overall mean. 
	
\end{itemize}

Comparison of two models [BHHR, Chapter 13] 
\begin{itemize}
	\item Smoothing: the two models both use weighted average of individual effects, the difference being the weights. Under the random effect model, $w_i^* = 1/(V_i + \tau^2)$, thus comparing with the fixed-effect model, the influences of extreme studies are smoothed: studies with very small variance would have lower weight under the random effect model, similarly, studies with very large variance would have larger weight. 
	
	\item Application/selection of model: in general, fixed-effect model is suited to studies where the design, experimental/intervention procedure, etc. are the same across studies (e.g. repeat of the same experiment), wherease random-effect model is more generally applicable to independent studies. One caveat is: when the number of studies is small, the estimation of $\tau^2$ is poor, and the random-effect model may be limited (Bayesian analysis would be better). 
	
	\item The practice of using fixed-effect first, then switch to random effect if the test of heterogeneity is significant. This should be strongly discouraged: the power of the test is often low. The decision of which model to use should be made before the analysis. 
\end{itemize}

Measuring and testing heterogeneity [BHHR, Chapter 15-16]
\begin{itemize}
	\item Motivation: in addition to estimating the summary effect, researchers often need to answer questions such as, is there difference in true effect size across studies? So we need to test this and quantify the extent of heterogeneity. 
	
	\item Intuition: the variation of data has two parts: true variation of effect size (between-study variation) and sampling error (within-study variation). Our goal is to extract heterogenity from the total variation: the idea is that we can compare the total variation with expected variation when there is no heterogeneity. 
	
	\item Interpretation of $Q$ and testing heterogeneity: from the defintion of $Q$, when there is no heterogeneity (null hypothesis), it follows chi-square distribution with dof equal to $k-1$. So this allows us to consutrct a chi-square test of $Q$. In addition, we have: 
	\begin{equation}
	\E(Q|H_0) = k-1
	\end{equation}
	However, $Q$ itself depends on the number of studies, so is not interpretable/comparable. 
	
	\item Measuring heterogeneity using $T^2$: from the definition of $T^2$, we know that it is a normalized measure of excess dispersion $Q - df$. It depends on the scale of effect size. 
	
	\item Measuring heterogeneity using $I^2$: because $Q-df$ measures the excess dispersion, we can ask what proportion of total dispersion is due to excess dispersion: 
	\begin{equation}
	I^2 = \frac{Q - df}{Q} \times 100\%
	\end{equation}
	It is a descriptive measure, ranging from 0-1 and insensitive to effect size scale and number of studies. Low values suggest that heterogeneity is probably low. 
	
	\item Comparison of $Q$, $T^2$ and $I^2$ in measuring heterogeneity: $Q$ and the $p$ value from $Q$ could test heterogeneity, but does not quantify the extent of heterogeneity. $T^2$ is sensitive to the scale of effect size and measures the variation of true effect size across studies. $I^2$ is entirely driven by $Q$ and df, so it does not reflect variation of true effect size (Figure 16.7 of the BHHR book).  
\end{itemize}

Combining $p$-values: [BHHR, Chapter 36]
\begin{itemize}
	\item Sign test: comparing the number of studies where the effect is one direction vs. the number of studies of the other direction. Very simple, not use all the information (magnitude of effect). 
	
	\item Comparison of using $p$-values and using effect sizes: the effect size meta-analysis is generally preferred because: 
	\begin{itemize}
		\item Effect size is often what is desired in practice, instead of $p$-values. 
		\item $p$ values also depend on sample size, and do not fully reflect the effect size. Ex. two studies may have the same $p$-values, but very different effect sizes (because of different sample sizes). 
	\end{itemize}
	
	\item When to use $p$-value based test: (1) the sample size is not known, thus impossible to back-compute effect sizes; (2) the studies are very diverse, and it's meaningless to ask about a single summary effect, whereas one could ask whether any of the effect size is zero. 
	
	\item Fisher's method of combining $p$-values:  $X^2 = -2 \sum_i \ln p_i$, where $p_i$ is one-sided $p$-value. Under $H_0$ of no effect in each study, $X^2$ follows $\chi^2$ distribution with df equal to $2k$.
	
	\item Stouffer's method of combining $Z$-scores: let $Z_i$ be the $Z$-score computed from $p$-value (one-sided so that $Z$ is standard normal distribution; if two-sided, $Z$ is always positive), define
	\begin{equation}
	Z_{\text{Stouffer}} = \sum_i w_i Z_i 	
	\end{equation}
	where $\sum_i w_i^2 = 1$. Then under $H_0$ of no effect in each study, $Z_{\text{Stouffer}}$ follows standard normal distribution (easy to check). In particular, when we combine results of multiple studies of different sample sizes (but the effect is the same across all studies), we have: 
	\begin{equation}
	Z_{\text{Stouffer}} = \sum_i \sqrt{n_i} Z_i / \sqrt{n} 	
	\end{equation}
	where $n$ is the total sample size. \\
	Proof: we consider the problem of testing $H_0: \mu = 0$ in normal distribution, and suppose there are only two groups. For the $i$-th group, the data points are i.i.d. $N(\mu, \sigma^2)$, with sample size $n_i$. The test statistic is the $Z$-score: 
	\begin{equation}
	Z_i = \sqrt{n_i} \frac{\bar{x}_i - \mu}{\sigma}	
	\end{equation}
	while follows $N(0,1)$ under $H_0$. It is easy to show that: 
	\begin{equation}
	\frac{\sqrt{n_1} Z_1 + \sqrt{n_2} Z_2}{\sqrt{n_1 + n_2}} = 	\frac{n_1 \bar{x}_1 + n_2 \bar{x}_2 - (n_1 + n_2) \mu}{\sigma \sqrt{n_1 + n_2}} = \sqrt{n_1 + n_2}\frac{\bar{x} - \mu}{\sigma}
	\end{equation}
	This is exactly the test statistic we would use if we combine the two datasets. This proves that combining $Z$-scores with the weighting above (meta-analysis) is equivalent to combining the two datasets (mega-analysis) when testing the mean of normal distributions. 
	
\end{itemize}
%%%%%%%%%%%%%%%%%%%%%%%%%%%%%%%%%%%%%%%%%%%%%%%%%%%%%%%%%%%%
%%%%%%%%%%%%%%%%%%%%%%%%%%%%%%%%%%%%%%%%%%%%%%%%%%%%%%%%%%%%
\chapter{Regression Analysis}
\section{Overview of Regression Analysis}

Some examples of linear regression: [Chatterjee]
\begin{itemize}
	\item Effect of union law on the cost of living: state-level data of the cost of living (response variable), whether the law was implemented (predictor), other variables such as income level. 
	\item Prediction of domestic immigrant rates: the total immigration into a state (could be negative). The predictors include state wage, unemployment rates, crime rates, etc. 
	\item The water quality at many rivers as a function of agriculture, forest, commercial/industrial use in the neighborhood. 
\end{itemize}

Perpsectives of linear regression: 
\begin{itemize}
	\item Perspective of variation: regression can be viewed as prediction from independent variables. But can be also viewed as the description of how variation of the response variable results from the variation of predictors, or equivalently, comparison of groups defined by values of independent variables. Thus if $Y$ depends on a feature $X_j$, we interpret as: the groups with different values of $X_j$ will have different $Y$. 
	
	\item Predictive vs counterfactual interpretations: (1) predictive: the difference of the outcome variable between two groups that differ by 1 on average in the relavant predictor (on average: the effect that other variables are held constant); (2) counterfactual: everything else fixed, changing the predictor of one object results in the change of outcome (Note: on individual objects instead of groups). 
	
	\item Linear regression vs. grouping/ANOVA: to study if $X$ influences $Y$, we can build a linear model of $Y$ vs. $X$; or study the groups defined by $X$, and see if their $Y$'s are different (or effects are different - this allows controlling other covariates). The difference is that with the grouping approach: no assumption of linearity (i.e. incorporating interactions).  
	\begin{itemize}
		\item Example: salary survey data, the question is whether education and management affects salary. Define groups of subjects by education $\times$ management, and test if the salaries of the groups are different, while controlling for other variables. 
	\end{itemize} 
\end{itemize}

General assumptions of linear model: 
\begin{itemize}
	\item Linearity: of coefficients. When the relation of $Y$ over $X_j$ is not linear, do transformation on $X_j$'s. This assumption means that the effect of $X_j$ on $Y$ (could be linear or not) does not depend on other explanatory variables. 
	
	\item Error: normality and independence. 
	
	\item No measurement error. 
\end{itemize}

Modeling procedure for regression: 
\begin{itemize}
	\item Formulate a problem: whether the goal is to test one particular variable (e.g. the union law example) or prediction. 
	\item Collecting data: selecting control variables, variation of the main variable to be tested. 
	\item Statistical model: choose a set of predictor variables, choose a form of model. May need to transform the variables, e.g. when $Y$ is a linear function of $X_1^2$, we will need to transform $X_1$ s.t. the model is linear.  
	\item Model fitting.   
	\item Criticism and analysis: check assumptions and model dignostics. Examples: (1) outliers in the data: in the domestic immigration example, Alaska and Hawii are outliers that should be removed. (2) For prediction tasks, whether the variables fall into the range of the training data. Specific steps may include: residual plot, outlier detection, sensitivity analysis. 
\end{itemize}

Setting up linear regression: 
\begin{itemize}
	\item Choose features/predictors: in general, a feature $X$ should be chosen if the groups defined by $X$ have different values of $Y$. 
	
	\item Control variables: in many problems, the goal is to investigate the effect of one treatment variable on the outcome (some type of causal inference). In these cases, it is important to remove the effect of other variables that may influence the outcomes: the control variables.
	
	\item Post-treatment variables: should not be used as control variables when investigating the causal effect of treatments. They should be correlated with treatment variables, thus when doing regression, the effect of the treatment variable may be masked by the post-treatment variables. 
	
	\item Feature expansion: may be functions of the independent variables - basis expansion. 
	
	\item Interaction between features: two features interact if the effect (coefficient/slope) of one feature depends on the value of another feature. Ex. the child IQ as an outcome variable of the mother's number of high school years ($mom.hs$), and the mother's IQ ($mom.IQ$). The effect (slope) of $mom.IQ$ depends on $mom.hs$: when $mom.hs$ is small (no high school), $mom.IQ$ plays a large effect; however when $mom.hs$ is large, the effect of $mom.IQ$ is considerably smaller (education makes up for the deficiency of mother IQ). 
	
\end{itemize}

General strategy for fitting a linear model: the idea is to find a model whose predictions agree with the observations. This can be done in two ways, broadly speaking
\begin{itemize}
	\item Summary statistics: most commonly the moment of the data, the histogram, and so on. The summary statistics under the model (expected values) should match the observed values. 
	\item Conditional distribution/expectation: intutively, we should have $y_i \approx \E(Y_i|x_i)$. 
\end{itemize}

Model diagnosis: any statistical model is based on some assumptions of the data distributions, which may not be true. So it is important to inspect these assumptions. For example, for linear model, we will need to check the linearity assumption and the normality of error assumption. The insights gained can be used to improve the model.  
\begin{itemize}
	\item Plotting: either by plot $y$ against $x$, or often, plot residual $e_i$ against $x_i$, one can explore the linearity of the relationship, and whether the error is constant at different $x$. More generally, plot can reveal the unexpected relationship among variables. If some non-linear relationship is found, one may transform the variables to make it more linear.  
	
	\item Measuring quality-of-fit: we can quantify how good a model fits the data. Two very general ideas are (1) The agreement of observations and predictions. (2) How much variation in the data is explained by our model. 
\end{itemize}

Regression model with known properties of variables: 
\begin{itemize}
	\item Motivation: in high-dim. regression problem, we may have information of the properties of variables. It may be desirable to include such properties to improve estimation of the effects. 
	
	\item Strategy 1: variable selection prior. The idea is that each variable has an indicator variable $Z_j$, whose prior depends on the properties through a regression model (or a similar model). 
	
	\item Strategy 2: variable grouping. The idea is that we can divide the variables by their properties (within a group, variables would have similar effect sizes - a common prior distribution), and then estimate these group-level effect sizes. 
	
	\item Example: to test if some annotation makes a SNP more likely to be causal SNP for a trait.
	\begin{itemize}
		\item Variable selection prior: define annotations of SNPs as features, do regression model on the SNP prior. 
		
		\item Variable grouping: define groups of SNPs, e.g. promoters, tissue-specific enhancers. Within a group, all SNPs have the same prior of effect sizes. Estimation of the effect sizes of each group: this would allow one to control other groups when estimating the effect of one group (important when LD is present). 
	\end{itemize} 
	
	\item Comparison of the strategies: similar to the situation where we can study the effect of $X$ on $Y$ through either regression or ANOVA (compare means of groups defiend by $X$). 
	\begin{itemize}
		\item Non-linear effects: variable grouping can more easily accommodate non-linear effects  (e.g. groups are defined by the product of two features, or clusters of features). 
		
		\item Overlapping groups: when the variable groups overlap, it is easier to model with variable selection prior. With the grouping approach, one needs extra assumption about the effect sizes of variables belonging to multiple groups. 
	\end{itemize}
\end{itemize}


%%%%%%%%%%%%%%%%%%%%%%%%%%%%%%%%%%%%%%%%%%%%%%%%%%%%%%%%%%%%
\section{Analysis of Variance (ANOVA)}
\begin{enumerate}
	
	\item{Introduction to ANOVA} [KNNL, Applied Linear Statistical Models, 5ed, Chapter 15]
	
	Experimental and observation studies: 
	\begin{itemize}
		\item Problem: the effect of some ``treatment'' of interest on some experimental units. The difficulty is that there are often confounding/nuisance factors that cause variation of the response variables (but the interest is only in the treatment effect). All those variables that may influence the response variables are called factors, including treatment and confounding factors. 
		
		\item Example: compare yield of varieties 1 and 2. Need to compare in multiple blocks. However, each block itself has effect on the yield. Need to compare the two varieties in an unbiased fahsion. 
		
		\item Experimental studies: randomization is employed to assign a set of treatments to the experimental units. The causal relationship can be established, as the differences between the treatment and control groups are averaged out, and the only difference is thus due to treatment. 
		
		\item Observational studies: random samples are obtained from multiple populations defined by the levels of one or more explanatory factors, referred to as observational factors. Usually, need external evidence to estabilish causal relationship. 
	\end{itemize}
	
	Experimental designs: 
	\begin{itemize}
		\item Complete randomized design: the treatments are randomly assigned. The linear statistical model for the response is: 
		\begin{equation}
		Y = [\text{Constant}]	+ [\text{Treatment Effect}] + [\text{Error}]
		\end{equation}
		
		\item Factorial design: multiple factors, and the treatment combinations are randomly assigned. Ex. two factors, one has two levels and the other three levels, lead to the $2 \times 3$ factorial design. The linear model: 
		\begin{equation}
		Y = [\text{Constant}]	+ [\text{First-order Treatment Effect}] + [\text{Interaction Effect}] + [\text{Error}]
		\end{equation}
		
		\item Randomized complete block design: the experimental units can be grouped into blocks, according some factors, and within the blocks, randomization of treatments is applied. The linear model: 
		\begin{equation}
		Y = [\text{Constant}]	+ [\text{Treatment Effect}] + [\text{Block Effect}] + [\text{Error}]
		\end{equation}
		
	\end{itemize}
	
	Observational studies: 
	\begin{itemize}
		\item Cross-sectional studies: Measurements of one or more subpopulations at a single time point or time interval. It provides a ``snapshot'' of the factors and the outcome variable. 
		
		\item Prospective studies: one or more groups are formed according to the levels of a hypothesized causal factor, and these groups are observed over time wrt. an outcome variable of interest. 
		
		\item Retrospective studies: groups are defined on the basis of an observed outcome, and the differences among the groups at an earlier time point are identified as potential causal effects. 
		
		\item Matching: similar to blocking, treatment is assigned to a pair of matched units, which are identical in all aspects except treatment. 
	\end{itemize}
	
	Overview of ANOVA strategy: 
	\begin{itemize}
		\item Strategy: the basic goal is to explain how the observed variables vary with treatments, and other groups. 
		\begin{itemize}
			\item Model of factor effects (means of groups): suppose there is only one treatment, then we could have a model like: 
			\begin{equation}
			Y_{ij} = \mu + \tau_i + \epsilon_{ij}	
			\end{equation}
			where $Y_{ij}$ is the $j$-th observation of the $i$-th group, $\tau_i$ is the effect of the $i$-th group. If we view the treatment also as random, then we could have the relation between random variables: 
			\begin{equation}
			Y = \bar{Y} + X + \epsilon	
			\end{equation}
			where $X$ is the variation due to the (random) treatment. 
			
			\item Variance partition: a consequence of the factor effect model is that the total variance of the data can be partitioned according to groups. For example, from the equation above, we see that: 
			\begin{equation}
			\Var Y = \Var X + \sigma^2	
			\end{equation}
			We replace the variance with the sample variance, and we have the variance partition relation. 
			
			\item Inference of factor effects: if there is no treatment effect, we have the sample variance across the treatment group ($\Var X$) close to 0. This suggest that we could assess the variance (in comparison with $\sigma^2$) to test if the treatment effect is 0. 
		\end{itemize}
		
		\item The idea of ANOVA (variance partition) can be applied in many cases: 
		\begin{itemize}
			\item Quantitative genetics: the phenotype $P = G + E$, where $G$ and $E$ represent the independent influence of genotype and environment, respectively. Thus we have the partition: $V_P = V_G + V_E$. The variance partition could also be understood through regression of phenotype on genotype, and the $SST$ of phenotype is the sum of $SSR$ (genotype) and $SSE$ (environment). 
			\item PCA: the variance is partitioned into orthogonal principal components
			\item Fisher's LDA: the variance is partitioned into within-class and between-class variances. 
		\end{itemize}
		Note that when applying ANOVA, it may be that the group/factor is treated as a RV, instead of constant. 
		
		\item Limitation of variance partition: (1) if some factor is important, but does not vary in an observational study, then its importance cannot be quantified in this way. Ex. essential genes will not vary in the population, thus the real importance not detectable. (2) orthogonality of sources is important, need correction if this does not hold. 
		
	\end{itemize}
	
	\item{One way ANOVA} [KNNL, Applied Linear Statistical Models, 5ed, Chapter 16]
	
	One way ANOVA: 
	\begin{itemize}
		\item Cell means model: $r$ groups/levels of the treatment factor, the mean of $i$-th group is $\mu_i$ (where the sample size is $n_i$), and the response variable can be expressed as: 
		\begin{equation}
		Y_{ij} = \mu_i + \epsilon_{ij}	\qquad 1 \leq j \leq n_i
		\end{equation}
		where $Y_{ij}$ is the outcome variable of the $j$-th sample in the $i$-th group. 
		
		\item Assumptions: $E(\epsilon_{ij}) = 0$, independent and normally distributed. Also in one way ANOVA, assume equal variance in groups, i.e. $\sigma_i^2 = \sigma^2$. The classical ANOVA hypothesis is: 
		\begin{equation}
		H_0: \mu_1 = \mu_2 = \cdots = \mu_k	
		\end{equation}
		
		\item Factor effects model: the ANOVA model can be written equivalently as: 
		\begin{equation}
		Y_{ij} = \mu + \tau_i + \epsilon_{ij}	
		\end{equation}
		where $\tau_i$ the effect of the $i$-th factor level, and $\epsilon_{ij}$ are independent $N(0,\sigma^2)$. To ensure identifiability of the model, we require that: $\sum_i \tau_i = 0$. 
		
		\item Relation to linear regression: ANOVA model is equivalent to linear regression model, where the factor levels (groups) are treated as indicator variables. The main diffrence between ANOVA and regression is: when the predictor variables are quantitative, ANOVA does not make any assumption about the nature of the statistical relation. 
	\end{itemize}
	
	Analysis of variance: 
	\begin{itemize}
		\item Partitioning of total sum of squares: we first note that the total variation (sum of square) $SSTO$ can be partitioned as the between group variation or treatment variation ($SSTR$) and the within group variation, or the error sum of squares ($SSE$): 
		\begin{equation}
		\sum_i \sum_j (Y_{ij} - \bar{Y})^2 = \sum_i n_i (\bar{Y}_{i.} - \bar{Y})^2 + \sum_i \sum_j (Y_{ij} - \bar{Y}_{i.})^2
		\end{equation}
		Or written equivalently: 
		\begin{equation}
		SSTO = SSTR + SSE	
		\end{equation}
		The term $SSTR$ measures the extent of differences between the estimated factor level means, and $SSE$ measures the random variation of the observations around the estimated factor level means. The degree of freedom: $SSTO$ - $n_T - 1$; $SSTR$ - $r - 1$; $SSE$ - $n_T - r$. 
		
		\item Mean sum of squares: Then the mean sum of square defined as: 
		\begin{equation}
		\begin{array}{ll}
		MSTR & = SSTR / (r-1)	\\
		MSE & = SSE / (n_T-r)
		\end{array}
		\end{equation}
		The expected values of $MSTR$ and $MSE$ are: 
		\begin{equation}
		\begin{array}{ll}
		E[MSE] & = \sigma^2\\
		E[MSTR] & = \sigma^2 + \frac{\sum_i n_i (\mu_i - \mu)^2}{r-1}
		\end{array}
		\end{equation}
		where $\mu = \sum_i n_i \mu_i / n_T$. 
		
		\item $F$ test: Note that when $\mu_i$'s are equal, we have $E[MSTR] = E[MSE]$, and if not, $E[MSTR]$ would be larger. The ratio of the two can thus indicate how different $\mu_i$'s are. The $F$ test statistic is simply:
		\begin{equation}
		F = \frac{MSTR}{MSE}
		\end{equation}
		the average between-group variation (difference of factor level means), normalized by the within group variation. 
		
	\end{itemize}
	
	Alternative approach to ANOVA test: [Casella, Chapter 11] 
	\begin{itemize}
		\item Goal: for some constants $a = (a_1, \cdots, a_k)$, test the hypothesis: 
		\begin{equation}
		H_0: \sum_{i=1}^k a_i \theta_i = 0  \qquad \text{vs.} \qquad H_1: \sum_{i=1}^k a_i \theta_i \neq 0
		\end{equation}
		
		\item Pooled estimator of within group variance: for a group $i$, the estimator of $\sigma^2$ is the sample variance: 
		\begin{equation}
		S_i^2 = \frac{1}{n_i - 1} \sum_{j=1}^{n_i} (Y_{ij} - \bar{Y}_{i.})^2
		\end{equation}
		where $\bar{Y}_{i.}$ is the mean of group $i$. Since $\sigma^2$ is shared, the pooled estimator is: 
		\begin{equation}
		S_i^2 = \frac{1}{N - k} \sum_{i=1}^k (n_i -1) S_i^2 = \frac{1}{N - k} \sum_{i=1}^k \sum_{j=1}^{n_i} (Y_{ij} - \bar{Y}_{i.})^2	
		\end{equation}
		
		\item $t$-test: $\sum_i a_i \bar{Y}_{i.}$ follows normal distribution with mean $\sum_i a_i \theta_i$, thus use $t$ test. We reject $H_0$ if: 
		\begin{equation}
		\left \lvert \frac{\sum_i a_i \bar{Y}_{i.}}{S_p \sqrt{\sum_i a_i^2/n_i}} \right \rvert > t_{N-k,\alpha/2}	
		\end{equation}
		The coefficient in the denominator is needed to account for $a_i$. 
		
		\item Contrast: a special case of the above test is for $a$ s.t. $\sum_i a_i = 0$. Ex. when $a = (1,-1, 0, \cdots, 0)$, this is the pairwise test: $\theta_1 = \theta_2$: reject $H_0$ if
		\begin{equation}
		\left \lvert \frac{\bar{Y}_1 - \bar{Y}_2}{S_p \sqrt{1/n_1 + 1/n_2}} \right \rvert > t_{N-k, \alpha/2}	
		\end{equation}
		
		\item $F$-test: for any given $a$, let $T_a$ be the test statistic ($t$-test above), i.e. we reject $H_{0a}$ if $T_a >k$ for some $k$. Then the classical ANOVA hypothesis is equivalent to for any contrast $a$, $\sum_i a_i \theta_i = 0$. Therefore if for some contrast $a$, the hypothesis $H_{0a}$ is reject, $H_0$ will be rejected. We use the maximum of $T_a^2$ as the test statistic: 
		\begin{equation}
		F = \frac{\frac{1}{k-1}\sum_i n_i (\bar{Y}_{i.} - \bar{Y})^2}{S_p^2} \sim F_{k-1,N-k}	
		\end{equation}
		where $\bar{Y}$ is the population mean. To see $F$ follows $F$ distribution, note that both the numerator and denominator follow $\chi^2$ distribution (independent). 
	\end{itemize} 
	
	\item{Two-way ANOVA} [KNNL, Applied Linear Statistical Models, 5ed, Chapter 19]
	
	Two-way ANOVA model: 
	\begin{itemize}
		\item Factor effects model: suppose we have two factors $A$ and $B$, where $A$ has $a$ levels (indexed by $i$) and $B$ has $b$ levels (indexed by $j$), then we can write the outcome of the $k$-th sample in the $ij$ cell as: 
		\begin{equation}
		Y_{ijk}	= \mu + \alpha_i + \beta_j + (\alpha \beta)_{ij} + \epsilon_{ijk}
		\end{equation}
		where $\alpha_i$ and $\beta_j$ are constants subject to the constraints: $\sum_i \alpha_i = 0$, $\sum_j \beta_j = 0$ and $(\alpha \beta)_{ij}$ are constants subject to the constraints: 
		\begin{equation}
		\begin{array}{ll}
		\sum_i (\alpha \beta)_{ij} = 0 & j = 1, 2, \cdots, b	\\
		\sum_j (\alpha \beta)_{ij} = 0 & i = 1, 2, \cdots, a	\\
		\end{array}
		\end{equation}
		$\epsilon_{ijk}$ are independent $N(0, \sigma^2)$. 
		
		\item Interpretation of parameters: $\alpha_i$ and $\beta_j$ are main effects. Ex. $\alpha_1 = \mu_{1 \cdot} - \mu$, where $\mu_{1 \cdot}$ is the mean of all samples with $A$ factor at level 1. The interaction effect $(\alpha \beta)_{ij}$ is defined as: 
		\begin{equation}
		(\alpha \beta)_{ij} = \mu_{ij} - (\mu + \alpha_i + \beta_j)	
		\end{equation}
		Thus it is the difference between $\mu_{ij}$ and the value that would be expected if the factors are additive (defined as $\mu_{ij} = \mu + \alpha_i + \beta_j$). The interaction effect is not equal to 0 if the effect of one factor depends on the level of another factor. 
	\end{itemize}
	
	ANOVA table: consider the case where each cell $ij$ has the same sample size $n$: 
	\begin{itemize}
		\item Partitioning of variances: we have: 
		\begin{equation}
		SST = SSA + SSB + SSAB + SSE	
		\end{equation}
		where $SSA$ and $SSB$ are computed from the variation of the $A$ and $B$ factor, respectively, and $SSAB$ from the combination of two factors: 
		\begin{equation}
		\begin{array}{ll}
		SSA & = nb \sum_i (\bar{Y}_{i \cdot} - \bar{Y})^2\\
		SSB & = na \sum_j (\bar{Y}_{\cdot j} - \bar{Y})^2\\
		SSAB & = n \sum_{i,j} (\bar{Y}_{ij} - \bar{Y}_{i \cdot} - \bar{Y}_{\cdot j} + \bar{Y})^2
		\end{array}	
		\end{equation}
		
		\item Mean squares: these are defined as: 
		\begin{equation}
		\begin{array}{ll}
		MSA = \frac{SSA}{a-1}\\
		MSB = \frac{SSB}{b-1}\\
		MSAB = \frac{SSAB}{(a-1)(b-1)}
		\end{array}	
		\end{equation}
		
		\item $F$ test: to test if all interaction effects are equal to 0: 
		\begin{equation}
		F = \frac{MSAB}{MSE}	
		\end{equation}
		To test for factor $A$ and $B$ main effects (all 0): $F = \frac{MSA}{MSE}$ and $F = \frac{MSB}{MSE}$. 
		
		\item Remark: the same test can be derived from regression approach (see the section ``Analysis of variance approach to regression''). In the regression approach, for any factor $A$, we could write $SSA = SSR(A)$ (variation of group means defined by $A$). 
		
	\end{itemize}
	
	\item{Random and mixed effects models} [KNNL, Applied Linear Statistical Models, 5ed, Chapter 25]
	
	Random effects model: 
	\begin{itemize}
		\item Two types of factors: some factors are properties of objects and are of intrinsic interest (i.e. to know their effects would be interesting); some other factors, however, are only random samples and of no intrinsic interest. Ex. to test the effect of a manufacturing procedure, multiple plants are selected and outcomes measured (with each plant: the outcome of different procedures for different samples). In this case, the selected plants are random, but they need to be treated as an experimental factor as their individual variation is important. 
		
		\item Random effects model: the second type of factors have effects that can be viewed as random samples of a larger population. Also called ANOVA model II. The questions of interest are generally about the larger population, not individual samples. 
		
		\item Mixed effects model: if there are multiple factors, some may have fixed effects and other random effects. 
	\end{itemize}
	
	Random effects model of 1 factor: random cell means model: 
	\begin{itemize}
		\item Model: the $j$-th sample of the $i$-th group has outcome: 
		\begin{align}
		Y_{ij} = \mu_i + \epsilon_{ij}
		\end{align}
		where $\mu_i$ are independent RVs of $N(\mu, \sigma_{\mu}^2)$ and $\epsilon_{ij}$ are independent $N(0, \sigma^2)$. 
		
		\item Questions: we are typically interested in: (1) whether there is significant variation across groups, i.e. $\sigma_{\mu}^2 = 0$; (2) estimate the average population mean $\mu$ and the variances $\sigma_{\mu}^2$ and $\sigma^2$. 
		
		\item Important features of the model: the expected value of $Y_{ij}$:
		\begin{equation}
		E(Y_{ij}) = \mu	
		\end{equation}
		The variance of $Y_{ij}$:  
		\begin{equation}
		\text{Var}(Y_{ij}) = \sigma_Y^2 = \sigma_{\mu}^2 + \sigma^2	
		\end{equation}
		The covariance: 
		\begin{equation}
		\text{Cov}(Y_{ij}, Y_{ij'}) = \sigma_{\mu}^2	
		\end{equation}
	\end{itemize}
	
	ANOVA test and estimation of random 1-factor model: 
	\begin{itemize}
		\item Mean sum of squares: the main test statistics are still $MSTR$ and $MSE$. Assuming the treatment sample size is $n$ for all cells: 
		\begin{equation}
		\begin{array}{l}
		E[MSE] = \sigma^2\\
		E[MSTR] = \sigma^2 + n \sigma_{\mu}^2
		\end{array}
		\end{equation}
		
		\item Test whether $\sigma_{\mu}^2 = 0$: this could be formed by the $F$ test: 
		\begin{equation}
		F = \frac{MSTR}{MSE}
		\end{equation}
		
		\item Estimation of $\mu$: an unbiased estimator of $\mu$ is: $\hat{\mu} = \bar{Y}$. The variance of this estimator is: 
		\begin{equation}
		\text{Var}(\bar{Y}) = \frac{\sigma_{\mu}^2}{r} + \frac{\sigma^2}{rn}
		\end{equation}
		And the unbiased estimator of the variance of $\bar{Y}$ is: 
		\begin{equation}
		s^2(\bar{Y}) = MSTR / (rn)	
		\end{equation}
		Thus $(\bar{Y} - \mu) / s(\bar{Y})$ follows $t$ distribution with df $r - 1$. 
		
		\item Estimation of $\sigma^2$ and $\sigma_{\mu}^2$: for $\sigma^2$, we have:
		\begin{equation}
		\frac{r(n-1) MSE}{\sigma^2} \sim \chi^2_{r(n-1)}	
		\end{equation}
		For $\sigma_{\mu}^2$, our unbiased estimator is:
		\begin{equation}
		s_{\mu}^2 = \frac{MSTR - MSE}{n}	
		\end{equation}
	\end{itemize}
	
\end{enumerate}
%%%%%%%%%%%%%%%%%%%%%%%%%%%%%%%%%%%%%%%%%%%%%%%%%%%%%%%%%%%%
\section{Linear Regression}

\subsection{Simple Linear Regression}

Reference: [KNNL, Applied Linear Statistical Models, 5ed, Chapter 1-2], [Chatterjee \& Hadi, Regression analysis by example, 4ed, Chapter 2]

Linear model with one predictor variable: 
\begin{itemize}
\item Model: for $1 \leq i \leq n$:
\begin{equation}
Y_i = \beta_0 + \beta_1 X_i + \epsilon_i	
\end{equation}
where $X_i$ is a known constant, $Y_i$ are independent RVs, and $\epsilon_i$ is a RV with mean 0 and variance $\text{Var}(\epsilon_i) = \sigma^2$, where $\epsilon_i$ and $\epsilon_j$ are uncorrelated for any $i,j$. It is often assumed that $\epsilon_i \sim N(0,\sigma^2)$. 

\item Alternative models: a dummy variable $X_0 = 1$: 
\begin{equation}
Y_i = \beta_0 \cdot 1 + \beta_1 X_i + \epsilon_i	
\end{equation}
Another model is: 
\begin{equation}
Y_i = \beta_0^* + \beta_1 (X_i - \bar{X}) + \epsilon_i 	
\end{equation}
where $\beta_0^* = \beta_0 + \beta_1 \bar{X}$. 

\item Remark: in a linear model, we assume $X_i$'s are constants. In some other situations, it may be easier to view $X_i$'s also as RVs, e.g. in  quantitative genetics: total variance is the sum of variance in predictor variable (genotype) and variance in environment (error term). 
\end{itemize}

Least square parameter estimation: 
\begin{itemize}
\item Nomral equations: minimizing the sum of squared error: 
\begin{equation}
Q = \sum_{i=1}^n (Y_i - \beta_0 - \beta_1 X_i)^2	
\end{equation}
leads to the normal equations that the estimated parameter ($b_0$ and $b_1$) should satisfy: 
\begin{equation}
\begin{array}{lll}
\sum_i Y_i & = & n b_0 + b_1 \sum_i X_i	\\
\sum_i X_i Y_i & = & b_0 \sum_i X_i + b_1 \sum_i X_i^2 
\end{array}
\end{equation}

\item LS estimators: solving the normal equations: 
\begin{equation}
b_1 = \frac{\sum (X_i-\bar{X}) (Y_i - \bar{Y}) }{\sum (X_i - \bar{X})^2}	
\end{equation}
\begin{equation}
b_0 = \bar{Y} - b_1 \bar{X}	
\end{equation}
Intuition of the estimator: the numerator $\sum (X_i-\bar{X}) (Y_i - \bar{Y})$ is the sample covariance, when $X_i - \bar{X}$ and $Y_i - \bar{Y}$ have the same signs, the term is positive. Thus, the sign of the sample covariance captures the relationship: when one is larger, whether the other is large too. 

\item Geometry of least square estimation: suppose $X$ and $y$ are centered (vectors) in $n$-dim. space, the objective function is to minimize the distance from $y$ to any point in the direction of $X$. This is solved by the projection of $y$ on $X$: 
\begin{equation}
\hat{\beta_1} = \frac{X^T y}{\norm{X}^2}	
\end{equation}
And $\hat{\beta_0} = 0$. When $X$ and $y$ are not centered, simply replace $X$ by $X - \bar{X}$ and $y$ by $y - \bar{y}$. 

\item Relation to MOM estimation: given $Y = \beta_0 + \beta_1 X + \epsilon$, we take the expectation: 
\begin{equation}
\E(Y) = \beta_0 + \beta_1 \E(X)	
\end{equation}
And consider the covariance between $X$ and $Y$: 
\begin{equation}
\Cov(X,Y) = \beta_1 \Var(X)	
\end{equation}
Solving the two equations gives the MOM estimator of $\beta_0$ and $\beta_1$, which are the same as LS estimators. 

\item Properties of estimators of coefficients: both $b_0$ and $b_1$ are linear estimators, i.e. they are linear combinations of $Y_i$'s: 
\begin{equation}
b_1 = \sum_i k_i Y_i \qquad \text{where} \qquad k_i = \frac{X_i - \bar{X}}{\sum (X_i - \bar{X})^2}	
\end{equation}
And $b_0$ and $b_1$ are unbiased estimators of $\beta_0$ and $\beta_1$, respectively: 
\begin{equation}
E(b_1) = \beta_1	\qquad E(b_0) = \beta_0
\end{equation}

\item Properties of fitted lines: we define residuals $e_i = Y_i - \hat{Y_i}$, where $\hat{Y_i}$ is the point estimator. We have the properties of residuals: (these can be proved easily)
\begin{equation}
\sum_{i=1}^n e_i = 0	
\end{equation}
\begin{equation}
\sum_{i=1}^n X_i e_i = 0	
\end{equation}
Furthermore: (1) $\sum e_i^2$ is minimized by $b_0$ and $b_1$ (from LS estimation); (2) the regression line passes through $(\bar{X}, \bar{Y})$. 

\item Estimator of $\sigma^2$: define sum of square error as: 
\begin{equation}
SSE = \sum_{i=1}^n (Y_i - \hat{Y_i})^2 = \sum_{i=1}^n e_i^2
\end{equation}
The SSE has two dof. (two parameters), the appropriate mean square is:
\begin{equation}
MSE = \frac{SSE}{n-2} = \frac{\sum e_i^2}{n-2}
\end{equation} 
It can be shown that: 
\begin{equation}
E(MSE) = \sigma^2	
\end{equation}
The intuition is that: $Y_i - \hat{Y_i}$ is normally distributed with variance $\sigma^2$, so using MOM estimation of normal distribution, the MSE is an estimator of $\sigma^2$. 
\end{itemize}

Sampling distributions of $b_0$, $b_1$: 
\begin{itemize}
\item $b_1$ as a linear function of $Y_i$: we could write $b_1 = \sum k_i Y_i$, where 
\begin{equation}
k_i = \frac{X_i - \bar{X}}{\sum (X_i - \bar{X})^2}	
\end{equation}
The coefficients $k_i$ have the properties: 
\begin{equation}
\begin{array}{lll}
\sum k_i & = & 0\\	
\sum k_i X_i & = & 1\\	
\sum k_i^2 & = & \frac{1}{\sum (X_i - \bar{X})^2}	
\end{array}
\end{equation}

\item Sampling distribution of $b_1$: $b_1$ follows a normal distribution with:
\begin{equation}
E(b_1) = \beta_1 \qquad \text{Var}(b_1) = \frac{\sigma^2}{\sum_i (X_i - \bar{X})^2}	
\end{equation}
Proof: since $b_1$ is a linear combination of $Y_i$'s and $Y_i$ are independent normal RVs, $b_1$ must follow normal distribution. Its mean: 
\begin{equation}
E(b_1) = E(\sum_i k_i Y_i) = \sum_i k_i (\beta_0 + \beta_1 X_i) = \beta_0 \sum_i k_i + \beta_1 \sum_i X_i k_i = \beta_1
\end{equation}
And its variance: 
\begin{equation}
\text{Var}(b_1) = \text{Var}(\sum_i k_i Y_i) = \sum_i k_i^2 \text{Var}(Y_i) = \sum_i k_i^2 \sigma^2 = \frac{\sigma^2}{\sum (X_i - \bar{X})^2}	
\end{equation}
We can estimate the variance of $b_1$ by using the unbiased estimator of $\sigma^2$, MSE: 
\begin{equation}
s^2(b_1) = \frac{MSE}{\sum (X_i - \bar{X})^2}		
\end{equation}

\item Intuition of $\Var(b_1)$: larger $\sigma^2$ (intrinstic error) leads to a larger $\text{Var}(b_1)$, and smaller $\sum_i (X_i - \bar{X})^2$ (small variation of $X$) also leads to a larger $\text{Var}(b_1)$. Intuitively, one could think of $\sum_i (X_i - \bar{X})^2$ as ``effective sample size'': when $X_i = \bar{X}$, sample $i$ has no information of $\beta_1$. To see why large $X_i - \bar{X}$ is more preferred (reducing the variance), we consider this example: 
\begin{equation}
Y_i \sim N(\beta t, \sigma^2)
\end{equation}
then $\hat{\beta} t = \bar{Y}$ is the estimator of $\beta$. Its variance is $\Var{\hat{\beta}} = \sigma^2 / (n t^2)$. 

\item Sampling distribution of SSE: $SSE/\sigma^2$ is independent of $b_1$ and $b_0$, and follows $\chi^2$ distribution with dof $n-2$. \\
Proof: similar to the proof that for normal distribution, sample variance divided by $\sigma^2$ follows $\chi^2$ distribution with dof $n - 1$. The difference now is that in SSE, we have $y_i - b_1 x_i - b_0$, thus we have two parameters (vs. normal distribution: only one parameter, the mean). 

\item $t$-test of $b_1$: $(b_1 - \beta_1) / s(b_1)$ follows $t$-distribution with dof $n-2$. \\
Proof: divide $\sigma(b_1)$ in both numerator and denomiator, the numerator follows the standard normal distribution, and denominator follows $\chi^2$ distribution, so the ratio is $t$-distribution. \\
This distribution can be used to construct $t$ test for the parameter value $b_1$. 

\item Sampling distribution of $b_0$: $b_0$ is also a linear combination of $Y_i$, thus follows normal distribution. Its mean: 
\begin{equation}
\E(b_0) = \E(\bar{Y}) - \E(b_1) \bar{X} = \beta_1 \bar{X} + \beta_0 - \beta_1 \bar{X} = \beta_0	
\end{equation}
To find out its variance, we first show that $\bar{y}$ and $b_1$ are independent. Both are linear functions of $Y_i$, and they are independent if the covariance is equal to 0:
\begin{equation}
\Cov(b_1, \bar{y}) = \Cov\left(\sum_i k_i Y_i, \frac{1}{n} \sum_i Y_i\right) = \sum_i k_i \cdot \frac{1}{n} \Var(Y_i) = \frac{\sigma^2}{n} \sum_i k_i = 0
\end{equation}
Next we find the variance: 
\begin{equation}
\Var(b_0) = \Var(\bar{Y}) + \bar{X}^2 \Var(b_1) = \sigma^2 \left[ \frac{1}{n} + \frac{\bar{X}^2}{\sum_i (X_i - \bar{X})^2}	\right]
\end{equation}
And an estimator of the variance of $b_0$: 
\begin{equation}
s^2(b_0) = MSE \left[ \frac{1}{n} + \frac{\bar{X}^2}{\sum_i (X_i - \bar{X})^2}	\right]
\end{equation}

\item $t$-test of $b_0$: $(b_0 - \beta_0) / s(b_0)$ follows $t$-distribution with dof $n-2$.

\item Covariance between $b_1$ and $b_0$: we use the fact that $\bar{Y}$ and $b_1$ are independent. 
\begin{equation}
\Cov(b_1, b_0) = \Cov(\bar{Y} - b_1 \bar{X}, b_1) = - \bar{X} \Var(b_1) = - \sigma^2 \frac{\bar{X}}{\sum_i (X_i - \bar{X})^2}	
\end{equation}

\end{itemize}

Point estimation of response: 
\begin{itemize}
\item Point estimation of mean response: for any value of $X$, the estimator of the respose variable is given by: 
\begin{equation}
\hat{Y} = b_0 + b_1 X	
\end{equation}

\item Sampling distribution of the point estimator of response: at a point $X_h$, the estimator of $Y_h$ is given by: 
\begin{equation}
\hat{Y_h} = b_0 + b_1 X_h	
\end{equation}
$\hat{Y_h}$ follows normal distribution with: 
\begin{equation}
E(\hat{Y_h}) = Y_h \qquad \text{Var}(Y_h) = \sigma^2 \left[ \frac{1}{n} + \frac{(X_h-\bar{X})^2}{\sum_i (X_i - \bar{X})^2}	\right]	
\end{equation}
To prove the latter equation, first show that $\bar{Y}$ and $b_1$ are independent (since $b_1$ and $b_0$ are dependent), and express the variance as a sum of $\bar{Y}$ and $b_1$, then apply the variances of the two. The estimator of the variance of $\hat{Y_h}$ can be obtained by replacing $\sigma^2$ with MSE. Note that at $X_h$ close to $\bar{X}$, the variance of the response estimator is smaller. 

\item Remark: the prediction problem can be studied in a way similar to parameter estimation: construct the estimator and the confidence interval (which often involves the standard error). 
\end{itemize}

Dignostics and assessing the quality of fit: 
\begin{itemize}
\item Covariance measures the linear dependency between two variables. Anscombe's quartet illustrates that when the linearity does not hold, very different relationships could result in the same correlation. It's important to use (scatter) plot to examine the relationship. 

\item Analysis: we can theoretically analyze how the results change when the assumptions are violated. In the linear model, two mains assumptions are: linearity and normality of errors. Suppose the second assumption is invalid, e.g. there are several outliers in the data, then from this equation: $\Var(\beta_1) = \hat{\sigma}^2 / \sum_i (x_i - \bar{x})^2$, the numerator will be inflated. We will then overestimate the variance of the estimator, which reduces the power of testing $\beta_1$. 

\item Measures of quality of fit: two ideas
\begin{itemize}
\item The correlation between $\hat{Y}$ (the predicted value) and $Y$ measures how prediction agrees with the observation. Furthermore, the measure can be easily generalized to multiple linear regression. 
\item Coefficient of determination: measures how much variance of $Y$ is explained by the variance of the predictor, $R^2 = SSR/SST$ (see the section of ``ANOVA approach to regression''). 
\end{itemize}

\item \textbf{Lessons}: both measures of quality of fit of a linear model are based on very general ideas (1) The agreement of observations and predictions. (2) How much variation in the data is explained by our model.  
\end{itemize}

Normal correlation models: 
\begin{itemize}
\item Bivariate normal model: in the regression model, $X$ is considered as constants. But we could also view $X$ as a RV, and consider the joint distribution of $(X,Y)$. According to the conditional distribution of $Y|X$ (the section ``bivariate normal distribution''), we have: 
\begin{equation}
E(Y|X) = \left(\mu_Y - \mu_X \rho_{YX} \frac{\sigma_Y}{\sigma_X} \right) + \rho_{YX} \frac{\sigma_Y}{\sigma_X}	X
\end{equation}
Thus $Y|X$ follows normal distribution and $E(Y|X)$ is a linear function of $X$. Therefore the linear regression model is equivalent to the conditional distribution of the bivariate normal distribution. In particular, we have the following relations under two models: 
\begin{equation}
\beta_1 = \rho_{YX} \frac{\sigma_Y}{\sigma_X} = \frac{\sigma_{XY}}{\sigma_X^2}
\end{equation}
\begin{equation}
\beta_0 = \mu_Y - \mu_X \rho_{YX} \frac{\sigma_Y}{\sigma_X} = \mu_Y - \beta_1 \mu_X	
\end{equation}

\item Inference on $\rho$: the MLE of $\rho$ is given by the Pearson product-moment correlation coefficient: 
\begin{equation}
r = \frac{\sum (X_i - \bar{X})(Y_i - \bar{Y})}{\sqrt{\sum (X_i - \bar{X})^2 \sum (Y_i - \bar{Y})^2}}	
\end{equation}
The test of $\rho = 0$ is equivalent to the test of $\beta_1 = 0$, and it can be shown that the $t$ test of $\beta_1$ can be expressed in $r$ as: 
\begin{equation}
t = \frac{r \sqrt{n-2}}{\sqrt{1 - r^2}}	
\end{equation}

\item Fisher $z$ transformation: the distribution of $r$ when $\rho \neq 0$ is complicated, so define 
\begin{equation}
z' = \frac{1}{2} \ln \frac{1+r}{1-r}	
\end{equation}
With large $n$, $z'$ is approximately normally distributed with: 
\begin{equation}
E(z') = \frac{1}{2} \ln \frac{1+\rho}{1-\rho} \qquad \sigma^2(z') = \frac{1}{n-3}	
\end{equation}

\item Relation between $R^2$ and $r$: (see ANOVA approach below) plug in the equation of $b_1$: 
\begin{equation}
R^2 = \frac{SSR}{SST} = b_1^2 \frac{\sum (X_i - \bar{X})^2}{\sum (Y_i - \bar{Y})^2}	= \frac{\left[\sum (X_i -\bar{X})(Y_i - \bar{Y}) \right]^2}{\sum (X_i - \bar{X})^2 \sum (Y_i - \bar{Y})^2} = r^2
\end{equation}
\end{itemize}

%%%%%%%%%%%%%%%%%%%%%%%%%%%%%%%%%%%%%%%%%%%%%%%%%%%%%%%%%%%%
\subsection{Multiple Linear Regression} 

Reference: [RABE, 5ed; Hastie, Section 3.2; KNNL, Applied Linear Statistical Models, 5ed]

Linear models: 
\begin{itemize}
\item Definition: the models are linear wrt. the parameters. The function form for predictors can be non-linear. 

\item Model: $p$ independent variables $X_j, j = 1 \cdots p$, and dependent variable $Y$ are related by: 
\begin{equation}
Y = \beta_0 + \sum_{j=1}^{p} \beta_j X_j + \epsilon
\end{equation}
Suppose the error follows normal distribution $N(0,\sigma^2)$. Suppose there are $n$ data points, $(x_i, y_i), 1 \leq i \leq N$ and each $x_i = (x_{i1}, \ldots, x_{ip})$ is a vector of features. We want to estimate $\beta = (\beta_0, \beta_1, \ldots, \beta_p)^T$. 
\end{itemize}

Least square parameter fitting: 
\begin{itemize}
\item Notation: Denote by $\bf{X}$ the $N$ by $(p+1)$ matrix with each row a data point and each column a feature (with 1 at the first position of each row: a dummy feature), and $\bf{y}$ the response vector (column vector). $\beta$ is assumed to be a column vector. $X$ is called the design matrix. Note: without this notation, we would have $X - \bar{X}$, and $y - \bar{y}$ in the equations below, instead of $X$ and $y$. 

\item Least square: To maximize the log likelihood is equivalent to minimizing the residue sum of squares: 
\begin{equation}
\text{RSS}(\beta) = \sum_{i=1}^N (y_i - f(x_i))^2 = \sum_{i=1}^N (y_i - \beta_0 - \sum_{j=1}^p x_{ij} \beta_j)^2
\end{equation}
Then we can write: 
\begin{equation}
\text{RSS}(\beta) = (\bf{y} - \bf{X} \beta)^T (\bf{y} - \bf{X} \beta)
\end{equation}
Take derivative and solve the equatoin: 
\begin{equation}
\hat{\beta} = (\bf{X}^T \bf{X})^{-1} \bf{X}^T y	
\label{eq:least_square}
\end{equation}
And the predicted value for a new data point $\mathbf{x_0}$ is simply $\mathbf{x_0} \hat{\beta}$, and the predicted values for the training input: 
\begin{equation}
\hat{\mathbf{y}} = X \hat{\beta} = X (X^T X)^{-1} X y	
\end{equation}
The $N \times N$ matrix $X (X^T X)^{-1} X$ is called the hat or projection matrix. 

\item Estimator of $\sigma^2$: The unbiased estimator of $\sigma^2$ is given by the sample variance of residuals: 
\begin{equation}
\hat{\sigma^2} = \frac{1}{N-p-1} \sum_{i=1}^N (y_i - \hat{y_i})^2 = \frac{SSE}{N - p -1}
\end{equation}
It is also called mean squared error (MSE). 
 
\end{itemize}

Deriving parameters by method-of-moment (MOM) approach: 
\begin{itemize}
\item Method: given the linear model, we consider the covariance beween $Y$ and $X_k$: 
\begin{equation}
\Cov(X_k, Y) = \sum_j \beta_j \Cov(X_k, X_j) 
\end{equation}
Write in matrix form:
\begin{equation}
\Cov(X, Y) = \Cov(X) \beta
\end{equation}
This describes the relationship between expected variance and covariance and $\beta$. To estimate $\beta$, we use the sample variance and covariance, and have the estimator of $\beta$ in Equation~\ref{eq:least_square}. 

\item Intepretation of the coefficients: $\hat{\beta}$ is determined by the covarances between $X_j$ vectors and $y$ ($X$ matrix is given): a special case, when $X_j$s are independent, $\hat{\beta}_j$ is proportional to $\Cov(X_j, y)$ - this is what we expect by intuition.  

\item Remark: 
\begin{itemize}
	\item This approach assumes centering of variables. Alternatively, when we have dummy variable $X_0 = 1$, we do not need centering. 
	
	\item The equation of $\hat{\beta}$ suggests a relationship between the rank of $X^T X$ and the variance of $\hat{\beta}$: when the matrix $X^T X$ is not full ranked, it is hard to estimate $\beta$, or the variance of $\beta$ is high. 
\end{itemize}
\end{itemize}

Interpretation of regression coefficients: 
\begin{itemize}
\item Geometric interpretation: we could write:
\begin{equation}
\text{RSS}(\beta) = \norm{\mathbf{y} - X \mathbf{\beta}}^2	
\end{equation}
In the $N$-dim. space, $y$ is a vector, $X \beta$ is a point in the subspace created by $X_1, \cdots, X_p$ (linear combination of these $p$ vectors), thus the objective function is the distance from $y$ to some point in the subspace. We choose $\hat{\beta}$ s.t. the residual vector $\mathbf{y} - X \hat{\beta}$ is orthogonal to the column space of $X$. The projection of $Y$ on the subspace is: 
\begin{equation}
\sum_j \beta_j X_j = [X_1 \cdots X_p] [\beta_1 \cdots \beta_p]^T = X \beta	
\end{equation}
The orthogonality implies that $Y - X \beta$ is orthogonal to any $X_j$, i.e. 
\begin{equation}
X_j^T (Y - X \beta) = 0	\qquad 1 \leq j \leq p	
\end{equation}
This is an important property of residuals: they are independent of $X_j$'s. This can be written in the matrix form (one equation above per row): 
\begin{equation}
X^T (Y - X \beta) = 0	
\end{equation}
This is called ``\textit{normal equation}''. The intuition is that the residues should be independent of $X_j$'s - an important property of residues. Solving it gives $\hat{\beta} = (X^T X)^{-1} X^T y$, which can be simply understood as inner product divided by the squared norm of $X$. 
	
\item Relation to simple regression/condtional regression: e.g. $\beta_2$ in a regression involving three explanatory variables should be the regression coefficient of $Y$ on $X_2$ after adjusting all other variables, for both $Y$ and $X_2$. First, we do regression of $Y$ on $X_1$ and $X_3$, and let residuals be $e_{Y \cdot X_1 X_3}$; next we do regression of $X_2$ on $X_1$ and $X_3$, and let residuals be $e_{X_2 \cdot X_1 X_3}$. Then the regression coefficient of $e_{Y \cdot X_1 X_3}$ on $e_{X_2 \cdot X_1 X_3}$ would give $\beta_2$. Thus regression coefficients in a multiple regression model are the \textit{partial regression coefficents}. 
\end{itemize}

Statistical significance of parameters: 
\begin{itemize}
\item The estimator of $\beta$: according to Equation~\ref{eq:least_square}, $\hat{\beta}$ is a linear combination of $\mathbf{y}$. Since $y_i$ follows normal distribution $N(x_i \beta, \sigma^2)$, thus $y$ follows a multivariate normal distribution with mean $X \beta$, and covariance matrix: $\sigma^2 I$. Using the results of linear function of multivariate normal RVs, the mean of $\hat{\beta}$ is: 
\begin{equation}
(X^T X)^{-1} X^T X \beta = \beta	
\end{equation}
Thus $\hat{\beta}$ is unbiased. And the covariance matrix of $\hat{\beta}$ is: 
\begin{equation}
(X^T X)^{-1} X^T \cdot \sigma^2 I \cdot ((X^T X)^{-1} X^T)^T = \sigma^2 (X^T X)^{-1} X^T X (X^T X)^{-1} = \sigma^2 (X^T X)^{-1}
\end{equation}
We use the fact that $X^T X$ is a symmetric matrix. This leads to:\\
\textbf{Theorem}: $\hat{\beta}$ follows normal distribution: 
\begin{equation}
\hat{\beta} \sim N(\beta, (\bf{X^T X})^{-1} \sigma^2)
\end{equation}
Its marginal distribution $\hat{\beta}_j \sim N(\beta_j, v_j \sigma^2)$ where $v_j$ is the $j$-th diagonal element of $(X^T X)^{-1}$. In practice, we often do not know $\sigma^2$, so we replace it with its estimator: 
\begin{equation}
s^2(\hat{\beta}) = (N-p-1)^{-1} (X^T X)^{-1} (y-X\hat{\beta})^T (y-X\hat{\beta})
\end{equation} 

\item Interpretation of the variance of $\beta$: it is proportional to the sample precision matrix of $X$: $\hat{\Cov}(X)^{-1}$. Intuitively, when $X_i$ is linearly dependent on other variables, $\sigma_{ii}$ (the partial covariance of $i$) is small, and thus its inverse is large, so $\Var(\hat{\beta}_j)$ is large. 

\item The estimator of $\sigma^2$: similar to the case of sample variance of the univariate Gaussian distribution: 
\begin{equation}
(N-p-1) \hat{\sigma^2} \sim \sigma^2 \chi^2_{N-p-1}
\end{equation}
In addition, $\hat{\beta}$ and $\hat{\sigma^2}$ are statistically independent. We could also state this as the distribution of the error $SSE$: 
\begin{equation}
\frac{SSE}{\sigma^2} \sim \chi^2_{N-p-1}	
\end{equation}

\item Testing individual coefficients: often we are interested in testing a particular coefficient while controlling for all other variables. To test the hypothesis that a particular coefficient $\beta_j = 0$, compute the standarized coefficient: 
\begin{equation}
t_j = \frac{\hat{\beta_j}}{\hat{\sigma} \sqrt{v_j}}	
\end{equation}
where $v_j$ is the $j$-th diagonal element of ${\bf (X^T X)}^{-1}$. Under the null hypothesis $\beta_j = 0$, $t_j$ is distributed as $t_{N-p-1}$, and if $\sigma$ is known, $t_j$ follows normal distribution. As sample size increasese, the difference between normal and $t$ distribution becomes negligible. 

\item Testing a group of parameters simultanesouly: suppose given $p_0$ coefficients, want to test if more parameters, $p_1$, gives a significant better fit (i.e. test the significance of extra $p_1 - p_0$ parameters). The $F$-test statistic:
\begin{equation}
F = \frac{(\text{RSS}_0 - \text{RSS}_1) / (p_1 - p_0)}{\text{RSS}_1 / (N - p_1 - 1)}	
\end{equation}
See the section on AVOVA below. 
\end{itemize}

Likelihood approach: 
\begin{itemize}
\item Likelihood function: 
\begin{equation}
L(\beta, \sigma^2) = \frac{1}{(2 \pi \sigma^2)^{N/2}} \exp \left[ -\frac{1}{2 \sigma^2} \sum_{i=1}^N (y_i - X_i \beta)^2\right]
\end{equation}
The log-likelihood function is: 
\begin{equation}
l(\beta, \sigma^2) = -\frac{1}{2} \left[\log(2\pi) +  \log \sigma^2 + \frac{1}{\sigma^2} \sum_{i=1}^N (y_i - X_i \beta)^2\right]
\end{equation}

\item MLE: maximizing likelihood is equivalent to minimize squared error, so we have $\hat{\beta} = \hat{\beta}_{\text{LS}}$, and 
\begin{equation}
\hat{\sigma^2} = \frac{\sum (y_i - \hat{y_i})^2}{n}	
\end{equation}
where $\hat{y}_i = X_i \hat{\beta}$. 

\item Likelihood ratio test: suppose we compare two models, one reduced model (R) and one full model (F). The dof. of the two models are $d_R$ and $d_F$ respectively ($n$ minus the number of free parameters), and the difference of the number of parameters in two models is thus $d_R - d_F$. To test if the parameters in the full model are equal to 0, we form the LRT, assuming $\sigma^2$ is known: 
\begin{equation}
-2 [l(\hat{\theta}|R - l(\hat{\theta}|F))] = \frac{\text{SSE}_R - \text{SSE}_F}{{\sigma}^2}
\end{equation}
which follows $\chi^2$ distribution of $d_R - d_F$. However, $\sigma^2$ is not known, so this test is not directly applicable (could use MLE of $\sigma^2$, a different test). But we use the fact that: for the full model:  
\begin{equation}
\frac{\text{SSE}_F}{\sigma^2} \sim \chi^2_{d_F}
\end{equation}
The ratio of the two $\chi^2$ distribution follows the $F(d_R - d_F, d_F)$ distribution: 
\begin{equation}
F = \frac{\frac{\text{SSE}_R - \text{SSE}_F}{{\sigma}^2} / (d_R - d_F)}{\frac{\text{SSE}_F}{\sigma^2} / d_F} = \frac{(\text{SSE}_R - \text{SSE}_F) / (d_R - d_F)}{\text{SSE}_F / d_F}
\end{equation}
This is exactly the same $F$-test above (also see the section on AVOVA approach on regression). 

\end{itemize}

Prediction and residuals [RABE, Chapter 4]:  the predictions and residuals are also random variable and we want to determine their distribution. This would allow us to estimate the accuracy of predictions and the distribution of residuals can be used to check for model violations. 
\begin{itemize}
\item Predictions: for sample $i$, its fitted value is $\hat{y}_i = x_i \hat{\beta}$, or in matrix form: 
\begin{equation}
\hat{y} = X \hat{\beta}
\end{equation}	
To write it in terms of $y_i$'s, we plug in $\hat{\beta}$: 
\begin{equation}
\hat{y} = X (X^T X)^{-1} X^T y = P y
\end{equation}
where $P = X (X^T X)^{-1} X^T$ is the projection matrix. Or: 
\begin{equation}
\hat{y}_i = p_{i1} y_1 + p_{i2} y_2 + \cdots + p_{in} y_n	
\end{equation}
For simple regression, we have:
\begin{equation}
p_{ij} = \frac{1}{n} + \frac{(x_i - \bar{x})(x_j - \bar{x}))}{\sum (x_i - \bar{x})^2}
\end{equation}
In particular, the term $p_{ii}$ is called leverage. Intuitively, when $x_i - \bar{x}$ is large, the leverage is bigger. 
	
\item Properties of the projection matrix: first, it is symmetric
\begin{equation}
P^T = P
\end{equation}
This is easy to check. Next, it is idempotent: 
\begin{equation}
P^2 = X (X^T X)^{-1} X^T \cdot X (X^T X)^{-1} X^T = X (X^T X)^{-1} X^T = P
\end{equation}
From this, it is easy to show that $(I-P)^2 = I-P$. 

\item Distribution of residuals: once a model is fit, we could compute the residuals: 
\begin{equation}
e_i = y_i - \hat{y}_i = y_i - x_i \hat{\beta}	
\end{equation}
To derive the distribution of $e_i$, we have $e = y - \hat{y} = (I-P)y$. Since $y$ is MVN, then clearly $e$ also follow MVN, and its variance is:
\begin{equation}
\Var(e) = (I-P) \Var(y) (I-P)^T = \sigma^2 (I-P)^2 = \sigma^2 (I-P)
\end{equation}
where we used the properties of the projection matrix above. So the variance of the $i$-th residual is given by: 
\begin{equation}
\text{Var}(e_i) = \sigma^2 (1 - p_{ii})	
\end{equation}
So the variance of all residuals are not equal: when $x_i$ is far from $\bar{x}$, we have larger leverage, then the variance of $e_i$ is smaller - easier to predict. Ex. when $x_i$ is 0 (or close to 0), then we cannot predict $y_i$, we have small leverage and large variance of $e_i$. 

\item Standardized and studentized residuals: to address the problem that $e_i$ are not comparable across samples, we standarize the residuals by: 
\begin{equation}
r_i = \frac{e_i}{\sigma \sqrt{1 - p_{ii}}}	
\end{equation}
It has mean 0 and variance 1. We can use unbiased estimator of $\sigma^2$: 
\begin{equation}
r_i = \frac{e_i}{\hat{\sigma} \sqrt{1 - p_{ii}}}	
\end{equation}
where $\hat{\sigma}$ is the estimated $\sigma$. It follows $t$-distribution.

\item Properties of residuals: RSS$(\beta)$ can be written in terms of residuals: 
\begin{equation}
\text{RSS}(\beta) = \sum_i e_i^2	
\end{equation}
Thus from minimization of RSS, we have: $\partial \text{RSS}(\beta)/\partial \beta_j = 0$ for $\hat{\beta}_j$. This leads to the following results: 
\begin{equation}
\sum_i e_i = 0	
\end{equation}
\begin{equation}
\sum_i e_i x_{ij} = 0 \qquad j = 1, 2, \cdots, p	
\end{equation}
\end{itemize}

Gauss-Markov Theorem: suppose we want to estimate $\theta = a^T \beta$ (this is predicting a new data point, if $a = x_0$). Let $\hat{\beta}$ be the least square estimator of $\beta$. The theorem states: 
\begin{itemize}
	\item Unbiased estimator: $E(a^T \hat{\beta}) = a^T \beta$. 
	\item Minimum variance: among all estimators that are linear to $\mathbf{y}$, $\tilde{\theta} = c^T \mathbf{y}$, we have: 
\begin{equation}
\text{Var}(a^T \hat{\beta}) \leq \text{Var}(c^T \mathbf{y})
\end{equation}	
\end{itemize}

Alternative notations of linear regression: (often in machine learning literature)
\begin{itemize}
\item Notation: a data point $\mathbf{x}$ will first be mapped to a feature space, using basis functions $\phi_j(\mathbf{x})$. Then a data point will be represented by a column vector $\phi(\mathbf{x}) = (\phi_1(\mathbf{x}), \cdots, \phi_p(\mathbf{x}))^T$. The parameters will be denoted by a column vector $\mathbf{w}$. The data matrix will be denoted as $N \times p$ matrix (design matrix), $\Phi$, where the $i$-th row is the $i$-th data point $\phi(\mathbf{x_i})^T$. 

\item Least square solutions: the solution can now be written as: 
\begin{equation}
\hat{\mathbf{w}} = (\Phi^T \Phi)^{-1} \Phi^T y	
\end{equation}
And the prediction for a new example $\mathbf{x}$ is: $\hat{y} = \hat{\mathbf{w}}^T \mathbf{x}$. 
\end{itemize}

%%%%%%%%%%%%%%%%%%%%%%%%%%%%%%%%%%%%%%%%%%%%%%%%%%%%%%%%%%%%
\subsection{Generalized Least Square}

Generalized least square (GLS): [Wiki]
\begin{itemize}
\item Motivation: in ordinary linear model, we assume that the errors are iid. When the errors are independent but not identical, or are correlated, we need to generalized the model. 

\item Model: let $X$ be the $N \times p$ design matrix, and $y$ be the response variable ($N \times 1$ vector), we have the linear model
\begin{equation}
y = X \beta + \epsilon \quad \epsilon \sim N(0, \Sigma)	
\end{equation}
The log-likelihood function of $\beta$ is:
\begin{equation}
l(\beta) = - \frac{1}{2} (y-X\beta)^T \Sigma^{-1} (y-X\beta) + \text{const}
\end{equation}
The derivative is: 
\begin{equation}
\frac{\partial l(\beta)}{\partial \beta} = - \frac{1}{2} \cdot 2 (y-X\beta)^T \Sigma^{-1} (-X)
\end{equation}
This leads to the estimating equation: $X^T \Sigma^{-1} (y - X \beta) = 0$, and the solution is:
\begin{equation}
\hat{\beta} = (X^T \Sigma^{-1} X)^{-1} X^T \Sigma^{-1} y	
\end{equation}

\item GLS approach: to generalize, we do not have to assume that the data follows MVN distribution. Rather, the errors only need to follow: $\E(\epsilon) = 0$ and $\Var{\epsilon} = \Sigma$, we minimize the generalized squared error, defined as: 
\begin{equation}
GSE = (y-X\beta)^T \Sigma^{-1} (y-X\beta)	
\end{equation}

\end{itemize}

Weighted least square (WLS):  This is a special case of GLS with $\Sigma$ a diagnoal matrix. Let $\sigma_i^2$ be the variance of the error of the $i$-th observation, then $\Sigma^{-1}$ is the diagnoal matrix, with the diagnoal term $1/\sigma_i^2$.

Diagnonalization of GLS model [personal notes]
\begin{itemize}
	\item Method: we do Spectrum Decomposition $\Sigma = U D U^T$, then we multiply $U^T$ to both sides of the regression model:
	\begin{equation}
	U^T y = U^T X \beta + U^T \epsilon
	\end{equation}
	Then the error follows $U^T \epsilon \sim N(0, U^T \Sigma U) = N(0, D)$. This means that we do variable substitution: let $y' = U^T y$ and $X' = U^T X$, then we have $y' = X' \beta + N(0, D)$ where the errors are independent. 
	
	\item Remark: the same method can be applied to factor analysis. Suppose we have $x_j$, $N \times 1$ vector:
	\begin{equation}
	x_j = \sum_k Z_k W_{jk} + \epsilon_j, \qquad \epsilon_j \sim N(0, \Sigma)
	\end{equation}
	We can use the same trick to diagonalize. This has been used in RSSp, where the errors of nearby SNPs are correlated because of LD; and in GWAS factor analysis of multiple traits, where the errors are correlated b/c sample overlapping.   
\end{itemize}
%%%%%%%%%%%%%%%%%%%%%%%%%%%%%%%%%%%%%%%%%%%%%%%%%%%%%%%%%%%%
\section{Analysis of Variance Approach to Regression} 

Reference: [KNNL, Applied Linear Statistical Models, 5ed, Sections 2.7-2.8, 6.5, 7.1-7.4]

Partitioning of total sum of squares: 
\begin{itemize}
\item Definitions: we define the total sum of squares of the response variable (total variation): 
\begin{equation}
SST = \sum_{i=1}^n (Y_i - \bar{Y})^2
\end{equation}
The measure of the variation of $Y_i$ after $X$ is taken into account is the error sum of squares:
\begin{equation}
SSE = \sum_{i=1}^n (Y_i - \hat{Y_i})^2
\end{equation} 
The regression sum of squares measures the deviation of the predicted $Y_i$ from $\bar{Y}$: 
\begin{equation}
SSR = \sum_{i=1}^n (\hat{Y_i} - \bar{Y})^2
\end{equation}

\item Partitioning: we could write, for every observation: 
\begin{equation}
Y_i - \bar{Y} = \hat{Y}_i - \bar{Y} + Y_i - \hat{Y}_i	
\end{equation}
Take the square in both sides and sum over $i$: 
\begin{equation}
\sum_i (Y_i - \bar{Y})^2 = \sum_i (\hat{Y}_i - \bar{Y})^2 + \sum_i (Y_i - \hat{Y}_i)^2 + 2 \sum_i (\hat{Y}_i - \bar{Y}) (Y_i - \bar{Y})
\end{equation}
It can be shown that the last term is 0, by replacing $(Y_i - \bar{Y})$ with $e_i$, $\hat{Y}_i$ with $x_i \hat{\beta}$, and applying the properties of residuals. This leads to the partition: 
\begin{equation}
SST = SSR + SSE	
\end{equation}
Thus we could view $SSR$ as the explained variation and $SSE$ as the unexplained variation.

\item Degree of freedom (df): the df of $SST$ is $N - 1$, the lost degree comes from the constraint that the sum of $Y_i - \bar{Y}$ must sum to 0. The df of $SSR$ is $p$, the number of free parameters ($\beta_0$ is not counted as we are only interested in the deviation from the mean). The df of $SSE$ is $N - p -1$, where $p + 1$ comes from the number of constraints of residuals. The df's also satisfy: 
\begin{equation}
df(SSE) = df(SSR) + df(SSE)	
\end{equation}

\item Mean squares: defined as the ratio of sum of squares and the df: 
\begin{equation}
MSR = \frac{SSR}{p}	
\end{equation}
\begin{equation}
MSE = \frac{SSE}{N-p-1}	
\end{equation}
The expected value of $MSE$ is simply:  
\begin{equation}
E[MSE] = \sigma^2	
\end{equation}
The expected value of $MSR$ is $\sigma^2$ plus a nonnegative number, e.g. for $p = 1$: 
\begin{equation}
E[MSR] = \sigma^2 + \beta_1^2 \sum_i (X_i - \bar{X})^2	
\end{equation}

\item Coefficient of determination: we could define a measure of goodness-of-fit as the fraction of explained variation: 
\begin{equation}
R^2 = \frac{SSR}{SST} = 1 - \frac{SSE}{SST}	
\end{equation}
Also note that it equals the square of the correlation coefficient between the $y$ and $\hat{y}$.   
\end{itemize}

ANOVA approach to univariate regression: 
\begin{itemize}
\item Idea: if $\beta_1$ is large, then $X$ and $Y$ are highly correlated, and we expect a significant part of the variance of $Y$ can be explained by $X$. So we could use the variance partitioning to infer/test the parameter of the linear model. 

\item $F$-test of $\beta_1 = 0$: from expected $MSE$ and $MSR$, we see that if $\beta_1 = 0$, they have the same expectation; and would be different if $\beta_1 \neq 0$. This motivates the $F$-test: 
\begin{equation}
F = \frac{MSR}{MSE}	
\end{equation}
$F$ statistic follows the $F(1,n-2)$ distribution, where $1$ and $n-2$ are df. of $SSR$ and $SSE$ respectively. \\
Proof: we know that $SSE / \sigma^2$ follows $\chi^2$ distribution with df equal to $n - 2$. For $SSR$, we know that it is equal to $b_1^2$ times constant. $b_1$ follows normal distribution, under $H_0: \beta_1 = 0$, we have: 
\begin{equation}
\frac{b_1}{\sigma/\sqrt{\sum_i (X_i - \bar{X})^2}} \sim N(0,1)	
\end{equation}
Its square, which is $SSR / \sigma^2$ thus follows $\chi^2$ distribution with df $1$. The ratio of the two $\chi^2$ distribution (divided by d.f.) thus follows $F$-distribution. \\
When $H_A: \beta_1 \neq 0$, $F$ follows the noncentral distribution. 

\item Equivalent of $F$ and $t$ test of $\beta_1$: use $SSR = b_1^2 \sum(X_i - \bar{X})^2$, and $s^2(b_1) = MSE / \sum (X_i - \bar{X})^2$, we could prove that: 
\begin{equation}
F = \frac{b_1^2}{s^2(b_1)} = t^2	
\end{equation}
where $t$ is the test statistic of the Student $t$-test. 

\item Descriptive measure of linear association: the partition of sum of square motivates the following measure, the coefficient of determination, of how good the linear model explains the data: 
\begin{equation}
R^2 = \frac{SSR}{SST}	
\end{equation}
It is interpreted as the fraction of variance explained by the predictor variable $X$. Generally, higher linear assocition means larger $R^2$. \\
Caution: if $X$ and $Y$ may not be linearly associated, $R^2$ may have very limited use. Ex. it is possible that $X$ and $Y$ are strongly dependent (but nonlinear), but $R^2$ is close to 0. 

\item Remark:  approach is a way of selecting models: for a variable $X$ to be selected, need to reject $H_0: \beta_1 = 0$. This only happens if adding $X$ explains a significant amount of variation ($SSR$ term), and simpler models are preferred because our prior belief is $H_0$ is more likely (the conservative nature of classical hypothesis testing). Comparing with approaches that explicitly penalize complex models such as Lasso and Bayesian model selection, ANOVA approach has a few drawbacks:
\begin{itemize}
	\item Penalization is implicit: thus the prior belief of how $H_0$ is likely vs $H_A$ is never specified, and not tested. 
	\item Multiple hypothesis testing: when selecting among multiple models, we are testing multiple hypothesis. Not clear how correction should be done. 
\end{itemize}
\end{itemize}

Partition using extra sum of squares: in multiple regression, we are interested in the question of marginal reduction of error when extra variables are introduced. 
\begin{itemize}
\item Extra sum of squares: suppose we want to know the marginal effect of adding $X_3$ to the regression model which already contains $X_1$ and $X_2$, we have two equivalent forms: 
\begin{equation}
SSR(X_3|X_1, X_2) = SSE(X_1, X_2) - SSE(X_1, X_2, X_3) = SSR(X_1, X_2, X_3) - SSR(X_1, X_2)	
\end{equation}
The equality of the two forms: the increase of explained variation should be equal to the decrease of unexplained variation. 

\item Decomposition of sum of squares: in general, we have many different ways of paritioning $SSR$ or $SST$, e.g.: 
\begin{equation}
\begin{array}{ll}
SSR(X_1, X_2, X_3) & = SSR(X_1) + SSR(X_2|X_1) + SSR(X_3|X_1,X_2)\\
SSR(X_1, X_2, X_3) & = SSR(X_2) + SSR(X_1,X_3|X_2)
\end{array}	
\end{equation}
The df of each conditional SSR is simply the number of extra parameters. And this allows one to define the conditional MSR, e.g. 
\begin{equation}
MSR(X_2,X_3|X_1) = \frac{SSR(X_2,X_3|X_1)}{2}	
\end{equation}

\item Coefficient of partial determination: this is defined as the fraction of explained variance by the extra variables over all the variance in the current model (with some variables already in the model). Example: 
\begin{equation}
R^2_{X_1|X_2,X_3} = \frac{SSR(X_1|X_2,X_3)}{SSE(X_2, X_3)}	
\end{equation}
This could also be explained as: suppose we regress $Y$ on $X_2$ and $X_3$ and obtain residuals, $e_i(Y|X_2, X_3)$; and we regress $X_1$ on $X_2$ and $X_3$ and obtain residuals, $e_i(X_1|X_2,X_3)$, then the coefficient of determination $R^2$ between the two sets of residuals is $R^2_{X_1|X_2,X_3}$. The square root of the coefficient of paritial determination is the partial correlation coefficient. 
\end{itemize}

$F$ test of general linear models: 
\begin{itemize}
\item Full model vs. reduced model: our test is whether the full model is significantly better than a resrictied/reduced model, e.g. some $\beta_k = 0$ where $\beta_k$ is the extra parameter (or parameters). Suppose $SSE(F)$ is the SSE of the full model, and $SSE(R)$ is the SSE of the reduced model, then the test statistic is a function of $SSE(R) - SSE(F)$: 
\begin{equation}
F = \frac{SSE(R)-SSE(F)}{df_R - df_F} / \frac{SSE(F)}{df_F}	
\end{equation}
The $F$ test statistic follows the $F$ distribution with df $df_R - df_F$ and $df_F$. 
\begin{itemize}
\item Ex. to test if $\beta_2 = \beta_3 = 0$ in a regression of 3 variables, we have: $SSE(F) = SSE(X_1, X_2, X_3)$, $SSE(R) = SSE(X_1)$, and $SSE(R) - SSE(F) = SSR(X_2, X_3|X_1)$.
\item Also note that, the constraints in the reduced model do not have to be some $\beta_k = 0$, it could also be, e.g. $\beta_1 = \beta_2$. In this case, the $F$ test still applies, but not the extra sum of square (conditional SSR).
\end{itemize}

\item Proof of the $F$ test: to show that $F$ under $H_0$ follows $F$ distribution, we first note that the denominator divided by $\sigma^2$ follows $\chi^2$ distribution. For the numerator, we note that it is the variance explained by the additional parameters in the $F$ model. Similar to our proof in the univariate case ($SSR/\sigma^2$ follows $\chi^2$ distribution), we can show that it also follows $\chi^2$ distribution. 
\begin{itemize}
	\item Remark: a rigorous proof could use Cochran's Theorem. 
\end{itemize}

\item Example: dealing with confounding variables. Suppose we want to test the effect of $A$, in the presence of a confoudning variable $B$. Thus the full model is $F = (A,B)$ and the reduce model is $R = B$. We have: $SSE(F) = SSE(A,B)$, and 
\begin{equation}
SSE(R) - SSE(F) = SSE(B) - SSE(A,B) = SSR(A|B)	
\end{equation}
Therefore, the $F$ statistic is determined by $SSR(A|B)$, the extra variation explained by $A$ conditioned on $B$. This relates to the idea of ``stratification'': $SSR(A|B)$ is computed on each stratum defined by $B$. 
\end{itemize}

Relationship between $R^2$ and regression coefficients [personal notes]
\begin{itemize}
\item Motivation: in statistical genetics, we want to estimate heritability (or proportion of variance explained or PVE) from the effect size estimates (and other related quantities, including the standard error). 

\item Simple regression: suppose the model is $y = x \beta + \epsilon$, we want to estimate $R^2$ from $\hat{\beta}$, $\Var(\beta)$ and $\Var(x)$. We take the variance of $y$: 
\begin{equation}
\Var(y) = \beta^2 \Var(x) + \sigma^2
\end{equation}
To determine $R^2$, we need $\sigma^2$. We use this equation: 
\begin{equation}
\Var(\hat{\beta}) = \frac{\sigma^2}{\Var(x)}
\end{equation}
In summary, our estimate of $R^2$: 
\begin{equation}
\hat{R^2} = \frac{\hat{\beta}^2 \Var(x)}{\hat{\beta}^2 \Var(x) + \hat{\sigma}^2} \qquad \hat{\sigma}^2 = \Var(\hat{\beta}) \Var(x)
\end{equation}

\item Multiple regression when the explanatory variables are independent: the variance of $y$ is given by: 
\begin{equation}
\Var(y) = \beta^2 \Var(x) + \sigma^2 = \sum_j \beta_j \Var(x_j) + \sigma^2
\end{equation}
And the standard error of $\hat{\beta}$ is given by: 
\begin{align}
\Var(\hat{\beta}) = \sigma^2 (X^T X)^{-1}
\end{align}
This leads to the estimate: 
\begin{equation}
\hat{R^2} = \frac{\sum_j \hat{\beta_j}^2 \Var(x_j)}{\sum_j \hat{\beta_j}^2 \Var(x_j) + \hat{\sigma}^2} \qquad \hat{\sigma}^2 = \Var(\hat{\beta}) (X^T X)
\end{equation}

\item Logistic model: when $Y$ is binary, we can still talk about the same questions, using a liability model. An alternative may be to use conditional distributions, when $X$ is also discrete. For example, to see how important $X$ is: $P(Y=1|X=1) - P(Y=1)$ estimates how many cases of $Y=1$ is due to $X = 1$. 
\end{itemize}

\subsection{Linear Regression with Categorical Variables}

Reference: [RABE, 5th ed, Chapter 5]

Motivation: 
\begin{itemize}
\item Qualitative variables: often we have such variables, such as gender or occupation, that may influence the response variables. Therefore we need to incorporate these variables in the analysis. But their model is different from continuous variables as we need to ``code'' them, and the value of the code itself does not have a numerical meaning. 

\item Groups: often the samples may have some group structure, and the response variables may differ across groups. So we could encode the groups as qualitative variables. The point here is that even if there is explicit/natural way of grouping (and naming it with a group variable), the group structure may still exist.  

\item Non-linear effects: sometimes the effect of a variable depends on some other conditions - one way to model this is to create groups where the effect is homogeneous, then the interaction between the variable and the group membership variable models the non-linear effect. 
\end{itemize}

Basic model of dealing with categorical variables (factors): 
\begin{itemize}
\item Factor: if we have a cateorical variable, then we call it a ``factor''. A factor can have multple levels. The basic strategy of modeling factors in regression is: define one level as reference/base/control level, and compare the other levels with this one. If we have $K$ levels, we have $K-1$ indicator variables, one for each level other than the reference. 
	
\item Salary survey data: we study the relationship $Y$ - salary, and some variables including $X$ - experience (continuous), $E$ - educator, HS (high-school) or BS (bachelor) or advanced (base level), and $M$ - management (1 or 0). Our model: 
\begin{equation}
Y = \beta_0 + \beta_1 X + \gamma_1 E_1 + \gamma_2 E_2 + \delta_1 M
\end{equation}
where $E_1, E_2$ are the two levels (HS and BS). The interpretation of coefficients: the effect of a variable (over base level), when controlling for all other variables. Effectively, controlling for categorical variables does not matter here: because the effect does not depend on the categorical variables, so we can simply assume all categorical variables are at the base level (coded as 0). 
	
\end{itemize}

Interaction model: 
\begin{itemize}
\item Detecting interactions: residual plot could help. Plot the residuals against the categorical variables and see if they are homogeneous against differnet groupings. 

\item Model: for the salary example, we found that there is an interaction between $E$ and $M$, so the new model: 
\begin{equation}
Y = \beta_0 + \beta_1 X + \gamma_1 E_1 + \gamma_2 E_2 + \delta_1 M + \alpha_1 (E_1 \cdot M) + \alpha_2 (E_2 \cdot M)
\end{equation}

\item The interpretations of coefficients: 
\begin{itemize}
	\item Marginal coefficients: The effect of a variable (vs. base) when controlling for all other variables (categorical variables at the base level). Example: $\beta_1$ means the effect of HS education when $M = 0$ and $X$ at the mean (or 0). 
	\item Interaction coefficients: The differential effect of one variable when the other variable is at the non-base level, and controlling for all other variables. Example: $\alpha_1$ is the difference of the effect of HS education when $M = 1$, vs. the effect of HS education when $M = 0$. In other words, $\beta_1$ is the effect of $E_1$ when $M = 0$, and $\beta_1 + \alpha_1$ is the effect of $E_1$ when $M = 1$ (adjusting for the effect of $M$ itself). 
\end{itemize}

\item Alternative model: we can also encode interaction simply by treating each combination as one group. In the salary example, $E$ could take three values and $M$ two values, then there are 6 possible groups. We can simply model them as one variable of 6 levels. We can also replace the intercept with the mean of the base group, then the model is simple (symmetric). For example, let $G_i$ be the $i$-th group, our model: 
\begin{equation}
y = \beta_1 G_1 + \beta_2 G_2 + \cdots \beta_6 G_6
\end{equation}
The coefficients now have simple interpretation: mean of a group. Statistical problem may become testing if coefficients are equal. 

\end{itemize}

Comparing two groups: systems of regression equation
\begin{itemize}
	\item Example: suppose we study the relationship between $X$ (test score) and $Y$ (job performance) on subjects of different races. The relationship between $Y$ and $X$ can be different in different races. To model this, we can create two regression models, one for subjects of each race. 
	
	\item Model: we could create a single regression model, and it would be easier to test hypothesis. The basic idea is to model interaction bwteen $X$ and the group variable. Example, let $Z$ be the race variable, our model is: 
	\begin{equation}
	Y = \beta_0 + \beta_1 X + \gamma Z + \delta (X \cdot Z) 
	\end{equation}
	So if $\gamma \neq 0$, it means that the race could affect the performance; if $\delta \neq 0$, it means that the effect of $X$ depends on the race. To test the hypothesis that there is discrimanation against race, we test: $H_0: \gamma = 0, \delta = 0$. This can be done with F-test.  
	
	\item Special case 1: same slope different intercepts. This is the special case above, and we do not have the interaction terms. We test $H_0: \gamma = 0$. 
	
	\item Special case 2: same intercepts different slopes. We do not have the term $\gamma Z$, and we test $H_0: \delta = 0$. 
\end{itemize}

ANOVA: a special case of regression model with categorical variable
\begin{itemize}
	\item Problem: suppose we test if the mean is different across $K$ groups $H_0: \mu_1 = \cdots = \mu_K$. 
	
	\item Model: we create group variables $X_1, \cdots, X_{K-1}$, which takes $K - 1$ values, and we test $H_0: \beta_1 = \cdots = \beta_{K-1} = 0$. 
\end{itemize}

Analysis of one example: study treatment effects across subjects [personal notes] 
\begin{itemize}
\item Problem: we want to study the treatment effects (potentially many kinds of treatments/drugs), but the effect may differ across subjects. We ask questions such as: is there any treatment effect (averaging across subjects)? How often the effect varies across individuals? 

\item Model: We have a factor of treatment with three levels: control, low dosage and high dosage. We define two variables $T_1$ for the effect of low dosage vs. control; and $T_2$ for high dosage vs. control. Then our basic model is :
\begin{equation}
Y = \beta_0 + \beta_1 T_1 + \beta_2 T_2 + \epsilon 
\end{equation}
where $\beta_1$ and $\beta_2$ describe the effects of low and high dosage, and $\beta_0$ is the baseline (control) of the response variable.

\item Paired design: when we do not care about subject difference, we assume it is the same, and we only need to test the effect, but adjusting for the background difference across subjects. So we have a paired design, where the treatment and control are paired in each subject. Our model: 
\begin{equation}
Y = \beta_0 + \beta_1 T + \beta_2 S + \epsilon
\end{equation}
where $T$ is treatment and $S$ subject. If we have two subjects, then $S$ is the difference between the two (0 for one, 1 for the other). 

\item Differential effect: to answer the question about variation of effects across subjects, we include interaction in the model. Suppose we have two subjects ($S_1$ is the referene), our model:
\begin{equation}
Y = \beta_0 + \beta_1 T + \beta_2 S + \gamma(S \cdot T) + \epsilon
\end{equation}
Then $\beta_1$ is the treatment effect in $S_1$ (reference subject), and $\beta_2$ the baseline of $S_2$ (the subject effect) and $\gamma$ the difference of the treatment effect in $S_2$ (so $\beta_1 + \gamma$ gives the treatment effect in $S_2$). 

\item Interpretation of coefficients: four cases
\begin{itemize}
	\item $\beta_1 = 0, \gamma= 0$: no effect in both subjects. 
	\item $\beta_1 \neq 0, \gamma= 0$: same effect in both subjects. 
	\item $\beta_1 = 0, \gamma \neq 0$: no effect in $S_1$, effect in $S_2$. 
	\item $\beta_1 \neq 0, \gamma \neq 0$: effect in both, but different sizes (including the case: effect in $S_1$, but no effect in $S_2$). 
\end{itemize}

\item Using alternative encoding: we could also model $S \times T$ as groups. Suppose there are four combinations, we have: 
\begin{equation}
Y = \beta_1 G_1 + \beta_2 G_2 + \beta_3 G_3 + \beta_4 G_4
\end{equation}
Suppose $G_1$ is $S_1 \times $control, $G_2$ is $S_2 \times $treatment, and similarly for $G_3$ and $G_4$. Then $\beta_2 - \beta_1$ is the treatment effect in $S_1$, and $\beta_4 - \beta_3$ the effect in $S_2$, and $(\beta_4 - \beta_3) - (\beta_2 - \beta_1)$ the difference of effect between $S_2$ and $S_1$.  

\item Estimating the fraction of differential effects: sometimes we may want to estimate how often each of the scenarios occur (among many treatments, e.g. in genomics, could be many genes). Ideally, we should use a Bayesian approach: mixture prior on the effect sizes, then estimate the proportion. 
\end{itemize}
%%%%%%%%%%%%%%%%%%%%%%%%%%%%%%%%%%%%%%%%%%%%%%%%%%%%%%%%%%%%
\section{Linear Regression in Practice} 

Reference: [Gelman07, chapter 2, 3; Hastie, chapter 3; KKNL, Applied Linear Statistical Models], [Chatterjee, Regression analysis by example (RABE), 4ed]

Assumptions of linear model and possible violations [RABE, Chapter 4]: 
\begin{itemize}
	\item Linearity assumption: about the coefficients. Data transformation is often used to make sure this assumption holds. 
	
	\item Normality assumption of the errors: this can be assessed with appropriate graphs of residuals. 
	
	\item Constant variance assumption of the errors (homoscedasticity). 
	
	\item Independent-error assumption: auto-correlation problem. 
	
	\item Collearity problem: when $X_j$'s are not linearly independent. 
\end{itemize}

Graphical methods [RABE, Chapter 4]
\begin{itemize}
\item Motivation: Anscombe quarter (Figure 4.1). Plot $Y$ against $X$ in scatter plot. The same correlation cofficient and regression model, but very different patterns. In (b) quadratic relation; (c) outlier; (d) influential point. 

\item Overview: use graphical methods to explore the relation between variables, recognize patterns such as clusters, detect errors, detect outliers and highly influential observations. 

\item Explorative analysis of data: include the distribution of variables and relationship between variables. 
\begin{itemize}
	\item 1D graphs: such as histogram, dot plot, box plot for the distribution of $X$'s or $Y$'s. May suggest necessary transformation, outlier, etc. 
	\item 2D graphs: scatter plot for $Y$ against $X$. In the mutiple regression case, plot matrix. However, note that, it is possible that even if the model is linear, the pairwise relationship can differ dramatically (e.g. no correlation). So one should control for other variables when creating the plots. 
\end{itemize}

\item Checking linearity and normality assumptions using residuals: if the linear model is a reasonable fit, then one should expect the standardized residuals to be $N(0,1)$ distributed. 
\begin{itemize}
	\item Normal probabability plot: directly check if the standardized residuals follows standard normal. 
	\item Index plot of the standardized residuals. 
	\item Standardized residuals against predictor variables
	\item standardized residuals against the fitted values ($\hat{Y}_i$). 
\end{itemize}

\item \textbf{Lesson}: to check for the validaty of a model, one can: 
\begin{itemize}
	\item Directly examine the assumption of the model: e.g. linear relationship between $Y$ and $X$. 
	\item Examine model predictions: in the linear model case, the predicted values should be close to the actual values of the response variables, more precisely, the errors should follow normal distribution. 
\end{itemize}
\end{itemize}

Outliers and influential poins: [RABE, Chapter 4]
\begin{itemize}
\item Outliers: in response variables. They can be detected by the residual plots. 

\item Influential points: outliers in predictors. These points have large leverage $p_{ii}$. High-leverage points may not be detected by residual plot because these points often dominate the model fitting thus have small residuals. 

\item Potential-residual plot: define the potential function as a function of leverage (monotically increasing) and residual function as a function of normalized residual. Then for each point, plot its potential and residual. The plot can aid in classifying unusual observations as high-leverage points, or outliers or both. 

\item What to do with outliers? They should not be automatically discarded. Instead, they may be very informative (model assumptions, etc.). One should examine these outliers, then take appropriate actions such as: down-weighting or deleting outliers, transforming data, considering a different model, etc. 
\end{itemize}

Adding variables to a regression equation [RABE, Chapter 4] After adding a variable, there are four cases
\begin{itemize}
\item The new variable has a insignicant coefficient and not change other parameters: the new variable can be ignored. 

\item The new variable has a significant coefficient and no change other parameters: the new variable should be included. 

\item Whenever the new variable substantially change other variables, should check for colinearity. 
\end{itemize}

Feature transformation: 
\begin{itemize}
\item Rationale: for choosing and transforming features are two fold: 
\begin{itemize}
	\item Scale/linearity: the scale of a feature $X$ should be chosen s.t. when $X$ increases by two-fold, we expect its contribution to $Y$ als increases by two fold. This is very important consideration, for example, we consider a model where $X$ is inverse of the allele frequency of a SNP, and $Y$ is phenotype. The bigger $X$, we expect the bigger $Y$, however, the scale of $X$ is incorrect: there is no simple linearity between $Y$ and $X$. 
	\item Additivity: important for combining features. Ex. the phenotype depends on all alleles of a gene (some strong, some weak), the additivity question is essentially the question of whether one strong site is equivalent to how many weak sites. 
\end{itemize}

\item Centering: often helps simplifies the model, $z_j = x_j - \bar{x}$. 

\item Standarization of features: often helps interpretation. Ex. the intercept $\beta_0$ is now the average outcome variable when the data point is average (if not standardized, the feature value at 0 may be meaningless, e.g. body height). It particularly helps interaction. Ex. consider the regression problem: 
\begin{equation}
\texttt{earn} \sim \texttt{height} + \texttt{male} + \texttt{height} \cdot \texttt{male}
\end{equation}
where $\texttt{male}$ is the binary variable of sex (1 if male, 0 female). The coefficients now have interpretation: (1) intercept: the earning of females of average height; (2) coefficient of $\texttt{height}$: effect of height in females; (3) coefficient of $\texttt{male}$: the average difference of male vs females; (4) coefficient of $\texttt{height} \cdot \texttt{male}$: the increase of the effect of height in males, relative to females. 

\item Unit length scaling: alternative way is to scale the features: $z_j = x_j / \norm{x}$, then $\norm{z_j} = 1$. 

\item Logarithmic transformation: when the variables are positive, often do log. transformation. Suppose we do log. transformation on the response variable $Y$, we have: 
\begin{equation}
\log Y = f(X) \qquad \log Y' = f(X + \Delta X) \Rightarrow \frac{Y'}{Y} = \exp[f(X+\Delta X) - f(X)]	
\end{equation}
Thus the coefficient of a feature $X$ means: the increase of $Y$ (in multiplicative terms) due to the increase of $X$ by a unit. Thus if coefficient is 0.081, it means $Y$ is increase by $\exp(0.081) = 1.084$, i.e. $Y$ increases 8.4\%. The same interpretation is useful in logistic regression. 

\item Categorical variables: if there are multiple categories, need to define multiple indicator variables, one for each category. But since the indicators variables are not independent (sum to 1), one category is chosen as the reference, and the effect of any other category is always relative to this reference category. 

\end{itemize}

Rank deficiency and feature correlation: 
\begin{itemize}
\item When $X$ is not of full rank, then $X^T X$ is singular, and the least square fit $\hat{\beta}$ is not well-defined. This could result from: (1) correlation of features; (2) $p > N$: more features than the number of data points.

\item Correlated features: if multiple features are correlated, then a true causal feature (assume it exists) may not be chosen, if its effect has alreay be explained by other correlated features. Alternatively speaking, there may be multiple ways of choosing the model (features) that explain the data equally well. Ex. [Hastie] in heart disease risk data, the feature, blood pressure is not chosen. 

\end{itemize}

Model diagnosis: the goal is to check if the model is sufficient to model the variation of data points. 
\begin{itemize}
\item Residual plot: we could test the assumptions by 
\begin{itemize}
	\item Scatter plot: $r_i$ should be uncorrelated with $\hat{y_i}$ or $x_i$, thus we could draw scatter plot of $r_i$ vs. $\hat{y_i}$ or $x_i$. 
	\item Probability plot: $r_i$ should follow $N(0,1)$ distribution, thus we could draw normal Q-Q plot and compare the observed $r_i$ distribution with the normal distribution. 
\end{itemize}

\item Outliers: several ways to deal with outliers in the residual plot: 
\begin{itemize}
	\item Sometimes they hint that our model assumptions are wrong.
	\item Down-weight or delete outlying data points.
\end{itemize} 

\end{itemize}

Robustness of linear model: 
\begin{itemize}
\item Problem: suppose we are testing the effect of $X$ on $Y$, but we need to control for $Z$. If the true relation between $Z$ and $Y$ is not-linear, e.g. $Y$ increases with larger $Z$, but reaches a plateau when $Z$ is large enough. Then fitting a linear model between $Z$ and $Y$ may overcorrect $Z$ when it is large, and undercorrect $Z$ when it is small. As a result, even if $Y$ is similar for different $X$, its residual (after subtracting the effect of $Z$) may show correlation with $X$. 

\item Example: [Large-Scale Psychological Differences Within China Explained by Rice Versus Wheat Agriculture, Science, 2014] Study how invention (number of patents) varies with rice/wheat agriculture, controlling GDP per capita. If GDP has a non-linear effect on patents, then one may not control for it correctly, and this can lead to false correlation between patents and rice. 
\end{itemize}
%%%%%%%%%%%%%%%%%%%%%%%%%%%%%%%%%%%%%%%%%%%%%%%%%%%%%%%%%%%%
\section{Generalized Linear Models}
\begin{enumerate}

\item{Introduction to generalized linear models (GLM)} 

Reference: [Agresti, Introduction to Categorical Data Analysis, 2ed, Chapter 3] [McCullagh \& Nelder, Section 2.2, Chapter 6], [KNNL, Chapter 14]

Problems of GLM: 
\begin{itemize}
\item Relation to contingency table analysis. 

\item Extensions of GLM: dependency between samples, different variance of errors. 
\end{itemize}
 
Lessons:
\begin{itemize}
\item When developing a GLM, ask if the assumption made by the model fit the characteristics of the data. For example, logistic function describes the sigmoid relation between $\pi$ and $x$. When the situation is: $\pi$ initially increases (linearly) with $x$ when $x$ is small, but approaches to 1 as $x$ increases, then sigmoid model may not fully capture this situation. 

\item Checking model assumptions: e.g. Poisson regression assumes that sample mean and sample variance are equal - this assumption can be tested in real data. 

\item Model checking using residuals (or more generally model predictions): we assess the model predictions at given samples, and compare those with observations. If the model fit is good, the residuals should follow some distributions and generally small. 
\end{itemize}
 
GLM: the three components. 
\begin{itemize}
\item The random component: the distribution of $Y$ given $X$. This may depend on some dispersion parameter $\phi$. 
	
\item The systematic component: the linear predictor
\begin{equation}
\eta = \sum_{j=1}^p x_j \beta_j	= X \beta
\end{equation}

\item The link function between the two components: let $\mu = \E(y|X)$: 
\begin{equation}
g(\mu_i) = X_i \beta
\end{equation}
The function $g(\cdot)$ should be a monotonically differentiable function. Equivalently, we could have: $\mu_i = g^{-1}(X_i \beta)$. For linear regression, the link function is identity link. When the link function is log, we have log-linear model; when it is logit function, it is the logistic regression model. 
\end{itemize}

Common GLM for binary data: [Agresti, 3.2], [GCSR, Chapter 16]
\begin{itemize}
\item Logistic regression model: the response variable follow Bernoulli distribution with parameter $\pi(x)$, and it increases with the independent variable $x$. The link function is logit: 
\begin{equation}
\log \frac{\pi(x)}{1 - \pi(x)} = \alpha + \beta x
\end{equation}
The parameter $\beta$ describes the direction and the magnitude of change of $\pi(x)$ with $x$, and $\alpha$ is similar to the ``threshold'' - when it is large (negative value), need large value of $x$ to overcome it. 

\item Probit model: when the response variable is between 0 and 1, we could view $Y_i$ as a probability. We can convert it to the $Z$-scores and model $Z$-scores as linear functions of $X$. More generally, using the CDF, we could relate a probability to the value of some random variable, let it be $X \beta$ for our regression purpose: 
\begin{equation}
\E(Y_i|X_i) = \Phi( X_i \beta)
\end{equation}
Thus this is a GLM with link function $g(\cdot) = \Phi^{-1}(\cdot)$. When the response variable is binary, we have: 
\begin{equation}
P(Y_i=1|X_i) = \Phi( X_i \beta)
\end{equation}
This is similar to the logistic regression model, except that the logit function is replaced by the CDF of standard normal distribution. 

\item Interpretation of discrete-data model in terms of latent continuous data: sometimes a model with discrete data may be equivalent to a model with latent continuous variable. Ex. for the probit model above, we imagine some latent variable $u_i$ corresponding to $X_i \beta$: $u_i \sim N(X_i \beta, 1)$, and $y_i$ is determined by $u_i$ via: 
\begin{equation}
y_i	= \left\{ \begin{array}{ll}
1 & \text{if $u_i \geq 0$}\\
0 & \text{if $u_i < 0$}
\end{array} \right.
\end{equation}
The advantage of this model is: the first step is a simple distribution, and the second step is deterministic, so it allows a simple Gibbs sampler.
\begin{itemize}
	\item Remark: this is a special case of data augmentation, which facilitates Gibbs sampling (or EM). 
\end{itemize}

\end{itemize}

Common GLM for count data [Agresti, 3.3], [GCSR, Chapter 16]
\begin{itemize}
\item Poission regression: we often need to model the count data, which is assumed to follow Poisson distribution. Suppose $Y$ given $X$ follows Poisson distribution with mean $\mu$, and the link function is $\log$, we have, $\log \mu_i = X_i \beta$, or, 
\begin{equation}
y_|X_i \sim \text{Poisson}(\exp(X_i\beta))	
\end{equation}
In the simple case (one covariate), we have: 
\begin{equation}
\log \mu_i = \alpha + \beta x_i
\end{equation}
The meaning of $\beta$ is: a one-unit increase of $x$ has a multiplicative effect of $e^{\beta}$ on $\mu$. The log-likelihood function is given by: 
\begin{equation}
l(\beta) = \sum_i y_i X_i \beta - \sum_i e^{X_i \beta} - \sum_i \log y_i!
\end{equation}
Note that the last term is constant. 

\item Often $x$ is continuous, then $\mu$ at a particular value of $x$ is not well-defined. We may need to divide $x$ into bins for this analysis.  

\item Overdispersion problem: Often the Poission assumption is violated because of overdispersion (the variance is larger than expected from Poisson distribution). This could be caused by for example, fail to include other independent variables (this would not be a problem for normal model, because it has a separate parameter for variance) - then there is additional heterogeneity between samples. 

\item Negative Binomial regression: we define the negative binomial by two parameters $\mu$ (mean) and $D$ (dispersion parameter): $y_i \sim \text{NB}(\mu_i, D)$, and the parameters are:
\begin{equation}
\E(Y) = \mu, \quad \Var(Y) = \mu + D \mu^2
\end{equation}
Negative binomial GLM is then similar to Poisson regression: using the log link function: 
\begin{equation}
\log \mu_i = \alpha + \beta x_i
\end{equation}

\item Count regression with rate data: in some problems, our data is rate: e.g. number of murders in a city of certain population size. So for each sample $Y_i$, we have an index $t_i$ (such as population size), the log-linear model has the form: 
\begin{equation}
\log (\mu_i / t_i) = \alpha + \beta x_i
\end{equation}
Or equivalently: 
\begin{equation}
\mu_i = t_i \exp(\alpha + \beta x_i)
\end{equation}

\item Identity vs log link function: the log link function is often used. Need to decide which one is a better model. When $x$ is binary, the form of link function does not make a difference though (simple transformation of parameter). 
\end{itemize}

Model checking: 
\begin{itemize}
\item Explorative analysis of the relation between $y_i$ and $x_i$: plot $y_i$ and $x_i$ to check if there is any correlation. Do log-transformation to see if that improves linearity: choose log-link function if $\log(y_i)$ is more linear to $x_i$. 

\item Examining the assumptions of the model: e.g. for to check if Poisson regression is a good fit, we collect all samples $Y_i$ under the same values of $x$, and see if the sample mean and the sample variance are equal. 

\item Model comparison/selection by the deviance: to select between two models of different complexity, calculate the deviance, defined as the -2 times the difference of log-likelihood. Two types of deviance: 
\begin{itemize}
	\item Null deviance: Null Deviance = 2(LL(Saturated Model) - LL(Null Model)) on df = df(Sat) - df(Null), where Saturated model uses one parameter for each observation. It should roughly follow chi-square distribution with df specified above. 
	\item Residual deviance: Residual Deviance = 2(LL(Saturated Model) - LL(Proposed Model)). 
\end{itemize}
Small deviance of the proposed model means the model fits the data relatively well. One can define the ``deviance residual'', which characterizes how good the model predicts a particular observation using the LL function. This is used in R \texttt{glm()} (standarized, thus roughly following normal distribution). 

\item Model diagnosis using residuals: similar to linear model, we can assess the model fit by $y_i - \hat{\mu_i}$ (under the normal linear model, this should follow normal distribution). For count regression, however, the variance of the residual depends on $y_i$, so we ``standardize'' the residual by dividing the standard error:
\begin{equation}
e_i = \frac{y_i - \hat{\mu_i}}{\sqrt{\hat{\Var}(y_i)}}
\end{equation}
For Poisson GLM, the Pearson residual is $e_i = (y_i - \hat{\mu}_i) / \sqrt{\hat{\mu}_i}$. The residuals follow approximately normal distribution when $\mu_i$ is large. Overdispersion under Poisson GLM can be detected if residuals tend to be larger at higher values of independent variables (or mean of $y_i$). 

\end{itemize}

Statistical inference of GLM: 
\begin{itemize}
\item Fitting GLM: the Fisher scoring algorithm, effectively Newton-Raphson algorithm for maximizing the log-likelihood function. 

\item Inference of parameter $\beta$: Wald, LRT or score tests are commonly used. Wald test is simpler, but LRT is more trustworthy. 
\end{itemize}

\item{Logistic regression} 

Reference: [Hastie, Section 4.4], [KNNL, Chapter 14], [GCSR, Chapter 16]

Logistic regression model: 
\begin{itemize}
\item Background: logit function. It is also called log-odds function: 
\begin{equation}
\text{logit}(p) = \log \frac{p}{1-p}	
\end{equation}
Logit function is the inverse of the logistic function, let $x = \text{logit}(p)$, then: 
\begin{equation}
p = \frac{1}{1 + e^{-x}}	
\end{equation}

\item Model: some explanatory variables increase the chance that the event occurs ($Y =1$); others decrease it; and yet others have no effect. Thus model the probability that the event occurs as a logistic function: 
\begin{equation}
P(Y=1|X,\beta) = \frac{1}{1 + e^{-X \beta}}	
\end{equation}
where $X \beta = \beta_0 + \sum_{j=1}^p \beta_j X_j$. Alternatively, let $\mu = P(Y=1|X,\beta)$, and $P(Y=0|X,\beta) = 1 - \mu$ (the probability that the event does not occur), the model can be equivalently defined using the logit as the link function: 
\begin{equation}
\text{logit}(\mu) = X \beta
\end{equation}

\item Intepretation of parameters: for the parameter $\beta_j$, suppose every other explanatory variable is fixed, and we increase $x_j$ by one unit, the log odds-ratio (OR) will change by $\beta_j$. In particular, when $x_j$ itself is also binary, the coefficient $\beta_j$ is the difference of log-odds of the group $x_j = 1$ vs. $x_j = 0$. 

\item Classification: To classify an object, we only need to test if $P(Y=1|X,\beta) > P(Y=0|X,\beta)$, or simply: $Y = 1$ if $X \beta> 0$, and $Y = 0$ otherwise. To apply to the multi-class case: for every two class, the log-odds follows a linear function (the coefficients depend on the class). Typically one class is chosen as the reference class. 

\item Assessing a model: to assess a fitted model, let the predicted probability of $Y = 1$ be: 
\begin{equation}
\hat{\pi} = \frac{1}{1 + \exp(-X\hat{\beta})}	
\end{equation}
Then we could test the model by comparing $\hat{\pi}_i$ and the observed $y_i$. 

\end{itemize}

Parameter estimation and inference: 
\begin{itemize}
\item Maximum-likelihood parameter estimation: let $\pi_i$ be the $P(Y=1|x_i,\beta)$, we have the log likelihood function: 
\begin{equation}
l(\beta) = \sum_{i=1}^N \left[ y_i \log \pi_i + (1 - y_i) \log (1 - \pi_i) \right]
\end{equation}
Plug in $\pi_i = P(y_i = 1|X_i, \beta)$, we have: 
\begin{equation}
l(\beta) = \sum_{i=1}^N \left[ y_i X_i \beta - \log(1 + \exp(X_i \beta) \right]
\end{equation}
To maximize:
\begin{equation}
\frac{\partial{l(\beta)}}{\partial(\beta)} = \sum_{i=1}^N (y_i - \pi_i) X_i = 0
\end{equation}
No closed form solution. Typically use Newton's method or gradient ascent to find $\hat{\beta}$. 

\item Wald test: the distribution of the MLE can be approximated by a normal distribution, thus the significance of coefficient $\beta_j$ is estimated by a $Z$ score.  

\item Likelihood ratio test: a special case is to test if some $\beta_j = 0$. This can be formulated as a nested test where $H_A$ has one more parameter, so the standard LRT can be applied. It is not, however, as general as the Wald test. 

\item Group comparison test: this is an analogy of $F$ test. To test a feature $X_j$, test if the groups defined by the value of $X_j$ have different frequencies of the two classess. If $X_j$ is discrete, and has $K$ different values, then the test of whether $K$ groups have equal frequencies is a test of $K \times 2$ table, and $\chi^2$ test or Fisher's exact test can be applied. 

\item Testing mode of feature action: the above test only tests if all groups have equal frequencies. More generally, we may want to know exactly how a feature acts: e.g. which group has the highest risk (frequency of the positive class), whether the effect of a feature is monotonic, etc.  
\end{itemize}

Extension of the basic model: 
\begin{itemize}
\item Feature selection and model improvement: to improve the model, we can do this repeatedly: remove the least significant feature (by $Z$ score) and retrain the model, until no feature can be removed. 

\item Regularization: maximize the objective function: 
\begin{equation}
\max_{\beta_0, \beta} \left[ l(\beta_0, \beta) - \lambda \sum_{j=1}^p \vert \beta_j \rvert \right ]	
\end{equation}
where $l(\beta_0, \beta)$ is the log-likelihood function. 
\end{itemize}

\item{Introduction to nonlinear regression} 

Reference: [Gallant, Am Stat, 1975], [Seber \& Wild, Nonlinear regression], [KNNL, Chapter 13]

Non-linear regression: 
\begin{itemize}
\item Model: the response variable is a function of (multiple) predictors, where the functional form wrt. parameters is non-linear: 
\begin{equation}
Y = f(X;\theta) + \epsilon	
\end{equation}
where $\epsilon$ is error of distribution $N(0,\sigma^2)$. Given $n$ observations $(x_i, y_i), 1 \leq i \leq n$, the goal is to estimate $\theta$ and assess its significance.

\item Example: exponential regression model. Suppose the response variable is given by: 
\begin{equation}
Y_i = \gamma_0 + \gamma_1 \exp(\gamma_2 X_i) + \epsilon_i	
\end{equation}
where $\epsilon_i \sim N(0,\sigma^2)$. This models the case where the response may increase with $X$ ($\gamma_1 >0)$), but is bounded as $X$ becomes very large ($\gamma_2 < 0$).  

\item Linearization: sometimes it is possible to do variable transformation s.t. the resulting model is linear. However, the error term after transformation may no longer be normally distributed. For the example above: 
\begin{equation}
\log(Y_i - \gamma_0) = \log \gamma_1 + \gamma_2 X_i	+ \log(\epsilon_i)
\end{equation}
If we define $Y_i' = \log(Y_i - \gamma_0)$, the model is linear, however, the error term is not normal. 

\item Paramter estimation: the estimator of $\theta$ is obtained by minimizing sum of square error: 
\begin{equation}
SSE(\theta) = \sum_i [y_i - f(x_i; \theta)]^2	
\end{equation}
The minimization is generally performed by the numerical method. The estimator of the variance is given by: 
\begin{equation}
s^2 = \frac{1}{n-p} SSE(\hat{\theta}) 	
\end{equation}
\end{itemize}

Significance test of nonlinear regression: 
\begin{itemize}
\item Linear approximation: suppose $\hat{\theta}$ is MLE of $\theta$, the idea is to linearize the function $f(x;\theta)$ in the neighborhood of $\hat{\theta}$ so that we can apply the results from linear regression to obtain the distribution of the estimator of $\theta$. Suppose $\theta^*$ is the true value of the parameter, then we have, for each $x_i$: 
\begin{equation}
f(x_i;\theta^*) \approx f(x_i;\hat{\theta}) + \nabla{f(x_i, \hat{\theta})} \cdot (\theta^* - \hat{\theta})
\end{equation}
Let $z_i = y_i - f(x_i, \hat{\theta})$, then we have: 
\begin{equation}
z_i = \nabla{f(x_i, \hat{\theta})} \cdot (\theta^* - \hat{\theta}) + \epsilon_i	
\end{equation}
This is a linear regresion with response variable $z_i$ and predictors $(\theta^* - \hat{\theta})$. Apparently the estimator of $\theta^*$ is $\hat{\theta}$. 

\item Inference: Let $F(\theta)$ be the matrix $[\partial{f(x_i,\theta)}/\partial{\theta_j}]$, then following the results from linear regression, $\hat{\theta}$ follows normal distribution with mean $\theta^*$ and variance-covariance matrix $\sigma^2 (F^T F)^{-1}$. Define: 
\begin{equation}
\hat{C} = [\hat{F(\theta)}^T \hat{F(\theta)}]^{-1}	
\end{equation}
Then the confidence interval of $\theta_i$ is $\hat{\theta_i} \pm t_{0.025} \sqrt{s^2 \hat{c_{ii}}}$, where $t_{0.025}$ is the critical value for $t$ distribution of $n-p$ d.o.f. The test is applicable if the function $f(x;\theta)$ satisfies some regularity conditions: most notably, second partial derivative must be continuous. 
\end{itemize}

\end{enumerate}
%%%%%%%%%%%%%%%%%%%%%%%%%%%%%%%%%%%%%%%%%%%%%%%%%%%%%%%%%%%%
\section{Linear Mixed Model}

Reference: [McCulloch \& Searle, Generalized, linear and mixed models]

Review of ANOVA: concepts [McCulloch, 1.1-1.2]
\begin{itemize}
	\item Factors: in ANOVA, our goal is to assess the effect of explanatory variable(s). If the variable is discrete, we call it a \emph{factor}, and the values of a factor are \emph{levels}. The ANOVA problem is to compare mean across different levels. 
	
	\item With multiple factors: they could be nested or crossed. Two factors could also have interaction effect. 
	
	\item Balanced data: if we have equal number of observations/samples in each cell. 
\end{itemize}

Fixed and random effects: [McCulloch, 1.3-1.4]
\begin{itemize}
	\item Fixed effect model: e.g. study the drug effect on blood pressure, at different dosages. The response of the $j$-th subject at dosage $i$:
	\begin{equation}
	\E(y_{ij}) = \mu + \alpha_i	
	\end{equation}
	where $\alpha_i$ is the effect of the dosage $i$ - fixed effect. 
	
	\item Random effect model: e.g. study the drug effect at different clinics. The response of the $j$-th subject at the $i$-th clinic is: 
	\begin{equation}
	\E(y_{ij}) = \mu + a_i	
	\end{equation}
	where $a_i$ is the effect of the drug at the $i$-th clinic. Different clinics may be different in some ways - doctors, equipments, etc., thus the effect may vary. On the other hand, each clinic represents just one sample of the population (all putative patients), so we treat $a_i$ as random effect: $a_i \sim N(0, \sigma_a^2)$. 
	
	\item Variance component: under a pure random effect model, we have $\Var(y_{ij}) = \sigma_a^2 + \sigma^2$, which has two components: random effects and error. 
	
	\item Choose fixed or random effect? Generally we choose random effect if we are interested in the population effect, and each group represents only one sample of the population. 
\end{itemize}

Introduction to linear mixed model: [McCulloch, 1.5-1.7]
\begin{itemize}
	\item ANOVA form: suppose for the drug treatment problem, we have both multiple clinics ($i$) and multiple dosages ($j$), the response can be written as: 
	\begin{equation}
	\E(y_{ij}) = \mu + a_i + \beta_j + c_{ij}		
	\end{equation}
	where $a_i$ is the effect of clinic (random), $\beta_j$ is the effect of dosage (fixed) and $c_{ij}$ is the interaction effect. 
	
	\item Regression form: suppose we are testing a drug on subjects, with longitudinal data. For the $i$-th patient, its $j$-th measurement of response is $y_{ij}$, and we have dosage $d_{ij}$. We are interested in the effect of the drug. Furthermore, the drug effect might depend on age, so we control for age well. The age at the $j$-th time point is $x_{ij}$. Our model is: 
	\begin{equation}
	\E(y_{ij}) = a_i + b_i d_{ij} + \gamma x_{ij}	
	\end{equation}
	where $a_i$ is the base-level of the $i$-th subject (random), $b_i$ is the response of the $i$-th subject (again random) and $\gamma$ is the age effect (fixed). We could write $b_i = \beta + b_i'$, then $\beta$ is the effect we want to estimate, and $b_i'$ is the individual variation. 
	
	\item Inference: generally, we will need to marginalize the random effects. To do that, we consider the marginal distribution of main random variables, typically, considering their variance or covariance. 
	\begin{itemize}
		\item REML (restricted ML): we remove all the fixed effects, and do ML estimation. 
		\item Quasi-likelihood: use mean and variance instead of the full distribution to estimate parameters. 
	\end{itemize}
\end{itemize}

Normal random effect model: [McCulloch, 2.2]
\begin{itemize}
	\item Model: let $\mu_i$ be the mean of the $i$-th group ($1 \leq i \leq m$, it is $\mu + a_i$, where $a_i$ is the random effect of the $i$-th group
	\begin{equation}
	y_{ij}|a_i \sim N(\mu + a_i, \sigma^2)
	\end{equation}
	where $1 \leq j \leq n$. And
	\begin{equation}
	a_i \sim N(0, \sigma_a^2)
	\end{equation}
	The ANOVO $H_0$: $\sigma_a^2 = 0$. 
	
	\item Variance components: the idea of inference is that the variance and covariance encode information of $\sigma^2$ and $\sigma_a^2$: 
	\begin{equation}
	\Var(y_{ij}) = \E_{a_i}\left[ \Var(y_{ij}|a_i)\right] + \Var_{a_i} \left[ \E(y_{ij}|a_i)\right]	 = \E_{a_i}(\sigma^2) + \Var_{a_i}(\mu+a_i) = \sigma^2 + \sigma_a^2 
	\end{equation}
	\begin{equation}
	\begin{array}{lll}
	\Cov(y_{ij}, y_{ik}) & = & \E_{a_i}\left[ \Cov(y_{ij}, y_{ik}|a_i)\right]	+ \Cov_{a_i} \left[ \E(y_{ij}|a_i), \E(y_{ik}|a_i)\right] \\
	& = & \E_{a_i}(0) + \Cov_{a_i}(\mu+a_i, \mu+a_i) \\
	& = & \sigma_a^2
	\end{array}
	\end{equation}
	Intuitively, the covariance of two samples within a group is not 0 because they share the same random effect in that group. 
	
	\item Likelihood: within the $i$-th group, the covariance matrix is given above, written in the matrix form: $V_i = \sigma^2 I_n + \sigma_a^2 J_n$ where $I_n$ is identity matrix and $J_n$ is $n \times n$ matrix of 1's. Thus the vector $\mathbf{y_i}$ follows MVN: 
	\begin{equation}
	\mathbf{y_i} \sim N(\mu \mathbf{1}_n, V_i)	
	\end{equation}
	When the data is balanced, the MLE is: $\hat{\mu} = \bar{y}$, 
	\begin{equation}
	\hat{\sigma}^2 = \text{MSE} = \frac{1}{m (n-1)} \sum_{i,j} (y_{ij} - \bar{y})^2 	
	\end{equation}
	\begin{equation}
	\hat{\lambda} = \hat{\sigma}^2 + n\hat{\sigma}_a^2 = \frac{1}{n} \text{SSA} = \frac{1}{m} \sum_{i=1}^m n (\bar{y}_i - \bar{y})^2
	\end{equation}
	Note that it's possible that $\hat{\sigma}_a^2 < 0$ in practice. 
	
\end{itemize}

Random intercept model: [McCulloch, 3.5]
\begin{itemize}
	\item Model: the $j$-th subject ($1 \leq j \leq n$) of the $i$-th group ($1 \leq i \leq m$) follows: 
	\begin{equation}
	y_{ij} | a_i \sim N(\mu + a_i + \beta x_{j}, \sigma^2)	
	\end{equation}
	Note that we assume $x_{ij}$ only depends on $j$, thus droping $x$. And $a_i \sim N(0, \sigma_a^2)$. The distribution can be written in the matrix form. First the vector of response variable in group $i$:  
	\begin{equation}
	\E(\mathbf{y_i}|a_i) = [\mathbf{1} \mathbf{x}] [\mu \beta]^T + \mathbf{1} a_i	
	\end{equation}
	where $x = [x_1, \cdots, x_n]^T$. And $\Var(\mathbf{y_i}) = \sigma^2 I_n + \sigma_a^2 J_n = V_0$. From this we can obtain the full matrix form of all $y_{ij}$'s using Kronecker production (see the text). 
	
	\item MLE: $\hat{\mu} = \bar{y} - \hat{\beta} \bar{x}$, and 
	\begin{equation}
	\hat{\beta} = \frac{S_{xy}}{S_{xx}} = \frac{\sum_j x_j \bar{y}_{\cdot j} - n \bar{x} \bar{y}}{ \sum_j x_j^2 - n \bar{x}^2}	
	\end{equation}
	\begin{equation}
	\hat{\sigma}^2 = \frac{SSR}{m (n-1)}	
	\end{equation}
	where SSR is the residual sum of square. Note that $\hat{\mu}$ and $\hat{\beta}$ do not depend on the unknown variance $\sigma_2$ and $\sigma_a^2$, and are exactly the same when $a_i$ effects are fixed or random or no effect. 
	
\end{itemize}

Introduction to Linear Mixed Model (LMM)
\begin{itemize}
	\item Ref: Chapter 2. [West and Welch, Linear Mixed Models, a practical guide, 2007]. \url{https://stats.idre.ucla.edu/other/mult-pkg/introduction-to-linear-mixed-models/}.  \url{http://www2.stat.duke.edu/~sayan/Sta613/2018/lec/LMM.pdf}
	
	\item Motivation: dependency in the data. Ex. (1) Grouping structure: patient data from multiple doctors. The patients from the same data are expected to share something in common. (2) Repeated measurements of the same subjects. (3) Longitudinal data: the same individual is measured multiple times. 
	
	\item Fixed and Random factors: fixed factors are given groups, e.g. by gender. Random factors are random sample of groups, e.g. patients from a doctor. Fixed and random effects: coefficients of a covariates that are the same for all samples; or associated with the levels of a random factor.
	
	\item Note: in the grouping structure case, the sharing of individuals within a group can be modeled in different ways: they could have the same mean (which is different from population mean); or the effect of a certain covariate is the same in a group, but vary across groups. These lead to random intercept or random slope models. 
	
	\item About multiple random effects: even if we model only random intercept, it is possible to have multiple random effects. Basically, a sample may come from different ways of grouping: e.g. in patient study, doctor is one factor, and study cohort (multiple studies) could be another factor. 
		
	\item LMM with a single random effect as random intercept: we have $N$ samples, and $p$ covariates (fixed effects). In addition, we have $m$ groups, with $n_i$ the size of group $i$. Note: $\sum_i n_i = N$. For each group $i$, its group mean may deviate from population mean, we have:
	\begin{equation}
	y_i = X_i \beta + Z_i u_i + \epsilon_i
	\end{equation}
	where $y_i$ is $n_i \times 1$ and $X_i$ is $n_i \times p$. $Z_i$ here is simply a $n_i$-dim. vector of 1 (we only look at samples in group $j$) and $u_i$ is a scalar (mean of group $i$). Across all groups, we use $Z$ to denote the $N \times m$ ``design matrix'', with $Z_{ij} = 1$ if sample $j$ belongs to the group $i$ and 0 o/w. Now we have
	\begin{equation}
	y = X \beta + Z u + \epsilon
	\end{equation}
	where $u$ is $m \times 1$ vector. Generally, we do not estimate $u$, but rather treat $u$ as random: $u_i \sim N(0, G)$. For random intercept model, $u_i \sim N(0, \sigma^2)$. More generally, we write $G = G(\theta)$, where $\theta$ is the parameter of $G$. Often we use $\epsilon \sim R$, and $R = \sigma_e^2 I$. Our problem is to infer $\beta, \theta, R$ given $y$ and $X$, $Z$. 
	
	\item General LMM: we could have $q$ random effects. Let $u_i$ be the vector of $q$ random effects, with $i$ group index, $1 \leq i \leq m$. Our model for group $i$ is:
	\begin{equation}
	y_i = X_i \beta + Z_i u_i + \epsilon_i
	\end{equation}
	where $Z_i$ is $n_i \times q$ design matrix, $u_i \sim N(0,D)$ is $q$ dim. random vector, and $D$ is the covariance matrix of $q$ random effects. We have $\epsilon_i \sim N(0, R_i)$ is $n_i$-dim. random vector, and $R_i$ captures across-sample correlation within group $i$. We can combine the model of all $m$ factors into a single model. Let $Z$ be the block diagonal matrix:
	\begin{equation}
	Z = \left(
	\begin{array}{llll}
	Z_1 & 0 & \cdots & 0 \\
	0 & Z_2 & \cdots & 0 \\
	& \cdots & & \\
	0 & \cdots & 0 & Z_m \\
	\end{array}
	\right)
	\end{equation}
	Now $Z$ is $N \times mq$ dimension. $u$ is now concatenation of all $u_i$ vectors, so it is $mq \times 1$ vector of $q$ random effects in $m$ groups. And $\epsilon$ is $N$-dim. random vector. We can define the prior distribution of $u$ and $\epsilon$ as: $u \sim N(0, G)$ and $\epsilon \sim N(0, R)$, where 
	\begin{equation}
	G = \left(
	\begin{array}{llll}
	D & 0 & \cdots & 0 \\
	0 & D & \cdots & 0 \\
	& \cdots & & \\
	0 & \cdots & 0 & D \\
	\end{array}
	\right)	
	\qquad 
	R = \left(
	\begin{array}{llll}
	R_1 & 0 & \cdots & 0 \\
	0 & R_2 & \cdots & 0 \\
	& \cdots & & \\
	0 & \cdots & 0 & R_m \\
	\end{array}
	\right)	
	\end{equation}
		
	\item Marginal linear model: to make inference, we can marginalize $u_i$ (consider single group for now). We define $\epsilon_i^* = Z_i u_i + \epsilon_i \sim N(0, V_i)$, where 
	\begin{equation}
	V_i = Z_i D Z_i^T + R_i
	\end{equation}
	Assuming $D$ and $R$ are given, we can obtain log-likelihood as function of $\beta$. More generally, let $\theta$ be the parameters of $D(\theta)$ and $R_i(\theta)$, we can compute:
	\begin{equation}
	l_i(\beta, \theta) = P(y_i | X_i, Z_i, \beta, \theta)
	\end{equation}
	We can also collapse all the $m$ groups, and have a single model with $G$ and $R$ are covariance matrix. 
	
	\item ML and REML estimation of $\theta$ : for a given $\theta$, we can do MLE of $\beta$, $\hat{\beta}(\theta)$, we can then plug in this estimate in the log-likelihood of $\theta$. This estimator is biased, because it does not take into account the uncertainty of $\hat{\beta}$. REML is preferred: it marginalizes $\beta$ using uniform prior. It maximizes this objective function:
	\begin{equation}
	l(\theta) = \ln \int L(\beta, \theta) d\beta
	\end{equation}
	
	\item Hypothesis testing: suppose we test parameter $\hat{\beta}$. If we use the variance of the estimator $V$, where $\theta$ (random effect parameters) are fixed from REML estimation, we ignore the uncertainty of $\hat{\theta}$, this leads to underestimated error and inflation of type I error. 
	
	\item Estimation of fixed and random effects $\beta$ and $u$: Once $\theta$ is estimated, we can treat as known and estimate $\beta$, this is Best Linear Unbiased Estimator (BLUE) - Equation (2.19) of the LMM book. We can also estimate $u$, assuming $G$ and $R$ are given. The result is Best Linear Unbiased Predictor (BLUP). The result is given by:
	\begin{equation}
	\hat{u} = G Z^T V^{-1} (Y - X \hat{\beta})
	\end{equation}
	Intuitively, this is a linear combination of the residual terms, with weights determined by $Z$ and other terms. Assuming a simple model where there is a single random effect and $m$ groups, the BLUP of $u$ of a group is given by the group mean, but also the prior. 
	
	\item Correlation of BLUP with errors: see `` Residual analysis for linear mixed models'' \url{https://www.ime.usp.br/~jmsinger/MAE0610/Mixedmodelresiduals.pdf}.  The BLUP of the random effect part is a linear function of the errors. Intuition: suppose we have a model with a single random factor with many levels/subgroups. Suppose in one subgroup, by chance (errors), the group mean is somewhat high, we will learn a higher group mean in this group - we attribute larger errors to large random effects. This leads to correlation of predicted random effects and errors. To address this problem, use cross-validated prediction (use all data except one to train the parameters). 
\end{itemize}

Linear mixed model [GCTA paper, personal notes]
\begin{itemize}
\item Model: let $X$ be covariates (fixed effect) and $G$ be genotypes ($p$ variants). Let $\beta$ be the fixed effect parameters, and $\gamma_j$ be the effect size of variable $G_j, 1 \leq j \leq p$. In other words: we have $p$ random factors - each individual is assigned randomly to one of two (or three) groups for each factor. The effect size for a variant is the difference of the mean of the two groups. We have $\gamma_j \sim N(0, \sigma_a^2)$. The model: 
\begin{equation}
Y = X \beta + G \gamma + \epsilon
\end{equation}
where $\epsilon \sim N(0, \sigma_e^2)$. 

\item Remark: in this setting, we have a large number of factors, with each factor only 2 or 3 levels. We do not learn a different random effect for each factor (SNP), rather, we assume that the random effects are shared across all factors. 
%\item Equivalent model: we define $u = Z \gamma = \sum_j Z_j \gamma_j$. Then given two samples, covariance of $u_i$ and $u_k$: 
%\begin{equation}
%\Cov(u_i, u_k) =  \Cov(Z_i \gamma, Z_k \gamma) = \sum_j \Cov(Z_{ij} \gamma_j, Z_{kj} \gamma_j) = \sum_j Z_{ij}  Z_{kj} \Cov(\gamma_j, \gamma_j) = Z_i Z_k^T \tau^2
%\end{equation}
%Let $A = Z Z^T$ be the similarity matrix (GRM in GCTA). The equivalent model is: 
%\begin{equation}
%Y = X\beta + u + \epsilon
%\end{equation}
%where $u \sim N(0, A \tau^2)$. A simpler proof is: $\gamma \sim N(0, \tau^2 I)$, then $Z \gamma \sim N(0, \tau^2 Z Z^T)$.  

\item Inference: the marginal model is given by $Y = X \beta + \epsilon^*$, where $\epsilon^* \sim N(0, V)$, where  
\begin{equation}
V = G \sigma_a^2 I G^T + \sigma_e^2 I = \sigma_a^2 G G^T + \sigma_e^2 I
\end{equation}
where $G G^T$ is the GRM (up to a constant). We can then solve $\sigma_a^2, \sigma_e^2$ by REML. Let $\hat{V} = \hat{\sigma_a}^2 G G^T + \hat{\sigma}_e^2 I$ be the estimated covariance matrix, the BLUP of random effect component is given by:
\begin{equation}
y_{\text{BLUP}} = \sigma_a^2 G G^T \hat{V}^{-1} (y - X \hat{\beta})
\end{equation}
If we use $\sigma_g^2 = M \sigma_a^2$ as the heritability, where $M$ is the number of variants, we can write it in terms of GRM $K = G G^T / M$:
\begin{equation}
y_{\text{BLUP}} = \sigma_g^2 K \hat{V}^{-1} (y - X \hat{\beta})
\end{equation}
Ref: GBAT paper [Xuanyao Liu]. 

\item Alternative view: viewing random effects as correlated error terms. We can also write $u = G \gamma$ as the total contribution of genetic background (over all SNPs)
\begin{equation}
Y = X \beta + u + \epsilon
\end{equation}
where $u \sim N(0, \sigma_a^2 A)$ is given by the GRM. Under this view, the genetic random effect can be thought of an error term that correlates across samples, with correlation given by the GRM. 

\end{itemize}

Variance component model with score test (SKAT): 
\begin{itemize}
\item Model: we have $y = \mu + X\beta + \epsilon$, and the prior $\beta_j \sim N(0, \tau)$, where $\mu$ is the intercept, and assume know. Our goal is to test $H_0: \tau=0$. We test this using the score test. We write the model in matrix form: 
\begin{equation}
y|\beta \sim N(X\beta + \mu, \sigma^2 I) \qquad \beta |\tau \sim N(0, \tau I)
\end{equation}
where $I$ Is the identity matrix. Using the property of MVN, Equation~\ref{eq:MVN_marginal}, the marginal of $y$ is: 
\begin{equation}
y | \tau \sim N(\mu, \sigma^2 I + \tau X X^T)
\end{equation}
The covariance is exactly the same equation we had before for LMM: $XX^T$ is the kinship matrix, and $\tau$ corresponds to heritability. 

\item Score test: the log-likelihood function is
\begin{equation}
l(\tau) = -\frac{1}{2} (y - \mu)^T [\sigma^2 I + \tau XX^T]^{-1} (y-\mu)
\end{equation}
The score is: 
\begin{equation}
S(\tau) = -\frac{1}{2} (y - \mu)^T [\sigma^2 I + \tau XX^T]^{-2} (XX^T) (y-\mu)
\end{equation}
At $\tau = 0$, we have: 
\begin{equation}
S = -\frac{1}{2 \sigma^4} (y-\mu)^T XX^T (y-\mu)
\end{equation}

\end{itemize}

%%%%%%%%%%%%%%%%%%%%%%%%%%%%%%%%%%%%%%%%%%%%%%%%%%%%%%%%%%%%
\section{Bayesian Linear Regression}
 
Reference: [Gelman04, chapter 14; Bishop, 3.3]

Bayesian simple linear regression: posterior and BF [personal notes; SuSiE paper]
\begin{itemize}
	\item Model: $y = x \beta + \epsilon$, where $\beta \sim N(0, \sigma_0^2)$ and $\epsilon \sim N(0, \sigma^2)$. The MLE and s.e. are given by:
	\begin{equation}
	\hat{\beta} = (x^T y)^{-1} x^T y \qquad s^2 = \frac{\sigma^2}{x^T x}
	\end{equation}
	The $Z$-score is given by: $Z = \hat{\beta} / s$.
	
	\item BF: Wakefield formula:
	\begin{equation}
	B = \sqrt{1- r} \cdot \exp\left(\frac{Z^2 r}{2}\right) \text{ where } r = \frac{\sigma_0^2}{\sigma_0^2 + s^2}
	\end{equation}
	
	\item Posterior distribution: $\beta | y \sim N(\mu_1, \sigma_1^2)$, where
	\begin{equation}
	\sigma_1^2 = \frac{1}{1/ s^2 + 1 /\sigma_0^2} \qquad \mu_1 = \hat{\beta} \cdot \frac{\sigma_1^2}{s^2}
	\end{equation}
	
	\item Alternative calculation of BF in terms of posterior parameter:
	\begin{equation}
	B = \sqrt{\frac{\sigma_1^2}{\sigma_0^2}} \cdot \exp \left( \frac{\mu_1^2}{2 \sigma_1^2}\right)
	\end{equation}
	Proof: we first consider the exponential part, using $\hat{\beta} = Z s$: 
	\begin{equation}
	\mu_1 = Z s \cdot \frac{\sigma_1^2}{s^2} = Z \cdot \frac{\sigma_1^2}{s} \Rightarrow \frac{\mu_1^2}{2 \sigma_1^2} = Z^2 \frac{\sigma_1^4}{s^2 \sigma_1^2} = Z^2 \frac{\sigma_1^2}{s^2} = Z^2 r
	\end{equation}
	And the proof of the quadratic part is simple.
\end{itemize}

Ordinary linear regression with noninformative prior:  
\begin{itemize}
\item Model: let $X$ be the data matrix, $y$ be the response variables, $\beta$ be the parameters (intercept and effects), and $\sigma^2$ be the sampling variance. The model: 
\begin{equation}
y|\beta, \sigma^2, X \sim N(X\beta, \sigma^2 I)	
\end{equation}
where $I$ is the $n \times n$ identity matrix. The prior is assumed to be uniform on $\beta$ and $\log(\sigma)$: 
\begin{equation}
p(\beta, \sigma^2) \propto \sigma^{-2}	
\end{equation}
Note that log. has the effect of ``compression'': for instance, if $\sigma$ belongs to 1 to 100, at the original scale, it only has 10\% of being less than 10, but at the log10 scale, 50\%. 

\item Likelihood function: 
\begin{equation}
p(y|\beta, \sigma^2, X) = \left( \frac{1}{\sqrt{2\pi} \sigma}\right)^n	\exp \left[ -\frac{1}{2 \sigma^2} \sum_{i=1}^n (y_i - X_i \beta)^2\right]
\end{equation}

\item Posterior distribution: plug in the prior, we have: 
\begin{equation}
p(\beta, \sigma^2|y) \propto \left(\frac{1}{\sigma^2}\right)^{\frac{n}{2} + 1}	\exp \left[ -\frac{1}{2 \sigma^2} \sum_{i=1}^n (y_i - X_i \beta)^2\right]
\end{equation}
We write the exponential in quadratic form of $\beta$: 
\begin{equation}
(y - X\beta)^T (y - X \beta) = \beta^T X^T X \beta - 2 \beta^T X^T y + y^T y	
\end{equation}
Apply the quadratic form of $x^T A x + x^T b + c$, we have: 
\begin{equation}
(y - X\beta)^T (y - X \beta) = (\beta - \hat{\beta})^T V_{\beta}^{-1} (\beta - \hat{\beta}) + C
\end{equation}
where $C$ is some constant and
\begin{equation}
\hat{\beta} = (X^T X)^{-1} X^T y	\qquad V_{\beta} = (X^T X)^{-1}
\end{equation}
To see what $C$ is (important for the distribution of $\sigma^2$), we write the quadratic form in a different way: 
\begin{equation}
(y - X\beta)^T (y - X \beta) = [y - X\hat{\beta} - X (\beta - \hat{\beta})]^T [y - X\hat{\beta} - X (\beta - \hat{\beta})]
\end{equation}
Expand this, and we can show that: 
\begin{equation}
(y - X\beta)^T (y - X \beta) = (\beta - \hat{\beta})^T X^T X (\beta - \hat{\beta}) + (y - X\hat{\beta})^T (y - X\hat{\beta})
\end{equation}
Let $s^2$ be the mean squared error: 
\begin{equation}
s^2 = \frac{1}{n-p} (y - X\hat{\beta})^T (y - X\hat{\beta})
\end{equation}
Then the posterior distribution can be written as: 
\begin{equation}
p(\beta, \sigma^2|y) \propto \left(\frac{1}{\sigma^2}\right)^{\frac{n}{2} + 1} \exp \left[ -\frac{1}{2 \sigma^2} (\beta - \hat{\beta})^T X^T X (\beta - \hat{\beta})\right]	\exp \left[ -\frac{(N-p) s^2}{2 \sigma^2}\right]
\end{equation}

\item Conditional posterior distribution of $\beta$ given $\sigma^2$: follows normal distribution:
\begin{equation}
\beta|\sigma^2,y \sim N(\hat{\beta}, V_{\beta} \sigma^2)	
\end{equation}
Interpretation: $\beta$ has mean $\hat{\beta}$, the least-square estimate of $\beta$, and covariance matrix $(X^T X)^{-1} \sigma^2$, which is the covariance matrix of the least square estimator under classical statistics. $\sigma^2$ has mean $s^2$, the MLE of $\sigma^2$, and dof. equal to $N - p$. 

\item We could understand/prove the results using relationship between posterior and estimator distribution (see notes in the Introduction of Bayesian). Suppose $\sigma^2$ is given, the MLE of $\beta$ has the distribution: 
\begin{equation}
\hat{\beta} | \beta \sim N(\beta, (X^TX)^{-1}\sigma^2)
\end{equation} 
Treat $\hat{\beta}$ as given, $\beta$ follows the distribution $N(\hat{\beta}, (X^TX)^{-1}\sigma^2)$ if $P(\beta)$ is uniform. 

\item Marginal posterior distribution of $\sigma^2|y$: we integrate out $\beta$ in the posterior distribution. Using Multivariate Gaussian integral: 
\begin{equation}
\int_{\infty}^{\infty} \exp \left[ -\frac{1}{2 \sigma^2} (\beta - \hat{\beta})^T X^T X (\beta - \hat{\beta})\right] d\beta = \frac{(2 \pi \sigma^2)^p}{|X^T X|^{1/2}} 
\end{equation}
Thus: 
\begin{equation}
p(\sigma^2|y) \propto \left(\frac{1}{\sigma^2}\right)^{\frac{n-p}{2} + 1} \exp \left[ -\frac{(n-p) s^2}{2 \sigma^2}\right]
\end{equation}
This is the scaled inverse-$\chi^2$ distribution: 
\begin{equation}
\sigma^2 |y \sim \text{Inv}-\chi^2(n-p, s^2)	
\end{equation}
  
\item Proper posterior distribution: for posterior to be proper (finite integral), two conditions must be satisfied: (1) $N > p$; (2) the rank of $X$ equals $p$, i.e. the variables (columns) of $X$ must be linearly independent.  
 
\item Sampling from posterior distribution: 
\begin{enumerate}
	\item Computation of $\hat{\beta}$, $V_{\beta}$ and $s^2$. 
	\item Draw $\sigma^2$ from the inverse-$\chi^2$ distribution. 
	\item Draw $\beta$ from the multivariate normal distribution, this is often done by first sampling independent standard normal then using the Cholesky decomposition.
\end{enumerate} 
Note that computation of $\hat{\beta}$ and $V_{\beta}$ involves the inverse of $X^T X$. This is typically done via QR factorization. See the book.  
 
\item Posterior predictive distribution: for a given $\tilde{x}$, given $\sigma^2$, we have: 
\begin{equation}
E(\tilde{y}|\sigma^2,y,\tilde{x}) = E(E(\tilde{y}|\beta, \sigma^2,y,\tilde{x})|\sigma^2,y,\tilde{x}) = E(\tilde{x}\beta|\sigma^2,y,\tilde{x}) = \tilde{x} \hat{\beta}
\end{equation}
And variance: 
\begin{equation}
\text{Var}(\tilde{y}|\sigma^2,y,\tilde{x}) = (I + \tilde{x} V_{\beta} \tilde{x}^T) \sigma^2	
\end{equation}
The posterior predictive distribution has mean $\tilde{x} \hat{\beta}$, and variance consisting of sample variance $\sigma^2$, and the uncertainty due to $\beta$. 

\item Example: estimate incumbency advantage in congressional election. Let $y_i$ be the proportion of votes, and $R_i$, the main variable of interest, be whether incumbent or not. Also control for vote proportion in the previous election and the incumbent party. The analysis: 
\begin{itemize}
	\item Data transformation: this is not needed here (typically we should make $y$ normally distributed). 
	\item Posterior inference: obtain the posterior intervals (quantiles) of the variables and offset. See how this changes with election years. 
	\item Model checking: (1) Plot standardized residual, $(y_i - X \hat{\beta}_i) / s$, it should be standard normal. Check if there are outliers. (2) Predictive distribution: use existing data, we sample $y_i$ from $X_i$, and obtain the residuals, and compare the proportion of high residuals vs. observed proportion. 
\end{itemize}
\end{itemize}

Weighted linear regression: 
\begin{itemize}
\item Model: a slight generalization of the ordinary linear regression model is the errors with unequal variance (but still independent). Suppose the error of the $i$-th observation has variance $\sigma_i^2 = \sigma^2 / w_i$. Then the likelihood function is: 
\begin{equation}
p(y|\beta, \sigma^2, X) \propto \sigma^{-n}	\exp \left[ -\frac{1}{2 \sigma^2} \sum_{i=1}^n w_i (y_i - X_i \beta)^2\right]
\end{equation}

\item Weighted least square: the MLE is thus minimization of least square, with weights $w_i$. Note that if we multiply each row $X_i$ and $y_i$ by $\sqrt{w_i}$, then it becomes the standard least square. So equivalently, we could say each observation has weight $\sqrt{w_i}$, with observations with low errors higher weights. 

\item Bayesian inference: we perform variable transformation $y_i' = \sqrt{w_i} y_i$ and $X_i' = \sqrt{w_i} X_i$. In matrix form, this is: 
\begin{equation}
X' = W^{1/2}	X \qquad y' = W^{1/2} y
\end{equation}
where $W = \text{diag}(w_1, \cdots, w_n)$. Then in terms of $X'$ and $y'$, it is the ordinary linear regression. The posterior distribution of $\beta$ in terms of the orignial $X$ and $y$: 
\begin{equation}
\hat{\beta} = (X^T W X)^{-1} X^T W y	\qquad V_{\beta} = (X^T W X)^{-1}	
\end{equation}

\end{itemize}

Linear regression with known covariance matrix: 
\begin{itemize}
\item Model: the covariance matrix is given $\Sigma_y$, and we have: 
\begin{equation}
y \sim N(X \beta, \Sigma_y)	
\end{equation}

\item Posterior distribution: following our approach of weighted linear regression, we use variable transformation s.t. the regression becomes ordinary. Let $y' = \Sigma_y^{-1/2} y$, we have (linear map of normal RV): 
\begin{equation}
y' = \Sigma_y^{-1/2} y \sim N(\Sigma_y^{-1/2} X \beta, \Sigma_y^{-1/2} \Sigma_y (\Sigma_y^{-1/2})^T) = N(\Sigma_y^{-1/2} X \beta, I)
\end{equation}
Thus we could define $X' = \Sigma_y^{-1/2} X$ and solve the resulting ordinary linear regression. The posterior distribution in terms of original data: 
\begin{equation}
\hat{\beta} = (X^T \Sigma_y^{-1} X)^{-1} X^T \Sigma_y^{-1} y	\qquad V_{\beta} = (X^T \Sigma_y^{-1} X)^{-1}	
\end{equation}

\end{itemize}

Linear regression with unknown covariance matrix: 
\begin{itemize}
\item Model: we assume prior $\beta|\Sigma_y$ is uniform, and the prior $p(\Sigma_y)$. The conditional posterior of $\beta$ given $\Sigma_y$ is already solved. To obtain the joint posterior sample, we need the marginal posterior distribution $p(\Sigma_y|y)$. See Equation (14.14) in the book. 

\item Remark: $\Sigma_y$ is $N \times N$ matrix, and in general, cannot be estimated (any sample point has its own parameter). Need strong informative prior or some structure of $\Sigma_y$ (e.g. diagonal, or grouping). 

\item Independent errors with the variance dependent on some (unknown) constant:
\begin{equation}
\Sigma_{ii} = \sigma^2 v(w_i, \phi)	
\end{equation}
where $v$ is a function, e.g. $v(w_i, \phi) = (1 - \phi) + \phi/w_i$, a mixture of constant variance and weighted variance. The inference problem is to find the posterior $p(\beta, \sigma^2, \phi|y)$. 

\item Group-specific errors: $n$ observations form $I$ groups, with all observations in one group the same error $\sigma_i^2, i = 1, \cdots, I$. The inference problem is to sample from the posterior $p(\beta, \sigma_1^2, \cdots, \sigma_I^2|y)$. Suppose we use the noninformative prior: $p(\beta, \Sigma_y) \propto \prod_i \sigma_i^{-2}$.
\begin{itemize}
\item The complete posterior distribution: 
\begin{equation}
p(\beta, \sigma_1^2, \cdots, \sigma_I^2|y) \propto \left( \prod_i \sigma_i^{-n_i -2}\right) \exp\left[ -\frac{1}{2}(y-X\beta)^T \Sigma_y^{-1} (y-X\beta)\right]
\end{equation}

\item Posterior mode of $p(\sigma_1^2, \cdots, \sigma_I^2|y)$: using the EM algorithm. Note that in the log-posterior distribution, for the $E$-step, we only need to consider the term that depends on $\beta$ (missing parameters), thus $E$-step involves computing: 
\begin{equation}
\E_{\text{old}}\left[ (y-X\beta)^T \Sigma_y^{-1} (y - X\beta)\right]	
\end{equation}
averaging over $\beta|\Sigma_y^{\text{old}},y$. We could solve this posterior of $\beta$ using weighted linear regression. Then evaluating the expectation above is equivalent to evaluting the expectation of quadratic forms of MVN random variables. 

\item Gibbs sampling: the distribution $p(\beta|\Sigma_y,y)$ is simply weighted linear regression. The distribution $p(\sigma_i^2|\beta,y)$ ($\sigma^i$ are independent) is also simple: scaled inverse-$\chi^2$ distribution from the Bayesian inference of normal distribution (with known mean). 

\end{itemize}
\end{itemize}

Linear regression with conjugate prior: [Banerjee, Bayesian linear models: gory details; Bishop; GCSR 14.8] 
\begin{itemize}
\item Prior distribution: use the conjugate prior, where $\sigma^2$ follows inverse-Gamma distribution: 
\begin{equation}
p(\sigma^2) = IG(\sigma^2|a_0, b_0) \propto \left(\frac{1}{\sigma^2}\right)^{a_0+1} \exp\left( - \frac{b_0}{\sigma^2}\right)
\end{equation}
\begin{equation}
p(\beta|\sigma^2) = N(\beta|\beta_0,\sigma^2 V_0)	
\end{equation}
We call the joint distribution as Normal-Inverse-Gamma (NIG) distribution: 
\begin{equation}
p(\beta,\sigma) = NIG(\beta_0,V_0, a_0, b_0) \propto \left(\frac{1}{\sigma^2}\right)^{a_0+\frac{p}{2}+1} \exp\left( - \frac{b_0}{\sigma^2}\right) \exp\left[ - \frac{1}{2\sigma^2} (\beta-\beta_0)^T V_0^{-1}(\beta-\beta_0)\right]
\end{equation}

\item Lkelihood function: this is the same as before: 
\begin{equation}
p(y|\beta, \sigma^2) = \left( \frac{1}{\sqrt{2\pi} \sigma}\right)^n	\exp \left[ -\frac{1}{2 \sigma^2} (y - X\beta)^T (y-X\beta)\right]	
\end{equation}

\item Conditional posterior distribution $\beta|\sigma^2,y$: we first obtain this distribution in order to get the joint posterior. Both $\beta$ and $y|\beta$ follow normal distribution under given $\sigma^2$: 
\begin{equation}
\beta|\sigma^2 \sim N(\beta_0,\sigma^2 V_0)	
\end{equation}
\begin{equation}
y|\beta,\sigma^2 \sim N(X\beta, \sigma^2 I)	
\end{equation}
Using the properties of MVN, we have: 
\begin{equation}
\beta|\sigma^2,y \sim N(\hat{\beta}_n, \sigma^2 V_n)	
\end{equation}
where
\begin{equation}
\hat{\beta}_n = V_N (V_0^{-1} \beta_0 + X^T y)	
\end{equation}
\begin{equation}
V_N = (V_0^{-1} + X^T X)^{-1}
\end{equation}

\item Posterior distribution $\sigma^2|y$: we first obtain the distribution $y|\sigma^2$. From the property of MVN: 
\begin{equation}
y|\sigma^2 \sim N(X\beta_0, \sigma^2(I + XV_0 X^T))	
\end{equation}
We have the posterior: 
\begin{equation}
p(\sigma^2|y) \propto \left(\frac{1}{\sigma^2}\right)^{a_0 + \frac{n}{2} +1} \frac{1}{|I + X V_0 X^T|^{1/2}} \exp \left\{ -\frac{1}{\sigma^2}\left[ b_0 + \frac{1}{2}(y-X\beta_0)^T (I + X V_0 X^T)^{-1}(y-X\beta_0)\right]\right\}	
\end{equation}
Ignoring the constant term, we recognize that this is Inverse-Gamma distribution $IG(a_n, b_n)$, where
\begin{equation}
a_n = a_0 + n \qquad b_n = b_0 + \frac{1}{2}(y-X\beta_0)^T (I + X V_0 X^T)^{-1}(y-X\beta_0)	
\end{equation}

\item Posterior distribution: in summary, we have $\sigma^2|y$ follows inverse-gamma and $\beta|\sigma^2,y$ follows normal distribution, so the joint distribution follows $NIG(\hat{\beta}_n, V_n, a_n, b_n)$, or: 
\begin{equation}
p(\beta,\sigma^2|y) \propto \left(\frac{1}{\sigma^2}\right)^{a_n+\frac{p}{2}+1} \exp\left( - \frac{b_n}{\sigma^2}\right) \exp\left[ - \frac{1}{2\sigma^2} (\beta-\hat{\beta}_n)^T V_n^{-1}(\beta-\hat{\beta}_n)\right]
\end{equation}

\item Linear regression with conjugation prior and general error term: slightly more general form:
\begin{equation}
\beta|\sigma^2 \sim N(\beta_0,\sigma^2 V_0)	
\end{equation}
\begin{equation}
y|\beta,\sigma^2 \sim N(X\beta, \Sigma)	
\end{equation}
where $\Sigma$ is the error covariance matrix (weighted or correlated errors). The solutions are given by:
Using the properties of MVN, we have: 
\begin{equation}
\beta|\sigma^2,y \sim N(\hat{\beta}_n, V_n)	
\end{equation}
where
\begin{equation}
\hat{\beta}_n = V_N (V_0^{-1} \beta_0 + X^T \Sigma^{-1} y)	
\end{equation}
\begin{equation}
V_N = (V_0^{-1} + X^T \Sigma^{-1} X)^{-1}
\end{equation}
It is easy to see that in the extreme case (data dominates the prior), $\hat{\beta}_n$ converges to the MLE: $(X^T \Sigma^{-1} X)^{-1} X^T \Sigma^{-1} y$. 
\end{itemize}

Linear regression with semi-conjugate prior: [Banerjee, Bayesian linear models: gory details; Bishop; GCSR 14.8] 
\begin{itemize}
\item Prior distribution: we have independent prior (suppose we use noninformative prior for $\sigma^2$): 
\begin{equation}
p(\sigma^2) \propto 1/\sigma^2 \qquad p(\beta) = N(\beta|\beta_0, V_0)	
\end{equation}

\item Conditional posterior distribution of $\beta|\sigma^2,y$: similar to the derivation before, we have: 
\begin{equation}
\beta | y, \sigma^2 \sim N(\hat{\beta}_n, V_N)	
\end{equation}
The mean and covariance matrix are given by: 
\begin{equation}
\hat{\beta}_n = V_n (V_0^{-1} \beta_0 + \frac{1}{\sigma^2} X^T y)	
\end{equation}
\begin{equation}
V_N^{-1} = V_0^{-1} + \frac{1}{\sigma^2} X^T X	
\end{equation}
In the special case where $\beta_0 = 0$ and $V_0 = \alpha^{-1} I$ (used below), we have: 
\begin{equation}
\hat{\beta}_N = \frac{1}{\sigma^2} V_N X^T y	
\end{equation}
\begin{equation}
V_N^{-1} = \alpha I + \frac{1}{\sigma^2} X^T X	
\end{equation}

\item Posterior distribution $\sigma^2|y$: we first obtain $y|\sigma^2$ as normal distribution:
\begin{equation}
y|\sigma^2 \sim N(X \beta_0, \sigma^2 I + X V_0 X^T)	
\end{equation}
The posterior: 
\begin{equation}
p(\sigma^2|y) \propto \sigma^{-2} \frac{1}{|\sigma^2 I + X V_0 X^T|^{1/2}} \exp \left[ -\frac{1}{2}(y-X\beta_0)^T (\sigma^2 I + X V_0 X^T)^{-1}(y-X\beta_0)\right]
\end{equation}
This is not a standard parametric distribution, but given any $\sigma^2$, it can be numerically evaluated. 

\item Alternative approach to informative prior [GSCR]: suppose we have the prior $\beta \sim N(\beta_0, \Sigma_{\beta})$. We can treat the prior distribution as $p$ prior ``data points'': for the $j$-th point, the explanatory variable is equal to the unit vector, and the response variable $\beta_{0,i}$. Thus this is reduced to the problem before with non-informative prior. 

\item MAP estimator: the log of the posterior distribution is given by: 
\begin{equation}
\ln p(\beta|y, \sigma^2) = - \frac{1}{2\sigma^2} \sum_i (y_i - x_i \beta)^2 - \frac{\alpha}{2} \beta^T \beta + \text{const}	
\end{equation}
Maximizing this function leads to ridge regression, where the second term corresponds to $L_2$ penalty of coefficients. The interpretation: the prior $\beta_0 = 0$, thus favor small parameter values (penalty), and $\alpha$ controls the strength of penalty: if $\alpha$ is small, the prior distribution is diffused, thus little penalty. 

\item Posterior predictive distribution: given $\tilde{x}$, the distribution is given by: 
\begin{equation}
\tilde{y} | \tilde{x}, y, \sigma^2 \sim N(\tilde{x} \hat{\beta}_n, \sigma^2 + \tilde{x} V_n \tilde{x}^T)	
\end{equation}
The mean follows the classical results, and the variance has two parts: the sampling variance and variance due the uncertainty of $\beta$. 

\end{itemize}

Linear regression with constraints and variable selection: 
\begin{itemize}
\item Inequality constraints [GCSR, 14.8]: e.g. $\beta \geq 0$ or $\beta_2 \leq \beta_3$. A simple way to handle constraints in is to ignore constraints in posterior sampling, and in the end, simply discard those samples that violate the constraints. 

\item Variable selection [GCSR, 15.5]: the idea is to put an informative prior on $\beta_j$, s.t. it has a significant probability of being 0. Ex. each variable is probability unimportant, but if it has an effect, it could be large, one could use a $t$ or other wide-tailed distribution for $p(\beta)$. 
\end{itemize}

Equivalent kernel: [Bishop]
\begin{itemize}
\item Alternative interpretation of posterior predictive mean: we consider the linear basis function, $\phi(x_i)$ instead of $x_i$. Then the posterior predictive mean can be written as:
\begin{equation}
\phi(\tilde{x}) \hat{\beta_N} = \frac{1}{\sigma^2} \phi(\tilde{x}) S_N \phi(X)^T y = \frac{1}{\sigma^2} \sum_{i=1}^N \phi(\tilde{x}) S_N \phi(x_i)^T y_i = \frac{1}{\sigma^2} \sum_{i=1}^N k(\tilde{x}, x_i) y_i
\end{equation}
where the equivalent kernel is defined as: 
\begin{equation}
k(x,x') = \frac{1}{\sigma^2} \phi(x) S_N \phi(x')^T	
\end{equation}
Thus the prediction is the weighted average of $y_i$ of the training data. Note that, for any point $x$, we have: 
\begin{equation}
\sum_{i=1}^N k(x,x_i) = 1	
\end{equation}
 
\item Interpretation of kernel: $k(x,x')$ is concentrated on the neighborhood of the $x'$, thus in the prediction, the points in the training data close to $\tilde{x}$ will make a large contribution than distant points. Furthermore, for posterior predictive mean of two points: $\tilde{y}(x)$ and $\tilde{y}(x')$, we have: 
\begin{equation}
\text{Cov}[\tilde{y}(x), \tilde{y}(x')] = \text{Cov}[\phi(x) \beta, \phi(x') \beta ] = \phi(x) S_N \phi(x')^T = \sigma^2 k(x,x')
\end{equation}
where $S_N$ is the covariance matrix of $\beta|y$ is used. Thus, the predictive mean at nearby points will be highly correlated. 

\item Kernel regression: the kernel interpretation of predictions leads to a general idea of regression: use a localized kernel directly and use this to make predictions for new input vector.  

\end{itemize}
%%%%%%%%%%%%%%%%%%%%%%%%%%%%%%%%%%%%%%%%%%%%%%%%%%%%%%%%%%%%
\section{Bayesian Hierarchical Linear Models}

Reference: [Gelman07, chapter 11-13; Gelman04, Chapter 15]

Motivations of multi-level regression: compromise between complete pooling and no pooling, thus allows to model group variation and let one borrow information from one group to another. 
\begin{itemize}
	\item Group structure: generally, we have group structure in the data. In regression problem: often at the sample level; but could also be at the parameter level. Different groups: share some common characteristics, but also different (unexplained by known factors). Use complete pooling: we ignore between-group variation; use no pooling, we ignore the common shared distribution. 
	
	\item Use all the data to perform inference for groups with small sample size - pooling. Caveat: the effect paramters of groups are modeled by a common distribution, based on the assumption that the groups share certain aspects (e.g. in the Radon example, the effect of uranium is constant). If this is not the case, i.e. group variation is so large, then there is no benefit. 
	
	\item Learn about treatment effects that vary with groups. If the interest lies in the effect on groups, then multi-level modeling is the natural appraoch. 
	
	\item Simple regression on all predictors (including both group-level and individal level ones) is not as good: it does not account for the additional between-group variation (not explained by the group-level predictors). In other words, different groups may differ (in their means) in some aspects not explained by the group-level explanatory variables (if not accounted for, effectively complete pooling). Thus, the missing variation from existing predictors is the important consideration when developing multi-level models. \\
	Ex. in presidential election, the election year affects the Dem. vote, in addition to other effects alreay modeled such as national economy (this additional variation may be: e.g. the events, the popularity of the candidates, etc.). 
\end{itemize}

Problems of multi-level modeling: 
\begin{itemize}
	\item Example: Radon data in housing sampled from multiple counties. Individual level data: $y_i$ is the Radon level in the $i$-th house, and $x_i$ is floor level (the individual-level predictor). Group level data: $j[i]$ is the county of the $i$-th house (group predictor), and $u_j$ is the county-level soil uranium level (the group-level predictor).  
	
	\item Common problems: 
	\begin{itemize}
		\item Problems about specific groups: e.g. what is the average Radon level of each group? 
		\item Group level problem: e.g. how uranium level affects Radon level? 
		\item Population level problem: e.g. what is the effect of the floor level on Radon (suppose this effect is independent of the county)? 
	\end{itemize}
\end{itemize}

A special case of hierarchical linear regression model: random effect model: 
\begin{itemize}
	\item A special case of the random effect model is the hierarical normal model: the mean of each group is a random effect. To model this as a regression problem, for each data point, we treat the group membership as explantory variables ($J$ groups - $J$ variables), then the $X$ is simply the $J \times J$ identity matrix. And $\beta_j$ in the linear model is the mean of the $j$-th group. 
	
	\item Model: more generally, the $j$-th coefficient: $\beta_j \sim N(\alpha, \sigma_{\beta}^2)$, or in vector form: 
	\begin{equation}
	\beta \sim N(\alpha \vec{1}, \sigma_{\beta}^2 I)	
	\end{equation}
	where $I$ is the identity matrix. 
	
	\item Relation with the MVN with intra-class correlation: suppose we have for the $i$-th group: $y_i \sim N(\beta_i, \sigma^2)$ and $\beta_i \sim N(\alpha, \sigma_{\beta}^2)$. Then we could derive the joint distribution of $y_1, \cdots, y_n$ as MVN. The variance of a sample in the $i$-th group is: 
	\begin{equation}
	\Var(y_i) = \E[\Var(y_i|\beta_i)] + \Var[\E(y_i|\beta_i)] = \sigma^2 + \Var(\beta_i) = \sigma^2 + \sigma_{\beta}^2
	\end{equation}
	The covariance of two samples in different groups is clearly 0, and the covariance of $y_{i1}$ and $y_{i2}$ from the $i$-th group is: 
	\begin{equation}
	\Cov(y_{i1},y_{i2}) = \E[\Cov(y_{i1},y_{i2}|\beta_i)] + \Cov[\E(y_{i1}|\beta_i), \E(y_{i2}|\beta_i)] = 0 + \sigma_{\beta}^2	= \sigma_{\beta}^2
	\end{equation}
	Therefore, hierarchical normal model is equivalent to a MVN with group structure. And similarly, the MVN with certain group structure can be modeled as a hierarchical normal model. 
	
\end{itemize}

Classical regression approach to multi-level data: 
\begin{itemize}
	\item Assumptions: $y_i$ of a house is determined by the floor level and the county effect (plus individual variation). The county effect is a function of the uranium level (however, the uranium level cannot explain all the effects of counties). We denote $\alpha_j$ as the county effect on the average Radon level (not observed).
	
	\item Complete pooling: pool all the samples from all groups and do regression (treating group-level predictors as individual predictors, and no group indicator). Ex. to answer the question of how uranium level affects Radon by complete pooling: 
	\begin{equation}
	y_i = \beta x_i + \gamma u_{j[i]}	+ \epsilon
	\end{equation}
	The average Radon level of each county is different (unexplained variation from county uranium level), and this additional group variation is ignored when performing individual-level regression. 
	
	\item No pooling: treat each group independently and perform regression on each group (assuming each group has its own parameter). This is equivalent to introducing group indicator as individual-level predictors: 
	\begin{equation}
	y_i = \beta x_i + \alpha_{j[i]} + \epsilon
	\end{equation}
	This approach ignores the fact that groups may be related, thus one group may carry information on another group (overestimate the group variation). 
	
	\item Two-step analysis: first do analysis on each group, and then perform group-level analysis. In this example, first estimate $\alpha_j$ for each group, then do: $\alpha_j \sim u_j$. Again, the problem is that the individual group analysis may alreay overestimate the group variation. 
	
\end{itemize}

Multi-level modeling approach with varying intercept: 
\begin{itemize}
	\item Partial pooling: pooling the data from multiple groups (s.t. information from different groups can be used), but only partially s.t. group variation is still accounted. Ex. in the case of estimating average level of each group, partial pooling amounts to a weighted average between: population-level average and group average (the weight depends on the size of the group and the population). 
	
	\item Varying intercept model: in the Radon example, we model the effect of the county (intercept) on house Radon level:  
	\begin{equation}
	y_i \sim N(\alpha_{j[i]} + \beta x_i, \sigma_y^2)
	\end{equation}
	\begin{equation}
	\alpha_j \sim N(\gamma_0 + \gamma_1 u_j, \sigma_{\alpha}^2)	
	\end{equation}
	
	\item Equivalent formulation using indicator variables: in general, the group-specific parameters can also be specified with group indicators. In the above example, the first equation can be written as: 
	\begin{equation}
	y_i \sim N(\sum_j \alpha_{j} I(j[i] = j) + \beta x_i, \sigma_y^2)	
	\end{equation}
	where $I(j[i] =j)$ is the indicator variable. 
\end{itemize}

Multi-level modeling with varying slopes and intercepts: 
\begin{itemize}
	\item Varying slope model: the group membership can affect the effect parameter (interaction between feature and group indicator). Aagain consider the Radon example, now assume that the effect of floor level (in addition to average Rado level) depends on the county, we have: 
	\begin{equation}
	y_i \sim N(\alpha_{j[i]} + \beta_{j[i]} x_i, \sigma_y^2)
	\end{equation}
	\begin{equation}
	\left( \begin{array}{l} \alpha_j\\ \beta_j \end{array} \right) \sim N\left( \left( \begin{array}{l} \gamma_0^{\alpha} + \gamma_1^{\alpha} u_j \\ \gamma_0^{\beta} + \gamma_1^{\beta} u_j \end{array} \right), \Sigma \right)
	\end{equation}
	where $\Sigma$ is the covariance matrix (see below why it is needed). 
	
	\item The varying slope model under classical regression: the varying slope effect can be captured using interaction between group indicator and invidual-level predictor. In the Radon example, we define invidual level predictor $v_i = u_{j[i]}$, and express $y_i$ in terms of only invidual-level predictors, including group indicator, and their interactions: 
	\begin{equation}
	y_i = a + b v_i + c_{j[i]} + d x_i + e v_i x_i + f_{j[i]} x_i + \epsilon_i
	\end{equation}
	
	\item Correlation between group-level intercepts and slopes: when the data points are not centered, larger $\alpha_j$ means smaller $\beta_j$ (as the line has to pass through the center). Centering the data can alleviate the problem. 
	
\end{itemize}

Relation to modeling interaction in regression: in general, interaction can be modeled with a multi-level model, however, this may be different from traditional approach. 
\begin{itemize}
	\item Tradiational model: suppose $Y$ is a function of $X_1$, and $X_1$ effect depends on $X_2$, then our regression: 
	\begin{equation}
	Y = \beta_1 X_1 + \beta_2 X_1 \cdot X_2 + \beta_0 + \epsilon	
	\end{equation}
	
	\item Multi-level model: we assume the regression:
	\begin{equation}
	Y = \beta_1 X_1 + \beta_0 + \epsilon	
	\end{equation}
	To model the dependence of $X_1$ effect on $X_2$, we assume $\beta_1 \sim N(\gamma_1 X_2 + \gamma_0, \sigma_{\beta_1}^2)$. The difference is: (1) traditional model: the dependence of $\beta_1$ is fixed on $X_2$; (2) multi-level model: the dependence on $X_2$ itself is random. Thus, the variance of $Y$ given $X_1$ and $X_2$ is constant given the traditional model, but depends on the value of $X_1$ under the multi-level model. 
	
\end{itemize}

General hierarchical regression model: 
\begin{itemize}
	\item Model: the likelihood: 
	\begin{equation}
	y|X,\beta,\Sigma_y \sim N(X \beta, \Sigma_y)	
	\end{equation}
	The population distribution: given the explanatory variables of $\beta$, denoted as $X_{\beta}$, and the coefficients $\alpha$, the distribution of $\beta$: 
	\begin{equation}
	\beta | X_{\beta}, \alpha, \Sigma_{\beta} \sim N(X_{\beta} \alpha, \Sigma_{\beta})	
	\end{equation}
	Finally, we have the hyperprior distribution of $\alpha$: 
	\begin{equation}
	\alpha | \alpha_0, \Sigma_{\alpha} \sim N(\alpha_0, \Sigma_{\alpha})	
	\end{equation}
	
	\item Equivalent to a single linear regression: by modeling the prior distribution of $\beta$ as additional ``data points'', we could treat the hierarchical regression model as a single linear regression model. See Equation (15.3) in [GCSR]. 
\end{itemize}

Inference of multi-level regression: 
\begin{itemize}
	\item MCMC: the parameters are $\beta$ (individual level regression coefficients), $\alpha$ (group level regression coefficients) and variance parameters $\Sigma_y$ and $\Sigma_{\beta}$. The update rules are: we use Gibbs sampling to sample:  
	\begin{equation}
	\forall j: \beta_j | \alpha, \Sigma_{\beta}, \Sigma_y, y
	\end{equation}
	This is regression of a single group: the prior of $\beta_j$ is determined by $\alpha$ and $\Sigma_{\beta}$. Next we use Gibbs sampling: 
	\begin{equation}
	\alpha | \beta, \Sigma_{\beta}
	\end{equation}
	This is a single regression for the higher-level parameters using $\beta$ ($\beta_j$ for each group is a data point). For the variance parameters, we can use Scaled-inverse-$\chi^2$ or MH update: 
	\begin{equation}
	\Sigma_y | y, \beta, \qquad \Sigma_{\beta} | \alpha, \beta 
	\end{equation}
	To initialize: regression with noninformative priors can be used to sample the initial values. 
	
	\item Traps in Gibbs sampling: the sampler could be slow when the group variance parameter is close to 0. Then the group parameters will all be forced close to the population mean, and in the next round, the group variance parameter will be sample close to 0 (as the group parameters are close), and so on. 
	
	\item Alternative Gibbs sampling: 
	\begin{itemize}
		\item All-at-once Gibbs sampler: treat the hierarchical model as the single-level regression model, and alternatively update regression coefficients and the variance parameters. 
		\item Scalar Gibbs sampler: update one parameter a time. In particular, for the regression coefficient, this is similar to the stepwise regression in non-Bayesian approach.  
	\end{itemize}
\end{itemize}

More complex multi-level models: 
\begin{itemize}
	\item Non-nested models: e.g. regression of earnings on ethnicity, age and height. The individuals are grouped by ethnicity and age, denoted as $j[i]$ and $k[i]$, respectively. Let $z_i$ be the height, the regression: 
	\begin{equation}
	y_i \sim N(\alpha_{j[i],k[i]} + \beta_{j[i],k[i]} z_i, \sigma_y^2)
	\end{equation}
	\begin{equation}
	\left( \begin{array}{l} \alpha_{j,k} \\ \beta_{j,k} \end{array} \right)	= \left( \begin{array}{l} \mu_0 \\ \mu_1 \end{array} \right) + \left( \begin{array}{l} \gamma_{0j}^{\text{eth}}\\ \gamma_{1j}^{\text{eth}} \end{array} \right) + \left( \begin{array}{l} \gamma_{0k}^{\text{eth}}\\ \gamma_{1k}^{\text{eth}} \end{array} \right) + \left( \begin{array}{l} \gamma_{0jk}^{\text{eth} \times \text{age}}\\ \gamma_{1jk}^{\text{eth}} \times \text{age} \end{array} \right)
	\end{equation}
	The last three terms can be modeled as normal distribution with mean 0. 
	
	\item Structure or modeling of regression predictors: 
	\begin{itemize}
		\item Regression coefficents of classical models: whether predictors are in the model can be viewed as a special case of the multi-level model. When the variance of a coefficient is 0, the predictor is out; if $\infty$, the predictor is in. 
		
		\item Grouping regression predictors: impose structures/grouping in the predictors. E.x. modeling presidential election outcome (Dem. vote), the explanatory variables include a number of economic measures: suppose they are all in the scale, we may assume $\beta_j$ are from a common distribution, thus making all $\beta_j$'s close togeter. 
		
		\item Modeling regression coefficients: e.g. regression of cancer risk on the food consumption (362 individuals with 87 foods). Each food can be characterized by the level of 35 nutrients, thus the data can be used to infer the effect of nutrient on cancer risk. Let $\beta_j$ be the effect of food $j$, it can be modeled as: 
		\begin{equation}
		\beta_j \sim N(z_j \gamma, \sigma_{\beta}^2)	
		\end{equation}
		where $z_j$ is the vector of nutrient level, and $\gamma$ is the effect of nutrients. 
	\end{itemize}
	
	\item Network structure: group memberships are not always disjoint.
	
\end{itemize}

%%%%%%%%%%%%%%%%%%%%%%%%%%%%%%%%%%%%%%%%%%%%%%%%%%%%%%%%%%%%
\section{Bayesian Generalized Linear Models}

Bayesian inference of GLM: [GCSR, Section 16.4]. 
\begin{itemize}
	\item Model: the model is specified by $g(\mu) = X \beta$, where $\mu = \E(y|X)$ and $g(\cdot)$ is the link function. Sometimes we have a dispersion parameter $\phi$ (e.g. for Negative Binomial regression).  
	
	\item General procedure: generally the posterior distribution of $\beta$ does not an analytic form, so need approximation/sampling. The general procedure is: 
	\begin{enumerate}
		\item Obtain posterior mode $(\hat{\beta}, \hat{\phi})$. 
		\item Normal approximation about the posterior mode as the starting point for simulation: $p(\beta|\hat{\phi}, y) \approx N(\beta|\hat{\beta}, V_{\beta})$ where $V_{\beta}$ is determined by the asymptotic approximation (second derivative of log-likelihood function). This is weighted regression problem. 
		\item Sample posterior by MH. 
	\end{enumerate}
	
	\item Normal approximation: let $\eta_i = X_i \beta$ be the predictor. Let $L(y_i|\eta_i, \phi)$ be the log-likelihood function. We use the second order Tayler expansion as approximation of $L$ (normal approximation): 
	\begin{equation}
	L(y_i|\eta_i, \phi) \approx -\frac{1}{2 \sigma_i^2} (z_i - \eta_i)^2 + \text{const}
	\end{equation}
	
	\item Posterior mode (iterative regression): the problem is essentially linear regression with prior equal to the estimation in the previous round. Thus Newton's method becomes iterated linear regression. 
\end{itemize}

Bayesian logistic regression [Bishop, Section 4.5]: 
\begin{itemize}
	\item Normal approximation of posterior distribution: suppose we have the logistic regression: 
	\begin{equation}
	P(Y=1|X,\beta) = \sigma(X\beta)	
	\end{equation}
	where $\sigma$ is the sigmoid function. The prior is $\beta \sim N(\beta_0, S_0)$. The posterior is given by: 
	\begin{equation}
	\log P(\beta|y) = \log P(\beta) + \log P(y|\beta) = \log N(\beta|\beta_0, S_0) + \sum_{i=1}^N \left[ y_i X_i \beta - \log(1 + \exp(X_i \beta) \right]
	\end{equation}
	Take the second derivative wrt. $\beta$, we have: 
	\begin{equation}
	- \frac{\partial^2}{\partial \beta^2} \log p(\beta|y) = S_0^{-1} + \sum_i \pi_i (1 - \pi_i) X_i^T X_i	
	\end{equation}
	where $\pi_i = P(y_i = 1|X_i, \beta)$. Using the normal approximation of posterior distribution, we have: $\beta|y \sim N(\hat{\beta}, S_n)$ where $\hat{\beta}$ is the MAP estimator of $\beta$ and $S_n$ satisfies: 
	\begin{equation}
	S_n^{-1} = S_0^{-1} + \sum_i \pi_i (1 - \pi_i) X_i^T X_i		
	\end{equation}
	
	\item Predictive distribution and model evidence: in both cases, $p(\beta)$ follows normal distribution: prior normal for model evidence and posterior normal approximation for posterior predictive distribution. Let $\beta \sim N(\mu_{\beta}, \Sigma_{\beta})$. We need to solve a problem of integration: a convolution between sigmoid and normal functions:
	\begin{equation}
	P(y=1|X) = \int \sigma(X \beta) p(\beta) d\beta
	\end{equation}
	We first show that in the integral, the dimension of variables can be reduced. The idea is to integrate over the variable $X\beta$ (but cannot directly apply Change of Variable Theorem because of the difference of dimensionality), over the region defined by $X\beta$. Let $a = X \beta$, we have (by Fubini's Theorem): 
	\begin{equation}
	\int \sigma(X \beta) p(\beta) d\beta = \int \left(\int \delta(a - X\beta) \sigma(a) da\right) p(\beta) d\beta = \int \sigma(a) p(a) da
	\end{equation}
	where 
	\begin{equation}
	p(a) = \int \delta(a - X \beta) p(\beta) d\beta	
	\end{equation}
	It can be shown that $p(a)$ is simply the multivariate normal distribution of $\beta$ subject to the linearity constraint, $X \beta = a$, and this marginal distribution is also normal with mean and variance: 
	\begin{equation}
	\mu_a = X \mu_{\beta}	
	\end{equation}
	\begin{equation}
	\sigma_a^2 = X \Sigma_{\beta}	X^T
	\end{equation}
	The integral can be approximated by using the probit function. For the sigmoid function, we have: 
	\begin{equation}
	\sigma(a) = \Phi(\lambda a)	
	\end{equation}
	where $\Phi$ is the CDF of standard normal, and $\lambda^2 = \pi/8$. The result is given by (see [Bishop]): 
	\begin{equation}
	P(y=1|X) \approx \int \sigma(a) p(a) da \approx \sigma \left( \frac{\mu_a}{\sqrt{1 + \pi \sigma_a^2 / 8}} \right)
	\end{equation}
	
\end{itemize}

%%%%%%%%%%%%%%%%%%%%%%%%%%%%%%%%%%%%%%%%%%%%%%%%%%%%%%%%%%%%
\section{Shrinkage Methods and Variable Selection}

Reference: [Hastie, Section 3.4]

Motivations for shrinkage: 
\begin{itemize}
\item Large $p$, small $N$ problem: when the number of features is large, or in the case of categorical variables, the number of categories is large, the number of data points would be small to learn regression coefficients. 

\item Parameter shrinkage: prefer models where most parameters are small or zero. 

\item Structure of predictors: impose additional structure on predictors. Most commonly: (1) group predictors s.t. each group of predictors have the same parameter, e.g. haplotype regression where haplotypes are clustered; (2) random effects of predictors of the same group. 
\end{itemize}

Feature standardization: it is often necessary to standardize features for regression procedures that penalize complex models by shrinkage of parameters. Without standardization, the parameters of different features are not comparable. Standardization consists of: 
\begin{itemize}
	\item Feature standardization: for the $j$-th feature, let $\bar{x_j}$ be its mean, and $\text{sd}(x_j)$ be its standard deviation, then we have: 
\begin{equation}
x_{ij}' = \frac{x_{ij} - \bar{x_j}}{\text{sd}(x_j)}	
\end{equation}
	
	\item Residual: substract mean from the response variable: 
\begin{equation}
y_i' = y_i - \bar{y}	
\end{equation}

	\item Intercept: after standardization, the intercept would be 0, so for the original model, we would have: $\beta_0 = \bar{y}$. 	
	
	\item Covariance matrix: after standardization, the $p \times p$ matrix $X^T X / N$ represents the sample covariance matrix of the features $X_1, \cdots, X_p$. 
\end{itemize}

Subset selection: 
\begin{itemize}
	\item The best subsets are not necessarily nested: the best subset of size 2 does not always contain the best subset of size 1. To see an example, consider a function: $Y = 0.9 X_1 + 1.2 X_2$. Suppose there is a feature $X_3 = X_1 + X_2$. At size 1, the best subset is $X_3$; at size 2, the best subset is $(X_1, X_2)$ with coefficients $(0.9,1,2)$. 
	
	\item Forward stepwise selection: start with the intercept, sequentially add the predictor that most improves the fit. The improvement of fit is often based on $F$-statistic. Stops when the improvement is no longer significant based on $F$ distribution. 

	\item Backward stepwise selection: start with the full model and sequenntially delete predictors with the smallest $Z$-score. 
\end{itemize}

Ridge regression: 
\begin{itemize}
\item Minimize a penalized residue sum of squares: 
\begin{equation}
\hat{\beta}^{\text{Ridge}} = \text{argmin}_{\beta} \left\{\sum_{i=1}^N (y_i - \beta_0 - \sum_{j=1}^p x_{ij} \beta_j)^2 + \lambda \sum_{j=1}^p \beta_j^2\right\}
\end{equation}
The parameter $\lambda$ is the complexity parameter. Also to apply the method, all input features need to be standarized. The analytic solution can be found: 
\begin{equation}
\hat{\beta}^{\text{Ridge}} = (\bf{X^T X} + \lambda \bf{I})^{-1} \bf{X}^T y	
\end{equation}

\item Bayesian perspective of Ridge regression: If assume the prior distribution $\beta \sim N(0, \tau I)$, then the Ridge regression is effectively maximizing the posterior distribution of $\beta$; and the parameter $\lambda$ effectively corresponds to the variance of the prior. 

\item Benefit of ridge regression: when the variables are correlated, the coefficients can be poorly determined and exhibit high variance: a large positive coefficient can be canceled by a large negative coefficient on the correlated feature. By imposing the size constraint on the coefficients, this problem is alleviated. 

\item Principal component interpretation of ridge regression: let SVD of $X$ be: 
\begin{equation}
X = U D V^T	
\end{equation}
where $U$ is eigenvectors of $X X^T$ and $V$ eigenvectors of $X^T X$. The least square solutions are: 
\begin{equation}
\begin{array}{lll}
X \hat{\beta}^{\text{ls}}	& = & X (X^T X)^{-1} X^T \mathbf{y} = UDV^T \cdot VD^{-2}V^T \cdot VD^TU^T \mathbf{y} \\
 & = & U \cdot \text{diag}(1,1,\cdots, 1, 0, \cdots, 0) \cdot U^T \mathbf{y} = \sum_{j=1}^p u_j u_j^T \mathbf{y}
\end{array}
\end{equation}
Note that $u_j^T \mathbf{y}$ are projections of $\mathbf{y}$ on the orthogonal basis $U$. Also note that only $p$ dimensions are used in this equation (if all $N$ directions are used, we would have the RHS equal to $y$). Similarly, the ridge solutions can be written as: 
\begin{equation}
X \hat{\beta}^{\text{Ridge}} = \sum_{j=1}^p u_j \frac{d_j^2}{d_j^2 + \lambda}	u_j^T \mathbf{y}
\end{equation}
Thus the effect of ridge regression is shrinkage of the $y$ coordinates by $d_j^2 / (d_j^2 + \lambda)$. This shrinkage is strongest for those $j$'s of small PCs: in these directions, data points have smaller variance, thus it would be more difficult to determine the gradient of $y$ in these directions. 
\end{itemize}
 
Lasso regression: 
\begin{itemize}
\item Lasso objective function: the Lasso estimator is defined by: 
\begin{equation}
\begin{array}{lll}
\hat{\beta}^{\text{Lasso}} & = & \text{argmin}_{\beta} \sum_{i=1}^N (y_i - \beta_0 - \sum_{j=1}^p x_{ij} \beta_j)^2 \\
& & \text{subject to} \sum_{j=1}^p \lvert \beta_j \rvert \leq t
\end{array}
\end{equation}
We note that this is the convex optimization problem with Slater's condition satisified, thus strong duality holds. We show that the problem is equivalent to minimizing the following objective function with $L_1$ penalty: 
\begin{equation}
\hat{\beta}^{\text{Lasso}} = \text{argmin}_{\beta} \left\{\sum_{i=1}^N (y_i - \beta_0 - \sum_{j=1}^p x_{ij} \beta_j)^2 + \lambda \sum_{j=1}^p \lvert \beta_j \rvert \right\}
\label{eq:lasso}
\end{equation}
To see this, we note that the Lagrangian of the constrained optimization problem is: 
\begin{equation}
L(\beta,\lambda) = \norm{y - X \beta}^2 + \lambda \left( \sum_j \abs{\beta_j} - t \right)	
\end{equation}
The primal optimal $\beta^*$ should minimize $L(\beta,\lambda^*)$ at the dual optimal $\lambda^*$. Ignoring the constant term $\lambda^* t$, $\beta$ should minimize the objective function defined in Equation~\ref{eq:lasso} (the constant only affects the minimum of the objective function, but not $\hat{\beta}$). Note that $t$ is not given, thus we do not have to know how the dual optimal $\lambda$ depends on $t$. 

\item $L_1$ regularization from Bayesian perspective [Murphy, Section 13.3]: we use Laplace prior for $\beta$, $p(\beta|\lambda) \propto \exp(-\lambda \norm{\beta})$. The MAP estimator is then given by: 
\begin{equation}
\hat{\beta} = \text{argmin}_{\beta} RSS(\beta) + \lambda \norm{\beta}
\end{equation}
This is the equivalent to Lasso. 

\item Geometric intuition of lasso: as for constrained optimization problems in general, we visualize the feasibility set and the coutour line of the objective function. In this case, the feasibility set is a square $\sum_j \abs{\beta_j} \leq t$, and the contour line is $\norm{y - X \beta}^2 = C$, an ellipsid. The solution of lasso is thus the intersection of the ellipsid at with the square. To see why the intersection often occurs in the corners (thus one of $\beta_j$ is equal to 0):
\begin{itemize}
\item Depending on the slope of the axis of the contour line of the ellipsid: at some range, the intersection may occur at the line of the square; but for the other cases (e.g. the slope is parallel to the x-axis in 2D case), the intersection occurs at the corner. 
\item In general, because of the discontinuity at the corner, the intersections tend to occur at the corner (the boundary point, as opposed to an interior point when the constraint is smooth). 
\end{itemize}

\item The tuning parameter of Lasso: the shrinkage factor is defined as: 
\begin{equation}
s = t/ \sum_{j=1}^p \lvert \hat{\beta_j} \rvert	
\end{equation}
where $\hat{\beta_j}$ are least square estimates. As $s = 1.0$, the least square estimate will automatically satisfy the constraint, so there is no effect of shrinkage. As $s \rightarrow 0$, the parameters decrease to 0. 

\item Optimization: the objective function is a quadratic function, thus can be solved with quadratic programming. Efficient algorithms exist for solving the entire Lasso path (as $\lambda$ varies) - least angle regression (LAR) (see below for cyclic coordinate descent algorithm). The basic idea is that as $\lambda$ changes, the selected variables change only at a few critical points, which can be determined. 

\item Remark: learning sparse models. Note that regularization itself (constraints on parameters) does not necessarily lead to sparse models, e.g. $L_2$ norm. The sparsity of the learned model is a consequence of the non-differentiability of the constraint ($L_1$ norm). In general, design the constraint s.t. the solution is a sparse model. 

\item Remark: Lasso implicitly penalize large regression coefficients. This may not be desired in practice. Ex. Lasso model for association test with both common and rare variants. Rare variants generally have large effects, but small explanatory power, so Lasso could penalize rare variants too much. 

\item Memory usage of Lasso [personal note]: glmnet can be memory expensive. One strategy is to use variable selection: from univariate analysis. However, this may not be safe as a general strategy, because the true effects of a variable is learned by adjusting all others. In GWAS problem, this is probably fine b/c most of SNPs are largely independent. 
\end{itemize}

Comparison of subset selection, Ridge and Lasso regression: 
\begin{itemize}
\item Shrinkage effect [Murphy 13.3.3]: all methods can be viewed as shrinkage of parameters (smaller number of parameters lead to simpler model, thus lower variance). Let $\hat{\beta}^{OLS}$ be least square estimator of $\beta$, the difference (using the case where the data input is orthonomal matrix, i.e. uncorrelated features as an example):  
\begin{itemize}
	\item Subset selection: drop all variables whose coefficients are ranked lowest. A form of ``hard thresholding''. 
	\item Ridge: proportional shrinkage, the estimator is $\hat{\beta}^{OLS} / (1 + \lambda)$. 
	\item Lasso: A form of ``soft thresholding''. Truncate parameters by a constant $\lambda/2$: the lasso estimator is: 
	\begin{equation}
	\hat{\beta}^{Lasso} = \text{sign}(\hat{\beta}^{OLS}) \left(\abs{\hat{\beta}^{OLS}} - \frac{\lambda}{2}\right)_+
	\end{equation}
	where the last term is 0 if $\abs{\hat{\beta}^{OLS}} < \frac{\lambda}{2}$. 
\end{itemize}

\item Subset selection vs Lasso or Ridge: subset selection is a discrete method, which tends to have high variance: in two independently generated datasets, two subsets may be chosen because of noises in the data. 

\item Lasso vs Ridge: (Figure 3.11) the constraint in Lasso regression has corners, thus with Lasso, there are many opportunities for the estimated parameters to be zero. 

\end{itemize}

Limitations of Lasso: 
\begin{itemize}
\item Correlation of variables: [Friedman, Regularization Paths for Generalized Linear Models via Coordinate Descent, 2009] when some explanatory variables are highly correlated, lasso will choose one arbitrarily. In contrast, ridge regression will split the weights among these variables, a preferred choice. Mixing the two (elastic net) may be a better option. 

\item Consistency of Lasso [Murphy, 13.3.5]: the estimators are always biased. Because of the penalty, it will not converge to the true $\beta$ even as $N \rightarrow \infty$ (not ``model selection consistent''). Ideally, we want our estimator of a variable $j$ to be close to its true value, if the effect of $j$ is large. 

\item Addressing consistency problem: (1) adaptive lasso, where the penalty $\lambda$ can be different for different variables. (2) Debiasing (Murphy Figure 13.9): use Lasso only to select variables, then do least-square estimator of selected variables. 

\item Statistical inference of Lasso: Lasso describes an algorithm, but does not directly permit inference. For example, what is the FDR of the features. Boostrap lasso (bolasso): approximate posterior inclusion probabilities, boostrap samples, and choose a variable if it occurs in at least 90\% of the sets returned by Lasso, for a given $\lambda$. 

\item Bayesian ideas of addressing the limitations of Lasso: for adaptive Lasso, instead of having a weight for each variable, we could have a hierarchical model of $\beta_j$ s.t. it has shrinkage property but the extent of shrinkage is also specific to each variable.
\end{itemize}

General form and elastic net: [Friedman, Regularization Paths for Generalized Linear Models via Coordinate Descent, 2009]
\begin{itemize}
\item The general form of the penalty term is: $\lambda \sum_j \lvert \beta_j \rvert^q$. The case $q = 1$ is Lasso, and $q = 2$ is ridge regression. However, in general, with $q > 1$, the penalty does not share the ability of Lasso to set many coefficients exactly to 0 (often preferred: simpler model and interpretation). 

\item Elastic net: the penalty provides a compromise between ridge and lasso: 
\begin{equation}
\frac{1}{2N} \sum_i (y_i - x_i \beta)^2 + \lambda \sum_{j=1}^p \left( \frac{\alpha}{2} \beta_j^2 + (1-\alpha) \lvert \beta_j \rvert \right)	
\end{equation}
For simplicity, we also assume $x_{ij}$ are standardized, i.e. $\sum_{i} x_{ij} = 0$ and $1/N \sum_{i} x_{ij}^2 = 1$. 

\item Cyclic coordinate descent algorithm: (conditional minimization algorithm) iteratively update the parameters. At the step $j$, we assume all parameters $\tilde{\beta}_l, l \neq j$ are known, and we find the optimal $\beta_j$. Note that the objective is a quadratic function of $\beta_j$. Denote by $R(\beta)$ the objective function defined above, the solution of $\beta_j$ should satisify the condition that the derivative is equal to 0 if $\beta_j \neq 0$. Suppose $\beta_j > 0$, we have: 
\begin{equation}
\frac{\partial R}{\partial \beta_j} = -\frac{1}{N} \sum_i x_{ij} (y_i - \tilde{\beta}_0 - \sum_{l \neq j} x_{il} \tilde{\beta}_l - x_{ij} \beta_j) + \lambda (1 - \alpha) \beta_j + \lambda \alpha
\end{equation}
Solving this equation (using the fact that $1/N \sum_{i} x_{ij}^2 = 1$) , we have: 
\begin{equation}
\beta_j = \frac{\frac{1}{N} \sum_i x_{ij}(y_i -\tilde{\beta}_0 - \sum_{l \neq j} x_{il} \tilde{\beta}_l ) - \lambda \alpha}{1 + \lambda ( 1- \alpha)}	
\end{equation}
Let the first term in the numerator be $z$, then if $z \geq \lambda \alpha$, the function is minimized at $\beta_j$ defined above; if $z < \lambda \alpha$, it is minimized at 0. The similar condition exists for $\beta_j < 0$ and $\beta_j = 0$. 

\item Interpretation of the update rule in cyclic coordinate descent algorithm: first, the simple least square fit of $\beta_j$ is obtained, between the $j$-th explanatory variable and the partial residual (fitting $y_i$ using all other features and the current estimates of all other coefficients). Next, we decide if the coefficient should be 0 or not (lasso constraint), by comparing the coefficient with $\lambda \alpha$. Finally, we apply the proportional shrinkage, $ 1 + \lambda (1 - \alpha)$, for the ridge penalty. 

\item Computational efficiency: see ``Covariant update'' in the paper. Basically the computation of terms in the update can be simplified by storing the reused terms, the inner product of $x_i$ and $y$. 

\item Positivity constraint (not verified): if we want $\beta_j \geq 0$, we simply assume that there is an implicit constraint that $\beta_j \geq 0$ in the previous problem. Every step above would be the same, except that we need to change the update rule: suppose $z$ is defined as before, we have (using the fact that the objective function is quadratic of $\beta_j$): 
\begin{equation}
\tilde{\beta_j} = \left\{ \begin{array}{ll}
\frac{z - \lambda \alpha}{1 + \lambda ( 1 - \alpha)} & \text{ if } z > \lambda \alpha\\
0 & \text{ otherwise}
\end{array}
\right.	
\end{equation}
\end{itemize}

Fused lasso: [Variable fusion: a new adaptive signal regression method, 1996; Sparsity and smoothness via the fused lasso; 2005]
\begin{itemize}
\item Motivation: the explanatory variables can be ordered, and in the correct model, the coefficients of the nearby variables should be close to each other. Ex. classification of cancer status with mass-spec. data of many compounds, clearly the nearby variables (compounds with similar $m/z$ ratio) have similar chemical properties and thus should have similar effect on cancer status. 

\item Fusion: our constraint is that the sum of the (absolute difference) of the coefficents of nearby features should be small: 
\begin{equation}
\sum_{j=2}^{p} \abs{\beta_j - \beta_{j-1}} \leq t
\end{equation}
The constraint can be geometrically represented as stripes. To see why this penalty leads to sparse solutions, we can assume we do variable substitution with $\gamma_j = \beta_j - \beta_{j-1}$, and the problem is formulated in terms of $\gamma$. This is then a Lasso regression and the solution would encourage $\beta_j = \beta_{j - 1}$. 

\item Fused lasso: to encourage both sparseness, and the closeness of nearby coefficients, we solve this problem: 
\begin{equation}
\text{Minimize } \sum_i (y_i - x_i \beta)^2 \qquad \text{subject to } \sum_j \abs{\beta_j} \leq s_1 \text{ and } \sum_j \abs{\beta_j - \beta_{j-1}} \leq s_2	
\end{equation}
\end{itemize}

Group lasso [Model selection and estimation in regression with grouped variables, JRSSB, 2006]:
\begin{itemize}
\item Motivation: suppose the explanatory variables can be grouped, and we expect that the variables tend to be selected as a group. Ex. in the association of multiple genetic markers and phenotype, the markers form groups (e.g. of the same pathway), and there should be only a few groups that are relevant to a phenotype. However, within the group, there is no additional constraint/preference (i.e. no sparsity within the group). 

\item Group lasso penalty: suppose we have $J$ groups of input variables, our objective is to minimize the least square function subject to: 
\begin{equation}
\sum_j \norm{\beta_j}_2	\leq t
\end{equation}
where 
\begin{equation}
\norm{\beta_j}_2 = \sqrt{\sum_k \beta_{jk}^2}	
\end{equation}
Note that the $L_2$ component above performs ridge penalty within a group, and the $L_1$ component (e.g. imagine for some groups, there is only one variable, then it effectively becomes Lasso) encourages sparse group selection. 

\item Example: consider three variables where the first two form one group, the constraint is then: 
\begin{equation}
\sqrt{\beta_{11}^2 + \beta_{12}^2} + \abs{\beta_2} \leq t	
\end{equation}
To see what this constraint set look like, we fix one parameter, and check the 2D picture of the constraint region of the other two. When $\beta_2$ is fixed at $c$, the feasibility set: 
\begin{equation}
\sqrt{\beta_{11}^2 + \beta_{12}^2} \leq t - c	
\end{equation}
This is a circle, and we have ridge penalty within the group. When $\beta_{11}$ (or $\beta_{12}$) is fixed at $c$, the feasibility set:
\begin{equation}
\sqrt{c^2 + \beta_{12}^2} + \abs{\beta_2} \leq t		
\end{equation}
The boundary of this set consists of two pieces of parabolas (the top and the bottom piece) joined together. The points where the two pieces joined (at the $x$-axis) are not differentiable, creating corners. This allows opportunity of learning sparse models, similar to Lasso. 

\end{itemize}

Lasso of multiple related response variables [CMU CS 10-170 lecture 18]:
\begin{itemize}
\item Motivation: suppose we have multiple response variables that are related, i.e. two related traits are likely to also share the same explanatory variables. The general idea is similar to fused lasso, where we penalize the difference between coefficients. 

\item Graph-guided fused lasso: we penalize the difference between coefficients (of the same explanatory variable) on highly correlated response variables. Let $\beta_{jm}$ be the coefficient of the $j$-th explanatory variable on the $m$-th response variable, then our constraint in addition to lasso is: 
\begin{equation}
\sum_{(m,l) \in G} f(r_{ml}) \sum_j |\beta_{jm} - \text{sign}(r_{ml})\beta_{jl}| \leq s
\end{equation}
where $G$ is the graph representing the relation/similarity between response variables, and $(m,l)$ denotes an edge in $G$, $r_{ml}$ is the correlation and $f(r_{ml})$ is some monotic function of $r_{ml}$, e.g. $f(r) = 1$ (unweighted) or $f(r) = |r|$. 


\item Temporarlly-smoothed lasso: suppose $Y_t$ are response variables over time $t$. Let $\beta_{j,t}$ be the coefficient of the $j$-th explanatory variable on $Y_t$, our constraint in addition to lasso constraint is: 
\begin{equation}
\sum_j \abs{\beta_{j,t+1} - \beta_{j,t}} \leq s
\end{equation}

\end{itemize}

Tree-guided group lasso [CMU CS 10-170 lecture 18]: 
\begin{itemize}
\item Motivation: a generalization of group lasso (in response variables). Suppose we have multiple response variables (traits) related by a tree: for the traits that are really close, we should choose them as a group (i.e. the same explanatory variables likely influence all); for the traits that are far aways, they should be unrelated (sparsity only). 

\item Idea: consider two subtrees of traits $T_L$ and $T_R$. Our penalty consists of two parts: (1) if the two subtrees are very close, then we should choose both subtrees (hence all offspring nodes) as a group (group penalty); (2) if the two subtrees are distant, then we should choose either one of them (lasso penalty). The final penalty should be a mix of the two types of penalty with the weight dependent on the distance between $T_L$ and $T_R$. 

\item Example: two leaf nodes $(Y_1,Y_2)$. Let $h$ be the height of the tree (the distance between $Y_1$ and $Y_2$), the penalty is: 
\begin{equation}
\lambda \sum_j \left[ (1 - h) \sqrt{\beta_{j1}^2 + \beta_{j2}^2} + h (\abs{\beta_{j1}} + \abs{\beta_{j2}})\right]	
\end{equation}

\item Example: three leaf nodes $((Y_1,Y_2),Y_3)$. Let $h_1$ be the height of the subtree $(Y_1,Y_2)$ and $h_2$ be the height of the remaining tree. The penalty 
\begin{equation}
\lambda \sum_j \left[ (1 - h_2) \sqrt{\beta_{j1}^2 + \beta_{j2}^2 + \beta_{j3}^2} + h_2 (\abs{C_1} + \abs{\beta_{j3}})\right]	
\end{equation}
where $C_1$ is the penalty of the two node tree (defined recursively): 
\begin{equation}
C_1 =	(1 - h_1) \sqrt{\beta_{j1}^2 + \beta_{j2}^2} + h_1 (\abs{\beta_{j1}} + \abs{\beta_{j2}})
\end{equation}

\item Remark: how the height of the tree is define? Ex. for the three-node example, $h_2$ should only include the distance from the common ancestor of $Y_1$ and $Y_2$ vs. $Y_3$. 
\end{itemize}
%%%%%%%%%%%%%%%%%%%%%%%%%%%%%%%%%%%%%%%%%%%%%%%%%%%%%%%%%%%%
\subsection{Bayesian Variable Selection}

Bernoulli-Gaussian model and $l_0$ regularization [Murphy, 13.2.2]
\begin{itemize}
	\item Model: $y = w x + \epsilon$, we can rewrite spike-and-slab prior of $w$ as: 
	\begin{equation}
	\gamma_j \sim \text{Ber}(\pi) \qquad w_j \sim N(0, \sigma_w^2)
	\end{equation}
	With this prior, we can write our model as: $y = \sum_j w_j \gamma_j x + \epsilon$. Under this model, only $w_j \gamma_j$ is identifiable. 
\end{itemize}

Automatic relevance determination (ARD) prior [Murphy, 13.7]
\begin{itemize}
	\item Model: we have $y = wx + \epsilon$, we use a normal prior for $w$, $w_j \sim N(0, 1/\alpha_j)$ and $\epsilon \sim N(0, 1/\beta I)$. ARD approach would do EB estimation of $\alpha$, and then obtain the posterior of $w$. The estimation can be done by EM. The procedure is simpler than spike-and-slab prior. 
	
	\item How the prior leads to sparsity? See Figure 13.20. We note that when $\alpha_j \to \infty$, we have $w_j \approx 0$, so this leads to variable selection. We claim that EB optimization of $\alpha$ would lead to $\alpha_j \approx \infty$ if a feature $j$ is irrelevant/independent of $y$. Consider a simple case of one independent variable $x$. We consider the distribution of $y$, marginalizing over $\beta$. This is given by:
	\begin{equation}
	y | x, \alpha \sim N\left( \frac{1}{\alpha} x x^T + \frac{1}{\beta} I \right)
	\end{equation}
	When $y$ is independent of $x$, we would expect $y$ to be independent of $x$, so the distribution of $y$ should be spherical. However, when $\alpha$ is finite, the marginal distribution would not be spherical, but this ``wastes'' probability mass. 
	\begin{itemize}
		\item Remark: similar to heritability analysis: the covariance of $y$ depends on GRM and heritability. When $y$ and $x$ are independent, REML should lead to $h^2_g = 0$. 
	\end{itemize}
\end{itemize}


Bayesian Variable Selection in Structured High-Dimensional Covariate Spaces With Applications in Genomics [Li \& Zhang, JASA, 2010]:
\begin{itemize}
	\item Motivations: when selecting the true predictors, there is a certain dependency s.t. if one variable is selected, another related one is likely selected too. Eamples:
	\begin{itemize}
		\item Gene expression modeling: from promoter composition (words as features) to expression level. The true predictors are correlated: if one word is selected, then its neighbor (defined by Hamming distance) has a higher probability of being related to gene expression too.  
		\item Cancer CNVs to survival outcome: predictors are CNVs, and there is a linear/spatial dependence. 
		\item fMRI data to behavior traits: the predictors are voxel intensities, but there is spatial smoothness in the selection of true predictions - true signals usually represent connected regions in the brain. 
	\end{itemize}
	
	\item Idea: use MRF prior for variable selection indicators. Computationally, the advantage is tha the MRF prior can structure the MCMC moves because effectively MCMC searches a smaller set of configurgations based on the MRF prior (e.g. only smooth configurations for spatially-motivated examples). 
	Phase transition problem: the configruation selected may be sensitive to some hyperparameters. 
	
	\item Model: consider a linear model $Y = X \beta + \epsilon$, let $\gamma_i$ be an indicator of whether $X_i$ is selected. The distribution of $\beta_i, 1 \leq i \leq p$ is: 
	\begin{equation}
	\beta_i | \gamma_i = 0 \sim I_0 \qquad 	\beta_i | \gamma_i = 1 \sim N(0, \sigma^2 \nu^2)
	\end{equation}
	where $I_0$ is a point mass at 0, $\sigma^2$ is the residual variance, and $\nu^2$ is the variance of $\beta_i$ (in the unit of $\sigma^2$). The prior of $\sigma^2$ follows the standard inverse-gamma conjugate prior. For the prior of $\gamma$, suppose we have a graph $G$ representing the dependency of variables, then we have the prior: 
	\begin{equation}
	P(\gamma) \propto \exp(a^T \gamma + \gamma^T B \gamma)	
	\end{equation}
	where $a = (a_1, \cdots, a_p)^T$ is a vector, and $B = (b_{ij})$ is $p \times p$ matrix. Usually, we assume $a_i < 0$, thus any $\gamma_i = 1$ will introduce penalty to $P(\gamma)$, and this leads to \emph{sparsity} of the model. For $B$, we assume $b_{ij} = 0$ if $(i,j) \notin G$, and $b_{ij} > 0$ otherwise. Then any edge $(i,j)$ s.t. $\gamma_i = \gamma_j = 1$ will be favored by the model, leading to \emph{smoothness} of the model. We also assume a single constant $a$ for all $a_i$'s, and similarly another constant $b$ for all $b_{ij}$'s. 
	\begin{itemize}
		\item The intuition of the Ising prior is that: we have a certain budget of $\gamma_i = 1$ (due to sparsity), and we want to allocate it s.t. $\gamma$ is smooth (the neighbors have the same $\gamma_i$). 
	\end{itemize}
	
	\item Inference/Gibbs sampling: we are searching the configruation space of $\gamma$. Using Gibbs sampling, we update $\gamma_i$ at each step, and need to compute $P(\gamma_i | \gamma_{-i}, y)$ where $y$ is all the data. Because $\gamma_i = 1$ or 0, this conditional probability can be computed as (similar to Bayesian model selection): 
	\begin{equation}
	\frac{P(\gamma_i=1|\gamma_{-i}, y)}{P(\gamma_i=0|\gamma_{-i}, y)}	= \frac{P(\gamma_i=1|\gamma_{-i})}{P(\gamma_i=0|\gamma_{-i})} \times \frac{P(y|\gamma_i=1,\gamma_{-i})}{P(y|\gamma_i=0,\gamma_{-i})}
	\end{equation}
	So the posterior odds is the prior odds multiplied by the Bayes factor. The computation of BF under a given $\gamma$ follows from standard Bayesian regression analysis, by integrating out $\beta$ and $\sigma$. The computation time for one iteration ($p$ variables) is $O(p p_i^2)$, where $p_i$ is the model size. 
	
	\item Phase transition of the Ising prior: the proportion of $\gamma_i = 1$ is sensitive to the hyperparameters $(a,b)$. In simulations assuming a regular graph (equal degree), the proportion can change sharply from all 0's to nearly all 1's when one varies the value of $b$, near the phase transition boundary. The intuition is some kind of positive feedback: as we increases $b$, it is favored to have more 1's, but as we have more 1's, at some point, it will favor even more 1's (if a lot of a node's neighbors are 1's, then this node should be 1 too). 
	\begin{itemize}
		\item For the Ising prior, the phase transition can be analyzed using the mean field theory: how the proportion of 1's depends on the parameters (temperature). 
		\item The posterior model has phase transition, but cannot be studied analytically. Some heuristics are offered about how to choose $a$ and $b$. 
	\end{itemize}
	
	\item Lessons: 
	\begin{itemize}
		\item In many problems, we favor a certain smoothness in the model (sequence data, spatial/image data, structure ...), and this can be encoded by an Ising prior. 
		\item Analysis of the model behavior/sensitivity to hyperparameters is important in Bayesian inference: in this case, how the results (proportion of $\gamma_i=1$) depends on the smoothness parameter.
	\end{itemize}
	
	\item Remark: the prior of $\gamma$ can be modified to incoprorate penalty for $(1,0)$ edges. 
\end{itemize}

Incorporating biological information into linear models: A Bayesian approach to the selection of pathways and genes [Stingo \& Vannucci, AAS, 2011]
\begin{itemize}
	\item Problem: given gene expression data and responses (e.g. survival outcomes), the goals are (1) a predictive model from expression to response; (2) identification of the relevant genes. The main motivation here is to incorporate pathway information. Methods such as GSEA can only identify genes, but not predict responses. 
	
	\item Background: some relevant works. Doing dim. reduction on the pathways (PCA) and use the PC (the ``supergene'') as explanatory variables. Priors in regression that incorporate gene-gene relationship. 
	
	\item Idea: pathway activities as explantory variables (similar to PCA); within pathways, the selection of genes can be enhanced by an Ising prior representing the network. 
	
	\item Model: suppose we have $K$ pathways, let $\theta_k$ be the indicator of the $k$-th pathway: whether it is selected. For each gene, we have $\gamma_i$ as gene indicator. The pathway level activity is the Partial Least Square Regression (PLS) of gene expression vs. response, using only genes in the pathway whose $\gamma_i = 1$. For the $k$-th pathway, we use $k(\gamma)$ to indicate the subset of genes that are selected. The linear model is: 
	\begin{equation}
	Y = \alpha + \sum_k T_{k(\gamma)} \beta_{k(\gamma)}	+ \epsilon
	\end{equation}
	where $T_{k(\gamma)}$ is the activity of the pathway derived from PLS. The prior of $\theta$ follows Bernoulli distributions. The prior of $\gamma$ is given by the Ising prior: $P(\gamma) \propto \exp(\mu \mathbf{1} \gamma + \eta \gamma^T R \gamma)$ where $R$ is the graph representing the gene relationship. To make the model identifiable, there are also constraints on $\theta$ and $\gamma$: no orphan gene - a gene cannot be selected if none of its pathway is selected. No empty pathway. A subset of genes may belong to multiple pathways: need to be resolved. 
	
	\item Inference: the regression parameters will be integrated out. Main parameters to be learned are: $(\theta, \gamma, \eta)$. Use Gibbs sampling. For $P(\theta, \gamma|\eta, D)$ where $D$ is the data, use MH algorithm. The MH moves are structured to implement the constraints of $\theta,\gamma$. For the posterior of $\eta$, it only depends on $\gamma$, so we sample $P(\eta|\gamma) \propto P(\eta) P(\gamma|\eta)$. 
	
	\item Remark: a main motivation of modeling pathway is that the genes can share information: if some genes in a pathway are chosen, then other genes in the same pathway are more likely to be selected as well. So $\gamma$ should depend on $\theta$, but this is not explicitly modeled. The dependence is only modeled as extra constraints that $\theta$ and $\gamma$ must satisfy. 
\end{itemize}

Bayesian variable selection regression for genome-wide association studies and other large-scale problems [Guan and Stephens, AAS, 2011]
\begin{itemize}
	\item Model: let $\tau^{-1}$ be the variance of the error (environmental effect) of the phenotype:  
	\begin{equation}
		y = \mu + X \beta + \epsilon \qquad \epsilon \sim N(0, \tau^{-1})
	\end{equation}
	We assume a sparse prior for $\beta$. Let $\gamma$ be the indicators of all SNPs: 
	\begin{equation}
		\gamma_j \sim \text{Ber}(\pi) \qquad \beta_j | \gamma_j = 0 \sim \delta_0 \qquad \beta_j | \gamma_j = 1 \sim N\left(0, \frac{\sigma_a^2}{\tau}\right)
	\end{equation}
	where $\sigma_a$ is the effect size, measured in unit of $\tau$. Ex. a SNP with $\sigma_a = 0.1$ means that the SNP changes $y$ by 0.1 standard deviation (note not the sd. from phenotypic variance).  
	
	\item Prior of $\pi$: we use $\log \pi \sim U(a,b)$, where $a$ is a small number, say $1/p$, where $p$ is the number of SNPs, and $b$ the upper bound (at most 1). Comparing with $\pi \sim U(a,b)$, this prior puts more probability mass on smaller values (at log scale, 0 to 0.001 becomes $-\infty$ to -3, clearly most probability mass are far from -3). 
	
	\item Prior of $\sigma_a^2$: if we set the prior as a constant, the issue is that the more variants we have, the larger PVE will be. This is undesirable. Instead, we assume a prior on PVE, and use the PVE to set the value of $\sigma_a^2$. Suppose we know $\gamma$, the variance explained by genotypes is: 
	\begin{equation}
		V_G(\gamma) = \frac{\sigma_a^2}{\tau} \sum_{j: \gamma_j = 1} s_j
	\end{equation}
	where $s_j$ is the variance of the SNP ($p_j(1-p_j)$). PVE is related to $V_G$ by $h = V_G / (V_G + 1/\tau)$. This allows to have: 
	\begin{equation}
		\sigma_a^2 = \frac{h}{1-h} \frac{1}{\sum_{j: \gamma_j = 1} s_j}
	\end{equation}
	Note that the constant term $\tau$ is canceled out. So in practice, we specify the prior on PVE: $h \sim U(0,1)$, and once $h$ and $\gamma$ is given (from Bernoulli prior), we can compute $\sigma_a^2$. 
	
	\item Relationship between expected PVE and effect size (personal notes): let $s_a$ be the average variance of SNPs, and $p$ be the number of SNPs, we have: 
	\begin{equation}
		V_G = p \pi s_a \sigma_a^2 / \tau
	\end{equation}
	And the PVE is given by: 
	\begin{equation}
		h = \frac{p \pi s_a \sigma_a^2}{p \pi s_a \sigma_a^2+1}
	\end{equation}
	This shows that at a given $h$, the more causal SNPs we have, the smaller effect size per SNP. This allows us to estimate PVE due to a single SNP. Suppose we have a SNP with effect $\sigma_j$ (in the unit of $\tau$) and variance $s_j$, the PVE of this SNP is: 
	\begin{equation}
		\text{PVE}_j = \frac{\sigma_j^2 s_j/\tau}{V_G + 1/\tau} = (1-h) \sigma_j^2 s_j
	\end{equation}
	Ex. a SNP with effect 0.2 sd, and AF 0.2, and $h = 0.5$, its PVE is 0.0032 = 0.32\%. The PVE explained by a single causal SNP on average is simply $h / (p \pi)$, the total PVE divided by the number of causal SNPs. 
	
	\item Inference: we use MCMC to sample the key parameters $h$ and $\pi$, and the configuration $\gamma$. 
	\begin{equation}
		P(h, \pi, \gamma | y) \propto P(y|h, \gamma) P(h) P(\gamma|\pi) P(\pi)
	\end{equation}
	Note that $P(y|h, \gamma)$ integrates out $\beta$ and $\tau$, which has a closed form. Some key ideas of MCMC: (1) mostly local move by MH, sometimes change many $\gamma_j$'s at once. (2) Sample $\gamma$ with large marginal association statistic. 
	
	\item Estimation of PVE, mapping causal variants and phenotype prediction: 
	\begin{itemize}
		\item PVE: once we sample $\beta$ and $\gamma$, we can obtain the actual PVE explained by the causal SNPs.
		\item Mapping: estimate $P(\gamma_j=1|y)$, this uses Rao-Blackwellisation. Intuitively, this is testing a causal SNP by conditioning on all other causal SNPs. 
		\item Prediction: use $\E(y_{n+1}|y) = x_{n+1} \E(\beta|y)$. 
	\end{itemize}
	
	\item Simulation procedure: 
	\begin{enumerate}
		\item Sample 10k SNPs with AF sampled from $U(0.05, 0.5)$. 
		\item Sample $h \sim U(0,1)$. 
		\item Sample causal SNPs (30). 
		\item From $h, \gamma$ and AF, determine $\sigma_a^2$. 
		\item Sample phenotypes. 
	\end{enumerate}
	
	\item PVE estimation: Figure 1: scatter plot of Estimated PVE vs. True PVE in simulation. Results: With a large number of causal SNPs, estimation of PVE is difficult (hard to distinguish null with a large number of variants with very small effects).
	
	\item Identification of causal SNPs: Figure 3: vary threshold of different methods, and estimate TP and FP rates using ROC. For single-SNP test: vary single SNP BF. For BVSR: vary PIP. For Lasso: first compute solution path as $\lambda$ varies, and then compute TP and FN rates as $\lambda$ varies. In simulation with LD, compare region level statistics. Results:  multi-SNP methods, BVSR and Lasso, better than single-SNP, due to controlling SNPs. 
	
	\item Prediction of phenotype: BVSR better than Lasso. This is due to problem with Lasso: single $\lambda$ controls both sparsity and shrinkage (Elastic Net would be better). 
	
	\item Calibration of posterior inclusion probability (PIP): Figure 5: proportion of True Positives vs. PIPs. Knowing $\pi, \sigma_a$ helps calibration. Note that the BFs are relatively insensitive when $\sigma_a$ is larger than the true value.  

	\item Application in real data: evaluation using region-based analysis, e.g. prob. that a region contains at least 1 causal SNP. 
\end{itemize}

Scalable variational inference for Bayesian variable selection in regression and its accuracy in genetic association studies [Carbonetto and Stephens, Bayesian Analysis, 2012]
\begin{itemize}
	\item Model: linear regression
	\begin{equation}
	y = \beta_0 + \sum_{k=1}^p X_k \beta_k + \epsilon \qquad \epsilon \sim N(0, \sigma^2)
	\end{equation}
	Use spike-and-slab prior for $\beta_k$: $\gamma_k \sim \text{Bern}(\pi)$, and $\beta_k | \gamma_k = 1 \sim N(0, \sigma_{\beta}^2 \sigma^2)$. The main unknowns are $\beta, \gamma$ and the hyperparameters are $\theta = (\pi, \sigma_{\beta}^2, \sigma^2)$. Setting the prior of hyperparameters: encourage sparsity, e.g. in Crohn's GWAS data, a normal prior on $\log \pi / (1-\pi)$ s.t. 95\% prob. mass are in the range of 0 to 70 causal variants. 
	
	\item VB inference of $\beta, \gamma$ given hyperparameters: we approximate $p(\beta, \gamma|y, X, \theta)$ by $q(\beta, \gamma) = \prod_k q(\beta_k, \gamma_k)$. This is valid when $X_j$'s are independent, but not in general. VB inference means that we should update $\beta_k, \gamma_k$ by the equation: 
	\begin{equation}
	\log q(\beta_k, \gamma_k) = \E_{q(\beta_{-k}, \gamma_{-k})} [\log p(\beta, \gamma|y, X, \theta)]
	\end{equation}
	where we take expectation over $q(\cdot)$ of other parameters. We note that the log posterior has three components:
	\begin{equation}
	\log p(\beta, \gamma|y, X, \theta) = \log p(\gamma| \pi) + \log p(\beta|\gamma, \theta) + \log p(y|X, \beta, \gamma) 
	\end{equation}
	We can expand these terms and take expectations. For the first one, we have: $\log \pi \sum_k \gamma_k + \log (1-\pi) \sum_k [p-\sum_k \gamma_k]$. For the last term, we have $-1/(2 \sigma^2) [(y-X\beta)^T (y-X\beta)]$, whose expectation over $\beta$ can be determined analytically. This leads to the VB iterative update (Equations 8-10) in terms of: $\alpha_k$, the probability that $\gamma_k$ is 1; and $\mu_k, s_k^2$ the mean and variance of $\beta_k$ if $\gamma_k = 1$. This is equivalent to solving univariate regression problem, where all other coefficients are given by their posterior mean:
	\begin{equation}
	y = X_k \beta_k + \sum_{j \neq k} X_j \E(\beta_j | D) + \epsilon
	\end{equation}
	\begin{itemize}
		\item Equation (8): for $s_k^2$, this is posterior variance of $\beta_k$ in the univariate regression. Since $\beta_j$'s are given, this is also the same as the simple regression: $y = X_k \beta_k + \epsilon$. 
		\item Equation (9): for $\mu_k$, this is the posterior mean of the regression above. 
		\item Equation (10): the posterior ratio is the product of prior ratio and BF. Note: the log-BF has the term $\mu_k^2 / s_k^2$, which is roughly the chi-square of variable $k$. 
	\end{itemize}
	
	\item Averaging over hyperparameters by importance sampling: PIPs of variable $k$ should be averaged over all hyperparameters $\theta$. However, we should weigh them by their posterior density $w(\theta)$. This is done by the approximation of marginal likelihood (summing over $\beta, \gamma$) by ELBO from VB inference.  
	
	\item VB algorithm: Figure 1. Outer loop: over 100-1000 hyperparameters, with each weighted by ELBO. Inner loop: compute $\beta, \gamma$ given $\theta$. Final results: average over hyperparameters. 
	
	\item Behavior of VB in simple simulations: two variable in different degrees of correlation. Tend to overetimate the mode. The independence of posterior assumption leads to problem. Ex. two perfectly correlated variables, in the true posterior, we should have $\gamma_1, \gamma_2$ highly correlated: PDF ellipse along the diagonal line. However, in the posterior, because the two are independent, we need only one variable $\gamma_1$ or $\gamma_2$ (either variable is sufficient to explain the data, and are two equal modes), so VB posterior is horizontal or vertical ellipse. 
	
	\item Behavior of VB in real data of genomic regions: (1) Accurate estimation of hyperparameters. (2) In regions of high LD: VB shows single SNPs, while MCMC captures uncertainty. However, VB still correctly calculates the expected number of causal SNPs in the block (Figure 9). 
	
	\item Comparison of MCMC vs. VB in real data: WTCCC, 4000 samples, 500K SNPs. Full VB takes a day. Some disagreement of VB and MCMC: however, possible that MCMC miss some regions because of convergence issues. 
	
	\item Extensions: can be used for binary traits, and other priors of effects.  
\end{itemize}

Bayesian structured sparsity from Gaussian fields [Engelhardt \& Adams, arxiv, 2014]
\begin{itemize}
	\item Background: Bayesian approach to sparsity
	\begin{itemize}
		\item Spike-and-slab prior (two group): $\beta_j$ follows a mixture prior, with one component point mass 0. The challenge is to search in an exponential space. 
		\item Continuous relaxation (one group): e.g. Laplacian prior. Often apply a threshold after learning $\beta_j$'s - this is called zero assumption (if an effect is very small, its true value is probably 0). 
	\end{itemize}
	
	\item Motivation: linear model where predictors are correlated. Ex. association analysis, SNPs in LD are correlated. The goal is to encourage ``dense within-group'' sparsity: the closely related predictors should be all 0's or all 1's. 
	
	\item Model: the idea is to represent the dependency between $z_j$'s (indicator variables) using MVN as an underlying distribution. Let $\Gamma$ be the indicator matrix, a diagonal matrix with $\Gamma_{j,j} = z_j$, where $z_j$ is the indicator of the $j$-th predictor. The prior can be written as: 
	\begin{equation}
	\beta | \Gamma \sim N(0, (\nu \lambda)^{-1} \Gamma)	
	\end{equation}
	where $\nu$ is residual precision and $\lambda$ the precision of $\beta$ in the unit of $\nu$. For $\Gamma$, instead of using independent Bernoulli distributions, we assume there is an underlying latent distribution (Gaussian field): 
	\begin{equation}
	\gamma \sim N(0, \Sigma)	
	\end{equation}
	where $\Sigma$ is positive definite matrix. And $\Gamma_{j,j} = I(\gamma_j > \gamma_0)$. 
	
	\item Application to eQTL: use posterior probability of inclusion (PPI) to select predictors (SNPs). The method has two properties: (1) It encourages sparsity at the group level (spike-and-slab prior); (2) it encourage dense-within-group sparsity: so if a SNP is chosen, another SNP in high LD may be chosen as well. The results show $>10$ times increase of significant cis-eQTL, but smaller number of genes with at least one cis-eQTL. 
	
	\item \textbf{Lesson}: model the dependency of discrete/binary RVs using a latent MVN distribution. 
\end{itemize}

Bayesian Variable Selection for Binary Outcomes in High Dimensional Genomic Studies Using Non-Local Priors [Nikooieneja \& V.E Johnson, Review for ASA, 2015]
\begin{itemize}
	\item Problem of existing Bayesian variable selection: the problem is that a prior of coefficient $f(\beta)$ that has mode at 0 would be hard to distinguish from 0. Suppose we compare two models $M_1$ and $M_2$ where $M_1$ has $\beta$, but $M_2$ does not. Suppose the variable does not actually influence $y$. Then $P(D|M_1)$ and $P(D|M_2)$ are similar. 
	
	\item Model idea: a model is specified by variables with non-zero coefficients; and we use the prior densities of each variable s.t. the density is 0 at $\beta = 0$. 
	
	\item Non-local prior densities: see Figure 1. Note that the model puts smaller penalty on large coefficients comparing with alternative models (decays quadratically, instead of exponentially).  
	
	\item Model prior: choose a prior form s.t. (1) the same for models with the same number of variables; (2) decreases with more variables; (3) marginal probability of a variable being selected is given by Beta$(a,b)$ for some parameters $a$ and $b$. Choose $a$ at approximately $\log(p)$, and $a + b = p$, where $p$ is the total number of explanatory variables. 
	
	\item Learning hyperparameters: two parameters $r$ and $\tau$. $r$ controls the tail behavior, and $\tau$ is similar to shrinkage parameters - it controls the penalty and determines the minimum value that the regression coefficient must have in order to be selected. The idea of choose $r$: based on how likely we will have very large effects. Choosing $\tau$: use simlulations, control the number of false positive variables to be included. We simulate data under null, and compare the null distribution of MLE of parametes vs. the prior densities. 
\end{itemize}

Regression with Summary Statistics (RSS) [Xiang Zhu and Stephens, 2016]
\begin{itemize}
	\item Goal: from the estimated effect size, its standard error, and LD, learn about the underlying distribution of effect sizes. The idea is that: we  treat the estimated effect size as data. Its mean is given by the true effect size, and its uncertainty by the observed standard error.  
	
	\item Related work: GCTA-COJO, ``Conditional \& joint analysis of GWAS summary statistics without individual level genotype data''. 
	
	\item The scale of $\hat{\beta}$: since we do not have genotype/phenotype data, we do not know the exact scale. But we can assume that $\beta$ is in the scale of log-OR for binary traits, and the effect on phenotypic standard deviation for quantitative traits. 
	
	\item Posterior of effect sizes from summary statistics: our main goal is to determine $P(\beta| \hat{\beta}, S, R)$ where $\beta$ is the effect size, $S$ its standard error (vector) and $R$ the LD (matrix). The definition of $R$ is the LD matrix (this assumes that genotypes are centered): 
	\begin{equation}
	R_{ij} = \frac{X_i^T X_j} {\sqrt{X_i^T X_i} \sqrt{X_j^T X_j}}
	\end{equation}
	To simplify, represent $S$ as a diagonal matrix with diagonal element $s_j$ being the standard error of $\hat{\beta}_j$. The posterior is given by:
	\begin{equation}
	P(\beta| \hat{\beta}, S, R) \propto P(\hat{\beta}|\beta, S, R) P(\beta)
	\end{equation} 
	So we will mainly need to specify the prior and determine $P(\hat{\beta}|\beta, S, R)$, which is similar to the likelihood (distribution of some statistic of data). 
	
	\item Single SNP summary statistics: using the standard results from linear model: 
	\begin{equation}
	\hat{\beta_j} = (X_j^T X_j)^{-1} X_j^T y \quad s_j^2 = (n X_j^T X_j)^{-1} (y - X_j \hat{\beta_j})^T (y - X_j \hat{\beta_j})
	\end{equation}
	We make the assumption that the effect size is small (or total variance explained by each individual loci is small). So we have $s_j^2 = (X_j^T X_j)^{-1} \sigma^2$, where $\sigma^2$ is the variance of $y$. We can also show that by: when $\sigma^2$ is known, $s_j^2 = \sigma^2 / (X_j^T X_j)$, from simple linear regression. 
	
	\item Rewriting $X_j^T X_j$ and $X_j^T X_k$ in terms of $s_j, R$ and $\sigma^2$: because $\hat{\beta}_j$ is expressed as functions of these covariance terms. We have: 
	\begin{equation}
	X_j^T X_j = \frac{\sigma^2}{s_j^2} \qquad X_j^T X_k = \frac{\sigma^2}{s_j s_k} R_{jk}.
	\label{eq:genotype_covar}
	\end{equation}
	
	\item Relationship of $X_j^T X_j$, $R$, $S$ and population standard deviation of genotypes: let $\sigma_{X,j}$ be the standard deviation of SNP $j$ in the population. Define $D = \text{diag}(\sigma_{X,j})$, and $S = \text{diag}(s_j)$. We have $X_j^T X_j = n \sigma_{X,j}^2$, or
	\begin{equation}
	\text{diag}((X_j^T X_j)^{-1}) = \frac{1}{n} D^{-2} \qquad S  = n^{-\frac{1}{2}} \sigma D^{-1} 
	\end{equation}
	See Proposition 2.2. And we can also link covariance matrix of $X$ with $R$ by: 
	\begin{equation}
	\frac{1}{n} X^T X = D R D
	\end{equation}
	See the beginning of Section 2.4. 
	
	\item Summary statistics of SNPs in LD: we can show that 
	\begin{equation}
	\hat{\beta} | \beta, S, R \sim N(SRS^{-1}\beta, SRS)
	\end{equation}
	The mean vector is: 
	\begin{equation}
	\E(\hat{\beta_j}) = s_j \sum_{i=1}^p R_{ij} s_i^{-1} \beta_i
	\label{eq:RSS_mean}
	\end{equation}
	The intuition is that $\hat{\beta_j}$ is the weighted sum of $\beta_i$ where the weight is given by $R_{ij} s_j / s_i$. The covariance of the summary statistics: 
	\begin{equation}
	\Cov(\hat{\beta_j}, \hat{\beta_k}|\beta, S, R)  = s_j s_k R_{jk}
	\end{equation}
	depends on the LD between $j$ and $k$. Or the correlation between the effects of two SNPs is simply $R_{jk}$. When $j=k$, $\Var(\hat{\beta}_j) = s_j^2$. 
	
	\item Proof of RSS likelihood: our idea is in the expression of $\hat{\beta}$, we replace $y$ with $X\beta + \epsilon$. From the equation of $\hat{\beta_j}$, 
	\begin{equation}
	\hat{\beta_j} = (X_j^T X_j)^{-1} X_j^T y = (X_j^T X_j)^{-1} X_j^T X \beta + (X_j^T X_j)^{-1} X_j^T \epsilon
	\end{equation}
	we have, 
	\begin{equation}
	\E(\hat{\beta_j}) = (X_j^T X_j)^{-1} X_j^T X \beta
	\end{equation}
	Plug-in Equation~\ref{eq:genotype_covar}, we have:
	\begin{equation}
	\E(\hat{\beta_j}) = \frac{s_j^2}{\sigma^2} [X_j^T X_1 \cdots X_j^T X_p] [\beta_1 \cdots \beta_p]^T = \frac{s_j^2}{\sigma^2} \sum_{i=1}^p (X_j^T X_i) \beta_i = \frac{s_j^2}{\sigma^2} \sum_i \frac{\sigma^2}{s_j s_i} R_{ji} \beta_i
	\end{equation}
	Now $\sigma^2$ cancels out, and we have Equation~\ref{eq:RSS_mean}. Next we prove the covariance of observed effects, we only need to consider the random terms (those related to $\epsilon$): 
	\begin{equation}
	\Cov(\hat{\beta_j}, \hat{\beta_k}) = \frac{s_j^2}{\sigma^2} \frac{s_k^2}{\sigma^2} \Cov(X_j^T \epsilon, X_k^T \epsilon) = \frac{s_j^2}{\sigma^2} \frac{s_k^2}{\sigma^2} \sigma^2 \cdot X_j^T X_k
	\end{equation}
	where we use the fact that Covariance of $\epsilon$ (vector) is diagonal with diagnoal entry $\sigma^2$. Now we plug in Equation~\ref{eq:genotype_covar}. \\
	Remark: even if $\beta_j$'s are independent, $\hat{\beta}_j$'s are not if SNPs are in LD. This is due to the fact that $X_j$ and $X_k$ are correlated if they are in LD. 
	
	\item Proof of RSS using matrix form: from the expression of $\hat{\beta}_j$, we can write the vector $\hat{\beta}$ in matrix form as:
	\begin{equation}
	\hat{\beta} = \left[ \begin{array}{c}
	(X_1^T X_1)^{-1} X_1^T \\
	\cdots \\
	(X_p^T X_p)^{-1} X_p^T
	\end{array}
	\right]
	X \beta + \left[ \begin{array}{c}
	(X_1^T X_1)^{-1} X_1^T \\
	\cdots \\
	(X_p^T X_p)^{-1} X_p^T
	\end{array}
	\right] \epsilon
	\end{equation} 
	Simplifying this, we have:
	\begin{equation}
	\hat{\beta} = \text{diag}( (X_j^T X_j)^{-1} ) X^T X \beta + \text{diag}( (X_j^T X_j)^{-1} ) X^T \epsilon
	\end{equation}
	where $\epsilon \sim N(0, \sigma^2 I)$ is a random vector. Thus $\beta$ is a linear function of a random vector, and we can derive its mean and variance. 
	\begin{equation}
	\E(\hat{\beta}) = \text{diag}( (X_j^T X_j)^{-1} ) X^T X \beta = \frac{1}{n} D^{-2} X^T X \beta = D^{-1} R D \beta = S R S^{-1} \beta
	\end{equation}
	The variance is given by: 
	\begin{equation}
	\Var(\hat{\beta}) = \Var \left(\frac{1}{n} D^{-2} X^T \epsilon \right) = \frac{1}{n^2} D^{-2} X^T \cdot \sigma^2 I \cdot X D^{-2} = \frac{\sigma^2}{n} D^{-1} R D^{-1} = S R S
	\end{equation}
	
	\item \textbf{Remark}: the key of proof is (1) Write $\hat{\beta}$ in terms of linear function of $\epsilon$, the errors as random vector. (2) Express quantities in the expression $(X_j^T X_j)$ and $X_j^T X$ in terms of $D$, the genotype standard deviation matrix, and $R$. Once we have expression in terms of $D$, we can relate to the standard errors $S$. 
	
	\item Distribution of $Z$ scores: if we define $Z_j = \hat{\beta}_j / s_j$, we can obtain the distribution of $Z_j$'s. Let $Z = S^{-1} \hat{\beta}$, using property of MVN, it is easy to show: 
	\begin{equation}
	Z | S, R, \beta \sim N(RS^{-1} \beta, R)
	\end{equation}
	
	\item RSS model under polygenic prior: suppose we have $\beta_j \sim N(0, \sigma^2)$, we can integrate out $\beta$ analytically. Write in matrix form: 
	\begin{equation}
	\beta | \sigma^2 \sim N(0, \sigma^2 I) \qquad \hat{\beta} | \beta, S, R \sim N(SRS^{-1}\beta, SRS)
	\end{equation}
	Let $M = SRS^{-1}$ we have: 
	\begin{equation}
	\hat{\beta} | \sigma^2, S, R \sim N(0, \sigma^2 M M^T + SRS)
	\end{equation}
	
	\item Priors of $\beta$: first, Bayesian sparse linear mixed model (BSLMM) prior:  
	\begin{equation}
	\beta_j \sim \pi N(0, \sigma_{B}^2 + \sigma_P^2) + (1-\pi) N(0, \sigma_P^2)
	\end{equation}
	where $\pi$ is the fraction or causal variants. This prior would induce sparsity because it will try to fit a Gassuain prior for even non-risk SNPs. Next is adaptive shrinkage (ASH) prior given by: 
	\begin{equation}
	\beta_j \sim \sum_k \omega_k N(0, \sigma_k^2), \quad \omega \sim \text{Dir}(\lambda, \cdots, \lambda)
	\end{equation} 
	Under this prior, we choose a certain number of effect sizes beforehand; but the bad ones will be effectively removed by the model (fit a small $\omega_k$). 
	
	\item Specifying BSLMM prior: similar to the BSLMM model before, we specify the prior using PVE $h$ and PGE $\rho$ (uniform). To relate the effect sizes in RSS with PVE and PGE, we first note the relationship between genotype variance, $\sigma_{x,j}^2$ and the std error in effect size $s_j$: 
	\begin{equation}
	s_j^2 = \frac{\sigma_y^2}{n \sigma_{x,j}^2}
	\end{equation}
	where $\sigma_y^2$ is the residual variance of SNP $j$ - which is effectively the phenotypic variance because of the assumption RSS makes. Next, we have the phenotypic variance explained by the sparse effects: 
	\begin{equation}
	V_B = \sigma_B^2 \pi \sum_j \sigma_{x,j}^2 = \sigma_y^2 \sigma_B^2 \pi \sum_j \frac{1}{n s_j^2}
	\end{equation}
	Similarly, we have the polygenic component: 
	\begin{equation}
	V_P = \sigma_P^2 \sum_j \sigma_{x,j}^2 = \sigma_y^2 \sigma_P^2 \sum_j \frac{1}{n s_j^2}
	\end{equation}
	Following the definitions: $h = (V_B + V_P) / \sigma_y^2$ and $\rho = V_B / (V_B + V_P)$, we can solve $\sigma_B$ and $\sigma_P$ as: 
	\begin{equation}
	\sigma_B^2 = h \rho \left( \pi \sum_j \frac{1}{n s_j^2}\right)^{-1} \qquad \sigma_P^2 = h (1-\rho) \left( \sum_j \frac{1}{n s_j^2}\right)^{-1}
	\end{equation}
	Remark: comparing with BVSR paper, we are using $\sigma_y^2$ as total phenotypic variance, whereas the BVSR paper uses $1/\tau$ as residual variance. 
	
	\item Inference: we parameterize by $\pi, h$. We use MCMC to sample both the parameters and $\gamma$, the indicator variables for all SNPs. The posterior is given by: 
	\begin{equation}
	P(\gamma, \pi, h|\hat{\beta}, S, R) \propto P(\pi) P(h) P(\gamma|\pi) P(\hat{\beta} | S, R, \gamma, \pi, h)
	\end{equation}
	The likelihood conditioned on $\gamma$ has a closed form under BVSR and BSLMM. The proposal distribution of $\gamma$ uses rank-based strategy: SNPs with small single-point $p$-values are sampled with higher probabilities. 
	
	\item Results: 
	\begin{itemize}
		\item Estimation of PVE: when the true PVE is large, an upward bias by RSS. Likely due to the problem with the assumption (each SNP explains a small heritability). 
		
		\item Detecting causal variants: compare BVSR-RSS with BVSR on individual level data, the results are highly correlated, and the AUC is almost the same. 
	\end{itemize}	
	
	\item Related work: one simple idea is that $\hat{\beta}$, $S$, and $R$ all depend on $X^TX$, $X^T y$ and $y^T y$, so we solve $X^T X$, $X^T y$ and $y^T y$ using MOM, then use them in the common linear model framework. The LD score regression approach converts SNP statistics into $\chi^2$ and solve the regression: 
	\begin{equation}
	\E(\chi_j^2|R) = a_0 + a_1 \sum_k {r_{kj}^2}
	\end{equation}
	This is similar to RSS in that the SNP summary statistics is a linear function of the true effects over multiple SNPs in LD, but the error term is not IID. The PAINTOR approach uses $Z$-scores, using non-centrality parameter $\lambda$. Under the alternative model: 
	\begin{equation}
	Z|R, \lambda \sim N(R\lambda, R)
	\end{equation}
	But the semantics of $\lambda$ is not well-defined. 
	
	%	\item Heritability estimation: from the posterior, heritability is related to these parameters: 
	%	\begin{equation}
	%	h = \pi \sigma_{\beta}^2 + \sigma_p^2 \sum_j \frac{1}{n_j s_j^2}
	%	\end{equation}
	%	where $n_j$ is the sample size of the $j$-th locus. 
	
	\item Application in other domains: e.g. image analysis, the observation at one pixel is a linear function of the true ``effect'' at adjacent pixels. One can use RSS kind of analysis to denoise the images. 
	
	\item Remark: 
	\begin{itemize}
		\item For the prior model of $\beta_j$, it is independent, while we expect that within one LD block, there is usually 1-2 causal SNPs. Does this cause any problem? 
		\item The model may not work well for molecular QTL, where effect sizes could be quite large. 
	\end{itemize}
\end{itemize}

A simple new approach to variable selection in regression, with application to genetic fine-mapping (SuSiE) [Wang and Stephens, biorxiv, 2019]
\begin{itemize}
	\item Motivation: in VB approach to BVSR [Carbonette, 2012], the posteriors are independent for each variable/SNP. This does not work well for groups of highly correlated variables. Intuitively, if we can group highly correlated variables, and define posterior on them, then the posterior are roughly independent. Use the ``single effect'' to capture such highly correlated variables. 
	
	\item Credible set: defined on single effects. Different from CAVIAR. 
	
	\item Single effect regression (SER): Equations (2.4) - (2.8). Let $\gamma$ be the indicator vector (1/0 for each variable), the model assumes $\gamma \sim \text{Mult}(1, \pi)$, where $\pi$ is the prior of all variables (vector). Then for the selected variable, its effect $b \sim N(0, \sigma_0^2)$. Posterior under SER: for variable $j$, we compute its BF (using only $X_j$): 
	\begin{equation}
	B_j = \frac{P(y|X_j, \sigma_0^2, \gamma_j = 1)}{ P(y|X_j, \gamma_j = 0)}
	\end{equation}
	The PIP of variable $j$ is then: 
	\begin{equation}
	\alpha_j = \frac{\pi_j B_j}{ \sum_j \pi_j B_j}
	\end{equation}
	The posterior of effect size $b_j | \gamma_j = 1$ would follow normal distribution $N(\mu_{1j}, \sigma_{1j}^2)$, defined as:
	\begin{equation}
	\mu_{1j} = \frac{\sigma_{1j}^2}{s^2} \hat{b}_j \qquad \sigma_{1j}^2 = \left(\frac{1}{s^2} + \frac{1}{\sigma_0^2}\right)^{-1}
	\end{equation}
	Note that the posterior mean is shrinked towards 0: as $\sigma_{1j}^2 > s^2$ because of prior. The posterior can be summarized as $(\alpha, \mu_1, \sigma_1^2)$. The method also needs first and second moment of $b_j$:
	\begin{equation}
	\E(b_j | y, X, \sigma^2, \sigma_0^2) = \alpha_j \mu_{1j} \qquad \E(b_j^2 | y, X, \sigma^2, \sigma_0^2) = \alpha_j (\sigma_{1j}^2 + \mu_{1j}^2)
	\end{equation}
	It is also easy to obtain credible set: just rank variables by PIPs. 
	
	\item EB approach to SER: we can compute the marginal likelihood of $\sigma_{0}^2$ as:
	\begin{equation}
	p(y|X, \sigma_0^2, \sigma^2) = p(y|\sigma^2) \sum_{j=1}^p \pi_j B_j(X_j, y; \sigma^2, \sigma_0^2)
	\end{equation}
	where $\sigma_2$ is the error of $y$. One can then estimate $\sigma_0^2$ and potentially decide if there is a single effect or not. 
	
	\item SuSiE model: suppose we have $L$ effects, we can then write the effect size vector as the sum of $L$ single effect models:
	\begin{equation}
	\mathbf{b} = \sum_{l=1}^L \gamma_l b_l \qquad \gamma_l \sim \text{Mult}(1, \pi) \qquad b_l \sim N(0, \sigma_{0l}^2)
	\end{equation}
	where $\gamma_l$ is a $p$-vector. 
	
	\item Iterative Bayesian Stepwise Regression (IBSS): in the simpler version, not fitting the parameters $\sigma_{0l}^2$. In the $l$-th step, solve SER model for $l$ given other $q$'s. Accounting for other $q$'s by regressing them out (posterior mean) and obtain the residual in the regression model (as response). Results: in iteration $l = 1, \cdots, L$, we obtain $\alpha_l$, $\mu_{1l}$ and $\sigma_{1l}$ (vector for each SNP). However, a SNP would not be selected in multiple steps, so in reality, we should consider only the union of SNPs chosen in multiple steps. 
	
	\item VB inference: approximate the posterior of $\mathbf{b_l}$ as: 
	\begin{equation}
	q(\mathbf{b_1}, \cdots,\mathbf{b_L}) = \prod_l q(\mathbf{b_l})
	\end{equation}
	Formulate a more general model (additive effect model):
	\begin{equation}
	y = \sum_{l=1}^L \mu_l + e \qquad \mu_l \sim g_l(\cdot)
	\end{equation}
	where $g_l(\cdot)$ is the prior of $\mu_l$. For SuSiE, we have $\mu_l = X \mathbf{b}_l$. The posterior approximation is then defined on $\mu_l$'s. The ELBO function is given by Eq (B.6):
	\begin{equation}
	F(\mathbf{q}, \mathbf{g}, \sigma^2; y) = -\frac{n}{2} \log (2 \pi \sigma^2) - \frac{1}{2 \sigma^2} \E_q \norm{y - \sum_l \mu_l}^2 + \sum_l \E_{q_l} \left[ \log \frac{g_l(\mu_l)}{q_l(\mu_l)}\right]
	\end{equation}
	Our goal is to optimize this function over $\mathbf{q}$ and $\mathbf{g}$. To show that the IBSS algorithm optimizes this function, we prove in two steps: (1) Show that coordinate descent leads to optimization at each step, a simpler problem Eq (B.11). The key of this step is to show that the expectation of SSE (sum of square error) over $q_l$ is the SSE where we the explanatory variables are replaced by their posterior mean. (2) Show that the solution of Eq. (B.11) leads to the solution of SER. 
	
	\item Remark: it may be possible to derive the IBSS algorithm using mean field equation, i.e. directly computing $\log q_l(\mu_l)$ by taking expectation of ELBO over $q_i$'s, $i \neq l$.  
	
	\item Estimating hyperparameters $\sigma_{0l}^2$: see Algorithm 4, an optional step before fitting SER model, estimate $\sigma_{0l}^2$ using the EB approach (marginal likelihood). 
	
	\item Determining $L$: the concern is that when $L >$ number of true effects, PIPs may get inflated. Two possible ideas (1) use the size of credible set: if there is no effect, the PIPs would be very diffused, so the credible set becomes large. However, this may not work well for small regions. (2) Use EB estimation of $\sigma_{0l}^2$. In practice, this number is either positive or 0, so it is easy to determine $L$. 
		
	\item Remark: the idea of ``group selection'' with single effect may be applicable to more general settings, e.g. factor analysis (for correlated expression), and HMRF (gene clusters). 
	
	\item Evaluation of SuSiE performance: comparison with DAG, FINEMAP and CAVIAR. Use 574 real genotypes from GTEx. Setting 1: $S$ (number of causal signals) from 1 to 5, and PVE of all variants from 0.05 to 0.4, 1000 SNPs. Setting 2: $S = 10$ and PVE = 0.3, 3,000 to 12,000 SNPs. To run SuSiE: use $L = 10$ in setting 1, and $L = 20$ in setting 2. 
	
	\item Results of simulations: (1) Calibration of PIPs: actual true positives vs. PIPs (Figure S1), most methods are calibrated. (2) Compare PIPs of individual variables between two methods - scatter plot (Figure 2A). Also color the variables by whether it has a true effect (red) or not. Overall correlated, but SuSiE is better: more red dots below the diagonal line. (3) Power vs. FDR (Figure 2B). 
	
	\item SuSiE-RSS: directly work on summary statistics, so use LD matrix that is a sum of reference LD and LD inferred from Z-scores $R' = R + \lambda Z^T Z$, where $Z$ is the $p$-dim. vector of Z scores. If all SNPs are null, then $Z^T Z$ should give the LD in the data. 
	
	\item Remark: SuSiE can be used with prior, which can be derived from e.g. logistic prior model. However, the hyperparameters can have large estimation error. It may be advantageous to accounting for uncertainty of prior/hyper-parameters, similar to Var-BVS. 
	
	\item Remark: about calibration of PIPs. The simulation setting mimics eQTL, the per SNP PVE is very large. So PIP calibration may be OK. In GWAS setting, the ``boundary case'' may lead to inflation. 
\end{itemize}

A fast and flexible Empirical Bayes approach for prediction in multiple regression (MR.ASH) [Youngseok Kim and Stephens, 2020]
\begin{itemize}
	\item Background: prediction accuracy of Lasso is limited by its tendency to overshrink estimates of the large effects. Elastic net: two tuning parameters. MCMC methods (e.g. BSLMM): convergence can be difficult to diagnose. 
	
	\item Background: EB methods, with spike-and-slab prior, CML method: conditions on a single best model (i.e. which predictors have non-zero coefficients) instead of summing over all models as a conventional likelihood. 
	
	\item Idea: similar to Var-BVS with two modifications: (1) In updating $q_j$ of individual variables, use ASH prior. (2) After updating all $q_j$'s, update the ASH prior: for a given component, $k$, $\pi_k$ is just the expected number of data points from $k$. 
	
	\item Model: we have $y = X b + \epsilon$, where $b_j \sim g(\cdot)$ and $\epsilon \sim N(0, \sigma^2)$. We assume that the effect size prior $g(\cdot)$ is scaled by $\sigma^2$, and parameterized by $\pi_k$ for component $k$. Also for simplicity, assume $x_j$ is normalized with variance 1. Let $q_j$ be the VB approximation of posterior of variable $b_j$. It is parameterized by $\phi_{jk} = P(\gamma_j = k | D)$, the probability that $b_j$ is from component $k$, and $\mu_{jk}$ the posterior mean. The VB optimizes:
	\begin{equation}
	\max_{q, g, \sigma^2} F(q, g, \sigma^2)
	\end{equation}
	This is done by Algorithm 1: iteratively update $q_j, 1 \leq j \leq p$ assuming other variables $q_{-j}$ and $g, \sigma^2$ given; and update $g$ and $\sigma^2$ given all $q_j$'s are given. When $g$, $\sigma^2$ and $q_{-j}$ are given, the update for $q_j$ is equivalent to solving the regression problem. Let $\bar{r}_j = y - X_{-j} \bar{b}_{-j}$ be the residual, where $\bar{b}_{-j}$ is the mean of all other coefficients. Then we solve this problem:
	\begin{equation}
	\bar{r}_j = X_j b_j + \epsilon \qquad b_j \sim g(\cdot)
	\end{equation}
	We can view this as Empirical Bayes Normal Mean (EBNM) problem. Consider the OLS estimator of this regression, $\tilde{b}_j = X_j^t \bar{r}_j$ (Note: variance of $X_j$ is 1). Then we have this EBNM problem:
	\begin{equation}
	\tilde{b}_j \sim N(b_j, \sigma^2) \qquad b_j \sim g(\cdot)
	\end{equation}
	This gives the $\phi_{jk} = \phi_k(\tilde{b}_j; g, \sigma^2)$, the component mixture proportions, and $\mu_{jk} = \mu_k(\tilde{b}_j; g, \sigma^2)$, component means. Once $q_j$'s are given, we can update $\pi_k$ in the ASH prior $g(\cdot)$ as: 
	\begin{equation}
	\hat{\pi}_k = \frac{1}{p} \sum_{j=1}^n \phi_{jk}
	\end{equation}
	Finally we can update $\sigma^2$.  
	
	\item Implementation: (1) the actual implementation does not require $X_j$ to have unit variance. (2) The program does not recompute $\bar{r}_j$ at each step: instead, it only computes the residual $\bar{r} = y - X \bar{b}$ once, then at each step, it adds back the variable $j$ being considered. This saves computation time. 

	\item Space and time complexity: running time is $O(n+K)p$ per iteration, and space is $O(n+p)$ (however, need $O(np)$ for storing $X$). 
	
	\item Practical issues: (1) Intercept term: usually centering the data. (2) Initialization: use Lasso (CV) to initialize $b_j$'s. 
	
	\item Connection with penalized regression methods: the penalized regression problem is often optimized by coordinate descent. The update rule is similar to MR.ASH with a different ``shrinkage operator''. With different priors, MR.ASH can mimic various shrinkage methods (Figure 1).   
	
	\item Simulation setting include varying $n$ - sample size. $p$ - number of variables, $s$ - the sparsity level and $h$ - the signal distribution. Baseline setting: $n = 500, p = 2000, s= 20$, PVE $= 0.5$ (used to set error variance) and $h = N(0,1)$. Also vary $X$: independent, or correlated or real genotype. Evaluation metric: RMSE in prediction (scaled by error variance). 
	
	\item Performance evaluation: (1) Several designs (dimensions, signal shape, $X$): show varying levels of $s$ vs. RMSE (Figure 2, 3). Results: Ridge does not work well in sparse signals and Lasso is worst among all other methods. (2) Varying $n$, PVE or $h$: the performance does not vary greatly.
	
	\item Computational efficiency: comparable to Lasso with CV, and much faster than Elastic net. Ex. $n = 500, p = 10000$, about 10s.  
	
	\item \textbf{Remark}: design simulation schemes to answer specific questions. Ex. Ridge is expected to perform worse with sparse signals. SuSiE is most advantagenous when $x_j$'s are correlated. 
		
	\item Remark: one main advantage of MR.ASH is flexible prior, this is however not extensively evaluated.
\end{itemize}

%%%%%%%%%%%%%%%%%%%%%%%%%%%%%%%%%%%%%%%%%%%%%%%%%%%%%%%%%%%%
\section{Extensions of Linear Models}

Statistical Methods with Varying Coefficient Models [Fan \& Zhang, Stat Interface, 2008]: 
\begin{itemize}
	\item Motivation: given a regression problem, there may be considerable heterogeneity in the effect of predictors on response variable, e.g. the effect may change over time or space. Thus it is desirable to relax the constraint that coefficients are constants in regression models. 
	
	\item Model: given data $(U, X, Y)$ where $U$ is some index variable (e.g. time). The model can be written as: 
	\begin{equation}
	Y = \sum_j a_j(U) X_j + \epsilon	
	\end{equation}
	where $a(U)$ is a vector function of $U$, with unknown functional form. 
	
	\item Inference: given the value of $u$, we want to estimate $a(u)$. The objective is: suppose $a(u)$ is the true model, and we apply it to all data points, the loss (error) over all data points. Since we are estimating only at $a(u)$, our model should perform well at points close to $u$, but not have to be good for distant points, thus the loss should be discounted for the distant sample points. The general form of the objective function: 
	\begin{equation}
	L = \sum_i L(y_i, \hat{y}_i) K_h(u_i - u)	
	\end{equation}
	where $\hat{y}_i$ is the predicted value of $y$ at $i$, and $K_h$ is the kernel function. Suppose our model at $u$ is $X^T a(u)$, then at $u_i$, the prediction when we apply the $a(u)$ model, should be: 
	\begin{equation}
	\hat{y}_i = x_i ^T a(u_i) = x_i^T \left[ a(u) + a'(u) (u_i - u)\right]	
	\end{equation}
	Plug in the prediction $y_i$ and use the $L_2$ loss, we have the objective function to be minimized at $u$: 
	\begin{equation}
	L(a,b) = \sum_i \left[y_i - x_i^T a - x_i^T b (u_i - u)\right]^2 K_h(u_i - u)
	\end{equation}
	
	\item Application in longitude analysis: the model can be written as: 
	\begin{equation}
	Y(t) = \beta_0(t) + X(t)^T \beta(t) + \epsilon(t)
	\end{equation}
	In some special models, only $\beta_0$ or $\beta$ is a function of $t$, but not both. 
	
	\item Remark: other strategies of dealing with heterogeneity of effects (coefficients) may include: hierarchical model (the coefficients follow a random distribution) or HMM (the coefficients follow a mixture distribution and switch over the index variable). These strategies, however, impose additional constraints on the data that may not be desired. 
	
	\item Remark: the general idea that underlying heterogeneity there exists some local structure can be exploited in other contexts. Ex. to model DNA evolution, the rate at different positions are different, but the rates at adjacent positions should be similar. 
\end{itemize}

Bayesian factor analysis for testing interactions [Federico Ferrari, interview, 2020]
\begin{itemize}
	\item Data: 56 blood or urine metabolites, 10K samples and outcome (e.g. BMI). Goal: detect interactions of metabolites. 
	
	\item Quadratic regression for modeling interactions: let $y_i$ be health outcome and $x_i$ be exposure, we have:
	\begin{equation}
	y_i = x_i^T \beta_x + x_i^T \Omega_x x_i + \epsilon_i
	\end{equation}
	where $\Omega_x$ is the interaction coefficients. Most methods use sparsity assumptions for interactions. 
	
	\item Motivation: the metabolites are often correlated, and they are products of some chemical agents that are likely the causal factors. Interactions probably happen at the level of causal factors. So the interaction terms may not be sparse. 
	
	\item Model: model exposures as function of factors, and outcome also function of factors, allowing interactions. 
	\begin{equation}
	x_i = \Lambda \eta_i + N(0, \Psi) \qquad y_i = \eta_i^T \omega + \eta_i^T \Omega \eta_i + \epsilon_i
	\end{equation}
	Inference: use Gibbs sampling, sample $\eta_i$ explicitly. Note: cannot marginalize $\eta_i$, because $y_i$ is product of two normal RVs. 
	
	\item Induced regression and interaction: the expectation and variance of $y_i$ have analytic form
	\begin{equation}
	\E(y_i | x_i) = \tr(\Omega V) + (\omega^T A) x_i + x_i^T (A^T \Omega A) x_i
	\end{equation}
	where $V$ and $A$ are some matrices. This helps interpretation of coefficients. 
	
	\item Dealing with missing data: in particular, measures below level of detection (LOD). Use Truncated Normal distribution to impute missing data. 
	
	\item Adding effect modifiers: let $Z_i$ be other covariates, e.g. age. Then we add interaction terms in $y_i$: $\eta_i^T \Delta z_i$. 
	
	\item Using sparse prior for $\Lambda$, the factor to exposure effects. Dirichlet-Laplace prior. 
	
	\item Discussion: improve power over quadratic regression? Model does not need normal assumption of $x_i$, however, need for $z_i$. 
	
\end{itemize}

\subsection{Linear discriminant analysis (LDA)} 

Ref: [Hastie, Section 4.3]

Class density approach: 
\begin{itemize}
\item Class prediction: let $f_k(x)$ be the class density of $k$, and $\pi_k$ be the prior probability of class $k$, then: 
\begin{equation}
P(G=k | X=x) = \frac{f_k(x) \pi_k}{\sum_l f_l(x) \pi_l}	
\end{equation}

\item Discriminant function: for a multi-class problem, functions $\delta_k(x)$ are discriminant functions if any $x$ is classified by the class with the largest value for its discriminant function. A class density approach may be formulated as discriminant function. 
\end{itemize}

LDA with class density: 
\begin{itemize}
\item Model: suppose each class density is multivariate Gaussian, $N(\mu_k, \Sigma_k)$ for class. Then the log likelihood ratio between any two classes is a linear function of $x$ if $\Sigma_k = \Sigma$ for all $k$; or a quadratic function of $x$ if not equal variance. The discriminant function: 
\begin{equation}
\delta_k(x) = x^T \Sigma^{-1} \mu_k - \frac{1}{2} \mu_k^T \Sigma^{-1} \mu_k + \log \pi_k	
\end{equation}
Note: $x$ and $\mu_k$ are considered column vector in this equation. 

\item Inference: the parameters are estimated as: $\hat{\pi_k} = N_k / N$, $\hat{\mu_k}$ as sample mean of class $k$, and $\hat{\Sigma}$ as the pooled estimator of $\Sigma$ using all classes. 

\item Quadratic discriminant function (QDA): not assume equal variance among classes. The discriminant is a quadratic function of $x$. 

\item Compromise bewteen LDA and QDA: this is similar to complete pool vs. no pooling in multi-level problems. A regularized covariance matrix may have the form: 
\begin{equation}
\hat{\Sigma_k(\alpha)} = \alpha \hat{\Sigma_k} + (1-\alpha) \hat{\Sigma}	
\end{equation}
where $\hat{\Sigma}$ is te pooled covariance matrix. 
\end{itemize}

Fisher's LDA (reduced-rank LDA): 
\begin{itemize}
\item Motivation: the class prediction is entirely determined by the (probability) distance to the class centroids. With $K$ classes, the centroids lie in at most $K-1$ dim. hyperplane, thus, the distance comparison can be performed in this hyperplane. This dimensionality-reduction can be achieved by Fisher's LDA.  

\item Fisher's problem: start with the two-class case, the intuition that the points in the positive class are closer to $\mu_+$ than to $\mu_-$ is the same as: most positive examples will occur in one side of the decision boundary, and negative examples in the other side. Consider the direction perpendicular to the decision bounday, then the projection of points in this direction have maximum spread between two classes, relative to the within-class spread. This can be formulated as: find the direction $a$ that maximizes the ratio of between-class variation to the within-class variation:
\begin{equation}
\max_{a,\norm a = 1} \frac{a^T B a}{a^T W a}	
\end{equation}
where $B$ is the between-class covariance matrix, defined as the covariance of the class centroids, and $W$ is the within-class covariance matrix, defined as (in the two class case) ($W_+ + W_-$), the sum of the covariance matrix of positive and negative points, respectively. The variance calculation follows Equation~\ref{eq:projection_variance}. 

\item Solution: The problem is solved by the generalized Rayleigh quotient, and the solution is the eigenvector of the maximum eigenvalue of $W^{-1}B$. All the eigenvectors define a vector subspace containing the variability between features (the smaller eigenvalues can be ignored), and the projections are called discriminant (or canonical) coordinates. Thus the distance comparison/classification of data points can be entirely performed in terms of discriminant coordinates (if only the top eigenvalues considered, then reduced dimension). 

\end{itemize}
	
\subsection{Generalized additive models and structural regression} 

Motivations: 
\begin{itemize}
\item The linear models can be generalized in different ways through, e.g.: 
\begin{itemize}
	\item More general basis functions (but still additive); 
	\item Form of the basis functions: allow higher-order terms, such as multiplication; 
	\item General non-linear functions. 
\end{itemize}

\item Example: predict some property of a sequence, the basis functions could be word counts, or some basic properties of the sequence. 
\end{itemize}

Generalized additive model: 
\begin{itemize}
\item Idea: why a non-linear model can be learned from the data? When all other variables are given, the relationship $Y \sim X_j$ is 1D, and can be approximated through, e.g. splines. 

\item Model: it has the form: 
\begin{equation}
E(Y|X_1, X_2, \cdots, X_p) = \alpha + f_1(X_1) + f_2(X_2) + \cdots + f_p(X_p)
\end{equation}
More generally, the LHS could be $g(\mu(X_1, X_2, \cdots, X_p))$, where $g$ is a link function, such as $\text{logit}$ function. 

\item Fitting additive models: the objective function is minimize error, with penality of non-smooth functions: 
\begin{equation}
\text{PRSS}(\alpha, f_1, \cdots, f_p) = \sum_{i=1}^N \left(y_i - \alpha - \sum_{j=1}^p f_j(x_{ij})\right)^2 + \sum_{j=1}^p \lambda_j \int f_j^{''}(t_j)^2 dt_j
\end{equation}
where the second term penalizes non-smooth functions. Fitting can be achieved in an iterative scheme: suppose all other $\hat{f}_k, k \neq j$ are known, then $\hat{f}_j$ can be estimated through a 1D nonlinear fitting (e.g. cubic spline) of the $j$-th feature and the residual terms (after removing all other variables):  
\begin{equation}
x_{ij} \sim y_i - \hat{\alpha} - \sum_{k \neq j} \hat{f}_k(x_{ik}), 1 \leq i \leq N
\end{equation}

\item \textbf{Remark}: in the problem of learning the effect of one predictor, need to control the other predictors. This can be achieved through regression of the predictor of interest and the residual terms (after applying other predictors). And this can be applied in an iterative fashion. The general idea can be applied in any regression type problems, not just linear models. 
\end{itemize}

Structured regression functions: [Hastie, Section 6.4] 
\begin{itemize}
\item A general form of the structured regression: 
\begin{equation}
f(X_1, X_2, \cdots, X_p) = \alpha + \sum_j g_j(X_j) + \sum_{k<l} g_{kl}(X_k,X_l) + \cdots	
\end{equation}
The additive model assume only the main effect terms (i.e. no higher-order terms). 

\item Varying coefficient model: a special form of structured regression. It generalizes the regression model: the parameters/coefficients are not constant, instead, they depend on the values of (at least some) variables. Let $X_1, X_2, \cdots, X_q$ be the main predictors, and $Z$ be the variables that may influence the effect of these main predictors. We could write regression as: 
\begin{equation}
f(X) = \alpha(Z) + \beta_1(Z) X_1 + \cdots \beta_q(Z) X_q	
\end{equation}

\end{itemize}
%%%%%%%%%%%%%%%%%%%%%%%%%%%%%%%%%%%%%%%%%%%%%%%%%%%%%%%%%%%%
%%%%%%%%%%%%%%%%%%%%%%%%%%%%%%%%%%%%%%%%%%%%%%%%%%%%%%%%%%%%
\chapter{Probabilistic Graphical Model and Causal Inference}
\section{Overview of Causal Inference}

Experimental and observational studies [Pearle, Causality, 2000]: consider an example where we want to test if smoking ($S$) has an causal influence on disease ($D$), e.g. lung cancer, and want to estimate the causal effect. We have three possible experimental stategies: 
\begin{itemize}
	\item Experimental study with control: treatment (smoking) on the same or identical individuals (e.g. twins). Then any difference in the response variable must be due to the treatment. 
	
	\item Experimental study with randomized control: treatment on two groups - case and control, where any confounding variables have been completely randomized in the two groups. Then the confounding variables do not statistically bias case or control group, and any difference is due to treatment.
	
	\item Observational study: the association between explanatory variable and response variable cannot be taken as causation. Ex. association bewteen smoking and disease can be explained by: 
	\begin{itemize}
		\item Disease $\rightarrow$ Smoking (disease affects the smoking habit). 
		\item $\text{Genotype} \rightarrow \text{Smoking}$ and $\text{Genotype}\rightarrow \text{Disease}$, then the same genotype that causes one person to smoking may also predispose the person to the disease. In other words, the sample of individuals with smoking is not completely random: they are biased towards certain genotypes.  
	\end{itemize}
	In general, failure to account for confounding variables would lead to false causality discoveries; and if the regression corrects for variables in the causal path from exposure to outcome (\textit{mediator}), it may cause over-adjustment. 
\end{itemize}

Regression from the graphical model perspective [personal notes]: 
\begin{itemize}
	\item Theorem: given variables $X, Y, Z$, if 
	\begin{equation}
	X \leftarrow Y \rightarrow Z
	\end{equation}
	then $X$ and $Z$ are dependent, but $X \perp Z | Y$. If we have the collider case: 
	\begin{equation}
	X \rightarrow Y \leftarrow Z
	\end{equation}
	Then $X \perp Z$, but $X$ and $Z$ are dependent given $Y$. Intuitively, given $Y$, then large $X$ may mean smaller $Z$. 
	
	\item Motivation: suppose we want to estimate the effect of $X$ on $Y$ via regression. But we have other covariates that may be associated with $Y$, denoted as $Z$. Shall we include $Z$? What happens if we do not include $Z$ (e.g. when $Z$ is actually hidden)? Consider a linear model $y = \beta x + \gamma z$. Ex. $x$ is the genotype, $y$ is phenotype, and $z$ is a possible confounding variable. 
	
	\item $Z$ not associated with $X$ but associated with $Y$ (\textbf{Covariate}): e.g. $z$ is an environmental exposure (e.g. air pollution) in the genetics example, but independent of genotype. Incorporating $Z$ will regress out some variance of $Y$, thus we should include $Z$ to increase the power of detecting $X$ effect. In other words, once $Z$ is included, the total variance of $Y$ (after adjusting for $Z$) is smaller, and the proportion explained by $X$ would be bigger. If $Z$ is not adjusted, it will not create false positives - only losing power. 
	\begin{itemize}
		\item Even if $Z$ does not has a direct effect on $Y$, $Z$ may be a marker of some hidden variables that affect $Y$.  
	\end{itemize}
	Analysis: when $z$ is not included, our estimated $\beta$ is: 
	\begin{equation}
	\hat{\beta} = \frac{\hat{\Cov}(x,y)}{\hat{\Var}(x)}	\approx \frac{\Cov(x, \beta x + \gamma z)}{\Var(x)} = \frac{\beta^2 \Var(x)}{\Var(x)} = \beta^2
	\end{equation}
	We could replace the approximation with expectation in the above argument to be rigorous. Therefore, the estimator is still about right (unbiased). However, it will have a bigger confidence interval: 
	\begin{equation}
	\Var(\hat{\beta}) = \frac{\sum_i e_i^2/(n-2)}{\Var(x)} > \frac{\sum_i \sigma_i^2/(n-2)}{\Var(x)}
	\end{equation}
	where $e_i$ is the residual when only $x$ is included, and $\sigma_i$ is the true redisual. Obviously we have on average $e_i > \sigma_i$ since $e_i$ now is the sum of $\sigma_i$ and $\gamma z_i$. The intuition is that: we will have to explain the missing $z$ through larger error terms, leading to lower power.
	
	\item \textbf{Confounding}: if $Z$ has an effect on $X$, and also independent effect on $Y$, conditioned on $X$,
	\begin{equation}
	X \leftarrow Z \rightarrow Y
	\end{equation}
	we should adjust for $Z$, otherwise false positive association between $X$ and $Y$. Ex. $z$ represents population ancestry (which may be associated with disease through culture, history, sampling bias, etc), then it can create false association. 
	
	\item Covariates affected by $X$: We have two cases: (1) If $X$ has a true effect on $Y$, 
	\begin{equation}
	Y \leftarrow X \rightarrow Z \qquad Z \sim Y
	\end{equation}
	Then adjusting for $Z$ will eliminate some signal of $X$, and is undesirable. This is \textbf{Mediation}. Ex. $z$ represents smoking habit, but is influenced by genetics. (2) In the second case, $X$ has no effect on $Y$, but $Z$ is associated with $Y$, then it may create false positive association. We can call this \textbf{Confounding via Mediation}. Ex. suppose a hidden variable $U$ affects both $Z$ and $Y$: 
	\begin{equation}
	Y \leftarrow U \rightarrow Z \leftarrow X 
	\end{equation}
	Then because $U$ and $X$ are now ``colliders'', and they become dependent conditioned on $Z$. So adjusting for $Z$ will create correlation of $U$ and $Z$, and $U$ would be a confounder. Ex. $Y$ is risk of kidney disease and $Z$ is obesity, $U$ diet (sugar content), then a SNP of obesity may be falsely associated with kidney disease because of confounding variable diet. The optimal strategy in this case may be: (1) Adjusting for $X$ in $Z$, i.e. define $Z' = Z - X \delta$, and then adjusting for $Z'$ when regressing $Y$ on $X$; (2) Joint analysis of $Y$ and $Z$.  
	
	\item Application to genetic association: suppose $X$ is SNP, $Y$ is expression of a gene tested and $Z$ is another gene or PC from expression. $Z$ cannot be a confounder now ($Z$ cannot affect $X$). It is possible that $X$ affects $Z$ (hertiable PC, or SNP is eQTL of $Z$). Generally, we will not adjust for $Z$, or we should adjust for only the part of $Z$ that is independent of $X$. 
	
	\item Multivariate regression (association): suppose we consider $X$ on $Y_1$ and $Y_2$, and $Y_1, Y_2$ are correlated (independent of $X$). Then testing regression of $Y_1$ and $Y_2$ on $X$ independently would not create false positive associations. 
	
	\item Reference: [Stephens, PLoS ONE, 2013], [Aschard, Musical Chairs paper, 2016] 
\end{itemize} 

Unobserved confounding variable in multi-trait association [personal notes]
\begin{itemize}
	\item Problem: suppose we have a QTL $Q$ of a gene $G_1$. Suppose $G_1$ highly correlates with another gene $G_2$. Could it be possible that we'll find $Q$ is also a QTL of $G_2$ because of the correlation? 
	
	\item Three cases: first, the correlation between $G_1$ and $G_2$ is created by an unobeserved variable $Z$: 
	\begin{equation}
	Q \rightarrow G_1 \quad G_1 \leftarrow Z \rightarrow G_2
	\end{equation}
	It is easy to see that $Q \bot G_2$ even if we do not control for $Z$. Essentially, $G_2$ only depends on $Z$, which is independent of $Q$. In the second case, 
	\begin{equation}
	Q \rightarrow G_1 \quad G_1 \rightarrow Z \rightarrow G_2
	\end{equation}
	It is clear that $G_2$ would depend on $Q$ through $Z$. In the third case, 
	\begin{equation}
	Q \rightarrow G_1 \quad G_1 \rightarrow Z \leftarrow G_2
	\end{equation}
	Then $Q$ would also depend on $G_2$; but if we control for $Z$, they will be independent. 
	
	\item Summary: whether the unobserved confounding variable creates problem depends on the nature of the correlation between $G_1$ and $G_2$. 
	
\end{itemize}

Mediation analysis [personal notes]: 
\begin{itemize}
	\item Mediation analysis [Wiki]: we are testing if a variable $M$ mediates the effect of $X$ on $Y$: $X \rightarrow M \rightarrow Y$. The analysis has three steps: 
	\begin{enumerate}
		\item Establish that $X \rightarrow Y$ by regression of $Y$ on $X$. 
		\item Establish that $X \rightarrow M$ by regression of $M$ on $X$. 
		\item Mediation: $Y = \beta_0 + \beta_1 X + \beta_2 M + \epsilon$, test if $\beta_2 \neq 0$ and $\beta_1$ should be smaller than the coefficient in step (1). Mediator should have some independent effect on $Y$ ($\beta_2$ significant), and after including $M$, the effect of $X \rightarrow Y$ should be reduced.  
	\end{enumerate}
	All three steps are needed. Ex. if we have only step 3, the true model could be $X \rightarrow Y \leftarrow M$. Application to genetic association: $X$ is SNP, $M$ and $Y$ are two traits. Then conditions 1 and 2 mean that $X$ needs to be QTL of both $M$ and $Y$. 
		
	\item Sobel test [Wiki]: we test the reduction of $X \rightarrow Y$ effect after controlling for the mediator. Suppose we estimate total effect first: 
	\begin{equation}
	X \xrightarrow{\tau} Y
	\end{equation}
	Next we estimate the mediated effect and the direct effect:  
	\begin{equation}
	X \xrightarrow{\alpha} M \xrightarrow{\beta} Y \qquad X \xrightarrow{\tau'} Y
	\end{equation}
	where the second part is the effect of $X$ on $Y$ not through $M$. Write this as:
	\begin{equation}
	M = X \alpha + \epsilon_M \qquad Y = X \tau' + M \beta + \epsilon_Y
	\end{equation}
	We can see now how $Y$ depends on $X$ by plugging in $M$ equation: 
	\begin{equation}
	Y = X \tau' + (X \alpha + \epsilon_M) \beta + \epsilon_Y = X (\tau' + \alpha \beta) + (\beta \epsilon_M + \epsilon_Y)
	\end{equation}
	We test $\tau - \tau'$, but note that $\tau - \tau' = \alpha \beta$. This says: $\tau$ has two parts, mediated effects $\alpha \beta$ and direct effect $\tau'$. So the test statistics is $\alpha \beta$, and the SE is $\sqrt{\alpha \sigma_{\beta}^2 + \beta \sigma_{\alpha}^2}$. Roughly, the test statistics of $\alpha$ is the regression of $M$ against $X$, and $\beta$ is the regression of $Y$ against $M$ when conditioned on $X$. 
	
	\item Does Mediation imply causality? For simplicity, assume steps 1 and 2 are causal (e.g in the case of genetics when $X$ is SNP). Consider a simple model of independent effect: 
	\begin{equation}
	X \rightarrow M, X \rightarrow Y
	\end{equation}
	Then conditioning on $M$ is as if we condition on part of $X$, so $Y \sim X$ would have reduced effect. The same can be said when there is a confounder $U$ acting on $M$ and $Y$: 
	\begin{equation}
	X \rightarrow M \qquad M \leftarrow U \rightarrow Y \qquad X \rightarrow Y
	\end{equation}
	So in summary, we could have mediation without any causal effect. 
	
	\item Does Mediation tell the direction of causality? Suppose there is a causal effect, of $M \rightarrow Y$ or $Y \rightarrow M$. We can show that this model where $Y$ affects $M$ also show significant mediation of $M$ on $Y$:
	\begin{equation}
	X \rightarrow Y \rightarrow M
	\end{equation}
	It's easy to verify that $M \sim X$ and $Y \sim X$. Additionally, conditioned on $M$ would remove some of the effect of $X$ on $Y$, so $Y \sim X + M$ would have reduced effects. The results would hold if there is $U$ acting on $M$ and $Y$. 
%		
%	The challenge is unobserved confounder $U$. Suppose our causal model is:
%	\begin{equation}
%	X \rightarrow M \rightarrow Y \qquad M \leftarrow U \rightarrow Y \qquad X \rightarrow Y
%	\end{equation}
%	where $M \rightarrow Y$ effect may be 0. The question is whether $\alpha \beta \neq 0$ implies non-zero effect of $M$ on $Y$. Let's assume the effect of $M$ on $Y$ is 0. Our causal model simplifies to:
%	\begin{equation}
%	X \rightarrow M \qquad M \leftarrow U \rightarrow Y \qquad X \rightarrow Y
%	\end{equation}	
%	First it's easy to see $X$ is associated with $M$ and $Y$ (from direct effect). Next we do association of $Y \sim M + X$. The association of $M$ should be non-zero, since there is a path from $M$ to $Y$: $M \leftarrow U \rightarrow Y$, which cannot be adjusted by $X$. Another way to see this is: the association of $X$ to $Y$ is entirely due to $X \rightarrow Y$ direct effect, since the non-causal path has a collider $M$. In the conditional model: $Y \sim \tau' X + \beta M$, we have $\tau' < \tau$, as $M$ explains away some of the association between $X$ and $Y$. This is because $M$ is downstream of $X$, so adjusting $M$ is equivalent to partially adjusting $X$, thus putting both $M$ and $X$ in the same model reduces the effect of $X$ to $Y$. 
	
	\item Interpretation of $\beta$ in Mediation: given the analysis above, we can think of $\beta$ as total association of $M$ with $Y$, through the causal effect and the confounder-induced correlation. With this interpretation: $\tau - \tau' = \alpha \beta$ is always true even when there is confounder. 
	
	%\item Alternative form of mediation: consider the model $G \rightarrow X \rightarrow Y$. Suppose $X = \theta G + \gamma G' + \epsilon_X$, and $Y = \lambda X + \epsilon_Y$, we have: $Y = \lambda \theta G + \lambda \gamma G' + \lambda \epsilon_X + \epsilon_Y$. So mediation analysis would test if conditioned on $G$, the remaining part of $X$ has any effect on $Y$. Under mediation model, we should have for any $G'$ (there could be many), conditioned on $G$, their effect on $Y$ is non-zero. This is the idea of Sherlock. 
	%\begin{itemize}
		%\item Remark: test if all coefficients are non-zero has low power, so we may use hierarchical model, and test if the variance of the effect size is 0. 
	%\end{itemize}
\end{itemize}

Lessons and questions for IV approach [personal notes]
\begin{itemize}
	\item How to combine information from multiple (weak) IVs of the same exposure? 
	
	\item A variable may not be a valid IV for $X$, e.g. it affecting another variable $X'$, which could have a causal effect on $Y$. But can we include $X'$ as well to infer the causal effect of $X$? 
\end{itemize}

%%%%%%%%%%%%%%%%%%%%%%%%%%%%%%%%%%%%%%%%%%%%%%%%%%%%%%%%%%%%
\section{Graphical Models}
\subsection{Directed Graphical Models} 

Reference: [Pearl, Causality, 2000;], [Bishop, Pattern Recognition and Machine Learning, Chapter 10], [Murphy, 2012, Chapter 8]

Modeling with Bayesian networks (BN):
\begin{itemize}
\item Constraints of graph structure: this is often important to limit the model complexity. In addition to some appropriate priors, the common ideas are: mixture models or hierarchical models that group variables. Examples: 
\begin{itemize}
	\item Bi-clustering of gene expression data: the genes and the conditions are clustered into groups, and the expression in the same gene-condition group follows the same distribution.
	\item Module networks: members of the same modules share the parents. 
\end{itemize}
Sometimes, we can introduce additional variables to impose some structure. Ex. if we know $X$ affects $Y_j$'s, and $Y_j$s' are correlated, but we don't know their relationship. We can introduce $U$ as a latent factor, and $X \rightarrow U$ and $U \rightarrow Y_j$. 

\item Application of BN in prediction problems: when the features $X_j$'s have certain known structure, it may be best to model $X_j$'s as well as $Y$ within a BN, treating $Y$ as just another variable while modeling the dependence of features. Examples: 
\begin{itemize}
	\item Prediction of gene interaction: the features are co-expression, PPIs and genetic interaction. However for each type of data, there may be multiple features, which should be highly correlated. Inference could be done through a hierarchical BN [Troyanskaya \& Botstein, PNAS, 2003]. 
	\item Prediction of interacting AA of a protein: the features are AA conservation, hydrophobicity, etc. A naive Bayesian model to infer AA state (interaction or not) [Needham \& Westhead, PLCB, 2007].
\end{itemize}

\item Directionality in modeling: this may not be obvious (corresponding to causal relations), and the BN may allow both directions. This is true, for instance, when some variables are introduced to model variable grouping (as in mixture or hierarchical models), thus having no or ambiguous physical meaning. Example: features of gene promoters ($X$) and gene cluster assignment ($Y$). $X \rightarrow Y$: the promoter determines which cluster a gene belongs to; $Y \rightarrow X$: which cluster a gene belongs to determines the characteristics of promoters (evolutionary consequence). 

\item Interventional vs. observational data: both can be used for learning causal models. However, the interventional data will generally be more informative than observational data in constructing causal networks. This should be taken into account when learning the causal model. Ref: [Sachs \& Nolan, Science, 2005]. 
\end{itemize}

Conditional independence (CI) [Wiki]: 
\begin{itemize}
\item Motivation: capture the structure of data (Occum's razor). Ex. the future does not depend on the past given the future, this is modeled by a Markov chain. 

\item Intuition: the relation, $X \bot Y | Z$, can be understood from different perspective. Take an example of working in a chemical factory ($X$) and cancer ($Y$), where $Z$ indicates chemical exposure. 
\begin{itemize}
\item Stratification: given $Z$, i.e. if the data are stratified by $Z$, then $X$ and $Y$ are independent. This means that for people with the same chemical exposure, then whether working for the chemical factory is not related to cancer risk. 
\item Explanation of correlation: the variables $X$ and $Y$ are correlated, but the correlation is caused by their common relationship with $Z$; if we take $Z$ out, the correlation between $X$ and $Y$ will be explained away. In our example, the correlation of working status and cancer risk is explained away by the chemical exposure. 
\end{itemize}

\item Properties of CI: we ignore the condition for simplicity of notations. 
\begin{itemize}
\item Symmetry: $X \independent Y \Leftrightarrow Y \independent X$. 
\item Decomposition: if $X \independent A, B$, then $X \independent A$ and $X \independent B$. 
\item Contraction: if $X \independent A |B$, and $X \independent B$, then $X \independent A,B$. The intuition is if $X \independent A|B$, but $B$ is independent of $X$, thus cannot explain the relation between $X$ and $A$, thus we must have $X$ independent of $A$. 
\item Intersection: if the probabilities of $X$, $A$ and $B$ are all positive, then $X \independent A|B$ and $X \independent B |A$ implies that $X \independent A, B$. 
\end{itemize}
\end{itemize}

Markov model: 
\begin{itemize}
\item Definition: a probability distribution defined on a directed acyclic graph (DAG) is Markovian if: 
\begin{equation}
P(v) = \prod_i P(x_i | \pi(x_i))	
\end{equation}
where $x_i$ is a node and $\pi(x_i)$ is the parent of $x_i$. 

\item Markov condition: the factorization and the joint condition is equivalent to the following Markov condition: for every variable $W$,
\begin{equation}
W \bot \tilde{W} | \pi(W)	
\end{equation}
where $\tilde{W}$ denotes all the other variables except the parents and descendants of $W$ (it may include co-parents of $W$).\\
Example: $X \rightarrow Y \rightarrow Z$, then it is easy to prove that $p(x,z|y) = p(x|y) p(z|y)$ from the factorization of $p(x,y,z)$. 

\item Graphical notation for representing models: (1) multiple independent RVs: by plate; (2) deterministic parameters: small solid circles; (3) hidden and observed variables: different colors or open/solid state. 

\item Causal interpretation of a graphical model: a model only represents a factorization of joint probability distribution, thus may not correspond to causality at all. Ex. the model $X \rightarrow Y \rightarrow Z$ can also be written as: $Z \rightarrow Y \rightarrow X \leftarrow Z$. Therefore, it is the interpretation and the attempt at modeling that gives a model the causal semantics. 
\end{itemize}

Directed Gaussian graphical model (Directed GGM): 
\begin{itemize}
\item Model: conditional distribution of a node, given its parents, follow the linear model. Suppose $t$ is a node, we have:
\begin{equation}
x_t - \mu_t = \sum_{s \in \pi(t)} w_{ts} (x_s - \mu_s) + \sigma_t \cdot z_t
\end{equation}
where $w_{ts}$ is the weight coefficient of $s \rightarrow t$ edge, and $z_t \sim N(0,1)$ is the error term. We could write the relation in matrix form: 
\begin{equation}
x - \mu = W (x- \mu) + Sz	
\end{equation}
where $S$ is the diagonal matrix with terms $\sigma_t$. We define $e = Sz$ as the error term for all variables. 

\item Joint distribution: we write the above relation as: 
\begin{equation}
x - \mu = (I - W)^{-1} e	= USz
\end{equation}
where $U = (I - W)^{-1}$. Thus the joint distribution is normal, with mean equal to $\mu$, and the covariance matrix: 
\begin{equation}
\Sigma = \Cov(x-\mu) = \Cov(USz) = US \Cov(z) SU^T = U S^2 U^T 
\end{equation}

\end{itemize}

Conditional independence in Markov model: 
\begin{itemize}
\item The semantics of a Markov model lies in the conditional independence (CI): it is the lack of arrows (conditional independence) that makes a certain model specific. Ex. for the model $X \rightarrow Y \rightarrow Z$, we can factorize the joint distribution in different ways, e.g. $P(Z) P(Y|Z) P(X|Y,Z)$, but then $X$ will depend on both $Y$ and $Z$, no CI. 

\item Simple examples of $d$-separation and $d$-connection (not $d$-separation): 
\begin{itemize}
	\item Non-collider: including chains and forks. E.g. $X \rightarrow Y \rightarrow Z$, or $X \leftarrow Y \rightarrow Z$, $X$ and $Z$ are $d$-connected, but $d$-separated given $Y$. 
	\item Collider: e.g. $X \rightarrow Y \leftarrow Z$, $X$ and $Z$ are $d$-separated, but $d$-connected given $Y$. Also note that for collider, $X \bot Z$, and this independence is true if conditioned on, e.g. predecessor of $X$ or $Z$. Ex. $X \rightarrow Y \rightarrow Z \leftarrow W$, then we have $Y \bot W | X$. 
	\item Collider descendant: in $X \rightarrow Y \leftarrow Z$, also $Y \rightarrow W$, the same relation holds: $X$ and $Z$ are $d$-separated, but $d$-connected given $Y$ or $W$. Thus a descendant of a collider has the same role in determining CI. 
\end{itemize}

\item Rules of $d$-separation and $d$-connection: $X$ and $Y$ are $d$-separated by a set of nodes $S$ if $S$ blocks all paths from $X$ to $Y$ (path is undirected). $S$ blocks a path $p$ if either of the two conditions satisfy:  
\begin{itemize}
	\item $p$ contains a chain ($i \rightarrow m \rightarrow j$) or a fork ($i \leftarrow m \rightarrow j$) s.t. the middle node $m$ is in $S$. 
	\item $p$ contains a collider ($i \rightarrow m \leftarrow j$) s.t. the middle node $m$ as well as its descendant is not in $S$ (they will make $i$ and $j$ $d$-connected). Note that a node $Z$ may block a path between $X$ and $Y$, even if $Z$ lies outside the path, due to the fact that colliders (alone, without conditioning on the middle node) create $d$-separation (thus conditioning on some variable outside the path is equivalent to not conditioning). 
\end{itemize}
Example: $X \rightarrow W \leftarrow V \rightarrow Y \leftarrow Z$. $X$ and $Y$ are $d$-separated by $V$ (from rule 1) and $Z$ (from rule 2), but not by $W$. 

\item Theorem: let $A$, $B$ and $C$ be disjoint sets of vertices, then $A \bot B | C$ iff $A$ and $B$ are $d$-separated by $C$. 

\item Remark: this is closely related to backdoor criterion for adjusting in confounding. If we adjust all variables s.t. $X$ and $Y$ are conditionally independent (ignoring $X \rightarrow Y$ edge), then regression of $Y$ on $X$ estimates the causal effect. 

\item Markov blanket: for a given variable $x_i$, its Markov blanket is the set of variables $Y$, s.t. 
\begin{equation}
P(x_i|x_{-i}) = P(x_i |Y)	
\end{equation}
where $x_{-i}$ denotes all other variables in the graph. In other words, the Markov blanket isolates a variable from the rest of graph. It comprises the set of parents, children and co-parents (other parents of its children) of the node: 
\begin{itemize}
	\item Children need to be included: parent of a variable $X$ does not block the path from $X$ to its children. 
	\item Co-parents: a co-parent and $X$ are $d$-connected because the common child is a collider. So if we include the child, we must include its other parent(s). 
\end{itemize}
The probability $P(x_i | x_{-i})$ is called the \textit{full conditional}. 

\end{itemize}

Markov equivalence: 
\begin{itemize}
\item Definition: two graphs with the same set of CI relations. 

\item Condition of Markov equivalence: Two DAGs $G_1$ and $G_2$ are Markov equivalent iff $\text{skeleton}(G_1) = \text{skeleton}(G_2)$ (skeleton: undirected graph by replacing all arrows with undirected edges) and $G_1$ and $G_2$ have the same set of $v$-structures, that is, two colliders whose tails are not connected by an arrow. Thus the structure of a graphical model can be uniquely determined only up to Markov equivalent class.  

\item To assess Markov Equivalence, we only need to see if collider structure are identical. Example: consider the standard MR model, with $G \rightarrow X \rightarrow Y$ and $U$ is a confounder. Consider the model where $U$ is a mediator of $X$ to $Y$. In the first case, $X$ is a collider, but in the second, $X$ is not. 

\item Directionality of edge: can be identified only if changing the direction will change the collider structure. Example: in the model $X \rightarrow Y \rightarrow Z$, the direction of $Y \rightarrow Z$ can be determined, but not the first edge. 
\end{itemize}

Inference of graphical models [Murphy, Chapter 20]
\begin{itemize}
	\item Inference vs. learning: inference is about estimating hidden variables.
	
	\item Learning (parameter estimation) from complete data [Murphy, 10.4]: MLE or MAP. The likelihood can be factorized into distributions of a node and its parents. With discrete RV for each node (multinomial distribution), the parameters can be analytically determined: posterior still independent, following Dirichlet distribution.
	
	\item Forward-backward algorithm for HMM: as message passing algorithm. $P(z_t | x[1..T])$, two parts, belief of $z_t$ using past data up to $t$, and conditional likelihood of future $x[t+1 .. T]$, written as $P(x[t+1 .. T]|z_t)$. Forward pass: update the believe from left to right; backward pass: using likelihood information from right to left.
	
	\item Belief propagation (BP) in tree: from root to leaf (forward) and leaf to root (backward).
	
	\item Variable elimination (VE) algorithm: pushing sums into products. A version is Peeling algorithm in genealogy trees.
\end{itemize}

\subsection{Tree Model}

Reference: [Ronen, Parameter Estimation of Dependence Tree Models Using the EM Algorithm, 1995; Kazemian \& Sinha, Quantitative analysis of the Drosophila segmentation regulatory network using pattern generating potentials, PLos Bio, 2010]

Tree model: 
\begin{itemize}
\item Likelihood: given $n$ RV's related by a tree, $\{X_1, \cdots, X_n \}$, where $X_1$ is the root node. Let $t_i$ be the length of the branch leading to $X_i$. Also assume that $X_1$ has a uniform distribution (prior). For a node $i$, we denote $\pi(i)$ as the parent of $i$, and $C(i)$ as the set of child nodes of $i$. The complete likelihood of the model is defined by: 
\begin{equation}
P(X) = P(X_1) \prod_{i=2}^n P(X_i|X_{\pi(i)})	
\end{equation}

\item Model with missing variables: assume that only the variables in the leaf nodes are observed (e.g. in phylogenetic analysis). We use $O_i$ to denote the observations at the subtree rooted at $i$, thus $O_1$ is our data. We are interested in two problems: (1) parameter estimation; and (2) estimation of the missing variables (internal nodes). For the former, let $\theta$ be model parameters, we compute the likelihood $P(O_1|\theta)$; and for the latter, we need to compute $P(X_i = m | O_1)$. 
\end{itemize}

Upward/downward algorithm: 
\begin{itemize}

\item Recurrence variables for the likelihood and missing variables: we have the decomposition for the likelihood: 
\begin{equation}
L(\theta)	= P(O_1|\theta) = \sum_m P(X_1 = m|\theta) \cdot P(O_1|X_1 = m,\theta)
\end{equation}
We also have the decomposition for the latent varialbes: 
\begin{equation}
P(X_i = m | O_1) = \frac{P(X_i = m, O_1)}{P(O_1)} = \frac{P(O_i | X_i = m) P(X_i = m, O_{1 \backslash i})}{P(O_1)} 	
\label{eq:tree_missing_var}
\end{equation}
where we use the notation $O_{i \backslash j}$ to denote the leaf nodes that are part of the subtree rooted at $i$, but not part of the subtree rooted at $j$. The two equations suggest that we need the following two types of variables: 
\begin{equation}
\beta_i(m) = P(O_i | X_i = m)	
\end{equation}
\begin{equation}
\alpha_i(m) = P(X_i = m, O_{1 \backslash i})	
\end{equation}
Using these recurrence variables, we have: $L(\theta) = \sum_i \beta_1(m)$ and
\begin{equation}
P(X_i = m | O_1) = \frac{\alpha_i(m) \beta_i(m)}{ \sum_{m'}\alpha_i(m') \beta_i(m') }
\end{equation}
We could compute the two types of variables using the upward-downward algorithm.   

\item Upward algorithm: this computes the upward variable $\beta_i(m)$. If $i$ is the leaf node, we have $\beta_i(m) = \delta_{m,x_i}$ (delta-function). If $i$ is an internal node, $\beta_i(m)$ can be decomposed using the child nodes of $i$: 
\begin{equation}
\beta_i(m) = \prod_{j \in C(i)} P(O_j | X_i = m) 
\end{equation}
We define $\gamma_i(m) = P(O_i | X_{\pi(i)} = m)$, thus $\beta_i(m)$ is a product of $\gamma_j(m)$ over the child nodes of $i$. We next consider the recurence of $\gamma_i(m)$, if $i$ is a leaf node: 
\begin{equation}
\gamma_i(m) = P(m \rightarrow x_i|t_i) \beta_i(x_i)
\end{equation}
If $i$ is an internal node, we have the recurrence: 
\begin{equation}
\gamma_i(m) = \sum_{m'} P(m \rightarrow m'|t_i) \beta_i(m')
\end{equation}
To implement the algorithm, we first compute the recurrence at the leaf nodes, then move up the tree. 

\item Downward algorithm: this computes the downward variable $\alpha_i(m)$. If $i$ is the root node, we have $\alpha_1(m) = P(X_1 = m)$, this is given by the prior distribution. If $i$ is an internal node, we have the recurrence: 
\begin{equation}
\alpha_i(m) = \sum_{m'} \alpha_{\pi(i)}(m') P(m' \rightarrow m|t_i) P(O_{\pi(i)\backslash i} | X_{\pi(i)} = m')
\end{equation}
For a binary tree, we have $O_{\pi(i)\backslash i} = O_{\text{Sib}(i)}$, where $\text{Sib}(i)$ is the sibling of $i$. We can then write the recurrence: 
\begin{equation}
\alpha_i(m) = \sum_{m'} \alpha_{\pi(i)}(m') P(m' \rightarrow m|t_i) \gamma_{\text{Sib}(i)}(m')
\end{equation}

\end{itemize}

A Spectral Algorithm for Latent Tree Graphical Models [ICML, 2011]
\begin{itemize}
	\item Motivation: in the graphical models, if there are latent variables, need to do EM to do parameter estimation, which is local optima. 
	
	\item Idea: do transformation s.t. the inference is based on transformed variables on observable data only. Suppose $O$ is the observed vector, $R$ is the root (hidden), then $P(O)$ (a vector) can be written as a product of $P(O|R)$ (a matrix) and $P(R)$ (a vector). In general, this is the message passing equation. 
\end{itemize}

\subsection{Markov Random Fields (MRF)} 

Reference: [Hastie, Chapter 17; Bishop, Section 8.3]

MRF idea: 
\begin{itemize}
\item Motivation: need a flexible way of modeling probability distribution of multiple random variables with certain dependencies. Ex. we may know that two RVs are correlated, but not the exact order of causality of all RVs (thus directed graphical models are not applicable). 
\item A general strategy of modeling probability distribution: simply define the energy of a system of multiple RVs, and the probability distribution follows Boltzmann distribution. How the energy of the system is defined encodes our knowledge/belief of the likely states of the system. 
\item Examples: (1) Spatial data: where the variable at a grid should be similar to the value of its adjacent grids. Model this as an energy function that rewards neighboring interaction/correlation. (2) Network data: where two linked nodes tend to be similar (e.g. social network, PPI network). Model this as an energy function that rewards correlated network neighbors. 
\end{itemize}

MRF model: 
\begin{itemize}
\item Motivation: Ising model. Consider a lattice, and each site in the lattice has a spin state ($+1$ or $-1$), $\sigma$ is the assignment of spin states of all sites. The energy of the system is given by: 
\begin{equation}
H(\sigma) = - \Sigma_{i,j} J_{ij} \sigma_i \sigma_j - \Sigma_j h_j \sigma_j
\end{equation}
where the first sum is over pairs of adjacent spins (every pair is counted once) and $h_j$ is the external field at $j$. The probability of $\sigma$ is then given by the Boltzmann distribution. The interesting statistical questions to ask are all in the limit of large numbers of spins:
\begin{itemize}
	\item In a typical configuration, are most of the spins $+1$ or $-1$, or are they split equally?
	\item If a spin at any given position $i$ is 1, what is the probability that the spin at position $j$ is also 1?
	\item If $\beta$ (temperature) is changed, is there a phase transition?
\end{itemize}

\item Probability distribution: suppose a potential function is defined for every clique of the graph (some type of mutual interaction), then the joint distribution: 
\begin{equation}
f(x) = \frac{1}{Z}\prod_C \Psi(x_C)	
\end{equation}
where $C$ is over all maximum cliques of the graph, $x_C$ is the RVs at $C$ and $\Psi(x_C)$ is the potential function of $x_C$. $Z$ is the partition function, defined as: 
\begin{equation}
Z = \sum_x \prod_C \Psi(x_C)	
\end{equation}
It is common to parameterize $\Psi(x_C)$ with an energy term, $E(x_C)$, using Boltzmann distribution: 
\begin{equation}
\Psi(x_C) = \exp[-E(x_C)]	
\end{equation}

\item Conditional independence: three sets of nodes $A,B,C$, we have: if $C$ separates $A$ and $B$, then $A \bot B | C$. CI allows one to decompose a graph into maximal cliques, thus the factorization above can also be expressed as a set of CI statements. 

\item Pairwse Markov graph: a potential function for each edge, and at most second-order (pairs) interactions are represented. This has the benefit of fewer parameters and easier to work with. Ex. for a three-node complete graph, we simply have: 
\begin{equation}
f(x,y,z) = \frac{1}{Z} \Psi(x,y) \Psi(y,z) \Psi(x,z)	
\end{equation}
\end{itemize}

Continuous MRF (Gaussian MRF): multivariate normal distribution
\begin{itemize}
\item Idea: suppose the set of RVs can be treated as multivariate normal distribution, then an associated pairwise Markov graph encodes the additional constraints of the covariance structure of these RVs. This would allow one to better estimate this distribution if the graph is known; or learn a simpler model if the graph is not known. 

\item Background: let $\Theta = \Sigma^{-1}$ be the precision matrix, if the $ij$ component of $\Theta$ is 0, then $i$ and $j$ are CI given the other variables. 

\item Learning the covariance matrix given the associated pairwise Makrov graph. Suppose the potential function of the state $x$ is given by:
\begin{equation}
\Psi(x) = \sum_i \theta_{ii} x_i^2 + \sum_{i \neq j} \theta_{ij} x_i x_j = x^T \Theta x
\end{equation} 
So the pdf is given by: $P(x) \propto \exp(- x^T \Theta x)$. When $x$ is centered, we see that this is just the pdf of MVN, with $\Theta = \Sigma^{-1} / 2$. The log-likelihood function is: 
\begin{equation}
l(\Theta) = \log \det \Theta - \text{tr}(S\Theta)	
\end{equation}
where $S$ is the sample covariance matrix. The maximization is subject to the equality constraint that $\Theta_{ij} = 0$ if $i$ and $j$ are not linked in the Markov graph. It can be shown that the parameter estimation can be done through a series of regression of the variable $i$ on $j$, conditioned 

\item Learning the Markov graph: when the graph is unknown, then effectively we want to learn a model with possibly as few edges as possible. This can be done by maximizing the penalized log-likelihood:
\begin{equation}
\log \det \Theta - \text{tr}(S\Theta)	- \lambda \norm{\Theta}_1	
\end{equation}
where $\norm{\Theta}_1$ is the $L_1$ norm. One can adapt the lasso to solve this problem. 

\end{itemize}

Graphical lasso: [Sparse inverse covariance estimation with the graphical lasso]
\begin{itemize}
	\item Motivation: to estimate a multivarate normal distribution, the correlation structure should be relatively sparse, specifically, most variables should be conditionally independent (given all other variables). This translates to: the number of non-zero terms in the precision matrix should be small. 
	
	\item Model: let $\Theta = \Sigma^{-1}$ be the precision matrix (non-negative definite), and $S$ be the sample covariance matrix, our goal is to maximize the penalized log-likelihood: 
	\begin{equation}
	\log \det \Theta - \tr(S \Theta) - \rho \norm{\Theta}_1
	\end{equation}
	where $\norm{\Theta}_1$ is the $L_1$ norm of the matrix: the sum of the absolute values of the elements of $\Theta$. 
	
	\item Optimization: block coordinate descent algorithm. One can show that to solve the conditional maximization problem is equivalent to the dual problem, which resembles a lasso regression (least square with $L_1$ constraint). 
	
	\item The issues to be solved/proved: 
	\begin{itemize}
		\item The interpretation of the elements of $\Sigma^{-1}$: conditional independence of variables. 
		\item Convexity of the optimization problem: prove the convexity in any line restriction.  
		\item Optimization: block coordinate descent and the dual problem. 
	\end{itemize}
	
	\item Remark: penalized log-likelihood method, where the penalty is based on the non-zero coefficients. 
\end{itemize}

Disrete MRF: most common when the variables are binary. Also called Ising model, Boltzmann machine. 
\begin{itemize}
\item Potential function: each edge of the variables $X_j$ and $X_k$ has the energy, $-\theta_{jk} X_j X_k$, which means favorable interaction when $X_j = X_k$. This leads to the probability of a state $X$: 
\begin{equation}
p(X) = \frac{1}{Z} \exp\left[\sum_{(j,k) \in E} \theta_{jk} X_j X_k\right]	
\end{equation}

\item Application: Image de-noising. Let $x_i$ be the value of the $i$-th grid in the true image (hidden variables), and $y_i$ be the corresponding variable in the noisy image (data). Then the solution $x$ should satisfy that: $x$ generally has similar values in the neighboring grids; and $x_i$ should be generally equal to $y_i$. Also we could favor $x_i$ to certain class (e.g. more $+1$ over $-1$). This can be expressed as the energy function: 
\begin{equation}
E(x,y) = h \sum_i x_i - \beta \sum_{i,j} x_i x_j - \eta \sum_{i} x_i y_i
\end{equation}
where $i$ and $j$ are neighboring grids. The goal is to find/sample $x_i$ according to the distribution specified by this energy function ((effectively minimize the energy). 

\end{itemize}

Remark: The alignment problem can serve as a general strategy for capturing the similarity/matching structure in a problem, and this can be used for finding the hidden variables (see examples below). 
\begin{itemize}
\item The alignment problem is often charcterized by: the missing variables/alignment should (1) the counterparts of different objects should match each other; (2) the missing variables should be consistent with their ``neighbors'' within each object. 

\item In particular, MRF is a general modeling framework for alignment problems; it is more general than directed models, e.g. for sequence alignment, one may have the situation where some sequences are similar but no direction (temporal) is known (example below). 
\end{itemize}

Some examples of MRF:
\begin{itemize}
\item Finding missing sequences: suppose we have multiple sequences from a group, each with part of sequence missing/error. Suppose the sequences are related to each other, e.g. some are from the same family, some from the same region, and we have a matrix characterizing mutual relationship of sequences. Then we have a generative model of each sequence (such as HMM), but also considering the similarity between sequences. 

\item Markov random topic models [Daume, ACL, 2009]: suppose documents are related to each other through a graph $G$, the topic proportions of documents should be consistent with the graph, i.e. the neighbors should have similar topics. Define a MRF on topic proportions $\theta$, the potential of a pair of documents $d, d'$ is $\psi(\theta_d, \theta_{d'}) = \exp[-l_{d,d'} \rho(\theta,\theta')]$, where $l_{d,d'}$ measures the strengh of link, and $\rho$ is a distance function. 

\item Machine translation (word alignment): suppose given a sentence in $L_1$, we want to find the best alignment of the words in $L_2$. This alignment should: (1) conform to the syntax/semantics of $L_2$; (2) match the sentence in $L_1$. 
\end{itemize}

Time-varying Indian Buffer Process [A Dynamic Relational Infinite Feature Model for Longitudinal Social Networks, AI-STAT, 2011]
\begin{itemize}
\item Idea: modeling the change of IBP over time. Define latent variables ($Z$) and weight matrix $W$ that models dependence of $Z$. Only the latent variables change over time, while $W$ remains constant. 
\end{itemize}

Learning Scale Free Networks by Reweighted L1 regularization [AI-STAT, 2011]:
\begin{itemize}
\item Favoring scale-free networks: by regularizing parameters. 

\item Optimization: minorize-maximization (MM) algorithm, which is a generalization of EM. The idea is for a non-convex function to be max'ed, approximate with a series of convex function, which provides a lower-bound of the max at each step. 
\end{itemize}
%%%%%%%%%%%%%%%%%%%%%%%%%%%%%%%%%%%%%%%%%%%%%%%%%%%%%%%%%%%%
\subsection{Graphical Model Structure Learning}

Graphical model structure learning [Murphy, Chapter 26]
\begin{itemize}
	\item Heuristic approach: relevance network with MI. Dependency network: use sparse regression for full conditional (one variable a time).
	
	\item Learning tree structure: equivalence of directed and undirected trees (26.3.1). It is easier to use undirected tree (simple factorization of probability). 
	
	\item ML tree structure: Chow-Liu algorithm, the likelihood function is the score of a tree. Finding maximum weight spanning tree.
	
	\item Mixture of tree model for general graph: to apply the tree algorithm to general graph. Remark: standard MR causal diagram is not a tree.
	
	\item Learning DAG structure with complete data: one can only find graphs with Markov equivalence. The results are PDAG. For a graph with no missing variable, we marginalize parameters:
	\begin{equation}
	P(D|G) = \int P(D|G, \theta) P(\theta) d\theta
	\end{equation}
	This will produce simpler models than ML (which would give complete graph). The model requires hyperparameters. 
	
	\item Example: college plan network. Figure 26.7, college decision depends on sex, IQ, SES, and parental encouragement (PE). Learns the graph from data: one model is much preferred than the rest with large BF. 
	
	\item Learning DAG with latent variables: BIC often gives overly simplified model. A better approach is VB-EM: let $z$ represent hiden variables
	\begin{equation}
	p(\theta, z_{1:N} | D) \approx q(\theta) \prod_i q(z_i)
	\end{equation}
	
	\item Example: college plan network with hidden variable - a common cause of SES and IQ. This model has much better BF. 
	
	\item Discovering hidden variables: use latent variables to explain dense clusters of variables. Could also extend to hierarchical latent variable model. Example: Google’s Rephil, to fit word-document model, several levels of latent variables
	
	\item SEM: all the relationships are linear, let $w_{ij}$ be the effect of $x_j$ on $x_i$, then we have:
	\begin{equation}
	x_i = \mu_i + \sum_{j \neq i} w_{ij} x_j + \epsilon_i
	\end{equation}
	where $\epsilon \sim N(0, \Psi)$. The model can be written in the matrix form: 
	\begin{equation}
	x = W x + \mu + \epsilon \Rightarrow x = (I-W)^{-1} (\mu + \epsilon)
	\end{equation}
	The joint distribution of $x$ is thus given by $x \sim N(\mu, \Sigma)$, where 
	\begin{equation}
	\Sigma = (I-W)^{-1} \Psi (I-W)^{-T}
	\end{equation}
	Note that in SEM, the graph can be cyclic. 
\end{itemize}

Learning causal DAG structures [Murphy, 26.6]
\begin{itemize}
	\item Example: Treatment $\rightarrow$ Effects, Gender as a confounder, or Blood pressure as mediator.
	
	\item Learning from observational data: up to PDAG (Markov equivalence class).
	
	\item Analysis: need assumptions of no missing data. Ex. given two nodes, $X \rightarrow Y$ and $X \leftarrow Y$ are Markov equivalent, however, the true model may be $U$ affects both.
	
	\item Analysis: often we include hidden variables to have a DAG with known structure, and only estimate parameters. Inference of PDAG with missing data may be difficult/un-identifiable. Special case: if we can bound some effects of hidden variables, we may be able to infer causal effects. Smoking-gene example.
	
	\item Learning from interventional data: (1) Do graph surgery in interventional data. (2) Inference.
\end{itemize}
%%%%%%%%%%%%%%%%%%%%%%%%%%%%%%%%%%%%%%%%%%%%%%%%%%%%%%%%%%%%
\section{The Book of Why [Judea Pearl]}

Chapter 1: The ladders of causation
\begin{itemize}
	\item Ladder 1: association, $P(Y|X)$. 
	
	\item Ladder 2: causation, $P(Y|do(X))$. Ex. what is the chance we will sell $Y$ if we set the price of $X$? 
	
	\item Ladder 3: counterfactual, imagined world, the reason for observed events. Ex. what is the probability that a customer who bought toothpaste would still have bought it if we had doubled the price? 
	
	\item Firing squad example: causal diagram (Figure 1.4) with Court Order to Captain to two soldiers A and B to death $D$. Intervention: A decides to shoot, we set $A = $ true, and remove all edges pointing to $A$. Counterfactual: suppose we have seen $D$, we ask if $A$ is responsible. We imagine the world where $A = $ false and remove all edges pointing to $A$. Given Court Order is true, we conclude that $D$ will still be true, so A should not be responsible.
	
	\item Small pox vs. vaccine: after vaccination, more people died from innoculation than small pox. Can we conclude that we should ban vaccine? Counterfactual: how many people will die if we set vaccination rate to be 0 percent? Use causal diagram to answer. 
	
	\item \textbf{Lesson}: to answer counterfactual questions, what would happen to $Y$ if some condition about $X$ had happened? We set $X$ in the causal diagram with $do(X)$ operation, and then estimate the distribution of $Y$. 
\end{itemize}

Chapter 2. From Buccaneers to Guinea Pigs: The Genesis of causal inference
\begin{itemize}
	\item Regression to the mean by Francis Galton: one can predict the height of father from son, and height of son from father by the regression line. The situation is symmetric (no causal implication). Stability of the population? Answer: HWE. 
	
	\item Abandoning causality by Karl Pearson: ``the ultimate scientific statement of description of the relation between two things can always be thrown back upon ... a contingency table''. 
	
	\item Problems of Karl Pearson: some correlations are just silly, called ``spurious''. However, how do we know which ones are meaningful, which spurious? Discovery of Simpson's paradox: skull length and breath are uncorrelated if analyzing males and females separately; but correlated if together. Explanation: if shorter, likely come from female, thus breath smaller. 
	
	\item Path Diagram of Sewall Wright: Figure 2.7, coat color in guinea pigs. $D$: developmental factors, $E$: environmental factors, $G$: genetic factors. Show how these factors pass through generations and influence the trait. Path diagram allows one to derive the correlation in terms of path coefficients (thus estimating unobserved path coefficients). 
	
	\item Application of path diagram: Observed that one more day in the womb leads to gain of 5.66g weight. Does it mean that it grows 5.66g per day? No, because longer gestation period means the growth condition is more favorable (smaller litter size). To estimate: birth weight $X$ depends on $P$ gestation period and $Q$ prenatal growth rate (unobserved). Both depends on $L$ litter size. Derive the total correlation of $P$ and $X$: which is the sum of direct effect of $P$ to $X$, and the correlated induced by $L$ ($L$ affects both $P$ and $Q$ and $Q$ affects $X$). 
	
	\item Debate between Sewall Wright and Samuel Karlin in AJHG: about path analysis (1) Karlin: one can adopt an essentially model-free approach, seeking to understand the data interactively by using a battery of displays, indices and contrasts''. (2) Wright: ``There can be no such analysis without a model''. 
	
	\item \textbf{Lesson}: in path diagram, one can derive the correlation of variables, which sum over all ``relevant'' paths:  any path that can induce correlations (backdoor paths other than causal paths). This includes: chain and confounding, but not collider. 
\end{itemize}

Chapter 3: From evidence to causes
\begin{itemize}
	\item Bayesian networks: \textbf{Belief propagation}. Mimic how a neural network works. Suppose we want to infer a variable $X$, let's say we have information of its child or parent $Y$, and we can update our belief of $X$. When $Y$ is child: we update $X$ by likelihood; when $Y$ is parent: we update $X$ by prior. Proof that belief propagation algorithm eventually converges. 
	
	\item Application of Bayesian networks: (1) Genetic relationship in pedigrees. (2) Error-correcting code: need several codes to encode a single word. BN of hidden information bit, codeword and noisy code words. 
	
	\item Building blocks of BNs: (1) Chain. (2) Fork: e.g. children with larger shoes tend to read at higher levels. (3) Collider: two independent variables affecting a common one (multiple causes of the same thing). Ex. beauty $\rightarrow$ celebrity $\leftarrow$ talent, then given a celebrity, usually negatively correlated.    
\end{itemize}

Chapter 4: Confounding and deconfounding: or, slaying the lurking variable
\begin{itemize}
	\item Why RCTs work? Fisher's problem: how yield may depend on various factors of interest. Challenge: many possible confounders. Experimental design can reduce but not remove all possible confounders. Solution: RCT. What it does in the causal diagram is: for $X$, remove all edges pointing to $X$, and replace it be a ``random card''. This eliminates all back-door paths.  
	
	\item What is confounding? Anything that makes $P(Y|do(X))$ different from $P(Y|X)$. The most important case is a fork: $X \leftarrow Z \rightarrow Y$. In general, if there is any non-causal path from $X$ to $Y$, then there is confounding: the variables in this path leads to confounding.  
	
	\item Back-door criteria to deconfound: block back-door path from $X$ to $Y$ (edge pointing to $X$) by adjusting variables. Basic rules: 
	\begin{itemize}
		\item Control for a mediator closes the back-door path;
		\item Control for a collider opens the back-door path;
		\item Control for descendant of a variable is like ``partially'' controlling for the variable itself. 
	\end{itemize}
	
	\item Remark: one example, suppose we have a gene candidate of a trait (e.g. expression correlation), but the effect may be confounded by e.g. [TF], which is not observed. We can use the descendant, mRNA levels of TF target genes. In general, we may formulate this as \textbf{missing data problem}, where we adjust for some unobserved variables but using their posterior distributions. 
\end{itemize}

Chapter 5: The smoke-filled debate: clearing the air
\begin{itemize}
	\item How to prove causality of smoking to lung cancer in the absence of RCT? (1) Cornfield's Inequality: suppose we have an unobserved confounder, Smoking Gene. Show that to explain the strong association of smoking and cancer, the confounder needs to have an effect so large that is unrealistic. (2) Hill's criteria: consistency, dosage effect, temporal order, coherence with other data. None of this by itself is sufficient. 
	
	\item Birth-weight paradox: smoking usually leads to low birth-weight, which increases mortality. So we expect that low birth-weight infants of smokers have higher mortality. But the data is opposite, why? Causal model: birth defect also affects birth weight, and usually have more severe effects on mortality. This creates a collider: 
	\begin{equation}
	\text{Smoking} \rightarrow \text{Birth weight} \leftarrow \text{Birth defect}
	\end{equation}
	So conditioned on Birth weight, smoking and birth defect are anti-correlated. Thus low birth-weight infants from smokers are less likely to have birth defect, thus lower mortality. 
\end{itemize}

Chapter 6. Paradoxes Galore!
\begin{itemize}
	\item Monte Hall problem: The decision of which door to open (by the host) depends on both the door you opened and the true location of the car. So our causal diagram is: Your Door $\rightarrow$ Door opened $\leftarrow$ Car location. So Door opened is a collider. 
	
	\item Berkson's Paradox: also collider bias. Even if two diseases are not associated in general population, they may be associated in hospitals. Causal diagram: Disease 1 $\rightarrow$ Hospitalization $\leftarrow$ Disease 2.  
	
	\item Simpson's Paradox: a drug is associated with higher risk in males and in females, but putting all data together, it is associated with lower risk. (1) Purely numerical explanation. (2) Explanation by causal diagram: gender can influence the risk, and it also has an effect on whether one takes a drug. So our causal model: 
	\begin{equation}
	\text{Drug} \rightarrow \text{Heart attack} \qquad \text{Drug} \leftarrow \text{Gender} \rightarrow \text{Heart attack}
	\end{equation}
	So Gender is a confounder. If we want to study the effect of drug, we should adjust for Gender (stratify). However, if our covariate is Blood pressure, then it may mediate the effect of drug, and we should NOT adjust for it. 
	\begin{equation}
	\text{Drug} \rightarrow \text{Heart attack} \qquad \text{Drug} \rightarrow \text{Blood pressure} \rightarrow \text{Heart attack}
	\end{equation}
		
	\item Simpson's Paradox in pictures: Figure 6.6. Cholesterol ($Y$) vs. Exercise ($X$): age is a confounder. So conditioned on age, exercise reduces Cholesterol; but across all samples, positive correlation of the two. 
\end{itemize}

Chapter 7. Beyond adjustment: the conquest of Mount Intervention
\begin{itemize}
	\item Backdoor adjustment: suppose our model is $X \rightarrow Y$ with $Z$ a confounder. Then to compute the causal effect of $X$ on $Y$: (1) Regression model: $Y \sim X + Z$, with partial correlation of $Y$ on $X$ conditioned on $Z$. \\
	(2) Probabilities: 
	\begin{equation}
	P(Y|do(X)) = \sum_z P(Y|X, Z=z) P(Z=z)
	\end{equation}
	Interpretation: conditioned on $Z$, how $Y$ depends on $X$. Note that we need to average over $P(Z = z)$, not $P(Z=z|X)$. 
	
	\item Frontdoor criterion: suppose we have an measured confounder $U$ on $X$ and $Y$. We cannot do backdoor adjustment, but if we can measure the mediator of $X$ on $Y$, say $Z$, we can adjust the confounder by frontdoor criterion:
	\begin{equation}
	X \rightarrow Z \rightarrow Y \qquad X \leftarrow U \rightarrow Y
	\end{equation}
	One example: to study smoking on cancer, we use tar as $Z$. The idea is: (1) We can learn the effect of $X$ to $Z$ (smoking to tar) with $Y$ being a collider; (2) We can learn the effect of $Z$ to $Y$ (tar to cancer) by adjusting for $X$ (smoking). Putting the two together, we can get $X$ to $Y$ effect. In probabilistic terms:
	\begin{equation}
	P(Y|do(X)) = \sum_z P(Z=z | X) \sum_x P(Y|Z=z, X=x) P(X=x)
	\end{equation}
	
	\item Generalization: axioms of do calculus. Three rules: 
	\begin{itemize}
		\item Rule 1: if $Z$ blocks all paths from $W$ to $Y$, then
		\begin{equation}
		P(Y | do(X), Z, W) = P(Y| do(X), Z)
		\end{equation}
		
		\item Rule 2: if $Z$ blocks all backdoor paths from $X$ to $Y$, then we have:
		\begin{equation}
		P(Y| do(X), Z) = P(Y|X, Z)
		\end{equation}
		
		\item Rule 3: if there is no causal path from $X$ to $Y$:
		\begin{equation}
		P(Y | do(X)) = P(Y)
		\end{equation}
	\end{itemize}
	How this leads to the front door criterion (Figure 7.4). 
	
	\item Algorithm and completeness theorem: three rules are enough to decide if the do-operation has a solution (in terms of conditional distributions). Also a polynomial time algorithm exists to find the solution. Extensions: when the problem is not solvable, can we find a variable $Z$ s.t. we can solve the problem. 
	
	\item IV approach: the story of Dr. Snow. Cholera epidemic. Hypothesis: caused by water purity (correlation with infection rate), however, many confounders such as poverty. IV: which water company provided the service. 
	\begin{equation}
	\text{Water company} \rightarrow \text{Water purity} \rightarrow \text{Cholera} \qquad \text{Water purity} \leftarrow \text{Poverty, etc.} \rightarrow \text{Cholera} 
	\end{equation}
	More generally, we can understand the validity of the IV approach by causal diagram: 
	\begin{equation}
	G \rightarrow X \rightarrow Y \qquad X \leftarrow U \rightarrow Y
	\end{equation}
	The $G \rightarrow X$ effect can be learned from regression of $X$ on $G$. The $G \rightarrow Y$ effect is only due to the causal path from $X$ to $Y$, because the backdoor path $G \rightarrow X \leftarrow U \rightarrow Y$ has a collider $X$ in it. 
	
	\item Caution about MR: IV (SNPs) represents cumulative life-time impact, while drugs cannot mimic that effect.
	
	\item \textbf{Lesson}: in practice, if a given causal diagram does not allow one to estimate causal effects, one can choose what additional variables to measure to circumvent the problem. Ex. surrogate of some confounders, creating frontdoor conditions, etc. Most important cases: to learn $X \rightarrow Y$, we can use two strategies (1) Use IV of $X$. (2) Measure mediator of $X$ to $Y$. 
	
	\item Remark: limitation of frontdoor criterion in practice. If there are multiple mediators, and one only measure one (part) of them, the approach would not work. This is relevant in genomics, where the effect of a risk factor may be mediated via multiple paths. 
\end{itemize}

Chapter 8. Counterfactuals: Mining worlds that could have been
\begin{itemize}
	\item Difference of causal inference and counterfactuals: causal inference concerns the average effect of one on another, so we use probabilistic models. Counterfactuals concerns the cause of a specific instance, so the influences of all variables on a quantity of interest are modeled in a deterministic fashion. Ex. phenotype depends on genotype and environment: in modeling average effects, we treat environmental effects as random (errors); in modeling an individual, the environmental effect is deterministic.  
	
	\item David Hume on causality: ``We may define a cause to be an object followed by another, and where all the objects, similar to the first, are followed by objects similar to the second. Or, in other words, where, \textit{if the first object has not been, the second never had existed}''. The second definition is a \textbf{counterfactual}.  
	
	\item General approach to counterfactuals: use Structural Causal Models (SCM). Similar to SEM, the difference is that in SCM, the relationship may not be linear. 
	
	\item Potential outcomes: salary estimation problem, salary $S$ depends on one's education $ED$ and one's experience (number of years) $EX$. What would the salary of someone had he have a different education degree? Notations: what is the value of $Y$ for individual $u$, had $X$ been assigned the value $x$: $Y_{X=x}(u)$, or simply $Y_x(u)$. Ex. 
	
	\item Imputation approach: why is this wrong? Suppose we are interested in the salary of Alice, had she had a college degree. We can match employees with Alice in every aspect, except the degree, and then estimate from those with college degree. Intuitively, $ED$ and $EX$ are not independent: if Alice has a higher degree, she will likely have less $EX$. So matching gives misleading results. 
	
	\item Solving the counterfactual problem by SCM: first, the causal diagram
	\begin{equation}
	\text{Education} \rightarrow \text{Experience} \rightarrow \text{Salary} \qquad \text{Education} \rightarrow \text{Salary} 
	\end{equation}
	We can fit the data to have the SCM equations as:
	\begin{equation}
	EX = 10 - 4 \times ED + U_{EX} \qquad S = 65,000 + 2,500 \times EX + 5,000 \times ED + U_S
	\end{equation}
	where $U_S$ are $U_{EX}$ are idiosyncratic factors (unique to each individual). Note this is a deterministic equation. With this, we solve the problem in three steps:
	\begin{itemize}
		\item Abduction: determine $U_S$ are $U_{EX}$ for Alice. 
		\item Action: apply do operation. This means setting $ED = 1$ for Alice (eliminating any edges to $ED$). 
		\item Prediction: estimate $S$ using the causal diagram and idiosyncratic factors. This means estimate the effect of $ED = 1$ on $EX$, and then on $S$. 
	\end{itemize}
	
	\item Why Durbin's approach is flawed? Potential outcome approach makes key assumption of ``ignorability''. However, this is hard to verify. 
	
	\item Necessary cause and sufficient cause: Let $X = 1$ be treatment and $Y = 1$ be observed outcome. Suppose we observe $X = 1, Y=1$, we would like to ask if $X$ causes $Y$. There are two ways: (1) Probability of necessity, PN. It captures ``but for'' cause. Someone would not die but for some condition. PN is defined as:
	\begin{equation}
	PN = P(Y_{X=0} = 0 | X= 1, Y=1)
	\end{equation}
	(2) Probability of sufficiency, PS. It captures proximate cause, defined as:
	\begin{equation}
	PS = P(Y_{X=1} = 1 | X=0, Y=0)
	\end{equation}
	It involves imagining $X = Y = 0$ (no death, no treatment), then if we have treatment, would it lead to the outcome? Ex. fire broke out after someone struck a match. Does oxygen cause the fire? It has high PN, but low PS. 	
	
	\item Climate change: can we attribute one extreme weather event to climate change?  
\end{itemize}

Chapter 9: mediation: the search for a mechanism
\begin{itemize}
	\item The story of Barbara Burks: Nature vs. Nurture. Estimate the effect of parental intelligence on children's IQ (total genetic effect). The causal diagram:
	\begin{equation}
	\text{Parental IQ} \rightarrow \text{Child IQ} \qquad \text{Parental IQ} \rightarrow \text{Social status} \rightarrow \text{Child IQ}
	\end{equation}
	So one should not adjust for social status, which is a mediator. Also suppose we have unknown factor affecting both child IQ and social status:
	\begin{equation}
	\text{Social status} \leftarrow X \rightarrow \text{Child IQ}
	\end{equation}
	Then Social status is a collider from parental IQ to social status to $X$ to child IQ. Adjusting it leads to collider bias. 
	
	\item Why we need mediation analysis? Suppose we have a causal diagram, and we estimate the effect of $X$ on $Y$. When we perform $do(X)$, the effect will be propagated to other variables, which may change $Y$. Our goal is to make statements of what are the ``mechanisms'' of $X$ effect on $Y$: does it go through a particular variable $M$ or directly? 
	
	\item Direct effects: Berkeley admission example. One wants to estimate if gender affects admission decision. However, gender may influence which department one applies, and the admission rates differ across departments. 
	\begin{equation}
	\text{Gender} \rightarrow \text{Admission} \qquad \text{Gender} \rightarrow \text{Department} \rightarrow \text{Admission}
	\end{equation}
	Our goal is to estimate the direct effect from Gender to Admission.  
	
	\item Defining direct effects: our model is:
	\begin{equation}
	X \rightarrow Y \qquad X \rightarrow M \rightarrow Y
	\end{equation}
	To define direct effects of $X$ to $Y$, we will use $do(X)$, and control for $M$ (s.t. the effect must be from $X$ to $Y$ directly). The problem is that when we change $X$, $M$ will be also, so at what value of $M$ shall we control? Natural Direct Effect (NDE) is defined as $M$ is set at the level it had been when there is no perturbation of $X$. Let's say $M_0$ is the value of $M$ for a sample whose $X = 0$. We have:
	\begin{equation}
	\text{NDE} = P(Y_{M=M_0} = 1 | do(X) = 1) - P(Y_{M=M_0} = 1 | do(X) = 0) 
	\end{equation}
	where the subscript indicates counterfactual. 	

	\item Defining indirect effects: we want to know the effect of $X$ on $Y$ that is mediated by $M$, so we should control for $X$ (do-operation on $X$), and assess the effect of $M$. But we should not just use do-operation for $M$: we would learn the effect of $M$ to $Y$, which may have nothing to do with $X$. So we should assess what happens to $Y$, if we change $M$ based on the effect of $X$ on $M$, while controlling $X$. This leads to Natural Indirect Effect:
	\begin{equation}
	\text{NIE} = P(Y_{M=M_1}=1 | do(X=0)) - P(Y_{M=M_0}=1 | do(X=0))
	\end{equation}
	where $M_1$ is the value of $M$ when $X = 1$, and $M_0$ is the value when $X = 0$. The formulat:
	\begin{equation}
	\text{NIE} = \sum_m [P(M=m | X=1) - P(M=m|X=0)] \times P(Y=1|X=0, M=m)
	\end{equation}
	In words, to assess the indirect effect of changing $X = 0$ to $X = 1$ on the probability of $Y = 1$: imagine we change $X$ from 0 to 1, how the values of $M$ are changed? Then given that value of $M$ and $X = 0$ (we do not want direct effect), what is the probability of $Y = 1$?  
	
	\item Smoking gene: increases the risk of cancer, estimation of direct vs. indirect effects. The indirect effect through smoking is very small: the gene makes people smoke only one more cigarette per day (small clinical effect). However, the effect of smoking gene on lung cancer is large only on individuals who smoke - an example of interaction. 
	
	\item Tourniquet saving soldiers: Tourniquet uses are not found to save lives. However, this is because doctors only collect who survived long to reach hospitals, thus some of Tourniquet effect is through pre-administrative survival, which is not measured. 
\end{itemize}

Chapter 10: big data, artificial intelligence and the big questions
\begin{itemize}
	\item Data fusion: e.g. merging several datasets from different states. The causal diagrams are different, e.g. in some state we have RCT data, and in different states, we need to adjust for different confounders. But causal model and do-calculus allow us to estimate the same causal effect across studies. 
	
	\item Selection bias: e.g. we only have data from hospitalized samples, then in the model, we may need to capture this aspect, by having a link from Hospital to the variable of interest. 
	
	\item Free will: this is basically an illusion. But what is the benefit? Possible explanation: voluntary actions are recognized by a trace they leave in short-term memory. This leaves the impression of ``consciousness''. 
	
	\item \textbf{How do we encode causal relations in our mind}? [Personal notes] Ex. we know that a car crash is bad even if we have never seen or experienced it. This is based on analogy: we know that crash is bad in general from our experience, and we can make reasonable predictions in new situations. In other words, we use the same pattern recognition mechanism for vision, speech, etc. to make causal predictions. 
\end{itemize}
%%%%%%%%%%%%%%%%%%%%%%%%%%%%%%%%%%%%%%%%%%%%%%%%%%%%%%%%%%%%
\section{Causality: Models, Reasoning and Inference [Judea Pearl]}

Problems of causal inference: 
\begin{itemize}
\item Learn causal model from the data: a model may not be identifiable, e.g. the simplest case, two associated variables $(X,Y)$, is not identifiable. 
\item Prediction: what would a variable $Y$ be given another variable $X$? 
\item Intervention/causal effect: given a causal model, estimate the causal effects of interest (or other causal aspects).  
\item Counterfactuals: e.g. given that a person who smokes developed lung cancer, would the person avoid lung cancer had he not smoke? 
\end{itemize}

Limitations of regression approach to observational studies:  
\begin{itemize}
\item Simple regression approach: Suppose we want to infer if $X$ has a causal influence on $Y$, or estimate the effect of $X$ on $Y$. Then we do regression of $Y$ on $X$, with other possible confounding variables, say $Z$, and we conclude: $X$ has an influence on $Y$ iff the coefficient, $\beta_X \neq 0$; and $\beta_X$ measures the strength of causation. Why is this approach insufficent?

\item Learning causal model:   
\begin{itemize}
	\item Backward edge: if $Y \rightarrow X$, then regressin of $Y$ on $X$ will have non-zero coefficient for $X$ (the parent variables of $Y$ generally cannot explain all the variations of $Y$). 
	\item Missing indirect effect: if $X$ acts indirectly on $Y$ through $Z$, and $Z$ is included in the features, then the effect of $X$ on $Y$ will not be seen. 
	\item Unobserved confounding variables: will create false association, e.g genotype that influences both smoking and disease. 
\end{itemize}

\item Estimating causal effect: suppose the causal model is known, the the coefficient suggests causal effect only when all features influence the response independently. Example: the effect of $G$ on $D$ (1) direct effect of $G \rightarrow D$; (2) indirect effect: $G \rightarrow S \rightarrow D$. The regression of $D$ on $G$ and $S$ will miss the indirect effect. 
\end{itemize}
 
Structural/functional causal models: 
\begin{itemize}
\item Structural model: specifies the deterministic relations among variables of interest, plus the disturbances (exogeneous) random variables. In general, a variable $x_i$ can be written as: 
\begin{equation}
x_i = f_i(\pi(x_i), u_i)	
\end{equation}
where $f_i$ is a deterministic function, $\pi(x_i)$ represents the variables that directly influences $x_i$, and $u_i$ be disturbance. In structural equation modeling (SEM), $f_i$ are linear functions. Also assume that $u_i$ are all independent, as otherwise, we would have latent variables that explain the correlation among $u_i$'s. 

\item Causal diagrams: structural model can be represented by the causal diagram. Example, consider the following structural model: 
\begin{equation}
x = f_X(u_X)
\end{equation}
\begin{equation}
y = f_Y(x,u_Y)	
\end{equation}
\begin{equation}
z = f_Z(y, u_Z)	
\end{equation}
It can be written as $X \rightarrow Y \rightarrow Z$, the random (exogenous) variables: $U_X, U_Y, U_Z$, are typically omitted. Note that the semantics of the model lies in the missing links, i.e. independence relations, e.g. $Z$ is independent of $X$, conditional on $Y$. 

\item Internventional interpretation: a structural model specifies how a variable in the LHS changes when the variables in the RHS are changed (by external force). Ex. $y = \beta x + \epsilon$, we have: $E(Y|do(x)) = \beta x$, or: 
\begin{equation}
\beta = \frac{\partial}{\partial{x}} E[Y|do(x)]	
\end{equation}

\item Markovian: the model $M$ is semi-Markovian if the causal diagram $G(M)$ is a DAG; if in addition, the background variables are independent, the model is Markovian. Note that any semi-Markovian model can be converted to a Markovian model by introducting latent variables: e.g. if the disturbance term $\epsilon_X$ and $\epsilon_Y$ are dependent, then introduce $U \rightarrow X$ and $U \rightarrow Y$. 

\item Counterfactuals: what is the probability $Y = y'$ under the treatment $X = x'$, given that the actual situation is $Y = y$ under the treatment $X = x$? 
\end{itemize}

Comparison of structural model and probabilistic graphical model: even though causal Bayesian network can be used as causal model, the structural model approach is preferred because: 
\begin{itemize}
\item Cyclic graph: with structural model, it is possible to have cyclic relationship, and model the dynamics. 
\item Intervention: more general interventions such as changing parameters of the deterministic equation can be handled in the structural model. 
\item Counterfactual: the counterfactual questions can be answered in the structural model. This is due to the ``stability'' of structural models, where the deterministic relationship does not change. On the other hand, this may not be the case for Bayesian network. Ex. the joint probability $P(x,y) = 1/4 \forall x,y $ for two binary RVs $X$ and $Y$ may correspond to different deterministic equations, and each provides a different answer to the counterfactual problem (page 36). 
\end{itemize}

Procedure of learning ounterfactuals in structural models: suppose we want to learn about $Y$ under $X= x$ given evidence $e$. Let $U$ be the unobserved variable, then: 
\begin{itemize}
\item Update the probability $P(u)$ to obtain $P(u|e)$;
\item Replace the equations: $X = x$; 
\item Use the modified model to compute $P(Y=y)$. 
\end{itemize}

\subsection{Inferring Cusation}

Model preference: to infer causality from data, some principles for model preference would be needed; otherwise, one could always come up with arbitrarily complex causal models. 
\begin{itemize}
\item Minimality (Occam's razor): minimal model that explains the dependency structure in the observation without any additional dependency that is not observed.
\item Stability: a stable model is preferred, where the dependency relationship does not change with specific parameters of the model. In general, a simpler model is always a special case of a more complex model, which reduces to the simpler model at special parameter values. 
\end{itemize}

General strategies: 
\begin{itemize}
\item Bayesian structure learning: a model is evaluated by posterior probability. Simpler models will be preferred because prior distribution penalize complex models. 
\item Condition independence: (1) determine all conditional independence (CI) relations from observation; (2) search for simplest model that is consistent with the CIs.
\end{itemize}

Inductive causation algorithm: consists of three main steps: 
\begin{itemize}
\item Determine CI: for any pair of nodes $a$ and $b$, search for a set $S_{ab}$ s.t. $a \bot b | S_{ab}$. Then connect an undirected edge for $a$ to $b$, if no set $S_{ab}$ is found. 
\item Identifying colliders: if $a$ and $b$ share a common neighbor $c$, check if $c \in S_{ab}$: if not, then create the arrows: $a \rightarrow c \leftarrow b$; otherwise, do nothing (as it could be a fork or chain). 
\item Orient as many undirected edges as possible: the orientations should not create a new $v$-structure (colliders) and not create a directed cycle. 
\end{itemize}

Local criteria for causal relations: to determine whether $X$ has a causal influence on $Y$.
\begin{itemize}
	\item Potential cause: $X$ has a potential causal influence on $Y$ if $X$ and $Y$ are dependent in every context; and there exists a variable $Z$ and context $S$ s.t. (1) $X$ and $Z$ are indepenedent given $S$, and (2) $Z$ and $Y$ are dependent given $S$. 
	\item Genuine cause: see Definition 2.7.2. the idea is: if $Z$ is a potential cause of $Y$, and the effect of $Z$ can be completely explained by $X$, i.e. $Z \bot Y | S \cup X$, then $X$ is a genuine cause of $Y$. 
\end{itemize}

Time and causality: 
\begin{itemize}
\item Statistical time: given a distribution $P$, it is defined as an ordering of variables that agree with at least one minimal causal structure consistent with $P$. Thus $P$ may have multiple statistical time (Markov equivalence class). 
\item Temporal bias: the physical time should coincide with at least one statistical time. 
\end{itemize}

Comments on condition independence (CI)-based learning: 
\begin{itemize}
\item Covariates: generally, to resolve $X \rightarrow Y$, one will need to introduce additional covariates to test more specific models. Ex. the model $X \rightarrow Z \rightarrow Y$ is testable, as it entails the CI: $X \bot Y | Z$. 

\item Limitations: there are often unobserved variables, which may make CI invalid. Ex. in the model $X \rightarrow Z \rightarrow Y$, if there is an observed variable $U$, s.t. $U \rightarrow X$ and $U \rightarrow Y$, then the CI: $X \bot Y | Z$ does not hold. While the CI: $X \bot Y | (Z, U)$ holds, it cannot be tested, as $U$ is not observed. To overcome this limitation, one would need to examine consequences of a model other than CI (which may be too restrictive). 
\end{itemize}

\subsection{Identification of causal effects} 

Ref: [Pearl, Causality, 2000, Chapter 3]

Intervention: the fundamental solution to the causality problem is intervention. 
\begin{itemize}
	\item Concept: given a structural model $M$, intervention amounts to the use of external forces to change the causal relations encoded by the model. Most commonly, this is to set one or more variables to some specified values. Thus intervention $do(x_i)$ means: change the value of $x_i$, and this results in a new model, $M_{x_i}$ where (1) the equation $x_i = f_i(\pi_i, u_i)$ is replaced by $x_i = x$; (2) any variable $x_i$ in other equations will be replaced by the value $x$. 
	
	\item Interpretation of intervention with structural models: suppose a system can be represented by a model $M$ without intervention. With intervention, the external force has to be incorporated into the model (the external force remains constant in the situation with no intervention, thus is not needed and omitted by the model $M$). Intervention can thus be represented by an augmented graph, where the external force of $do(x)$ is explicitly represented as a node $F_x$: $x$ will be given by $f(x)$ when $F_x = \text{idle}$; $x$ will be equal to the value given by $F_x$, when it is not idle. 
	
	\item Intervention and randomization: In the smoking example, to estimate the effect of smoking on diseases, one should subject the patients to treatment (smoking), and compare the difference of diseases. Doing this, the two groups (smoking and not) are random wrt. genotypes, thus any difference in diseases must be attributable to the difference of treatment. This is like in physical experiments, where everything has to be controlled equal with only treatment differs; the population level, this means, the case and control groups must be randomized with everything else except treatment.
\end{itemize}

Causal effect estimation and causal model selection: 
\begin{itemize}
	\item These are generally two different problems: the technique of causal effect estimation is insufficient for selecting causal models. Ex. to establish the model $Q \rightarrow X \rightarrow Y$ (direct influence of $X$ on $Y$), one needs to consider the conditional independence: $Q \bot Y | X$ (this is not held in the alternative model $Q \rightarrow Y \rightarrow X$; while causal effect estimatation of $X$ on $Y$ will not distinguish two models. 
	\item For special problems: causal effect estimation may be used to choose between alternative models, most importantly test if any edge is necessary. Ex. given a model $Q \rightarrow X \rightarrow Y, Q \rightarrow Y$, test if the direct effect of $X$ on $Y$ is 0. 
	\item The CI test can be formulated as a problem of estimating causal effect: 
\end{itemize}

Defining causal effects: 
\begin{enumerate}
\item The causal effect of $X$ on $Y$ can be defined in terms of the probability of $Y$ under the model $M_x$, written as: 
\begin{equation}
P(Y=y|do(x)) = P(Y=y|M_x)	
\end{equation}
Thus the effect can be defined as, e.g. 
\begin{equation}
E(Y|do(x')) - E(Y|do(x))	
\end{equation}
The causal effect reflects adjusting/controlling for confounding variables. 

\item Estimating causal effects: $P(Y=y|M_x)$ can be computed in terms of probability distribution in the original model $M$. Some examples (suppose $X$ is smoking, $Y$ is disease, $Z$ is genotype and $W$ is some lifestyle that may be affected by smoking): 
\begin{itemize}
	\item $X \rightarrow W \rightarrow Y$: marginalization of $W$ (average over the intermediate cause): 
\begin{equation}
P(Y=y|do(x_0)) = \sum_w p(w|x_0) p(y|w)
\end{equation}

	\item $X \rightarrow Y \leftarrow Z$ and $Z \rightarrow X$: adjusting for confounding variable $Z$: 
\begin{equation}
P(Y=y|do(x_0)) = \sum_z p(y|x_0,z) p(z)
\end{equation}	
\end{itemize}

\end{enumerate}

Causal effects in the presence of unobserved variables: 
\begin{enumerate}
\item Why it is possible: it is not necessary to include all variables in analysis, e.g some variables that do not directly affect $X$ and $Y$ are already randomized. Consider another case, $X \rightarrow U \rightarrow Y$, then the information of effect of $U$ has already been included in the conditional distribution $P(y|x)$, in fact: 
\begin{equation}
P(y|\hat{x}) = \sum_u P(u|x) P(y|u) = P(y|x)	
\end{equation}

\item Adjusting for direct causes: let $\pi_i$ be the set of direct causes of $X_i$, and $Y$ be any set of variables disjoint of $X_i \cap \pi_i$, then: 
\begin{equation}
P(y|\hat{x}_i') = \sum_{\pi_i} P(\pi_i) P(y|x_i', \pi_i)	
\end{equation}
Proof: the joint distribution of $M_{x_i}$ differs $M$ by only the term $P(x_i'|\pi_i)$, and express this conditional probability in terms of $P(\pi_i)$ and the probability where $\pi_i$ is in the conditional part. 

\item Back-door criterion: a set of variables $Z$ is sufficent (or admisible) for adjustment if: 
\begin{itemize}
\item No element of $Z$ is a descendant of $X$: no need to worry about these variables as their effects are already included in $X$. 
\item The elements of $Z$ block all ``back-door'' paths from $X$ to $Y$, namely all paths that end with an arrow pointing to $X$: these variables (pointing to $X$) are responsible for creating spurious association (e.g. genotype in the smoking example), thus should be measured. 
\end{itemize}
Given a sufficent set $Z$, we have: 
\begin{equation}
P(y|\hat(x)) = \sum_z P(z) P(y|x,z) 	
\end{equation}
Proof: consider the augmented graph with the new node $F_x$. Let $P'$ be its joint distribution, then we have: 
\begin{equation}
P(y|\hat{x}) = \sum_z P'(z|F_x) P'(y|z,x,F_x)	
\end{equation}
The first term is equal to $P(z)$ as $Z$ is non-descendants of $X$ (the causes of $X$ should not be affected by the intervention). And for the second term, we have: $Y \bot F_x | X,Z$, from the $d$-seperation of $Y$ and $F_x$ by $(X,Z)$ ($Z$ blocks all back paths, and $X$ blocks all other paths from $Y$ to $F_x$), thus it is equal to $P(y|x,z)$. 

\item Front-door criterion:  a set of variables $Z$ satisfy front-door criteria if: 
\begin{itemize}
	\item $Z$ intercepts all directed paths from $X$ to $Y$; 
	\item There is no back-door path from $X$ to $Z$; 
	\item All back-door paths from $Z$ to $Y$ are blocked by $X$.
\end{itemize}
Intuitively, $Z$ is the children of $X$ that could influence $Y$, and there is no other variable that influence $Z$. We have: 
\begin{equation}
P(y|\hat{x}) = \sum_z P(z|x) \sum_{x'} P(x') P(y|x'z)	
\end{equation}
Thus it is possible to estimate causal effect of $X$ to $Y$, even if covariates are ``post-treatment'' variables (those that can be affected by $X$). 
\end{enumerate}

Identifiability: 
\begin{itemize}
	\item In the presence of latent variables, a causal effect may not be identifiable. Ex. in a simplest model, $U \rightarrow X, U \rightarrow Y, X \rightarrow Y$, the effect of $X$ on $Y$ is not identiable, as any such effect may also be due to the unobserved $U$. 
	\item Examples of identifiable causal diagrams: those specified by the back-door and front-door criteria are identifiable. There are other cases as well. Note that if there is no back-door path to $X$, then the effect must be identifiable. 
	\item Examples of non-identiable causal diagrams: often involving cofounding variables on $X$ (i.e. some unobserved variable $U$ influences both $X$ and some other variable). 
\end{itemize}

Application of causal effect estimation to structural learning: 
\begin{itemize}
\item Structural learning: when the causal model is unknown, we can propose a model, and test if the predicted causal effects from the model are consistent with this model. For a variable $X$, which should affect $Y$ according to the model, if there is no causal effect, estimated from data,   (i.e. intenvention of $X$ has no effect on $Y$), then the proposed model is wrong. 
	
\item Exmaple: suppose we want to test if smoking ($X$) causes disease ($Y$), we could add a covariate $Z$, the amount of tar deposited in the lung, and test the model $X \rightarrow Z \rightarrow Y$ (using CI). However, with unobserved variable $U$, s.t. $U \rightarrow X$, $U \rightarrow Y$, CI testing does not work. But we could estimate the causal effect of $X$ on $Y$, even in the presence of $U$, with the help of $Z$ (front-door criterion). 
\end{itemize}

Extensions:
\begin{itemize}
	\item Cyclic causal diagrams: need a general definition of $d$-separation. 
	\item General intervention: e.g. intervention changes the functional form or parameters in the deterministic equation. 
\end{itemize}

\subsection{Linear Structural Model (SEM)} 

Ref: [Pearl, Causality, Chapter 5, 2000]

Problems of parameteric structural model: in particular, the linear model, or structural equation modeling (SEM): 
\begin{itemize}
\item Model testing: check the assumptions of the model. The general idea is: calculating the covariance structure from the model and compare those with the sample covariance structure in the data. 
\item Parameter identification: the parameters of the determistic model. In traditional SEM: MLE; in structural theory: through causal effects (see below). 
\end{itemize}

Global vs local parameter fitting and model testing: 
\begin{itemize}
	\item Global approach: parameter estimation through MLE; then model testing is based on these parameters: calculate the partial correlation coefficients according to these parameters (theoretical values) and compare with the observed values in data. 
	\item Local approach: parameter estimation through local causal effects (below), and the conditional independence only depends on the model structure (not parameters), which can be directly verified. 
	\item Comparison: global approach requires MLE of the entire model, however, some paramaters may not be identifiable, and thus the estimated parameters may be unstable. Local approach is taken by the structural theory. 
\end{itemize}

$d$-separation and partial correlation coefficient (PCC): 
\begin{itemize}
\item Partial correlation and regression coefficient: given variables $X$, $Y$ and $Z$, the partial variance, covariance and correlation coefficient are defined as the term conditioned on $Z$: $\sigma^2_{X|Z}$, $\sigma^2_{XY|Z}$ and $\rho_{XY|Z}$ (note that these are well-defined because they are not dependent on the value of $Z$). The partial regression coefficient is given by: 
\begin{equation}
r_{YX|Z} = \rho_{YX|Z} \frac{\sigma_{Y|Z}}{\sigma_{X|Z}}	
\end{equation}
It is equal to the coefficient of $X$ in the linear regression of $Y$ on $X$ and $Z$, i.e. $r_{YX|Z} = \beta_X$ in the regression: $Y = \beta_0 + \beta_X X + \beta_Z Z$. More generally, the coefficient of $X$ in a linear regression: 
\begin{equation}
Y = aX + b_1 Z_1 + \cdots + b_p Z_p	
\end{equation}
is given by $a = r_{YX|Z_1Z_2 \cdots Z_p}$. 

\item Markovian models: a partial correlation coefficient $\rho_{XY|Z} = 0$ iff $X$ and $Y$ are $d$-separated by $Z$. 

\item Semi-Markovian and general models: for a linear model which may include cycles and bidirected arcs (dependent errors), the PCC $\rho_{XY|Z} = 0$ if $X$ and $Y$ are $d$-separated by $Z$ (where a bidrected arc $i \leftrightarrow j$ is interpreted as a latent common parent $i \leftarrow L \rightarrow j$). 
\end{itemize}

Testing conditional independences (causal assumptions of the model): 
\begin{itemize}
\item Local testing: test the CI implied by the structure model with PCC, estimatable from the data. 

\item Graphical basis: not all PCCs need to be tested, as some would imply the other ones. In a DAG model, let $Z_{ij}$ be any set of nodes that $d$-separates $i$ from $j$ for a nonadjacent pair $i$ and $j$. Then the set of pairs with PCC $= 0$, one element per nonadjacent pair, constitutes a basis for the set of all zero PCCs entailed by the model (i.e. sufficent to verify all zero PCCs). 

\item Model equivalence: for Markovian models, observational equivalence (i.e. covariance equivalence) is equivalent to the same CI relations (i.e. the same skeleton plus $v$-structure). For semi-Markovian models, the CI relations are necessary but not sufficient for observational equivalence, i.e. it is possible that the same CI may imply different covariance equivalence. 

\item Model equivalence limits our ability of testing models: a model can only be tested/verified up to its equivalence class. 
\end{itemize}

Parameter identification in linear models: in general semi-Markovian models. The basic strategy is to relate the causal parameters (path coefficients) to the observable partial correlation (or regression) coefficients. 
\begin{itemize}
\item Direct effects: to determine $\alpha$, the path coefficient associated with $X \rightarrow Y$. Let $G_{\alpha}$ be the diagram when $X \rightarrow Y$ is removed from $G$, then $\alpha = r_{YX|Z}$ for a set of variables $Z$ if: (1) $Z$ contains no descendent of $Y$; and (2) $X$ and $Y$ are $d$-separated by $Z$ in $G_{\alpha}$. The intuition is: $X$ may have both direct, defined by $\alpha$, and indirect effects on $Y$. If $Z$ blocks all indirect effects ($Z$ should not contain descendant of $Y$ - the post-treatment variable), then any effect of $X$ on $Y$ (partial regression conditional on $Z$) must be due to $\alpha$. 

\item Back-door criterion: the total effect of $X$ on $Y$ is given by $r_{YX|Z}$ for a set of variables $Z$ if: (1) no member of $Z$ is a descendant of $X$; and (2) $X$ and $Y$ are $d$-separated by $Z$ in $G_{\underline{X}}$ formed by deleting all arrows emanating from $X$. The total effect is defined by the sum of the products of path coefficients along all directed paths from $X$ to $Y$. Ex. $X \stackrel{\alpha}{\rightarrow} Y, X \stackrel{\beta}{\rightarrow} Z \stackrel{\gamma}{\rightarrow} Y$, the total effect of $X$ on $Y$ is: $\alpha + \beta \gamma$. 

\item Instrumental variables: in some cases where a parameter may not be identifiable, introducing instrumental variables may make it identifiable. Ex. in $X \stackrel{\alpha}{\rightarrow} Y, X \leftrightarrow Y$, $\alpha$ is not identifiable, however, adding $Z \stackrel{\beta}{\rightarrow} X$, we have: $\beta = r_{XZ}$ and $\alpha \beta = r_{YZ}$. 

\item Example: $Z \stackrel{\beta}{\rightarrow} X \stackrel{\alpha}{\rightarrow} Y, Z \leftrightarrow Y$, we have: $\alpha = r_{YX|Z}, \beta = r_{XZ}$. 

\item Example: $X \stackrel{\alpha}{\rightarrow} W \stackrel{\beta}{\rightarrow} Y, X \stackrel{\delta}{\rightarrow} Z \stackrel{\gamma}{\rightarrow} Y, X \leftrightarrow Z, W \leftrightarrow Y$, we have: (1) the direct effect of $X$ on $W$: $X$ and $W$ are $d$-separated in $G_{\alpha}$, thus $\alpha = r_{WX}$; (2) the total effect of $X$ on $Y$: $X$ and $Y$ are $d$-separated by $Z$ in $G_{\underline{X}}$, thus $\alpha \beta = r_{YX|Z}$; (3) the direct effect of $Z$ on $Y$: they are $d$-separated by $X$ in $G_{\gamma}$, thus $\gamma = r_{YZ|X}$. 
\end{itemize}

Application of SEM on linear regression: in general, in the regression problem, the predictors $X_1, \cdots, X_p$ are not independent, so it's possible to model the dependence of $X_j$'s and $Y$. 
\begin{itemize}
\item Example: $X_1 \rightarrow Y, X_2 \rightarrow Y$, we have $\beta_1 = r_{YX_1}, \beta_2 = r_{YX_2}$, note that since $X_1$ and $X_2$ are independent, we don't need conditioning here. If in addition, we have $X_1 \leftrightarrow X_2$, then conditioning on other explanatory variables is necessary: $\beta_1 = r_{YX_1|X_2}, \beta_2 = r_{YX_2|X_1}$. 

\item Correlated features: even features are correlated, the model is still identifiable: $\beta_1$ from the correlation of $X_1$ and $Y$ in the stratum of $X_2$; and similar for $\beta_2$. However, correlation reduces the variablity within the stratum (thus parameter estimation is less stble); in the extreme case, $X_1$ and $X_2$ are perfectly correlated, then there is no variation of $X_1$ in the stratum of $X_2$, thus not possible to estimate $\beta_1$. 
\end{itemize}

Identification of parameters in nonparameteric models: 
\begin{itemize}
	\item Limitations of parameter identification: only from the probility distribution. Thus it would be impossible to identify two differenent models if they lead to the same distribution. 
	\item In general, the non-parameter models (functions) have many-to-one relations with probability distributions, thus not identifiable from observational data. However, different models have different interventional properties. 
\end{itemize}

\subsection{Counterfactuals and Applications} 

Ref: [Pearl, 2009]

Applications of counterfactual analysis:
\begin{itemize}
\item Direct vs total effect: in some cases, we are only interested in direct causal effect. Ex. whether gender ($X$) directly influences hiring ($Y$). Let $Z$ be the qualification, then the direct effect of $X$ on $Y$, as opposed to that mediated by $Z$, can be defined as, the controlled direct effect: 
\begin{equation}
\text{CDE} := E(Y|do(x'), do(z)) - E(Y|do(x), do(z))
\end{equation}
Or in counterfactual notation: 
\begin{equation}
\text{CDE} := E(Y_{x'z}) - E(Y_{xz})
\end{equation}
Thus CDE controls the mediating variable, and the difference must be due to $X$. In case CDE depends on the value of $Z$, we should consider the ``natural direct effect'', where $Z$ is set the counterfactual at given values of $X$: 
\begin{equation}
\text{DE}_{x,x'}(Y) := E(Y_{x',Z_x}) - E(Y_x)	
\end{equation}
Under some assumptions, it can be shown that it is the weighted average of CDE: 
\begin{equation}
\text{DE}_{x,x'}(Y) = \sum_z P(z|do(x)) \left[ E(Y|do(x',z)) - E(Y|do(x,z))\right]	
\end{equation}

\item Probability of causation: we are interested in the question, given $X=x$ and $Y=y$, what is the probability that $Y$ would be different had $X = x'$. Ex. in the smoking example, this means, for a patient with smoking and disease, what is the chance that he would not have the disease had he not smoke. This would be a measure of causation of disease due to smoking. The probability of causation or necessarity is defined as: 
\begin{equation}
PN(x,y) = P(Y_{x'} = y'|X=x, Y=y)	
\end{equation}

\end{itemize}
 
\subsection{Comparison with Other Approaches}

Potential outcome framework by Neyman-Rubin: 
\begin{itemize}
\item Missing data approach: given a causal model, to assess the effect of $X$ on $Y$, introduce random variables $Y_x$, which is the value of $Y$ had the treatment be $x$ (even if the actual treatment may be different). Given an observation $(X=x,Y=y)$, we have: 
\begin{equation}
Y_x = y	
\end{equation}
i.e. the counteractual is equal to the actual observation. Thus, counterfactuals are treated as missing data (half of them are observed, assuming $X$ is binary), and the usual statistical apporach can be applied on the joint distribution $P*$ defined on all variables plus counterfactuals.  

\item Solving the counterfactuals: the casuality assumptions involved in specifying the causal model will be needed. Ex. if a set of covariates satisfy CI: $X \bot Y_x | Z$, then we have: 
\begin{equation}
P*(Y_x = y) = \sum_x P(y|x,z) P(z)	
\end{equation}
which is similar to the equation under the structural theory. The intuition: we should control for confounding variable $Z$; if given $Z$, $X$ and $Y_x$ are independent (e.g. given genotype, smoking and smoking suspectibility are independent), then we could estimate the causal effect. 

\item Difficulties with potential outcome framework: the causality assumptions are not explicitly formulated, thus it may be difficult to verify the probabilistic properties of the distribution involving counterfactuals (i.e. a calculus of counterfactuals, in addition to probability distribution, is missing in this framework). 

\end{itemize}
%%%%%%%%%%%%%%%%%%%%%%%%%%%%%%%%%%%%%%%%%%%%%%%%%%%%%%%%%%%%
\section{Structural Equation Modeling}
\begin{enumerate}
	
	\item{Introduction to Structural Equation Modeling (SEM)}
	
	Reference: [Bollen, Structural Equations with Latent Variables, Chapter 1, 2]
	
	Modeling strategy of SEM: 
	\begin{itemize}
		\item Latent variable model: the basic SEM strategy is that the true causal relations in the model are encoded by latent variables. Normally, define $\xi$ the $n$-dim. exogeneous variables (not explained by the variables within the model), and $\eta$ $m$-dim. the endogenous variables. Then $\eta$ can be written as linear models of $\xi$ and $\eta$: 
		\begin{equation}
		\eta = B \eta + \Gamma \xi + \zeta	
		\end{equation}
		where $B$ is $m \times m$ matrix, and $\Gamma$ is $m \times n$ matrix, $\zeta$ is the unexplained random error  of $\eta$ (independent of the latent variables). Ex. a model of one exogeneous variable and two endogenous variables: 
		\begin{equation}
		\eta_1 = \gamma_{11} \xi_1 + \zeta_1	
		\end{equation}
		\begin{equation}
		\eta_2 = \beta_{21} \eta_1 + \gamma_{21} \xi_1 + \zeta_2	
		\end{equation}
		The error assumption: $\zeta_i$ is (1) homoscedastic: same variance across multiple observations; (2) non-autocorrelated: independent across multiple observations. 
		
		\item Measurement error model: the observations are measurements or proxies of the latent variables. Let $x$ be the $q$-dim. measurement of the exogenous variables and $y$ be the $p$-dim. measurements of the endogenous variables: 
		\begin{equation}
		x = \Lambda_x \xi + \delta	
		\end{equation}
		\begin{equation}
		y = \Lambda_y \eta + \epsilon	
		\end{equation}
		Note that: $x$ and $\xi$ (or $y$ and $\eta$) are not necessarily in the same scale, thus there are additional coefficients, encoded by $\Lambda_x$. 
		
		\item Notations: normally assume that all the variables in the model have zero mean. Thus covariance matrices in the latent variable model: 
		\begin{equation}
		\Phi = \E(\xi \xi^T)	
		\end{equation}
		is the covariance matrix of the endogenous variable, and 
		\begin{equation}
		\Psi = \E(\zeta \zeta^T)	
		\end{equation}
		is the covarance matrix of the random error of the endogenous variables. For the measurement model, 
		\begin{equation}
		\Theta_{\epsilon} = \E(\epsilon \epsilon^T) \qquad \Theta_{\delta} = \E(\delta \delta^T)	
		\end{equation}
		are the covariance matrices of $\epsilon$ and $\delta$ respectively. 
		
	\end{itemize}
	
	SEM representation:
	\begin{itemize}
		\item Path diagram: (Table 2.6) square box for observed variables, circles for latent variables, single-headed arrows for causal relations, disturbrance or errors in unenclosed variables, two-headed arrows for associations. 
		
		\item Reciprocal or feedback relations: allowed in SEM. Represented by two single-headed arrows.  
	\end{itemize}
	
	Problems of SEM: 
	\begin{itemize}
		\item Identification: the basic strategy of SEM estimation is the MOM type of estimator. A SEM implies certain covariance/correlation among all observed variables, let it be $\Sigma(\theta)$, a function of $\theta$, where $\theta$ represents all model parameters - this is the population covariance. The sample covariance meanwhile can be determined from data. Thus we should have: 
		\begin{equation}
		\Sigma = \Sigma(\theta)	
		\label{eq:SEM_basic}
		\end{equation}
		In general, if the number of covariance terms is greater than the number of parameters, we have overdetermination; and if smaller, we have underdetermination. Examples: 
		\begin{itemize}
			\item Overdetermination: suppose we have a model of $p$ variables, now adding one more variable (with one or two more parameters) linked to one node. The increase of the number of covariance terms is $p$, and this leads to overdetermination. 
			\item Underdetermination: given a model of two observables ($X \rightarrow Y$), we could estimate the effect of $X$ on $Y$ by sample $\Cov(X,Y)$. Adding one more latent variable, $X \rightarrow Z \rightarrow Y$, not possible to determine the parameters related to $Z$. Effectively, $\Cov(X,Y)$ can result from the direct effect or indirect effect through $Z$. 
		\end{itemize}
		
		\item Estimation: Equation~\ref{eq:SEM_basic} is the basic equation for estimation (MOM). However, the system may be overdetermined, so we need to find $\theta$ s.t. $\Sigma(\theta)$ is closest to $\Sigma$. This can be done via maximum likelihood or other measures of distance between the two matrices. The inference challenges: 
		\begin{itemize}
			\item Testing some parameters or effects: e.g. through LRT. 
			\item Assessing goodness-of-fit of the model: assessing how good the causal model is. However, in general, we cannot prove the causality even if we have good model fitting (could be other models that fit as well). 
		\end{itemize}
		
		\item Effects: the total effect of one variable on another is the sum of direct and indirect effects (product of multiple terms). Ex. given a model: 
		\begin{equation}
		\xi_1 \xrightarrow{\gamma_{21}} \eta_2 \qquad \xi_1 \xrightarrow{\gamma_{11}} \eta_1	\qquad \eta_1 \xrightarrow{\beta_{21}} \eta_2 
		\end{equation}
		The total effect of $\xi_1$ on $\eta_2$ is thus: $\gamma_{21} + \gamma_{11} \beta_{21}$. 
		
	\end{itemize}
	
	\item{SEM with observed variables}
	
	Reference: [Bollen, Structural Equations with Latent Variables, Chapter 4]
	
	Model specification: 
	\begin{itemize}
		\item Model: no latent variables, let $x$ be $q$-dim. exogenous variables (or explanatory variables) and $y$ be $p$-dim. endogenous variables, we have: 
		\begin{equation}
		y = B y + \Gamma x + \zeta	
		\end{equation}
		where $B$ is $p \times p$ matrix, $\Gamma$ is $p \times q$ matrix, and $\zeta$ is the $p$-dim. error vector. The standard assumption is that errors are uncorrelated with $x$. We denote $\Psi = \E(\zeta \zeta^T)$ the covariance matrix of the errors, and $\Phi = \E(x x^T)$ the covariance matrix of the explanatory variables. 
		
		\item Recursive and nonrecurive models: 
		\begin{itemize}
			\item Recursive models: contain no reciprocal relation or feedback loops. For these models, $B$ is lower triangular matrix (or can be rearranged s.t. it is lower triangular), and the covariance matrix of the errors, $\Psi$, is diagonal. However, the covariance matrix of $x$, $\Phi$, can be nondiagonal. 
			\item Nonrecurive models: reciprocal relation or feedback loops. 
		\end{itemize}
		
		\item Implied covariance matrix: assuming the mean of each variable is 0, we have: 
		\begin{equation}
		\Sigma_{yy}(\theta) = \E(yy^T) = (I-B)^{-1} (\Gamma \Phi \Gamma^T + \Psi) (I-B)^{-T}	
		\end{equation}
		where $-T$ denotes the transpose of the inverse. And similarly, 
		\begin{equation}
		\Sigma_{xx}(\theta) = \Phi \qquad \Sigma_{xy}(\theta) = \Phi \Gamma^T (I - B)^{-T}
		\end{equation}
		
	\end{itemize}
	
	Model identification: 
	\begin{itemize}
		\item $t$-rule: for the model to be identified, the number of parameters ($t$) must be less than the number of free terms in the covariance matrix, so the necessary condition is: 
		\begin{equation}
		t \leq \frac{1}{2}(p+q) (p+q+1)	
		\end{equation}
		The condition is not sufficient because it is possible that some parameters remain unidentified even though overall the number of parameters is small. 
		
		\item Null $B$ rule: a sufficient condition of model identification is that $B = 0$, i.e. the endogenous variables are only caused by exogenous variables. Remark: this is simply the case of multivariate regression. 
		
		\item Recursive rule: if the model is recursive, i.e. $B$ is lower triangular and $\Psi$ is diagonal, then the model is identified. 
		
		\item Order and other conditions: when the model is nonrecursive, a necessary condition for an equation to be identified is that the number of variables excluded from that equation be at least $p - 1$.  
	\end{itemize}
	
	Estimation: important when the model is overidentified. 
	\begin{itemize}
		\item MLE: the joint distribution follows multivariate normal distribution (with mean 0), so the likelihood is a function of $\Sigma(\theta)$. The log-likelihood function according to MVN: 
		\begin{equation}
		\log L(\theta) = - \frac{N(p+q)}{2} \log (2\pi) - \frac{N}{2} \log \abs{\Sigma(\theta)} - \frac{N}{2} \tr(S \Sigma^{-1}(\theta))	
		\end{equation}
		where $S$ is the sample covariance matrix. The log-likelihood at the sample covariance matrix $\Sigma = S$ is given by (replacing $\Sigma$ with S): 
		\begin{equation}
		\log L(S) =	- \frac{N(p+q)}{2} \log (2\pi) - \frac{N}{2} \log \abs{S} - \frac{N}{2} (p+q)
		\end{equation}
		From this, we obtain the likelihood ratio test statistic:
		\begin{equation}
		-2 [\log L(\theta) - \log L(S)] = N \cdot F_{\text{ML}}(\theta)	
		\end{equation}
		where:  
		\begin{equation}
		F_{\text{ML}}(\theta) = \log \abs{\Sigma(\theta)} + \tr(S \Sigma^{-1}(\theta)) - \log \abs{S} - (p+q)
		\label{eq:SEM_obs_MLE}
		\end{equation}
		MLE is equivalent to minimizing the function $F_{\text{ML}}(\theta)$ above (similar to squared error). The confidence interval can be obtained from the asymptotic covariance matrix of $\hat{\theta}$. 
		\begin{itemize}
			\item Remark: in the book, the LRT statistic is $(N - 1) \cdot F_{\text{ML}}(\theta)$. 
		\end{itemize}
		
		\item Least square: unweighted least square method that minimizes the function: 
		\begin{equation}
		F_{\text{ULS}} = \frac{1}{2} \tr \left[ (S - \Sigma(\theta))^2\right]	
		\end{equation}
		And similarly one can define weighted least square. These estimators are intuitive, however, they are not scale invariant, thus the values change with any change of scale. 
	\end{itemize}
	
	Other issues: 
	\begin{itemize}
		\item Causal interpretation of coefficients \& comparison of explanatory variables: the causal interpretation is, when $X$ changes by a unit, the change of $Y$, i.e.
		\begin{equation}
		\beta = \frac{\Delta Y}{\Delta X}	
		\end{equation}
		To compare different explanatory variables, however, we need to make them in the same scale. First, elasticity is defined using the percent change of $X$ and $Y$: 
		\begin{equation}
		\text{Elasticity} := \frac{\Delta Y/Y}{\Delta X / X} = \beta \frac{X}{Y}	
		\end{equation}
		Normally, it is evaluated at $\bar{X}$ and $\bar{Y}$. However, when $X$ and $Y$ are standarized, i.e. zero mean, elasticity is not well-defined. The alternative is to measure the change using the standard deviation as units: 
		\begin{equation}
		\text{Standarized Coefficient} := \frac{\Delta Y / \sigma_Y}{ \Delta X / \sigma_x} = \beta \frac{\sigma_X}{\sigma_Y}	
		\end{equation}
		Note however, in practice, $\beta$, $\sigma_X$ and $\sigma_Y$ needs to be replaced by their sample versions. 
		
		\item Interaction terms: if there is an interaction, say between $X_1$ and $X_2$, we simply introduce a new variable $X_3 = X_1 X_2$, and the rest is similar. However, $X_3$ is no longer normally distributed (the conditional distribution of $Y$ still normal). 
	\end{itemize}
	
	\item{Confirmatory factor analysis}
	
	Reference: [Bollen, Structural Equations with Latent Variables, Chapter 6,7]
	
	Measurement model: 
	\begin{itemize}
		\item Modeling strategy: when we have abstract concepts that cannot be directly measured, we treat them as latent variables (e.g. anxiety, terroism), and provide operational definition, i.e. (multiple) proxy variables that are determined by these concepts. We specify the relation between measurement and the latent variables, as the measurement error model. 
		
		\item Lessons for developing measurement models: example, we have latent factors for democracy at 1960 and 1965, and some observables dependent on democracy. Some modeling lessons: 
		\begin{itemize}
			\item Abstraction: the key concepts, in this case, the level of democracy. 
			\item Causal constraints: e.g. a variable can only influence the variables in a later time point. In this example, the factor at 1965 could not affect observables at 1960. 
			\item Additional relations: e.g. in 1960 and 1965, the measurement model is the same (i.e. the same coefficients of how the observations depend on the latent factor).  
		\end{itemize}
		
		\item Exploratory and confirmatory factor analysis: factor analysis is to use a smaller number of (latent) factors to explain the correlation between observed variables. Exploratory factor analysis (EFA) mainly for determining the number of factors. However, statistical methods of EFA has limited power (better to use causal knowledge). 
	\end{itemize}
	
	Confirmatory factor analysis (CFA) model: 
	\begin{itemize}
		\item Model specification: suppose we have $n$-dim. latent factors (exogenous variables) $\xi$, and $n$-dim. observed variables $x$, our model: 
		\begin{equation}
		x = \Lambda_x \xi + \delta	
		\end{equation}
		where $\delta$ is the measurement error. We generally assume that $\E(\delta) = 0$ and $\E(\xi \delta) = 0$. 
		
		\item Definitions: factor loading - the value of $\Lambda_{ij}$, factor complexity - the number of latent factors that influence an observed variable. 
		
		\item Implied covariance matrix: it is given by: 
		\begin{equation}
		\Sigma(\theta) = \E(x x^T) = \Lambda_x \Phi \Lambda_x^T + \Theta_{\delta}	
		\end{equation}
		where $\Phi$ is the covariance matrix of the latent factors, and $\Theta_{\delta}$ is the covariance matrix of the measurement error terms. We consider a special case: one factor, multiple proxy variables (indicators): 
		\begin{equation}
		x_i = \lambda_i \xi + \delta_i, \qquad i = 1, 2, \cdots, q	
		\end{equation}
		Then we have the covariance terms: 
		\begin{equation}
		\Var(x_i) = \lambda_i^2 \phi + \Var(\delta_i)
		\end{equation}
		where $\phi$ is the variance of $\xi$. And: 
		\begin{equation}
		\Cov(x_i, x_j) = \lambda_i \lambda_j \phi	
		\end{equation}
		
	\end{itemize}
	
	Identification of CFA model: 
	\begin{itemize}
		\item Scale of latent factors: for any laten factor, $\xi_i$, the scale needs to be specified, otherwise, it is obviously not identified. Typically, let $\lambda_i = 1$ for one of the indicator variable of $\xi_i$ (instead of the variance of any observations). 
		
		\item One factor - multiple indicator case: suppose there are $q$ indicators, then there are $\frac{1}{2} q (q+1)$ terms in the sample covariance matrix. The number of free parameters is: 2 per indicator (one for $\lambda$ and one for the variance of $\delta$ term), and one parameter for $\phi$ (the variance of $\xi$), but we need to subtract 1 (the coefficient of one indicator is equal to 1), so the total number of parameters is $2q$. The model is identified if $q \geq 3$. 
		
		\item $t$-rule: let $t$ be the number of parameters, then a necessary condition of model identification is $t < \frac{1}{2} q (q+1)$. 
		
		\item Three-indicator rule: as discussed before, if a factor has three or more indicators and $\Theta_{\delta}$ diagonal, then it is identified. 
		
		\item Two-indicator rule: suppose the load complexity is exactly one (i.e. each observed variable depends only one latent factor), and there are at least two indicators per factor, then the model is identified if each row of $\Phi$ has at least one non-zero off-diagonal element and $\Theta_{\delta}$ is diagonal. 
		\begin{itemize}
			\item Idea: if there are only two indicators, but the factors are correlated, then one can borrow information from indicators of correlated factors (s.t. the effective number of indicators is higher).
			
			\item Example: we have the following model: 
			\begin{equation}
			x_1 \xleftarrow{1} \xi_1 \xrightarrow{\lambda_2} x_2 \qquad x_3 \xleftarrow{1} \xi_2 \xrightarrow{\lambda_2} x_4	\qquad \xi_1 \leftrightarrow \xi_2
			\label{eq:two_indicator}
			\end{equation}
			where the last part specifies correlation between the two latent factors. The model is identified, in particular, we have, the covariance bdtween the two latent factors:
			\begin{equation}
			\phi_{12} = \Cov(x_1, x_3)	
			\end{equation}
		\end{itemize}
		
		\item Local identification using information matrix: $\theta$ is identified at some point if and only if the inverse of the information matrix exists at that point. 
	\end{itemize}
	
	Estimation: 
	\begin{itemize}
		\item Maximum likelihood: Similar to the case of SEM with observed variables (Equation~\ref{eq:SEM_obs_MLE}), we minimize the function: 
		\begin{equation}
		F_{\text{ML}}(\theta) = \log \abs{\Sigma(\theta)} + \tr(S \Sigma^{-1}(\theta)) - \log \abs{S} - q
		\label{eq:factor_model_MLE}
		\end{equation}
		
		\item Improper solutions: the parameter values that are impossible in the population, e.g. negative covariance. This may be caused by several factors: 
		\begin{itemize}
			\item The true values may be close to the boundary, and because of sample fluctuations, the estimated value may appear to be improper. 
			\item Outliers: because of the assumptions made (multivariate normality), an outlier that violates these assmptions may create improper estimates. 
			\item Fundamental problem in the model specification. 
		\end{itemize}
	\end{itemize}
	
	Model evaluation: overall fit 
	\begin{itemize}
		\item Problem: once we formulate an SEM model, we need to evaluate how good the model fits the data (overall model fit), or test specific components/parameters of the model. For the latter, for example: 
		\begin{itemize}
			\item Effect of latent factor on observed variable: test if $\lambda_{ij} = 0$ for some factor $i$ and observed variable $j$. 
			\item Correlation/covariance between latent factors: e.g. in the two-indicator model (Equation~\ref{eq:two_indicator}), test if $\phi_{12} = 0$. 
		\end{itemize}
		
		\item Likelihood ratio test: we are comparing two models: $H_0$ a restricted model using MLE from the SEM model, and $H_1$ the full model using the sample covariance matrix. We use Equation~\ref{eq:factor_model_MLE} and the relation between LRT and $F_{\text{ML}}$: 
		\begin{equation}
		-2 \log(L_0 - L_1) = (N - 1) \cdot F_{\text{ML}} = (N - 1) \cdot \left[\log \abs{\hat{\Sigma}} + \tr(S \hat{\Sigma}^{-1}) - \log \abs{S} - q \right]
		\end{equation}
		where $\hat{\Sigma}$ is computed from the MLE of $\theta$. It follows $\chi^2$ test with the d.f. $\frac{1}{2} q (q+1) - t$. 
		
		\item Residuals: the quality of the fit can be judged by how close $\Sigma(\hat{\theta})$ is to the sample covariance $S$. So a simple statistic is the absolute value of mean (or median) of the elements of the residual matrix $S - \hat{\Sigma}$. May be corrected via: (1) correlation residuals; (2) correct for sample size. 
		
		\item Incremental fit measure: similar to the LRT, instead of comparing a model with the full model using $S$, we compare a maintained model ($m$) with a very restrictive baseline model ($b$). The percent reduction of the error ($F$ function) is a measure of how much gain is produced by using the model $m$:
		\begin{equation}
		\Delta_1 = \frac{F_b - F_m}{F_b}	
		\end{equation}
		However, the measure does not control for: 
		\begin{itemize}
			\item Degree of freedom: clearly, a complex model would have higher percent reduction. 
			\item Sample size: while both $F_m$ and $F_b$ may decline with large $N$, their rate may be different. In particular, $F_b$ may decline more slowly as $N$ increases (a poor model may not benefit much as sample size increases). 
		\end{itemize}
		See the text for various normalizations. 
	\end{itemize}
	
	Model comparison: specific components
	\begin{itemize}
		\item Likelihood ratio test: suppose we compare a restricted model ($r$) and unrestricted model ($u$). Example, for testing the effect size, the restricted model may have some $\lambda$ term equal to 0. The test statistic: 
		\begin{equation}
		-2 \left[ \log L(\hat{\theta}_r) - \log L(\hat{\theta}_u) \right] = (N - 1) (F_r - F_u)	
		\end{equation}
		
		\item Other large-sample tests can also be used, including Score test and Wald test. 
		
		\item Example: two-indicator model, Equation~\ref{eq:two_indicator}. We want to test $H_0: \phi_{12} = 0$. The challenge is that the model $H_0$ is not identified (only two indicators, without correlation). 
		
	\end{itemize}
	
	\item{Error-in-variable (EIV) model} 
	
	Reference: [Fuller, Measurement Error Models], [Casella, Statistical Inference, Chapter 12], [Total Least Squares and Errors-in-Variables Modeling: Bridging the Gap between Statistics, Computational Mathematics and Engineering, Van Huffel, 2004]
	
	EIV model overview: 
	\begin{itemize}
		\item EIV model: our observations are $(x_i, y_i)$, $x_i$ are measurements of the true variables $\xi_i$, with Gaussian errors. In addition, $y_i$ is related to the latent variable by a linear model. So we have: 
		\begin{equation}
		x_i = \xi_i + u_i \qquad u_i \sim N(0,\sigma_{u}^2)	
		\end{equation}
		\begin{equation}
		y_i = \beta_1 \xi_i + \beta_0 + e_i \qquad e_i \sim N(0,\sigma_{e}^2)	
		\end{equation}
		This model is different from the ordinary regression model by the measurement error of explanatory variables. We can write the model in a regression form by eliminating $\xi_i$: 
		\begin{equation}
		y_i = \beta_1 (x_i - u_i) + \beta_0 + e_i = \beta_1 x_i + \beta_0 + (e_i - u_i \beta_1)	
		\end{equation}
		This is not standard regression, however, as $x_i$ is correlated with the error term (covariance equal to $-\beta_1 \sigma_{u}^2$). 
		
		\item Functional and structural model: if we view $\xi_i$ (the true explanatory variables) as unknown constants, the model is known as a functional relationship; if we view $\xi_i$ as random variables and independent of errors, the model is a structural relationship. In particular, if: 
		\begin{equation}
		\xi_i \sim N(\mu_{\xi}, \sigma_{\xi}^2)	
		\end{equation}
		then it is a structural model. 
		
		\item Relationship between the two models: [Casella]
		\begin{itemize}
			\item Consistent estimators in the functional model are also consistent in the structural model. 
			\item If a parameter is identifiable in the functional model, then it is identifiable in the structural model. 
		\end{itemize}
		The intuition is that the functional model is a special case of the structural model: if a parameter is identifiable in the functional model, then we could do some iterative scheme to identify the parameter in the structural model (at each step assuming $\xi_i$ are known). 
	\end{itemize}
	
	Model simplification and identification: 
	\begin{itemize}
		\item Identification of the functional model: let $\theta = (\beta_0, \beta_1, \xi_1, \cdots, \xi_n, \sigma_u^2, \sigma_e^2)$, the likelihood function is: 
		\begin{equation}
		L(\theta|\mathbf x, \mathbf y) = \frac{1}{(2 \pi \sigma_u \sigma_e)^{n}} \exp\left[-\sum_i \frac{(x_i - \xi_i)^2}{2 \sigma_u^2} \right] \exp\left[-\sum_i \frac{(y_i - \beta_1 \xi_i - \beta_0)^2}{2 \sigma_e^2} \right]
		\end{equation}
		The problem is: it does not have a finite maximum. To see this, take $\xi_i = x_i$, and let $\sigma_u^2 \to 0$. Thus we need additional constraints on the parameters, most commonly, this is $\sigma_e^2 / \sigma_u^2 = \lambda$. 
		
		\item Structural model and bivariate normal distribution: we could show that with the structural model, the distribution of $(x_i,y_i)$ follows independent bivariate normal distribution with mean: 
		\begin{equation}
		\E(X) = \mu_{\xi} \qquad \E(Y) = \beta_0 + \beta_1 \mu_{\xi}	
		\end{equation}
		and covariance matrix: 
		\begin{equation}
		\Var(X) = \sigma_{\xi}^2 + \sigma_u^2 \qquad \Var(Y) = \beta_1^2 \sigma_{\xi}^2 + \sigma_e^2 \qquad \Cov(X,Y) = \beta_1 \sigma_{\xi}^2	
		\end{equation}
		
		\item Identification of the structural model: our model has six parameters $(\mu_{\xi}, \sigma_{\xi}^2, \sigma_u^2, \beta_0, \beta_1, \sigma_e^2)$, while the bivariate normal distribution has only five parameters. Suppose we perform MOM parameter estimation, we will have five equations (sample mean and variance) for 6 parameters, thus not all parameters are identified. 
		\begin{itemize}
			\item Some parameters may still be identified, e.g. $\mu_{\xi} = \Bar{X}$, regardless of other parameters. 
			\item Intuition: in the above five equations, suppose we have any value of one parameter, e.g. $\beta_1$, we could solve the others. Intuitively, we could fit the model in different ways: strong correlation between the variables (large $\beta_1$) but large measurement error; or weak correlation (small $\beta_1$) but small measurement error. 
		\end{itemize}
		
		\item Identifiability conditions: the most commonly used model assumes that $\lambda = \sigma_{e}^2 / \sigma_{u}^2$ is known. Other identifiable cases include: the measurement error, $\sigma_u^2$ is known. 
	\end{itemize}
	
	Model with known measurement error: 
	\begin{itemize}
		\item MOM estimation: we consider the structural model. Suppose the variance of the measurement error, $\sigma_{\xi}$ is known. Let the sample means be $\bar{X}$ and $\bar{Y}$, and the same variance/covariance be $S_{XX}$, $S_{XY}$ and $S_{YY}$. Since $(X,Y)$ is normally distributed, the same mean and variance coverger to the population mean and variance. Solving the five equations, we have: 
		\begin{equation}
		\hat{\beta}_1	= (S_{XX} - \sigma_u^2)^{-1} S_{XY}
		\end{equation}
		\begin{equation}
		(\hat{\sigma}_{\xi}^2, \hat{\sigma}_{e}^2) = (S_{XX} - \sigma_u^2, S_{YY} - \hat{\beta}_1 S_{XY})
		\end{equation}
		\begin{equation}
		(\hat{\mu}_{\xi}, \hat{\beta}_0) = (\bar{X}, \bar{Y} - \hat{\beta}_1 \bar{X})	
		\end{equation}
		
		\item Sampling distribution of the estimators: the basic strategy is (1) use Delta Method to express the estimators as linear functions of sample mean and sample covariance; (2) the limiting distribution of sample covariance is given by (extended) CLT. The last step is to replace the true parameter values in the limiting distribution with the MLE (to form the consistent estimator of the limiting distribution).  
		\begin{itemize}
			\item First, we introduce the variables $v_i$: 
			\begin{equation}
			v_i = y_i - \beta_1 x_i - \beta_0 = e_i - \beta_1 u_i, i = 1, \cdots, n	
			\end{equation}
			$v_i$ is similar to the residuals in ordinary linear regression. Clearly, $\E(v_i) = 0$. We could then define the population and sample covariance involving $v_i$: 
			\begin{equation}
			\sigma_{Xv} = \Cov(x_i, v_i) = -\beta_1 \Var(u_i) = -\beta_1 \sigma_u^2	= \sigma_{uv}
			\end{equation}
			\begin{equation}
			S_{Xv} = \frac{1}{n-1} \sum_i (x_i - \bar{x}) v_i	
			\end{equation}
			
			\item Expansion of estimators: Taylor expansion of $\hat{\beta_1}$ as a function of $S_{XY}$ and $S_{XX}$, near the true values $\sigma_{XY}$ and $\sigma_{XX}$: 
			\begin{equation}
			\hat{\beta}_1 \approx \frac{\sigma_{XY}}{\sigma_{XX} - \sigma_u^2} + \frac{S_{XY} - \sigma_{XY}}{\sigma_{XX} - \sigma_u^2} - \frac{\sigma_{XY}(S_{XX} - \sigma_{XX})}{(\sigma_{XX} - \sigma_u^2)^2}
			\end{equation}
			Simplify the equation using $\sigma_{XX} - \sigma_u^2 = \sigma_{\xi}^2$, $\sigma_{XY} = \beta_1 \sigma_{\xi}^2$, and the equations of $S_{Xv}$ and $\sigma_{Xv}$, we have: 
			\begin{equation}
			\hat{\beta}_1 \approx \beta_1 + \frac{1}{\sigma_{\xi}^2} (S_{Xv} - \sigma_{Xv})	
			\end{equation}
			And similarly, we could have: 
			\begin{equation}
			\hat{\beta}_0 \approx	\beta_0 - (\hat{\beta}_1 - \beta_1) \mu_{\xi}
			\end{equation}
			
			\item Limiting distribution: we use the limiting distribution of $S_{Xv} - \sigma_{Xv}$ from CLT, and this allows us to obtain the limiting distribution of $\hat{\beta}_1$ and  $\hat{\beta}_0$. 
		\end{itemize}
		
		\item Theorem: the vector $\sqrt{n}[\hat{\beta}_1 - \beta_1, \hat{\beta}_0 - \beta_0]$ converges in distribution to a normal random vector with zero mean and covariance matrix: 
		\begin{equation}
		\Gamma = \left[ \begin{array}{ll}
		\mu_{\xi}^2 \sigma_{\xi}^{-4} (\sigma_{XX} \sigma_{vv} + \sigma_{Xv}^2) + \sigma_{vv} & -\mu_{\xi}^2 \sigma_{\xi}^{-4} (\sigma_{XX} \sigma_{vv} + \sigma_{Xv}^2)\\
		-\mu_{\xi}^2 \sigma_{\xi}^{-4} (\sigma_{XX} \sigma_{vv} + \sigma_{Xv}^2) & \sigma_{\xi}^{-4} (\sigma_{XX} \sigma_{vv} + \sigma_{Xv}^2)\\
		\end{array} \right]	
		\end{equation}
		For the consistent estimator of $\Gamma$, denoted as $\hat{V}(\hat{\beta}_0, \hat{\beta}_1)$, see [Fuller, Theorem 1.2.1]. 
		
		\item Testing $\beta_1$: a $t$-test of $\beta_1$ is given by: 
		\begin{equation}
		t = (\hat{\beta}_1 - \beta_1) / \sqrt{\hat{V}(\hat{\beta}_1)}	
		\end{equation}
		
		\item Remark: an alternative strategy is to use the asymptotic distribution of MLE (the MOM estimator is also MLE). However, the log-likelihood function is a complex function of the parameters, and its second derivartive wrt. the parameters are even more complex. 
		
		\item Estimating the latent variables: suppose we have the parameter values. We consider two cases:
		\begin{itemize}
			\item Functional model: treating $\xi_i$ as a constant, then we could view $x_i$ and $y_i$ as linear function of $\xi_i$: 
			\begin{equation}
			x_i = \xi_i \cdot 1 + u_i	
			\end{equation}
			\begin{equation}
			y_i = \xi_i \cdot \beta_1 + e_i	
			\end{equation}
			Treating this as a linear model, where $\xi_i$ is the coefficient. 
			
			\item Structural model: the joint distribution of $\xi_i, x_i, y_i$ follows MVN distribution. The problem is thus inferring the conditional distribution of one compoment given the other components in a MVN. 
		\end{itemize}
	\end{itemize}
	
	Functional model with known variance ratio: 
	\begin{itemize}
		\item Least square estimator: the expectation of the LS estimator: 
		\begin{equation}
		\E(\hat{\beta}_1^{LS}) = \E\left[ \frac{\Cov(X,Y)}{\Var(X)} \right] = \beta_1 \frac{\sigma_{\xi}^2}{\sigma_{\xi}^2 + \sigma_u^2}
		\end{equation}
		Clearly the LS estimator is biased, and the ratio: 
		\begin{equation}
		\kappa =	\frac{\sigma_{\xi}^2}{\sigma_{\xi}^2 + \sigma_u^2}
		\end{equation}
		is called the reliability ratio. It is below 1 because of the measurement error of $X$. It is similar to the heritability in the genetic context. 
		
		\item Orthogonal regression for functional model: under the assumption that $\lambda = 1$, the likelihood: 
		\begin{equation}
		L(\beta_1, \beta_0, \sigma_{\delta}^2, \xi_1, \cdots, \xi_n) \propto \sigma_{\delta}^{-2n} \exp\left\{ -\frac{1}{2 \sigma_{\delta}^2} \sum_i \left[ (x_i - \xi_i)^2 + (y_i - \beta_1 \xi_i - \beta_0)^2\right] \right\}
		\end{equation}
		At any value of $\sigma_{\delta}^2$ (it is independent of other parameters), we are minimizing the sum of square of orthogonal distances from data points to thel line. This is the orthogonal regression. 
	\end{itemize}
	
	Structural model with known variance ratio: 
	\begin{itemize}
		\item Bayesian inference for structural model [The Bayesian Estimation of a Linear Functional Relationship, Lindley \& El-Sayyad, JRSSB, 1968]: assume the prior distribution $\xi_i \sim N(0,\tau)$, the likelihood of the $i$-th observation is: 
		\begin{equation}
		p(x_i,y_i|\beta,\sigma_{\xi}^2,\sigma_{\epsilon}^2,\tau) = \int p(\xi_i|\tau) p(x_i|\xi_i,\sigma_{\xi}^2) p(y_i|\xi_i,\beta,\sigma_{\epsilon}^2) d\xi_i	
		\end{equation}
		All the three distributions are normal, and we have $(x_i,y_i)$ follows bivariate normal distribution with zero mean, and: 
		\begin{equation}
		\Var(x_i) = \tau + 	\sigma_{\xi}^2 \qquad \Var(y_i) = \beta_1^2 \tau + \sigma_{\epsilon}^2 \qquad \Cov(x_i,y_i) = \beta_1 \tau	
		\end{equation}
		The posterior distribution and its approximation can be found at [Lindley68]. 
		
	\end{itemize}
\end{enumerate}
%%%%%%%%%%%%%%%%%%%%%%%%%%%%%%%%%%%%%%%%%%%%%%%%%%%%%%%%%%%%
%%%%%%%%%%%%%%%%%%%%%%%%%%%%%%%%%%%%%%%%%%%%%%%%%%%%%%%%%%%%
\chapter{Advanced Statistics}
%%%%%%%%%%%%%%%%%%%%%%%%%%%%%%%%%%%%%%%%%%%%%%%%%%%%%%%%%%%%
\section{Gaussian Process}

Gaussian process in regression: [Murphy, Section 15.1-15.2]
\begin{itemize}
	\item Motivation: when learning a function $f$, regularize $f$ s.t. if $x_i$ and $x_j$ are close, $f_i$ and $f_j$ should be close. Directly regularize a function is of course difficult, but we can consider the data points of $f$ at $x_1, \cdots, x_N$. To predict is then roughly interpolate using given data points. 
	
	\item Model: Figure 15.1, a PGM, let $f_i$ be $f(x_i)$, and $y_i$ be observed response at $x_i$. Our model is: $f_i | x_i$ is given by a Gaussian graphical model, that encourages similarity of $f_i$'s when $x_i$'s are close. And $y_i | f_i \sim N(f_i, \sigma_y^2)$, the noise. Formally, we write $p(f|X) = N(\mu, K)$ where $\mu$ is the mean function, and $K$ is the covariance, given by the kernel, $\kappa(, \cdot ,)$. 
	
	\item The choice of mean function: commonly use $\mu(X) = 0$, then the model simply smooths the function $f$. If we use $\mu(X)$ as a linear function of $X$, then this is similar to linear regression, where the errors are correlated, with the error structure given by $K$. 
	
	\item Inference and prediction: prediction is given by
	\begin{equation}
	p(y_*| y, X, x_*) = \int p(y_* | f, x_*) p(f | X, y) df
	\end{equation}
	We can show that the posterior mean of $f_*$ is:
	\begin{equation}
	\bar{f_*} = \sum_{i=1}^N \alpha_i \kappa(x_i, x_*) \qquad \alpha = K_y^{-1} y
	\end{equation}
	where $K_y = K + \sigma_y^2 I$. So this is similar to nearest neighbor estimates, where the weight of a training data point $i$, depends on $\kappa(x_i, x_*)$. 
	
	\item Impact of hyperparameters: e.g. squared exponential (SE) kernel. It has two parameters, $l$ controls the horizontal scale (smoothness of function) and $\sigma_f^2$ controls the vertical scale of the function. See Figure 15.3. 
	
	\item Estimation of kernel parameters: EB estimation. It is possible to margianlize $f$ to compute $p(y|X) = \int p(y|f) p(f|X) df$. To do parameter estimations, use gradient based methods. Trade-offs between $l$ and $\sigma_f$ can lead to local optimum, see Figure 15.5: smooth function, but large prediction error; or wiggly function with small prediction error. 
\end{itemize}

\section{Spatial Statistics}

Overview of spatial statistics: use disease count data as an example [Wakefield, Disease mapping and spatial regression with count data, Biostatistics, 2007]]
\begin{itemize}
\item Disease mapping problem: estimate the RR of each area.  

\item Spatial regression problem: identify the potential risk factors of disease. 

\item Why need a different approach: e.g. with disease mapping problem, the variance of the estimator is high for low density regions. 

\item General ideas: need to model the variation (spatial in this case) across regions, as well as, the dependency (correlation) between neighbors. The two are generally independent. 
\end{itemize}

Nonspatial model for disease mapping [Wakefield, 2007]
\begin{itemize}
\item Poisson model with Gamma random effects: suppose the count in region $i$ is $Y_i$, and $E_i$ be the expected number (given). The difference of $Y_i$ over $E_i$ is RR, and it is due to fixed effects (known covariates $x_i$) and random effects (variation across regions not explained by fixed effects). Let $\mu_i$ be the fixed effect term and $\theta_i$ be the random effect, then: 
\begin{equation}
Y_i \sim \text{Pois}(E_i \mu_i \theta_i)
\end{equation}
where $\mu_i = \mu(x_i, \beta)$ and 
\begin{equation}
\theta_i \sim \text{Ga}(\alpha, \alpha)
\end{equation}
This prior is chosen s.t. the prior mean is 1. The marginal distribution of $Y_i$ is Negative Binomial: 
\begin{equation}
\E{Y_i} = E_i \mu_i \qquad \Var{Y_i} = \E{Y_i} (1 + \E{Y_i}/\alpha)
\end{equation}
Inference can be done with EB, which estimates the MLE of $\beta$ and $\alpha$. Then $\theta_i$ can be infered, as a weighted combination of $Y_i$ and the prior. 

\item Poisson model with Gamma random effects (slightly different):
\begin{equation}
Y_i \sim \text{Pois}(E_i \theta_i) \qquad \theta_i \sim \text{Ga}(\mu_i \alpha, \alpha)
\end{equation}
where $\mu_i = f(x_i, \beta)$ is the fixed effect. 

\item Poisson model with lognormal random effects: 
\begin{equation}
Y_i \sim \text{Pois}(E_i \mu_i e^{V_i}) \qquad V_i \sim N(0, \sigma_v^2)
\end{equation}
where $V_i$ are area-specific random effects that capture the unexplained log RR in area $i$. 
\end{itemize}

Spatial model for disease mapping: incorporate spatial dependence [Wakefield, 2007]
\begin{itemize}
\item Joint model: let $\mu_i$ be the fixed effect term and $U_i, V_i$ be random effects ($U_i$ non-spatial and $V_i$ spatial), we have
\begin{equation}
Y_i \sim \text{Pois}(E_i \mu_i e^{U_i+V_i})
\end{equation}
The fixed effect term has two components: one due to known covarates, the other due to large-scale spatial trend ($S_i$ be the spatial location):  
\begin{equation}
\log \mu_i = f(x_i, \beta) + g(S_i, \gamma) 
\end{equation}
The non-spatial random effects: $V_i \sim N(0, \sigma_v^2)$. The spatial random effects, the vector $U$, can be modeled as a MVN where correlation depends on the distance $d_{ij}$: 
\begin{equation}
\Var{U_i} = \sigma_u^2 \qquad \text{corr}(U_i, U_j) = \rho^{d_{ij}}
\end{equation}
where $\rho$ is a parameter that determines the extend of correlation. 

\item Conditional model: Intrinsic conditional autoregressive (ICAR) prior, under this model, the random effect $U_i$ depends on its neighbors $\partial_i$: 
\begin{equation}
U_i | U_j, j \in \partial_i \sim N(\bar{U_i}, \omega_u^2 / m_i)
\end{equation}
where $m_i$ is the number of neighbors and $\bar{U_i}$ is the mean of the spatial random effects of neighbors. 

\end{itemize} 

Bayesian Multiscale Models for Poisson Processes [Kolaczyk, JASA, 1999]
\begin{itemize}
	\item Model: number of events $X_i$ in the $i$-th interval, with $X_i \sim \text{Pois}(\Lambda_i)$. Our goal is to estimate spatially smooth $\Lambda_i$. To do that, we partition the data into smaller intervals. At the top level, the total count is $X_{00}$, where the first index is for level, and the second for position. The distribution:
	\begin{equation}
	X_{00} \sim \text{Pois}(\Lambda_{00})
	\end{equation}
	Next, we consider two halves of the total interval, let the counts be $X_{10}$ and $X_{11}$, respectively. Let $R_{10} = \Lambda_{10}/\Lambda_{00}$, the conditional distribution: 
	\begin{equation}
	X_{10} | X_{00} \sim \text{Bin}(X_{00}, R_{10})
	\end{equation}
	and so on. So the total likelihood of all data can be written as the product of conditional distributions, and the model is parameterized by $\Lambda_{00}$, and $R_{jk}$'s. More formally, we have: 
	\begin{equation}
	P(X|\Lambda) = P(X_{00} | \Lambda_{00}) \prod_j \prod_k P(X_{j+1,2k}|X_{jk}, R_{jk})
	\end{equation}
	
	\item Prior of $R_{jk}$: the intuition of $R_{jk}$ is that, to be spatially smooth, most of the time it should be 1/2. So we have this prior: 
	\begin{equation}
	R_{jk} \sim \gamma_{jk} 0.5 + (1 - \gamma_{jk}) B_{jk}
	\end{equation}
	where 
	\begin{equation}
	\gamma_{jk} \sim \text{Ber}(p_j) \qquad B_{jk} \sim \text{Beta}(a_j, a_j)
	\end{equation}
	
	\item Model interpretation: When $p_j$ is large, most of the time, $R_{jk}$ is 0.5, thus we have equal rates. So $p_j$ measures the spatial homogeneity. $a_j$ on the other hand controls $B_{jk}$, corresponding to the ``magnitude of effect'': if $a_j$ is large, then $B_{jk}$ is close to 1/2; if $a_j$ is small, then $B_{jk}$ has large variation. In the case of peak detection, $p_j$ controls the number and width of peaks, and $a_j$ controls the variability of the magnitude of peaks. Ex. broader peaks captured by small $p_j$ at higher level (more spatial heterogeneity) while shorter peaks represented by small $p_j$ at lower level. 
	
	\item Remark: the model favors smoothness, but it cannot learn/enforce certain ``shapes'' of peaks, e.g. it cannot capture the notion that in peak detection, we generally have peaks above the background, but not below. Even if a particular shape occur repeatedly, the model wouldn't capture that. 
\end{itemize}
 
SMASH: multi-scale (multi-seq) Poisson model[Tom Xing, Sep, 2016]
\begin{itemize}
	\item Poisson Model: let $p_j$ be the probability of $j$-th interval in the multi-scale Poisson model [Kolaczyk 1999] (the conditional probability of binomial), under $H_0$, $p_j = 0.5$. In ASH, we model:
	\begin{equation}\label{key}
	\log \frac{p_j}{1-p_j} =  \alpha_j + \beta_j x
	\end{equation}
	where $\alpha$ captures spatial smoothness and $\beta_j$ the effect of treatment $x$. Both $\alpha$ and $\beta$ are defined at many scales, and we shrink $\alpha_j$ towards 0 at each scale. This model can be used for both treatment and control (differed by $x$). Note: does not have an explicit model to shrink more at higher spatial resolution, rather, estimate the parameters using EB. 
	
	\item Use multi-seq to detect difference between samples: the idea is that we have a linear model of $\alpha_j$, with treatment condition as a covariate. LRT to test if the coefficient of the covariate is 0. 
	
	\item Remark: does treatment changes the overall shape, e.g. number and width of peaks or specific peaks? In the model, $\beta_j$ is defined at every location $j$. So any a given scale, we may have log-OR equals to 0 in the background, but $= \beta_j$ in treatment. The parameter $\beta_j$ would reflect the difference at the specific location $j$. 
	
	\item Remark: the peak locations (consider the case of smoothing only, no covariate) may not match the scales we have defined. Ex. a peak may have 3/4 in the first interval and 1/4 in the second interval defined in the multi-scale model. How do we capture this? Or how do we make the results translational invariant?  Idea: try all possible rotations of data. Naive algorithm $O(n^2)$ time; with smarter strategy $O(n \log n)$.    
	
	\item Joint analysis of DNase-seq and ChIP-seq data: let $D$ be DNase data and $C$ be ChIP-seq data. We infer $P(D|C)$, the expected read count at each position of a TFBS (TF footprint). 
\end{itemize}

Smoothing via Adaptive Shrinkage (smash): denoising Poisson and
heteroskedastic Gaussian signals [Xing and Stephens, 2017]
\begin{itemize}
	\item Normal model with known $\sigma$: suppose we have $Y = \mu + \epsilon$, where $\mu$ is mean of $Y$ (spatially smooth) and $\epsilon \sim N(0,D)$, where $D$ is the diagnoal matrix with entries $\sigma_1^2, \cdots, \sigma_T^2$. We transform the data using Discrete Wavelet Transform by multiplying a matrix $W$: 
	\begin{equation}
	W Y = W \mu + W \epsilon
	\end{equation}
	which we write as: $\tilde{Y} = \tilde{\mu} + \tilde{\epsilon}$, where $\tilde{\epsilon} \sim N(0, W D W^T)$. Note that $\tilde{\mu}$ now represent the coefficients of wavelet functions, and are assumed to be sparse. For simplicity, we assume $\tilde{Y}_j$'s are independent, and use ASH prior for $\tilde{\mu}$, and this can be fit with ASH. 
	
	\item Additional assumptions: average results over all $T$ rotations of data. Apply ASH to each level of wavelet coefficients. 
	
	\item Normal model with known $\mu$ but unknown $\sigma$: we consider $Z_t^2 = (Y_t - \mu_t)^2$, then estimating $\sigma$ is now a mean estimation problem.  
	
	\item Normal model with unknown $\mu, \sigma$: iterate, estimate $\mu$ assuming $\sigma$ known; and estimate $\sigma$ assuming $\mu$ known. 

	\item Including covariates in the normal model: we simply consider the residual, and fit the normal model. Could also do this iteratively: from initial fit of spatially smooth model, estimate the residual again, and refit. 
	
	\item Poisson model: the difference with the standard model is, we parameterize with $\alpha_j = \log p_j / (1-p_j)$, which follows ASH prior at each level. 
\end{itemize}

SMASH-GEN: extensions of SMASH [\url{https://dongyuexie.github.io/smash-gen/index.html}, 2018]
\begin{itemize}
	\item Claim: if $X \sim \text{Pois}(\lambda)$, then we can approximate it as: $Y = \log(X)$, and $Y \sim N(\mu, \sigma^2)$ where $\mu = \log(\lambda)$ and $\sigma = 1/\lambda$. The problem of estimating $\lambda$ can be then reduced to estimating $\mu$ under a normal distribution. 
	
	\item Proof: The log-likelihood function of Poisson distribution is: 
	\begin{equation}
	l(\mu) = \log P(x|\mu) = x \log \lambda - \lambda = x \mu - e^{\mu}
	\end{equation}
	We can use Taylor expansion near $\mu_0$ to approximate the LL:  
	\begin{equation}
	l(\mu) \approx l(\mu_0) + l'(\mu_0) (\mu - \mu_0) + \frac{l^{''}(\mu_0)}{2} (\mu - \mu_0)^2
	\end{equation}	
	The derivatives are given by: 
	\begin{equation}
	l(\mu_0) = x \mu_0 - e^{\mu_0} \qquad l'(\mu_0) = x - e^{\mu_0} \qquad l^{''}(\mu_0) = -e^{\mu_0}
	\end{equation}
	The MLE of $\mu$ is $\log(x)$, so we choose $\mu_0 = \log(x)$. Plug in this, we have:
	\begin{equation}
	l(\mu) \approx x \log(x) - x - \frac{x}{2} (\mu - \log(x))^2
	\end{equation}
	This is the log-likelihood of normal RV with mean $\log(x) = \mu$ and variance $1/x = 1/\lambda$. 
		
	\item Intuitions of the algorithm: our analysis suggests that we can focus on fitting $\log(X_i)$, which is roughly normal. We can consider the Taylor approximation of $\log (X_i/\lambda_i)$: 
	\begin{equation}
	\log \frac{X_i}{\lambda_i} = \log\left(1 + \frac{X_i - \lambda_i}{\lambda_i}\right) = \frac{X_i - \lambda_i}{\lambda_i} - \frac{1}{2} \left(\frac{X_i - \lambda_i}{\lambda_i}\right)^2 + \cdots 
	\end{equation}
	This leads to the approximation: 
	\begin{equation}
	\log(X_i) \approx \log \lambda_i + \frac{X_i - \lambda_i}{\lambda_i} \equiv Y_i.
	\end{equation} 
	Now we can write $Y_i = \mu_i + \epsilon_i$, where $\mu_i$ is spatially smoothed mean, and $\epsilon_i$ the heteroscedastic variance: $1/\lambda_i$ (easy to prove). $Y_i$ can be thought of as the `normal component'' of Poisson data. We will develop iterative algorithm to fit $Y_i$.  
	
	\item Algorithm: start with some initial estimate of $\lambda_i$, we 
	fit spatially smoothed model of $Y_i = \mu_i + \epsilon_i$ using SMASH (wavelet smoothing). Then we have new estimate of $\mu_i$, and we update our definition of $Y_i = \mu_i + (X_i - \lambda_i)/\lambda_i$, and do wavelet smoothing again. Repeat this process until convergence. 
	
	\item Remark: 
	\begin{itemize}
		\item The algorithm focuses on fitting $Y_i$ instead of $\log(X_i)$. Intuitively, only $Y_i$ represents the normal part of data, and can be approximated by normal wavelet method. The extra deviation of $\log(X_i)$ from $Y_i$ is introduced by Poisson likelihood, and cannot be captured by normal approximation. 
		
		\item Behavior of the algorithm: suppose we plot $Y_i$ against $i$, with the curve $\log (\lambda_i)$. Initially, the fitted results looks spiky because $\log \lambda_i$ are not smooth (which occurs in the definition of $Y_i$). The deviation of $Y_i$ from $\log (\lambda_i)$ is roughly normal. As the algorithm proceeds, $\log (\lambda_i)$ curve now becomes smooth. The errors are still roughly normal with similar magnitude as before. 
	\end{itemize}

	\item Bayesian interpretation of the iterative algorithm: we should expand near the posterior mean of $\mu$, so intuitively, we should use the spatially smoothed estimate. This gives the iterative procedure above.
	
	%\item Outline of algorithm: suppose we are fitting $X_i \sim \text{Pois}(\lambda_i)$, where $\lambda_i$ should be spatially smooth. We define $Y_i = \log(X_i)$, we can approximate it as: $Y_i = \mu_i + \epsilon_i$, where $\epsilon_i \sim N(0, 1/\lambda_i)$. If we know $\lambda_i$, we can simply fit spatially smooth model wrt. $\mu_i$ using SMASH-normal. However, since $\lambda_i$'s are unknown, this motivates an iterative algorithm: start with initial estimate of $\mu_i$ or $\lambda_i$, fit SMASH-normal, then update the estimate of $\lambda_i$ and repeat SMASH-normal. 
	
	\item Does $X_i = 0$ create a problem? $Y_i$ is still well-defined at $X_i=0$. Intuitively, the normal approximation of Poisson log-likelihood remain valid even when $\lambda$ is small (close to 0). 
	
	\item Including covariates in the Poisson model: suppose the effects of covariates can be accounted for by a constant term $t_i$ for data point $i$. Our model is then: $X_i \sim \text{Pois}(t_i \lambda_i)$. We can now write the approximation as: 
	\begin{equation}
	\log(X_i) \approx \log t_i + \log \lambda_i + \frac{X_i - \lambda_i t_i}{\lambda_i t_i} \equiv Y_i.
	\end{equation}
	So we fit the model $Y_i = \log t_i + \mu_i + \epsilon_i$, where $\epsilon_i$ has variance $1/(\lambda_i t_i)$.  	

	\item Nugget effect: we define $Y_i$ as: 
	\begin{equation}
	Y_i = \log \lambda_i + \frac{X_i - \lambda_i}{\lambda_i} + N(0, \sigma^2)
	\end{equation}
	where the last term is the Nugget effect. This is important for RNA-seq data, large base level variations that are not captured by spatial effects. 
		
	\item Remark: how does the model behave in the case where we have hotspots? Initially, suppose our $\lambda_i$ is flat, then when we fit $Y_i$, in the hotspot region, $Y_i$ is now large (because of large $X_i$). The model has to use a large $\lambda_i$ (mean term) to accommodate, so in next run, it fits larger values of $\lambda_i$ while striving for smoothness, which is achieved by wavelet (the coefficient of the basis function corresponding to the hotspot range would be non-zero).  
	
	\item \textbf{Lesson}: to estimate parameters, we could transform the data (variable substitution), and work on the new data, whose distribution may be simpler. Ex. we approximate a distribution by normal using appropriate variable substitution. 
\end{itemize}

\subsection{Bayesian Disease Mapping: Hierarchical Modeling in Spatial Epidemiology}

Section 5.1-5.2: Model of disease
\begin{itemize}
	\item The case event (Poisson process) model: let $\lambda(s)$ be the continuous density at location $s$, defined over a region $T$. We have events at $s_1, \cdots, s_m$. Our likelihood is: 
	\begin{equation}
	L(s_1, \cdots, s_m|\Psi) = \prod_{i=1}^m \lambda(s_i|\Psi) \exp(-\Lambda_T) \qquad \text{where } \Lambda_T = \int_T \lambda(u|\Psi) du
	\end{equation}
	The proof can be done by discretization of space: the Poisson probability for grids with events is proportional to $\lambda(s_i) \exp(-\lambda(s_i))$, and the probability for grids without events is proportion to $\exp(-\lambda(s))$. 
	
	\item The conditional logistic model: if we have case-control data, with density $\lambda_1(s|\Psi)$ and $\lambda_0(s|\Psi)$. We consider the conditional distribution, and this leads to the Bernoulli distribution for $y_i \sim \text{Bern}(p_i)$, where 
	\begin{equation}
	p_i = \frac{\lambda_1(s_i|\Psi_1)}{1 + \lambda_1(s_i|\Psi_1)}
	\end{equation}
	
	\item Poisson model for count data in small areas: $y_i \sim \text{Pois}(\mu_i)$, with $\mu_i = e_i \theta_i$, where $e_i$ is the expected rate, and $\theta_i$ relative risk. 
	
	\item Model specification: let $\eta_i$ be the log expected rate under Poisson process model (or log relative risk under Poisson model). Typical model for $\eta_i$ is: $\eta_i = x_i \beta + z_i$, where $x_i$ are covariates and $z_i$ random effects. Also non-linear models: e.g. dependency on distance, and polynomial function on coordinates. 
\end{itemize} 

Section 5.4. Correlated heterogeneity models
\begin{itemize}
	\item The count model and let $\theta_i$ be the RR, we have: 
	\begin{equation}
	\log(\theta_i) = x_i \beta + u_i + v_i
	\end{equation}
	where $u_i$ and $v_i$ are correlated and uncorrelated random effects. It is recommended to have both terms. The model is not identifiable, but the sum of $u_i$ and $v_i$ is. 
	
	\item Conditional autoregressive (CAR) model: improper model (ICAR). The prior of $u_i$ is given by: 
	\begin{equation}
	p(u|r) \propto \frac{1}{r^{m/2}} \exp \left[-\frac{1}{2r} \sum_i \sum_{j \in \delta_i} (u_i - u_j)^2 \right]
	\end{equation}
	where $\delta_i$ is the neighborhood of $i$, e.g. all neighbors within a distance. This prior would penalize the difference of neighbors. One can also show that this is the MRF with the weight matrix given by $1/r$ times a matrix with 1 in diagonal, and -1 in neighbors, and 0 elsewhere. The prior of $v_i$ is given by: $v_i \sim N(0, \sigma^2)$, and are independent. The inference (MCMC) is made simpler by the conditional distribution of $u_i$: 
	\begin{equation}
	u_i | u_{-i}, y, x, r, \sigma^2 \beta \sim N(\bar{u}_i, r / n_{\delta_i}) 
	\end{equation}
	where $\bar{u}_i$ is the average of neighbors, and $n_{\delta_i}$ is the number of neighbors. 
	
	\item Proper CAR model: with ICAR model, the conditional expectation of $u_i$ is exactly the same as neighbors. Instead, we can allow the conditional expectation as a regression function of neighbors. So the new model introduce a parameter $\phi$: conditional expectation of $u_i$ is $\phi$ times average of neighbors. See the book for the case with $x_i \beta$. 

\end{itemize}

Section 5.6: model comparison and diagnosis
\begin{itemize}
	\item Model comparison and evaluation: use posterior predictive loss (PPL). For data point $i$ (test data), suppose $\hat{y}_{ij}$ is the j-th prediction of $i$ from posterior sample, the PPL is defined as the average loss over posterior sample of prediction: 
	\begin{equation}
	PPL_i = \frac{1}{G} \sum_j f(y_i, \hat{y}_{ij})
	\end{equation}
	where $f(\cdot)$ is the loss function. One can inspect PPL over testing data to explore the model goodness-of-fit. 
	
	\item Assessing spatial correlation (autocorrelation) of residuals: we want to test if there is any remaining spatial structure in the data. The most common auto-correlation statistic is Moran's I: 
	\begin{equation}
	I = e^T W e / e^T e
	\end{equation}  
	where $e$ is the residual (standardized), and $W$ the 0-1 adjacency matrix (diagonal 0). Note the definition is similar to correlation between two vectors. In the special case where only adajcent regions have $w_{ij} = 1$, and the rest 0, this is the correlation between adjacent regions. We can estimate it by fitting a regression: 
	\begin{equation}
	e_i = a_0 + \rho e_i^* + \epsilon_i
	\end{equation}
	where $e_i^* = \sum_{j \neq i} w_{ij} e_j$. 
\end{itemize}

Chapter 6: Disease cluster detection
\begin{itemize}
	\item Definitions of clusters: (1) hot-spots: could be on single areas; (2) pre-defined groups, and (3) clusters defined with residuals.
	
	\item Residuals: under conditional logistic model, let $y_i$ be count (0 or 1) and $p_i$ be the model prediction, then the residual can be defined as: 
	\begin{equation}
	r_i = \frac{y_i - \hat{p_i}}{\sqrt{\hat{p_i} (1-\hat{p_i})}}
	\end{equation}
	Under the Poisson count models, suppose our model is $y_i \sim \text{Pois}(e_i \theta_i)$, and we estimate $\theta_i$ using the model, the residual defined as: 
	\begin{equation}
	r_i = \frac{y_i - e_i \hat{\theta_i}}{\sqrt{e_i \hat{\theta_i}}}
	\end{equation}
	The idea of Bayesian residuals: we obtain posterior sample of parameters, and hence $\hat{p}_i$ or $\hat{\theta}_i$, so we have a posterior sample of $r_i$. This allows us to define $P(r_i > c)$ for some threshold $c$.  
	
	\item Cluster models: suppose we believe the data has $K$ clusters in terms of RRs. For each cluster, it generates some RR in its neighborhood. For Poisson count data, $y_i \sim \text{Pois}(e_i \theta_i)$, our model becomes: 
	\begin{equation}
	\log \theta_i = \alpha_0 + \alpha_1 \sum_{j=1}^K \phi_j h(C_i - c_j; \tau_h)
	\end{equation}
	where $\phi_j$ is the RR of cluster $j$ (modeled as random effects), and $h(\cdot)$ captures the influence of clusters. $C_i$ are positions of $i$, and $c_j$ the cluster centroid. Two special cases: $h(\cdot)$ is normal, or is constant when a point is within a neighborhood of $c_j$ and 0 otherwise. 
	
	\item Partition models and tree models: all data can be partitioned, and within a partition, RRs are constant, and modeled as random effects. The partition model can be extended to tree models. 
	
\end{itemize} 

%%%%%%%%%%%%%%%%%%%%%%%%%%%%%%%%%%%%%%%%%%%%%%%%%%%%%%%%%%%%
\section{Functional Analysis and Networks}

Bayesian Inference and Testing of Group Differences in Brain Networks [Durante \& Dunson, arXiv, 2015]
\begin{itemize}
	\item Problem: suppose we have brain connectivity data of individuals and we also have phenotypic variable (e.g. creatitivity) for individuals. Find the difference of brain network between the two groups (high vs. low creativitiy). The connectivity data can be represented as networks of $V$ nodes ($V=68$), $A_i$, and the phenotype data $y_i$. 
	
	\item Existings methods: 
	\begin{itemize}
		\item Test individual edges: whether the edge presence correlates with $y$. Limitation: need multiple testing correction, not taken into account the network dependency structure, thus lose power. 
		
		\item Summary statistics of graphs, and correlation with phenotype: e.g. degree, other topological measures. Limitation: lose information of specific regions.  
	\end{itemize}
	 
	\item Intuition: the network edges are highly correlated: for some sets of edges (regions), all edges are either 0 or 1 simultaneously in one individuals. This motivates the use of factor analysis: relatively few latent variables explain the covariance pattern of many variables (edge occurrence). 
	 
	\item Model: directly model the network-valued variable $A_i$, conditioned on $y_i$. The naive representation of $A$ has $2^{V(V-1)/2}$ variables (number of possible graphs). Propose a mixture of low-rank factorizations: each component of the mixture specifies the edge probability $\pi_l$ for edge $l = 1, \cdots, V(V-1)/2$. To model the $h$-th component, $\pi^{(h)}$, assume it is generated from $S^{(h)}$ (logistic regression model), which has a low-rank factorization. 
	
	\item \textbf{Lesson}: whenever we have many random variables, some of which may be correlated, we can use latent variable model (factor analysis). 
\end{itemize}

Bayesian Functional Quantile regression (FQR) [Yusha Liu from Jeff Morris’ group]
\begin{itemize}
	\item Problem: MS data, comparison of cancer vs. normal.
	
	\item Motivation for quantile regression: two distributions may have similar mean, but differ in other quantiles (e.g. top quantile). Do quantile regression $Y = X \beta^{\tau} + \epsilon^{\tau}$ where $\beta$ is defined as the effect on a certain quantile. Typically we do this for different $\tau$'s (instead of running on multiple quantiles in parallel). 
	
	\item Basic model: model the entire dataset, response at $t$
	\begin{equation}
	Y(t) = X B^{\tau}(t) + E^{\tau}(t)
	\end{equation}
	where $X$ is covariates (cancer or normal), $B^{\tau}(t)$ is the effect ot treatment on the quantile $\tau$ at point $t$. Error $E^{\tau}(t)$: asympotic Laplacian (AL) distribution. For simplicity, drop $\tau$ in the notations. 
	
	\item Dealing with spatial continuity of effects: the effects of $B$ of a covariate should be spatially correlated. Model $B(t)$ as a wavelet. Let $B^*_{ajh}$ be the coefficient of covariate $a$, of scale $j$ and specific wavelet $h$. Use global-local prior of wavelet coefficients to shrink most to 0 and borrow information across the wavelets of the same scale
	\begin{equation}
	B^*_{ajh} \sim N(0, \lambda_{ajh}^2 \psi_{aj}^2)
	\end{equation}
	where $\lambda_{ajh}$ follows $g_1$ prior (global), e.g. Laplace prior, and $\psi_{aj}$ follows $g_2(\Psi_{aj})$ prior (local).  
		
	\item Discussion: how to do predictions if $X$ differs between groups of $Y$ only in quantile, but not mean?
	
	\item Remark: in the error model, still independent across $t$’s.
	
	\item Lesson: quantile regression can capture the effects that only change extreme quantiles but not mean. The challenge is to model error distribution. 
	
	\item \textbf{Lesson}: Global-local prior can allow one to borrow information across groups of variables. Similar to group-Lasso?
	
	\item \textbf{Lesson}: modeling of functional data, to capture spatial continuous effects using wavelet transform. Write the true effects as sums of wavelets and learn about the wavelet coefficients.
\end{itemize}
%%%%%%%%%%%%%%%%%%%%%%%%%%%%%%%%%%%%%%%%%%%%%%%%%%%%%%%%%%%%
\section{Misc. Methods}

Nonparametric methods: 
\begin{itemize}
	\item What is nonparametric methods? Defined by the lack of parametric models of the underlying process. The result would not depend on the parametric distribution (thus distribution-free), and the test can be applied to some general statement of the population without using parameters, e.g. the trend, the randomness, etc. (nonparametric test). 
	
	\item Advantages of nonparametric methods: robustness to the underlying distribution, usually very generally and applicable. The computation and null distribution may often be simple as well. 
	
	\item Comparing tests: robustness is not a main criterion, but power is. However, it may be difficult to compare nonparametric tests because the power would generally depend on the alternative hypothesis, which is not known (the reason why nonparametric test is used in the first place). 
\end{itemize}


\begin{enumerate}

\item{Modeling extra evidence in ranking and prediction with latent variables}

Problem: some examples of a common/generic problem: 
\begin{itemize}
	\item Predict regulatory sequences from expression pattern: suppose we want to predict or rank sequences, $S_i$, that predict the expression of gene $G_j$, denoted as $y_j$. The model of how $S_i$ is releated to $y_j$ is available, and now we want to incorporate extra evidence of the sequences, $x_i$, which could be conservation, distanct to TSS, histone modification, etc. 
	\item Ranking pages to some query: the goal is to rank pages, $P_i$, that are relevant to query $Q_j$. The relevance function is defined according to the content, and now we want to incorporate evidence of pages, such as page importance (link structure), time, etc. 
\end{itemize}

Model: 
\begin{itemize}
	\item Probabilistic model: take the example of regulatory sequence prediction, let $Z_{ij}$ be the indicator variable of whether $S_i$ regulates $G_j$. Our model is: for any given gene $G_j$: first sample $Z_{ij}$ in all $S_i$'s; and for the chosen sequence, generate $y_j$ from $S_i$. Without extra evidence, the prior probability $P(Z_{ij})$ is uniform; with extra evidence, we model $Z_{ij}$ via logistic regression of $x_i$, where the regression coefficients would favor sequences that are more conserved, close to TSS, etc. 
	
	\item Inferring hidden indicator variables: the posterior probability of a sequence: 
	\begin{equation}
	P(Z_{ij} = 1| x_i, y_j) \propto P(Z_{ij} = 1| x_i) P( y_j | Z_{ij} = 1)	
	\end{equation}
	where the first term is the prior probability and the second the evidence of $S_i$. Take log. of the equation, and we note that the final score is the sum of the prior evidence and the LL score. If we treat the LL score as one feature of $S_i$, this is similar to a classification problem based on all features (the log. of prior is approximately a linear function, if prior is logistic regression). The difference here is: inference could be done without any training data. 
	
	\item Extensions: the distribution of $Z_{ij}$ can further include the properties of $G_j$ or $Q_j$. Ex. for certain categories of queries, certain features of pages are generally important. 
\end{itemize}

Alternative model based on prior distribution of parameters: 
\begin{itemize}
	\item Modeling prior: let $\beta_{ij}$ be the influence of $S_i$ on $G_j$, the extra evidence can then be modeled as the distribution of $\beta_{ij}$, which should be higher with extra favorable evidence, e.g. through a linear regression on features representing the extra evidence. 		  	\item Comparison with latent variable approach: in the latern variable approach, the assumption is there is only one that has true influence, which needs to be determined; and in prior modeling approach, each could have influence, while the extent of influence may vary. Ex. in finding causal variation in GWAS, the latent variable approach may be more appropriate [Veyrieras \& Pritchard, PG, 2008]; while in finding the possible regulators of expression of many genes, the prior modeling approach may be better [Lee \& Koller, Lirnet, PG, 2009]. 
\end{itemize}

Remark:
\begin{itemize}
	\item The difficulty of this type of problems is the extra evidence cannot be easily modeled/connected to the data. The idea here is to introduce latent variables ($Z_{ij}$), and the extra evidence is incorporated via influencing $Z_{ij}$. 
	\item Latent variable modeling: a general strategy of probabilistic modeling. The usual models such as regression, stochastic models, etc. can be applied to the latent variables. One special case: the labels of data (classes) are missing. 
\end{itemize}

\item{Time series analysis}

Autocorrelation and cross-correlation [Modern Applied Statistics with S, Section 14.1]: 
\begin{itemize}
	\item Aim: assess the correlation of two time series, allowing time-lag between the two. 
	
	\item Autocorrelation: we start with the case of correlation within a time series. Suppose we have a series $X(\tau)$, where $\tau = 1, 2, \ldots, n$, we want to see the correlation of this series with $X(\tau+t)$: 
	\begin{equation}
	\gamma_X(t) = \text{Cov}(X(\tau+t),X(\tau))	
	\end{equation}
	\begin{equation}
	\rho_X(t) = \text{Corr}(X(\tau+t),X(\tau))
	\end{equation}
	Assuming the signal is second-order stationarity, i.e. both quantities do not depend on $\tau$, we have the same mean for $X(\tau)$ and $X(\tau+t)$, let it be $\bar{X}$. We have the estimator for $\gamma_X(t)$: 
	\begin{equation}
	c_X(t) = \frac{1}{n} \sum_s (X(s+t) - \bar{X}) (X(s) - \bar{X})	
	\end{equation}
	where $s$ takes the range $1, \ldots, n-t$ if $t>0$ and $1-t, \ldots, n$ if $t < 0$. We note that, $\text{Var}(X)$ is exactly $c_X(0)$ (auto-covarance), so the estimator of $\rho_X(t)$ is given by: 
	\begin{equation}
	r_X(t) = \frac{c_X(t)}{c_X(0)}	
	\end{equation}
	
	\item Cross-correlation: when we have two time series $X(\tau)$ and $Y(\tau)$ observed on the same interval, we could do similar analysis. The estimators for covarance and correlation are: 
	\begin{equation}
	c_{XY}(t) = \frac{1}{n} \sum_s (X(s+t) - \bar{X}) (Y(s) - \bar{Y})	
	\end{equation}
	\begin{equation}
	r_{XY}(t) = \frac{c_{XY}(t)}{\sqrt{c_X(0) \cdot c_Y(0)}}	
	\end{equation}
	
	\item Remark: the implementation in R follows the above definitions. However, the common/mathematical definition is slightly different (which assumes mean is 0), see Wiki. 
\end{itemize}

\item{Sequential analysis}

Reference: \texttt{http://www.hsph.harvard.edu/betensky/bio276.html}

Sequential testing ideas: [Wiki, Sequential probability ratio test]
\begin{itemize}
	\item Model: suppose we are testing $H_0: \theta = \theta_0$ vs. $H_1: \theta=\theta_1$. Let $S_n$ be the log-likelihood ratio at data points from 1 to $n$. The stopping rule is that: $S_n < a$ or $S_n > b$. To determine the boundary, $a$ and $b$ should be chosen s.t. the type I and II error are satisfied, i.e. 
	\begin{equation}
	P(S_n < a \text{ or } S_n > b|\theta_0) \leq \alpha \quad P(S_n < a \text{ or } S_n > b|\theta_1) \leq \beta
	\end{equation}
	
\end{itemize}

Brownian motion in sequential testing: 
\begin{itemize}
	\item Brownian motion: $W(t)$ follows the properties: (1) $W(0) = 0$. (2) $W(t) - W(s)$ follows normal distribution $N(\mu(t-s), \sigma^2 (t-s)$. (3) For any $t_i$ and $s_i$, $W(t_i) - W(s_i)$ are independent. It is easy to prove that: 
	\begin{itemize}
		\item The joint distribution of $W(t_1), \cdots, W(t_n)$ is Gaussian. 
		\item $W(t)$ is Gaussian with mean $\mu t$ and variance $\sigma^2 t$. 
		\item The covariance between $W(t_i)$ and $W(t_j)$ is $\sigma^2 t_i$ (assuming $t_i$ is small - the shared time between the two random variable). To see this, we write $W(t_j) = W(t_i) + (W(t_j) - W(t_i))$, where the two terms are independent, so the covariance is simply the variance of $W(t_i)$. 
	\end{itemize}
	
	\item Sequential testing by Brownian motion: example, suppose we have $X_1, \cdots, X_n$ iid. $N(\mu,1)$ and we are testing if $\mu=0$. The test will be based on the partial sum: $S_n = \sum_i X_i$. Then $S_n$ is a random walk (discrete), and the contiuous relaxation leads to Brownian motion. 
\end{itemize}

\item{Hypothesis testing of correlations}

Pearson's correlation coefficient [Correlation, Wiki]:
\begin{itemize}
	\item Pearson's correlation coefficient: defined for two random variables $X$ and $Y$, as: 
	\begin{equation}
	\rho_{X,Y} = \frac{\text{Cov}(X,Y)}{\sigma_X \sigma_Y}	
	\end{equation}
	When $X$ and $Y$ are jointly normal, they are independent if and only if correlation equals zero. But if not normal, this is not true: independent then $\rho = 0$, but $\rho = 0$ does not necessarily mean they are independent. The sample correlation is computed by: 
	\begin{equation}
	r_{xy} = \frac{\sum{(x_i - \bar{x})(y_i - \bar{y})}}{(n-1) s_x s_y}	
	\end{equation}
	where $\bar{x}$ and $\bar{y}$ are sample means, and $s_x$, $s_y$ are sample standard deviations. Thus $r_{xy}$ is also sample covarance divided by the product of sample deviations. 
	
	\item Interpretation of Pearson's correlation coefficient: 
	\begin{itemize}
		\item Geometric interpretation: if the vectors of samples are standardized (shifted by sample mean), then it is the cosine of the angle between two vectors. 
		\item Linear regression: do a linear regression of $y$ on $x$, then the coefficient is $r s_y / s_x$. Furthermore, the coefficient of determination (the variance explained) is the square of correlation coefficient. 
	\end{itemize}
	
	\item Application of Pearson's correlation coefficient: it measures the strength of a linear relationship between two variables that are normal. May not work well if the assumptions are not held, for example, Anscombe's quartet. In particular, one outlier may be enough to produce a high correlation coefficient, even though the relationship between the two variables is not linear.
\end{itemize}

Non-parameteric correlations [Correlation, Wiki]:
\begin{itemize}
	\item Spearman's correlation coefficient: a special case of the Pearson product-moment coefficient in which two sets of data $X_i$ and $Y_i$ are converted to rankings $x_i$ and $y_i$ before calculating the coefficient. Significance: best by permutation test. 
	
	\item Mutual information: Formally, the mutual information of two discrete random variables X and Y can be defined as:
	\begin{equation}
	I(X;Y) = \sum_{x \in X, y \in Y} p(x,y) \log \frac{p(x,y)}{p_1(x) p_2(y)}	
	\end{equation}
	where $p_1(x)$ and $p_2(y)$ are marginal distributions. Normally base 2 is chosen for $\log$ function. Mutual information is a measure of dependence in the following sense: $I(X; Y) = 0$ if and only if $X$ and $Y$ are independent random variables. 
\end{itemize}

Hypothesis testing of correlation coefficients:
\begin{itemize}
	\item Significance of correlation coefficient by Fisher transformation: let $N$ be the sample size, and $r$ be the sample correlation coefficient, define the transformation: 
	\begin{equation}
	z = \frac{1}{2} \log \frac{1+r}{1-r}	
	\end{equation}
	If $(X,Y)$ has a bivariate normal distribution, then $z$ is approximately normally distributed with mean $\frac{1}{2} \log \frac{1+\rho}{1-\rho}$, and standard deviation: $\frac{1}{\sqrt{N - 3}}$. This could be used for constructing confidence interval for $\rho$. 
	
	\item Difference between two correlation coefficients from independent samples: let $r_1$ and $r_2$ be correlation coefficients of two independent samples, first convert them to $z_1$ and $z_2$ respectively with Fisher transformation, then the statistic is $z_1 - z_2$, and its standard error is 
	\begin{equation}
	\sigma_{z_1-z_2} = 	\sqrt{\frac{1}{N_1 - 3} + \frac{1}{N_2 - 3}}
	\end{equation}
	
	\item Difference between two correlation coefficients from dependent samples: suppose we want to test if $r_{XY}$ is significantly different from $r_{ZY}$. Let $n$ be the number of points, compute: 
	\begin{equation}
	t = (r_{XY} - r_{ZY}) \cdot \sqrt{\frac{(n-3) (1 + r_{XZ})}{2 (1-r_{XY}^2 - r_{XZ}^2 - r_{ZY}^2 + 2 r_{XY} r_{XZ} r_{ZY})}}
	\end{equation}
	Then $t$ should be $t$ distribution with degree of freedom $n-3$. 
	
	\item Reference: 
	\begin{itemize}
		\item Google ``Confidence Interval on Pearson's Correlation'' and ``Confidence Interval, Difference between Independent Correlations''. 
		\item Cohen \& Cohen, Applied Multiple Regression/Correlation Analysis for the Behavioral Sciences, 2003, Section 2.8, and 5.1.
		\item \url{http://talkstats.com/showthread.php?t=9011} Or, Blalock, H., 1972. Social Statistics. NY: McGraw-Hill. Page 406-7
		\item Google ``How to compare sample correlation coefficients drawn from the same sample''
	\end{itemize}
\end{itemize}

\end{enumerate}
%%%%%%%%%%%%%%%%%%%%%%%%%%%%%%%%%%%%%%%%%%%%%%%%%%%%%%%%%%%%
%%%%%%%%%%%%%%%%%%%%%%%%%%%%%%%%%%%%%%%%%%%%%%%%%%%%%%%%%%%%
\chapter{Machine Learning}

\section{Introduction to Statistical Machine Learning}
	
Reference: [Bishop, Pattern Recognition and Machine Learning, Chap. 1], [Hastie, Elements of Statistical Learning, Section 2.3], [Murphy, Chapter 1]

Challenges of learning: 
\begin{itemize}
	\item Model over-fitting: complex models produce very small training error, but have poor generalization performance. 
	\begin{itemize}
		\item Example, in polynomial curve fitting, high order of polynomial leads to overfitting. Intuitively, the more flexible polynomials with larger values of $M$ (order) are becoming increasingly tuned to the random noise on the target values.
		\item Maximum-likelihood method for parameter estimation suffers from over-fitting. 
	\end{itemize}
	
	\item Curse of dimensionaility: at high dimension, the neighbors of any point must be sparse, thus it is theoretically difficult to apply local regression/smoothing. 
\end{itemize}

Paradigms of supervised learning: 
\begin{itemize}
	\item Problem: predict response $Y$ from $X$, given data $(x_i, y_i)$, where $1 \leq i \leq n$, and the dimensionality of $X$ is $p$. 
	
	\item Model-based learning: fit a global model that explains the training data, e.g. linear models with least square fitting. 
	\item Instance-based learning: the prediction on $x$ is determined by the neighbors of $x$ in the training data, e.g. k-nearest neighbor method (KNN): 
	\begin{equation}
	\hat{Y}(x) = \frac{1}{k} \sum_{x_i \in N_k(x)} y_i	
	\end{equation}
	where $N_k(x)$ is the neighborhood of $x$ defined by $k$ nearest points of $x$ in the training sample. 
\end{itemize}

Extending simple methods for supervised learning: 
\begin{itemize}
	\item Kernel methods that uses weights that decrease smoothly to zero with the distance to the target point, rather than 0/1 as in nearest neighbor method (discrete selection tends to lead to high variance). 
	\item Local regression that fits linear models by locally weighted least squares.  
	\item Basis expansion of the linear methods. 
\end{itemize}

Unsupervised learning: problems and ideas
\begin{itemize}
	\item Statistical perspective: we are given the output data only, but not input data (alternatively, only data, but not labels). The goal is to estimate the density, $p(x|\theta)$, from the data of $x$. This is in contrast to the supervised learning problem, where the goal is to estimate $p(y|x,\theta)$. The challenge is when $x$ has a large number of dimensions,and there is no simple parameteric form of the distribution of $x$, how to estimate the density. 
	
	\item Clustering: the simplest structure to take advange of is that some data points form clusters. Ex. in the height-weight data, there is a natural structure: two genders (clusters).  
	
	\item Discovering latent factors: more generally, to impose structure on the data, one assumes that the data is generated from some latent variables, and in the space of the latent variables, the data points fall in a low-dim. space (dim. reduction). 
	
	\item Graph structure of the variables: the dependence and/or correlation between variables, or causal models. 
	
	\item Imputation: the goal is to infer the missing data from given data. Ex. collaborative filtering - missing matrix.
\end{itemize}

Bayesian methods and roughness penalty: penalize the complex models (regularization) [Hastie, Elements of Statistical Learning, Section 2.7-2.8]
\begin{itemize}
	\item In general, the function $f$ is chosen to minimize: 
	\begin{equation}
	PRSS(f; \lambda) = RSS(f) + \lambda J(f)	
	\end{equation}
	where $RSS(f)$ is the residum sum of square of $f$ in the training data, and $J(f)$ penalizes functions that change rapidly over small regions. For example, the cubic smoothing spline: 
	\begin{equation}
	J(f) = \int [f''(x)]^2 dx	
	\end{equation}
	
	\item These methods can be cast in a Bayesian framework, where the penalty $J$ correponds to log-prior, and minimizing $PRSS$ amounts to finding the posterior mode. 
\end{itemize}

Kernel methods and local regression: e.g. Nadaraya-Watson kernel regression method predicts $y$ of a point $x_0$ by averaging $y_i$'s weighted by the distance of $x_i$ to $x_0$: 
\begin{equation}
\hat{f}(x_0) = \frac{\sum_i K_{\lambda}(x_0, x_i) y_i}{\sum_i K_{\lambda}(x_0, x_i)}	
\end{equation}
where $\lambda$ controls the size of the neighborhood. 

Cross validation [Hastie, Section 7.10]: 
\begin{itemize}
	\item Cross validation: divide the data into K equal parts. For the $k$-th part, train the model on the rest $K-1$ parts and calculate the prediction error of the fitted model when predicting the $k$-th part of the data. We combine the $K$ estimates of the prediction error. 
	
	\item Cross validation for model selection: given a set of models $f(x,\alpha)$ indexed by a tuning parameter $\alpha$, we have $CV(\alpha)$ as the estimates of the prediction error of the model $f(x,\alpha)$. We should choosen $\alpha$ that minimizes $CV(\alpha)$, and the final model is $f(x,\hat{\alpha})$.
\end{itemize}
%%%%%%%%%%%%%%%%%%%%%%%%%%%%%%%%%%%%%%%%%%%%%%%%%%%%%%%%%%%%	
\subsection{Assessing Estimators and Statistical Decision Theory} 

Ref: [Hastie, Elements of Statistical Learning, Section 2.4]
	
Assessing estimators: 
\begin{itemize}
	\item Mean squared error (MSE): suppose $W$ is an estimator of $\tau(\theta)$, where $\theta$ stands for model parameter(s). Then $W$ can be assessed by its MSE: 
	\begin{equation}
	\text{MSE(W)} = E(W - \tau(\theta))^2
	\end{equation}
	Note the frequentist interpretation: the expectation is averaged over all possible datasets generated from the probability distribution. 
	
	\item Unbiased estimator: often look for unbiased estimator of $\theta$, i.e. $E(\hat{\theta}) = \theta$. However, $\tau(\hat{\theta})$ is not necessarily an unbiased estimator of $\tau(\theta)$: 
	\begin{equation}
	E(\tau{\theta}) \neq \tau(E \hat{\theta}) = \tau(\theta)	
	\end{equation}
	
	\item Prediction problem: to predict the value of a function for a given $x_0$: $\hat{y} = f(x_0;\hat{\theta})$, one can view this as estimating a function of parameters. Ex. in linear regression problem, estimating $\mathbf{a}^T \theta$, this is prediction for a new data point $\mathbf{x_0} = \mathbf{a}$. 
	
\end{itemize}
	
Loss function approach to regression: 
\begin{itemize}
	\item Loss function criterion: the function should be chosen to minimize the expected loss or expected prediction error (EPE) over the joint distribution of $(X,Y)$, $p(x,y)$. For regression problem, the loss function is often chosen as squared error loss: $L(Y, f(X) = (Y - f(X))^2$. This leads to the criterion for choosing $f$ (suppose $\hat{f}$ is a predictor function): 
	\begin{equation}
	\text{EPE}(\hat{f}) = E(Y-\hat{f}(X))^2 = \int{(y-\hat{f}(x))^2 p(x,y) dx dy}	
	\label{eq:EPE}
	\end{equation}
	Note that in this equation, $\hat{f}$ is a deterministic function; if it is estimated from data $D$, then the EPE of $\hat{f}$ also needs to be averaged over $D$ (see bias-variance decomposition below). 
	
	\item Optimal pointwise predictor for squared loss [Murphy, Section 5.7]: at a given $x$, one can show that with squared loss, the optimal predictor of $x$ is: 
	\begin{equation}
	\hat{f}(x) = E(Y|X=x)	
	\end{equation}
	assuming that the true distribution is known. To see this, consider $L(y, a) = (y-a)^2$ where $y$ is observed response and $a$ predicted. The posterior expected loss is given by: 
	\begin{equation}
	\rho(a|x) = \E[(y-a)^2|x] = \E(y^2|x) - 2 a \E(y|x) + a^2
	\end{equation}
	The optimal estimator is thus: 
	\begin{equation}
	\frac{\partial \rho(a|x)}{\partial a} = -2 \E(y|x) + 2a = 0 
	\end{equation}
	Solving this leads to $\hat{y} = \E(y|x)$.
	
	\item Remark: 
	\begin{itemize}
		\item The difficulty of applying this theorem is: the data may not contain the value of $X$ to be predicted, thus need assumptions about how the information of related points can be used to make inference for an unseen $X$. 
		\item Some methods are motivated by direct approximation of conditional mean, e.g. the KNN method. 
	\end{itemize}
	
	\item Optimal predictor for $L_1$ (absolute) loss: we have $L(y,a) = \abs{y -a}$. The optimal predictor is the median of $p(y|X)$. 
\end{itemize}
	
Loss function approach to classification [Murphy, Section 5.7]: 
\begin{itemize}
	\item Zero-one loss function: the loss function is $L(y,a) = 1$ if there is a mis-classification and 0 otherwise. This function can be generalized to include different costs for FPs and FPs. The optimal pointwise prediction is given by maximizing the posterior probability of the class label: 
	\begin{equation}
	\hat{y}(x) = \text{argmax}_g P(g|X=x)
	\end{equation}
	Proof: given $X = x$, the problem is then to find the best point approximation of the conditional distribution $P(Y|X = x)$, and this is given by the Theorem of Point Approximation of Discrete RV. 
	
	\item Surrogate loss function - \textit{logloss}: 0-1 loss function is not smooth, so difficult for optimizers to work. In practice, log-loss function may be preferred. We consider binary logistic regression $y_i \in \{-1,1\}$. Our \textit{decision function} is defined as the log odds ratio: 
	\begin{equation}
	f(x_i) = \log \frac{P(y_i = 1 | x_i)}{P(y_i = -1|x_i)}
	\end{equation}
	The optimizer will then minimize the log-loss function: 
	\begin{equation}
	L(y, f(x)) = -\log P(y|x) = \log (1 + e^{-y f(x)})
	\end{equation}
	where $y$ is true label (1 or -1) and $f(x)$ the decision function at input $x$. In other words, we should choose the parameters of $f(x)$ s.t. it has high probability of predicting/generating observed labels $y$. The optimal pointwise predictor (population minimizer) is given by: 
	\begin{equation}
	\hat{f}(x_i) = \frac{1}{2} \log \frac{\pi_i}{1-\pi_i}
	\end{equation}
	where $\pi_i = P(y_i = 1 | x_i)$. 
	
	\item See Table 16.1 [Murphy] for a list of loss functions and ``population minimizers''. 
\end{itemize}

Remarks of loss function: 
\begin{itemize}
	\item Selection of loss/error function: this should depend on the characteristics of problems. The considerations: weighting of different data points (e.g. positive or negative examples may be weighed differently), the sensitivity to outliers (exponential or $L_2$ error are more sensitive to outliers than $L_1$ error), etc. 
	
	\item Loss function approach: a general way of expressing an inference/decision problem. For inference problem: loss may be interpreted as the departure from the truth; for decision problem: loss as the consequence if wrong decision is made. Ex. for a clustering problem where the goal is to learn class label assignment, $\pi$, it may be reasonable to define the loss function (use $K$ points to approximate all data points) as: 
	\begin{equation}
	L(\pi) = \sum_k \sum_{x_i \in C_k} (x_i - \mu_k)^2	
	\end{equation}
	
	\item More applications of loss function approach: 
	\begin{itemize}
		\item Parameter estimation problem: viewed as given $X_1, \cdots, X_n$, predict $X_i$ using the expectation of $X$ (dependent on parameter). Ex. $X_1, \cdots, X_n$ iid. Bernoulli with probability $p$, define loss function as $(x_i - p)^2$ (prediction for any data point is $p$), then minimize the loss gives the estimator $\hat{p} = \sum_i x_i/n$. 
		\item Dimensionality reduction problem: essentially prediction of high dimensional values using low-dimensional projections, thus could define a loss function and perform minimization. 
	\end{itemize}
\end{itemize}

%%%%%%%%%%%%%%%%%%%%%%%%%%%%%%%%%%%%%%%%%%%%%%%%%%%%%%%%%%%%	
\subsection{Model selection and Bias-Variance Tradeoff} 

Ref: [Hastie, Elements of Statistical Learning, Section 2.9; Bishop, Pattern Recognition and Machine Learning, Section 1.5]
	
Bias-variance decomposition of estimators: suppose $W$ is an estimator of $\tau(\theta)$, the MSE of $W$ can be decomposed as: 
\begin{equation}
E(W - \tau(\theta))^2 = [EW - \tau(\theta)]^2	+ \text{Var}(W)
\end{equation}
The first term is the square of bias, and the second variance of $W$. 

Bias-variance decomposition of predictor function: 
\begin{itemize}
	\item Pointwise prediction: given the function $\hat{f}$, and a point $x$, and the optimal predictor function is $h(x) = \E(Y|X = x)$, the EPE of $\hat{f}$ is given by the Theorem of Point Approximation, applied to the distribution $Y|x$: 
	\begin{equation}
	\text{EPE}(\hat{f}|x) = \E[Y|x - \hat{f}(x)]^2 = [\hat{f}(x) - h(x)]^2 + \int [y - h(x)]^2 p(x,y)dy	
	\end{equation}
	The second term does not depend on the choice of $\hat{f}$. 
	
	\item Averaging over data: since $\hat{f}$ is estimated from data (instead of a deterministic function), EPE should also be averaged over possible data $D$, thus we write $\hat{f}$ as $\hat{f}_D$. We only need to consider the first term, averaging over $D$: 
	\begin{equation}
	\E_D [\hat{f}_D(x) - h(x)]^2 = [\E_D(\hat{f}_D(x)) - h(x)]^2 + \E_D[\hat{f}_D(x) - \E_D(\hat{f}_D(x))]^2
	\end{equation}
	by applying the bias-variance decomposition of estimator ($\hat{f}_D(x)$ is an estimator of $h(x)$). 
	
	\item Now we average over $x$, and obtain: the full decomposition of the EPE [Bishop, Section 3.2]: 
	\begin{equation}
	\text{EPE}(\hat{f}) = (\text{bias})^2 + \text{variance} + \text{noise}
	\label{eq:EPE_bias_variance}	
	\end{equation}
	where:
	\begin{equation}
	(\text{bias})^2 = \int [\E_D(\hat{f}_D(x)) - h(x)]^2 p(x) dx
	\end{equation}
	\begin{equation}
	\text{variance} = \int \E_D[\hat{f}_D(x) - \E_D(\hat{f}_D(x))]^2 p(x) dx	
	\end{equation}
	\begin{equation}
	\text{noise} = \int [h(x) - y]^2 p(x,y)dxdy	
	\end{equation}
	The bias term depends on the truth ($h(x)$), and the variance term is determined by the property of the $\hat{f}_D$, and the noise term is the intrinsic noise in the data (not dependent on the choice of $\hat{f}_D$. 
	
\end{itemize}

Bias-variance tradeoff: 
\begin{itemize}
	\item Simple vs complex models: consider the pointwise prediction $\hat{f}(x_0)$, and we want to analyze its EPE (the case of parameter estimation is similar): 
	\begin{itemize}
		\item Simple models: the expected prediction will be relatively distant from the true values because simpler models just cannot capture the data, thus high bias. 
		\item Complex models: the models are excessively tuned for the noises in the training data (thus different for each new data point), thus high variance. Another way of seeing this is: complex models have more parameters, with each parameter contributing to the total variance. 
	\end{itemize}
	
	\item Bias-variance decompoisition provides a way to analyze model performance. When designing an algorithm, analyze how model performance (bias and variance) depends on the algorithm parameters, e.g. the number of parameters, data processing procedure (such as discretization), etc. In general, choose the complexity parameter to balance bias and variance to minimize EPE. Alternatively, we could say, choose simpler models to reduce variance, without sacrificing too much bias. 
	
	\item Complexity parameters: often through cross validation on training data. Choose the complexity parameter that minimizes the EPE. 
	
\end{itemize}

Example: suppose data are generated from $Y = f(X) + \epsilon$, with $\text{Var}(\epsilon) = \sigma^2$. For the KNN method, the EPE at $x_0$ is: 
\begin{equation}
\text{EPE}_k(x_0) = E[(Y - \hat{f}_k(x_0))^2|X = x]	= \sigma^2 + [f(x_0) - \frac{1}{k} \sum_{l=1}^k f(x_{(l)}) ]^2 + \frac{\sigma^2}{k}
\end{equation}
where $(l)$ indicates the indices of points in the neighborhood of $x_0$. Then: 
\begin{itemize}
	\item When $k$ is small (small neighborhood, more irregular function, thus more complex): only points near $x_0$ are used, thus small bias; but large variance. 
	\item When $k$ is large (large neighborhood, more regular function, thus simpler): distant points of $x_0$ are used, thus large bias; but small variance. 
\end{itemize}

Example: linear regression
\begin{itemize}
	\item Model variance: we are interested in the variance of the prediction at a point $x$: 
	\begin{equation}
	\text{Var}(x^T \hat{\beta}) = \text{Var} (x_1 \hat{\beta}_1 + \cdots + x_p \hat{\beta}_p)
	\end{equation}
	The variance of $\hat{\beta}$ is given by the normal distribution: 
	\begin{equation}
	\hat{\beta} \sim N(\beta, (X^TX)^{-1} \sigma^2)	
	\end{equation}
	Note that in this distribution, $X^T X$ is fixed (conditioned on $X$), but in our analysis of the predictor variance, we should consider the variance of $X$ as well (this is what leads to the fine-tuning of model to specific data). Assume features are normalized and independent, then $X^T X$, the covariance matrix of features, is diagonal, and thus its inverse is also diagonal:
	\begin{equation}
	(X^T X)^{-1} = \text{diag}(1/\sigma_1^2, \cdots, 1/\sigma_p^2)	
	\end{equation}
	Therefore, $\hat{\beta_j}$ are independent, we have: 
	\begin{equation}
	\text{Var}(x^T \hat{\beta}) = \sum_j x_j^2 \text{Var}(\hat{\beta}_j) = \sum_j x_j^2 \sigma^2 / \sigma_j^2	
	\end{equation}
	
	\item Model preference: the total variance of the predictor is thus the sum of variance of each feature, thus reducing the number of features (simpler model) would lead to lower variance (Lasso regression). Furthermore, among all features, those with large variance should be preferred (imagine a low variance feature would lead to instability of parameter estimation).   
\end{itemize}
	
\subsection{Basis Expansion} 

Ref: [Hastie, ESL, 2.6.3, 2.8.3; Bishop, 3.1]
	
Basis functions: e.g. polynormial fitting can be understood as using basis functions $x, x^2, x^3, \ldots$; and trignometric functions are used as basis functions for Fourier analysis. 

Basis expansion: many methods involve basis functions, upon which more complex models are constructed. A natural idea would be to extend with more basis functions. 
\begin{itemize}
	\item Simple basis expansion: e.g. in linear regression, replace, $x_j$ with higher order terms such as $x_j^2$, and $x_i x_j$. 
	\item Adaptive basis expansion: e.g. spline methods for curve fitting, the basis functions are polynomials (and the final function is piecewise polynomial). Or another example, radias basis function: 
	\begin{equation}
	f_{\theta}(x) = \sum_{m} \theta_m K_{\lambda_m}(\mu_m, x)	
	\end{equation}
	where $K_{\lambda_m}(\mu_m, x)$ may be a Guassian kernel. In these cases, the basis functions contain additional parameters that need to be learned from data (thus the basis functions are ``adaptive'').  
\end{itemize}

Common basis functions: 
\begin{itemize}
	\item Application in linear models: 
	\begin{equation}
	y = \beta_0 + \sum_j \beta_j \phi_j(x) + \epsilon	
	\end{equation}
	where $\phi_j(x)$ is a basis function. Since the model is still linear to $\beta_j$, all the usual results of linear model still apply. 
	
	\item Splines: piecewise polynomials, where within each region (defined by ``knots''), the basis function is a polynomial. A spline is order $M$ if it has continuous derivative of order $M - 2$. 
	
	\item Gaussian basis functions: suppose $\mu_j$ is a point in $\mathbb{R}^p$, want to define a function that is large when close to $\mu_j$, and small when distant, and the decay follows a Gaussian function: 
	\begin{equation}
	\phi_j(x) = \exp \left[ - \frac{(x - \mu_j)^2}{2s^2}\right]	
	\end{equation}
	
	\item Other basis functions: sigmoidal function, Fourier series and wavelets (similar to Fourier, but localized in both space and frequency). 
\end{itemize}

Remarks: basis expansion and feature expansion - introduce additional features that may be more predictive of response variable.
\begin{itemize}
	\item Mathematical expansion: e.g. replace linear term with other nonlinear function, either parametric or nonparameteric (e.g. generalize additive model). 
	
	\item Composite features: defined on basic features, e.g. interaction terms (model combinations), objects in image analysis, topics in text analysis. The features may be latent variables: the model may take this into account, but the idea is similar. 
	
	\item Strutured predictors: e.g. $X$ is time-series expression data, then it could be expressed as a combination of ``basic'' profiles (expression in the beginning, in the middle, in the end, etc.), then these ``basic'' profiles can be used as features. The idea is related to semi-supervised learning, where unsupervised data is used. 
\end{itemize}
%%%%%%%%%%%%%%%%%%%%%%%%%%%%%%%%%%%%%%%%%%%%%%%%%%%%%%%%%%%%
%%%%%%%%%%%%%%%%%%%%%%%%%%%%%%%%%%%%%%%%%%%%%%%%%%%%%%%%%%%%
\section{Partition-based Methods and Model Averaging}

Ideas and questions [personal notes]
\begin{itemize}
	\item Questions: how to control the model complexity of boosting methods, the number of weak learners $m$? 
\end{itemize}

Motivations/ideas [Hastie, ESL]: 
\begin{itemize}
\item Partition-based methods: a single global model for the entire data space is not realistic in most cases; on the other hand, the instance-based methods model the neighborhood of each data point, and this does not seem necessary. A compromise is then: partition the data space into regions, and each region is fit or dominated by a single local model. 
\item Model averaging: suppose there are multiple models, each perhaps explaining part of the data, then combining the models will improve the performance. Usually, fit multiple models with different input regions. 
\item \textbf{Remark}: the crucial idea is ``heterogeneity'', that the laws governing the objects of interest are not uniform, and depend on objects(many local models instead of one global model) . The key step addressing heterogeneity is to recognize the regions within which the laws are uniform, but across which the laws may be different. For instance: 
\begin{itemize}
	\item In some problems, there are natural partitions, e.g. in human genetics, the populations form natural partitions. 
	\item Partition according to some features: this leads to the tree-based methods. 
	\item Partitions are grouping of the objects: model parameters are different across groups, but could be considered as samples from a larger population. This is the hierarchical model approach. 
	\item Implicity partitions: data points that conform to simple (local) models. The boosting method. 
\end{itemize}

\item Connection with \textit{adaptive basis function model} (ABM) [Murphy 16.1]: we would like ot fit the model of the form
\begin{equation}
f(x) = w_0 + \sum_{m=1}^M w_m \phi_m(x)
\end{equation}
where $\phi_m(x)$ is a basis function. This basis function can be linear, or a decision tree (certain combination of features). Our idea is to learn $\phi_m(x)$, each explaining part of the data, and combine the results. Geometrically, each $\phi_m(x)$ can be thought of as a partition of data (decision boundaries).  

\end{itemize}

Tree-based methods [Hastie, ESL]:
\begin{itemize}
\item Idea: a simple way of partition the input data space is by the values of the predictors. Ex. for any one predictor, the space can be partitioned into two regions, depending on whether its value is greater than a threshold. Multiple predictors can be combined to partition the space into rectangluar regions. 
\item Tree model: assume within each partition, the value of the response variable is equal. So the model can be written as: 
\begin{equation}
\hat{f}(X) = \sum_{m=1}^M c_m I(X \in R_m)	
\end{equation}
where $m$ is an index of region, and $R_m$ is the $m$-th region. 
\end{itemize}
 
Learning regression and classification trees: 
\begin{itemize}
\item Choosing split variable and point: Intuition: split a variable s.t. in each partition, the labels are common/non-uniform. For any variable to split, its value in partition (purity within each region) can be assessed by SSE in the regression setting (total within-group variance), and measures such as misclassification error or Gini index. Entropy or Gini index are preferred than misclassification error [Murphy, 16.2.2.2].  

\item Pruning: finding the optimal tree is NP-hard, so usually learn a large tree from data, and prune the tree afterwards. In a regression tree, we define the criterion of the tree as: 
\begin{equation}
C(T) = \sum_{m=1}^{\lvert T \rvert} N_m Q_m(T) + \alpha \lvert T \rvert	
\end{equation}
where $m$ is a terminal node, $Q_m(T)$ is the average variance within the region defined by $m$, $N_m$ is the number of data points in $R_m$, and $\alpha$ penalizes large trees. Under pruning, the best subtree of $T$ that minmizes the above function can be found with a greedy algorithm: the weakest link is successively pruned, and regions collapsed. 
\end{itemize}

Practical issues of learning trees: 
\begin{itemize}
\item Categorical predictors and multi-way split: in general, multiway splits fragment the data too quickly, leaving insufficient data at the next level down. For binary outcome, one can order a categorical predictor by the fraction of class 1 instances in each group. 
\item Tree instability: a major problem with trees is their high variance: the effect of an error in the top split is propagated down to all of the splits below it. Bagging method averages many trees may reduce the variance. 
\item Difficulty in capturing additive structure: it could be captured in the model, but not particularly easy, e.g. if $Y$ is a linear combination of $X_1, X_2, X_3$, then need to first split at $X_1$, then at every region in the next level, split by $X_2$; then at next level, split by $X_3$ at every region, etc.
\end{itemize} 

Bagging (bootstrap aggregation) and random forest [Bishop, Chapter 14]: 
\begin{itemize}
\item Idea: reduce the variance by averaging over multiple models - bias will not be changed by averaging, but variance can be reduced.  

\item Committee prediction: suppose we form $M$ bootstrap datasets (sample $N$ instances from the original data, with replacement), and train one model $\hat{f}_m(x)$ for each dataset, the committee prediction is given by: 
\begin{equation}
\hat{f}_{\text{COM}}(x) = \frac{1}{M} \sum_m \hat{f}_m(x)	
\end{equation}

\item Variance reduction: if the errors of each of $\hat{f}_m(x)$ is uncorrelated, then the expected error is $1/M$ of the error of any single model. However, since the models and thus errors are highly correlated, the actual reduction is usually much smaller. 

\item Random forest [Murphy, 16.2.5]: to reduce correlations, use random subsets of data, and random subsets of variables. Also Bayesian approach, Bayesian adaptive regression tree (BART). 
\end{itemize}

Motivation of Boosting [Bishop, Chapter 14]: 
\begin{itemize}
	\item Tree-based methods perform ``hard'' partition of the data points, and enforce one model per region. A more flexible way is to learn multiple models, where each model dominates some data points.
	 
	\item The partitions of data points are formed from data points (thus not stric rules such as cubic regions), with smooth boundaries (which correspond to the decision boundaries of the weak learners).
\end{itemize}

Boosting [Murphy, 16.4]
\begin{itemize}
	\item Overview: our goal is to fit ABF: $f(x) = w_0 + \sum_m w_m \phi_m(x)$ where each basis function $\phi_m(x)$ is a ``weak learner''. Boosting is a greedy algorithm for fitting this model. Boosting is among the best off-the-shelf program for classification. It has the advantage of resistance to overfitting. The most commonly used weak learner is a shallow CART, where $\phi_m(x) = I(x \in R_m)$, where $R_m$ specifies the decision boundary.    
	
	\item Forward stagewise additive modeling: our goal is to solve the problem: 
	\begin{equation}
	\min_f \sum_{i=1}^N L(y_i, f(x_i))
	\end{equation}
	We have different $f$ given different loss functions (see Table 16.1). Ex. for $L_2$ loss, the optimal $f$, assuming the distribution $P(y|X)$ is given is: 
	\begin{equation}
	f^*(x) = \E(y|X)
	\end{equation}
	Finding optimal $f$ is hard, so we tackle it sequentially: similar to numerical optimization, we use gradient methods to find optimum of a function. Our function at step $m$ is: 
	\begin{equation}
	f_m(x) = f_{m-1}(x) + \beta_m \phi(x; \gamma_m)
	\end{equation}
	Suppose we have already known $f_{m-1}(x)$ (the solution from the previous step), we choose $\beta_m, \gamma_m$ to minimize the loss function:  
	\begin{equation}
	(\beta_m, \gamma_m) = \text{argmin}_{\beta, \gamma} \sum_i L(y_i, f_m(x_i))
	\end{equation}
	We continue this for a certain number of iterations or decide by model selection criteria. 
	
	\item $L_2$ boosting: the loss function at step $m$ for a sample $i$ is: 
	\begin{equation}
	L(y_i, f_{m-1}(x_i) + \beta_m \phi_m(x_i; \gamma_m) = (r_{im} - \beta_m \phi_m(x_i; \gamma_m))
	\end{equation}
	where $r_{im} = y_i - f_{m-1}(x_i)$ is the residual at point $i$. So at each step, the model is trying to fit/explain the residual from the previous step. Intuitively, this is similar to step-wise regression, where we choose one variable a time, and each time fitting a regression model using residuals as response variable. 
	
	\item AdaBoost: exponential loss function,
	\begin{equation}
	L_m(\phi) = \sum_i \exp\left[- y_i (f_{m-1}(x_i) + \beta \phi(x_i))\right] = \sum_i w_{i,m} \exp \left(-\beta y_i \phi(x_i) \right)
	\end{equation}
	where the weights are given by: $w_{i,m} = \exp(-y_i f_{m-1}(x_i))$ is the fit of the function at the previous step with observed value. Intuitively, if the prediction is already good at $i$, it will have low weight in next step. LogitBoost: similar, but use log-loss function. 
	
	\item Gradient boosting: a generic version that works on any loss function. Our goal is to minimize the function $f$ over a given loss function $L(f)$. We imagine the optimization is done over parameters, $f(x_1), \cdots, f(x_N)$. Let $g_m$ be the gradient at step $m$: 
	\begin{equation}
	g_{im} = \frac{\partial L(y_i, f(x_i))}{\partial f(x_i)}
	\end{equation}
	The function (vector of $N$ points) is then updated by: $f_m = f_{m-1} = \rho_m g_m$ where $\rho_m$ is step size, which is chosen by minimizing the loss along the direction of $g_m$ (line search). This update is called \textit{functional gradient descent}. This function only optimizes $N$ data points, but we can generalize to optimize a weak learner at each step. Intuitively, we optimizes $f$ by following its gradient along the direction of weak learners: the difference, $f_m - f_{m-1}$, is constrained to have the form of weak learners. 
	
	\item Intuitions of how boosting works: At step $m$, we learn the $m$-th weak learner and its weight $w_m$. This is done by fitting the weak learner to the ``residual'' from previous steps. For binary classification problem, where residuals are not defined, we can also view this as data ``weighting'': for data points fit well by previous steps, their weights are low. Two specific examples of boosting: 
	\begin{itemize}
		\item Linear model with variable selection ($L_2$ boosting): each weak learner is a linear model with a single variable. At each stage, we choose a variable that explains the ``residuals'' from previous stages. The residual of a data point can be thought of its weight: when residual is 0, it has low weight. 
				
		\item Weak learner is a shallow decision tree (AdaBoost): e.g. Figure 16.10, two variables $x_1, x_2$, and weak learners are decision boundaries (or intervals for each variable). Initially, very crude decision boundary. Later, boosting will try to separate the points in the boundaries (of earlier steps), based on which data points are not classified well - leading to more refined boundaries. This typically involves the use of different features in later stages. 
	\end{itemize} 
	Why we don't need to re-update parameters iteratively? In other words, when we fit later data points, we are updating the function parameters, which may lead to poorer fitting of earlier data points. Intuitively, this is not a problem, as the later stages uses different variables to refine decision boundaries.
	
	\item How boosting works from functional gradient perspective: at each step, we are minimizing the loss function by moving along the direction of gradients, but constrained by the fact that the functions need to be sum of weak learners. 
\end{itemize}

AdaBoost algorithm [Bishop, Chapter 14]: 
\begin{itemize}
\item Intuition: repeat $M$ steps, at each step learn a model $\hat{f}_m$. The instances are weighted differently at each step, so that some models are learned to classify some data points, while the other models learned for other data points. A simple example in 2D: the model $f_1$ classify according the value of $x_1$, and $f_2$ according to $x_2$, thus the combination of $f_1$ and $f_2$ can classify according to more complex boundary. 

\item Algorithm: at each step $m$, first train a model by minimizing the weighted error function according to the current weights $w_i^{(m)}, 1 \leq i \leq N$:
\begin{equation}
L_m = \sum_i w_i^{(m)} I(\hat{f}_m(x_i) \neq y_i)
\end{equation}
Then the weights are updated by: 
\begin{equation}
w_i^{(m+1)} = w_i^{(m)}	\exp\left[ \alpha_m I(\hat{f}_m(x_i) \neq y_i)\right]
\end{equation}
where $\alpha_m$ is the measure of how good the classifier $\hat{f}_m$ is (small if the model is poor). The intuition: if $\hat{f}_m$ misclassifies $x_i$, then it should have a higher weight; if $\hat{f}_m$ is already a good classifier (thus larger $\alpha_m$), then should put more weight on those examples that it fails. The final function is given by: 
\begin{equation}
\hat{f}(x) = \text{sgn} \left( \sum_{m=1}^M \alpha_m \hat{f}_m(x) \right)	
\end{equation}

\item Interpretation: the algorithm minimizes the exponential error function, which is equivalent to find the log-odds ratio at each point $x$: 
\begin{equation}
\hat{f}(x) = \frac{1}{2} \ln \frac{P(y=1|x)}{P(y=-1|x)}	
\end{equation}
When all previous $\alpha_m$ and $\hat{f}_m$ are given, it can be shown that minimizing the exponential error function leads to the update formulat of the AdaBoost algorithm. Also note this error function is much more sensitive to the outliers, comparing with cross-entropy, or other common errors. 

\end{itemize}

Mixture of linear regression [Bishop, Chapter 14]: 
\begin{itemize}
\item Idea: there may be hidden/unmeasured variables $Z$ (e.g. in a population of individuals, geneder may be such unmeasured variable), and it is reasonable to assume that the linear models for groups with different values of $Z$ may be different. 

\item Model: let the hidden group assignment of a data point $x_i$ be $Z_i$. The group assignment follows the multinomial distribution $\pi_k$ for group $k$, and the linear model for the group $k$ is given by $N(w_k^T x, \sigma^2)$. 

\item Inference: EM algorithm can be applied, similar to Gaussian mixture model.  
\end{itemize}

Mixture of experts [Bishop, Chapter 14]: 
\begin{itemize}
\item Idea: instead of hidden groups, assume that the group assignment depends on input, i.e. grouping data points by their input values. The interpretation is: the data point determine which region it belongs to, and with each region, some expert determines the function. 

\item Model: can be written as: 
\begin{equation}
p(y|x) = \sum_{k=1}^K \pi_k(x) p_k(y|x)	
\end{equation}
where $\pi_k(x)$ is called the gating function (determine which region, $k$), and $p_k(y|x)$ is called the expert function for the region $k$. In simple models, both $\pi_k(x)$ and $p_k(y|x)$ can be modeled with linear functions. 
\end{itemize}
%%%%%%%%%%%%%%%%%%%%%%%%%%%%%%%%%%%%%%%%%%%%%%%%%%%%%%%%%%%%
\section{Kernel and Prototype Methods}

Reference: [Murphy, Chapter 14], [Bishop, Chapter 6], [Hastie, Chapter 13]

Motivation: why kernel methods? 
\begin{itemize}
\item Non-linear decision boundary or regression function: several standard examples for understanding the need of learning based on instances: 
\begin{itemize}
\item XOR function: $f(x_1, x_2) = x_1 \text{XOR} x_2$. The decision boundary is clearly not linear. 
\item Circles: $y = 1$ if $\norm{x} \leq 1$. The decision boundary is circle. 
\item Two moons: the positive and negative examples fall into non-convex sets. 
\item sinc function: for regression analysis, $f(x) = \sin x / x$. 
\end{itemize}

\item Ideas for dealing with non-linearity: (1) feature expansion: e.g. for XOR function, we define features based on logic functions of $x_1$ and $x_2$. (2) Learning from instances: e.g. KNN, then for XOR function, there are four clusters in the data, and by predicting the label of a new instance based on its distance to the other instances, the decision can be non-linear. The kernel methods provide a framework to draw inference from examples. 

\item Example: using product features for circles. The decision boundary of a circle in $\mathbb{R}^2$: $x_1^2 + x_2^2 \leq R^2$, if define features as: 
\begin{equation}
(x_1, x_2) \mapsto (z_1 = x_1^2, z_2 = x_2^2, z_3 = x_1 x_2)	
\end{equation}
Then the decision boundary in $\mathbb{R}^3$ becomes linear: $z_1 + z_2 \leq R^2$. 

\item Use similarity for learning: kernel methods. Define kernel functions (similarlity) between objects, and the predictions of an object $x$ is based on its kernel function $\kappa(x,x')$, where $x'$ is existing data, and $y'$. Any methods using these functions can be thought as kernel methods. 
\begin{itemize}
	\item A particularly important example is: prediction of structured objects, e.g. strings and tree. No obvious feature representation is available, and it is more natural to define a kernel on these objects (e.g. the edit distance between two objects). 
\end{itemize}

\item Remark: the connection between the two perspectives on dealing with non-linearity. (1) Kernel methods (Mercel kernel): effectively a new feature representation; (2) Prototypes used in kernel methods: represent a particular combination of features, thus could also be viewed as a special kind of feature expansion. 
\end{itemize}

Constructing kernel functions: 
\begin{itemize}
\item A canonical example for object similarity: suppose we have $x$ and $x'$ as two objects in $\mathbb{R}^D$, their similarlity can be measured by correlation or the angle between the two vectors. First, correlation: 
\begin{equation}
r_{x,x'} = \frac{\sum_i (x_i - \bar{x}) (x'_i - \bar{x'})}{\sqrt{\sum_i (x_i - \bar{x})^2} \sqrt{\sum_i (x'_i - \bar{x'})^2}} = \frac{x \cdot x'}{\norm{x} \norm{x'}}
\end{equation}
So the correlation is proportional to the inner prodcut of $x$ and $x'$. Next, we see that it is exactly $\cos (\theta)$ where $\theta$ is the angle between $x$ and $x'$. In short, the object similarlity can be measured by the inner product, up to a constant. 

\item Kernel and inner product: To generalize the idea above, we want to define kernels that are effectively inner products. These are called Mercel kernels: a kernel function is Mercel kernel if for any given $N$ inputs, $x_i$, the Gram matrix, defined by: $K = (\kappa(x_i, x_j))$ is positive definite. We take the Choleksy decomposition of $K$, or
\begin{equation}
K_{ij} = \phi(x_i)^T \phi(x_j)	
\end{equation}
where $\phi(x_i)$ is some linear function of $x_i$ (based on the decomposition). And we see that $K_{ij}$ is an inner product. More generally, for any Mercel kernel, there exists a feature mapping from $x$ to $\mathbb{R}^D$ s.t. 
\begin{equation}
\kappa(x, x') = \langle \phi(x), \phi(x') \rangle	
\end{equation}
The RBF kernels (below) and cosine similarity kernels are all Mercel kernels. Thus Mercel kernels can be understood as inner product in some new feature space. 

\item Expanding kernels: Equations (6.13) to (6.22) in [Bishop, Chapter 6]. Most importantly, if $k_1(\mathbf{x}, \mathbf{x'})$ is a kernel, then, a polynormal function of $k_1()$ is also a kernel. And if $A$ is a symmetric psd. matrix, then 
\begin{equation}
k(\mathbf{x}, \mathbf{x'})	= \mathbf{x}^T A \mathbf{x'}
\end{equation}
is also a kernel. 
\end{itemize}

Examples of kernel functions: 
\begin{itemize}
\item Gaussian (RBF) kernel: the Guassian kernel is defined by: 
\begin{equation}
k(\mathbf{x}, \mathbf{x'}) = \exp\left(-\frac{1}{2} (x - x')^T \Sigma^{-1} (x-x')\right)	
\end{equation}
In the special case where $\Sigma$ is diagonal, this can be written as: 
\begin{equation}
k(\mathbf{x}, \mathbf{x'}) = \exp\left(-\frac{1}{2} \sum_{j=1}^D \frac{1}{\sigma_j^2} (x_j - x'_j)^2\right)	
\end{equation}
The term $\sigma_j$ is the characteristic length scale of the $j$-th dimension. And when $\sigma_j$ are all equal, we have: 
\begin{equation}
k(\mathbf{x}, \mathbf{x'}) = \exp\left(- \frac{\norm{\mathbf{x} - \mathbf{x'}}^2}{2\sigma^2}\right)	
\end{equation}
where $\sigma$ is called the bandwidth. 

\item Linear kernel: the simplest kernel is $\kappa(x,x') = x^T x'$, i.e. the feature mapping is $\phi(x) = x$. This is useful when the problem is linearly separable, and there is no need of working on other feature space. 

\item String kernel: suppose we have a dictionary of strings, and we want to define the kernel of two strings. Let $s$ be any substring, and we could define the string kernel as the number of times $s$ appears in both the input strings $x$ and $x'$, summing over al possible $s$. Or in other words, the kernel is roughly the number of shared substrings. 

\end{itemize}

Probabilistic kernels: 
\begin{itemize}
\item Motivation: if we know the process that generates $x$, then we could use this to define an appropriate kernel. For instance, if $x$ and $x'$ are from a normal distribution, then the appropriate scale for their distance/kernel is the standard deviation. The general idea is that the kernel function between two objects reflects how likely the two are generated from the same distribution (process). 

\item Probability product kernel: if $x_1$ and $x_2$ are close, then they should add similar information to the underlying distribution. From the Bayesian perspective, the posterior distributions $p(x|x_1)$ and $p(x|x_2)$ should be similar. In practice, we often assume $p(x|x_1)$ is close to $p(x|\hat{\theta}(x_1))$ where $\hat{\theta}(x_1)$ is the MLE of the parameter given $x_1$, and the same for $p(x|x_2)$. The similarlity of the two distributions can be defined via the kernel: 
\begin{equation}
\kappa(x_1, x_2) = \int p(x|x_1)^{\rho} p(x|x_2)^{\rho}	dx
\end{equation}
where $\rho > 0$. It can be seen, for example, that the kernel is maximum when the two distributions are equal (Cauchy-Scharwz Inequality, the norm is equal to 1). Examples: 
\begin{itemize}
\item RBF kernel: when the data is form normal distribution, $p(x|\theta) = N(\mu, \sigma^2 I)$, and $\rho = 1$, we have the RBF kernel. 

\item The data can be viewed as generated from a mixture model, then the kernel can be defined according to the probabilisty that two objects are sampled from the same component: 
\begin{equation}
k(\mathbf{x}, \mathbf{x'}) = \sum_i p(\mathbf{x}|i) p(\mathbf{x'}|i) p(i)
\end{equation}

\item For two sequences, the kernel can be defined assuming they are generated from the same HMM, following the same path: 
\begin{equation}
k(\mathbf{x}, \mathbf{x'}) = \sum_{\mathbf{z}} p(\mathbf{x}|\mathbf{z}) p(\mathbf{x'}|\mathbf{z}) p(\mathbf{z})
\end{equation}
where $\mathbf{z}$ is the hidden path. 
\end{itemize}

\item Fisher kernel: if $x$ and $x'$ are close, then to say that they add the same information, or suggesting the same values of $\theta$, is equivalent to saying the log-likelihood function $\log p(x|\theta)$ and $\log p(x'|\theta)$ should be similar, near $\hat{\theta}$. Define the gradient of the log-likelihood or the score vector, 
\begin{equation}
g(x) = \nabla_{\theta} \log p(x|\theta)	
\end{equation}
Then $g(x)$ should be close to $g(x')$ evaluated at $\hat{\theta}$. This is defined by their generalized inner product: 
\begin{equation}
\kappa(x,x') = g(x)^T F^{-1} g(x')	
\end{equation}
where $F$ is the Fisher information matrix evaluated at $\hat{\theta}$. 

\end{itemize}
 
Motivation of prototype methods: 
\begin{itemize}
	\item Idea: use prototypes to represent the data points in both positive and negative classes, then classification of one instance can be performed by looking at the nearest prototype(s) of this instance. The advantage: Highly unstructured, not dependent on statistical assumptions, can be very effective and often among the best in real data problems. Particularly when the decision boundary is highly irregular. 
	\item Implementations: (1) KNN and local methods: use the nearest neighbors/prototypes to make predictions, weighted by distance; (2) kernel representation: the distance of an instance to all protypes is a new representation of the instance. 
	\item Remark: this is similar to the method of fitting a curve with a piecewise smooth function. The key problem is to select a set of ``pivotal'' points to cover the data space. These prototypes can simply be all training points (e.g. KNN), or some points to be learned from data (e.g. RBF method). 
\end{itemize}

$K$-means based methods: 
\begin{itemize}
	\item $K$-means: cluster data points with $K$-means algorithm, on both positive and negative classes. And classify any instance to the nearest prototype. 
	\item Gaussian mixture: a ``soft'' version of clustering. 
\end{itemize}

Learning vector quantization (LVQ): 
\begin{itemize}
	\item Idea: $K$-means clusters are found for each class separately. But we want to prototypes that may discriminate different classes. 
	\item Algorithm: suppose we have processes all training instances up to $x_i$, now with the new training instance $x_i$, we first find the nearest prototype. If the class label of the prototype is the same as $y_i$, then we move the prototype a bit closer to $x_i$:
\begin{equation}
m_j(k) \leftarrow m_j(k) + \epsilon (x_i - m_j(k))	
\end{equation}
where $k$ is the class label, and $j$ the index of the prototype. Otherwise, it will be moved a bit away from $x_i$. 
\end{itemize}

KNN algorithm: 
\begin{itemize}
	\item Algorithm: no need to fit the model, for any instance to predict, $x_0$, find its $K$ nearest neighbors in the training data, and classify $x_0$ according to majority vote. 
	\item The algorithm can be very effective in a large number of applications. It is particularly good when the decision boundary is very irregular.  Most methods (even with $K$-means selection of prototypes) would depend on statistical assumption e.g. natural grouping of data points, which may not be true; in contrast, KNN does not make such assumptions, and the number of prototypes is not fixed a priori. 
\end{itemize}

Asymptotic performance of KNN: 
\begin{itemize}
	\item Model: we consider a data point to be classified, $x$, and analzye the average loss. The class lable of $x$ is inherently probabilistic as the probability distribution of different classes overlap at $x$. This uncertainty can be expressed as $p_k(x)$, which is the conditional probability for class $x$, assume the true models are known. 
	\item Optimal method (Bayes error: posterior probability of class label): let $k^*$ be the dominant class, then the Bayes error is $1 - p_{k^*}(x)$. 
	\item 1-NN: in the asymptotic case, make the correct prediction only if the correct class label of the nearest training point is correct (for the same reason that the class label of the training point is also inherently probabilistic), thus the error is (weighted by the probability of the true class label of $x$): 
\begin{equation}
\text{1NN error} = \sum_{k=1}^K p_k(x) (1 - p_k(x))	\geq 1 - p_{k^*}(x)
\end{equation}
For $K =2$: the error rate is $2 p_{k^*}(x) (1 - p_{k^*}(x)) \leq 2 (1 - p_{k^*}(x))$. 
	 
\end{itemize}

%RBF kernel method: 
%\begin{itemize}
%\item Prototype methods: suppose the data space can be represented by ``prototypes'', e.g. data consists of clusters, then centroids of clusters are prototypes. The function value at $x$ depends on which protoype $x$ is close, and in general, it can be written as a linear combination: 
%\begin{equation}
%f(x) = \sum_j \beta_j K_{\lambda_j}(x,\xi_j)
%\end{equation}
%where $\xi_j$ is the centroid, and $\beta_j$ is the function value at $\xi_j$. 
%
%\item Radial basis function (RBF): the common choice in the above method is the RBF kernel, defined as the function whose value depends only on the distance. 
%\end{itemize}

Kernel machines: 
\begin{itemize}
\item Kernalized feature vector: suppose we represent a data point $x$ by its kernel function wrt. $K$ prototypes, called ``kernelized feature vector'': 
\begin{equation}
\phi(x) = [\kappa(x,\mu_1), \cdots, \kappa(x,\mu_K)]	
\end{equation}
where $\mu_k$ is the $k$-th prototype (could be from a clustering algorithm or could be the data points themselves, see below). Then the usual GLM can be applied to the kernelized features - kernel machines. If the RBF kernel is used, this is called an RBF network. 

\item Why the kernel machines could solve the non-linearity problem? Example: in XOR function, the four prototypes represent four different logic functions (combinations) of the features $x_1$ and $x_2$. With these new features, the function is linear. 
\begin{itemize}
\item Example: application of RBF network in 1D fitting of $y = f(x)$ that is highly non-linear (Figure 14.3 of Murphy). The prototypes are a subset of points $(x_i, y_i)$: clearly the relationship between these points are not necessarily linear. 
\item Bandwidth: low bandwidth leads to very wiggy functions: e.g. for an $x$, if it is not in the close neighborhood of any prototypes, then its value $y$ is predicted to be 0 (low-bias, high variance). High bandwidth leads to lower variance, but higher bias. 
\end{itemize}
\end{itemize}

An example of kernel machine: we illsutrate how to make inference and predictions using kernel machines. 
\begin{itemize}
\item Consider a simple linear model in the kernelized feature space: 
\begin{equation}
y = \sum_{j=1}^K \beta_j \kappa(x, \mu_j) + \epsilon
\end{equation}
where $K$ is the number of prototypes, $\mu_j$ is the $j$-th prototype, and $\epsilon$ the error term (no need of change). 

\item Interpretation of $\beta_j$: imagine $\mu_j$ are well-separated, then $\kappa(\mu_j, \mu_k) \approx 0$ if $j \neq k$. Let $y_j$ be the response variable at $\mu_j$, then: $y_j \approx \beta_j \kappa(\mu_j, \mu_j)$. From this, we see that, $\beta_j \approx y_j / \kappa(\mu_j, \mu_j)$. For RBF kernel, we have $\kappa(\mu_j, \mu_j) = 1$, thus $\beta_j \approx y_j$. So under this special case, we have: 
\begin{equation}
y \approx \sum_{j=1}^K y_j \kappa(x, \mu_j)	
\end{equation}
Thus the value of $y$ at a point $x$ is simply the weighted average of $y_j$, with weights determined by the distance of $x$ to $\mu_j$. 

\item Parameter estimation and prediction: we define the kernelized feature vector for the data point $x_i$ as, $z_i = [\kappa(x_i, \mu_1), \cdots, \kappa(x_i, \mu_K)]$, then apply the usual linear model on $z_i$, we have: 
\begin{equation}
\hat{\beta} = (Z^T Z)^{-1} Z^T y	
\end{equation}
where $Z$ is the $n \times K$ matrix of $z_i$'s. To predict the value of $y$ at a point $x^*$, we form $z^*$ first, and then our prediction $\hat{f}(x^*) = z^* \hat{\beta} = z^* (Z^T Z)^{-1} Z^T y$. 

\item Special case of $K = n$: each $x_i$ is a protoptype, then the design matrix is simply $n \times n$ matrix $K = [\kappa(x_i, x_j)]$. When we use the inner product (instead of kernel), it is also $K = X X^T$, the Gram matrix. For symmetric kernels, we have: 
\begin{equation}
\hat{\beta} = (K^T K)^{-1} K^T y = K^{-1} y		
\end{equation}
And prediction $\hat{f}(x^*) = [\kappa(x^*, x_1), \cdots, \kappa(x^*, x_n)]K^{-1} y$. To see what this means, imagine $K$ is diagonal (the $n$ data points are well-separated), then it's easy to show that: 
\begin{equation}
\hat{f}(x^*) = \sum_i \frac{\kappa(x^*, x_i)}{\kappa(x_i,x_i)} y_i	
\end{equation}
This is effectively the kernel-weighted average that we'll discuss later, and corresponds to our intuition described above. 
\end{itemize}

Sparse vector machines: 
\begin{itemize}
\item In the high-dim. case, there is no good way of choosing a small number of prototypes. However, choosing a large number of prototypes makes the inference difficult: $D$ is close to $n$ (the number of features is close to or equal to the number of data points). A special case is every $x_i$ is a prototype: 
\begin{equation}
\phi(x) = [\kappa(x,\mu_1), \cdots, \kappa(x,\mu_N)]		
\end{equation}
Then $D = N$. The only way to solve this problem is to use the sparsity-promoting priors for the coefficients in the model. This is called a ``sparse vector machine''. 

\item Types of sparse vector machines: $L_1$ regularization vector machine (L1VM), L2VM, relevance vector machine (RVM) and SVM. Except L2VM, the others are sparse, so there are a small set of positive and negative prototypes among all training examples: prediction of an instance is based on how close it is to these prototypes. These are support vectors in SVM. 

\item \textbf{Question}: Sparsity is formulated in terms of the kernelized feature vectors. How should one formuate the model in terms of the original features?  
\end{itemize}

Kernel trick: 
\begin{itemize}
\item Claim: we do not have to explicitly model the kernel machines using the kernelized feature vectors. Instead, to make prediction at a new example, we only need to formulate the algorithm in terms of inner product between data points, and between data point and new example. Then we simply replace the inner products with kernels. This is the ``kernel trick''. 

\item Kernel trick: if an algorithm can be formulated in terms of inner product $\langle \mathbf{x}, \mathbf{x'} \rangle$, then we could define a feature mapping $\mathbf{x} \mapsto \phi(\mathbf{x})$, and the inner product in the new feature space is a kernel: 
\begin{equation}
k(\mathbf{x}, \mathbf{x'}) = \langle \phi(\mathbf{x}), \phi(\mathbf{x'}) \rangle	
\end{equation}
In this notation, $k(\mathbf{x}, \mathbf{x'})$ is a general measure of similarity bewteen two instances. Then all the computation can be carried out with this kernel function, instead of the explicit feature mapping $\phi()$. The kernel function should be Mercel kernel so that it can be viewed as inner product. 

\item Computational advantage: the feature mapping into a very high (or even infinite) dim. space is now computationally possible, as long as the kernel function is well-defined and computable. Also, the feature mapping may not need to be explicitly defined, and one only needs to specify the ideas of similarity using kernel function, then the same algorithm can be still applied. 

\item To prove this claim, we consider the linear model example. We try to write the predicted value of $y$ at example $x$ using innner product. Following Equation~\ref{eq:kernelized_ridge_regression}, this can be written as: 
\begin{equation}
\hat{f}(x^*) = x^* w = x^* X^T (K^{-1} y)	
\end{equation}
Clearly, if we replace inner product in this equation with kernels, we obtain the results discussed before with explicit kernelized feature representation. 
\end{itemize}

Kernelized distance-based methods: 
\begin{itemize}
\item Kernelalized KNN classification: to kernelize KNN algorithm, we write the distance as inner products: 
\begin{equation}
\norm{x - x'}^2 = \langle x, x \rangle + \langle x', x' \rangle - 2 \langle x, x' \rangle
\end{equation}

\item Kernelalized $K$-medoids clustering: in the algorithm, one key step is to choose a data member in a cluster, so that the total distance of this member to all other members is minimum. To kernelize it, we replace the distance with kernels. 
\end{itemize}

Kernelized ridge regression: 
\begin{itemize}
\item Primal problem: minimze the error function: 
\begin{equation}
J(\mathbf{w}) = (y - X w)^T (y - X w) + \lambda \norm{w}^2
\end{equation}
And the new prediction: $y(\mathbf{x}) = \mathbf{w}^T \mathbf{x}$. The solution is given by: 
\begin{equation}
w = (X^T X + \lambda I_D)^{-1} X^T y	
\end{equation}
Note that $w$ is written in terms of the covariance between explanatory variables ($X^T X$), and the covariance between explanatory and response variables. 

\item Dual problem: we can rewrite $\mathbf{w}$ in terms of the inner product of data vectors. Using the matrix inversion Lemma: 
\begin{equation}
w = X^T (X X^T + \lambda I_N)^{-1} y	
\label{eq:kernelized_ridge_regression}
\end{equation}
where $K = X X^T$ is the Gram matrix (inner product between any two instances). Now we still have $X^T$, however, we note that once we compute the prediction $w^T x$, we will move this term into an inner product as well. Specifically let $\alpha = (K + \lambda I_N)^{-1} y$ as dual variables, then we have: 
\begin{equation}
\hat{f}(x) = w^T x = (X^T \alpha) x = \sum_{i=1}^N \alpha_i \kappa(x, x_i)	
\end{equation}

\item Analysis: behavior of kernelized ridge regression. We consider the case where the bandwidth of the kernel is very small, thus $K$ is approximately diagonal. And we ignore the regularization term, so $\alpha \approx K^{-1} y$ with: 
\begin{equation}
\alpha_i \approx \frac{y_i}{\kappa(x_i, x_i)}	
\end{equation}
Then the predicted value of $y$ is given by: 
\begin{equation}
\hat{f}(x) = \sum_{i=1}^N \alpha_i \kappa(x, x_i) \approx \sum_{i=1}^N y_i \frac{\kappa(x, x_i)}{\kappa(x_i, x_i)}
\end{equation}
Thus $\hat{f}(x)$ is the average of $y_i$, weighted by the kernel function $\kappa(x, x_i)$ (similar to local regression). 

\item Remark: the idea of dual representation is very general: an algorithm/inference procedure can be stated in terms of how variables are related, but can also be stated in terms of the relation between data vectors. The advantage of the dual representation is that when $D$ is large, and $n$ is relatively small, the computational cost in the dual representation is much lower. 
\end{itemize}

\subsection{Smoothing Kernels and Local Methods} 

Reference: [Murphy, Chapter 14], [Hastie, Chapter 6]

Kernel smoothing methods: 
\begin{itemize}
\item Motivation: two related problems: 
\begin{itemize}
	\item Estimating probability density function: given data points $x_i$, find $p(x)$ for each $x$. 
	\item Estimating a function: given $(x_i, y_i)$, find a function $y = f(x)$.   
\end{itemize}
Our goal is, for both cases, find smooth functions/density. 

\item Idea of local methods: probability distribution and functions (to be learned) are generally smooth, i.e. the probability densities at close points are close, and if $x \approx x'$, it's likely that $f(x) \approx f(x')$, etc. This smoothness property can be exploited to learn functions and probability distributions locally. 

\item Kernel smoothing: to implement the local methods for a point of interest $x_0$, express the desired quantity as the sum of contributions of many examples, and weigh the examples according to how close they are to $x_0$, defined by a kernel function, $K_{\lambda}(x,x_0)$. 
\end{itemize}

Smoothing kernels: 
\begin{itemize}
\item Smoothing kernel: weighting of points depends on their distance to some reference point, usually 0. It is a function of one argument which satisfies the properties: 
\begin{equation}
\int \kappa(x) dx = 1 \qquad 	\int x \kappa(x) dx = 0 \qquad \int x^2 \kappa(x) dx > 0
\end{equation}
This will gurantee that the weights sum to 1, and by itself will not introduce bias. For vector input, and non-zero reference point ($x_0$), we could define a smoothing kernel as: 
\begin{equation}
K_h(x, x_0) = \kappa_h(\norm{x - x_0})	
\end{equation}

\item Gaussian kernel: the simplest case is the standard normal pdf: 
\begin{equation}
\kappa(x) = \frac{1}{\sqrt{2 \pi}} e^{-x^2/2}	
\end{equation}
In general, we introduce a bandwidth parameter $h$: $\kappa_h(x) = \frac{1}{h} \kappa(\frac{x}{h})$. In the multivariate case, we have: 
\begin{equation}
\kappa_h(x) = \frac{1}{h^D (2 \pi)^{D/2}} \prod_{j=1}^D \exp\left( -\frac{1}{2 h^2} x_j^2\right)	
\end{equation}

\item Other common smoothing kernels: Epanechnikov kernel (bounded on [-1,1]), tri-cube kernel, boxcar kernel (uniform distribution on [-1,1]), etc.
\end{itemize}

Kernel density estimation (KDE): the goal is to estimate the PDF of a RV at $x_0$, $p(x_0)$ from an iid sample $x_1, \cdots, x_N$. 
\begin{itemize}
\item Method: consider a local neighborhood of $x_0$ with width $\lambda$, $N(x_0)$, the simple local estimate would be:
\begin{equation}
\hat{p}(x_0) = \frac{1}{N \lambda} \sum_i I(x_i \in N(x_0))
\end{equation}
where $I(\cdot)$ is the indicator function. This estimate is ``bumpy'', to smooth the estimate, replace the step function with a kernel function, we have: 
\begin{equation}
\hat{p}(x_0) = \frac{1}{N \lambda} \sum_i K_{\lambda}(x_i, x_0)
\end{equation}
The kernel function is chosen s.t. it is large when close to $x_0$, and small when distant. 

\item Interpretation: the PDF can be approximated as a Gaussian mixture model, with each centroid at the data point $x_i$. Assume the mixture component has equal $\sigma$, and mixture weight $1/N$. The total probability density at a point $x_0$ is the sum of density from every component (weighted by $1/N$). 

\item Density estimation and classification: any density estimation methods can be used for classification - estimate density separately for each class, and then do classification using Bayesian theorem. The same idea can be applied for regression, where density estimation would include the joint distribution of $(x,y)$, see below and [Bishop, section 6.3]. 

\end{itemize}

Density estimation under other perspectives: [personal notes]
\begin{itemize}
\item Bayesian nonparametric density estimation: suppose we discretize the PDF into $K$ intervals, then our problem is to estimate a multinomial distribution $(p_1, \cdots, p_K)$ from the data $(x_1, \cdots, x_K)$ where $x_k$ is the number of points falling in the $k$-th interval. The MLE of $p_k$ is simply $x_k / N$. The Bayesian inference puts a prior on $p_k$ s.t. it is close to $p_{k-1}$. For instance, we could have $p_k$ from a stochastic process, where $\E(p_k) = p_{k-1}$. 

\item Sparse model for density estimation: following the same discretization scheme, we want to estimate $p_k$ through penalized log-likelihood. For example: 
\begin{equation}
l(p|x) = \sum_{k=1}^K x_k \log p_k - \lambda \sum_k (p_k - p_{k-1})^2	
\end{equation}
This is similar to fused Lasso in regression setting. 

\item Direct regularization of density function: in general, we could work directly on the density function. For instance, using the penalized log-likelihood, we have: 
\begin{equation}
l(f) = \sum_{i=1}^N \log f(x_i) - \lambda	\int \norm{f'(x)}^2 dx
\end{equation}
\end{itemize}

Local likelihood method: 
\begin{itemize}
\item Idea: instead of creating a global model, for any point to study, $x_0$, create a local model around $x_0$. Then we assume all data points are generated according to this likelihood model, and estimate the parameters accordingly with weighting of data points. This could be used for a prediction problem, or more generally learning a likelihood model. 

\item Local likelihood method: let $\theta(x_0)$ be the parameter of the local model, maximize the local likelihood function: 
\begin{equation}
l(\theta(x_0)) = \sum_{i=1}^N K_{\lambda}(x_0, x_i) l(x_i;\theta(x_0))	
\end{equation}
where $l(x_i,\theta)$ is the likelihood of a data point. This is simply the usual log-likelihood function with distance weighting. Specifically, we assume a function near $x_0$ parameterized by $\theta$, and the data $x_i$ are generated from this local function. This function is smooth: local, linear, polynomial, etc. 

\item Density estimation using local likelihood: we create intervals, and let $y_k$ be the number of data points in interval $k$, and we estimate the density near $x_0$. Assume density is constant near $x_0$, $p_0 = p(x_0)$. Our model is $y_k \sim \text{Bin}(N, p_0)$ where $N$ is the sample size. The local log-likelihood is: 
\begin{equation}\label{key}
l(p_0) = \sum_k K_{\lambda}(y_k, x_0) [y_k \log p_0 + (N-y_k) \log(1-p_0)]
\end{equation}
Maximizing this function, and use the fact that $\sum_k K_{\lambda}(y_k, x_0) = \sum_i K_{\lambda}(x_i, x_0)$, we can obtain the KDE. 
\end{itemize}

Local prediction: the optimal predictor of $x_0$ is given by $E(Y|X=x_0)$ from statistical decision theory. 
\begin{itemize}
\item Naive method: the $k$-nearest neighbor (KNN) average of the point $x_0$: 
\begin{equation}
\hat{f}(x_0) = \text{Ave}(y_i | x_i \in N_k(x))	
\end{equation}
The function is distinuous, and not optimal (as all points within a region have equal weight). 

\item Nadaraya-Watson kernel weight average: 
\begin{equation}
\hat{f}(x_0) = \frac{\sum_i K_{\lambda}(x_0, x_i) y_i}{\sum_i K_{\lambda}(x_0, x_i)}
\label{eq:kernel_weight_average}
\end{equation}
The kernel is usually chosen s.t. only points within a window (metric window, as it is defined by $x$, not by the rank) can contribute. 

\item Remark: the local prediction method can be viewed as estimating $E(Y|X=x_0)$ using the estimation of local density at $x_0$: 
\begin{equation}
f(x_0) = E[Y|X=x_0] = \frac{\sum_y y p(x_0,y)}{p(x = x_0)}	
\end{equation}
Plug-in the density estimate (as a mixture of $N$ Gaussian distributions) and we obtain the result.  

\item Remark: this could also be viewed as an application of local likelihood method. Assuming the local model at $x_0$ is $y_i \sim N(\theta(x_0), \sigma^2)$, where $\theta(x_0)$ is the expected value of $y$ at $x_0$ (to be estimated). The likelihood of this model at a point $(x_i, y_i)$ is: 
\begin{equation}
l(\theta(x_0)|x_i,y_i) = \log P(y_i|x_i, \theta(x_0))	= -\frac{1}{2 \sigma^2} (y_i - \theta(x_0))^2
\end{equation}
The local likelihood model is: 
\begin{equation}
l(\theta(x_0)) = -\frac{1}{2 \sigma^2} \sum_i K_{\lambda}(x_0, x_i) (y_i - \theta(x_0))^2
\end{equation}
Maximize this function by taking derivative of $\theta(x_0)$, and we obtain Equation~\ref{eq:kernel_weight_average}. 
\end{itemize}

Local regression: 
\begin{itemize}
\item Idea: the kernel weight average method still assumes a constant near $x_0$ to be estimated. More generally, fit a local model at $x_0$. The model should minimize the error, where the errors in regions distant to $x_0$ will be discounted. 

\item Local linear regression: at $x_0$, fit the local linear model, $y = \alpha(x_0) + \beta(x_0) x$: 
\begin{equation}
\min_{\alpha(x_0),\beta(x_0)} \sum_{i=1}^N K_{\lambda}(x_0,x_i) \left[ y_i - \alpha(x_0) - \beta(x_0) x_i\right]^2	
\end{equation}
The solution is similar to the result of linear regression, except that $X^T$ will be replaced by $X^T W(x_0)$, where $W(x_0)$ is the $N \times N$ diagonal matrix with $i$-th diagonal element $K_{\lambda}(x_0,x_i)$. The result is a linear combination of $y_i$: 
\begin{equation}
\hat{f}(x_0) = \sum_i l_i(x_0) y_i	
\end{equation}

\item Varying coefficient model: more generally, this is the varying coefficient model, or structured regression [Hastie, 6.4.2]. Suppose among all features, we can divide them into two groups, $x$ and $z$. The response variable $y$ depends on $x$ in a linear model, but on $z$ in a very non-linear fashion. Then we could formulate this as a model, where the coefficients of $y$ on $x$ is a local model of $z$. Specifically, we solve the regression problem for each $z_0$:  
\begin{equation}
\min_{\alpha(z_0),\beta(z_0)} \sum_{i=1}^N K_{\lambda}(z_0,z_i) \left[ y_i - \alpha(z_0) - \beta(z_0) x_i\right]^2	
\end{equation}
An example of this is the application to time-series data, where $z_i$ is the time point. We learn a linear model for each time point, but with the constraint that the models of the nearest time points should have similar coefficients. 

\item Bayesian perspective: we are trying to estimate a local model, however, the estimator using only local data points is a poor estimator (very few data points, thus high variance), so it's better to use other neighbors as priors. Similar to Bayesian density estimation, we could formulate a hierarchical model, where $\beta(z)$ follows random effects, with its prior favoring similar values of $\beta$ for close $z$. 
\end{itemize}

Selecting the width of kernel $\lambda$ in kernel smoothing methods: 
\begin{itemize}
\item Kernel width: defined by a parameter $\lambda$. For the Gaussian, it is the standard deviation; for KNN, it is the radius spanned by the nearest $k$ members; for tri-cube kernel, the radius of the support region. 

\item Bias-variance tradeoff: small $\lambda$, a small number of points are used to estimate $\hat{f}(x_0)$, thus large variance, but small bias as each of $f(x_i)$ should be close to $f(x_0)$. Large $\lambda$ is opposite: small variance, but large bias.  

\item Dependency on density: the variance (determined by the number of points near $x_0$) is inverstly proportional to the local density. Thus, at regions of low density, it may be better to use larger $\lambda$. More generally, an adaptive neighborhood can be used: 
\begin{equation}
K_{\lambda}(x_0,x) = D\left( \frac{x-x_0}{h_{\lambda}(x_0)}\right)	
\end{equation}

\end{itemize}
%%%%%%%%%%%%%%%%%%%%%%%%%%%%%%%%%%%%%%%%%%%%%%%%%%%%%%%%%%%%
\section{Unsupervised Learning}
\begin{enumerate}

\item{Overview of unsupervised learning} 

Challenges of unsupervised learning [Murphy, Chapter 1]: 
\begin{itemize}
\item Goal: discover the ``structure'' from the data, in a general sense. This could be: find a simple representation of data, categorize the data, etc. so that when new instances are presented, one could find similar objects in ``memory'', categorize the new instances, etc. 

\item Statistical perspective: the goal is to learn a density function $p(x|\theta)$, as opposed to learning a conditional density $p(y|x,\theta)$ in the classification problem. 

\item Cluster analysis: discover the clusters (objects belong to the same category) in the data. Ex. from a set of images, group them into categories such as animals, fruits, furnitures, etc. 

\item Discover latent factors: the data may be best explained through the action of certain latent factors. For example,the variability of a set of images of the same object but under different background, light conditions, etc. can be explained by a small number of latent factors such as lighting, pose, etc. 

\item Graph (dependency) structure: the multiple variables in the data are related to each other in a certain way. 

\item Imputation (filling the missing data): an application of the unsupervised learning is that given part of the data (object), learn the missing part. Ex. image inpainting, collaborative filtering (we know some movies a user like, and to predict the rest). 
\end{itemize}

\item{Cluster analysis} [Hastie, Chapter 14]

Distance/similarity metrics:  
\begin{itemize}
\item Distance vs. similarity: often distance or dissimilarity is used, instead of similarity, but this would depend on the clustering algorithms. $K$-means and hierarchical clustering use distance measures, while spectral clustering uses similarity measures. To convert distance to similarity (e.g. to apply spectral clustering), one can use the Gaussian kernel, $a = \exp(-d^2/2 \sigma^2)$. 

\item The choic of distance metric strongly depends on the problems of hand, this is similar to the choice of error/loss function. Considerations may include: the sensitivity to large difference of features (if so, then sensitive to outliers), the asymptotic behavior (e.g. the distance may approach a constant when the difference is large), the weighting of features, etc. 

\item Common distance metrics for feature vectors: (1) for ordinal variables represented by $M$ continguous integers: replace the value by $(i - \frac{1}{2})/M$, and treat as quantitative variables; (2) for categorical variables: suppose there are $M$ categories, then $M \times M$ matrix for every possible pair. 
\item Feature weighting: let $w_j$ be the weight of feature $j$ when computing distance, then the average distance of all pairs: 
\begin{equation}
\bar{D} = \sum_{j=1}^p w_j \bar{d}_j	
\end{equation}
where $\bar{d}_j$ is the average distance of the $j$-th feature, it is also equal to $2 \times \text{var}_j$. Thus the relative importance of each feature is proportional to its variance over the dataset (intuitively clear as large variance features should make the data more separable). Note that under common standardization, the variance of all feature is equal to 1, and this may not be good. 

\item Distance/similarity metrices for general objects, e.g. sequences: (1) edit distance, this includes variants that considers the alignment of two objects (e.g. cross-correlation for time-series data); (2) feature representation of objects: e.g. Fourier transform of time-series data, and use the coefficients as the features; (3) hidden dimensions: e.g. divergence time between two sequences, spatial distance betwen two objects. 

\item Properties of distance measure: generally need nonnegativity, and $d(x,x) = 0$, symmetry. Triangle inequality may be desired, but not always satisfied. 

\item Symmetric distance: if a distance measure is not symmetric, one can often transform it as a symmetric measure as: $[d(x,y) + d(y,x)]/2$. 

\item Reference: [Liao, Clustering of time series data - a survey, Pattern Recognition, 2005]. 
\end{itemize}

Probablity-based distance/similarity measures: 
\begin{itemize}
\item KL divergence: suppose $\theta_x$ and $\theta_y$ are the models that generate the two objects $x$ and $y$, respectively, then the distance between $x$ and $y$ can be defined as $KL(\theta_x||\theta_y)$. Note that KL divergence has a likelihood interpretation: the divergence bewteent two distributions $P$ and $Q$ is the likelihood of generating data of $P$ using the distribution $Q$, when the data size approaches infinity. 

\item Likelihood based measure: for instance, for two sequences, $x$ and $y$, suppose the model of $x$ is $\theta_x$ (MLE), then the similarity between $x$ and $j$ can be defined as the normalized log likelihood:  
\begin{equation}
l_{xy} = \frac{1}{\text{length}(y)} \log P(y|\theta_x)	
\end{equation}
To make it symmetric, we simply have: $d_{xy} = (l_{xy} + l_{yx})/2$. Transformation may be desired to make this a distance, in particular, to use the difference of log likelihood as the distance, e.g.:
\begin{equation}
d_{xy}^{BP} = \frac{1}{2} \left( \frac{l_{xy} - l_{xx}}{l_{xx}} + \frac{l_{yx} - l_{yy}}{l_{yy}} \right)
\end{equation}
See [Garcia-Garcia, A new distance measure for model-based sequence clustering, IEEE Pattern Analysis and Machine Learning]. 

\item Hypothesis testing: this is to test the hypothesis, $H_0$: the objects $x$ and $y$ are from the same distribution, vs. $H_A$: they are from different distributions. Suppose we form the LRT of the hypothesis, $\lambda$, then the distance can be defined as $\lambda$ or the CDF function at $\lambda$ (effectively normalize $\lambda$ s.t. it is between 0 and 1). See [Kumar, Clustering Seasonality Patterns in the Presence of Errors, KDD02], [Bagnall, A likelihood ratio distance measure for the similarity between the fourier transform of time series]. 
\end{itemize}

Combinatorial algorithms for clustering: one major paradigm for clustering. 
\begin{itemize}
\item Idea: find cluster assignment s.t. the total distance within clusters is minimum, or the distance between clusters is maximum. These two objectives are equal: suppose the total distance is $T$, and $W(C)$ is the within-cluster distance and $B(C)$ the between-cluster distance, then: 
\begin{equation}
T = W(C) + B(C)	
\end{equation}

\item Search strategy: often greedy algorithm that converge to local optima. 
\end{itemize}

$K$-means and $K$-mediods algorithm: 
\begin{itemize}
\item $K$-means: alternate two steps: (1) given the cluster assignment, find the means of the clusters; (2) given the cluster means, find the best cluster assignment. 
\item $K$-mediods: when the data cannot be treated as Gaussian variables, cluster means are not well-defined, thus for each cluster, use one cluster member instead. The same algorithm can be applied, except that the cluster means are replaced by the cluster centers. 
\item Choosing $K$: plot the reduction of total within-cluster distance as a function of $K$. The reduction should be large when $K$ is less than the natural/ideal $K$, but slow down when $K$ is bigger than the natural value. So look for the ``kink'' in the plot. The idea can be formulated by the gap statistic: the reduction of distance comparing with the reduction under the uniform (no cluster-structure) data. 
\item Application: image compression/vector quantization. No need to represent every pixel (8-bit, for 256 colors) in an image: there are a much smaller number of basic patterns (e.g. all-black, all-white), thus apply $K$-means to cluster all basic blocks and only record the cluster index of each block. 
\end{itemize}

%\item{Mixture models} [Zhu, ICML Tutorial, 2007; Ghahramani \& Jordan, NIPS, 1993; Nigram, ML, 1999]
%EM algorithm: need to maximize the data likelihood: 
%\beq
%l(\theta) = \sum_{i \in L}\log [P(y_i|\theta) P(x_i|y_i,\theta)] + \sum_{i \in U}\log \left[ \sum_{y} P(y|\theta) P(x_i|y,\theta) \right]
%\eeq
%The complete data likelihood is given by: 
%\beq
%\log P(X_l, Y_l, X_u, Y_u|\theta) = \sum_{i \in L}\log [P(y_i|\theta) P(x_i|y_i,\theta)] + \sum_{i \in U}\log [P(y_i|\theta) P(x_i|y_i,\theta)]
%\eeq
%Taking expectation over $P(y_i|x_i,\theta^t)$ for $i \in U$, where $\theta^t$ is the current value of $\theta$: 
%\beq
%\begin{array}{lll}
%Q(\theta|\theta^t) & = & E_{Y_u}[\log P(X_l, Y_l, X_u, Y_u|\theta)] \\
%& = & \sum_{i \in L}\log [P(y_i|\theta) P(x_i|y_i,\theta)] + \sum_{i \in U} \sum_{y} P(y|x_i, \theta^t) \log [P(y|\theta) P(x_i|y,\theta)]
%\end{array}
%\eeq
%This is the quantity to be maximized in M-step. 
%
%Extensions of the basic EM algorithm: 
%\ben
%\item Weighting labeled and unlabeled data: might to put more emphasis on the labeled data, thus maximize the following function instead: 
%\beq
%l(\theta) = \sum_{i \in L}\log [P(y_i|\theta) P(x_i|y_i,\theta)] + \lambda \sum_{i \in U}\log \left[ \sum_{y} P(y|\theta) P(x_i|y,\theta) \right]
%\eeq
%where $\lambda$ controls the importance of the unlabeled data, $0 \leq \lambda \leq 1$. In the case where the EM update uses the count from the data point, this will be equivalent to, weight the count of each data point by $\lambda$ or 1, depending on if this data point is unlabeled or labeled [Nigram, ML, 1999]. 
%
%\item Multiple mixture components per class. 
%\een

\item{Spectral clustering}

Overview of spectral clustering: 
\begin{itemize}
\item Global vs. local structure: the common clustering algorithms, are based on the global structure of the data. E.g. the $k$-means algorithm is based on the distance of a point to the center of a cluster. The local structure cannot be captured. For instance, suppose the cluster consists of points in a circle, then the points are mutually close, but not necessarily close to any center. 
\item Non-convex sets: A particular example of the problem is that the clusters may not for convex-sets, which are generally not captured by $k$-means, $k$-medoid, etc. 
\item Idea of spectral clustering: find clusters s.t. the points with a cluster are mutually close, and the points across clusters are distant. The difference with algorithms such as $k$-means: the distances of close neighbors are easier defined (e.g. use $\epsilon$-neighborhood graph, see below) than the distance of a point to the center. 
\item Reference: [Luxburg, A Tutorial on Spectral Clustering, 2007]
\end{itemize}

Similarity graph: 
\begin{itemize}
\item $\epsilon$-neighborhood graph: connect all points whose pairwise distances are less than $\epsilon$, and the rest are not directly linked. Usually used as unweighted graph. 

\item $k$-nearest neighbor graph: for each point, connect to its $k$ nearest neighbors. However, this is asymmetric. To have an undirect (symmetric) graph, one can link $i$ and $j$ if $i$ is the among the top $k$ nearest neighbors of $j$, and vice versa. 

\item Fully connected graph: for any two points, link the two with the weight defined as a function of the distance between two points. One commonly use the Gaussian similarity function: 
\begin{equation}
s(x_i, x_j) = \exp\left[-\norm{x_i - x_j}^2 / (2 \sigma^2)\right]	
\end{equation}
\end{itemize}

Unnormalized graph Laplacian: 
\begin{itemize}
\item Definitions: the weighted adjacency matrix $W = (w_{ij}) \geq 0, 1 \leq i, j \leq n$, if $i$ and $j$ are linked by an edge, and it is 0 otherwise. The degree matrix is a diagonal matrix $D = (d_i)$, where $d_i = \sum_j w_{ij}$. We define the weights between two sets $A$ and $B$ as: 
\begin{equation}
W(A,B) = \sum_{i \in A, j \in B} w_{ij}
\end{equation}
The volume of a set $A$ is defined as: $\text{vol}(A) = \sum_{i \in A} d_i$. The unnormalized graph Laplacian is defined as: 
\begin{equation}
L = D - W	
\end{equation}

\item Graph Laplacian and smoothness of the graph: the primary motivation of defining a graph Laplacian is that it is related to the smoothness measure of a graph. Suppose we have labels of the nodes in the graph, $f_i$ (e.g. the cluster membership). The ``smoothness'' of $f_i$, i.e. how abruptly $f_i$ changes in neighbors (in general, we want highly-weighted neighbors to have simliar labels), can be defined as the sum of $(f_i - f_j)^2$, weighted by the edge weight, over all edges. This smoothness penalty is related to the graph Laplacian by: 
\begin{equation}
f^T L f = \frac{1}{2} \sum_{i,j} w_{ij} (f_i - f_j)^2
\label{eq:graph_Laplacian}
\end{equation}
for any $f \in \mathbb{R}^n$. 

\item Eigenvalues and eigenvectors of graph Laplacian: $L$ has the following properties: 
\begin{itemize}
\item $L$ is symmetric and postive semi-definite. The proof follows from Equation~\ref{eq:graph_Laplacian}: $f^T L f \geq 0$ for any $f$. 
\item $L$ has $n$ non-negative eigenvalues with $0 = \lambda_1 \leq \lambda_2 \leq \cdots \leq \lambda_n$, and the eigenvector corresponding to $\lambda_1 = 0$ is the unit vector $\mathbf{1}$. This is easy to prove by checking that $L \mathbf{1} = 0 \mathbf{1}$. 
\end{itemize}
 
\item Graphs with multiple connected components: for a graph with $k$ connected components, $A_1, \cdots, A_k$, the multiplicity of the eigenvalue 0 is equal to $k$, and the corresponding eigenvectors are $\mathbf{1}_{A_1}, \cdots, \mathbf{1}_{A_k}$ respectively, where $\mathbf{1}_{A}$ represents a vector whose $i$-th compoment is equal to 1 if $i \in A$ and 0 otherwise. \\
Proof: first, when $k = 1$, i.e. the graph is connected, let $f$ be the eigenvector of 0, then $Lf = 0$ and thus: 
\begin{equation}
f^T L f = 0 = \frac{1}{2} \sum_{i,j} w_{ij} (f_i - f_j)^2	
\end{equation}
Thus every term in the RHS is 0. If $i$ and $j$ are linked, then $w_{ij} > 0$, thus $f_i = f_j$. From this, we see that $f$ must be proportional to the vector $\mathbf{1}$. \\
For arbitrary $k$, we could order the vertices s.t. the matrix $W$ has a block form, and so with $L$, with $k$ blocks, $L_1, \cdots, L_k$ (and all other terms in $L$ are equal to 0). We apply the previous result to each of $L_k$. 

\end{itemize}


Normalized graph Laplacian: 
\begin{itemize}
\item Two normalized graph Laplacian: if we normalize the label $f$ of a graph, as: $\tilde{f}_i = f_i / \sqrt{d_i}$, or $\tilde{f} = D^{-\frac{1}{2}} f$, then the smoothness penalty is the quandratic from of the matrix, $L_{\text{sym}} = D^{-\frac{1}{2}} L D^{-\frac{1}{2}}$: 
\begin{equation}
f^T L_{\text{sym}} f = \frac{1}{2} \sum_{i,j} w_{ij} \left( \frac{f_i}{\sqrt{d_i}} - \frac{f_j}{\sqrt{d_j}} \right)^2
\end{equation}
A second normalized graph laplacian is defined as (used in random walk): 
\begin{equation}
L_{\text{rw}}	= D^{-1} L = I - D^{-1} W
\end{equation}

\item Eigenvalues and eigenvectors of symmetric graph Laplacian: the matrix $L_{\text{sym}}$ is symmetric and p.s.d, and have $n$ non-negative eigenvalues with $0 = \lambda_1 \leq \lambda_2 \leq \cdots \leq \lambda_n$, and the eigenvector corresponding to $\lambda_1 = 0$ is $D^{\frac{1}{2}} \mathbf{1}$. \\
Proof: check that: 
\begin{equation}
L_{\text{sym}} D^{\frac{1}{2}} \mathbf{1} = D^{-\frac{1}{2}} L D^{-\frac{1}{2}} D^{\frac{1}{2}} \mathbf{1} = D^{-\frac{1}{2}} L \mathbf{1} = 0
\end{equation}

\item Eigenvalues and eigenvectors of random walk graph Laplacian: 
\begin{itemize}
\item $\lambda$ is an eigenvalue of $L_{\text{rw}}$ with eigenvector $u$ if and only if $\lambda$ is an eigenvalue of $L_{\text{sym}}$ with eigenvector $w = D^{\frac{1}{2}} u$. 
\item $\lambda$ is an eigenvalue of $L_{\text{rw}}$ with eigenvector $u$ if and only if $\lambda$ and $u$ solves the generalized eigen-problem $L u = \lambda D u$. 
\item The matrix $L_{\text{rw}}$ is symmetric and p.s.d, and have $n$ non-negative eigenvalues with $0 = \lambda_1 \leq \lambda_2 \leq \cdots \leq \lambda_n$, and the eigenvector corresponding to $\lambda_1 = 0$ is $\mathbf{1}$. 
\end{itemize}

\item Graphs with multiple connected components: for a graph with $k$ connected components, $A_1, \cdots, A_k$, the multiplicity of the eigenvalue 0 of both $L_{\text{sym}}$ and $L_{\text{rw}}$ is equal to $k$. For $L_{\text{rw}}$, the corresponding eigenvectors are $\mathbf{1}_{A_1}, \cdots, \mathbf{1}_{A_k}$ respectively. For $L_{\text{sym}}$, the corresponding eigenvectors are $D^{\frac{1}{2}} \mathbf{1}_{A_1}, \cdots, D^{\frac{1}{2}} \mathbf{1}_{A_k}$ respectively.   \\
\end{itemize}

Spectral clustering idea: 
\begin{itemize}
\item Simple case: a graph with $K$ connected components. The graph Laplacian has $K$ eigenvectors with $u_k = \mathbf{1}_k, 1 \leq k \leq K$. We could form the matrix $U = (u_1, \cdots, u_K)$. Clearly, $U$ stores the membership indices: for the $k$-th component (cluster), only those $x_i$'s in this cluster has $U_{ik} = 1$. Thus we could view the $i$-th row of $U$ as a $K$-dimensional representation of $x_i$ ($u_1, \cdots, u_k$ form an orthogonal basis). 

\item General case: perturbation of the simple case, where there may be (weak) edges between components. The $k$-th eigenvector $u_k$ would not be exactly equal to $\mathbf{1}_k$, but will be close. $u_k$ stores the projection of all $x_i$'s on the $k$-th dimension, and it will be dominated by the $k$-th cluster (with small contributions from other clusters weakly connected to $k$). 

\item Algorithmic idea: to partition a graph into $K$ clusters, we compute the $K$ eigenvectors corresponding to the $K$ smallest eigenvalues, and form the matrix $U$. Then the $i$-th row is a representation of $x_i$ in the $K$-dim. eigenspace. We then use usually clustering algorithm, e.g. $K$-means, on the new representation. 

\item Three graph Laplacian and three spectral clustering algorithms: all three Laplacian, $L$, $L_{\text{sym}}$ and $L_{\text{rw}}$ encodes the information of connected components, and cluster structure (through perturbation analysis), so we can use the eigenvectors corresponding to any of these Laplacian, leading to three Spectral Clustering algorithms. However, for $L_{\text{sym}}$, since the norm of the eigenvectors of connected components are not constant, we need to normalize the row of the matrix $U$. 
\end{itemize}

Graph cut: 
\begin{itemize}
\item Motivation: partition the graph into $k$ clusters s.t. the weights (similarity) within components are high and the weights across components are low. Since the sum of all weights is constant, we will only need to focus on the weights across the components. Given a graph partition, $A_1, \cdots, A_k$, we define the cut as: 
\begin{equation}
\text{cut}(A_1, \cdots, A_k) = \frac{1}{2} \sum_{i=1}^k W(A_i, \bar{A_i})	
\end{equation}

\item Problem of unnormalized graph cut: two ways of seeing the problem: 
\begin{itemize}
\item Trivial solution: to minimize $f^T L f$ without any additional constraint, the trivial solution is $f = 0$. 
\item Number of edges between components: Consider the case of $k = 2$, clearly, if $|A| = 1$, there are $n - 1$ (upper bound) edges between $A$ and $\bar{A}$; at $|A| = 2$, there are $2 (n-2)$ edges between $A$ and $\bar{A}$; and so on. The number of edges is not constant. 
\end{itemize}
So we will need to either reformulat the objective function (a normalized version), or impose additional constraints. 

\item Normalization: we could define the cut as the average weight between two components: 
\begin{equation}
\text{RatioCut}(A_1, \cdots, A_k) = \frac{1}{2} \sum_{i=1}^k \frac{W(A_i, \bar{A_i})}{|A_i|}
\end{equation}
We could also define the normalized cut as: 
\begin{equation}
\text{Ncut}(A_1, \cdots, A_k) = \frac{1}{2} \sum_{i=1}^k \frac{W(A_i, \bar{A_i})}{\text{vol}(A_i)}
\end{equation}
\end{itemize}

RatioCut at $k = 2$: 
\begin{itemize}
\item Objective function: we have the optimization problem: 
\begin{equation}
\min_{A \subset V} \text{RatioCut}(A, \bar{A}) 
\end{equation}
We define a membership representation: $f_i = a > 0$ if $v_i \in A$, and $f_i = -b < -$ if $v_i \in \bar{A}$. Clearly, finding $A$ is equivalent to finding $f$ s.t. some constraint. 

\item Express the RatioCut as a function of $f$: we have: 
\begin{equation}
f^T L f = \frac{1}{2} \sum_{i,j} w_{ij} (f_i - f_j)^2 = (a + b)^2 \text{cut}(A, \bar{A})	
\end{equation}
If we choose $a = \sqrt{|\bar{A}| / |A|}$ and $b = \sqrt{|A|/|\bar{A}|}$, then we have: 
\begin{equation}
f^T L f = |V| \text{RatioCut}(A, \bar{A})	
\end{equation}
The vector $f$ is subject to the constraint that $\sum_i f_i = 1$, i.e. $f \bot \mathbf{1}$, and $\norm{f}^2 = n$. Our problem is thus to minimze $f^T L f$ subject to the constraints of $f$ (discrete). 

\item Continuous relaxiation: the problem is NP-hard because $f$ can only take two values. We perform the continuous relaxation, and this leads to the problem: 
\begin{equation}
\min_{f \in \mathbb{R}^n} f^T L f \text{ subject to }	f \bot \mathbf{1}, \norm{f} = \sqrt{n}
\end{equation}
We could understand the contraints as: (1) The labels should be balanced, i.e. $\sum_i f_i = 0$; (2) The scale of the labels should be equal to 1 (a constant), otherwise, we could scale $f_i$ to be arbitrarily small. This means the mean of $f_i$ norm should be 1, or $\norm{f} = \sqrt{n}$. 

\item Solving the optimization problem: Using the Rayleigh quotient, it is to see that the solution is the eigenvector corresponding to the second smallest eigenvalue of $L$ (recall that the smallest eigenvalue of $L$ is 0 with eigenvector $\mathbf{1}$). Once we have $f$, we simply cluster all points by $f$ (1-D clustering). 
\end{itemize}

RatioCut for arbitrary $k$: 
\begin{itemize}
\item Idea: express the RatioCut at the general cut, as the sum of $\text{cut}(A_k, \bar{A_k}), 1 \leq k \leq K$. Each of the $K$ terms is related to the graph Laplacian. If we define a membership matrix $H = (h_{ik}), 1 \leq i \leq n, 1 \leq k \leq K$, where $h_{ik} = 1/\sqrt{|A_k|}$ if $v_i \in A_k$, and 0 otherwise, then we have: 
\begin{equation}
\text{RatioCut}(A_1, \cdots, A_K) = \sum_{k=1}^K h_k^T L h_k = \tr(H^T L H)	
\end{equation}
The matrix $H$ is subject to the constraint $H^T H = I$. 

\item Algorithm: do the relaxation, and solve the trace minimization problem. The resulting algorithm: find the top $K$ eigenvectors (corresponding to $K$ smallest eigenvalues) of $L$, and use the $K$-means algorithm to cluster the rows of the matrix of the $K$ eigenvectors. 
\end{itemize}

Ncut algorithm: 
\begin{itemize}
\item Idea for $k = 2$: similar to RatioCut, define the cluster membership function $f$ (in the case of $K = 2$), s.t. the quadratic form $f^T L f$ corresponds to the Ncut. Specifically, we are solving the problem: 
\begin{equation}
\min_{f \in \mathbb{R}^n} f^T L f	\text{ subject to } D f \bot \mathbf{1}, f^T D f = \text{vol}(V)
\end{equation}
This is the problem of Generalized Rayleigh Quotient, and it can be solved by the transformation, $g = D^{1/2} f$. 

\item Intuition of the optimization problem: similar to the RatioCut case, we could understand the contraints as: (1) The labels should be balanced with weighting, i.e. $\sum_i d_i f_i = 0$; (2) The weighted average of the norm of $f_i$ should be equal to 1, or $\sum_i d_i f_i^2 = \sum_i d_i$. 

\item Algorithm in the general case: find the first $K$ eigenvectors of the matrix $L_{\text{rw}}$, or the first $K$ generalized eigenvectors of $L u = \lambda D u$. Then cluster the $n$ rows of the matrix consisting of the $K$ eigenvectors. 
\end{itemize}

Random walk interpretation of Ncut: 
\begin{itemize}
\item Transition between clusters: suppose $G$ is connected and non-bipartite, suppose $X_0$ is the stationary distribution, we define $P(B|A) = P(X_1 \in B|X_0 \in A)$, then we have: 
\begin{equation}
\text{Ncut}(A,\bar{A}) = P(A|\bar{A}) + P(\bar{A}|A)	
\end{equation}
Proof: for any set $A$ and $B$, we have: 
\begin{equation}
P(A,B) = \sum_{i \in A,j \in B} P(X_0 = i, X_1 = j) = \sum_{i \in A,j \in B} \pi_i q_{ij} = \frac{1}{\text{vol}(V)} \sum_{i \in A,j \in B} w_{ij}
\end{equation}
We could easily obtain $P(A)$, and thus the conditional probability $P(B|A)$. Plug in the terms for $P(A|\bar{A})$ and $P(\bar{A}|A)$, respectively. 

\item Random walk interpretation: minizing Ncut is equivalent to finding a partition of the graph s.t. the transitions between the two clusters have low probabilities. 
\end{itemize}

How spectral clustering works? 
\begin{itemize}
\item Community/clique structure: if there is a clique structure in the graph, then it is costly to break the elements of the clique into multiple components (large inter-cluster weights). Thus the algorithm will try to put any clique-like structure into the same cluster. 

\item Boundary at low-density regions: if a region has low density, then the cost of split the points in the region into multiple clusters is relatively low. Thus the boundary of the clusters from the spectral clustering algorithm tends to be in the low-density regions: for many problems, if the clusters are separable, then these should match the true cluster boundaries. 
\end{itemize}

Considerations of spectral clustering: 
\begin{itemize}
\item Lesson: in general, a statistical learning method makes some assumptions of the data, and if these assumptions do not hold, the performance of the method may be poor. It is important to: (1) understand the implications of these assumptions: what kind of solutions will be ``favored'' by the methods; (2) examine these assumptions in a problem, e.g. by examining some plausible special cases. 

\item Overlapped clusters: if clusters highly overlap, then the assumption that cluster boundaries are in the low-density regions does not hold.  

\item Different cluster sizes: the RatioCut and Ncut algorithms all make some implicit assumptions of the cluster sizes: e.g. RatioCut tends to favor equal cluster sizes. When the real cluster sizes are very different, this may create a problem. 

\item Different cluster densities: this would imply that the degrees of nodes within clusters are very different across clusters. This is related to the issue of how to define the similarity graph: e.g. a fixed $K$-NN graph may not be a good choice. 

\item Which graph Laplacian? $L_{\text{rw}}$ is generally preferred, it minimizes the inter-cluster similarity, and also maximizes the intra-cluster similarity. It has a random-walk interpretation and better consistency properties than $L_{\text{sym}}$. 
\end{itemize}

Relation to Laplace operator: 
\begin{itemize}
\item The smoothness function: describe how variable $f$ is across the graph. This is similar to the problem of describing how a function varies in a region. Let $w_{ij} = 1/ d_{ij}^2$, then: 
\begin{equation}
w_{ij} (f_i - f_j)^2 = \left(\frac{f_i - f_j}{d_{ij}}	\right)^2
\end{equation}
is like a difference quotient. Then to minimize the function $\sum_{ij} w_{ij} (f_i - f_j)^2$ is like to minimize $\int_V \norm{\nabla f}^2 dV$, and the solution satisfies the Laplace's Equation $\Delta f  = 0$. 

\end{itemize}

Questions of Spectral Clustering: 
\begin{itemize}
\item How to choose the number of clusters? And how the eigenvalues can be used? 
\end{itemize}

\end{enumerate}
%%%%%%%%%%%%%%%%%%%%%%%%%%%%%%%%%%%%%%%%%%%%%%%%%%%%%%%%%%%%
\section{Manifold Learning}

Visualizing Data Using t-SNE [Laurens van der Maaten, Google Talks], An illustrated introduction to the t-SNE algorithm, \url{https://www.oreilly.com/learning/an-illustrated-introduction-to-the-t-sne-algorithm}
\begin{itemize}
	\item Embedding problem: represent high-D. data points in a low-D, such that the distance in high-D is preserved. The challenge is what distance metric to be preserved. For non-linear data, generally manifold distance is more preferred. 
	
	\item Example: images of 10 digits. Why PCA does not work? PCA maximizes variance, it cares about large distance. Ex. two very different digits 0 and 1, should be separated in PC space (maximum variance requirement). But it does not force close points in high-D to be close in low-D. 
	
	\item Idea of t-SNE: let $x_i$ be original data, and $y_i$ be low-dim. embedding. We define $p_{ij}$ as distance of $x_i$ and $x_j$ in the original space, and $q_{ij}$ in the low-D space. The idea is to place $y_i$’s s.t. $p_{ij}$ are similar to $q_{ij}$; more precisely, if $p_{ij}$’s are small, $q_{ij}$ should also be small. 
	
	\item High-dim. distance, we use Gaussian kernel $p_{ij}$. Because density varies at different data points, so use conditional Gaussian condition, $p_{j|i} \propto \exp(- \norm{x_i - x_j}^2 / 2 \sigma_i^2)$, where $\sigma_i$ varies with $i$. Similarly define $p_{i|j}$, and the average of the two is $p_{ij}$. 
	
	\item Low-dim. distance: similar idea, but use t-distribution. The idea is to use a long tail distribution: since we mainly care about preserving local structure. As long as two points close in high-D space close are also close in low-D, that's fine. 
	
	\item Minimizing KL divergence between $P$ and $Q$: 
	\begin{equation}
	KL(P||Q) = \sum_{i,j} p_{ij} \log \frac{p_{ij}}{q_{ij}}
	\end{equation}
	The effect of this is: when $p_{ij}$ is large, but $q_{ij}$ small, then large penalty (we wrongly place distant points in the original space close to each other in the manifold). But when $p_{ij}$ is small, and $q_{ij}$ large, small penalty. 
	
	\item Gradient interpretation of the algorithm (Physical analogy): N-body problem, the gradient of the objective function represents the force acting on an object. The solution is the equilbrium of the system. 
	
	\item Computational efficiency: group data points that are close, and represent them by their centroid. Do this recursively (tree). 
	
	\item Applications of t-SNE to the digit example: 10 clusters, with each representing a digit. The distance between clusters are not meaningful in t-SNE. 
	
	\item More applications: use deep learning to extract features, then visualize with t-SNE. 
	
	\item \textbf{Lesson}: to make inference, a different class of methods focus on inferring the underlying parameters/processes that ``preserves'' the structure of data, in terms of pairwise distance, instead of generating data itself. 
\end{itemize}


\subsection{Semi-supervised learning}

Reference: [Zhu \& Goldberg, Introduction to Semi-supervised Learning, 2009]

Why semi-supervised learning is possible? 
\begin{itemize}
\item Better decision boundary: e.g. in a 1D case, given a small number of labeled instances, there are many possible boundaries to separate the positive and negative instances. Suppose we add the unlabeled data, and choose the boundary between two clusters, then the decision boundary is close to the real boundary. 

\item Label propagation: by propagating labels to nearest instances (most confident), we add new data points to training data, and this may help (self-training is a simple way of using this idea). 

\item Caveats: the unlabeled data may not always help, e.g. when the clusters highly overlap. 
\end{itemize}

Inductive and transductive learning: 
\begin{itemize}
\item Inductive learning: predict the labels on future test data.

\item Transductive learning goal: predict the labels on the unlabeled instances in the training sample.
\end{itemize}

Self-training: 
\begin{itemize}
\item Notation: $(X_l,Y_l)$ are labeled data and $X_u$ are unlabeled data, want to learn the function $f:X \rightarrow Y$. 

\item Algorithm: 
\begin{itemize}
\item Train f from $(X_l, Y_l)$;
\item Predict on $x \in X_u$;
\item Add $(x, f(x))$ to the labeled data;
\item Repeat. 
\end{itemize}
An example is the propgating 1-nearest neighbor (1-NN) algorithm, which adds the nearest neighbor of each labeled instance in the unlabeled data at each step. 

\item Variations of the basic algorithm: 
\begin{itemize}
\item Add a few most confident $(x, f(x))$ to labeled data
\item Add all $(x, f(x))$ to labeled data
\item Add all $(x, f(x))$ to labeled data, weighting each by confidence
\end{itemize}

\item Remark: the self-training algorithm makes the assumption that the high-confience predictions tend to be correct. However, when the algorithm makes a mistake at the beginning, the error will be amplified. The algorithm is sensitive to outliers, e.g. the ones that are close to both clusters, and easy to misclassifer. 
\end{itemize}

Mixture model approach: 
\begin{itemize}
\item The likelihood: suppose we have the labeled data, $(x_i, y_i), 1 \leq i \leq l$, and unlabeled data, $x_i, l+1 \leq i \leq l+u$. The log-likelihood is: 
\begin{equation}
\log P(D|\theta) = \sum_{i=1}^l \log p(y_i|\theta) p(x_i|y_i,\theta) + \sum_{i=l+1}^{l+u} p(x_i|\theta)	
\end{equation}
We then perform a mixture-model estimation plus the labeled data. EM algorithm can be similarly used. The intuition (for GMM): at each step, we update the mean of each cluster using all labeled data in that cluster plus all predicted data in that cluster weighted by the posterior probabilities. 

\item Extension: typically, we have a weighted log-likelihood so that the labeled data contribute more: 
\begin{equation}
\sum_{i=1}^l \log p(y_i|\theta) p(x_i|y_i,\theta) + \lambda \sum_{i=l+1}^{l+u} p(x_i|\theta)	\qquad 0 \leq \lambda \leq 1
\end{equation}
As $\lambda \to 0$, the problem is reduced to supervised learning using only the labeled data. 
\end{itemize}

Cluster-and-label approach:
\begin{itemize}
\item Idea: if two data points belong to the same cluster, then they are likely to share the same label. 

\item Algorithms: the basic framework is: cluster data into multiple clusters, then apply a classifier in each cluster using the labeled instances in that cluster. Some options for the classifier: 
\begin{itemize}
\item Majority vote: run clustering first on $X_l$ and $X_u$, then label a data point in $X_u$ by the majority of points in that cluster. 
\item Cluster kernel: (used with discriminative learning framework) the kernel function of two data points depends on how often they are clustered together, supposing a clustering algorithm, e.g. K-means, is run multiple times (or on subset of data via sampling). [Weston \& Noble, Bioinfo, 2005]
\end{itemize}
\end{itemize}

Graph-based method: MinCut algorithm: 
\begin{itemize}
\item Model: let $f_i \in \{-1, +1\}$ be the predicted label of the $i$-th node, our goal is to minimize the cut: 
\begin{equation}
\min_{f: f_i \in \{-1, +1\}}	\sum_{i,j} w_{ij} (f_i - f_j)^2 \text{ subject to } f_i = y_i, 1 \leq i \leq l
\end{equation}

\item Algorithm: there exists efficient algorithm for solving the MinCut problem above (max. flow?). 

\item Remark: a number of problems with this version of MinCut, the multiplicity of solutions, lack of normalization, etc. 
\end{itemize}

Gaussian random field (GRF) method: 
\begin{itemize}
\item Reference: [Zhu \& Lafferty, Semi-Supervised Learning Using Gaussian Fields and Harmonic Functions, ICML, 2003]

\item GRF model: suppose we have labels $f$, which satisfy $f_i = y_i$ for labeled data. We want to choose $f$ on the unlabeled data s.t. the unlabeled points that are nearby in the
graph have similar labels. This is accompolished by the energy function: 
\begin{equation}
E(f) = \frac{1}{2} \sum_{i,j} w_{ij} (f_i - f_j)^2	
\end{equation}
The probability distribution of $f$ thus follows the Boltzman distribution, $P(f) = \frac{1}{Z} \exp[-\beta E(f)]$. Instead of sampling from $P(f)$, we find $f$ that minimizes $E(f)$. \\
Note: in the Ising model, the energy function has the form $w_{ij} f_i f_j$, and this is different from GRF model. 

\item Solving the GRF model: we write the optimization problem in matrix form: 
\begin{equation}
\min E(f) = f^T L f \text{ subject to } f_l = y_l
\end{equation}
where $f_l$ and $y_l$ represent the vector of $f$ and $y$ on the labeled data. We use the Lagrange multiplier method, define: 
\begin{equation}
\Omega(f, \lambda) = f^T L f + \lambda (f_l - y_l)	
\end{equation}
Take the derivative of $\Omega(f,\lambda)$: 
\begin{equation}
\frac{\partial \Omega(f,\lambda)}{\partial f}	= 2 L f + \left[ \begin{array}{l} \lambda\\ 0 \end{array} \right] = 0
\end{equation}
We have the solution: $Lf = 0$ on unlabeled data, and $f_l = y_l$ on the labeled data. To have a closed-form solution of $f$, we solve $Lf = 0$ or $Wf = Df$ on the unlabeled data: 
\begin{equation}
\left[ \begin{array}{cc}
W_{ll} & W_{lu} \\
W_{ul} & W_{uu} 
\end{array} \right] 
\left[ \begin{array}{c}
f_l \\
f_u 
\end{array} \right] 
= 
\left[ \begin{array}{c}
D_l f_l \\
D_u f_u 
\end{array} \right] 	
\end{equation}
The solution: $f_u = (D_{uu} - W_{uu})^{-1} W_{ul} f_l$.  

\item Harmonic function analogy: this is similar to the problem of minimizing the Dirichlet energy of $f$ on a region, and the solution is given by the Laplace's Equation, $\Delta f=0$, i.e. the solution $f$ is a harmonic function. 

\item Interpretation of the solution: expand $Lf = 0$ on the unlabeled data, we have, for any point $j$, its label $f_j$ is the weighted average of the labels of its neighbors: 
\begin{equation}
f_j = \frac{1}{d_j} \sum_i w_{ij} f_i	
\end{equation}
We could also write this as: 
\begin{equation}
f = Q f	
\end{equation}
where $Q = D^{-1} W$ is the transition probability matrix. This leads to a fixed point iteration algorithm: starting with any random assignment, at each step, update the label of a node with the weighted average of its neighbors. To prove the convergence, we use the fact that the eigenvalues of $Q$ is in $[-1,1]$ (Perron-Frobenius Theorem). 

\item Normalization by class mass normalization (CMN): the problem with the GRF model is that it tends to produce severely unbalanced classification (the same reason that spectral clustering without normalization tends to produce very unbalanced clusters). To address this problem, we choose the decision threshold to reflect the prior ratio of positive and negative classes. Let $q$ and $1 - q$ be the postive and negative class priors, and define the mass of class 1 as $\sum_i f_u(i)$, and the mass of class 0 as $\sum_i (1-f_u(i))$. We classify a node $i$ as 1 iff: 
\begin{equation}
q \frac{f_u(i)}{\sum_i f_u(i)} > (1 - q) 	\frac{1-f_u(i)}{\sum_i (1- f_u(i))}
\end{equation}
 
\item Incorporating external classifier (label prior): suppose we have a prior of unlabeled data, $h_u$, e.g. from an external classifier. The $h_u$ is similar to vertex potentials in the random field. We could incorporate $h_u$ s.t. a node with high $h_u$ is likely to be positive. To do this, we add an auxiliary node for each unlabeled $i$ with label $h_i$, and add an edge between $i$ and its auxilary node with weight $\eta$ (and split the weights of the rest of neighbors). 
\end{itemize}

Relation of GRF model to random walk and electric resistance: 
\begin{itemize}
\item Relation to random walk: suppose our label is 0 or 1, and we define a random walk on $G$ with the labeled instances as sinks. The label $f_i$ is the probability that a particle, starting from the node $i$, hits a labeled node with label 1. To see this, let $Q$ be the transition probability matrix, $q_{ij} = w_{ij} / d_i$, then the ``sinking probability'' of a node $j$ is given by: (moving one step further)
\begin{equation}
f_j = \sum_i q_{ji} f_i = \sum_i \frac{w_{ji}}{d_j} f_i = \frac{1}{d_j} \sum_i w_{ij} f_i
\end{equation}

\item Relation to electric resistance network: suppose the labeled positive nodes are linked to a hihg-voltage point (say 1), and the negative nodes to the ground. Let $w_{ij}$ be the conductance of the edge $(i,j)$. Then $f_i$ is the voltage of the $i$-node. To see this, we apply Kirchoff's Law on the node $j$: 
\begin{equation}
\sum_i (f_i - f_j) w_{ij} = 0 \Rightarrow \sum_i w_{ij} f_i = d_j f_j
\end{equation}
\end{itemize}

Issues/extensions of the basic GRF model: 
\begin{itemize}
\item Allowing errors in the labeled data: obviously, the labels may not be perfect, to allow some disagreement with the given labels, we solve the problem: 
\begin{equation}
\min_{f} \sum_i (f_i - y_i)^2 + \lambda f^T L f
\end{equation}

\item Balancing clusters: the objective function of the basic GRF algorithm is not normalized, thus it has the same problem as the unnormalized graph cut for spectral clustering. In particular, the classes (clusters) may not be balanced. A post-processing by CMN might alleviate the problem, but it's more preferred to change the objective function for normalization. 

\item Scaling of labels: similar to spectral clustering, the basic objective function favor small $f_i$'s. The labeled instances may alleviate the problem, but cannot completely solve it, e.g. when the unlabeled nodes dominate and the clusters are not well separable. 

\end{itemize}

Learning with Local and Global Consistency [Zhou \& Scholkopf, NIPS, 2003]: 
\begin{itemize}
\item Using normalized graph Laplacian: we want the label $f_i$ to have roughly the same norm 1 (we are doing contiuous relaxation, and in the discrete version, $f_i$ should be either 1 or -1), or in other words, the weighted average of $f_i$ to be close to 1. The problem is in the weighted average, the nodes with high degree dominate, thus we divide $f_i$ by $\sqrt{d_i}$, so that the contribution of each node to the weighted average is about the same (the weight is $\sqrt{d_i}$). And we know that: 
\begin{equation}
f^T L_{\text{sym}} f = \frac{1}{2} \sum_{i,j} w_{ij} \left( \frac{f_i}{\sqrt{d_i}} - \frac{f_j}{\sqrt{d_j}}\right)^2	
\end{equation}
where $L_{\text{sym}} = D^{-1/2} L D^{-1/2}$. 

\item Model: we have a regulazation framework, (1) the fitting constraint: $f_i$ should be close to known labels $y_i$ (0 for unlabeled nodes); (2) the smoothness constraint: the labels of the neighbors should be close (with weighting). So we have the objective function: 
\begin{equation}
Q(f) = \frac{1}{2} \sum_{i,j} w_{ij} \left( \frac{f_i}{\sqrt{d_i}} - \frac{f_j}{\sqrt{d_j}}\right)^2 + \mu \sum_i (f_i - y_i)^2
\end{equation}
where $\mu > 0$ is a regularization parameter. Written in the matrix form: 
\begin{equation}
Q(f) = 	f^T L_{\text{sym}} f + \mu (f - y)^T (f - y)
\end{equation}

\item Optimization: first we write $L_{\text{sym}} = D^{-1/2} (D - w) D^{-1/2} = I - S$, where $S = D^{-1/2} W D^{-1/2}$. It is given by: 
\begin{equation}
S_{ij} = \frac{w_{ij}}{\sqrt{d_i d_j}}	
\end{equation}
Thus it is simply a normalized version of $W$, but unlike $Q = D^{-1} W$, the stochastic matrix, the normalization is symmetric wrt. rows and columns. The derivative of $Q(f)$: 
\begin{equation}
\frac{\partial Q}{\partial f}	= 2 [(I - S) f + \mu (f-y)] = 0
\end{equation}
Solving the equation above, let $\alpha = 1 / (1 + \mu)$:
\begin{equation}
f = \alpha S f + (1 - \alpha) y	
\end{equation}
The closed-form solution is: $f = (1 - \alpha) (I - \alpha S)^{-1} y$. 

\item Iterative algorithm: is more efficient in solving the linear equation, especially when the graph is sparse. 
\begin{itemize}
\item Background: to solve a linear equation $Ax = b$, we could use an iterative algorithm that converges to the correct solution. In general, suppose we have a recurrence equation: $x_n = A x_{n-1} + b$, when the absolute value of the largest eigenvalue (absolute value) of $A$ is less than 1, then the algorithm converges to $x^* = A x^* + b$. 

\item Iterative algorithm: to solve our problem, we define the recurrence: 
\begin{equation}
f(t+1) = \alpha S f(t) + (1 - \alpha) y		
\end{equation}
Since $0 \leq \alpha < 1$ and the eigenvalues of $S$ are in $[-1,1]$ ($S$ is similar to the stochastic matrix), the algorithm converges. 
\end{itemize}

\item Intepretation of the iterative algorithm: the algorithm performs label propagation: the lable of the node $i$ in $t+1$ is the weighted average of (1) the labels of its neighbors in $t$, and (2) its initial label (label bias). Note that the label propagation from the neighbors: is now weighted by $s_{ij}$ instead of $w_{ij}$ (symmetric). 

\item Remark: the graph does not have self-loops, i.e. $w_{ii} = 0$. 
\end{itemize}

Questions of GRF model: 
\begin{itemize}
\item The difference between different normalization methods (CMN vs. normalized graph Laplacian)? 

\item Setting the initial label bias $y$, and how it affects the results? How to control for class balancing? 
\end{itemize}
%%%%%%%%%%%%%%%%%%%%%%%%%%%%%%%%%%%%%%%%%%%%%%%%%%%%%%%%%%%%
\section{Multi-Modal Machine Learning}

A Survey of Multi-View Representation Learning [Li and Zhang, IEEE, 2019]
\begin{itemize}
	\item Problem: learn common representations using paired data, e.g. document in two languages, image-text/caption. 
	
	\item Multi-view representation alignment: we project both views in a common space, using $f(X)$ and $g(Y)$. The projections of two views from the same pair should be close. Ex. Cross-modal Factor Analysis minimizes the distance of two projections:
	\begin{equation}
	\min_{W_x, W_y} \sum_i (x_i^T W_x - y_i^T W_y)^2 + r_x(W_x) + r_y(W_y)
	\end{equation}
	where $r_x(W_x)$ and $r_y(W_y)$ are regularization terms.  
	
	\item CCA: let $X$ and $Y$ be paired random vectors. After projection, we want projected $X$'s and $Y$'s to be correlated (smallest angle). Find vectors $w_X$ and $w_Y$ s.t. $\rho = \text{corr}(w_x^T X, w_y^T Y)$ is maximized. In the matrix form (averaging over all paired samples), let $X$ and $Y$ be data matrix, we want:
	\begin{equation}
	\max \text{corr}(w_x^T X, w_y^T Y) \propto w_x^T C_{XY} w_y
	\end{equation}
	where $C_{XY}$ is the covariance matrix. 
	
	\item Deep CCA: MLP applied to $X$ and $Y$, the objective is to maximize the correlation of the hidden states. 
	
	\item Multi-view representation fusion: learn a common representation to relate the two views. Graphical model approach: $p(x, y, z)$, and the representation is given by $p(z|x,y)$. Ex. probabilistic collective matrix factorization (PCMF), let $u_i$ be common hidden variable, and $x_i, y_i$ be two views, the model:
	\begin{equation}
	x_i \sim N(V_X u_i, \sigma_X^2 I) \qquad y_i \sim N(V_Y u_i, \sigma_Y^2 I) 
	\end{equation}
	
	\item Multi-modal LDA: image and text data. For image data, $N$ regions, and text data, $M$ words. The image data at region $n$, $r_n$ depends on the latent variable $z_n$ (topic at region $n$), following MVN. For the word at position $m$, $w_m$: first needs to sample which region is belongs to $y_m$, and then use the topic at this region to sample words. 
	
	\item Multi-modal DBM: use a common hidden layer as the common representations. 
	
	\item Bimodal autoencoder: e.g. audio, video. Let $\hat{x} = f_{\theta}(x)$ be the output of the audio autoencoder, and $\hat{y} = g_{\theta'}(y)$ be the output of the video autoencoder. The two autoencoder share a common bottleneck layer. Choose parameters to minimize the reconstruction error for both $x$ and $y$. 
\end{itemize}

Multimodal learning with Deep Boltzmann Machines [Srivastava, JMLR, 2014]
\begin{itemize}
	\item Motivation: we have paired image and text data (tags of images). The goal is to learn a model that can predict text of images and images from text. 
	
	\item Multinomial output of RBMs: Replicated Softmax Model (Figure 2). To model words at a document: for each word, its multinomial probability, $P(v_i = 1 | h)$, is a softmax function of hidden variables $h$. And the weights are repeated over all words. 
	
	\item Multimodal DBM of joint image-text data (Figure 3): we have DBM for both image and text data (two hidden layers each). In addition, a hidden layer $h^{(3)}$ that is connected to $h^{(2)}$ of both image and text DBM. Intuition: need the hidden layer to connect the hidden representations of image and text. Ex. a `nature' theme in the $h^{(3)}$ would correlate to words such as ``river'', ``mountain'', ``tree'' in text DBM, and to relevant image features in image DBM. 
	
	\item Applications: (1) Learning joint representations: $P(h^{(3)}|v^m, v^t)$. (2) Impute missing modalities: $P(v^t | v^m)$, see Figures 5 and 6. 
\end{itemize}

%%%%%%%%%%%%%%%%%%%%%%%%%%%%%%%%%%%%%%%%%%%%%%%%%%%%%%%%%%%%
\section{Misc. Topics in Machine Learning}
\begin{enumerate}

\item{Feature Selection}
	
Reference: [Guyon, JMLR, 2003]

Motivations: choose variables to: 
\begin{itemize}
\item Improving the prediction performance;
\item Providing faster and more cost-effective predictors;
\item Providing a better understanding of the underlying process that generated the data: the most important featurs. 
\end{itemize}

Variable ranking:
\begin{itemize}
\item Correlation: correlation of varaibles and responses. Most commonly Pearson's correlation. In the case of classification, Pearson's correlation is similar to t-test. By the same token, other two-sample tests can be used for classification: Mann Whitney test, etc. 
\item Single variable classifiers: choose a threshold for the variable. The performance is measured by error rate, or some other criteria defined via FP rate and FN rate, e.g. break-even point (the hit rate for a threshold value for FP rate = FN rate), AUC of ROC.
\item Information theoretical criteria: mutual information. Beause it needs to know the distributions of the variable and class label, it is most often applied to discrete cases. If continues variables, may discretize or use kernel density estimates. 
\end{itemize}

Feature relevance and redundancy: 
\begin{itemize}
\item Redundant variables can help each other: e.g. (Fig. 1) a variable $Z$ that is a linear combination of variables $X$ and $Y$ (thus redundant), can increase the information gain (by projecting data in a differentn dimension, making them more dispersed in two classes). 
\item A useless variable, when used with other ones, can be useful: e.g. XOR function, each of the features is completely useless. 
\end{itemize}

Variable subset selection: 
\begin{itemize}
\item Types: wrapper method (utilize the learning machine of interest as a black box to score subsets of variable according to their predictive power); embedded method (perform variable selection in the process of training and are usually specific to given learning machines), e.g. decision tree method such as CART. 
\item Search strategies: forward selection, backward elimiation. 
\item Objective functions for evaluating variable subsets: goodness-of-fit, and regularization term. 
\end{itemize}

Feature construction: 
\begin{itemize}
\item Clustering: of features by their class labels. 
\item Matrix factorization: e.g. SVD - form a set of features that are linear combinations of the original variables. 
\end{itemize}

Significant features/features subsets: 
\begin{itemize}
\item Variable ranking: fake variables, e.g. random values, or permutation test (permutate the feature vectors). 
\item Model selection (choose variable subsets): validation (by partitioning training data into training and validating sets), or cross-validation. 
\end{itemize}

\item{Bioinformatics}

Factor regression model in gene expression data: 
\begin{itemize}
\item Model: suppose we are solving a regression problem $y = f(x)$, we assume all the observations (including predictors and response) are determined by a (smaller) set of latent variables. Then we have a joint linear model of $(X,Y)$ as functions of $Z$ (latent variables). 

\item Sparse factor model in genomics [High-dimenstional sparse factor modeling: applications in gene expression genomics, JASA, 2008]: suppose for the $i$-th sample, we have $k$ latent variables, $\lambda_i$, representing the hidden pathway/module/regulator activities. Then expression of any gene $g$ in the $i$-th sample is thus: 
\begin{equation}
\E(x_{g,i}) = \mu_g + \alpha_g^T \lambda_i	
\end{equation}
where $\mu_g$ is the intercept of the gene $g$, and $\alpha_g$ is the effect of $\lambda$ on the gene. And the class of the $i$-th sample is similar: 
\begin{equation}
\E(y_{i}) = \mu_y + \alpha_y^T \lambda_i	
\end{equation}
This is a joint linear model of $p$ genes and class label on the $k$ latent varaibles. For the model to be identifiable, need to add constraints, including sparse factor loading (each latent factor is associated with only a small set of observed variables). 
\end{itemize}

Context-specific independence mixture modeling for positional weight matrices [ISMB, 2006]: 
\begin{itemize}
\item Idea: to learn a mixture of PWMs from sequence data, if assume each component of the mixture model has its own PWM, overparameterization. One may assume that different PWMs may share distributions in certain columns. This would reduce the model complexity.

\item Model: suppose there are $K$ PWMs to learn, and there are $p$ columns. We assume a structure $G$ represents the sharing of columns among the $K$ PWMs: i.e. at position $j$, $G_j$ is the partition of $K$ into groups, where each group has the same distribution at the position $j$. We define the prior distribution on $G$ that favors simpler models (more sharing)
\begin{equation}
P(G) = \prod_j \alpha^{z_j}
\end{equation}
where $z_j$ is the number of groups in the partition $G_j$ and $\alpha < 1$ is a hyperparameter. 

\item Inference: learn the model $M = (G, \theta_M)$ through an iterative algorithm, update/sample $G$ at each cycle according to the posterior distribution, and the parameter then can be estimated through EM. 

\item Remark: the CSI idea is that one may define a strcuture to represent the commonality/sharing among distributions, and a prior on the structure that favor simpler ones. For example, this could be applied to a dynamic model, where coefficients may change over time $\beta(t)$, but tend to stay the same. The associated structure can be represented by a binary matrix $M$, where $M_{jt}$ denotes if $\beta_j$ changes at time $t$ (if change, then resample $\beta_j$ using prior). 
\end{itemize}

\item{Network models}

Supervised random walks: Predicting and Recommending Links in Social Networks [Backstrom \& Leskovec, WSDM11]:
\begin{itemize}
\item Motivation: link prediction in social networks. Depends on both network topology (closeness) and attributes of nodes/edges. Ex. if two people share a lot of friends, then the two are also likely to be friends (network topology); if the two work in the same company, the chance they are friends is increased (attributes). 

\item Random walk with restart model: let $a_{uv}$ be the weigth of the edge $(u,v)$, then $Q'_{uv}$ is the normalized transition probability from the node $u$ to $v$: 
\begin{equation}
Q'_{uv} = \frac{a_{uv}}{\sum_w a_{uw}}
\end{equation}
And $Q'_{uv} = 0$ if $(u,v)$ is not linked. To keep the random walk around the starting node, we introduce the restart probability $\alpha$, i.e. the probability of jumping back to the seed node $s$ at each step. Thus the true transition probability is: 
\begin{equation}
Q_{uv} = (1 - \alpha) Q'_{uv} + \alpha \mathbf{1}(v = s)	
\end{equation}

\item Incorporating attributes: let $\Psi_{uv}$ be the node and edge attributes of $(u,v)$. We assume the edge weight is a function of these attributes, $a_{uv} = f_w(\Psi_{uv})$. The goal is to estimate $w$ s.t. (given a source node $s$), the stationary probability $p_d > p_l$, where $d$ is some node that are linked to $s$ and $l$ is the node not linked to $s$ (training data). 
\begin{itemize}
	\item Edge type: an important attributes. For an edge $(u,v)$, define its edge type as $(0,1)$, $(1,1)$, etc., depending on the distance of $u$ and $v$ to the source node $s$. It was found that the edge type is an important attribute. 
\end{itemize}

\item Optimization: the objective function is a function of $w$, taking the above criterion into account, with regularization term (s.t. few attributes have non-zero weight in $f$). This can be achieved by any method using the gradient, where the gradient can be computed in an iteriative fashion (similar to PageRank). 

\end{itemize}

\item{NLP}

Named entity recognition: [ACL, 2011]
\begin{itemize}
\item Background: in general to recognize mentionings with entities, use the discourse, background knowledge or string similarity. 

\item Idea: a graphical model of strings (mentions), define distance between string, and define potentials: affinity between close strings, and repulsion between distant strings. The method essentially do clustering on strings. 
\end{itemize}

Unsupervised predicate extraction from documents [ACL, 2011]:
\begin{itemize}
\item Idea: a generative model of predicate sentences consisting of (1) arguments; (2) syntactic realization. 
\end{itemize}

\item{Computer Vision}

Image cosegmentation: [Distributed Cosegmentation via Submodular Optimization on Anisotropic Diffusion, Gunhee Lee et al., ICCV, 2011]
\begin{itemize}
\item Background: image segmentation - extract salient features/divide an image into areas with salient features. Image cosegmentation - from multiple images, segment them simutaneously s.t. the features in different image align with each other (e.g. the same kind of objects). 

\item Application of heat diffusion models in image analysis: when putting heat source in some space, heat will diffused according to the heat equation, and the space with high thermal conductivity will tend to have similar temperatures. This is analogous to image analysis: if we map similarity between adajcent pixels to thermal conductivity, and the clustering of pixels (i.e. similar pixels belonging to the same clusters) to the temperature distribution, then we have regions with high similarity should be put into the same cluster. 

\item Remark: this is potentially much faster than MRF, which fails to take into account the spatial structure (each pixel is treated independently). 
\end{itemize}

Hierarchical model for image classification: 
\begin{itemize}
\item Reference: A Bayesian Hierarchical Model for Learning Natural Scene Categories [ICCV, 2005]; and some recent papers

\item Idea: to recognize an object in an image, we may be able to recognize the common feature of a class of objects; and identifying the class would help identify individual objects (the class provides prior for individual objects). This can be captured by a hierarchical model. 

\item Example: a class of objects: animal, vehicle; and objects within animal class: horse, cow, etc. 
\end{itemize}

\item{Maximum margin methods}

Graphical model learning by max margin and max entropy [Jun Zhu's work, 10/17/2011]:
\begin{itemize}
\item Background: 
\begin{itemize}
	\item Use graphical model for prediction (multi-class labels), the structure of input-ouput variables. 
	\item Models: CRF, maximum margin Markovian network (M3N).
	\item Challenges of graphical models: sparseness, prior, allow latent variables, stationarity, etc.
\end{itemize}

\item Maximum entropy discrimination (MED) learning [Jaakkola, 1999]
\begin{itemize}
\item Classical predictions: minimize loss function plus some penalty for regularization, e.g. logistic regression (squared error) and SVM (hinge loss)
\item Comparison of likelihood and max-margin methods: likelihood method is easy to allow latent variables, etc.; max-margin is free of probabilistic models. 
\item Motivation: combine the advantages of the two, e.g. averaging the parameters in the max-margin methods (Bayesian perspective), mixture of SVM models, etc. 
\item Maximum entropy discrimination (MED) learning (Jaakkola, 1999): prediction of label $y$ is the predicted value averaged over the parameters $w$. The learning of parameters is the minimization of KL divergence between the prior of the parameters and posterior (?) of parameters, subject to some contraints. 
\end{itemize}

\item Structured MaxEnt Discrimination: extend the MED learning to graphical models. 
\begin{itemize}
\item Under some specific conditions: the loss function (linear function) and the linear slack variables, and uniform prior of parameters, then Structured MaxEnt reduces to M3N. 
\item Extension to models with latent variables. 
\item Application to supervised learning with unlabelled data: traditionally, LDA $+$ SVM. The new method would guide the LDA process towards better discrimination. 
\end{itemize}

\item Infinity SVM: infinite number of mixture components of SVM models. 
\end{itemize}

\item{Multi-class problems}

Hierarchical binary classifiers: [Koller]
\begin{itemize}
\item Idea: suppose we want to classify objects into K classes, say 1 to 6. There is structure in the K classes, e.g. classes 1 and 2 are more related; 5 and 6 are more related, etc. To exploit this structure, we could first build a binary classifier that classify classess 1,2 (positive), 5,6 (negative) and 3,4 (ignore); then repeat this process (another binary classifer) on objects in classes 1,2,3,4 and on objects in class11429es 3,4,5,6; and so on. 

\item Objective function: at each level, learn a labeling of classes (positive, negative and ignore) while simulatenously perform binary classification. 

\item Remark: each level is independent, thus the errors in the top level would be propagated into lower levels. 
\end{itemize}

\end{enumerate}
%%%%%%%%%%%%%%%%%%%%%%%%%%%%%%%%%%%%%%%%%%%%%%%%%%%%%%%%%%%%
%%%%%%%%%%%%%%%%%%%%%%%%%%%%%%%%%%%%%%%%%%%%%%%%%%%%%%%%%%%%
\chapter{Artificial Neuron Networks}

Reference: 
\begin{itemize}
	\item Neural Networks for Machine Learning [Hinton, Coursera, 2016]
	\item Stanford Computer Vision class, \url{http://cs231n.stanford.edu}. 
	\item Goodfellow, Deep Learning, 2016
\end{itemize} 

Chapter 1: Introduction [Goodfellow et al]
\begin{itemize}
	\item Deep learning approach: Create new \textbf{representations}. Ex. from pixel representation to geometric shapes. Mathematically, new representation may make the problem much easier, e.g. linearly separable. 
	
	\item Deep learning approach: Make \textbf{abstractions}. The challenge is often the ``factors of influences'', e.g. age, gender and accent in speech. We create new representations that are invariant of such factors. The new representations are defined in terms of simpler representations: eg. edge in terms of pixels, and corners/contours in terms of edges, and object parts in terms corners and contours.  
\end{itemize}

Deep learning [LeCun \& Hinton, Nature, 2016]
\begin{itemize}
	\item Representation learning: existing classifiers require expert-curated ``features''. Learning these features itself is what's called \textbf{representation learning}. For example, in vision, we have multiple layers of neurons: one layer learns edges, the next layer learns shapes from the previous layer (combination of edges), and the next layer learns objects, and so on. 
	
	\item Why deep learning can do better than conventional ``shallow'' classifiers? The challenge of shallow classifiers is that it does not have a representation of high-level features. Ex. in vision, pixels are the basic features, and edges represents combination of these features. In linear classifiers, one would need to encode them as ``interaction terms'', which cannot be made very complex. 
	\begin{itemize}
		\item Challenge of machine learning: e.g. learn images of wolf vs. a dog breed that is similar to wolf. Without feature extraction, two images of the same wolf can be much more different (e.g. in different positions, illumuniations) than two images of wolf and dog. 
		
		\item Non-linearity of NN: as shown in Figure 1, even if the input functions are non-linear (i.e. positive and negative class has a non-linear boundary), the hidden layer can distort the input s.t. they become linearly separable. 
		
		\item The limitation of kernel methods: cannot generalize far enough beyond existing training examples (because they are based on similarity with existing examples). 
	\end{itemize}
	
	\item Pre-training and resurgence of ANN: pre-training is a form of unsupervised learning that learns representation. This is achieved by minimizing ``reconstruction error''. Autoencoder? 
	
	\item Convolutional neural networks (ConvNets): designed for array data, e.g. image. Its design: between two layers, not fully connected. It has two features: 
	\begin{itemize}
		\item Local features: e.g. edge is a feature derived from local pixels. In the convolutional layer, we have feature maps, where each feature map is linked only to some feature maps in the previous layer, representing a local motif (the links are called filter-banks). 
		\item Semantic similarity of features: slight variation of a feature (e.g. shifting position) can be captured by the same neurons. This is achieved by a pooling layer. 
	\end{itemize}
	
	\item Recurrent neural networks: designed for sequential/time series data. To predict output of the next element, we have neurons for the current input $x_t$, and neurons representing all past input $s_{t-1}$. The result is the node $s_t$. RNN uses all the data to train the weights (if unfolded wrt time, we have many layers correpsonding to each time point, with shared weights). 
	\begin{itemize}
		\item Importance of memory: proposal that we create memory cells to represent the past. 
	\end{itemize}
	
	\item Future of deep learning: (1) Importance of unsupervised learning. (2) Combination of representation learning and complex reasoning.
	
	\item Question: ConvNets, how to represent a feature, e.g. edge, at different positions? 
\end{itemize}

Can we open the black box of AI? [Nature, 2016]
\begin{itemize}
	\item Deep Dream: e.g. a deep NN trained for recognizing animal faces. Given input of flower, continue to modify the input images to increase the response rates. After some iterations, we see images where animal faces emerging in flowers - halluciation. 
	
	\item General problem: neural networks are surprisingly easy to fool with images that to people look like random noise. 
\end{itemize}

Lecture 1: Introduction to Deep Learning [Coursera]
\begin{itemize}
	\item Examples of machine learning tasks: handwritten digit recognition (MNIST database), ImageNet (1000 classes, millions of images), speech recognition. 
	
	\item How brain works? Each neuron receives multiple inputs (dendrites) and has one output (axon). Synapses: the strength can be changed (adapt). Human brain: $10^{11}$ neurons and each $10^4$ weights. 
	
	\item Model of neurons (activation function): let $x$ be input and $z = \sum_i w_i x_i$ be the weighted sum of input, and $y = f(x)$ is the activation function. It can be linear, binary, rectified linear, sigmoid or stochastic binary. 
	
	\item Example of handwritten digit recognition: for each digit, two layered neurons, one for input the other output. The learned weights match the template of the digit. However, cannot capture variations. 
	
	\item Unsupervised learning: learn internal representation of data. Main applications: 
	\begin{itemize}
		\item Useful for later supervised learning. 
		\item Low-dim. representation of data. PCA is one very limited example. 
		\item Denoising data: economic representation of input. 
		\item Clustering: if one view each cluster as a feature, then clustering is a simple sparse coding, where for each sample, only one feature is non-zero. 
	\end{itemize}
\end{itemize}

%%%%%%%%%%%%%%%%%%%%%%%%%%%%%%%%%%%%%%%%%%%%%%%%%%%%%%%%%%%%
\section{Feedforward Neuron Networks}

Lecture 2: Perceptron
\begin{itemize}
	\item Types of NN: feedforward neuron networks and recurrent neuron networks (RNN). RNN is designed for sequential data: it allows neurons to form circles. RNN is similar to very deep NN, where each layer corresponds to one time point (but the same weights).  
	
	\item Perceptron: binary threshold neurons, two layers, input-output. Its output is 1 if $w \cdot x + b > 0$, and 0 otherwise. We could transform the perceptron s.t. it has an extra input with constant value 1 (then we remove the bias term).  
	
	\item Geometric intuition of perceptron: we consider the weight space. For each $x_i$, suppose $y_i = 1$, then we should have $w \cdot x_i > 0$ (ignoring threshold). This poses \textit{constraint} on $w$: it must be in one side of the hyperplane defined by $w \cdot x_i = 0$. Each $x_i$ limits the possible locations of $w$. Ex. for $x_1$ and $x_2$, the feasible set may be a cone. The problem is thus to determine the feasible region/solution from all constraints posed by $x_i$'s. 
	\begin{itemize}
		\item Remark: this is basically a linear programming problem. And when we maximize the margin while finding a feasible set, this becomes SVM. 
	\end{itemize} 
	
	\item Perceptron learning algorithm: if we make a mistake with $x_i$ using the current weight $w$, we add or subtract $x_i$ to $w$, depending on the mistake. Intuitively, if $y_i = 1$, but our $w \cdot x_i < 0$, we make $w' = w + x_i$, then $w' x_i = w \cdot x_i + x_i \cdot x_i$, and its becoming more positive. Proof idea: we can always choose a solution in the ``generous feasible'' region (margin at least $x_i$), then this rule will guarantee that the weight will become closer to the target solution. 
	\begin{itemize}
		\item Remark: the difficult part to understand, how do we know that a new $w$ will not violate previous $x_i$'s. 
		\item Example 1: a single $x_1$ with $y_1 = 1$. The rule will keep increasing $w \cdot x_1$ until $w \cdot x_1 > 0$. 
		\item Example 2: we have two data points $x_1$ and $x_2$, repeatedly. Suppose the feasible set is the cone defined by two input vectors, one can see that eventually $w$ will reach the feasible region.  
	\end{itemize} 
	One can show that the objective function is convex. 
	
	\item Limitations of perceptron: cannot learn XOR (not linearly separable), and the example of translation wrap-around (which forms a group). Intuition: for each $w_j$, its value depends on how many times the corresponding $x_j$ is activated in positive vs. negative examples. The perceptron cannot distinguish two patterns if a pixel is equally likely to be activated. Group Invariance Theorem - cannot distinguish transformations of a pattern. 
	
	\item Questions: is RNN biologically motivated? 
\end{itemize}

Back-propagation [Wiki; personal notes]: 
\begin{itemize}
	\item Derivative of logistic function: suppose we use logistic function as our activation function, we have this simple result: 
	\begin{equation}
	\frac{dy}{dx} = y (1-y)
	\end{equation}
	
	\item Our NN is $f(x,w)$ where $w$ needs to be learned. The error is $E(w) = \frac{1}{1} \sum_i (f(x_i;w) - t_i)^2$, where $t_i$ is the target output of input $x_i$. We estimate $w$ via gradient descent. Let $w_{ij}$ be the weight of the link from neuron $i$ to $j$. For neuron $j$, let $o_j$ be its output, $z_j$ be its total input, we have: 
	\begin{equation}
	\frac{\partial E}{\partial w_{ij}} = \frac{\partial E}{\partial o_j} \frac{\partial o_j}{\partial z_j} \frac{\partial z_j}{\partial w_{ij}}
	\end{equation}
	The second term is determined by the activation function of neurons:
	\begin{equation}
	\frac{\partial o_j}{\partial z_j} = o_j (1-o_j)
	\end{equation}
	The third term is simply $o_i$. For the first term, it is simple when $j$ is in the output layer. If not, we note that $E$ is a function of $o_u$, where $u$ is a neuron in next layer of $j$, and $o_u$ depends on $o_j$ (this is how $o_j$ may affect output). We use the chain rule to write: 
	\begin{equation}
	\frac{\partial E}{\partial o_j} = \sum_u \frac{\partial E}{\partial o_u} \frac{\partial o_u}{\partial z_u} \frac{\partial z_u}{\partial o_j}
	\end{equation}
	Plug in the relevant terms: 
	\begin{equation}
	\frac{\partial E}{\partial o_j} = \sum_u \frac{\partial E}{\partial o_u} o_u (1-o_u) w_{ju}
	\end{equation}
	This gives the recurrence in terms of $\partial{E} / \partial {o_j}$, and we can solve them using dynamic programming. Once we have the gradient, we update $w_{ij}$ using gradient descent: $\delta w_{ij} = - \alpha \frac{\partial E}{\partial w_{ij}}$, where $\alpha$ is the step size.  
	
	\item Algorithm: at each iteration, we first have the forward pass to obtain $o_j$ for every neuron $j$; next we use backward pass to obtain the derivatives $\partial{E} / \partial{o_j}$, and the derivatives wrt. the weights. Then we update the weights using gradient descent. 
	
	\item Analysis: why backpropgation is better than simple numerical differentiation? Let $m$ be the number of edges and $n$ be number of neurons. Simple numerical derivative needs $O(mn)$ computations (compute $m$ times, and each time, compute the whole NN). While backpropgation needs $O(n)$ steps for computing $\frac{\partial E}{\partial o_j}$ and $O(m)$ steps for each of the $m$ edges.
	
	\item Backpropagation with constraints: sometimes (e.g. in CNN or RNN), we have constraints on parameters, say $w_1 = w_2 = w$. To implement these constraints, we use this as gradient on $w$: 
	\begin{equation}
	\frac{\partial E}{\partial w} = \frac{\partial E}{\partial w_1} + \frac{\partial E}{\partial w_2}
	\end{equation} 
	So we do the same backpropagation on $w_1$, $w_2$, but use the sum of gradients on all $w_i$'s as the gradient of $w$. 
\end{itemize}

Lecture 3: Backpropagation [Coursera]
\begin{itemize}
	\item Iterative optimization: an optimization problem, e.g Least square problem, can often be solved by iterative algorithm (even if other solutions exist). They may be slower, but very easy to generalize. 
	\begin{itemize}
		\item Different versions of iterative algorithm: batch version (using all training examples to update parameters), and online algorithm (update parameters after each example). 
		
		\item Pictures of iterative algorithm: gradient descent would move parameters along the direction of gradient, while online algorithm move in a zig-zag fashion. 
	\end{itemize}
	
	\item Delta-rule for linear neurons: consider input-output NN (single output). Let $i$ be index of example and $j$ be index of feature. Our error $E(w) = \frac{1}{2} \sum_i (t_i - y_i)^2$, where $y_i = \sum_j w_j x_{ij}$. It is easy to show that: 
	\begin{equation}
	\frac{\partial E}{\partial w_j} = - \sum_i x_{ij} (t_i - y_i)
	\end{equation}
	The update rule is $\delta w_j = - \epsilon \frac{\partial E}{\partial w_j}$. So the update is similar to perceptron learning (proportional to input vector), except that the change is weighted by learning rate $\epsilon$ and the difference between $t_i$ and $y_i$. 
	
	\item Delta-rule for logistic neurons: let $z_i = \sum_j w_j x_{ij}$ be the total input of the output neuron for example $i$. We can use chain rule to derive: 
	\begin{equation}
	\frac{\partial E}{\partial w_j} = \sum_i \frac{\partial E}{\partial y_i} \frac{\partial y_i}{\partial z_i} \frac{\partial z_i}{\partial w_j} = - \sum_i x_{ij} y_i (1-y_i) (t_i - y_i)
	\end{equation}
	So it's similar to delta rule except the extra weight $y_i (1-y_i)$, which is the slope of the logistic function.
	
	\item Backpropagation:  Intuition of why backprogation works: simple numerical derivative wrt. $w_j$ is equivalent to perturb every weight. Backprogation effectively perturbs the activity of each hidden neuron: we consider $\partial E / \partial y_i$, where $y_i$ is the activity of neuron $i$. We can compute this derivative for all neurons simultaneously. As the number of hidden neurons is smaller in number than that of weights, the algorithm more efficient. Formally, let $y_j$ be the output of $i$-th neuron and $z_j$ be its total input, and $w_{ij}$ be the weight of neuron $i$ to $j$ (assuming there is only one training example). Using chain rule: 
	\begin{equation}
	\frac{\partial E}{\partial w_{ij}} = \frac{\partial E}{\partial y_j} \cdot \frac{\partial y_j}{\partial z_j} \cdot \frac{\partial z_j}{\partial w_{ij}} 
	\end{equation}
	
\end{itemize}

Automatic differentiation (autodiff) [Wiki] Automatic Differentiation, Explained [Towards Data Science]
\begin{itemize}
	\item Idea: to compute the derivative of a function $y = f(x)$, we write it as some function $g$ of $z = h(x)$, we can then apply the chain rule to reduce the derivative to some simpler derivatives, which are easier to compute. The rule can be repeatedly applied. More generally, we represent the computation as a graph.
	
	\item Note: different from symbolic calculation, which can be very cumbersome. Also different from numerical method (finite difference), which could have large numeric errors.  
	
	\item Example: see Figure 2, Automatic Differentiation, Explained [Towards Data Science]. Suppose we have $z = y - \max(0, wx +b)$ and we want to compute $\partial z / \partial w$. We build a computation graph: where each leaf node is the ``input'': $w, x, b, y$ and the output node is $z$. The internal node represents the intermediate expressions. Next, we compute the partial derivatives of each connection (output vs. input), evaluated at a given input. With these, we can then compute $\partial z / \partial w$ in terms of these connections. 
\end{itemize}
%%%%%%%%%%%%%%%%%%%%%%%%%%%%%%%%%%%%%%%%%%%%%%%%%%%%%%%%%%%%
\section{Convoulutional Neuronal Networks}

Lecture 5, Coursera 
\begin{itemize}
	\item Challenges of object recognition in images: the main challenge is that an object might have different appearances, because of: 
	\begin{itemize}
		\item Lighting: affecting pixel intensity. 
		\item Viewpoint: esp. a problem for 3D object. 
		\item Deformation/abstraction: e.g. a chair may have different shapes. Another example: tri-angles. 
	\end{itemize} 
	Other challanges may include: segmentation (other objects in images). 
	\begin{itemize}
		\item Perspective: the appearance of an object is some transformations of the original object. Our problem is to find such ``inverse transformation''. 
	\end{itemize}
	
	\item Ideas for addressing object invariance:
	\begin{itemize}
		\item Normalization by putting a box: the challenge is of course without knowing the object, it's hard to know where to put the box. Idea: try all box positions. 
		\item Replicate features (ConvNet) s.t. different neurons represent all possible variations of the same underlying feature. 
	\end{itemize} 
	
	\item ConvNet: 
	\begin{itemize}
		\item Convolution layer: each neuron represents a feature in a particular position (possible variation). 
		\item Pooling layer: pool neurons from convolution layers, this would indicate whether a feature is present. 
		\item Fully-connected (FC) layer: topmost layers, combine information of all features. 
	\end{itemize}
	The problem is that the position information is lost after pooling, which can be important when we recognize composite object (relative positions of multiple basic objects matter). This may be solved by representing features in particular regions (after pooling, a neuron represents if a feature is present in a given region). 
	
	\item Training of ConvNet: backpropagation with constraint. Ex. to satisfy $w_1 = w_2$, we calculate $\frac{\partial E}{\partial w_1} + \frac{\partial E}{\partial w_2}$ and use it to update both $w_1$ and $w_2$. 
	
	\item Role of prior knowledge: in general, we use prior knowledge in two ways: (1) architecture of the NN; (2) training data. One idea is to enhance training data, by generating variations/transformations from the original training data.  
	
	\item Example: recognition of ``+'' in an image, allowing positional variation. Suppose we have a 10 by 10 image, and the sign is 3 by 3. We first divide the image into 3 $\times$ 3 regions. In convolution layer, each neuron is connected to a 3 by 3 region, and is activated when a plus is present. In the pooling layer, one neuron to represent whether the plus sign is present in any of the regions. 
	
	\item Example: recognition of circle on top of plus signs. Convulution layer: recognize circle and plus signs. Pooling: circle (with region information, i.e. pool only circles within a region) and plus. FC layer: match circle and plus in the same regions. 
		
	\item Success stories: LeNet for hand-digit recognition and ImageNet. The key tricks of ImageNet include: ReLU, dropout (remove half of the neurons during training), hardware. 
	
	\item Question: how to model scale invariance? Ex. rectangle, where the lengths of edges can vary. 
\end{itemize}

Stanford lecture notes: ConvNets (Lecture 4)
\begin{itemize}
	\item Convolution layer: we have filters, which represent features. Each neuron of a filter represents a feature in a particular region (size defined by \textit{receptive field}). Each filter has many neurons, with adjacent neurons defined by \textit{strides}. All filters of a given region are called the \textit{depth column} (representing all features in this region). The parameters of these strides, number of filters, etc. are hyperparameters. 
	\begin{itemize}
		\item Volume analysis: volume of a layer is defined by number of elements in width $\times$ height, and depth (for input: number of colors; for other layers, number of filters). 
	\end{itemize} 
	
	\item Pooling layer: reduce the representation in convolution layers. Ex. a neuron in the pooling layer may represent 4 adjacent regions. MAX is most common. 
	
	\item Architecture: commonly, convolution, then pool, and fully-connected (FC) as the last layer. Pooling layer may be less important. 
	
	\item Question: we only defines filters in conv. layer, how do we ensure that we actually learn distinct features? And what are the features? 
\end{itemize}
%%%%%%%%%%%%%%%%%%%%%%%%%%%%%%%%%%%%%%%%%%%%%%%%%%%%%%%%%%%%
\section{Sequential Neural Networks}

Lecture 6-7, Coursera
\begin{itemize}
	\item Linear Dynamic System and HMM. 
	
	\item RNN model: let $y_t$ be the neuron activity at time $t$, we have $y_t = f(y_{t-1}; w)$, where $f$ is the activating function and $w$ the weights. It is similar to a feedforward NN except that the weights are constrained to be the same across time. 
	
	\item Architecture of RNN? 
	
	\item Training RNN: we can use backprop, but the challenge is the exploding or vanishing gradients. In backprop, the relationship between gradients in successive layers is linear (not logistic), thus the weights can get exponentially large or small over many layers (many time steps). 
	
	\item Long short-term memory (LSTM): a memory cell with self-weight of 1 (thus in each time step, keep its value). It has multiple gates: Write, Keep and Read. 
\end{itemize}

Lecture 10 [Stanford]
\begin{itemize}
	\item Problems solved by Recurrent Neuron Networks (RNN): some examples
	\begin{itemize}
		\item One to many: image caption. 
		\item Many to one: sentiment of sentence. 
		\item Many to many: machine translation. 
	\end{itemize} 
	The problem differs from image recognition problem in that: one needs to have a representation of what has been seen so far (not all at once); and the sequence can be arbitrarily long. 

	\item Design/architecture of RNN: we consider the case where we have $x = \{x_t\}$ as input, and $y$ as output, which could be single (or a few), or a sequence as well. The key elements of designing RNN: 
	\begin{itemize}
		\item Use hidden neurons to represent what has been seen at a point. The hidden neuron should depend on the hidden neuron in the previous time step and the current input. 
		
		\item The functions of how output depends on the hidden neurons, and how hidden neurons depend on previous time step and input should be invariant to time. 
	\end{itemize}
	We denote $x_t, h_t, y_t$ as input, hidden neuron and output at time $t$ (each a vector), then we have: 
	\begin{equation}
	h_t = f_W(h_{t-1}, x_t) \qquad y_t = g_W(h_t)
	\end{equation}
	Consider a specific form: 
	\begin{equation}
	h_t = \tanh(W_{hh} h_{t-1} + W_{xh} x_t) \qquad y_t = W_{hy} h_t
	\end{equation}
	Ex. predicting next character: the input vocabularoy is 26, and let's say there are 100 hidden neurons. Then the dimensions of $W_{xh}$, $W_{hh}$ and $W_{hy}$ are $26 \times 100$, $100 \times 100$ and $100 \times 26$ respectively.  
	
	\item How RNN works- what do the hidden neurons represent? In the character prediction example, hidden neurons generally represent the information of sentence we have seen. Ex. we may have a neuron for the prefix ``ho'', and two neurons for the words ``hors'' and ``hous'': they are activated by the letter ``e''. 
	
	\item Training of RNN: how would backpropagation work? We do forward pass through all time steps to obtain $h_t$ and $y_t$. Then do backward pass through time to obtain gradients. Q: constraint that $W$ is constant over time? 
	
	\item Connecting CNN and RNN: image captioning. The output of CNN (not the final output, but the layer before) is supplied as the first hidden layer $h_0$ (time 0) of the RNN. Then we use RNN to generate the sequence of words as caption: we use the current output (sampling) as the input of next time step. Intuitively: the hidden neurons have two functions: representing image information and record what has been produced in the sentence. 
	
	\item Challenge of RNN: vanishing gradient problem. LSTM. 
\end{itemize}

RNN, LSTM and Transformer [Ben Lai, Group meeting, 2020]. Visualizing A Neural Machine Translation Model [Jay Alammer] \url{https://jalammar.github.io/visualizing-neural-machine-translation-mechanics-of-seq2seq-models-with-attention/}
\begin{itemize}
	\item Motivation of sequential model: suppose we have DNA sequence and ChIP-seq data of whether a TF is bound. Our goal is to predict TF binding (experimental data) from DNA sequences.
	
	\item RNN: at each time step, input $x_t$, and previous hidden state $a_{t-1}$, and compute the new hidden state $a_t$, and the output $y_t$ as a function of $a_t$. All the parameters are shared across time steps. In TFBS example: our hidden state records the motifs read so far, and if we enter a new motif, we update hidden state (in motif) and predict output (TF binding). If we exit a motif, we update hidden state (out of motif now) and predict output (no TF binding).
	
	\item Problem of RNN: hard to capture long-term dependency, and vanishing gradient.
	
	\item Motivation of LSTM: e.g. CTCF as boundaries of enhancers. To predict TFBS, only the sequence in an enhancer is relevant.
	
	\item LSTM: at each step, the hidden state is more complex: it has 4 gates, one gate allows you to forget (reset hidden states). LSTM still slow and ineffective for long sequences.
	
	\item Applications in genomics: DanQ, combine CNN and LSTM. The output of CNN: hidden features, are used as inputs of LSTM. Learning deep protein representation (prediction of missing residuals) and applications in protein engineering.
	
	\item Seq2seq model (Neural machine translation): encoder and decoder, both RNNs. Input: word embedding. (1) Encoder: input to context vector.  (2) Decoder: context vector to output (another language). Note: similar to autoencoder, context is the bottleneck layer.
	
	\item Attention: the idea is to focus only on relevant parts of the input sequences. Use all hidden states as input of the decoder (all times steps): however, weigh the time points - hence Attention. Then use the sum of the weighted hidden states as the context vector in the Decoder. At each time step of the Decoder, use different weights, thus focusing on different parts of the input sequence.
	
	\item Transformer: see, The Illustrated Transformer, \url{http://jalammar.github.io/illustrated-transformer/}
\end{itemize}
%%%%%%%%%%%%%%%%%%%%%%%%%%%%%%%%%%%%%%%%%%%%%%%%%%%%%%%%%%%%
\section{Deep Generative Models}

Boltzmann machines [Chapter 20, Deep Learning, Goodfellow et al, 2016]
\begin{itemize}
	\item Goal: model probability distribution over $d$-dim. binary vectors. This can be done with MRF:
	\begin{equation}
	P(x) \propto \exp(-E(x)) \qquad E(x) = -x^T U x - b^T x
	\end{equation}
	where $E(x)$ is the energy function. 
	
	\item Boltzmann machines: when not all variables are observed. Let $h$ be the hidden variables and $v$ be observed ones, or $x = (v,h)$. Then the energy function becomes:
	\begin{equation}
	E(v, h) = -v^T R v - v^T W h - h^T S h - b^T v - c^T h
	\end{equation}
	This allows one to compute $P(v|h)$ and $P(h|v)$. 
	
	\item Restricted Boltzmann machines (RBMs): a special type of Boltzmann machines where all the $v$'s are in the same layer, and all the $h$'s are in the different layer. Only connections across the layers, but not within a layer - Figure 20.1 (a). The energy function is simpler: $E(v,h) = -b^T v - c^T h - v^T W h$. The conditional distribution has closed form. This enables efficient block Gibbs sampling. 
\end{itemize}

Deep belief Network (DBN) and Deep Boltzmann machines (DBM) [ibid]
\begin{itemize}
	\item DBN: multiple hidden layers (undirected network), and directed graph from $h$ to $v$ (observed). Figure 20.1 (b). Fall out of fashion now. 
	
	\item DBM: multiple hidden layers and one observed layer, all undirected network, Figure 20.1 (c). Inference: use VB mean-field approximation. 
	
	\item Gaussian-Bernoulli RBMs: generalize to real valued data. This can be simply modeled as observed data is a linear function of hidden layers: $P(v|h) = N(v| Wh, \beta^{-1})$, where $\beta$ is the precision matrix. 
\end{itemize}

Variational auto-encoders (VAE) 
\begin{itemize}
	\item Ref: \url{https://jaan.io/what-is-variational-autoencoder-vae-tutorial/}. [Tutorial on Variational Autoencoders, arxiv, 2016]. Alan Selewa presentation. 
	
	\item Background: stochastic gradient descent (SGD). In gradient descent, suppose we want to optimize $Q(w)$, our update rule for parameters:
	\begin{equation}
	w := w - \eta \nabla Q(w) = w - \eta \sum_{i} \nabla Q_i(w) /n
	\end{equation}
	where $\nabla Q_i(w)$ is the gradient at sample $i$. When the number of training samples is large, a single iteration of gradient descent is slow. SGD updates $w$ immediately after seeing one training example: 
	\begin{equation}
	w := w - \eta \nabla Q_i(w)
	\end{equation}
	So SGD can potentially be much faster. It may osculate between different values, but in practice, they are close to the local (or global) optimum.  
	
	\item Background: autoencoder [Hinton and Salakhutdinov, Science, 2006]. Figure 3: better classification of digits than PCA. 
	
	\item Neural nets perspective: we have an encoder network $x$ to latent variable $z$: denoted as $q_{\theta}(z|x)$ and decoder network $z$ to $x$: denoted as $p_{\phi}(x|z)$. The objective (loss) function for a single example $x_i$ is given by: 
	\begin{equation}
	l_i(\theta, \phi) = -\E_{z \sim q_{\theta}(z|x_i)}[\log p_{\phi}(x_i|z)] + \text{KL}(q_{\theta}(z|x_i) || p(z))
	\end{equation}
	where $p(z)$ is the prior distribution of $z$. The loss function has two parts: (1) reconstruction loss at $x_i$: the encoder and decoder network should be chosen s.t. the probability of generating $x_i$ is large. Note that this prob. is averaging over the distribution $z$ given $x_i$. (2) Regularization: the posterior of $z$ given $x_i$ should be close to the prior. 
	
	\item Probability model perspective: VAE as a graphical model: $z \sim N(0, I)$ (prior), and $x_i | z \sim p(x|z)$. Why this is a good model? Even with simple normal distribution of $z$, we can generate very complex distributions. See the example in Tutorial: 2D normal distribution generates data points in a ring. 
	
	\item Variational inference of VAE: let $\phi$ be the model parameters. Our goal should be to find $\phi$ that max. $P(x)$. However, this is difficult. Our intuition is to use $q_{\theta}(z|x)$ to approximate the posterior $P(z|x)$ by min. KL divergence between the two. We can express the KL divergence in terms of ELBO: 
	\begin{equation}
	\log P(x) = \text{ELBO}(\lambda) + \text{KL}(q_{\theta}(z|x) || p(z|x))
	\end{equation}
	where $\lambda$ indexes the $q$ distribution and the ELBO function is defined as: 
	\begin{equation}
	\text{ELBO}(\lambda) = \E_{z \sim q_{\theta}(z|x)}[\log p_{\phi}(x|z)] - \text{KL}(q_{\theta}(z|x) || p(z))
	\end{equation}
	Since $\log P(x)$ is constant (independent of $\theta$), so min. KL divergence is equivalent to max. ELBO. Since $\phi$ is not known, we maximize ELBO in terms of both $\theta$ and $\phi$: this maximizes $P(x)$ while minimizing KL divergence $q$ and $P(z|x)$. The objective function for SGD of a single example is then: 
	\begin{equation}
	\text{ELBO}_i(\theta, \phi) = \E_{z \sim q_{\theta}(z|x_i)}[\log p_{\phi}(x_i|z)] - \text{KL}(q_{\theta}(z|x_i) || p(z))	
	\end{equation} 
	So our objective is to maximize data likelihood (first term), with regularization (second term). 
	
	\item Form of $q(z|x)$ function: normal distribution, mean and covariance depends on data and $\theta$. This is the encoder network. 
	
	\item Example: image generation, the pixel (0 or 1) is sampled from Bernoulli distribution. The $q$ function is: 
	\begin{equation}
	p_{\theta}(x|z) \sim \text{Ber}(\pi_z) \qquad q_{\phi}(z) \sim N(\mu, \text{diag}(\sigma_1^2, \cdots, \sigma_d^2))
	\end{equation}
	And $\pi_z$ are sigmoid function of $z$. For this model, ELBO has simple/closed forms: where the first term matches the binomial likelihood and the second term penalize large $\mu$ and $\sigma$ very different from 1. 
	
	\item Optimization: stochastic gradient descent, ELBO at each data point $i$. Problem is that the NN has stochastic nodes $z$ and backpropagation does not work. Reparameterization trick: sample $\epsilon$, and add to $z$, then do back-propagation. 
	
	\item Application of VAE: pictures of humans, and learn the latent variables: smile, gender, beard, glass, etc. 
\end{itemize}
%%%%%%%%%%%%%%%%%%%%%%%%%%%%%%%%%%%%%%%%%%%%%%%%%%%%%%%%%%%%
%%%%%%%%%%%%%%%%%%%%%%%%%%%%%%%%%%%%%%%%%%%%%%%%%%%%%%%%%%%%

\bibliographystyle{named}
\bibliography{Statistics-ref}

\end{document}
